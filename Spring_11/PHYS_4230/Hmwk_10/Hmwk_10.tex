\documentclass[11pt]{article}

\usepackage{latexsym}
\usepackage{amssymb}
\usepackage{amsthm}
\usepackage{enumerate}
\usepackage{amsmath}
\usepackage{cancel}
\numberwithin{equation}{section}

\setlength{\evensidemargin}{.25in}
\setlength{\oddsidemargin}{-.25in}
\setlength{\topmargin}{-.75in}
\setlength{\textwidth}{6.5in}
\setlength{\textheight}{9.5in}
\newcommand{\due}{April 15th, 2011}
\newcommand{\HWnum}{10}
\newcommand{\grad}{\bold\nabla}
\newcommand{\vecE}{\vec{E}}
\newcommand{\scrptR}{\vec{\mathfrak{R}}}
\newcommand{\kapa}{\frac{1}{4\pi\epsilon_0}}
\newcommand{\emf}{\mathcal{E}}
\newcommand{\unit}[1]{\ensuremath{\, \mathrm{#1}}}
\newcommand{\real}{\textnormal{Re}}
\newcommand{\Erf}{\textnormal{Erf}}
\newcommand{\sech}{\textnormal{sech}}
\newcommand{\scrO}{\mathcal{O}}
\newcommand{\levi}{\widetilde{\epsilon}}
\newcommand{\partiald}[2]{\ensuremath{\frac{\partial{#1}}{\partial{#2}}}}
\newcommand{\norm}[2]{\langle{#1}|{#2}\rangle}
\newcommand{\inprod}[2]{\langle{#1}|{#2}\rangle}
\newcommand{\ket}[1]{|{#1}\rangle}
\newcommand{\bra}[1]{\langle{#1}|}





\begin{document}
\begin{titlepage}
\setlength{\topmargin}{1.5in}
\begin{center}
\Huge{Physics 3320} \\
\LARGE{Principles of Electricity and Magnetism II} \\
\Large{Professor Ana Maria Rey} \\[1cm]

\huge{Homework \#\HWnum}\\[0.5cm]

\large{Joe Becker} \\
\large{SID: 810-07-1484} \\
\large{\due} 

\end{center}

\end{titlepage}



\section{Problem 6.12}
Given that a molecule of cyanogen (CN) has a first excited state energy of $4.7\times10^{-4}\unit{eV}$ from the ground state with 3 degenerate states at this energy level. And that for every 10 molecules found at ground state there are 3 molecules at the first excited energy level we can say that
$$\frac{P(s_2)}{P(s_1)} = \frac{3}{10}$$
we can find the temperature this system is held at by using the \emph{Boltzmann Distribution}
$$\frac{P(s_2)}{P(s_1)} = e^{-[E(s_2)-E(s_1)]/k_BT}$$
by solving for $T$
\begin{align*}
\frac{P(s_2)}{P(s_1)} = \frac{3}{10} &= e^{-[E(s_2)-E(s_1)]/k_BT}\\
&\Downarrow\\
\frac{E(s_1)-E(s_2)}{k_BT} &= \ln\left(\frac{3}{10}\right)\\
&\Downarrow\\
T &= \frac{E(s_1)-E(s_2)}{k_B\ln(3/10)}\\
&= \frac{-4.7\times10^{-4}\unit{eV}}{(1.38\times10^{-23}\unit{J\ K^{-1}})(-1.20)}\\
&= \frac{-7.53\times10^{-23}\unit{J}}{(1.38\times10^{-23}\unit{J\ K^{-1}})(-1.20)}\\
&= 4.52\unit{K}
\end{align*}


\section{Problem 6.13}

\section{Problem 6.17}
\begin{enumerate}[(a)]
\item
\item
\item
\item
\end{enumerate}

\section{Problem 6.18}
\section{Problem 6.20}
\begin{enumerate}[(a)]
\item
\item
\item
\item
\item
\end{enumerate}

\section{Problem 6.39}
\begin{enumerate}[(a)]
\item
\item
\item
\end{enumerate}

\end{document}

