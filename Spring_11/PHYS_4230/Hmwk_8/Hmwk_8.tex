\documentclass[11pt]{article}

\usepackage{latexsym}
\usepackage{amssymb}
\usepackage{amsthm}
\usepackage{enumerate}
\usepackage{amsmath}
\usepackage{cancel}
\numberwithin{equation}{section}

\setlength{\evensidemargin}{.25in}
\setlength{\oddsidemargin}{-.25in}
\setlength{\topmargin}{-.75in}
\setlength{\textwidth}{6.5in}
\setlength{\textheight}{9.5in}
\newcommand{\due}{March 18th, 2011}
\newcommand{\HWnum}{8}
\newcommand{\grad}{\bold\nabla}
\newcommand{\vecE}{\vec{E}}
\newcommand{\scrptR}{\vec{\mathfrak{R}}}
\newcommand{\kapa}{\frac{1}{4\pi\epsilon_0}}
\newcommand{\emf}{\mathcal{E}}
\newcommand{\unit}[1]{\ensuremath{\, \mathrm{#1}}}
\newcommand{\real}{\textnormal{Re}}
\newcommand{\Erf}{\textnormal{Erf}}
\newcommand{\sech}{\textnormal{sech}}
\newcommand{\scrO}{\mathcal{O}}
\newcommand{\levi}{\widetilde{\epsilon}}
\newcommand{\partiald}[2]{\ensuremath{\frac{\partial{#1}}{\partial{#2}}}}
\newcommand{\norm}[2]{\langle{#1}|{#2}\rangle}
\newcommand{\inprod}[2]{\langle{#1}|{#2}\rangle}
\newcommand{\average}[1]{\left\langle{#1}\right\rangle}
\newcommand{\ket}[1]{|{#1}\rangle}
\newcommand{\bra}[1]{\langle{#1}|}
\newcommand{\Resid}[2]{\ensuremath{\textnormal{Res}\left[{#1},{#2}\right]}}





\begin{document}
\begin{titlepage}
\setlength{\topmargin}{1.5in}
\begin{center}
\Huge{Physics 3310} \\
\LARGE{Principles of Electricity and Magnetism 1} \\
\Large{Professor Thomas R. Schibli} \\[1cm]

\huge{Homework \#\HWnum}\\[0.5cm]

\large{Joe Becker} \\
\large{SID: 810-07-1484} \\
\large{\due} 

\end{center}

\end{titlepage}



\section{Problem 5.1}
For one mole of argon at room temperature and one atmosphere of pressure we can calculate the energy using the equipartition theorem 
\begin{equation}
U = \frac{f}{2}Nk_BT 
\label{equipart}
\end{equation}
where for argon we have $f=3$ degrees of freedom so the energy is
\begin{align*}
U &= \frac{f}{2}Nk_BT \\
&= \frac{3}{2}(6.022\times10^{23})(1.38\times10^{-23}\unit{J\ K^{-1}})(300\unit{K}) \\
&= 3.74\times10^{3}\unit{J}
\end{align*}
Now we know the entropy of an ideal gas is given by
\begin{equation}
S = Nk_B\left[\ln\left(\frac{V}{N}\left(\frac{4\pi mU}{3Nh^2}\right)^{3/2}\right)+\frac{5}{2}\right]
\label{Sank}
\end{equation}
But we want the volume in terms of the pressure so by the ideal gas law we know that
$$V = \frac{Nk_BT}{P}$$
so equation \ref{Sank} gives
\begin{align*}
S &= Nk_B\left[\ln\left(\frac{Nk_BT}{NP}\left(\frac{4\pi mU}{3Nh^2}\right)^{3/2}\right)+\frac{5}{2}\right]\\
&= Nk_B\left[\ln\left(\frac{k_BT}{P}\left(\frac{4\pi mU}{3Nh^2}\right)^{3/2}\right)+\frac{5}{2}\right]\\
&= Nk_B\left[\ln\left(\frac{(1.38\times10^{-23}\unit{J\ K^{-1}})(300\unit{K})}{1.01\times10^{5}\unit{Pa}}\left(\frac{4\pi (6.63\times10^{-26}\unit{kg})(3.74\times10^{3}\unit{J})}{3(6.022\times10^{23})(6.62\times10^{-34}\unit{J\ s})^2}\right)^{3/2}\right)+\frac{5}{2}\right]\\
&= (6.022\times10^{23})(1.38\times10^{-23}\unit{J\ K^{-1}})\left[\ln\left((4.10\times10^{-26}\unit{m^3})(2.47\times10^{32}\unit{m^{-3}})\right)+\frac{5}{2}\right]\\
&= (6.022\times10^{23})(1.38\times10^{-23}\unit{J\ K^{-1}})(1.86\times10^{1})\\
&= 1.54\times10^{2}\unit{J\ K^{-1}}
\end{align*}
Now we can find the enthalpy by
$$H \equiv U + PV$$
again we use the ideal gas law to relate volume and pressure so
\begin{align*}
H &= U + P\frac{Nk_BT}{P}\\
&= U + Nk_BT\\
&= (3.74\times10^{3}\unit{J}) + (6.022\times10^{23})(1.38\times10^{-23}\unit{J\ K^{-1}})(300\unit{K})\\
&= (3.74\times10^{3}\unit{J}) + (6.022\times10^{23})(1.38\times10^{-23}\unit{J\ K^{-1}})(300\unit{K})\\
&= 6.23\times10^{3}\unit{J}
\end{align*}
We can calculate the \emph{Helmholtz Free Energy} by
\begin{equation}
F\equiv U - TS
\label{Helm}
\end{equation}
which for our argon is
\begin{align*}
F &= U - TS\\
&=  (3.74\times10^{3}\unit{J})- (300\unit{K})(1.54\times10^{2}\unit{J\ K^{-1}})\\
&=  (3.74\times10^{3}\unit{J})- (300\unit{K})(1.54\times10^{2}\unit{J\ K^{-1}})\\
&= -4.25\times10^4\unit{J}
\end{align*}
And we can find the \emph{Gibbs Free Energy} by
\begin{equation}
G\equiv U - TS + PV
\label{Helm}
\end{equation}
Which for this argon is
\begin{align*}
G &= U - TS + Nk_BT\\
&=  (3.74\times10^{3}\unit{J})- (300\unit{K})(1.54\times10^{2}\unit{J\ K^{-1}}) + (6.022\times10^{23})(1.38\times10^{-23}\unit{J\ K^{-1}})(300\unit{K})\\
&= -4.00\times10^{4}\unit{J}
\end{align*}

\section{Problem 5.2}
The production of ammonia from nitrogen and hydrogen given by
$$\textnormal{N}_2 + 3\textnormal{H}_2 \rightarrow 2\textnormal{NH}_3$$
at $298\unit{K}$ and $1\unit{bar}$ has the values $\Delta H = -46.11\unit{kJ}$ and $S = 192.45\unit{J\ K^{-1}}$. We can find $\Delta G$ by
$$\Delta G = \Delta H - T\Delta S$$
so we need to find the change in entropy. We look up that H$_2$ has the entropy of $130.68\unit{J\ K^{-1}}$ and N$_2$ has $S = 191.62\unit{J\ K^{-1}}$ so the total change in entropy is 
$$\Delta S = 2(192.45\unit{J\ K^{-1}}) - 191.62\unit{J\ K^{-1}} - 3(130.68\unit{J\ K^{-1}}) = -198.76\unit{J\ K^{-1}}$$
Note that this is the entropy of creation for two ammonia molecules, so for just one we have $\Delta S = -99.4\unit{J\ K^{-1}}$. So we can calculate $\Delta G$ by
\begin{align*}
\Delta G &= \Delta H - T\Delta S\\
&= -4.611\times10^{4}\unit{J} - (298\unit{K})(-99.4\unit{J\ K^{-1}})\\
&= -1.65\times10^{4}\unit{J}
\end{align*}
Note this is in agreement with the value in the back of the book.

\section{Problem 5.5}
\begin{enumerate}[(a)]
\item
For a fuel cell that uses the reaction
$$\textnormal{CH}_4+2\textnormal{O}_2\rightarrow2\textnormal{H}_2\textnormal{O}+\textnormal{CO}_2$$
note we assume that this reaction takes place at room temperature and at one atmosphere of pressure. We can find the change in the Gibbs Free Energy by taking the $\Delta G$ of the final products and subtracting the initial $\Delta G$
\begin{align*}
\Delta G_{tot} &= \Delta G_{f} - \Delta G_{i}\\
&= \left[\frac{}{}2(\Delta G_{H20}) + \Delta G_{CO2}\right] - \left[\frac{}{}\Delta G_{CH4} + 2(\Delta G_{O2})\right]\\
&= \left[\frac{}{}2(-228.57\unit{kJ}) -394.36\unit{kJ}\right] - \left[\frac{}{}-50.72\unit{kJ} + 2(0\unit{kJ})\right]\\
&= -8.01\times10^{5}\unit{J}
\end{align*}
Note that we assume that we have gaseous water. Now the Helmholtz free energy follows the same way
\begin{align*}
\Delta H_{tot} &= \Delta H_{f} - \Delta H_{i}\\
&= \left[\frac{}{}2(\Delta H_{H20}) + \Delta H_{CO2}\right] - \left[\frac{}{}\Delta H_{CH4} + 2(\Delta H_{O2})\right]\\
&= \left[\frac{}{}2(-241.82\unit{kJ}) - 393.51\unit{kJ}\right] - \left[\frac{}{}-74.81\unit{kJ} + 2(0\unit{kJ})\right]\\
&= -8.02\times10^{5}\unit{J}
\end{align*}

\item
We know that $\Delta G_{tot}$ represents the minimum other work required to make this reaction go. This implies that the electrical work we can get out of the cell is the same as $\Delta G_{tot}$ or $8.01\times10^{5}\unit{J}$ for a mole of methane.

\item
We know that $\Delta G$ is given by
$$\Delta G = \Delta H - T\Delta S$$
where we note that $T\Delta S$ is the heat produced by the system so if we just subtract off the $\Delta H$ term we will find the heat produced for a mole of methane.
\begin{align*}
Q &= \Delta G - \Delta H\\
&= -8.01\times10^{5}\unit{J} - (-8.02\times10^{5}\unit{J})\\
&= 1.00\times10^{3}\unit{J}
\end{align*}

\item
Given that the steps of reaction are
\begin{align*}
\textnormal{at - electrode:}& \ \textnormal{CH}_4+2\textnormal{H}_2\textnormal{O}\rightarrow\textnormal{CO}_2+8\textnormal{H}^++8\textnormal{e}^-\\
\textnormal{at + electrode:}& \ 2\textnormal{O}_2 + 8\textnormal{H}^+ + 8\textnormal{e}^- \rightarrow\textnormal{H}_2\textnormal{O}
\end{align*}
\end{enumerate}
So we see that for each reaction we have 8 electrons moving between the electrodes. This implies that the energy per electron is 
$$\frac{8.01\times10^{5}\unit{J}}{8(6.022\times10^{23})} = 1.66\times10^{-19}\unit{J} = 1.04\unit{eV}$$
and we know that an electron volt is the voltage required to give an electron that much energy so the total voltage is just the same number or $1.04\unit{V}$

\section{Problem 5.9}
See attached for the graph.

\section{Problem 5.10}
We know that the change in the Gibb's Free Energy goes by
$$dG = -SdT-VdP+\mu dN$$
now if we assume that we stay at constant pressure and number of molecules we have
$$dG = -SdT$$
So if we raise the temperature of the water by $\Delta T =5\unit{K}$ where the entropy of one mole of water is $S = 69.91\unit{J\ K^{-1}}$ the change in $G$ is
$$\Delta G = -S\Delta T = -(69.91\unit{J\ K^{-1}})(5\unit{K}) = -3.50\times10^{2}\unit{J}$$
now if we wanted to keep $G$ the same when we raise the temperature (while keeping $N$ constant) we need to increase the pressure so that $V\Delta P$ is the same as $-S\Delta T$ so
\begin{align*}
\Delta G = 0 &= -S\Delta T + V\Delta P\\
&\Downarrow\\
\Delta P &= \frac{S\Delta T}{V}\\ 
&= \frac{(69.91\unit{J\ K^{-1}})(5\unit{K})}{1.8068\times10^{-5}\unit{m^3}}\\ 
&= 1.93\times10^{6}\unit{Pa}
\end{align*}


\end{document}

