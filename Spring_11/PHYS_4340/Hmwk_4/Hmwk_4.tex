\documentclass[11pt]{article}

\usepackage{latexsym}
\usepackage{amssymb}
\usepackage{amsthm}
\usepackage{enumerate}
\usepackage{amsmath}
\usepackage{cancel}
\numberwithin{equation}{section}

\setlength{\evensidemargin}{.25in}
\setlength{\oddsidemargin}{-.25in}
\setlength{\topmargin}{-.75in}
\setlength{\textwidth}{6.5in}
\setlength{\textheight}{9.5in}
\newcommand{\due}{February 10th, 2011}
\newcommand{\HWnum}{4}
\newcommand{\grad}{\bold\nabla}
\newcommand{\vecE}{\vec{E}}
\newcommand{\scrptR}{\vec{\mathfrak{R}}}
\newcommand{\kapa}{\frac{1}{4\pi\epsilon_0}}
\newcommand{\emf}{\mathcal{E}}
\newcommand{\unit}[1]{\ensuremath{\, \mathrm{#1}}}
\newcommand{\real}{\textnormal{Re}}
\newcommand{\Erf}{\textnormal{Erf}}
\newcommand{\sech}{\textnormal{sech}}
\newcommand{\scrO}{\mathcal{O}}
\newcommand{\levi}{\widetilde{\epsilon}}
\newcommand{\partiald}[2]{\ensuremath{\frac{\partial{#1}}{\partial{#2}}}}
\newcommand{\norm}[2]{\langle{#1}|{#2}\rangle}
\newcommand{\inprod}[2]{\langle{#1}|{#2}\rangle}
\newcommand{\average}[1]{\left\langle{#1}\right\rangle}
\newcommand{\ket}[1]{|{#1}\rangle}
\newcommand{\bra}[1]{\langle{#1}|}
\newcommand{\Resid}[2]{\ensuremath{\textnormal{Res}\left[{#1},{#2}\right]}}





\begin{document}
\begin{titlepage}
\setlength{\topmargin}{1.5in}
\begin{center}
\Huge{Physics 3310} \\
\LARGE{Principles of Electricity and Magnetism 1} \\
\Large{Professor Thomas R. Schibli} \\[1cm]

\huge{Homework \#\HWnum}\\[0.5cm]

\large{Joe Becker} \\
\large{SID: 810-07-1484} \\
\large{\due} 

\end{center}

\end{titlepage}



\section{Problem \#1}
\begin{enumerate}[(a)]
\item
For a plane $hkl$ in a crystal lattice we can see that the plane intersects the axis of the crystal cell at the points $\vec{b_1}/h$, $\vec{b_2}/k$, and $\vec{b_3}/l$. So we can say that the vector from the point $\vec{b_1}/h$ to the point $\vec{b_2}/k$ given by 
$$\vec{A} = \frac{\vec{b_1}}{h} - \frac{\vec{b_2}}{k}$$
and the vector from the point $\vec{b_2}/k$ to the point $\vec{b_3}/l$ given by 
$$\vec{B} = \frac{\vec{b_2}}{k} - \frac{\vec{b_3}}{l}$$
Now by showing that these vectors are perpendicular to the reciprocal lattice vector 
$$\vec{G} = h\vec{b_1}+k\vec{b_2}+l\vec{b_3}$$
we show that $\vec{G}$ is perpendicular to the plane $(hkl)$. So we can see that
\begin{align*}
\vec{A}\cdot\vec{G} &= \left(\frac{\vec{b_1}}{h} - \frac{\vec{b_2}}{k}\right)\cdot \left(h\vec{b_1}+k\vec{b_2}+l\vec{b_3}\right)\\
\vec{A}\cdot\vec{G} &= \frac{\vec{b_1}}{h}h\vec{b_1} - \frac{\vec{b_2}}{k}k\vec{b_2}+l\vec{b_3}(0)\\
&= 1 - 1 = 0
\end{align*}
and
\begin{align*}
\vec{B}\cdot\vec{G} &= \left(\frac{\vec{b_2}}{k} - \frac{\vec{b_3}}{l}\right)\cdot \left(h\vec{b_1}+k\vec{b_2}+l\vec{b_3}\right)\\
&= h\vec{b_1}(0) + \frac{\vec{b_2}}{k}k\vec{b_2} - \frac{\vec{b_3}}{l}l\vec{b_3}\\
&= 1 - 1 = 0
\end{align*}
Therefore the reciprocal lattice vector is perpendicular to the plane $(hkl)$.

\item
To find the distance between two parallel planes we want to find the length of the unit normal vector along the point $\vec{b_1}/h$. From part (a) we know that $\vec{G}$ is normal to the plane so we can say the unit normal vector is
$$\hat{n} = \frac{\vec{G}}{|\vec{G}|}$$
So now we can find $d(hkl)$ by calculating $\hat{n}\cdot\vec{a_1}/h$. So
\begin{align*}
d(hkl) &= \hat{n}\cdot\frac{\vec{a_1}}{h}\\
&= \frac{\vec{G}}{|\vec{G}|}\cdot\frac{\vec{a_1}}{h}\\
&= \frac{2\pi}{|\vec{G}|}
\end{align*}

\item
Given that for a simple cubic lattice the primitive reciprocal lattice vectors are
\begin{align*}
\vec{b_1} &= \frac{2\pi}{a}h\hat{x}\\
\vec{b_2} &= \frac{2\pi}{a}k\hat{y}\\
\vec{b_3} &= \frac{2\pi}{a}l\hat{z}
\end{align*}
We can see that the distance squared is given by part (b)
\begin{align*}
d^2 &= \frac{(2\pi)^2}{|\vec{G}|^2}\\
&= (2\pi)^2\frac{a^2}{(2\pi)^2}\frac{1}{h^2+k^2+l^2}\\
&= \frac{a^2}{h^2+k^2+l^2}
\end{align*}
\end{enumerate}

\section{Problem \#2}
\begin{enumerate}[(a)]
\item
Given the primitive translation vectors of the hexagonal space lattice 
\begin{align}
\vec{a_1} &= (3^{1/2}a/2)\hat{x} + (a/2)\hat{y}\\
\vec{a_2} &= -(3^{1/2}a/2)\hat{x} + (a/2)\hat{y}\\
\vec{a_3} &= c\hat{z}
\end{align}
We can find the volume of the primitive cell by
$$\vec{a_1}\cdot(\vec{a_2}\times\vec{a_3})$$
So we can find the cross product as
\begin{align*}
\vec{a_2}\times\vec{a_3} &= \det\left(\begin{array}{ccc}
			\hat{x}		&\hat{y}	&\hat{z}\\
			-(3^{1/2}a/2) 	&a/2		&0\\
			0		&0		&c
				\end{array}\right)\\
&= ac/2\hat{x} + c(3^{1/2}a/2)\hat{y}
\end{align*}
So we can calculate 
\begin{align*}
\vec{a_1}\cdot(\vec{a_2}\times\vec{a_3}) &= ((3^{1/2}a/2)\hat{x} + (a/2)\hat{y})\cdot(ac/2\hat{x} + c(3^{1/2}a/2)\hat{y})\\
&= \frac{3^{1/2}a}{2}\frac{ac}{2} + \frac{a}{2}\frac{c(3^{1/2}a)}{2}\\
&= \frac{3^{1/2}a^2c}{4} + \frac{3^{1/2}a^2c}{4}\\
&= \frac{3^{1/2}a^2c}{2}
\end{align*}

\item
We can find the primitive translations of the reciprocal lattice by the definition 
$$\vec{b_1} = 2\pi\frac{\vec{a_2}\times\vec{a_3}}{\vec{a_1}\cdot(\vec{a_2}\times\vec{a_3})}$$
Note that we found each part in the previous part so it follows that 
\begin{align*}
\vec{b_1} &= 2\pi\frac{\vec{a_2}\times\vec{a_3}}{\vec{a_1}\cdot(\vec{a_2}\times\vec{a_3})}\\
&= 2\pi\frac{ac/2\hat{x} + c(3^{1/2}a/2)\hat{y}}{3^{1/2}a^2c/2}\\
&= \frac{2\pi}{3^{1/2}a}\hat{x} + \frac{2\pi}{a}\hat{y}
\end{align*}
Now to find 
$$\vec{b_2} = 2\pi\frac{\vec{a_3}\times\vec{a_1}}{\vec{a_1}\cdot(\vec{a_2}\times\vec{a_3})}$$
so we can calculate the cross product 
\begin{align*}
\vec{a_3}\times\vec{a_1} &= \det\left(\begin{array}{ccc}
			\hat{x}		&\hat{y}	&\hat{z}\\
			0		&0		&c\\
			(3^{1/2}a/2) 	&a/2		&0
				\end{array}\right)\\
&= -ac/2\hat{x} + c(3^{1/2}a/2)\hat{y}
\end{align*}
So we can see that
\begin{align*}
\vec{b_2} &= 2\pi\frac{\vec{a_3}\times\vec{a_1}}{\vec{a_1}\cdot(\vec{a_2}\times\vec{a_3})}\\
&= 2\pi\frac{-ac/2\hat{x} + c(3^{1/2}a/2)\hat{y}}{3^{1/2}a^2c/2}\\
&= -\frac{2\pi}{3^{1/2}a}\hat{x} + \frac{2\pi}{a}\hat{y}
\end{align*}
Now for
$$\vec{b_3} = 2\pi\frac{\vec{a_1}\times\vec{a_2}}{\vec{a_1}\cdot(\vec{a_2}\times\vec{a_3})}$$
And we calculate the cross product 
\begin{align*}
\vec{a_1}\times\vec{a_2} &= \det\left(\begin{array}{ccc}
			\hat{x}		&\hat{y}	&\hat{z}\\
			(3^{1/2}a/2) 	&a/2		&0\\
			-(3^{1/2}a/2) 	&a/2		&0
				\end{array}\right)\\
&= 3^{1/2}a^2/2\hat{z}
\end{align*}
So we can calculate $\vec{b_3}$ as 
\begin{align*}
\vec{b_3} &= 2\pi\frac{\vec{a_1}\times\vec{a_2}}{\vec{a_1}\cdot(\vec{a_2}\times\vec{a_3})}\\
&= 2\pi\frac{3^{1/2}a^2/2\hat{z}}{3^{1/2}a^2c/2}\\
&= \frac{2\pi}{c}\hat{z}
\end{align*}

\item
See attached

\end{enumerate}

\section{Problem \#3}
To show that the volume of the Brillouin zone is $(2\pi)^3/V_c$ where we define
$$V_c\equiv \vec{a_1}\cdot(\vec{a_2}\times\vec{a_3})$$
So the volume is given by $\vec{b_1}\cdot(\vec{b_2}\times\vec{b_3})$ where $b_n$ are the reciprocal basis vectors given by
\begin{align*}
\vec{b_1} &= 2\pi\frac{\vec{a_2}\times\vec{a_3}}{V_c}\\
\vec{b_2} &= 2\pi\frac{\vec{a_3}\times\vec{a_1}}{V_c}\\
\vec{b_3} &= 2\pi\frac{\vec{a_1}\times\vec{a_2}}{V_c}
\end{align*}
So we calculate the Brillouin zone volume as
\begin{align*}
\vec{b_1}\cdot(\vec{b_2}\times\vec{b_3}) &= \frac{(2\pi)^3}{V_c^3}(\vec{a_2}\times\vec{a_3})\cdot((\vec{a_3}\times\vec{a_1})\times(\vec{a_1}\times\vec{a_2}))\\
&= \frac{(2\pi)^3}{V_c^3}(\vec{a_2}\times\vec{a_3})\cdot(\vec{a_3}\cdot(\vec{a_1}\times\vec{a_2}))\cdot\vec{a_1}\\
&= \frac{(2\pi)^3}{V_c^3}(\vec{a_3}\cdot(\vec{a_1}\times\vec{a_2}))((\vec{a_2}\times\vec{a_3})\cdot\vec{a_1})\\
&= \frac{(2\pi)^3}{V_c^3}V_c^2\\
&= \frac{(2\pi)^3}{V_c}
\end{align*}
Note that we are only concerned with the magnitude $V_c$ so we say that
$$V_c = |\vec{a_3}\cdot(\vec{a_1}\times\vec{a_2})| = |(\vec{a_2}\times\vec{a_3})\cdot\vec{a_1}|$$

\section{Problem \#4}
If we were to construct primitive vectors from the reciprocal vector we can say that
$$\vec{a_1} = 2\pi\frac{\vec{b_2}\times\vec{b_3}}{\vec{b_1}\cdot(\vec{b_2}\times\vec{b_3})}$$
We can show that this is true using the result of problem 3
\begin{align*}
2\pi\frac{\vec{b_2}\times\vec{b_3}}{\vec{b_1}\cdot(\vec{b_2}\times\vec{b_3})} &=2\pi\frac{V_c}{(2\pi)^3}\frac{2\pi}{V_c}(\vec{a_3}\times\vec{a_1})\times\frac{2\pi}{V_c}(\vec{a_1}\times\vec{a_2}) \\
&=\frac{1}{V_c}(\vec{a_3}\times\vec{a_1})\times(\vec{a_1}\times\vec{a_2}) \\
&=\frac{1}{V_c}(\vec{a_3}\cdot(\vec{a_1}\times\vec{a_2}))\vec{a_1} \\
&=\frac{1}{V_c}V_c\vec{a_1} = \vec{a_1}
\end{align*}
\end{document}

