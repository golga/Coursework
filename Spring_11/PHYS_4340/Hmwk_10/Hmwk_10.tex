\documentclass[11pt]{article}

\usepackage{latexsym}
\usepackage{amssymb}
\usepackage{amsthm}
\usepackage{enumerate}
\usepackage{amsmath}
\usepackage{cancel}
\numberwithin{equation}{section}

\setlength{\evensidemargin}{.25in}
\setlength{\oddsidemargin}{-.25in}
\setlength{\topmargin}{-.75in}
\setlength{\textwidth}{6.5in}
\setlength{\textheight}{9.5in}
\newcommand{\due}{April 7th, 2011}
\newcommand{\HWnum}{10}
\newcommand{\grad}{\bold\nabla}
\newcommand{\vecE}{\vec{E}}
\newcommand{\scrptR}{\vec{\mathfrak{R}}}
\newcommand{\kapa}{\frac{1}{4\pi\epsilon_0}}
\newcommand{\emf}{\mathcal{E}}
\newcommand{\unit}[1]{\ensuremath{\, \mathrm{#1}}}
\newcommand{\real}{\textnormal{Re}}
\newcommand{\Erf}{\textnormal{Erf}}
\newcommand{\sech}{\textnormal{sech}}
\newcommand{\scrO}{\mathcal{O}}
\newcommand{\levi}{\widetilde{\epsilon}}
\newcommand{\partiald}[2]{\ensuremath{\frac{\partial{#1}}{\partial{#2}}}}
\newcommand{\norm}[2]{\langle{#1}|{#2}\rangle}
\newcommand{\inprod}[2]{\langle{#1}|{#2}\rangle}
\newcommand{\ket}[1]{|{#1}\rangle}
\newcommand{\bra}[1]{\langle{#1}|}





\begin{document}
\begin{titlepage}
\setlength{\topmargin}{1.5in}
\begin{center}
\Huge{Physics 3320} \\
\LARGE{Principles of Electricity and Magnetism II} \\
\Large{Professor Ana Maria Rey} \\[1cm]

\huge{Homework \#\HWnum}\\[0.5cm]

\large{Joe Becker} \\
\large{SID: 810-07-1484} \\
\large{\due} 

\end{center}

\end{titlepage}



\section{Problem \#1}
\begin{enumerate}[(a)]
\item
We can find the \emph{donor ionization energy} of indium antimonide by
\begin{equation}
E_d = \frac{e^4m}{2\epsilon^2\hbar^2}
\label{IonEn}
\end{equation}
where in CGS units we have
$$E_d = \frac{e^4m}{2\epsilon^2\hbar^2} = \left(\frac{13.6}{\epsilon^2}\frac{m^*}{m}\right)\unit{eV}$$
Note that $m^*$ is the effective mass given as $m^* = 0.015m$ and $\epsilon$ is the dielectric constant given as $\epsilon=18$. So we can calculate $E_d$ for indium antimonide as
\begin{align*}
E_d = \frac{e^4m}{2\epsilon^2\hbar^2} &= \left(\frac{13.6}{\epsilon^2}\frac{m^*}{m}\right)\unit{eV}\\
&= \left(\frac{13.6}{(18)^2}\frac{0.015m}{m}\right)\unit{eV}\\
&= 6.29\times10^{-4}\unit{eV}
\end{align*}

\item
We can find the Bohr radius of the donor by
\begin{equation}
a_d = \frac{\epsilon\hbar^2}{m^*e^2}  
\label{Bohr}
\end{equation}
which in CGS units we can say that
$$a_d = \frac{\epsilon\hbar^2}{m^*e^2} = \left(\frac{0.53\epsilon}{m^*/m}\right)\unit{\AA}$$
where we use the same $m^*$ and $\epsilon$ from part (a). So we calculate
\begin{align*}
a_d = \frac{\epsilon\hbar^2}{m^*e^2} &= \left(\frac{0.53\epsilon}{m^*/m}\right)\unit{\AA}\\
&= \left(\frac{0.53(18)}{0.015}\right)\unit{\AA}\\
&= 6.35\times10^{2}\unit{\AA}
\end{align*}

\item
If we assume that the Bohr radius of the donor $a_d$ we found in part (a) forms a sphere of radius $a_d$ we find its volume to be
\begin{align*}
V &= \frac{4}{3}\pi a_d^3\\
&= \frac{4}{3}\pi (635\unit{\AA})^3\\
&= 1.08\times10^{9}\unit{\AA^3} = 1.08\times10^{-15}\unit{cm^3}
\end{align*}
So if this is the volume occupied by the doped atom, we will have overlap when the concentration of atoms $n$ is equal to the inverse of this volume. So
\begin{align*}
n = \frac{1}{V} &= \frac{1}{1.08\times10^{-15}\unit{cm^3}}\\
&= 9.27\times10^{14}\unit{cm^{-3}}
\end{align*}
So we need a donor concentration of $9.27\times10^{14}$ donors per $\unit{cm^{3}}$.
\end{enumerate}

\section{Problem \#2}
\begin{enumerate}[(a)]
\item
For a semiconductor with $N_d = 10^{13}$ donors/cm$^{3}$ with an ionization energy $E_d = 1\unit{meV}$ and an effective mass of $m^*=0.01m$ we can estimate the concentration of conduction electrons $n$ at $4\unit{K}$ by
\begin{equation}
n \approx (n_0N_d)^{1/2}\exp(-E_d/2k_BT)
\label{EleCon}
\end{equation}
Where $n_0$ is defined as
$$n_0 \equiv 2\left(\frac{m^*k_BT}{2\pi\hbar^2}\right)^{3/2}$$
which for this semiconductor we calculate as
\begin{align*}
n_0 &\equiv 2\left(\frac{m^*k_BT}{2\pi\hbar^2}\right)^{3/2}\\
&= 2\left(\frac{(0.01)(9.11\times10^{-31}\unit{kg})(1.38\times10^{-23}\unit{J\ K^{-1}})(4\unit{K})}{2\pi(1.05\times10^{-34}\unit{J\ s})^2}\right)^{3/2}\\
&= 2\left(\frac{5.03\times10^{-55}\unit{J\ kg}}{6.92\times10^{-68}\unit{J^2\ s^2}}\right)^{3/2}\\
&= 3.92\times10^{19}\unit{m^{-3}} = 3.92\times10^{13}\unit{cm^{-3}}
\end{align*}
Note that we used the mass of an electron to find the effective mass. Now we can calculate equation \ref{EleCon}
\begin{align*}
n &\approx (n_0N_d)^{1/2}\exp(-E_d/2k_BT)\\
&= ((3.92\times10^{13}\unit{cm^{-3}})(10^{13}\unit{cm^{-3}}))^{1/2}\exp(-(1.60\times10^{-22}\unit{J})/2(1.38\times10^{-23}\unit{J\ K^{-1}})(4\unit{K}))\\
&= (1.98\times10^{13}\unit{cm^{-3}})(2.34\times10^{-1})\\
&= 4.65\times10^{12}\unit{cm^{-3}}
\end{align*}

\item
We can use the value for $n$ we calculated in part (a) to find the \emph{Hall coefficient} by the equation
\begin{equation}
R_H = -\frac{1}{nec}
\label{Hall}
\end{equation}
Note that this equation is given in CGS units in SI units we have
$$R_H = -\frac{1}{ne}$$
where $e$ is the elementary charge given as $e = 1.60\times10^{-19}\unit{C}$. So we calculate $R_H$ as
\begin{align*}
R_H = -\frac{1}{ne} &= -\frac{1}{(4.65\times10^{12}\unit{cm^{-3}})(1.60\times10^{-19}\unit{C})} = -1.34\times10^{6}\unit{cm^3\ C^{-1}}
\end{align*}
Note we can calculate this in CGS units where $c=3.00\times10^{10}\unit{cm\ s^{-1}}$ and $e = 4.80\times10^{-10}\unit{esu}$. Note the CGS unit for charge is the esu which is defined as $\unit{g^{1/2}\ cm^{3/2}\ s^{-1}}$. So equation \ref{Hall} yields
$$R_H = -\frac{1}{(4.65\times10^{12}\unit{cm^{-3}})(4.80\times10^{-10}\unit{esu})(3.00\times10^{10}\unit{cm\ s^{-1}})} = 1.49\times10^{-14}\unit{cm^{1/2}\ g^{-1/2}\ s^2}$$
\end{enumerate}

\section{Problem \#3}
See attached for the plots of the Band structures for Si, Ge, and GaAs.



\end{document}

