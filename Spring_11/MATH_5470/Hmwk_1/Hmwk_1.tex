\documentclass[11pt]{article}

\usepackage{latexsym}
\usepackage{amssymb}
\usepackage{amsthm}
\usepackage{enumerate}
\usepackage{amsmath}
\usepackage{cancel}
\numberwithin{equation}{section}

\setlength{\evensidemargin}{.25in}
\setlength{\oddsidemargin}{-.25in}
\setlength{\topmargin}{-.75in}
\setlength{\textwidth}{6.5in}
\setlength{\textheight}{9.5in}
\newcommand{\due}{January 2011}
\newcommand{\HWnum}{1}
\newcommand{\grad}{\bold\nabla}
\newcommand{\vecE}{\vec{E}}
\newcommand{\scrptR}{\vec{\mathfrak{R}}}
\newcommand{\kapa}{\frac{1}{4\pi\epsilon_0}}
\newcommand{\emf}{\mathcal{E}}

\begin{document}
\begin{titlepage}
\setlength{\topmargin}{1.5in}
\begin{center}
\Huge{Physics 3310} \\
\LARGE{Principles of Electricity and Magnetism 1} \\
\Large{Professor Thomas R. Schibli} \\[1cm]

\huge{Homework \#\HWnum}\\[0.5cm]

\large{Joe Becker} \\
\large{SID: 810-07-1484} \\
\large{\due} 

\end{center}

\end{titlepage}



\section{\S 1.1 Exercise \#1}
\begin{enumerate}[(a)]
\item
For the PDE
$$4u_{xx}+2u_{yy} + 27u_y = 0$$
we apply the classification rule 
$$d = AC -B^2$$
where $A=4$ $B=0$ and $C=2$. So it is easy to see that $d>0$ and we have an elliptic PDE.

\item
For
$$4u_{xx}-2u_{yy} = 0$$
we see that 
$$d = AC - B^2 = (4)(-2) - 0^2 = -8$$
so $d<0$ and the PDE is hyperbolic.

\item
For 
$$4u_{xx}-u_{y} = 0$$
we can calcuate $d$ as
$$d = AC - B^2 = (4)(0) - 0^2 = 0$$
So $d=0$ therefore the PDE is parabolic.
\end{enumerate}

\section{\S 1.1 Exercise \#2}
\begin{enumerate}[(a)]
\item
Given that $u = \ln(r)$ and $r = (x^2+y^2)^{1/2}\ne 0$. We will show that 
$$\Delta_2 u = 0$$
where 
$$\Delta_2 u = u_{xx}+u_{yy}$$
So if we solve for the given $u$ we find that
\begin{align*}
\Delta_2 u &= \frac{\partial^2}{\partial x^2}\ln\left((x^2+y^2)^{1/2}\right) + \frac{\partial^2}{\partial y^2}\ln\left((x^2+y^2)^{1/2}\right)\\
&= \frac{\partial^2}{\partial x^2}\frac{1}{2}\ln\left(x^2+y^2\right) + \frac{\partial^2}{\partial y^2}\frac{1}{2}\ln\left(x^2+y^2\right)\\
&= \frac{\partial}{\partial x}(x^2+y^2)^{-1}(x) + \frac{\partial}{\partial y}(x^2+y^2)^{-1}(y)\\
&= -(x^2+y^2)^{-2}(2x^2) + (x^2+y^2)^{-1} - (x^2+y^2)^{-2}(2y^2) + (x^2+y^2)^{-1}\\
&= -2(x^2+y^2)^{-2}\left(x^2+y^2\right)+ 2(x^2+y^2)^{-1}\\
&= -2(x^2+y^2)^{-1} + 2(x^2+y^2)^{-1}\\
&= 0
\end{align*}

\item
For the Gaussian 
$$u = (4\pi t)^{-1/2}e^{-x^2/4t}$$
we will show that is satisfies the Heat equation
$$u_t - u_{xx} = 0$$
by first calculating
\begin{align*}
u_t &= \frac{\partial}{\partial t}(4\pi t)^{-1/2}e^{-x^2/4t} \\
&= e^{-x^2/4t}\frac{\partial}{\partial t}(4\pi t)^{-1/2} + (4\pi t)^{-1/2}\frac{\partial}{\partial t}e^{-x^2/4t} \\
&= -\frac{1}{2}e^{-x^2/4t}(4\pi t)^{-3/2}(4\pi) + (4\pi t)^{-1/2}e^{-x^2/4t}\left(\frac{-x^2}{4}(-t^{-2})\right) \\
&= -2\pi e^{-x^2/4t}(4\pi t)^{-3/2} + (4\pi t)^{-1/2}e^{-x^2/4t}\left(\frac{x^2}{4t^2}\right)
\end{align*}
Next we calculate the 2nd derivative of $u$ with respect to $x$
\begin{align*}
u_{xx} &=  \frac{\partial^2}{\partial x^2}(4\pi t)^{-1/2}e^{-x^2/4t} \\
&=  (4\pi t)^{-1/2}\frac{\partial^2}{\partial x^2}e^{-x^2/4t} \\
&=  (4\pi t)^{-1/2}\frac{\partial}{\partial x}e^{-x^2/4t}\left(\frac{-2x}{4t}\right) \\
&=  -\frac{(4\pi t)^{-1/2}}{2t}\left(x\frac{\partial}{\partial x}e^{-x^2/4t} + e^{-x^2/4t}\frac{\partial}{\partial x}x \right) \\
&=  -\frac{(4\pi t)^{-1/2}}{2t}\left(e^{-x^2/4t}\left(\frac{-x^2}{2t}\right) + e^{-x^2/4t}\right) 
\end{align*}
Now we calculate
\begin{align*}
u_t - u_{xx} &= -2\pi e^{-x^2/4t}(4\pi t)^{-3/2} + (4\pi t)^{-1/2}e^{-x^2/4t}\left(\frac{x^2}{4t^2}\right) + \frac{(4\pi t)^{-1/2}}{2t}\left(e^{-x^2/4t}\left(\frac{-x^2}{2t}\right) + e^{-x^2/4t}\right) \\
&= -2\pi e^{-x^2/4t}(4\pi t)^{-3/2} + (4\pi t)^{-1/2}e^{-x^2/4t}\left(\frac{x^2}{4t^2}\right) - (4\pi t)^{-1/2}e^{-x^2/4t}\left(\frac{x^2}{4t^2}\right) + \frac{(4\pi t)^{-1/2}}{2t}e^{-x^2/4t} \\
&= -2\pi e^{-x^2/4t}(4\pi t)^{-3/2}  + \frac{(4\pi t)^{-1/2}}{2t}e^{-x^2/4t} \\
&= (4\pi t)^{-1/2}e^{-x^2/4t}\left(\frac{-2\pi}{4\pi t}  + \frac{1}{2t}\right)\\
&= (4\pi t)^{-1/2}e^{-x^2/4t}\left(\frac{-1}{2t}  + \frac{1}{2t}\right)\\
&= 0
\end{align*}

\item
For a moving sine wave
$$u = \sin(x+t)$$
we will show that it satisfies the wave equation
$$u_{tt} - u_{xx} = 0$$
by calculating
\begin{align*}
u_{tt} &= \frac{\partial^2}{\partial t^2}\sin(x+t)\\
&= \frac{\partial}{\partial t}\cos(x+t)\\
&= -\sin(x+t)
\end{align*}
and
\begin{align*}
u_{xx} &= \frac{\partial^2}{\partial x^2}\sin(x+t)\\
u_{xx} &= \frac{\partial}{\partial x}\cos(x+t)\\
&= -\sin(x+t)
\end{align*}
So it is obvious to see that
$$u_{tt} - u_{xx} = -\sin(x+t) - (-\sin(x+t)) = 0$$
\end{enumerate}

\section{\S 1.2 Problem \#1}
\begin{enumerate}[(a)]
\item
For the  Dirichlet problem
$$f(x,y) = \left\{\begin{array}{llr}
		x 	&\textnormal{for}\ 0\le x\le1, &y=0\\
		1 	&\textnormal{for}\ x=1, &0\le y\le1\\
		x 	&\textnormal{for}\ 0\le x\le1, &y=1\\
		0 	&\textnormal{for}\ x=0, &0\le y\le1
		\end{array}\right. $$
We can reason that the solution looks like a plane that goes by $f(x,y)=x$ for all $y\in\Omega$. This conclusion follows from the idea that we are in a minimal surface problem, and the minimal surface over the boundary conditions is the plane described.

\item
Now if we change the boundary conditions so that
$$f(x,y) = \left\{\begin{array}{llr}
		x^2 	&\textnormal{for}\ 0\le x\le1, &y=0\\
		1 	&\textnormal{for}\ x=1, &0\le y\le1\\
		x^2 	&\textnormal{for}\ 0\le x\le1, &y=1\\
		0 	&\textnormal{for}\ x=0, &0\le y\le1
		\end{array}\right. $$
the problem is no longer as simple as it is in part (a), this means a plane is not the solution. We can go further and show that a polynomial solution of the form
$$f(x,y) = \sum_{i=0}^{\infty}\sum_{j=0}^{\infty} a_{ij}x^iy^j$$
will not be able to fit this data and solve the Dirichlet problem
$$\Delta_2 f(x,y) = 0$$
which we can show by
\begin{align*}
\Delta_2 f(x,y) &=  \frac{\partial^2}{\partial x^2}f(x,y) + \frac{\partial^2}{\partial y^2}f(x,y)\\
\Delta_2 f(x,y) &=  \frac{\partial^2}{\partial x^2}\sum_{i=0}^{\infty}\sum_{j=0}^{\infty} a_{ij}x^iy^j + \frac{\partial^2}{\partial y^2}\sum_{i=0}^{\infty}\sum_{j=0}^{\infty} a_{ij}x^iy^j\\
\Delta_2 f(x,y) &=  \sum_{i=0}^{\infty}\sum_{j=0}^{\infty} a_{ij}(i)\frac{\partial}{\partial x}x^{i-1}y^j + \sum_{i=0}^{\infty}\sum_{j=0}^{\infty} a_{ij}(j)x^i\frac{\partial}{\partial y}y^{j-1}\\
\Delta_2 f(x,y) &=  \sum_{i=0}^{\infty}\sum_{j=0}^{\infty} a_{ij}(i)(i-1)x^{i-2}y^j +  a_{ij}(j)(j-1)x^iy^{j-2}
\end{align*}
Now we see that the only way to make this sum zero is to make $x$ or $y$ to be zero and this is the trivial solution.

\item
\end{enumerate}

\section{\S 1.3 Problem \#2}

\section{\S 1.4 Problem \#1}
\begin{enumerate}[(a)]
\item
\item
\item
\end{enumerate}

\section{\S 1.4 Problem \#2}
\begin{enumerate}[(a)]
\item
\item
\item
\end{enumerate}

\section{\S 1.4 Problem \#3}
\begin{enumerate}[(a)]
\item
\item
\item
\end{enumerate}

\end{document}

