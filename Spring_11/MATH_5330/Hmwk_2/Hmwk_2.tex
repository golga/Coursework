\documentclass[11pt]{article}

\usepackage{latexsym}
\usepackage{amssymb}
\usepackage{amsthm}
\usepackage{enumerate}
\usepackage{amsmath}
\usepackage{cancel}
\numberwithin{equation}{section}

\setlength{\evensidemargin}{.25in}
\setlength{\oddsidemargin}{-.25in}
\setlength{\topmargin}{-.75in}
\setlength{\textwidth}{6.5in}
\setlength{\textheight}{9.5in}
\newcommand{\due}{February 4th, 2011}
\newcommand{\HWnum}{2}
\newcommand{\grad}{\bold\nabla}
\newcommand{\vecE}{\vec{E}}
\newcommand{\scrptR}{\vec{\mathfrak{R}}}
\newcommand{\kapa}{\frac{1}{4\pi\epsilon_0}}
\newcommand{\emf}{\mathcal{E}}
\newcommand{\unit}[1]{\ensuremath{\, \mathrm{#1}}}
\newcommand{\real}{\textnormal{Re}}
\newcommand{\Erf}{\textnormal{Erf}}
\newcommand{\sech}{\textnormal{sech}}
\newcommand{\scrO}{\mathcal{O}}
\newcommand{\levi}{\widetilde{\epsilon}}
\newcommand{\partiald}[2]{\ensuremath{\frac{\partial{#1}}{\partial{#2}}}}
\newcommand{\norm}[2]{\langle{#1}|{#2}\rangle}
\newcommand{\inprod}[2]{\langle{#1}|{#2}\rangle}
\newcommand{\ket}[1]{|{#1}\rangle}
\newcommand{\bra}[1]{\langle{#1}|}





\begin{document}
\begin{titlepage}
\setlength{\topmargin}{1.5in}
\begin{center}
\Huge{Physics 3320} \\
\LARGE{Principles of Electricity and Magnetism II} \\
\Large{Professor Ana Maria Rey} \\[1cm]

\huge{Homework \#\HWnum}\\[0.5cm]

\large{Joe Becker} \\
\large{SID: 810-07-1484} \\
\large{\due} 

\end{center}

\end{titlepage}



\section{Problem \#1}
To find the \emph{Fourier Coefficients} $\hat{f}(n)$ for a given function $f(\theta)$ we use the formula
\begin{equation}
\hat{f}(n) = \frac{1}{2\pi}\int_{-\pi}^{\pi}f(\theta)e^{-in\theta}d\theta
\label{Four}
\end{equation}
So for $f(\theta) = \theta$ equation \ref{Four} becomes
\begin{align*}
\hat{f}(n) &= \frac{1}{2\pi}\int_{-\pi}^{\pi}\theta e^{-in\theta}d\theta
\end{align*}
To solve this we use \emph{Integration by parts} where
\begin{align*}
u = \theta\ \  &dv = e^{-in\theta}d\theta\\
du = d\theta\ \  &v = -\frac{1}{in}e^{-in\theta}
\end{align*}
So if we follow the formula for \emph{Integration by Parts}
\begin{equation}
\int udv = uv - \int vdu
\label{Int}
\end{equation}
we can say that
\begin{align*}
\hat{f}(n) &= \frac{1}{2\pi}\int_{-\pi}^{\pi}\theta e^{-in\theta}d\theta\\
&= \frac{1}{2\pi}\left(-\frac{1}{in}\theta e^{-in\theta} - \frac{-1}{in}\int_{-\pi}^{\pi}e^{-in\theta}d\theta\right)\\
&= \frac{1}{2\pi}\left(\left.-\frac{1}{in}\theta e^{-in\theta}\right|_{-\pi}^{\pi} - \frac{-1}{in}\int_{-\pi}^{\pi}e^{-in\theta}d\theta\right)\\
&= \frac{1}{2\pi}\left(-\frac{1}{in}\left((\pi)e^{-in\pi} - (-\pi)e^{in\pi}\right) + \left(\frac{1}{in}\frac{-1}{in}e^{-in\theta}\right|_{-\pi}^{\pi}\right)\\
&= \frac{1}{2\pi}\left(-\frac{\pi}{in}\left(e^{-in\pi} + e^{in\pi}\right) + \frac{1}{n^2}\left(e^{-in\pi} - e^{in\pi}\right)\right)\\
&= \frac{1}{2\pi}\left(-\frac{2\pi}{in}\cos(n\pi) + \frac{2i}{n^2}\sin(n\pi)\right)
\end{align*}
Now we now that for all $n=1,2,3...$ that $\sin(n\pi) = 0$ and that $\cos(n\pi)$ alternates between $1$ and $-1$ for odd and even $n$ respectively. We can call this $(-1)^n$ so we see that
\begin{align*}
\hat{f}(n) &= \frac{1}{2\pi}\left(-\frac{2\pi}{in}\cos(n\pi) + \cancelto{0}{\frac{2i}{n^2}\sin(n\pi)}\right)\\
&= -\frac{1}{2\pi}\frac{2\pi}{in}(-1)^n\\
&= \frac{1}{in}(-1)(-1)^n\\
&= \frac{(-1)^{n+1}}{in}
\end{align*}
So the Fourier series is
$$f(\theta) = \sum_{n=-\infty}^{\infty}\frac{(-1)^{n+1}}{in}e^{in\theta}$$

\section{Problem \#2}
For $f(\theta) = |\theta|$ we calculate the Fourier coefficient by equation \ref{Four}, but we need to make $f(\theta)$ into a piecewise function 
$$f(\theta) = \left\{\begin{array}{lc}
		\theta	&\textnormal{for}\ \theta>0\\
		-\theta	&\textnormal{for}\ \theta<0
			\end{array}\right.$$
\begin{align*}
\hat{f}(n) &= \frac{1}{2\pi}\int_{-\pi}^{\pi}f(\theta)e^{-in\theta}d\theta\\
&= \frac{1}{2\pi}\int_{-\pi}^{0}-\theta e^{-in\theta}d\theta + \frac{1}{2\pi}\int_{0}^{\pi}\theta e^{-in\theta}d\theta
\end{align*}
Now we need to do integration by parts for each integral. So for the first integral we let
\begin{align*}
u = \theta\ \  &dv = e^{-in\theta}d\theta\\
du = d\theta\ \  &v = -\frac{1}{in}e^{-in\theta}
\end{align*}
So it follows from equation \ref{Int} 
\begin{align*}
-\frac{1}{2\pi}\int_{-\pi}^{0}\theta e^{-in\theta}d\theta &= -\frac{1}{2\pi}\left(\left.\frac{-1}{in}\theta e^{-in\theta}\right|_{-\pi}^{0}-\int_{-\pi}^{0}e^{-in\theta}d\theta \right)\\
&= -\frac{1}{2\pi}\left(\frac{-1}{in}\left((0)e^{-in(0)} - (-\pi)e^{in\pi}\right)-\frac{-1}{in}\left(e^{-in\theta}\right|_{-\pi}^{0} \right)\\
&= -\frac{1}{2\pi}\left(\frac{-\pi}{in}e^{in\pi}-\frac{-1}{in}\left(e^{-in(0)} - e^{in\pi}\right)\right)\\
&= -\frac{1}{2\pi}\left(\frac{-\pi}{in}e^{in\pi}-\frac{-1}{in}\left(1 - e^{in\pi}\right)\right)\\
&= \frac{1}{2\pi}\frac{1}{in}\left(\pi e^{in\pi}+ e^{in\pi}- 1 \right)
\end{align*}
Note for the second integral we choose the same variables for integration by parts
\begin{align*}
u = \theta\ \  &dv = e^{-in\theta}d\theta\\
du = d\theta\ \  &v = -\frac{1}{in}e^{-in\theta}
\end{align*}
So equation \ref{Int} yields
\begin{align*}
\frac{1}{2\pi}\int_{0}^{\pi}\theta e^{-in\theta}d\theta &= \frac{1}{2\pi}\left(\left.\frac{-1}{in}\theta e^{-in\theta}\right|_{0}^{\pi}-\int_{0}^{\pi}e^{-in\theta}d\theta \right)\\
&= \frac{1}{2\pi}\left(\frac{-1}{in}\left(\pi e^{-in\pi} - (0)e^{-in(0)}\right) - \frac{-1}{in}\left(e^{-in\theta}\right|_{0}^{\pi} \right)\\
&= \frac{1}{2\pi}\left(\frac{-\pi}{in} e^{-in\pi} - \frac{-1}{in}\left(e^{-in\pi} - e^{-in(0)}\right)\right)\\
&= \frac{1}{2\pi}\left(\frac{-\pi}{in} e^{-in\pi} - \frac{-1}{in}\left(e^{-in\pi} - 1\right)\right)\\
&= -\frac{1}{2\pi}\frac{1}{in}\left(\pi e^{-in\pi} - e^{-in\pi} + 1\right)\\
\end{align*}
Now we can find the sum of these two parts to find the total integral
\begin{align*}
\hat{f}(n) &= \frac{1}{2\pi}\int_{-\pi}^{0}-\theta e^{-in\theta}d\theta + \frac{1}{2\pi}\int_{0}^{\pi}\theta e^{-in\theta}d\theta\\
&= \frac{1}{2\pi}\frac{1}{in}\left(\pi e^{in\pi}+ e^{in\pi}- 1 \right) - \frac{1}{2\pi}\frac{1}{in}\left(\pi e^{-in\pi} - e^{-in\pi} + 1\right)\\
&= \frac{1}{2\pi}\frac{1}{in}\left(\pi e^{in\pi}+ e^{in\pi}- 1 - \pi e^{-in\pi} + e^{-in\pi} - 1\right)\\
&= \frac{1}{2\pi}\frac{1}{in}\left(\pi \left(e^{in\pi}-e^{-in\pi}\right) + e^{in\pi} + e^{-in\pi} - 2\right)\\
&= \frac{1}{2\pi}\frac{1}{in}\left(\cancelto{0}{2i\pi\sin(n\pi)} + 2\cos(n\pi) - 2\right)\\
&= \frac{1}{2\pi}\frac{2}{in}\left((-1)^n - 1\right)\\
&= \frac{(-1)^n - 1}{in\pi}
\end{align*}
So the Fourier series is 
$$f(\theta) = \sum_{n=-\infty}^{\infty}\frac{(-1)^n - 1}{in\pi}e^{in\theta}$$
note that for $n\rightarrow$ even $\hat{f}(n) = 0$.
\end{document}
