\documentclass[11pt]{article}

\usepackage{latexsym}
\usepackage{amssymb}
\usepackage{amsthm}
\usepackage{enumerate}
\usepackage{amsmath}
\usepackage{cancel}
\numberwithin{equation}{section}

\setlength{\evensidemargin}{.25in}
\setlength{\oddsidemargin}{-.25in}
\setlength{\topmargin}{-.75in}
\setlength{\textwidth}{6.5in}
\setlength{\textheight}{9.5in}
\newcommand{\due}{March 1st, 2011}
\newcommand{\HWnum}{4}
\newcommand{\grad}{\bold\nabla}
\newcommand{\vecE}{\vec{E}}
\newcommand{\scrptR}{\vec{\mathfrak{R}}}
\newcommand{\kapa}{\frac{1}{4\pi\epsilon_0}}
\newcommand{\emf}{\mathcal{E}}
\newcommand{\unit}[1]{\ensuremath{\, \mathrm{#1}}}
\newcommand{\real}{\textnormal{Re}}
\newcommand{\Erf}{\textnormal{Erf}}
\newcommand{\sech}{\textnormal{sech}}
\newcommand{\scrO}{\mathcal{O}}
\newcommand{\levi}{\widetilde{\epsilon}}
\newcommand{\partiald}[2]{\ensuremath{\frac{\partial{#1}}{\partial{#2}}}}
\newcommand{\norm}[2]{\langle{#1}|{#2}\rangle}
\newcommand{\inprod}[2]{\langle{#1}|{#2}\rangle}
\newcommand{\average}[1]{\left\langle{#1}\right\rangle}
\newcommand{\ket}[1]{|{#1}\rangle}
\newcommand{\bra}[1]{\langle{#1}|}
\newcommand{\Resid}[2]{\ensuremath{\textnormal{Res}\left[{#1},{#2}\right]}}





\begin{document}
\begin{titlepage}
\setlength{\topmargin}{1.5in}
\begin{center}
\Huge{Physics 3310} \\
\LARGE{Principles of Electricity and Magnetism 1} \\
\Large{Professor Thomas R. Schibli} \\[1cm]

\huge{Homework \#\HWnum}\\[0.5cm]

\large{Joe Becker} \\
\large{SID: 810-07-1484} \\
\large{\due} 

\end{center}

\end{titlepage}



\section{Problem \HWnum.1}
\begin{enumerate}[(a)]
\item
If the \emph{covariant derivative} defined as
\begin{equation}
\grad_{\mu}V^{\nu} \equiv \partial_{\mu}V^{\nu} + \Gamma^{\nu}_{\mu\rho}V^{\rho}
\label{CovarDer}
\end{equation}
is a rank $(1,1)$ tensor if it follows the tensor transformation law. So for the covariant derivative in the primed coordinates has to transform like 
$$\grad_{\mu'}V^{\nu'} = \partiald{x^{\alpha}}{x^{\mu'}}\partiald{x^{\nu'}}{x^{\beta}}\grad_{\alpha}V^{\beta}$$
Note that the partial derivative acting on a vector does not follow this transformation rule (this is the motivation for creating the covariant derivative). The transformation of the partial derivative is
$$\partial_{\mu'}V^{\nu'} = \partiald{x^{\alpha}}{x^{\mu'}}\partiald{x^{\nu'}}{x^{\beta}}\partial_{\alpha}V^{\beta} + \partiald{x^{\alpha}}{x^{\mu'}}\frac{\partial^2x^{\nu'}}{\partial x^{\alpha}\partial x^{\beta}}V^{\beta}$$
Note that this does not follow the tensor transformation law. Now if we assume that the covariant derivative is a rank $(1,1)$ tensor combined with equation \ref{CovarDer} we can see that
\begin{align*}
\grad_{\mu'}V^{\nu'} &= \partiald{x^{\alpha}}{x^{\mu'}}\partiald{x^{\nu'}}{x^{\beta}}\grad_{\alpha}V^{\beta}\\
&= \partiald{x^{\alpha}}{x^{\mu'}}\partiald{x^{\nu'}}{x^{\beta}}\left(\partial_{\alpha}V^{\beta} + \Gamma^{\beta}_{\alpha\rho}V^{\rho}\right)\\
&= \partiald{x^{\alpha}}{x^{\mu'}}\partiald{x^{\nu'}}{x^{\beta}}\partial_{\alpha}V^{\beta} + \partiald{x^{\alpha}}{x^{\mu'}}\partiald{x^{\nu'}}{x^{\beta}}\Gamma^{\beta}_{\alpha\rho}V^{\rho}
\end{align*}
Now if we look at the right hand side of the equation we can see that by the definition of a covariant derivative 
\begin{align*}
\grad_{\mu'}V^{\nu'} &= \partial_{\mu'}V^{\nu'} + \Gamma^{\nu'}_{\mu'\rho'}V^{\rho'}\\
&= \partiald{x^{\alpha}}{x^{\mu'}}\partiald{x^{\nu'}}{x^{\beta}}\partial_{\alpha}V^{\beta} + \partiald{x^{\alpha}}{x^{\mu'}}\frac{\partial^2x^{\nu'}}{\partial x^{\alpha}\partial x^{\beta}}V^{\beta} + \Gamma^{\nu'}_{\mu'\rho'}V^{\rho'}
\end{align*}
Now we can equate these two expressions as they both are equal to $\grad_{\mu'}V^{\nu'}$ so
\begin{align*}
 \cancel{\partiald{x^{\alpha}}{x^{\mu'}}\partiald{x^{\nu'}}{x^{\beta}}\partial_{\alpha}V^{\beta}} + \partiald{x^{\alpha}}{x^{\mu'}}\frac{\partial^2x^{\nu'}}{\partial x^{\alpha}\partial x^{\beta}}V^{\beta} + \Gamma^{\nu'}_{\mu'\rho'}V^{\rho'} &= \cancel{\partiald{x^{\alpha}}{x^{\mu'}}\partiald{x^{\nu'}}{x^{\beta}}\partial_{\alpha}V^{\beta}} + \partiald{x^{\alpha}}{x^{\mu'}}\partiald{x^{\nu'}}{x^{\beta}}\Gamma^{\beta}_{\alpha\rho}V^{\rho}\\
 \partiald{x^{\alpha}}{x^{\mu'}}\frac{\partial^2x^{\nu'}}{\partial x^{\alpha}\partial x^{\beta}}V^{\beta} + \Gamma^{\nu'}_{\mu'\rho'}\partiald{x^{\rho'}}{x^{\rho}}V^{\rho} &= \partiald{x^{\alpha}}{x^{\mu'}}\partiald{x^{\nu'}}{x^{\beta}}\Gamma^{\beta}_{\alpha\rho}V^{\rho}
\end{align*}
Note that if we rename the dummy index $\beta$ in the first term to $\rho$ we can cancel out the vector $V$.
\begin{align*}
 \partiald{x^{\alpha}}{x^{\mu'}}\frac{\partial^2x^{\nu'}}{\partial x^{\alpha}\partial x^{\rho}}\cancel{V^{\rho}} + \Gamma^{\nu'}_{\mu'\rho'}\partiald{x^{\rho'}}{x^{\rho}}\cancel{V^{\rho}} &= \partiald{x^{\alpha}}{x^{\mu'}}\partiald{x^{\nu'}}{x^{\beta}}\Gamma^{\beta}_{\alpha\rho}\cancel{V^{\rho}}\\
 \partiald{x^{\alpha}}{x^{\mu'}}\frac{\partial^2x^{\nu'}}{\partial x^{\alpha}\partial x^{\rho}} + \Gamma^{\nu'}_{\mu'\rho'}\partiald{x^{\rho'}}{x^{\rho}} &= \partiald{x^{\alpha}}{x^{\mu'}}\partiald{x^{\nu'}}{x^{\beta}}\Gamma^{\beta}_{\alpha\rho}\\
&\Downarrow\\
  \Gamma^{\nu'}_{\mu'\rho'}\partiald{x^{\rho'}}{x^{\rho}} &= \partiald{x^{\alpha}}{x^{\mu'}}\partiald{x^{\nu'}}{x^{\beta}}\Gamma^{\beta}_{\alpha\rho} - \partiald{x^{\alpha}}{x^{\mu'}}\frac{\partial^2x^{\nu'}}{\partial x^{\alpha}\partial x^{\rho}}\\
  \Gamma^{\nu'}_{\mu'\rho'}\partiald{x^{\rho}}{x^{\rho'}}\partiald{x^{\rho'}}{x^{\rho}} &= \partiald{x^{\rho}}{x^{\rho'}}\left(\partiald{x^{\alpha}}{x^{\mu'}}\partiald{x^{\nu'}}{x^{\beta}}\Gamma^{\beta}_{\alpha\rho} - \partiald{x^{\alpha}}{x^{\mu'}}\frac{\partial^2x^{\nu'}}{\partial x^{\alpha}\partial x^{\rho}}\right)\\
  \Gamma^{\nu'}_{\mu'\rho'} &= \partiald{x^{\alpha}}{x^{\mu'}}\partiald{x^{\nu'}}{x^{\beta}}\partiald{x^{\rho}}{x^{\rho'}}\Gamma^{\beta}_{\alpha\rho} - \partiald{x^{\alpha}}{x^{\mu'}}\partiald{x^{\rho}}{x^{\rho'}}\frac{\partial^2x^{\nu'}}{\partial x^{\alpha}\partial x^{\rho}}
\end{align*}
So we have found the condition for $\grad_{\mu}V^{\nu}$ to be a rank $(1,1)$ tensor. The connection must transform by
$$\Gamma^{\nu'}_{\mu'\rho'} = \partiald{x^{\alpha}}{x^{\mu'}}\partiald{x^{\nu'}}{x^{\beta}}\partiald{x^{\gamma}}{x^{\rho'}}\Gamma^{\beta}_{\alpha\gamma} - \partiald{x^{\alpha}}{x^{\mu'}}\partiald{x^{\beta}}{x^{\rho'}}\frac{\partial^2x^{\nu'}}{\partial x^{\alpha}\partial x^{\beta}}$$
Note that we changed the dummy indices from $\rho$ to $\gamma$ on the first term and from $\rho$ to $\beta$ to avoid confusion with the primed and unprimed coordinates.

\item
We can show that the difference of two different connections given by
$$S^{\nu}_{\ \mu\lambda} = \Gamma^{\nu}_{\mu\lambda} - \tilde{\Gamma}^{\nu}_{\mu\lambda}$$
is a tensor by proving that $S^{\nu}_{\ \mu\lambda}$ transforms by the tensor transformation law. So lets start with 
$$S^{\nu'}_{\ \mu'\lambda'} = \Gamma^{\nu'}_{\mu'\lambda'} - \tilde{\Gamma'}^{\nu}_{\mu'\lambda'}$$
where we use the result from part (a), so we can say that
\begin{align*}
\Gamma^{\nu'}_{\mu'\lambda'} = \partiald{x^{\alpha}}{x^{\mu'}}\partiald{x^{\nu'}}{x^{\beta}}\partiald{x^{\gamma}}{x^{\lambda'}}\Gamma^{\beta}_{\alpha\gamma} - \partiald{x^{\alpha}}{x^{\mu'}}\partiald{x^{\beta}}{x^{\lambda'}}\frac{\partial^2x^{\nu'}}{\partial x^{\alpha}\partial x^{\beta}}\\
\tilde{\Gamma}^{\nu'}_{\mu'\lambda'} = \partiald{x^{\alpha}}{x^{\mu'}}\partiald{x^{\nu'}}{x^{\beta}}\partiald{x^{\gamma}}{x^{\lambda'}}\tilde{\Gamma}^{\beta}_{\alpha\gamma} - \partiald{x^{\alpha}}{x^{\mu'}}\partiald{x^{\beta}}{x^{\lambda'}}\frac{\partial^2x^{\nu'}}{\partial x^{\alpha}\partial x^{\beta}}
\end{align*}
Now we can find $S^{\nu'}_{\ \mu'\lambda'}$ by
\begin{align*}
S^{\nu'}_{\ \mu'\lambda'} &= \Gamma^{\nu'}_{\mu'\lambda'} - \tilde{\Gamma'}^{\nu}_{\mu'\lambda'}\\
&= \partiald{x^{\alpha}}{x^{\mu'}}\partiald{x^{\nu'}}{x^{\beta}}\partiald{x^{\gamma}}{x^{\lambda'}}\Gamma^{\beta}_{\alpha\gamma} - \partiald{x^{\alpha}}{x^{\mu'}}\partiald{x^{\beta}}{x^{\lambda'}}\frac{\partial^2x^{\nu'}}{\partial x^{\alpha}\partial x^{\beta}} - \partiald{x^{\alpha}}{x^{\mu'}}\partiald{x^{\nu'}}{x^{\beta}}\partiald{x^{\gamma}}{x^{\lambda'}}\tilde{\Gamma}^{\beta}_{\alpha\gamma} + \partiald{x^{\alpha}}{x^{\mu'}}\partiald{x^{\beta}}{x^{\lambda'}}\frac{\partial^2x^{\nu'}}{\partial x^{\alpha}\partial x^{\beta}}\\
&= \partiald{x^{\alpha}}{x^{\mu'}}\partiald{x^{\nu'}}{x^{\beta}}\partiald{x^{\gamma}}{x^{\lambda'}}\Gamma^{\beta}_{\alpha\gamma} - \partiald{x^{\alpha}}{x^{\mu'}}\partiald{x^{\nu'}}{x^{\beta}}\partiald{x^{\gamma}}{x^{\lambda'}}\tilde{\Gamma}^{\beta}_{\alpha\gamma}\\ 
&= \partiald{x^{\alpha}}{x^{\mu'}}\partiald{x^{\nu'}}{x^{\beta}}\partiald{x^{\gamma}}{x^{\lambda'}}\left(\Gamma^{\beta}_{\alpha\gamma} - \tilde{\Gamma}^{\beta}_{\alpha\gamma}\right)\\ 
&= \partiald{x^{\alpha}}{x^{\mu'}}\partiald{x^{\nu'}}{x^{\beta}}\partiald{x^{\gamma}}{x^{\lambda'}}S^{\beta}_{\ \alpha\gamma}
\end{align*}
So we see that $S^{\nu}_{\ \mu\lambda}$ transforms by the tensor transformation law. This implies the fact that $S^{\nu}_{\ \mu\lambda}$ is a tensor of rank $(1,2)$.

\item
For any parameter $\lambda$ that satisfies the \emph{Geodesic equation}
\begin{equation}
\frac{d^2x^{\mu}}{d\lambda^2} + \Gamma^{\mu}_{\nu\rho}\frac{dx^{\nu}}{d\lambda}\frac{dx^{\rho}}{d\lambda} = 0
\label{GeoDes}
\end{equation}
we can show that this satisfies the \emph{affine parameter}
\begin{equation}
\frac{d}{d\lambda}\left(g_{\mu\nu}\frac{dx^{\mu}}{d\lambda}\frac{dx^{\nu}}{d\lambda}\right) = 0
\label{GeoDes}
\end{equation}
by using the chain rule
\begin{align*}
\frac{d}{d\lambda}\left(g_{\mu\nu}\frac{dx^{\mu}}{d\lambda}\frac{dx^{\nu}}{d\lambda}\right) &= \frac{dx^{\mu}}{d\lambda}\frac{dx^{\nu}}{d\lambda}\frac{d}{d\lambda}g_{\mu\nu} + g_{\mu\nu}\frac{d^2x^{\mu}}{d\lambda^2}\frac{dx^{\nu}}{d\lambda} + g_{\mu\nu}\frac{dx^{\mu}}{d\lambda}\frac{d^2x^{\nu}}{d\lambda^2}\\
&= \frac{dx^{\mu}}{d\lambda}\frac{dx^{\nu}}{d\lambda}\frac{d}{d\lambda}g_{\mu\nu} + g_{\mu\nu}\frac{d^2x^{\mu}}{d\lambda^2}\frac{dx^{\nu}}{d\lambda} + g_{\nu\mu}\frac{dx^{\nu}}{d\lambda}\frac{d^2x^{\mu}}{d\lambda^2}
\end{align*}
Note that we renamed the indices in the last term so that $\mu\rightarrow\nu$ and $\nu\rightarrow\mu$. Now we use the fact that the metric $g_{\mu\nu}$ is symmetric so $g_{\mu\nu} = g_{\nu\mu}$. So it follows that
\begin{align*}
\frac{d}{d\lambda}\left(g_{\mu\nu}\frac{dx^{\mu}}{d\lambda}\frac{dx^{\nu}}{d\lambda}\right) &= \frac{dx^{\mu}}{d\lambda}\frac{dx^{\nu}}{d\lambda}\frac{d}{d\lambda}g_{\mu\nu} + 2g_{\mu\nu}\frac{d^2x^{\mu}}{d\lambda^2}\frac{dx^{\nu}}{d\lambda} 
\end{align*}
Now we can use equation \ref{GeoDes} to replace the $d^2x^{\mu}/d\lambda^2$ so that
\begin{align*}
2g_{\mu\alpha}\frac{d^2x^{\mu}}{d\lambda^2}\frac{dx^{\alpha}}{d\lambda} &= -2g_{\mu\alpha}\Gamma^{\mu}_{\nu\rho}\frac{dx^{\nu}}{d\lambda}\frac{dx^{\rho}}{d\lambda}\frac{dx^{\alpha}}{d\lambda}
\end{align*}
Note that we changed the dummy index from $\nu$ to $\alpha$ so that we only have one pair of $\nu$ indices.  Now we replace the connection with the \emph{Christoffel connection} given by
$$\Gamma^{\mu}_{\nu\rho} = \frac{1}{2}g^{\mu\sigma}\left(\partial_{\nu}g_{\sigma\rho} + \partial_{\rho}g_{\nu\sigma} - \partial_{\sigma}g_{\nu\rho}\right)$$
So it follows that
\begin{align*}
2g_{\mu\alpha}\frac{d^2x^{\mu}}{d\lambda^2}\frac{dx^{\alpha}}{d\lambda} &= -2g_{\mu\alpha}\frac{1}{2}g^{\mu\sigma}\left(\partial_{\nu}g_{\sigma\rho} + \partial_{\rho}g_{\nu\sigma} - \partial_{\sigma}g_{\nu\rho}\right)\frac{dx^{\nu}}{d\lambda}\frac{dx^{\rho}}{d\lambda}\frac{dx^{\alpha}}{d\lambda}\\
&= -\frac{dx^{\nu}}{d\lambda}\frac{dx^{\rho}}{d\lambda}\frac{dx^{\alpha}}{d\lambda}\delta_{\alpha}^{\sigma}\left(\partial_{\nu}g_{\sigma\rho} + \partial_{\rho}g_{\nu\sigma} - \partial_{\sigma}g_{\nu\rho}\right)
\end{align*}
Note that the metric contracted over on index with the inverse metric is the \emph{Kronecker delta}. This implies that $\sigma = \alpha$ so we can say that
\begin{align*}
2g_{\mu\alpha}\frac{d^2x^{\mu}}{d\lambda^2}\frac{dx^{\alpha}}{d\lambda} &= -\frac{dx^{\nu}}{d\lambda}\frac{dx^{\rho}}{d\lambda}\frac{dx^{\alpha}}{d\lambda}\delta_{\alpha}^{\sigma}\left(\partial_{\nu}g_{\sigma\rho} + \partial_{\rho}g_{\nu\sigma} - \partial_{\sigma}g_{\nu\rho}\right)\\
&=  -\frac{dx^{\nu}}{d\lambda}\frac{dx^{\rho}}{d\lambda}\frac{dx^{\alpha}}{d\lambda}\left(\partial_{\nu}g_{\alpha\rho} + \partial_{\rho}g_{\nu\alpha} - \partial_{\alpha}g_{\nu\rho}\right)\\
&=  -\frac{dx^{\nu}}{d\lambda}\frac{dx^{\rho}}{d\lambda}\frac{dx^{\alpha}}{d\lambda}\partial_{\nu}g_{\alpha\rho} - \frac{dx^{\nu}}{d\lambda}\frac{dx^{\rho}}{d\lambda}\frac{dx^{\alpha}}{d\lambda}\partial_{\rho}g_{\nu\alpha} + \frac{dx^{\nu}}{d\lambda}\frac{dx^{\rho}}{d\lambda}\frac{dx^{\alpha}}{d\lambda}\partial_{\alpha}g_{\nu\rho}
\end{align*}
Now we can rename the indices of the last term so that $\alpha\rightarrow\nu$, and $\nu\rightarrow\alpha$ so we see that we can cancel terms
\begin{align*}
2g_{\mu\alpha}\frac{d^2x^{\mu}}{d\lambda^2}\frac{dx^{\alpha}}{d\lambda} &=  -\frac{dx^{\nu}}{d\lambda}\frac{dx^{\rho}}{d\lambda}\frac{dx^{\alpha}}{d\lambda}\partial_{\nu}g_{\alpha\rho} - \frac{dx^{\nu}}{d\lambda}\frac{dx^{\rho}}{d\lambda}\frac{dx^{\alpha}}{d\lambda}\partial_{\rho}g_{\nu\alpha} + \frac{dx^{\alpha}}{d\lambda}\frac{dx^{\rho}}{d\lambda}\frac{dx^{\nu}}{d\lambda}\partial_{\nu}g_{\alpha\rho}\\
&=  -\frac{dx^{\nu}}{d\lambda}\frac{dx^{\rho}}{d\lambda}\frac{dx^{\alpha}}{d\lambda}\partial_{\rho}g_{\nu\alpha} 
\end{align*}
Now we can see that 
\begin{align*}
\frac{d}{d\lambda}\left(g_{\mu\nu}\frac{dx^{\mu}}{d\lambda}\frac{dx^{\nu}}{d\lambda}\right) &= \frac{dx^{\mu}}{d\lambda}\frac{dx^{\nu}}{d\lambda}\frac{d}{d\lambda}g_{\mu\nu} + 2g_{\mu\nu}\frac{d^2x^{\mu}}{d\lambda^2}\frac{dx^{\nu}}{d\lambda} \\
&= \frac{dx^{\mu}}{d\lambda}\frac{dx^{\nu}}{d\lambda}\frac{d}{d\lambda}g_{\mu\nu} - \frac{dx^{\nu}}{d\lambda}\frac{dx^{\rho}}{d\lambda}\frac{dx^{\alpha}}{d\lambda}\partial_{\rho}g_{\nu\alpha} 
\end{align*}
Now by chain rule we can see that
$$\frac{d}{d\lambda} = \frac{dx^{\alpha}}{d\lambda}\partial_{\alpha}$$
So we can see that
$$\frac{d}{d\lambda}\left(g_{\mu\nu}\frac{dx^{\mu}}{d\lambda}\frac{dx^{\nu}}{d\lambda}\right) = \frac{dx^{\mu}}{d\lambda}\frac{dx^{\nu}}{d\lambda}\frac{dx^{\alpha}}{d\lambda}\partial_{\alpha}g_{\mu\nu} - \frac{dx^{\nu}}{d\lambda}\frac{dx^{\rho}}{d\lambda}\frac{dx^{\alpha}}{d\lambda}\partial_{\rho}g_{\nu\alpha}$$
So now we can rename the indices of the second term so that $\rho\rightarrow\alpha$, $\nu\rightarrow\mu$, and $\alpha\rightarrow\nu$
\begin{align*}
\frac{d}{d\lambda}\left(g_{\mu\nu}\frac{dx^{\mu}}{d\lambda}\frac{dx^{\nu}}{d\lambda}\right)  &= \frac{dx^{\mu}}{d\lambda}\frac{dx^{\nu}}{d\lambda}\frac{dx^{\alpha}}{d\lambda}\partial_{\alpha}g_{\mu\nu} - \frac{dx^{\mu}}{d\lambda}\frac{dx^{\alpha}}{d\lambda}\frac{dx^{\nu}}{d\lambda}\partial_{\alpha}g_{\mu\nu} \\
&=0
\end{align*}
\end{enumerate}

\section{Problem \HWnum.2}
\begin{enumerate}[(a)]
\item
Given the functional 
$$I \equiv \frac{1}{2}\int d\lambda \left(g_{ij}\frac{dx^i}{d\lambda}\frac{dx^{j}}{d\lambda}\right)$$
Note that we are on the two-sphere so the functional for becomes
$$I = \frac{1}{2}\int d\lambda \left(\left(\frac{d\theta}{d\lambda}\right)^2 + \sin^2(\theta)\left(\frac{d\phi}{d\lambda}\right)^2\right)$$
and now we preform the variation $\theta\rightarrow\theta+\delta\theta$ So 
$$\delta I = \frac{1}{2}\int d\lambda \left(\left(\frac{d}{d\lambda}(\theta+\delta\theta)\right)^2 + \sin^2(\theta+\delta\theta)\left(\frac{d\phi}{d\lambda}\right)^2\right)$$
So we can expand 
\begin{align*}
\left(\frac{d}{d\lambda}(\theta+\delta\theta)\right)^2 &= \left(\frac{d\theta}{d\lambda} + \frac{d(\delta\theta)}{d\lambda}\right)^2\\
&= \left(\frac{d\theta}{d\lambda}\right)^2 + \left(\frac{d(\delta\theta)}{d\lambda}\right)^2 + 2\frac{d\theta}{d\lambda}\frac{d(\delta\theta)}{d\lambda}
\end{align*}
Now we will neglect the higher order terms and just say that
$$\left(\frac{d}{d\lambda}(\theta+\delta\theta)\right)^2 \approx  2\frac{d\theta}{d\lambda}\frac{d(\delta\theta)}{d\lambda}$$
Now we Taylor expand the sine squared around $\theta$ and evaluate at the point $\theta+\delta\theta$ to get
\begin{align*}
\sin^2(\theta+\delta\theta) &= \sin^2(\theta) + \left(\frac{d}{d\theta}\sin^2(\theta)\right)(\theta+\delta\theta - \theta) + ...\\
&= \sin^2(\theta) + 2\sin(\theta)\cos(\theta)(\delta\theta)\\
&\approx 2\sin(\theta)\cos(\theta)(\delta\theta)
\end{align*}
Note that we neglected any term of order $\sin^2(\theta)$ or higher. Now if we replace these terms into our functional we get
\begin{align*}
\delta I &= \frac{1}{2}\int d\lambda \left(\left(\frac{d}{d\lambda}(\theta+\delta\theta)\right)^2 + \sin^2(\theta+\delta\theta)\left(\frac{d\phi}{d\lambda}\right)^2\right)\\
&= \frac{1}{2}\int d\lambda 2\frac{d\theta}{d\lambda}\frac{d(\delta\theta)}{d\lambda} + 2\sin(\theta)\cos(\theta)(\delta\theta)\left(\frac{d\phi}{d\lambda}\right)^2\\
&= \int d\lambda \frac{d\theta}{d\lambda}\frac{d(\delta\theta)}{d\lambda} + \sin(\theta)\cos(\theta)(\delta\theta)\left(\frac{d\phi}{d\lambda}\right)^2
\end{align*}
Now we use integration by parts (or the product rule) to see that
\begin{align*}
\frac{d}{d\lambda}\left(\frac{d\theta}{d\lambda}\delta\theta\right) &= \delta\theta\frac{d^2\theta}{d\lambda^2} + \frac{d\theta}{d\lambda}\frac{d(\delta\theta)}{d\lambda}\\
&\Downarrow\\
 \frac{d\theta}{d\lambda}\frac{d(\delta\theta)}{d\lambda} &=  \frac{d}{d\lambda}\left(\frac{d\theta}{d\lambda}\delta\theta\right) - \delta\theta\frac{d^2\theta}{d\lambda^2}
\end{align*}
So now our functional becomes
\begin{align*}
\delta I &= \int d\lambda \left(\frac{d}{d\lambda}\left(\frac{d\theta}{d\lambda}\delta\theta\right) - \frac{d^2\theta}{d\lambda^2}\delta\theta + \sin(\theta)\cos(\theta)\left(\frac{d\phi}{d\lambda}\right)^2\delta\theta\right)\\
&= \cancelto{0}{\left.\frac{d\theta}{d\lambda}\delta\theta\right|} + \int d\lambda \left(-\frac{d^2\theta}{d\lambda^2}\delta\theta  + \sin(\theta)\cos(\theta)\left(\frac{d\phi}{d\lambda}\right)^2\delta\theta\right)\\
&= \int d\lambda \left(-\frac{d^2\theta}{d\lambda^2} + \sin(\theta)\cos(\theta)\left(\frac{d\phi}{d\lambda}\right)^2\right)\delta\theta
\end{align*}
Note that we used the fact that when we evaluate the integral at the boundary the terms go to zero. Now if we demand that $\delta I =0$ we see that the integrand must be zero so we get
$$-\frac{d^2\theta}{d\lambda^2} + \sin(\theta)\cos(\theta)\left(\frac{d\phi}{d\lambda}\right)^2 = 0$$
or 
$$\frac{d^2\theta}{d\lambda^2} - \sin(\theta)\cos(\theta)\left(\frac{d\phi}{d\lambda}\right)^2 = 0$$
Now if we do the variation $\phi\rightarrow\phi+\delta\phi$ we get the functional
$$\delta I = \frac{1}{2}\int d\lambda \left(\left(\frac{d\theta}{d\lambda}\right)^2 + \sin^2(\theta)\left(\frac{d}{d\lambda}(\phi+\delta\phi)\right)^2\right)$$
So we can expand 
\begin{align*}
\left(\frac{d}{d\lambda}(\phi+\delta\phi)\right)^2 &= \left(\frac{d\phi}{d\phi} + \frac{d(\delta\phi)}{d\lambda}\right)^2\\
&= \left(\frac{d\phi}{d\lambda}\right)^2 + \left(\frac{d(\delta\phi)}{d\lambda}\right)^2 + 2\frac{d\phi}{d\lambda}\frac{d(\delta\phi)}{d\lambda}
\end{align*}
Again we neglect the higher order terms so
$$\left(\frac{d}{d\lambda}(\phi+\delta\phi)\right)^2 \approx  2\frac{d\phi}{d\lambda}\frac{d(\delta\phi)}{d\lambda}$$
Now our functional becomes
$$\delta I = \frac{1}{2}\int d\lambda \left(\left(\frac{d\theta}{d\lambda}\right)^2 + 2\sin^2(\theta)\frac{d\phi}{d\lambda}\frac{d(\delta\phi)}{d\lambda}\right)$$
Note that we neglect the squared term of $d\theta/d\lambda$ as we are varying $\phi$ and there is not a $\phi$ dependence so
$$\delta I = \int d\lambda\sin^2(\theta)\frac{d\phi}{d\lambda}\frac{d(\delta\phi)}{d\lambda}$$
So we use integration by parts again to see that
\begin{align*}
\frac{d}{d\lambda}\left(\sin^2(\theta)\frac{d\phi}{d\lambda}\delta\phi\right) &= 2\sin(\theta)\cos(\theta)\frac{d\theta}{d\lambda}\frac{d\phi}{d\lambda}\delta\phi + \sin^2(\theta)\frac{d^2\phi}{d\lambda^2}\delta\phi + \sin^2(\theta)\frac{d\phi}{d\lambda}\frac{d(\delta\phi)}{d\lambda}\\
&\Downarrow\\
\sin^2(\theta)\frac{d\phi}{d\lambda}\frac{d(\delta\phi)}{d\lambda} &= \frac{d}{d\lambda}\left(\sin^2(\theta)\frac{d\phi}{d\lambda}\delta\phi\right) - 2\sin(\theta)\cos(\theta)\frac{d\theta}{d\lambda}\frac{d\phi}{d\lambda}\delta\phi - \sin^2(\theta)\frac{d^2\phi}{d\lambda^2}\delta\phi 
\end{align*}
Now our functional becomes 
\begin{align*}
\delta I &= \int d\lambda\sin^2(\theta)\frac{d\phi}{d\lambda}\frac{d(\delta\phi)}{d\lambda}\\
&= \int d\lambda\left(\frac{d}{d\lambda}\left(\sin^2(\theta)\frac{d\phi}{d\lambda}\delta\phi\right) - 2\sin(\theta)\cos(\theta)\frac{d\theta}{d\lambda}\frac{d\phi}{d\lambda}\delta\phi - \sin^2(\theta)\frac{d^2\phi}{d\lambda^2}\delta\phi \right)\\
&= \cancelto{0}{\left.\left(\sin^2(\theta)\frac{d\phi}{d\lambda}\delta\phi\right)\right|} + \int d\lambda\left(- 2\sin(\theta)\cos(\theta)\frac{d\theta}{d\lambda}\frac{d\phi}{d\lambda}\delta\phi - \sin^2(\theta)\frac{d^2\phi}{d\lambda^2}\delta\phi \right)\\
&=  \int d\lambda\left(-2\sin(\theta)\cos(\theta)\frac{d\theta}{d\lambda}\frac{d\phi}{d\lambda} - \sin^2(\theta)\frac{d^2\phi}{d\lambda^2}\right)\delta\phi 
\end{align*}
So again we require $\delta I = 0$ so the integrand must be zero
\begin{align*}
-2\sin(\theta)\cos(\theta)\frac{d\theta}{d\lambda}\frac{d\phi}{d\lambda} - \sin^2(\theta)\frac{d^2\phi}{d\lambda^2} &= 0\\
\frac{-1}{\sin^2(\theta)}\left(-2\sin(\theta)\cos(\theta)\frac{d\theta}{d\lambda}\frac{d\phi}{d\lambda} - \sin^2(\theta)\frac{d^2\phi}{d\lambda^2}\right) &= 0\\
\frac{d^2\phi}{d\lambda^2} + 2\cot(\theta)\frac{d\theta}{d\lambda}\frac{d\phi}{d\lambda}  &= 0
\end{align*}

\item
If we take equation \ref{GeoDes}
$$\frac{d^2x^{\mu}}{d\lambda^2} + \Gamma^{\mu}_{\nu\rho}\frac{dx^{\nu}}{d\lambda}\frac{dx^{\rho}}{d\lambda} = 0$$
which is the general form of the geodesic equation, we can match this up to the geodesic equations we found in part (a). For the equation
$$\frac{d^2\theta}{d\lambda^2} - \sin(\theta)\cos(\theta)\left(\frac{d\phi}{d\lambda}\right)^2 = 0$$
we see that $\mu=\theta$ and $\nu=\rho=\phi$ so we can see the first non-zero connection is given by
$$\Gamma^{\theta}_{\phi\phi} = -\sin(\theta)\cos(\theta)$$
we can also find the connection by 
\begin{equation}
\Gamma^{i}_{jk} = \frac{1}{2}g^{im}\left(\partial_{j}g_{mk}+\partial_{k}g_{jm} - \partial_{m}g_{jk}\right)
\label{connect}
\end{equation}
So we can calculate by equation \ref{connect} and we get
$$\Gamma^{\theta}_{\phi\phi} = \frac{1}{2}g^{\theta m}\left(\partial_{\phi}g_{m\phi}+\partial_{\phi}g_{\phi m} - \partial_{m}g_{\phi\phi}\right)$$
but we can see that the inverse metric $g^{\theta m}$ is only non-zero for $m=\theta$ so it follows that
\begin{align*}
\Gamma^{\theta}_{\phi\phi} &= \frac{1}{2}g^{\theta m}\left(\partial_{\phi}g_{m\phi}+\partial_{\phi}g_{\phi m} - \partial_{m}g_{\phi\phi}\right)\\
&= \frac{1}{2}g^{\theta \theta}\left(\partial_{\phi}g_{\theta\phi}+\partial_{\phi}g_{\phi \theta} - \partial_{\theta}g_{\phi\phi}\right)\\
&= \frac{1}{2}(1)\left(0+0- \partial_{\theta}\sin^2(\theta)\right)\\
&= -\frac{1}{2}\partial_{\theta}\sin^2(\theta)\\
&= -\frac{1}{2}2\sin(\theta)\cos(\theta) = -\sin(\theta)\cos(\theta)
\end{align*}
Now for the second geodesic equation of part (a)
$$\frac{d^2\phi}{d\lambda^2} + 2\cot(\theta)\frac{d\theta}{d\lambda}\frac{d\phi}{d\lambda}  = 0$$
we see that $\mu=\rho=\phi$ and $\nu=\theta$ so the connection is
$$\Gamma^{\phi}_{\theta\phi} = 2\cot(\theta)$$
or by equation \ref{connect} we get
$$\Gamma^{\phi}_{\theta\phi} = \frac{1}{2}g^{\phi m}\left(\partial_{\theta}g_{m\phi}+\partial_{\phi}g_{\theta m} - \partial_{m}g_{\theta\phi}\right)$$
Now the only non-zero inverse metric is $g^{\phi\phi} = \left(\sin^2(\theta)\right)^{-1}$ so
\begin{align*}
\Gamma^{\phi}_{\theta\phi} &= \frac{1}{2}g^{\phi \phi}\left(\partial_{\theta}g_{\phi\phi}+\partial_{\phi}g_{\theta \phi} - \partial_{\phi}g_{\theta\phi}\right)\\
&= \frac{1}{2}\frac{1}{\sin^2(\theta)}\left(\partial_{\theta}\sin^2(\theta)\right)\\
&= \frac{1}{2}\frac{1}{\sin^2(\theta)}\left(2\sin(\theta)\cos(\theta)\right)\\
&= \cot(\theta)
\end{align*}
Note the factor of 2 comes from the fact that the Christoffel connection is torsion free so the bottom indices can be switched and they can be summed over.

\item
Since we are varying over $\phi$ we use the geodesic equation 
$$\frac{d^2\phi}{d\lambda^2} + 2\cot(\theta)\frac{d\theta}{d\lambda}\frac{d\phi}{d\lambda}  = 0$$
but we are considering the line of constant $\phi$ so 
$$\frac{d\phi}{d\lambda} = 0$$
this implies that
$$\cancelto{0}{\frac{d^2\phi}{d\lambda^2}} + 2\cot(\theta)\frac{d\theta}{d\lambda}\cancelto{0}{\frac{d\phi}{d\lambda}}  = 0$$
So both terms go to zero. This implies that a line of constant $\phi$ is a geodesic on the two sphere. For a line of constant $\theta$ we use the geodesic equation
$$\frac{d^2\theta}{d\lambda^2} - \sin(\theta)\cos(\theta)\left(\frac{d\phi}{d\lambda}\right)^2 = 0$$
Now we are on a line of constant $\theta$ so the equation becomes
$$\sin(\theta)\cos(\theta)\left(\frac{d\phi}{d\lambda}\right)^2 = 0$$
and this is zero only for $\cos(\theta) = 0$ or $\sin(\theta) = 0$. So $\cos(\theta)$ is only zero at the poles ($\theta=0,\pi$), but these are not lines they are points on the sphere so these constant $\theta$ are not geodesics. So the only geodesic is the constant $\theta$ that satisfies $\sin(\theta) = 0$. This is $\theta=\pi/2$ so the equator is a geodesic of the two sphere.

\item
To find the components of the vector $V^{\mu}$ after a parallel transport we use the equation
\begin{equation}
\frac{dx^{\nu}}{d\lambda} \grad_{\nu}V^{\mu} = 0
\label{parallel}
\end{equation}
where we parallel transport around a circle of constant latitude. We parametrize this path using 
$$x^{\mu}(\lambda) = (\theta_0,\lambda)$$
note that $\theta_0$ is a constant so
$$\frac{x^{\mu}(\lambda)}{d\lambda} = (0,1)$$
So for this path we can see that the only non zero part of equation \ref{parallel} is
\begin{align*}
\frac{dx^{\nu}}{d\lambda} \grad_{\nu}V^{\mu} &= \grad_{\phi}V^{\mu}\\
&= \partial_{\phi}V^{\mu} + \Gamma^{\mu}_{\phi\rho}V^{\rho}
\end{align*}
So we see from this that we get two equations. First we take $\mu=\theta$ and we get
$$\partial_{\phi}V^{\theta} + \Gamma^{\theta}_{\phi\rho}V^{\rho} =0$$
Now we see from part (c) that the only non zero Christoffel connection is $\Gamma^{\theta}_{\phi\phi} = -\sin(\theta)\cos(\theta)$. So our equation of the parallel transport becomes
$$\partial_{\phi}V^{\theta} -\sin(\theta)\cos(\theta)V^{\phi} =0$$
Now for the $\mu=\phi$ equation we get
$$ \partial_{\phi}V^{\phi} + \Gamma^{\phi}_{\phi\rho}V^{\rho} = 0$$
Again we see that the only non zero Christoffel connection is $\Gamma^{\phi}_{\phi\theta} = \cot(\theta)$ so we can see that
$$ \partial_{\phi}V^{\phi} + \cot(\theta)V^{\theta} = 0$$
Now we need to combine these two equations so we take a derivative with respect to $\phi$ so
\begin{align*}
\partial^2_{\phi}V^{\theta} -\sin(\theta)\cos(\theta)\partial_{\phi}V^{\phi} &= 0\\
&\Downarrow\\
\partial_{\phi}V^{\phi} &= \frac{\partial^2_{\phi}V^{\theta}}{\sin(\theta)\cos(\theta)}
\end{align*}
Now we replace this into our other equation to get
\begin{align*}
 \frac{\partial^2_{\phi}V^{\theta}}{\sin(\theta)\cos(\theta)} + \cot(\theta)V^{\theta} &= 0\\
&\Downarrow\\
 \frac{\partial^2_{\phi}V^{\theta}}{\sin(\theta)\cos(\theta)} &=  -\cot(\theta)V^{\theta}\\
\partial^2_{\phi}V^{\theta} &=  -\sin(\theta)\cos(\theta)\cot(\theta)V^{\theta}\\
\partial^2_{\phi}V^{\theta} &=  -\cos^2(\theta)V^{\theta}
\end{align*}
Note that since the derivative is with respect to $\phi$ the $\cos(\theta)$ is constant so we can say the solution to this differential is
$$V^{\theta} = \alpha\sin(\cos(\theta)\phi)+\beta\cos(\cos(\theta)\phi)$$
we can now find the $\theta$ component by 
\begin{align*}
V^{\phi} &= \frac{\partial_{\phi}V^{\theta}}{\sin(\theta)\cos(\theta)}\\
&= \frac{\alpha\partial_{\phi}\sin(\cos(\theta)\phi)+\beta\partial_{\phi}\cos(\cos(\theta)\phi)}{\sin(\theta)\cos(\theta)}\\
&= \frac{\alpha\cos(\cos(\theta)\phi)\cos(\theta)-\beta\sin(\cos(\theta)\phi)\cos(\theta)}{\sin(\theta)\cos(\theta)}\\
&= \frac{\alpha\cos(\cos(\theta)\phi)-\beta\sin(\cos(\theta)\phi)}{\sin(\theta)}
\end{align*}
Now we need to use the fact that the vector starts at $(\theta,\phi) = (\theta,0)$ and has initial components of $(V^{\theta},V^{\phi}) = (1,0)$. So 
\begin{align*}
V^{\theta}(\phi=0) = 1 &= \alpha\sin(\cos(\theta)0)+\beta\cos(\cos(\theta)0)\\
&\Downarrow\\
1 &= \beta
\end{align*}
And for the other component
\begin{align*}
V^{\phi}(\phi=0) = 0 &= \frac{\alpha\cos(\cos(\theta)0)-\beta\sin(\cos(\theta)0)}{\sin(\theta)}\\
V^{\phi}(\phi=0) = 0 &= \frac{\alpha}{\sin(\theta)}\\
&\Downarrow\\
0 &= \alpha
\end{align*}
So the vector after parallel transport is
\begin{align*}
V^{\theta} &= \cos(\cos(\theta)\phi)\\
V^{\phi} &= -\frac{\sin(\cos(\theta)\phi)}{\sin(\theta)}
\end{align*}
So if we go around for $\phi=2\pi$ we get
\begin{align*}
V^{\theta} &= \cos(\cos(\theta)2\pi)\\
V^{\phi} &= -\frac{\sin(\cos(\theta)2\pi)}{\sin(\theta)}
\end{align*}
\end{enumerate}

\section{Problem \HWnum.3}
\begin{enumerate}[(a)]
\item
The proper time of a clock for the metric 
$$ds^2 = -(1+2\Phi(r))dt^2 + (1-2\Phi(r))dr^2 + r^2(d\theta^2+\sin^2(\theta)d\phi^2)$$
where $\Phi(r)$ is given by
$$\Phi(r) = -\frac{GM}{r}$$
we know that the proper time is when $dr = d\theta = d\phi = 0$ or in a frame that is not moving. But we know the metric is invariant and that $ds^2 = -d\tau^2$ where $d\tau$ is the proper time. So the proper time is given by
$$d\tau^2 = (1+2\Phi(r))dt^2$$
Note that $\Phi(r)$ is inversely proportional to $r$ so a clock on a tall building will have a larger $r$ than a clock on the ground. This implies that $\Phi(r)$ is smaller for a clock on a tall building so the proper time is fast for a clock on the ground compared to a clock on a tall building.

\item
For this metric we can say the functional $I$ is 
$$I = \frac{1}{2}\int d\tau\left(-(1+2\Phi(r))\left(\frac{dt}{d\tau}\right)^2 + (1-2\Phi(r))\left(\frac{dr}{d\tau}\right)^2 + r^2\left(\frac{d\theta}{d\tau}\right)^2+r^2\sin^2(\theta)\left(\frac{d\phi}{d\tau}\right)^2\right)$$
So we can vary $t$ such that $t\rightarrow t+\delta t$ so that
$$\delta I = \frac{1}{2}\int d\tau\left(-(1+2\Phi(r))\left(\frac{d}{d\tau}(t+\delta t)\right)^2 + (1-2\Phi(r))\left(\frac{dr}{d\tau}\right)^2 + r^2\left(\frac{d\theta}{d\tau}\right)^2+r^2\sin^2(\theta)\left(\frac{d\phi}{d\tau}\right)^2\right)$$
We are only concerned with the variation on $t$ we we neglect the other terms so that
\begin{align*}
\delta I &= \frac{1}{2}\int d\tau\left(-(1+2\Phi(r))\left(\frac{d}{d\tau}(t+\delta t)\right)^2\right)\\
&= \frac{1}{2}\int d\tau\left(-(1+2\Phi(r))\left(\frac{dt}{d\tau}+\frac{d(\delta t)}{d\tau}\right)^2\right)\\
&= \frac{1}{2}\int d\tau\left(-(1+2\Phi(r))\left(\left(\frac{dt}{d\tau}\right)^2+\left(\frac{d(\delta t)}{d\tau}\right)^2 + 2\frac{dt}{d\tau}\frac{d(\delta t)}{d\tau}\right)\right)\\
&= -\int d\tau(1+2\Phi(r))\frac{dt}{d\tau}\frac{d(\delta t)}{d\tau}
\end{align*}
Note we neglected the higher order terms. Now we use integration by parts to show
\begin{align*}
\frac{d}{d\tau}\left((1+2\Phi(r))\frac{dt}{d\tau}\delta t\right) &=\frac{d}{d\tau}(1+2\Phi(r))\frac{dt}{d\tau}\delta t + (1+2\Phi(r))\frac{d^2t}{d\tau^2}\delta t + (1+2\Phi(r))\frac{d(\delta t)}{d\tau}\frac{dt}{d\tau}\\
&=2\frac{d\Phi(r)}{dr}\frac{dr}{d\tau}\frac{dt}{d\tau}\delta t + (1+2\Phi(r))\frac{d^2t}{d\tau^2}\delta t + (1+2\Phi(r))\frac{d(\delta t)}{d\tau}\frac{dt}{d\tau}\\
&\Downarrow\\
(1+2\Phi(r))\frac{d(\delta t)}{d\tau}\frac{dt}{d\tau} &= \frac{d}{d\tau}\left((1+2\Phi(r))\frac{dt}{d\tau}\delta t\right) - 2\frac{d\Phi(r)}{dr}\frac{dr}{d\tau}\frac{dt}{d\tau}\delta t - (1+2\Phi(r))\frac{d^2t}{d\tau^2}\delta t 
\end{align*}
So our functional becomes
\begin{align*}
\delta I &= -\int d\tau(1+2\Phi(r))\frac{dt}{d\tau}\frac{d(\delta t)}{d\tau}\\
&= -\int d\tau\left(\frac{d}{d\tau}\left((1+2\Phi(r))\frac{dt}{d\tau}\delta t\right) - 2\frac{d\Phi(r)}{dr}\frac{dr}{d\tau}\frac{dt}{d\tau}\delta t - (1+2\Phi(r))\frac{d^2t}{d\tau^2}\delta t \right)\\
&=  -\left.(1+2\Phi(r))\frac{dt}{d\tau}\delta t\right| + \int d\tau \left(2\frac{d\Phi(r)}{dr}\frac{dr}{d\tau}\frac{dt}{d\tau} + (1+2\Phi(r))\frac{d^2t}{d\tau^2}\right)\delta t\\
&=  \int d\tau \left(2\frac{d\Phi(r)}{dr}\frac{dr}{d\tau}\frac{dt}{d\tau} + (1+2\Phi(r))\frac{d^2t}{d\tau^2}\right)\delta t
\end{align*}
So if we enforce the condition $\delta I = 0$ we see that we get the integrand to be zero. So it follows that
\begin{align*}
2\frac{d\Phi(r)}{d\tau}\frac{dr}{d\tau}\frac{dt}{d\tau} + (1+2\Phi(r))\frac{d^2t}{d\tau^2} &= 0\\
\frac{d^2t}{d\tau^2} + \frac{2}{(1+2\Phi(r))}\frac{d\Phi(r)}{dr}\frac{dr}{d\tau}\frac{dt}{d\tau}  &= 0
\end{align*}
Now if we vary $r\rightarrow r+\delta r$ the functional becomes
\begin{align*}
\delta I &= \frac{1}{2}\int d\tau\left(-(1+2\Phi(r+\delta r))\left(\frac{dt}{d\tau}\right)^2 + (1-2\Phi(r+\delta r))\left(\frac{d}{d\tau}(r+\delta r)\right)^2 + (r+\delta r)^2\left(\frac{d\theta}{d\tau}\right)^2\right.\\
&\ \ \ \ \ \left. +(r+\delta r)^2\sin^2(\theta)\left(\frac{d\phi}{d\tau}\right)^2\right)\\
%&= \frac{1}{2}\int d\tau\left(-2\Phi(r+\delta r))\left(\frac{dt}{d\tau}\right)^2 -2\Phi(r+\delta r))\left(\frac{d}{d\tau}(r+\delta r)\right)^2 + (r+\delta r)^2\left(\frac{d\theta}{d\tau}\right)^2\right.\\
%&\ \ \ \ \ \left. +(r+\delta r)^2\sin^2(\theta)\left(\frac{d\phi}{d\tau}\right)^2\right)\\
&= \frac{1}{2}\int d\tau\left(-\left(1+2\frac{d\Phi(r)}{dr}\right)(\delta r)\left(\frac{dt}{d\tau}\right)^2 + \left(1-2\Phi(r+\delta r)\right)\left(\frac{d}{d\tau}(r+\delta r)\right)^2 + 2r\delta r\left(\frac{d\theta}{d\tau}\right)^2 \right. \\
& \ \ \ \ \ \left. + 2r\delta r\sin^2(\theta)\left(\frac{d\phi}{d\tau}\right)^2\right)
\end{align*}
Note that we used low order approximation for all the terms that did not have two $\delta r$. The term that did have two $\delta r$ we have to treat more carefully. We use the Taylor expansion of $\Phi(r)$ about the point $r$ evaluated at $r+\delta r$ so
\begin{align*}
\Phi(r+\delta r) &= \Phi(r) + \frac{d\Phi(r)}{dr}(r+\delta r - r) + ...\\
&= \Phi(r) + \frac{d\Phi(r)}{dr}(\delta r)
\end{align*}
and we expand 
\begin{align*}
\left(\frac{d}{d\tau}(r+\delta r)\right)^2 &= \left(\frac{dr}{d\tau}\right)^2 + \left(\frac{d(\delta r)}{d\tau}\right)^2 + 2\frac{dr}{d\tau}\frac{d(\delta r)}{d\tau}
\end{align*}
Now we take the product of these two expansions a keep the lower order terms and terms that do not vary with $\delta r$
\begin{align*}
\Phi(r+\delta r)\left(\frac{d}{d\tau}(r+\delta r)\right)^2  &= \left(\Phi(r) + \frac{d\Phi(r)}{dr}(\delta r)\right)\left(\left(\frac{dr}{d\tau}\right)^2 + \left(\frac{d(\delta r)}{d\tau}\right)^2 + 2\frac{dr}{d\tau}\frac{d(\delta r)}{d\tau}\right)\\
&= \Phi(r)\left(\frac{dr}{d\tau}\right)^2 + \Phi(r)\left(\frac{d(\delta r)}{d\tau}\right)^2 + 2\Phi(r)\frac{dr}{d\tau}\frac{d(\delta r)}{d\tau} + \frac{d\Phi(r)}{dr}(\delta r)\left(\frac{dr}{d\tau}\right)^2 \\
& \ \ \ \ \ + \frac{d\Phi(r)}{dr}(\delta r)\left(\frac{d(\delta r)}{d\tau}\right)^2 + 2\frac{d\Phi(r)}{dr}(\delta r)\frac{dr}{d\tau}\frac{d(\delta r)}{d\tau}\\
&= 2\Phi(r)\frac{dr}{d\tau}\frac{d(\delta r)}{d\tau} + \frac{d\Phi(r)}{dr}(\delta r)\left(\frac{dr}{d\tau}\right)^2
\end{align*}
So now our functional becomes 
\begin{align*}
\delta I &= \frac{1}{2}\int d\tau\left(-2\frac{d\Phi(r)}{dr}(\delta r)\left(\frac{dt}{d\tau}\right)^2 - 2\left(2\Phi(r)\frac{dr}{d\tau}\frac{d(\delta r)}{d\tau} + \frac{d\Phi(r)}{dr}(\delta r)\left(\frac{dr}{d\tau}\right)^2\right) + 2r\delta r\left(\frac{d\theta}{d\tau}\right)^2 \right.\\
&\ \ \ \ \ \left.+ 2r\delta r\sin^2(\theta)\left(\frac{d\phi}{d\tau}\right)^2\right)
\end{align*}
Now we can use integration by parts to deal with the partial of $\delta r$ 
\begin{align*}
\frac{d}{d\tau}\left(\Phi(r)\frac{dr}{d\tau}\delta r\right) &= \frac{d\Phi(r)}{dr}\frac{dr}{d\tau}\frac{dr}{d\tau}\delta r + \Phi(r)\frac{d^2r}{d\tau^2}\delta r + \Phi(r)\frac{dr}{d\tau}\frac{d(\delta r)}{d\tau}\\
&= \frac{d\Phi(r)}{dr}\left(\frac{dr}{d\tau}\right)^2\delta r + \Phi(r)\frac{d^2r}{d\tau^2}\delta r + \Phi(r)\frac{dr}{d\tau}\frac{d(\delta r)}{d\tau}\\
&\Downarrow\\
\Phi(r)\frac{dr}{d\tau}\frac{d(\delta r)}{d\tau} &= \frac{d}{d\tau}\left(\Phi(r)\frac{dr}{d\tau}\delta r\right) - \frac{d\Phi(r)}{dr}\left(\frac{dr}{d\tau}\right)^2\delta r - \Phi(r)\frac{d^2r}{d\tau^2}\delta r 
\end{align*}
So the functional becomes
\begin{align*}
\delta I &= \frac{1}{2}\int d\tau\left(-2\frac{d\Phi(r)}{dr}(\delta r)\left(\frac{dt}{d\tau}\right)^2\right. \\
&\ \ \ \ \ - 2\left(2\frac{d}{d\tau}\left(\Phi(r)\frac{dr}{d\tau}\delta r\right) - 2\frac{d\Phi(r)}{dr}\left(\frac{dr}{d\tau}\right)^2\delta r - 2\Phi(r)\frac{d^2r}{d\tau^2}\delta r  + \frac{d\Phi(r)}{dr}(\delta r)\left(\frac{dr}{d\tau}\right)^2\right)\\
&\ \ \ \ \ \left. +2r\delta r\left(\frac{d\theta}{d\tau}\right)^2 + 2r\delta r\sin^2(\theta)\left(\frac{d\phi}{d\tau}\right)^2\right)\\
&= \frac{1}{2}\int d\tau\left(-2\frac{d\Phi(r)}{dr}(\delta r)\left(\frac{dt}{d\tau}\right)^2 - 2\left(-\frac{d\Phi(r)}{dr}\left(\frac{dr}{d\tau}\right)^2\delta r - 2\Phi(r)\frac{d^2r}{d\tau^2}\delta r \right)\right.\\
&\ \ \ \ \ \left. +2r\delta r\left(\frac{d\theta}{d\tau}\right)^2 + 2r\delta r\sin^2(\theta)\left(\frac{d\phi}{d\tau}\right)^2\right)\\
&= \int d\tau\left(-\frac{d\Phi(r)}{dr}\left(\frac{dt}{d\tau}\right)^2 + \frac{d\Phi(r)}{dr}\left(\frac{dr}{d\tau}\right)^2 + 2\Phi(r)\frac{d^2r}{d\tau^2} + r\left(\frac{d\theta}{d\tau}\right)^2 + r\sin^2(\theta)\left(\frac{d\phi}{d\tau}\right)^2\right)\delta r
\end{align*}
So enforcing that $\delta I = 0$ we get
$$\frac{d^2r}{d\tau^2} - \frac{1}{2\Phi(r)}\frac{d\Phi(r)}{dr}\left(\frac{dt}{d\tau}\right)^2 +\frac{1}{2\Phi(r)} \frac{d\Phi(r)}{dr}\left(\frac{dr}{d\tau}\right)^2  +\frac{r}{2\Phi(r)} \left(\frac{d\theta}{d\tau}\right)^2 + \frac{r\sin^2(\theta)}{2\Phi(r)}\left(\frac{d\phi}{d\tau}\right)^2 = 0$$
Now we vary $\theta\rightarrow\theta+\delta\theta$ so our functional becomes
$$\delta I = \frac{1}{2}\int d\tau\left(-(1+2\Phi(r))\left(\frac{dt}{d\tau}\right)^2 + (1-2\Phi(r))\left(\frac{dr}{d\tau}\right)^2 + r^2\left(\frac{d}{d\tau}(\theta+\delta\theta)\right)^2+r^2\sin^2(\theta+\delta\theta)\left(\frac{d\phi}{d\tau}\right)^2\right)$$
Note that we neglect any term that does not vary with respect to $\theta$ so our functional is now
$$\delta I = \frac{1}{2}\int d\tau\left(r^2\left(\frac{d}{d\tau}(\theta+\delta\theta)\right)^2+r^2\sin^2(\theta+\delta\theta)\left(\frac{d\phi}{d\tau}\right)^2\right)$$
So we can approximate the first term as
\begin{align*}
\left(\frac{d}{d\tau}(\theta+\delta\theta)\right)^2 &= \left(\frac{d\theta}{d\tau}\right)^2 + \left(\frac{d(\delta\theta)}{d\tau}\right)^2 + 2\frac{d\theta}{d\tau}\frac{d(\delta\theta)}{d\tau}\\
&\approx 2\frac{d\theta}{d\tau}\frac{d(\delta\theta)}{d\tau}
\end{align*}
And we Taylor Expand the $\sin^2(\theta+\delta\theta)$ about $\theta$ to get
\begin{align*}
\sin^2(\theta+\delta\theta) &= \sin^2(\theta) + \left(\frac{d}{d\theta}\sin^2(\theta)\right)(\theta+\delta\theta-\theta)\\
&= \sin^2(\theta) + 2\sin(\theta)\cos(\theta)(\delta\theta)\\
&\approx 2\sin(\theta)\cos(\theta)(\delta\theta)
\end{align*}
Now we plug these back into the functional to get
$$\delta I = \int d\tau\left(r^2\frac{d\theta}{d\tau}\frac{d(\delta\theta)}{d\tau}+r^2\sin(\theta)\cos(\theta)(\delta\theta)\left(\frac{d\phi}{d\tau}\right)^2\right)$$
Now we use integration by parts to pull the $\delta\theta$ out of the first term by
\begin{align*}
\frac{d}{d\tau}\left(r^2\frac{d\theta}{d\tau}(\delta\theta)\right) &= 2r\frac{dr}{d\tau}\frac{d\theta}{d\tau}(\delta\theta) + r^2\frac{d^2\theta}{d\tau^2}(\delta\theta) + r^2\frac{d\theta}{d\tau}\frac{d(\delta\theta)}{d\tau}\\
&\Downarrow\\
r^2\frac{d\theta}{d\tau}\frac{d(\delta\theta)}{d\tau} &= \frac{d}{d\tau}\left(r^2\frac{d\theta}{d\tau}(\delta\theta)\right) - 2r\frac{dr}{d\tau}\frac{d\theta}{d\tau}(\delta\theta) - r^2\frac{d^2\theta}{d\tau^2}(\delta\theta)  
\end{align*}
Again we have a new functional given by
\begin{align*}
\delta I &= \int d\tau\left(\frac{d}{d\tau}\left(r^2\frac{d\theta}{d\tau}(\delta\theta)\right) - 2r\frac{dr}{d\tau}\frac{d\theta}{d\tau}(\delta\theta) - r^2\frac{d^2\theta}{d\tau^2}(\delta\theta)  +r^2\sin(\theta)\cos(\theta)(\delta\theta)\left(\frac{d\phi}{d\tau}\right)^2\right)\\
&= \cancelto{0}{\left.r^2\frac{d\theta}{d\tau}(\delta\theta)\right|} + \int d\tau\left( - 2r\frac{dr}{d\tau}\frac{d\theta}{d\tau}(\delta\theta) - r^2\frac{d^2\theta}{d\tau^2}(\delta\theta)  +r^2\sin(\theta)\cos(\theta)(\delta\theta)\left(\frac{d\phi}{d\tau}\right)^2\right)\\
&=  \int d\tau\left( - 2r\frac{dr}{d\tau}\frac{d\theta}{d\tau} - r^2\frac{d^2\theta}{d\tau^2}  +r^2\sin(\theta)\cos(\theta)\left(\frac{d\phi}{d\tau}\right)^2\right)\delta\theta
\end{align*}
Now by the condition $\delta I =0$ we see that
\begin{align*}
- 2r\frac{dr}{d\tau}\frac{d\theta}{d\tau} - r^2\frac{d^2\theta}{d\tau^2}  +r^2\sin(\theta)\cos(\theta)\left(\frac{d\phi}{d\tau}\right)^2 &= 0\\
r\frac{d^2\theta}{d\tau^2} + 2\frac{dr}{d\tau}\frac{d\theta}{d\tau}   - r\sin(\theta)\cos(\theta)\left(\frac{d\phi}{d\tau}\right)^2 &= 0\\
\frac{d^2\theta}{d\tau^2} + \frac{2}{r}\frac{dr}{d\tau}\frac{d\theta}{d\tau}   - \sin(\theta)\cos(\theta)\left(\frac{d\phi}{d\tau}\right)^2 &= 0
\end{align*}
Now for varying $\phi\rightarrow\delta\phi$ the functional becomes
$$\delta I = \frac{1}{2}\int d\tau\left(-(1+2\Phi(r))\left(\frac{dt}{d\tau}\right)^2 + (1-2\Phi(r))\left(\frac{dr}{d\tau}\right)^2 + r^2\left(\frac{d\theta}{d\tau}\right)^2+r^2\sin^2(\theta)\left(\frac{d}{d\tau}(\phi+\delta\phi)\right)^2\right)$$
Again we neglect the terms that do not vary with $\delta\phi$ so
$$\delta I = \frac{1}{2}\int d\tau\left(r^2\sin^2(\theta)\left(\frac{d}{d\tau}(\phi+\delta\phi)\right)^2\right)$$
So we can say that
$$\left(\frac{d}{d\tau}(\phi+\delta\phi)\right)^2 \approx 2\frac{d\phi}{d\tau}\frac{d(\delta\phi)}{d\tau}$$
So 
$$\delta I = \int d\tau\left(r^2\sin^2(\theta)\frac{d\phi}{d\tau}\frac{d(\delta\phi)}{d\tau}\right)$$
Now we use integration by parts to see that
\begin{align*}
\frac{d}{d\tau}\left(r^2\sin^2(\theta)\frac{d\phi}{d\tau}(\delta\phi)\right) &= 2r\frac{dr}{d\tau}\sin^2(\theta)\frac{d\phi}{d\tau}(\delta\phi) + r^22\sin(\theta)\cos(\theta)\frac{d\theta}{d\tau}\frac{d\phi}{d\tau}(\delta\phi) + r^2\sin^2(\theta)\frac{d^2\phi}{d\tau^2}(\delta\phi) \\
&\ \ \ \ \ \ + r^2\sin^2(\theta)\frac{d\phi}{d\tau}\frac{d(\delta\phi)}{d\tau}\\
&\Downarrow\\
r^2\sin^2(\theta)\frac{d\phi}{d\tau}\frac{d(\delta\phi)}{d\tau} &= 
\frac{d}{d\tau}\left(r^2\sin^2(\theta)\frac{d\phi}{d\tau}(\delta\phi)\right) - 2r\sin^2(\theta)\frac{dr}{d\tau}\frac{d\phi}{d\tau}(\delta\phi) - 2r^2\sin(\theta)\cos(\theta)\frac{d\theta}{d\tau}\frac{d\phi}{d\tau}(\delta\phi)\\
&\ \ \ \ \ -r^2\sin^2(\theta)\frac{d^2\phi}{d\tau^2}(\delta\phi) 
\end{align*}
Now we can the functional becomes
\begin{align*}
\delta I &= \int d\tau\left(\frac{d}{d\tau}\left(r^2\sin^2(\theta)\frac{d\phi}{d\tau}(\delta\phi)\right) - 2r\sin^2(\theta)\frac{dr}{d\tau}\frac{d\phi}{d\tau}(\delta\phi) - 2r^2\sin(\theta)\cos(\theta)\frac{d\theta}{d\tau}\frac{d\phi}{d\tau}(\delta\phi)-r^2\sin^2(\theta)\frac{d^2\phi}{d\tau^2}(\delta\phi) \right)\\
&= \cancelto{0}{\left.r^2\sin^2(\theta)\frac{d\phi}{d\tau}(\delta\phi)\right|} + \int d\tau\left( - 2r\sin^2(\theta)\frac{dr}{d\tau}\frac{d\phi}{d\tau}(\delta\phi) - 2r^2\sin(\theta)\cos(\theta)\frac{d\theta}{d\tau}\frac{d\phi}{d\tau}(\delta\phi)-r^2\sin^2(\theta)\frac{d^2\phi}{d\tau^2}(\delta\phi) \right)\\
&=  \int d\tau\left(-2r\sin^2(\theta)\frac{dr}{d\tau}\frac{d\phi}{d\tau} - 2r^2\sin(\theta)\cos(\theta)\frac{d\theta}{d\tau}\frac{d\phi}{d\tau}-r^2\sin^2(\theta)\frac{d^2\phi}{d\tau^2} \right)\delta\phi
\end{align*}
Now we enforce the condition $\delta I = 0$ so
\begin{align*}
-2r\sin^2(\theta)\frac{dr}{d\tau}\frac{d\phi}{d\tau} - 2r^2\sin(\theta)\cos(\theta)\frac{d\theta}{d\tau}\frac{d\phi}{d\tau}-r^2\sin^2(\theta)\frac{d^2\phi}{d\tau^2} &= 0\\
r^2\sin^2(\theta)\frac{d^2\phi}{d\tau^2} + 2r\sin^2(\theta)\frac{dr}{d\tau}\frac{d\phi}{d\tau} + 2r^2\sin(\theta)\cos(\theta)\frac{d\theta}{d\tau}\frac{d\phi}{d\tau}   &= 0\\
\frac{d^2\phi}{d\tau^2} + \frac{2r\sin^2(\theta)}{r^2\sin^2(\theta)}\frac{dr}{d\tau}\frac{d\phi}{d\tau} + \frac{2r^2\sin(\theta)\cos(\theta)}{r^2\sin^2(\theta)}\frac{d\theta}{d\tau}\frac{d\phi}{d\tau}   &= 0\\
\frac{d^2\phi}{d\tau^2} + \frac{2}{r}\frac{dr}{d\tau}\frac{d\phi}{d\tau} + 2\cot(\theta)\frac{d\theta}{d\tau}\frac{d\phi}{d\tau}   &= 0
\end{align*}
So the four geodesic equations are
\begin{align}
\label{GeoD1}
0 &= \frac{d^2t}{d\tau^2} + \frac{2}{(1+2\Phi(r))}\frac{d\Phi(r)}{dr}\frac{dr}{d\tau}\frac{dt}{d\tau}  \\
\label{GeoD2}
0 &= \frac{d^2r}{d\tau^2} - \frac{1}{2\Phi(r)}\frac{d\Phi(r)}{dr}\left(\frac{dt}{d\tau}\right)^2 +\frac{1}{2\Phi(r)} \frac{d\Phi(r)}{dr}\left(\frac{dr}{d\tau}\right)^2  +\frac{r}{2\Phi(r)} \left(\frac{d\theta}{d\tau}\right)^2 + \frac{r\sin^2(\theta)}{2\Phi(r)}\left(\frac{d\phi}{d\tau}\right)^2 \\
\label{GeoD3}
0 &= \frac{d^2\theta}{d\tau^2} + \frac{2}{r}\frac{dr}{d\tau}\frac{d\theta}{d\tau}   - \sin(\theta)\cos(\theta)\left(\frac{d\phi}{d\tau}\right)^2 \\
\label{GeoD4}
0 &= \frac{d^2\phi}{d\tau^2} + \frac{2}{r}\frac{dr}{d\tau}\frac{d\phi}{d\tau} + 2\cot(\theta)\frac{d\theta}{d\tau}\frac{d\phi}{d\tau}   
\end{align}

\item
For a timelike geodesic $x^{\mu}(\tau)$ that corresponds to a circular orbit around the equator of the earth. This implies that $r$ is constant given by $r=R$ and $\theta$ is constant given by $\theta=\pi/2$. Since we $\theta$ and $r$ are not varying we do not use equations \ref{GeoD2} and \ref{GeoD3}. So for the varying $\phi$ we use equation \ref{GeoD4}
\begin{align*}
0 &= \frac{d^2\phi}{d\tau^2} + \frac{2}{R}\cancelto{0}{\frac{dr}{d\tau}}\frac{d\phi}{d\tau} + 2\cot(\theta=\pi/2)\frac{d\theta}{d\tau}\frac{d\phi}{d\tau}   \\
0 &= \frac{d^2\phi}{d\tau^2} + 2(0)\frac{d\theta}{d\tau}\frac{d\phi}{d\tau}   \\
0 &= \frac{d^2\phi}{d\tau^2}
\end{align*}
So the second derivative of $\phi$ with respect to $\tau$ is zero this implies that the first derivative is a constant which we define as
$$\omega \equiv \frac{d\phi}{d\tau}$$
Now for the varying time component we use equation \ref{GeoD1}
\begin{align*}
0 &= \frac{d^2t}{d\tau^2} + \frac{2}{(1+2\Phi(R))}\frac{d\Phi(R)}{dr}\cancelto{0}{\frac{dr}{d\tau}}\frac{dt}{d\tau}  \\
0 &= \frac{d^2t}{d\tau^2} 
\end{align*}
again this implies that the first derivative is constant which we define as
$$\gamma\equiv \frac{dt}{d\tau}$$

\item
We see that in the non-relitivistic limit and for a constant $\theta=\pi/2$ and $r=R$ we see that equation \ref{GeoD2} becomes
\begin{align*}
0 &= \cancelto{0}{\frac{d^2r}{d\tau^2}} - \frac{1}{2\Phi(r)}\frac{d\Phi(r)}{dr}\gamma^2 +\frac{1}{2\Phi(r)} \frac{d\Phi(r)}{dr}\cancelto{0}{\left(\frac{dr}{d\tau}\right)^2}  +\frac{r}{2\Phi(r)} \cancelto{0}{\left(\frac{d\theta}{d\tau}\right)^2} + \frac{r\sin^2(\theta)}{2\Phi(r)}\omega^2\\
&= -\frac{1}{2\Phi(R)}\frac{d\Phi(R)}{dr}  + \frac{R\sin^2(\pi/2)}{2\Phi(R)}\omega^2\\
&\Downarrow\\
\frac{1}{2\Phi(R)}\frac{d\Phi(R)}{dr}  &= \frac{R\sin^2(\pi/2)}{2\Phi(R)}\omega^2\\
\frac{1}{2}\frac{R}{GM}\frac{-R^2}{GM}  &= \frac{1}{2}\frac{R^2}{GM}\omega^2\\
\frac{-R}{GM}  &= \omega^2
\end{align*}
\end{enumerate}

\section{Problem \HWnum.4}
\begin{enumerate}[(a)]
\item
Given the identity from linear algebra 
$$\det A = \exp(\textnormal{Tr}\log A)$$ 
we can show that the equation
$$\Gamma^{\mu}_{\mu\nu} = \frac{1}{\sqrt{|g|}}\partial_{\nu}\sqrt{|g|}$$
follows from the definition of the Christoffel connection
$$\Gamma^{\mu}_{\rho\nu} = \frac{1}{2}g^{\mu\sigma}\left(\partial_{\rho}g_{\sigma\nu}+\partial_{\nu}g_{\rho\sigma} - \partial_{\sigma}g_{\rho\nu}\right)$$
Note that $g$ is the determinate of the metric $g_{\mu\nu}$ which we will call $G$ in matrix form. So if we find the derivative using chain rule
\begin{align*}
\partial_{\nu}\sqrt{|g|} &= \frac{1}{2}\frac{1}{\sqrt{|g|}}\partial_{\nu}|g|\\
&= \frac{1}{2}\frac{1}{\sqrt{|g|}}\frac{\partial}{\partial x^{\nu}}\left(\exp(\textnormal{Tr}\log G)\right)\\
&= \frac{1}{2}\frac{1}{\sqrt{|g|}}\exp(\textnormal{Tr}\log G)\frac{\partial}{\partial x^{\nu}}\left(\textnormal{Tr}\log G\right)\\
&= \frac{1}{2}\frac{1}{\sqrt{|g|}}\exp(\textnormal{Tr}\log G)\textnormal{Tr}\left(\frac{\partial}{\partial x^{\nu}}\log G\right)\\
&= \frac{1}{2}\frac{1}{\sqrt{|g|}}\exp(\textnormal{Tr}\log G)\textnormal{Tr}G^{-1}\left(\frac{\partial}{\partial x^{\nu}}G\right)\\
&= \frac{1}{2}\frac{1}{\sqrt{|g|}}|g|g^{\mu\sigma}\partial_{\nu}g_{\mu\sigma}
\end{align*}
So our equation becomes
\begin{align*}
\Gamma^{\mu}_{\mu\nu} &= \frac{1}{\sqrt{|g|}}\partial_{\nu}\sqrt{|g|}\\
&= \frac{1}{\sqrt{|g|}}\frac{1}{2}\frac{1}{\sqrt{|g|}}|g|g^{\mu\sigma}\partial_{\nu}g_{\mu\sigma}\\
&= \frac{1}{2}g^{\mu\sigma}\partial_{\nu}g_{\mu\sigma}
\end{align*}
Now we see that this follows from the definition of the Christoffel connection by
\begin{align*}
\Gamma^{\mu}_{\mu\nu} &= \frac{1}{2}g^{\mu\sigma}\left(\partial_{\mu}g_{\sigma\nu}+\partial_{\nu}g_{\mu\sigma} - \partial_{\sigma}g_{\mu\nu}\right)\\
&= \frac{1}{2}\left(g^{\mu\sigma}\partial_{\mu}g_{\sigma\nu} + g^{\mu\sigma}\partial_{\nu}g_{\mu\sigma} - g^{\mu\sigma}\partial_{\sigma}g_{\mu\nu}\right)\\
&= \frac{1}{2}\left(g^{\mu\sigma}\partial_{\mu}g_{\sigma\nu} + g^{\mu\sigma}\partial_{\nu}g_{\mu\sigma} - g^{\sigma\mu}\partial_{\mu}g_{\sigma\nu}\right)\\
&= \frac{1}{2}g^{\mu\sigma}\partial_{\nu}g_{\mu\sigma} 
\end{align*}
Note we renamed the dummy indices in the last term by $\mu\rightarrow\sigma$ and $\sigma\rightarrow\mu$ and used the fact that the metric is symmetric. We see that we got the same result from the identity given.

\item
Using the result from part (a) we see that the divergence of the vector $V^{\mu}$ is
\begin{align*}
\grad_{\mu}V^{\mu} &= \partial_{\mu}V^{\mu}+\Gamma^{\mu}_{\mu\nu}V^{\nu}\\
&= \partial_{\mu}V^{\mu}+ \left(\frac{1}{\sqrt{|g|}}\partial_{\nu}\sqrt{|g|}\right)V^{\nu}\\
&= \partial_{\mu}V^{\mu}+ \left(\frac{1}{\sqrt{|g|}}\partial_{\mu}\sqrt{|g|}\right)V^{\mu}\\
&= \frac{1}{\sqrt{|g|}}\partial_{\mu}\left(\sqrt{|g|}V^{\mu}\right)
\end{align*}
Note that we reversed the chain rule in the last step.

\item
To find the Laplacian in spherical coordinates we first define the Laplacian as
$$\grad_{\mu}\grad^{\mu}f$$
where we find that 
\begin{align*}
\grad^{\mu}f &= g^{\mu\nu}\grad_{\nu}f\\
&= g^{\mu\nu}\partial_{\nu}f
\end{align*}
Note that the connection term is zero because $f$ is a scalar function. So using the identity from part (b) we see that
\begin{align*}
\grad_{\mu}\grad^{\mu}f &= \grad_{\mu}g^{\mu\nu}\partial_{\nu}f\\
&= \frac{1}{\sqrt{|g|}}\partial_{\mu}\left(\sqrt{|g|}g^{\mu\nu}\partial_{\nu}f\right)\\
&= \frac{1}{r^2\sin(\theta)}\partial_{\mu}\left(r^2\sin(\theta)g^{\mu\nu}\partial_{\nu}f\right)
\end{align*}
Now we look at each component of $\nu$ individually so for $\nu=r$ we get
\begin{align*}
\left(\grad_{\mu}\grad^{\mu}f\right)_{r}&= \frac{1}{r^2\sin(\theta)}\partial_{\mu}\left(r^2\sin(\theta)g^{\mu r}\partial_{r}f\right)\\
\end{align*}
but we know that the metric is a diagonal matrix so this forces $\mu=r$ so
\begin{align*}
\left(\grad_{\mu}\grad^{\mu}f\right)_{r}&= \frac{1}{r^2\sin(\theta)}\partial_{r}\left(r^2\sin(\theta)g^{rr}\partial_{r}f\right)\\
&= \frac{1}{r^2\sin(\theta)}\sin(\theta)\partial_{r}\left(r^2(1)\partial_{r}f\right)\\
&= \frac{1}{r^2}\partial_{r}\left(r^2\partial_{r}f\right)
\end{align*}
Now for $\nu=\mu=\theta$ again the inverse metric forces the indices to be the same for all no zero terms.
\begin{align*}
\left(\grad_{\mu}\grad^{\mu}f\right)_{\theta} &= \frac{1}{r^2\sin(\theta)}\partial_{\theta}\left(r^2\sin(\theta)g^{\theta\theta}\partial_{\theta}f\right)\\
&= \frac{1}{r^2\sin(\theta)}\partial_{\theta}\left(r^2\sin(\theta)\frac{1}{r^2}\partial_{\theta}f\right)\\
&= \frac{1}{r^2\sin(\theta)}\partial_{\theta}\left(\sin(\theta)\partial_{\theta}f\right)
\end{align*}
And for $\nu=\mu=\phi$ we have
\begin{align*}
\left(\grad_{\mu}\grad^{\mu}f\right)_{\phi} &= \frac{1}{r^2\sin(\theta)}\partial_{\phi}\left(r^2\sin(\theta)g^{\phi\phi}\partial_{\phi}f\right)\\
&= \frac{1}{r^2\sin(\theta)}\partial_{\phi}\left(r^2\sin(\theta)\frac{1}{r^2\sin(\theta)}\partial_{\phi}f\right)\\
&= \frac{1}{r^2\sin(\theta)}\partial_{\phi}\left(\frac{1}{\sin(\theta)}\partial_{\phi}f\right)\\
&= \frac{1}{r^2\sin^2(\theta)}\partial_{\phi}\left(\partial_{\phi}f\right)\\
&= \frac{1}{r^2\sin^2(\theta)}\left(\partial^2_{\phi}f\right)
\end{align*}
So the total Laplacian is 
$$ \grad_{\mu}\grad^{\mu}f = \frac{1}{r^2}\partial_{r}\left(r^2\partial_{r}f\right) +\frac{1}{r^2\sin(\theta)}\partial_{\theta}\left(\sin(\theta)\partial_{\theta}f\right) +\frac{1}{r^2\sin^2(\theta)}\left(\partial^2_{\phi}f\right)$$
And this is the same equation for the Laplacian as what is given in the NRL Plasma Formulary 
$$\grad^2f = \frac{1}{r^2}\frac{\partial}{\partial r}\left(r^2\partiald{f}{r}\right) + \frac{1}{r^2\sin(\theta)}\partiald{}{\theta}\left(\sin(\theta)\partiald{f}{\theta}\right) + \frac{1}{r^2\sin^2(\theta)}\frac{\partial^2f}{\partial\phi^2}$$

\item
We can find the divergence of a vector $V^{\mu}$ by
\begin{align*}
\grad_{\mu}V^{\mu} &= \frac{1}{\sqrt{|g|}}\partial_{\mu}\left(\sqrt{|g|}V^{\mu}\right)\\
&= \frac{1}{r^2\sin(\theta)}\partial_{\mu}\left(r^2\sin(\theta)V^{\mu}\right)\\
&= \frac{1}{r^2\sin(\theta)}\partial_{r}\left(r^2\sin(\theta)V^{r}\right) + \frac{1}{r^2\sin(\theta)}\partial_{\theta}\left(r^2\sin(\theta)V^{\theta}\right) + \frac{1}{r^2\sin(\theta)}\partial_{\phi}\left(r^2\sin(\theta)V^{\phi}\right)\\
&= \frac{1}{r^2}\partial_{r}\left(r^2V^{r}\right) + \frac{1}{\sin(\theta)}\partial_{\theta}\left(\sin(\theta)V^{\theta}\right) + \partial_{\phi}V^{\phi}
\end{align*}

\item
We see that part (d) is not the divergence we find in the NRL Plasma Formulary. We see that the divergence is given as 
$$\grad_{\mu}V^{\mu} = \frac{1}{r^2}\partial_{r}(r^2V^{r}) + \frac{1}{r\sin(\theta)}\partial_{\theta}(\sin(\theta)V^{\theta}) + \frac{1}{r\sin(\theta)}\partial_{\phi}V^{\phi}$$
Note that we used index notation so that the two equations would look similar. So if we require that the divergence is the same regardless of basis we can equate the equations as see that 
\begin{align*}
\vec{e}_{r} &= \vec{e}_{\hat{r}}\\
\vec{e}_{\theta} &= \frac{1}{r}\vec{e}_{\hat{\theta}}\\
\vec{e}_{\phi} &= \frac{1}{r\sin(\theta)}\vec{e}_{\hat{\phi}}
\end{align*}
Where the hatted coordinates are the coordinates of the basis in the NRL plasma formulary. Note that $\vec{e}_{\hat{\mu}}\cdot\vec{e}_{\hat{\nu}} = \delta^{\mu}_{\nu}$
\end{enumerate}
\end{document}

