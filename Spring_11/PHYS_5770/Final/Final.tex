\documentclass[11pt]{article}

\usepackage{latexsym}
\usepackage{amssymb}
\usepackage{amsthm}
\usepackage{enumerate}
\usepackage{amsmath}
\usepackage{cancel}
\numberwithin{equation}{section}

\setlength{\evensidemargin}{.25in}
\setlength{\oddsidemargin}{-.25in}
\setlength{\topmargin}{-.75in}
\setlength{\textwidth}{6.5in}
\setlength{\textheight}{9.5in}
\newcommand{\due}{May 3rd, 2011}
\newcommand{\grad}{\bold\nabla}
\newcommand{\vecE}{\vec{E}}
\newcommand{\scrptR}{\vec{\mathfrak{R}}}
\newcommand{\kapa}{\frac{1}{4\pi\epsilon_0}}
\newcommand{\emf}{\mathcal{E}}
\newcommand{\unit}[1]{\ensuremath{\, \mathrm{#1}}}
\newcommand{\real}{\textnormal{Re}}
\newcommand{\Erf}{\textnormal{Erf}}
\newcommand{\sech}{\textnormal{sech}}
\newcommand{\scrO}{\mathcal{O}}
\newcommand{\levi}{\widetilde{\epsilon}}
\newcommand{\partiald}[2]{\ensuremath{\frac{\partial{#1}}{\partial{#2}}}}
\newcommand{\norm}[2]{\langle{#1}|{#2}\rangle}
\newcommand{\inprod}[2]{\langle{#1}|{#2}\rangle}
\newcommand{\average}[1]{\left\langle{#1}\right\rangle}
\newcommand{\ket}[1]{|{#1}\rangle}
\newcommand{\bra}[1]{\langle{#1}|}
\newcommand{\Resid}[2]{\ensuremath{\textnormal{Res}\left[{#1},{#2}\right]}}





\begin{document}
\begin{titlepage}
\setlength{\topmargin}{1.5in}
\begin{center}
\Huge{Physics 3310} \\
\LARGE{Principles of Electricity and Magnetism 1} \\
\Large{Professor Thomas R. Schibli} \\[1cm]

\huge{Homework \#\HWnum}\\[0.5cm]

\large{Joe Becker} \\
\large{SID: 810-07-1484} \\
\large{\due} 

\end{center}

\end{titlepage}



\section{Problem 1}
\begin{enumerate}[(a)]
\item
Given the Nordstr\"{o}m theory of gravity defined by
$$R = \kappa g_{\mu\nu}T^{\mu\nu}$$
where $R$ is the Ricci scalar and $\kappa$ is a constant, and the metric $g_{\mu\nu}$ is defined as
\begin{equation}
g_{\mu\nu} = \Omega(x)\eta_{\mu\nu}
\label{Prob1Metric}
\end{equation}
where $\eta_{\mu\nu}$ is the \emph{Minkowski metric}. We can take the weak-field limit where $\Omega(x) = 1+\epsilon(x)$ with $\epsilon(x)$ small so we have
$$g_{\mu\nu} = \eta_{\mu\nu} + \epsilon(x)\eta_{\mu\nu}$$
Now we can find the Christoffel symbols in first order of $\epsilon(x)$ by
\begin{equation}
\Gamma^{\lambda}_{\mu\nu} = \frac{1}{2}g^{\lambda\rho}\left(\partial_{\mu}g_{\nu\rho}+\partial_{\nu}g_{\mu\rho} - \partial_{\rho}g_{\mu\nu}\right)
\label{Christ}
\end{equation}
where the inverse metric is just given by
$$g^{\mu\nu} = \eta^{\mu\nu} + \epsilon(x)\eta^{\mu\nu}$$
due to the fact that the Minkowski metric is equal its inverse. So equation \ref{Christ} becomes
$$\Gamma^{\lambda}_{\mu\nu} = \frac{1}{2}\left(\eta^{\lambda\rho}+\epsilon(x)\eta^{\lambda\rho}\right)\left(\partial_{\mu}g_{\nu\rho}+\partial_{\nu}g_{\mu\rho} - \partial_{\rho}g_{\mu\nu}\right)$$
Now we can use the fact that the derivative of Minkowski metric is zero as it does not depend on any space-time coordinates to say that
\begin{align*}
\partial_{\mu}g_{\nu\rho} &= \partial_{\mu}\left(\eta_{\nu\rho} + \epsilon(x)\eta_{\nu\rho}\right)\\
&= \cancelto{0}{\partial_{\mu}\eta_{\nu\rho}} + \partial_{\mu}\epsilon(x)\eta_{\nu\rho}\\
&= \epsilon(x)\cancelto{0}{\partial_{\mu}\eta_{\nu\rho}} + \eta_{\nu\rho}\partial_{\mu}\epsilon(x) = \eta_{\nu\rho}\partial_{\mu}\epsilon(x) 
\end{align*}
So we can rewrite equation \ref{Christ} as
$$\Gamma^{\lambda}_{\mu\nu} = \frac{1}{2}\left(\eta^{\lambda\rho}+\epsilon(x)\eta^{\lambda\rho}\right)\left(\eta_{\nu\rho}\partial_{\mu}\epsilon(x)+\eta_{\mu\rho}\partial_{\nu}\epsilon(x) - \eta_{\mu\nu}\partial_{\rho}\epsilon(x)\right)$$
but we only want the first order in $\epsilon(x)$ so we have
$$\Gamma^{\lambda}_{\mu\nu} = \frac{1}{2}\eta^{\lambda\rho}\left(\eta_{\nu\rho}\partial_{\mu}\epsilon(x)+\eta_{\mu\rho}\partial_{\nu}\epsilon(x) - \eta_{\mu\nu}\partial_{\rho}\epsilon(x)\right)$$


\item
\item
\end{enumerate}

\section{Problem 2}
\begin{enumerate}[(a)]
\item
\item
\item
\item
\item
\item
\end{enumerate}

\end{document}

