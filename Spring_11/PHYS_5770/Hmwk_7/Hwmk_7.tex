\documentclass[11pt]{article}

\usepackage{latexsym}
\usepackage{amssymb}
\usepackage{amsthm}
\usepackage{enumerate}
\usepackage{amsmath}
\usepackage{cancel}
\numberwithin{equation}{section}

\setlength{\evensidemargin}{.25in}
\setlength{\oddsidemargin}{-.25in}
\setlength{\topmargin}{-.75in}
\setlength{\textwidth}{6.5in}
\setlength{\textheight}{9.5in}
\newcommand{\due}{April 5th, 2011}
\newcommand{\HWnum}{7}
\newcommand{\grad}{\bold\nabla}
\newcommand{\vecE}{\vec{E}}
\newcommand{\scrptR}{\vec{\mathfrak{R}}}
\newcommand{\kapa}{\frac{1}{4\pi\epsilon_0}}
\newcommand{\emf}{\mathcal{E}}
\newcommand{\unit}[1]{\ensuremath{\, \mathrm{#1}}}
\newcommand{\real}{\textnormal{Re}}
\newcommand{\Erf}{\textnormal{Erf}}
\newcommand{\sech}{\textnormal{sech}}
\newcommand{\scrO}{\mathcal{O}}
\newcommand{\levi}{\widetilde{\epsilon}}
\newcommand{\partiald}[2]{\ensuremath{\frac{\partial{#1}}{\partial{#2}}}}
\newcommand{\norm}[2]{\langle{#1}|{#2}\rangle}
\newcommand{\inprod}[2]{\langle{#1}|{#2}\rangle}
\newcommand{\average}[1]{\left\langle{#1}\right\rangle}
\newcommand{\ket}[1]{|{#1}\rangle}
\newcommand{\bra}[1]{\langle{#1}|}
\newcommand{\Resid}[2]{\ensuremath{\textnormal{Res}\left[{#1},{#2}\right]}}





\begin{document}
\begin{titlepage}
\setlength{\topmargin}{1.5in}
\begin{center}
\Huge{Physics 3310} \\
\LARGE{Principles of Electricity and Magnetism 1} \\
\Large{Professor Thomas R. Schibli} \\[1cm]

\huge{Homework \#\HWnum}\\[0.5cm]

\large{Joe Becker} \\
\large{SID: 810-07-1484} \\
\large{\due} 

\end{center}

\end{titlepage}



\section{Problem \HWnum.1}
\begin{enumerate}[(a)]
\item
For a massive particle moving in the radial direction with the instantaneous velocity $v = dr/dt$. We can find the components of the four-velocity $U^{\mu}$ by using the normalization condition 
$$g_{\mu\nu}U^{\mu}U^{\nu} = -1$$
where we take the $\theta$ and $\phi$ components to be constant. Note that the $\mu=r$ component is given by chain rule so that
$$\frac{dr}{d\tau} = \frac{dr}{dt}\frac{dt}{d\tau} = vU^{t}$$
So the normalization becomes 
\begin{align*}
g_{\mu\nu}U^{\mu}U^{\nu} &= -1\\
&\Downarrow\\
-(1+2\Phi)\left(\frac{dt}{d\tau}\right)^2 + (1+2\Phi)^{-1}\left(\frac{dr}{d\tau}\right)^2 &= -1 \\
-(1+2\Phi)\left(\frac{dt}{d\tau}\right)^2 + (1+2\Phi)^{-1}\left(\frac{dr}{dt}\frac{dt}{d\tau}\right)^2 &= -1 \\
-(1+2\Phi)\left(\frac{dt}{d\tau}\right)^2 + (1+2\Phi)^{-1}\left(v\frac{dt}{d\tau}\right)^2 &= -1 \\
\left(-(1+2\Phi) + (1+2\Phi)^{-1}v^2\right)\left(\frac{dt}{d\tau}\right)^2 &= -1 \\
&\Downarrow\\
\frac{dt}{d\tau} &= U^{t} = \frac{1}{\sqrt{(1+2\Phi) - v^2(1+2\Phi)^{-1}}}
\end{align*}
And from our definition of $dr/d\tau$ using chain rule we have
$$U^{r} = \frac{v}{\sqrt{(1+2\Phi) - v^2(1+2\Phi)^{-1}}}$$

\item
For a stationary observer we can find its four-velocity $U^{\mu}_{\scrO}$ by taking $U^{\mu}$ that we found in part (a) and take the limit as $v\rightarrow 0$ which gives us 
$$U^{t}_{\scrO} = \frac{1}{\sqrt{(1+2\Phi)}}$$
Now we can find the kinetic energy as seen by the observer by using
$$E_{kin} = -g_{\mu\nu}p^{\mu}U^{\nu}_{\scrO}$$
where $p^{\mu}$ is the four-momentum given by $p^{\mu} = mU^{\mu}$. Note that $U^{\mu}_{\scrO}$ is only nonzero for $\mu=t$ so it picks out the $t$ component. So
\begin{align*}
E_{kin} &= -g_{\mu\nu}p^{\mu}U^{\nu}_{\scrO}\\
&= -g_{tt}mU^{t}U^{t}_{\scrO}\\
&= (1+2\Phi)m\left(\frac{1}{\sqrt{(1+2\Phi) - v^2(1+2\Phi)^{-1}}}\right)\frac{1}{\sqrt{(1+2\Phi)}}\\
&= \frac{m(1+2\Phi)}{\sqrt{(1+2\Phi)^2 - v^2}}
\end{align*}
Now if we want to approximate the kinetic energy we want to do some algebra to make the approximation easier
\begin{align*}
E_{kin} &= \frac{m(1+2\Phi)}{\sqrt{(1+2\Phi)^2 - v^2}}\\
&= \frac{m(1+2\Phi)}{(1+2\Phi)\sqrt{1 - \frac{v^2}{(1+2\Phi)^2}}}\\
&= \frac{m}{\sqrt{1 - \frac{v^2}{(1+2\Phi)^2}}}
\end{align*}
Now we assume that $v<<1$ and $\Phi<<1$ so we can use a binomial approximation 
\begin{equation}
(1+x)^{\alpha} \approx 1+\alpha x
\label{bio}
\end{equation}
So we can approximate that
$$\frac{v^2}{(1+2\Phi)^2} \approx v^2(1-4\Phi)$$
Now we assume that this term is still small so we can approximate that
\begin{align*}
E_{kin} &\approx \frac{m}{\sqrt{1 - v^2(1-4\Phi)}} \\
&\approx m\left(1 + \frac{1}{2}v^2(1-4\Phi)\right)\\ 
&\approx m + \frac{1}{2}mv^2-2mv^2\Phi\\
&\approx m + \frac{1}{2}mv^2
\end{align*}
Note that we neglected the $v^2\Phi$ term. This approximation yields two nonrelativistic terms the first is the rest mass energy and the second is the classical kinetic energy.

\item
We can calculate the conserved energy by using the equation
\begin{equation}
E = - g_{\mu\nu}p^{\mu}K^{\nu}
\label{EnCon}
\end{equation}
where $K^{\mu}$ is the timelike killing vector given by
$$K^{\mu} = (1,0,0,0)$$
so again we the only nonzero component is for the $\mu=\nu=t$ so we have
\begin{align*}
E &= - g_{\mu\nu}p^{\mu}K^{\nu}\\
&= - g_{tt}mU^{t}K^{t}\\
&= (1+2\Phi)m\left(\frac{1}{\sqrt{(1+2\Phi) - v^2(1+2\Phi)^{-1}}}\right)\\
&= \frac{m(1+2\Phi)}{\sqrt{(1+2\Phi) - v^2(1+2\Phi)^{-1}}}\\
&= \frac{m(1+2\Phi)}{\sqrt{1+2\Phi}\sqrt{1 - v^2(1+2\Phi)^{-2}}}\\
&= \frac{m\sqrt{1+2\Phi}}{\sqrt{1 - v^2(1+2\Phi)^{-2}}}
\end{align*}
Again we can use the binomial approximation given by equation \ref{bio} to say that 
$$\sqrt{1+2\Phi} \approx 1+\Phi$$
and
$$(1+2\Phi)^{-2} \approx 1-4\Phi$$
So this yields 
$$E \approx \frac{m(1+\Phi)}{\sqrt{1 - v^2(1-4\Phi)}}$$
So now we use the binomial approximation on the $v^2(1-4\Phi)$ term. Note that we neglect the $v^2\Phi$ terms.
\begin{align*}
E &\approx \frac{m(1+\Phi)}{\sqrt{1 - v^2(1-4\Phi)}}\\
&\approx m(1+\Phi)\left(1 + \frac{1}{2}v^2(1-4\Phi)\right)\\
&\approx (m+m\Phi)\left(1 + \frac{1}{2}v^2-2v^2\Phi\right)\\
&\approx (m+m\Phi)\left(1 + \frac{1}{2}v^2\right)\\
&\approx m + \frac{1}{2}mv^2 + m\Phi - \cancel{\frac{1}{2}mv^2\Phi}\\
&\approx m + \frac{1}{2}mv^2 + m\Phi 
\end{align*}
We see that this approximation is the same as the result we found in part (b) where $m$ is the rest mass energy and $1/2mv^2$ is the kinetic energy. The new term $m\Phi$ is the gravitational potential. This makes sense since this energy is the total energy so we need to include the potential term.
\end{enumerate}

\section{Problem \HWnum.2}
\begin{enumerate}[(a)]
\item
For the \emph{Schwartzschild Solution} we find the effective potential
\begin{equation}
V(r) = -\frac{GM}{r} + \frac{L^2}{2r^2} - \frac{GML^2}{r^3}
\label{EffPot}
\end{equation}
Now we can define the variable $x$ as
$$x\equiv \frac{GMr}{L^2}$$
which if we solve for $r$ we get
$$r = \frac{xL^2}{GM}$$
Under this change in variable the effective potential given by equation \ref{EffPot} becomes
\begin{align*}
V(r) &= -\frac{GM}{r} + \frac{L^2}{2r^2} - \frac{GML^2}{r^3}\\
&\Downarrow\\
V(x) &= -GM\frac{GM}{xL^2} + \frac{L^2}{2}\left(\frac{GM}{xL^2}\right)^2 - GML^2\left(\frac{GM}{xL^2}\right)^3\\
&= -\frac{(GM)^2}{xL^2} + \frac{(GM)^2}{2x^2L^2} - \frac{(GM)^4}{x^3L^4}
\end{align*}
Now we define a constant $l$ as
$$l\equiv \frac{L}{GM}$$
the effective potential becomes
\begin{align*}
V(x) &= -\frac{(GM)^2}{xL^2} + \frac{(GM)^2}{2x^2L^2} - \frac{(GM)^4}{x^3L^4}\\
&\Downarrow\\
V(x) &= -\frac{1}{l^2x} + \frac{1}{2l^2x^2} - \frac{1}{l^4x^3}
\end{align*}
Now if we bring the $l^2$ to the left hand side we have
$$l^2V(x) = -\frac{1}{x} + \frac{1}{2x^2} - \frac{1}{l^2x^3}$$
We see that $l$ represents the ratio of the angular momentum $L$ to half of the Schwarzschild radius $R_s = 2GM$. We can see that $R_s$ is like the classical idea of the \emph{impact parameter}. Classically as the impact parameter goes to zero the particle goes directly to the center. Note as $l\rightarrow\infty$ we find that the $x^{-3}$ term goes to zero and we end up in the Newtonian limit. This is because as $l$ goes to infinity the Scwarzschild radius goes to zero. 

\item
We can find the circular orbits of the potential from part (a) by taking the derivative of $V(x)$ with respect to $x$
\begin{align*}
l^2V'(x) &= \frac{1}{x^2} - \frac{1}{x^3} + \frac{3}{l^2x^4}
\end{align*}
Now we find the $x$ for $V'(x)=0$ so
\begin{align*}
l^2V'(x) &= 0\\
&\Downarrow\\
0 &= \frac{1}{x^2} - \frac{1}{x^3} + \frac{3}{l^2x^4}\\
0 &= x^4\left(\frac{1}{x^2} - \frac{1}{x^3} + \frac{3}{l^2x^4}\right)\\
0 &= x^2 - x + \frac{3}{l^2}
\end{align*}
Note that we can multiply by $x^4$ since the right hand side is equal to zero. Now we can find $x$ by the quadratic formula
\begin{equation}
x = \frac{-b \pm \sqrt{b^2-4ac}}{2a}
\label{Quad}
\end{equation}
where $a$, $b$, and $c$ are the coefficients of the quadratic. So equation \ref{Quad} yields
\begin{align*}
x &= \frac{-(-1) \pm \sqrt{1 - 4(1)(3/l^2)}}{2(1)}\\
&= \frac{1 \pm \sqrt{1 - 12l^{-2}}}{2}
\end{align*}
Now we can change back from our defined variables $x$ and $l$ to get 
\begin{align*}
x &= \frac{1 \pm \sqrt{1 - 12l^{-2}}}{2}\\
&\Downarrow\\
\frac{GMr}{L^2} &= \frac{1 \pm \sqrt{1 - 12\frac{(GM)^2}{L^2}}}{2}\\
&\Downarrow\\
r &= \frac{L^2}{2GM}\frac{1 \pm \sqrt{1 - 12\frac{(GM)^2}{L^2}}}{2}\\
&= \frac{1}{2GM}\left(L^2 \pm L^2\sqrt{1 - 12\frac{(GM)^2}{L^2}}\right)\\
&= \frac{1}{2GM}\left(L^2 \pm \sqrt{L^4 - 12L^4\frac{(GM)^2}{L^2}}\right)\\
r_c &= \frac{1}{2GM}\left(L^2 \pm \sqrt{L^4 - 12(GML)^2}\right)
\end{align*}

\item
We can look at the two limits of $l$ for the solution we found in part (b) 
$$x = \frac{1 \pm \sqrt{1 - 12l^{-2}}}{2}$$
we see for large $l$ the $12l^{-2}$ term under the radical becomes negligible. This implies that
\begin{align*}
x &\approx  \frac{1 \pm \sqrt{1}}{2}\\
&\approx  \frac{1 \pm 1}{2}
\end{align*}
So we see that the only stable orbit for large $l$ is when $x=1$ the other is the origin and being at the origin is not an orbit. So $r$ is given by the definition of $x$ 
\begin{align*}
x \equiv \frac{GMr}{L^2} &= 1\\
&\Downarrow\\
r_c &= \frac{L^2}{GM}
\end{align*}
This is the classical limit for large angular momentum. Now for small $l$ we see that the solution for $x$
$$x = \frac{1 \pm \sqrt{1 - 12l^{-2}}}{2}$$
becomes imaginary for values of $l<\sqrt{12}$ so we assume that the smallest $l$ can be is $\sqrt{12}$. Note that this point is where the two values for $x$ become the same so we see that
\begin{align*}
x &= \frac{1 \pm \sqrt{1 - 12(\sqrt{12})^{-2}}}{2}\\
&= \frac{1 \pm \sqrt{1 - 1}}{2} = \frac{1}{2}
\end{align*}
So by the definition of $x$ we have
$$r_c = \frac{L^2}{2GM}$$
but we recall that we let $l = \sqrt{12}$. This implies by the definition of $l$ that
\begin{align*}
\frac{L}{GM} &= \sqrt{12}\\
&\Downarrow\\
L &= \sqrt{12}GM
\end{align*}
So by replacing $L$ with this term we get
\begin{align*}
r_c &= \frac{L^2}{2GM}\\
&= \frac{(\sqrt{12}GM)^2}{2GM}\\
&= \frac{12(GM)^2}{2GM}\\
&= 6GM
\end{align*}
So for $L = \sqrt{12}GM$ the two circular orbits converged at the radius $r_c = 6GM$.
\end{enumerate}

\section{Problem \HWnum.3}
\begin{enumerate}[(a)]
\item
Given Einstein's equations with a non-zero cosmological constant
\begin{equation}
R_{\mu\nu} - \frac{1}{2}g_{\mu\nu}R + \Lambda g_{\mu\nu} = 0
\label{Cosmo}
\end{equation}
where $\Lambda$ is the cosmological constant and $g_{\mu\nu}$ is the metric, $R$ is the Ricci scalar, and $R_{\mu\nu}$ is the Ricci tensor. We can contract this equation with the inverse metric $g^{\mu\nu}$ to get
$$ g^{\mu\nu}R_{\mu\nu} - \frac{1}{2}g^{\mu\nu}g_{\mu\nu}R + \Lambda g^{\mu\nu}g_{\mu\nu} = 0$$
We know that when we contract the tensor with its inverse we always have
$$g^{\mu\nu}g_{\mu\nu} = 4$$
and we know that the Ricci scalar comes from the Ricci Tensor by
$$R = R^{\mu}_{\ \mu} = g^{\mu\nu}R_{\mu\nu}$$
so we see that the first term is just $R$ so we have
\begin{align*}
g^{\mu\nu}R_{\mu\nu} - \frac{1}{2}g^{\mu\nu}g_{\mu\nu}R &+ \Lambda g^{\mu\nu}g_{\mu\nu} = 0\\
&\Downarrow\\
R - \frac{1}{2}4R + \Lambda 4 &= 0\\
R - 2R + \Lambda 4 &= 0\\
-R + 4\Lambda  &= 0\\
&\Downarrow\\
R &= 4\Lambda 
\end{align*}
Now if we use this relation with equation \ref{Cosmo} we get
\begin{align*}
R_{\mu\nu} - \frac{1}{2}g_{\mu\nu}R + \Lambda g_{\mu\nu} &= 0\\
&\Downarrow\\
R_{\mu\nu} - \frac{1}{2}g_{\mu\nu}(4\Lambda) + \Lambda g_{\mu\nu} &= 0\\
R_{\mu\nu} - 2\Lambda g_{\mu\nu} + \Lambda g_{\mu\nu} &= 0\\
R_{\mu\nu} - \Lambda g_{\mu\nu} &= 0\\
&\Downarrow\\
R_{\mu\nu} &= \Lambda g_{\mu\nu} 
\end{align*}

\item
We take a time-independent ansatz for the metric as
\begin{equation}
ds^2 = -e^{2\alpha(r)}dt^2 + e^{2\beta(r)}dr^2 + r^2d\Omega^2
\label{ansatz}
\end{equation}
where $d\Omega$ is defined as the spherically symmetric solid angle given as
$$d\Omega^2 \equiv d\theta^2 + \sin(\theta)d\phi^2$$
Now we can find solutions for $\alpha(r)$ and $\beta(r)$ by using the relation we found in part (a)
\begin{equation}
R_{\mu\nu} = \Lambda g_{\mu\nu}
\label{parta}
\end{equation}
We are given the non-zero components of the Ricci Tensor as
\begin{align*}
R_{tt} &= e^{2(\alpha-\beta)}\left(\alpha''+(\alpha')^2-\alpha'\beta'+\frac{2\alpha'}{r}\right)\\
R_{rr} &= -\alpha''-(\alpha')^2 + \alpha'\beta'+\frac{2\beta'}{r}\\
R_{\theta\theta} &= e^{-2\beta}\left(r(\beta'-\alpha')-1\right)+1\\
R_{\phi\phi} &= \sin(\theta)R_{\theta\theta}
\end{align*}
Note that primes denote derivatives with respect to $r$. We can see that $R_{tt}$ and $R_{rr}$ are equal and opposite except for the scaling factor so we can take the difference by
\begin{align*}
e^{2(\beta-\alpha)}R_{tt} + R_{rr} &= \cancelto{1}{e^{2(\beta-\alpha)}e^{2(\alpha-\beta)}}\left(\alpha''+(\alpha')^2-\alpha'\beta'+\frac{2\alpha'}{r}\right)  -\alpha''-(\alpha')^2 + \alpha'\beta'+\frac{2\beta'}{r}\\
&= \alpha''+(\alpha')^2-\alpha'\beta'+\frac{2\alpha'}{r}  -\alpha''-(\alpha')^2 + \alpha'\beta'+\frac{2\beta'}{r}\\
&= \frac{2\alpha'}{r}   + \frac{2\beta'}{r} = \frac{2}{r}(\alpha'+\beta')
\end{align*}
Now we can use equation \ref{parta} to find that
\begin{align*}
e^{2(\beta-\alpha)}R_{tt} + R_{rr} &= e^{2(\beta-\alpha)}\Lambda g_{tt} + \Lambda g_{rr}\\
&= \Lambda\left(e^{2(\beta-\alpha)}\left(-e^{2\alpha}\right) + e^{2\beta}\right)\\
&= \Lambda\left(-e^{2(\beta-\alpha)+2\alpha} + e^{2\beta}\right)\\
&= \Lambda\left(-e^{2\beta} + e^{2\beta}\right) = 0
\end{align*}
So we can combine these two results to find that
\begin{align*}
\frac{2}{r}(\alpha'+\beta') &= 0\\
&\Downarrow\\
\alpha' &= -\beta'
\end{align*}
So by integration we can find that 
$$\alpha(r) = -\beta(r) + C$$
where $C$ is a constant of integration. We see that the metric is independent of time therefore we can rescale time such that the constant $C$ is zero so we have
$$\alpha(r) = -\beta(r) $$
Now we can take the $R_{\theta\theta}$ component and replace the $\beta$ with the $\alpha$ using the relation we found
\begin{align*}
R_{\theta\theta} &= e^{-2\beta}\left(r(\beta'-\alpha')-1\right)+1\\
&= e^{2\alpha}\left(r(-\alpha'-\alpha')-1\right)+1\\
&= e^{2\alpha}\left(-2r\alpha'-1\right)+1\\
&= -e^{2\alpha}\left(2r\alpha' + 1\right)+1\\
&= -\partial_r\left(re^{2\alpha}\right)+1
\end{align*}
Note that we used the reverse of the chain rule to condense the term into a single differential. So now we can use equation \ref{parta} to find that
\begin{align*}
R_{\theta\theta} &= \Lambda g_{\theta\theta}\\
&\Downarrow\\
-\partial_r\left(re^{2\alpha}\right)+1 &= \Lambda r^2\\
-\partial_r\left(re^{2\alpha}\right) &= \Lambda r^2 - 1\\
\partial_r\left(re^{2\alpha}\right) &= -\Lambda r^2 + 1\\
&\Downarrow\\
re^{2\alpha} &= -\frac{\Lambda}{3}r^3 + r + R_s\\
e^{2\alpha} &= -\frac{\Lambda}{3}r^2 + 1 + \frac{R_s}{r}
\end{align*}
And we can say that
$$e^{-2\beta} = e^{2\alpha} = -\frac{\Lambda}{3}r^2 + 1 +\frac{R_s}{r}$$
Note that $R_s$ is a constant of integration and to get the Schwartzschild metric we set it to $R_s = -2GM$. So our metric become
$$g_{\mu\nu} = -\left(-\frac{\Lambda}{3}r^2 + 1 - \frac{2GM}{r}\right)dt^2 + \left(-\frac{\Lambda}{3}r^2 + 1 - \frac{2GM}{r}\right)^{-1}dr^2 + r^2d\Omega^2$$
Note that as $\Lambda\rightarrow 0$ we get the Schwartzschild limit.

\item
We can find the conserved energy by 
\begin{equation}
E = -K_{\mu}\frac{dx^{\mu}}{d\lambda}
\label{ConEn}
\end{equation}
Where $K_{\mu}$ is the Killing vector with its index lowered given by
$$K_{\mu} = g_{\mu\nu}K^{\nu}$$
 The Killing vector that is associated with the conservation of energy is the timelike Killing vector 
$$K^{\mu} = (1,0,0,0)$$
so we see that in equation \ref{ConEn} the $t$ component is picked out so
\begin{align*}
E &= -K_{\mu}\frac{dx^{\mu}}{d\lambda}\\
&= -g_{\mu\nu}K^{\nu}\frac{dx^{\mu}}{d\lambda}\\
&= -g_{\mu t}K^{t}\frac{dx^{\mu}}{d\lambda}\\
&= -g_{tt}K^{t}\frac{dt}{d\lambda}\\
&= \left(-\frac{\Lambda}{3}r^2 + 1 - \frac{2GM}{r}\right)\frac{dt}{d\lambda}
\end{align*}
For the angular momentum we have the conserved quantity
\begin{equation}
L = R_{\mu}\frac{dx^{\mu}}{d\lambda}
\label{AngMon}
\end{equation}
Where $R_{\mu}$ is the Killing vector given by
$$R_{\mu} = g_{\mu\nu}R^{\nu}$$
where
$$R^{\mu} = (0,0,0,1)$$
note that this picks out the $\phi$ component. So equation \ref{AngMon} yields
\begin{align*}
L &= R_{\mu}\frac{dx^{\mu}}{d\lambda}\\
&= g_{\mu\nu}R^{\nu}\frac{dx^{\mu}}{d\lambda}\\
&= g_{\phi\phi}R^{\phi}\frac{d\phi}{d\lambda}\\
&= r^2\sin^2(\theta)\frac{d\phi}{d\lambda}
\end{align*}
Note that we can take the angle $\theta$ to be any arbitrary angle so we take $\theta = \frac{\pi}{2}$. So this makes the angular momentum become
$$L = r^2\frac{d\phi}{d\lambda}$$

\item
To find the radial geodesic equation $r(\lambda)$ by using 
$$-g_{\mu\nu}\frac{dx^{\mu}}{d\lambda}\frac{dx^{\nu}}{d\lambda} = 1$$ 
Note that we already take $\theta = \frac{\pi}{2}$ and the $g_{\mu\nu}$ is zero for $\mu\ne\nu$ so we have
\begin{align*}
g_{tt}\left(\frac{dt}{d\lambda}\right)^2 + g_{rr}\left(\frac{dr}{d\lambda}\right)^2 + g_{\phi\phi}\left(\frac{d\phi}{d\lambda}\right)^2 &= -1\\
-\left(-\frac{\Lambda}{3}r^2 + 1 - \frac{2GM}{r}\right)\left(\frac{dt}{d\lambda}\right)^2 +  \left(-\frac{\Lambda}{3}r^2 + 1 - \frac{2GM}{r}\right)^{-1}\left(\frac{dr}{d\lambda}\right)^2 + r^2\left(\frac{d\phi}{d\lambda}\right)^2 &= -1
\end{align*}
Now we can multiply by $g_{tt}$ to cancel out the term in front of the $r$ component. 
$$-\left(-\frac{\Lambda}{3}r^2 + 1 - \frac{2GM}{r}\right)^2\left(\frac{dt}{d\lambda}\right)^2 +  \left(\frac{dr}{d\lambda}\right)^2 + \left(-\frac{\Lambda}{3}r^2 + 1 - \frac{2GM}{r}\right)\left(r^2\left(\frac{d\phi}{d\lambda}\right)^2 + 1\right) = 0$$
Now we see that we have relations to the conserved angular momentum and energy by
\begin{align*}
E^2 &= \left(-\frac{\Lambda}{3}r^2 + 1 - \frac{2GM}{r}\right)^2\left(\frac{dt}{d\lambda}\right)^2\\
\frac{L^2}{r^2} &= r^2\left(\frac{d\phi}{d\lambda}\right)^2
\end{align*}
So we have
$$-E^2 + \left(\frac{dr}{d\lambda}\right)^2 + \left(-\frac{\Lambda}{3}r^2 + 1 - \frac{2GM}{r}\right)\left(\frac{L^2}{r^2} + 1\right) = 0$$
Now we want to write this in the form 
$$\frac{1}{2}\left(\frac{dr}{d\lambda}\right)^2 + V(r) = \emf$$
so we move $E$ to the right hand side and multiply by one half to get
\begin{align*}
-E^2 + \left(\frac{dr}{d\lambda}\right)^2 + \left(-\frac{\Lambda}{3}r^2 + 1 - \frac{2GM}{r}\right)\left(\frac{L^2}{r^2} + 1\right) &= 0\\
&\Downarrow\\
\frac{1}{2}\left(\frac{dr}{d\lambda}\right)^2 + \frac{1}{2}\left(-\frac{\Lambda}{3}r^2 + 1 - \frac{2GM}{r}\right)\left(\frac{L^2}{r^2} + 1\right) &= \frac{1}{2}E^2\\
\frac{1}{2}\left(\frac{dr}{d\lambda}\right)^2 + \frac{1}{2}\left(-\frac{\Lambda}{3}r^2 + 1 - \frac{2GM}{r}-\frac{\Lambda L^2}{3} + \frac{L^2}{r^2} - \frac{2GML^2}{r^3}\right) &= \frac{1}{2}E^2\\
\frac{1}{2}\left(\frac{dr}{d\lambda}\right)^2 + \frac{1}{2} - \frac{\Lambda L^2}{6}  - \frac{\Lambda}{6}r^2 - \frac{GM}{r} + \frac{L^2}{2r^2} - \frac{GML^2}{r^3} &= \frac{1}{2}E^2
\end{align*}
We see that in this form we have
$$\emf = \frac{1}{2}E^2$$
and
$$V(r) =  \frac{1}{2} - \frac{\Lambda L^2}{6}  - \frac{\Lambda}{6}r^2 - \frac{GM}{r} + \frac{L^2}{2r^2} - \frac{GML^2}{r^3}$$
We see that we have the typical effective potential for the Schwarzschild solution but in this case we have an extra two terms dependent on the cosmological constant.

\item
We see by the effective potential we found in part (d) that the cosmological constant dominates the motion for large $r$. So the potential term for large $r$ is
$$V(r) = -\frac{\Lambda}{6}r^2$$
Now we can see the force associated with this potential by using the relation
$$F = -\frac{dV}{dr}$$
so this potential gives the force
$$F = \frac{\Lambda}{3}r$$
So for a positive cosmological constant we have a repulsive force and for a negative cosmological constant we have an attractive force.

\item
We can think of the cosmological constant as a perfect fluid where we think that $\Lambda$ is proportional to the energy density $\rho$. We say that
$$-p = \rho = \frac{\Lambda}{8\pi G}$$
Note that we flipped the negative sign then what was given in the problem. We can see that if $\Lambda>0$ we have that $\rho>0$ this implies that
$$\rho+3p = -2\rho<0$$
this implies that gravity is a repulsive force. This is in agreement with what we found in part (e). And for $\Lambda<0$ we have $\rho<0$ which implies that
$$\rho+3p = -2\rho>0$$
this implies that gravity is an attractive force. Again this agrees with the result we found in part (e).


\end{enumerate}

\end{document}

