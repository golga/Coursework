\documentclass[11pt]{article}

\usepackage{latexsym}
\usepackage{graphicx}
\usepackage{amssymb}
\usepackage{amsthm}
\usepackage{enumerate}
\usepackage{amsmath}
\usepackage{cancel}
\numberwithin{equation}{section}

\setlength{\evensidemargin}{.25in}
\setlength{\oddsidemargin}{-.25in}
\setlength{\topmargin}{-.75in}
\setlength{\textwidth}{6.5in}
\setlength{\textheight}{9.5in}
\newcommand{\due}{April 20th, 2016}
\newcommand{\HWnum}{6}
\newcommand{\grad}{\bold\nabla}
\newcommand{\vecE}{\vec{E}}
\newcommand{\scrptR}{\vec{\mathfrak{R}}}
\newcommand{\kapa}{\frac{1}{4\pi\epsilon_0}}
\newcommand{\emf}{\mathcal{E}}
\newcommand{\unit}[1]{\ensuremath{\, \mathrm{#1}}}
\newcommand{\real}{\textnormal{Re}}
\newcommand{\Erf}{\textnormal{Erf}}
\newcommand{\sech}{\textnormal{sech}}
\newcommand{\scrO}{\mathcal{O}}
\newcommand{\levi}{\widetilde{\epsilon}}
\newcommand{\partiald}[2]{\ensuremath{\frac{\partial{#1}}{\partial{#2}}}}
\newcommand{\norm}[2]{\langle{#1}|{#2}\rangle}
\newcommand{\inprod}[2]{\langle{#1}|{#2}\rangle}
\newcommand{\ket}[1]{|{#1}\rangle}
\newcommand{\bra}[1]{\langle{#1}|}





\begin{document}
\begin{titlepage}
\setlength{\topmargin}{1.5in}
\begin{center}
\Huge{Physics 3320} \\
\LARGE{Principles of Electricity and Magnetism II} \\
\Large{Professor Ana Maria Rey} \\[1cm]

\huge{Homework \#\HWnum}\\[0.5cm]

\large{Joe Becker} \\
\large{SID: 810-07-1484} \\
\large{\due} 

\end{center}

\end{titlepage}



\section{Problem \#1}
For a diatomic molecule where the nuclei move in a potential 
$$V(r) = -2D\left(\frac{1}{\rho}-\frac{1}{2\rho^2}\right)$$
where $\rho=r/a$, $a$ is the characteristic size, this effective potential is given by
$$V_{eff}(r) = V(r) + \frac{\hbar^2}{2\mu{r^2}}K(K+1)$$
where $K$ is an integer. We can approximate the effective potential as a harmonic oscillator
about the minimum, $r_0$, which we calculate as
\begin{align*}
\left.\partiald{V_{eff}}{r}\right|_{r_0} = 0 &= -2D\left(-\frac{a}{r_0^2}+\frac{a^2}{r_0^3}\right) - \frac{\hbar^2}{\mu{r_0^3}}K(K+1)\\
&\Downarrow\\
\frac{2Da}{r_0^2} &= \frac{2Da^2}{r_0^3} + \frac{\hbar^2}{\mu{r_0^3}}K(K+1)\\
&\Downarrow\\
r_0 &= a + \frac{\hbar^2}{2\mu{Da}}K(K+1)
\end{align*}
Now we can Taylor expand the effective potential about $r_0$ to get a parabolic approximation
\begin{align*}
V_{eff}(r) &= V_{eff}(r_0) + \cancelto{0}{\left.\partiald{V_{eff}}{r}\right|_{r_0}}(r-r_0) + \frac{1}{2}\left.\partiald{^2V_{eff}}{r^2}\right|_{r_0}(r-r_0)^2\\
&= V_{eff}(r_0)  + \frac{1}{2}\left.\partiald{^2V_{eff}}{r^2}\right|_{r_0}(r-r_0)^2\\
&\Downarrow\\
&= V_{eff}(r_0)  + \frac{1}{2}\mu\omega^2(r-r_0)^2
\end{align*}
Note that this approximation transformed the effective potential into a harmonic oscillator
potential where
\begin{align*}
\mu\omega^2 = \left.\partiald{^2V_{eff}}{r^2}\right|_{r_0} &= -2D\left(\frac{2a}{r_0^3}-\frac{3a^2}{r_0^4}\right) + \frac{3\hbar^2}{\mu{r_0^4}}K(K+1)\\
&= -4Da\left(\frac{2\mu{Da}}{2\mu{Da^2}+\hbar^2K(K+1)}\right)^3 + 3\frac{2\mu{Da^2}+\hbar^2K(K+1)}{\mu}\left(\frac{2\mu{Da}}{2\mu{Da^2}+\hbar^2K(K+1)}\right)^4\\
&= -\frac{2}{\mu}\frac{(2\mu{Da})^4}{(2\mu{Da^2}+\hbar^2K(K+1))^3} + \frac{3}{\mu}\frac{(2\mu{Da})^4}{(2\mu{Da^2}+\hbar^2K(K+1))^3}\\
&\Downarrow\\
\omega &= \frac{(2\mu{Da})^2}{\mu(2\mu{Da^2}+\hbar^2K(K+1))^{3/2}}
\end{align*}
which gives us the energy levels
$$E = V(r_0) + \frac{\hbar^2K(K+1)}{2I} + \frac{4\hbar\mu({Da})^2}{(2\mu{Da^2}+\hbar^2K(K+1))^{3/2}}\left(n+\frac{1}{2}\right)$$
where $I=\mu{r_0^2}$. Note the first two terms are constant.

\pagebreak

\section{Problem \#2}
We can use the variational method with the trial function
$$\psi = Ae^{-\beta{r}}$$
to find the ground state energy of the hydrogen atom with has the Hamiltonian
$$H = -\frac{\hbar^2}{2\mu}\frac{1}{r^2}\partiald{}{r}\left(r^2\partiald{}{r}\right) - \frac{e^2}{r}$$ 
First we find $A$ by the normalization condition
\begin{align*}
\int\psi^*\psi{dr} = 1 &= |A|^24\pi\int_{0}^{\infty}e^{-2\beta{r}}r^2dr\\
&= |A|^2\frac{\pi}{\beta^3}\\
&\Downarrow\\
A &= \sqrt{\frac{\beta^{3}}{\pi}}
\end{align*}
Then using the condition of the variational method which states
$$E_0\le\bra{\psi}H\ket{\psi}$$
so we calculate
\begin{align*}
\bra{\psi}H\ket{\psi} = E(\beta) &= \frac{\beta^3}{\pi}4\pi\int_{0}^{\infty}e^{-\beta{r}}\left(-\frac{\hbar^2}{2\mu}\frac{1}{r^2}\partiald{}{r}\left(r^2\partiald{}{r}\right) - \frac{e^2}{r}\right)e^{-\beta{r}}r^2dr\\
&= 4\beta^3\int_{0}^{\infty}\frac{\hbar^2}{2\mu}2\beta{e^{-2\beta{r}}}r - \frac{\hbar^2}{2\mu}\beta^2e^{-2\beta{r}}r^2 - e^2e^{-2\beta{r}}rdr\\
&= \frac{\hbar^2}{2\mu}\beta^2 - e^2\beta
\end{align*}
Now we minimize $E(\beta)$ to find that
\begin{align*}
\left.\partiald{E(\beta)}{\beta}\right|_{\beta_0} = 0 &= \frac{\hbar^2}{\mu}\beta_0 - e^2\\
\Downarrow\\
\beta_0 &= \frac{e^2\mu}{\hbar^2} = \frac{1}{a}
\end{align*}
where $a$ is the \emph{Bohr Radius}. This yields the ground state energy
\begin{align*}
E_0 \le E(\beta_0) &= \frac{\hbar^2}{2\mu}\beta_0^2 - e^2\beta_0\\
&= \frac{\hbar^2}{2\mu}\frac{e^4\mu^2}{\hbar^4} - \frac{e^4\mu}{\hbar^2}\\
&= -\frac{e^4\mu}{2\hbar^2}
\end{align*}
Note that this recovers the exact result for the ground state of hydrogen. Also we see that we
recover the exact ground state wave function
$$\psi(r) = \frac{1}{\sqrt{\pi{a^3}}}e^{-r/a}$$

\pagebreak

\section{Problem \#3}
Two particles have equal spins $s_1=s_2=1$. We can find the wave function describing the 
states with the overall spin $S=1$ and $2$ and $S_z=+1$ and $-1$, as well as $S=1$ and $S_z=0$
in the $s_{1z}s_{2z}$ representation by first noting that the states with $S=2$ and $S_z=+2$ 
and $-2$ in the $s_{1z}s_{2z}$ representation are
$$\Psi_{2,2} = \left(\begin{array}{c} 1\\ 0\\ 0\\\end{array}\right)_{1}\left(\begin{array}{c} 1\\ 0\\ 0\\\end{array}\right)_{2} \qquad
\Psi_{2,-2} = \left(\begin{array}{c} 0\\ 0\\ 1\\\end{array}\right)_{1}\left(\begin{array}{c} 0\\ 0\\ 1\\\end{array}\right)_{2} $$
Using these states we can act the ladder operators on the state where
$$\hat{L}_{\pm}\Psi_{S,S_z} = \sqrt{(S\mp{S_z})(S\pm{S_z}+1)}\Psi_{S,S_z-1}$$
where we can take the ladder operator in $s_{1z}s_{2z}$ representation as
$$\hat{L}_{\pm} = \hat{L}_{1\pm}+\hat{L}_{2\pm}$$
so we can act of the state with $S=2$ and $S_z=+2$ as
\begin{align*}
\hat{L}_{-}\Psi_{2,2} &= (\hat{L}_{1-}+\hat{L}_{2-})\left(\begin{array}{c} 1\\ 0\\ 0\\\end{array}\right)_{1}\left(\begin{array}{c} 1\\ 0\\ 0\\\end{array}\right)_{2} \\
&\Downarrow\\
2\Psi_{2,1} &= \sqrt{2}\left(\begin{array}{c} 0\\ 1\\ 0\\\end{array}\right)_{1}\left(\begin{array}{c} 1\\ 0\\ 0\\\end{array}\right)_{2} +
\sqrt{2}\left(\begin{array}{c} 1\\ 0\\ 0\\\end{array}\right)_{1}\left(\begin{array}{c} 0\\ 1\\ 0\\\end{array}\right)_{2} \\
\Psi_{2,1} &= \frac{1}{\sqrt{2}}\left[\left(\begin{array}{c} 0\\ 1\\ 0\\\end{array}\right)_{1}\left(\begin{array}{c} 1\\ 0\\ 0\\\end{array}\right)_{2} +
\left(\begin{array}{c} 1\\ 0\\ 0\\\end{array}\right)_{1}\left(\begin{array}{c} 0\\ 1\\ 0\\\end{array}\right)_{2}\right]
\end{align*}
And by raising the $S=2$ $S_z=-2$ state we get
\begin{align*}
\hat{L}_{+}\Psi_{2,-2} &= (\hat{L}_{1+}+\hat{L}_{2+})\left(\begin{array}{c} 0\\ 0\\ 1\\\end{array}\right)_{1}\left(\begin{array}{c} 0\\ 0\\ 1\\\end{array}\right)_{2} \\
&\Downarrow\\
2\Psi_{2,-1} &= \sqrt{2}\left(\begin{array}{c} 0\\ 1\\ 0\\\end{array}\right)_{1}\left(\begin{array}{c} 0\\ 0\\ 1\\\end{array}\right)_{2} +
\sqrt{2}\left(\begin{array}{c} 0\\ 0\\ 1\\\end{array}\right)_{1}\left(\begin{array}{c} 0\\ 1\\ 0\\\end{array}\right)_{2} \\
\Psi_{2,-1} &= \frac{1}{\sqrt{2}}\left[\left(\begin{array}{c} 0\\ 1\\ 0\\\end{array}\right)_{1}\left(\begin{array}{c} 0\\ 0\\ 1\\\end{array}\right)_{2} +
\left(\begin{array}{c} 0\\ 0\\ 1\\\end{array}\right)_{1}\left(\begin{array}{c} 0\\ 1\\ 0\\\end{array}\right)_{2}\right]
\end{align*}
We note that these state have parallel spins as the wave function is additive. This yields
a $S=2$ state if we make the wave function anti-parallel we can find the $S=1$ states as
\begin{align*}
\Psi_{1,1} &= \frac{1}{\sqrt{2}}\left[\left(\begin{array}{c} 0\\ 1\\ 0\\\end{array}\right)_{1}\left(\begin{array}{c} 1\\ 0\\ 0\\\end{array}\right)_{2} -
\left(\begin{array}{c} 1\\ 0\\ 0\\\end{array}\right)_{1}\left(\begin{array}{c} 0\\ 1\\ 0\\\end{array}\right)_{2}\right]\\
\Psi_{1,-1} &= \frac{1}{\sqrt{2}}\left[\left(\begin{array}{c} 0\\ 1\\ 0\\\end{array}\right)_{1}\left(\begin{array}{c} 0\\ 0\\ 1\\\end{array}\right)_{2} -
\left(\begin{array}{c} 0\\ 0\\ 1\\\end{array}\right)_{1}\left(\begin{array}{c} 0\\ 1\\ 0\\\end{array}\right)_{2}\right]
\end{align*}
From here we can easily find the $S=1$ $S_z=0$ state by using a ladder operator
\begin{align*}
\hat{L}_{-}\Psi_{1,1} &= \frac{1}{\sqrt{2}}(\hat{L}_{1-}+\hat{L}_{2-})\left[\left(\begin{array}{c} 0\\ 1\\ 0\\\end{array}\right)_{1}\left(\begin{array}{c} 1\\ 0\\ 0\\\end{array}\right)_{2} -
\left(\begin{array}{c} 1\\ 0\\ 0\\\end{array}\right)_{1}\left(\begin{array}{c} 0\\ 1\\ 0\\\end{array}\right)_{2}\right]\\
&\Downarrow\\
\sqrt{2}\Psi_{1,0} &= \frac{1}{\sqrt{2}}
\left[\sqrt{2}\left(\begin{array}{c} 0\\ 0\\ 1\\\end{array}\right)_{1}\left(\begin{array}{c} 1\\ 0\\ 0\\\end{array}\right)_{2} -
\sqrt{2}\left(\begin{array}{c} 0\\ 1\\ 0\\\end{array}\right)_{1}\left(\begin{array}{c} 0\\ 1\\ 0\\\end{array}\right)_{2}\right. + 
\left.\sqrt{2}\left(\begin{array}{c} 0\\ 1\\ 0\\\end{array}\right)_{1}\left(\begin{array}{c} 0\\ 1\\ 0\\\end{array}\right)_{2} -
\sqrt{2}\left(\begin{array}{c} 1\\ 0\\ 0\\\end{array}\right)_{1}\left(\begin{array}{c} 0\\ 0\\ 1\\\end{array}\right)_{2}\right]\\
\Psi_{1,0} &= \frac{1}{\sqrt{2}}\left[\left(\begin{array}{c} 0\\ 0\\ 1\\\end{array}\right)_{1}\left(\begin{array}{c} 1\\ 0\\ 0\\\end{array}\right)_{2} -
\left(\begin{array}{c} 1\\ 0\\ 0\\\end{array}\right)_{1}\left(\begin{array}{c} 0\\ 0\\ 1\\\end{array}\right)_{2}\right]
\end{align*}

\end{document}

