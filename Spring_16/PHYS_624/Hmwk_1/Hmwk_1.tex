\documentclass[11pt]{article}

\usepackage{latexsym}
\usepackage{graphicx}
\usepackage{amssymb}
\usepackage{amsthm}
\usepackage{enumerate}
\usepackage{amsmath}
\usepackage{cancel}
\numberwithin{equation}{section}

\setlength{\evensidemargin}{.25in}
\setlength{\oddsidemargin}{-.25in}
\setlength{\topmargin}{-.75in}
\setlength{\textwidth}{6.5in}
\setlength{\textheight}{9.5in}
\newcommand{\due}{February 7th, 2016}
\newcommand{\HWnum}{1}
\newcommand{\grad}{\bold\nabla}
\newcommand{\vecE}{\vec{E}}
\newcommand{\scrptR}{\vec{\mathfrak{R}}}
\newcommand{\kapa}{\frac{1}{4\pi\epsilon_0}}
\newcommand{\emf}{\mathcal{E}}
\newcommand{\unit}[1]{\ensuremath{\, \mathrm{#1}}}
\newcommand{\real}{\textnormal{Re}}
\newcommand{\Erf}{\textnormal{Erf}}
\newcommand{\sech}{\textnormal{sech}}
\newcommand{\scrO}{\mathcal{O}}
\newcommand{\levi}{\widetilde{\epsilon}}
\newcommand{\partiald}[2]{\ensuremath{\frac{\partial{#1}}{\partial{#2}}}}
\newcommand{\norm}[2]{\langle{#1}|{#2}\rangle}
\newcommand{\inprod}[2]{\langle{#1}|{#2}\rangle}
\newcommand{\ket}[1]{|{#1}\rangle}
\newcommand{\bra}[1]{\langle{#1}|}





\begin{document}
\begin{titlepage}
\setlength{\topmargin}{1.5in}
\begin{center}
\Huge{Physics 3320} \\
\LARGE{Principles of Electricity and Magnetism II} \\
\Large{Professor Ana Maria Rey} \\[1cm]

\huge{Homework \#\HWnum}\\[0.5cm]

\large{Joe Becker} \\
\large{SID: 810-07-1484} \\
\large{\due} 

\end{center}

\end{titlepage}



\section{Problem \#1}
For the case of a time-independent Hamiltonian the propagator, $K(q,t;q',t'=0)$ satisfies the
equation
$$\hat{q}(-t)K(q,t;q',t'=0) = cK(q,t;q',t'=0),$$
where $\hat{q}(t)$ is an operator in the Heisenberg representation. We can find $c$ by noting
that the Heisenberg representation of the operator $\hat{q}(t)$ is given by
$$\hat{q}(t) = S^{-1}(t)\hat{q}S(t)$$
where $S$ is the unitary transformation given by 
$$S = \exp\left(-\frac{i}{\hbar}\hat{H}t\right)$$
Therefore we see that the operator becomes $\hat{q}(-t) = S^{-1}(-t)\hat{q}S(-t)$. This 
allows us to treat the propagator as a wave function dependent on the parameter $q$ and 
$t$ that is being acted on by the transformation $S(-t)$. We note that $S(-t)$ evolves the
wave function that it acts upon backwards $-t$ in time. This implies that
$$S(-t)K(q,t;q',t'=0) = K(q,0;q',0) = \delta(q'-q)$$
we note that the delta function arises by the property of the propagator when $t'=t$. Using
the fact that $\delta(q'-q)$ is an eigenfunction of the operator $\hat{q}$ we note that
\begin{align*}
\hat{q}(-t)K(q,t;q',t'=0) &=  S^{-1}(-t)\hat{q}S(-t)K(q,t;q',t'=0)\\
&=  S^{-1}(-t)\hat{q}\delta(q'-q)\\
&=  q'S(t)\delta(q'-q)
\end{align*}
We note that the inverse transformation on the negative time step $-t$ is the same as the
transformation with a positive time step $t$. This transformation acting on the delta 
function, $\delta(q'-q)$, recovers the propagator with implies that
$$\hat{q}(-t)K(q,t;q',t'=0) = q'K(q,t;q',t'=0)$$
or that $c=q'$. Therefore when we act $\hat{q}(-t)$ on the propagator we recover the initial
condition $q'$.


\pagebreak

\section{Problem \#2}
\begin{enumerate}[(a)]
\item We can find the Green's Function, $G_E(x,x')$ of the Schr\"{o}dinger's Equation for a
free particle with $E<0$ vanishing at $|x-x'|\rightarrow\infty$ by noting that $G_E(x,x')$
solves Schr\"{o}dinger's Equation by
$$(\hat{H}-E)G_E(x,x')\equiv-\frac{\hbar^2}{2m}\partiald{^2}{x^2}G_E(x,x')-EG_E(x,x') = \delta(x-x').$$
We can solve this equation by using the general solution of $G_E$ given by
$$G_E(x,x') = A(x')e^{k|x-x'|} + B(x')e^{-k|x-x'|}$$
where $k=\sqrt{-2mE}/\hbar$ note that we assume that $E<0$ which makes the solution an 
exponential. Applying the assumption that the particle vanishes for $|x-x'|\rightarrow\infty$
we see that $A(x')=0$ must be true. In order to find $B(x')$ we note that $G_E$ can be 
considered as the solution to the delta function potential which yields a discontinuity in 
the derivative of $G_E$ at $x=x'$ which implies that
\begin{align*}
\frac{dG_E(x',x')}{dx} = -\frac{B(x')}{k}e^{-k|x-x'|} &= -\frac{2m}{\hbar^2}\\
&\Downarrow\\
B &= \frac{m}{k\hbar^2}
\end{align*}
Note the factor of $2m/\hbar^2$ follows from the solution to the delta function potential. 
Therefore, the Green function is 
$$G_E(x,x') = \frac{m}{k\hbar^2}e^{-k|x-x'|}$$

\item We can use this Green's function to represent the Schr\"{o}dinger's equation with a
short-range potential $U(x)$ where $U(x)\rightarrow0$ as $x\rightarrow\infty$. This makes the
Schr\"{o}dinger's equation become
$$-\frac{\hbar^2}{2m}\partiald{^2}{x^2}\psi(x)-E\psi(x) = U(x)\psi(x)$$
which allows us to use the general fact that a Green's function gives the solution to an
inhomogeneous differential equation by an integral
\begin{equation}
y(x) = \int_{-\infty}^{\infty}G(x,x')f(x')dx'
\label{Green}
\end{equation}
So for this potential we have equation \ref{Green} become
$$\psi(x) = \int_{-\infty}^{\infty}-\frac{m}{k\hbar^2}e^{-k|x-x'|}U(x')\psi(x')dx'$$

\item In order to find the momentum representation of $G_E$ we note that for any linear 
operator there is an associated Green's function given by
$$\hat{L}\psi(\xi)\equiv\int L(\xi,\xi')\psi(\xi')d\xi'$$
this implies that there exists an operator associated with $G_E$ which we will represent with
$\hat{G}_E$. Given that the Green's function is a solution to the equation
$$(\hat{H}-E)G_E(x-x') = \delta(x-x')$$
we can write this relation independent of representation using the fact that the delta 
function is the position space representation of the identity operator, $\hat{I}$. Therefore,
\begin{align*}
(\hat{H}-E)\hat{G}_E &= \hat{I}\\
&\Downarrow\\
\hat{G}_E &= (\hat{H}-E)^{-1}
\end{align*}
Now we can simply write the Hamiltonian in momentum representation as $\hat{H} = p^2/2m$ 
which yields
$$G_E(p) = \frac{1}{p^2/2m-E}$$

\end{enumerate}

\section{Problem \#3}
\begin{enumerate}[(a)]
\item Given a particle in the field generated by a uniform force given by
$$U = -\mathbf{F_0}\cdot\mathbf{r}$$
we can find the coordinate representation of the propagator, $K(\mathbf{r},t,\mathbf{r'},t_0)$
by first using the relation found in Problem 1 which states in three dimensions
$$\mathbf{\hat{r}}(-t)K(\mathbf{r},t,\mathbf{r'},t_0) = \mathbf{r'}K(\mathbf{r},t,\mathbf{r'},t_0)$$
where $\mathbf{\hat{r}}(t)$ is the Heisenberg representation of the position operator given 
by
\begin{align*}
\mathbf{\hat{r}}(t) = \hat{S}^{-1}(t)\hat{\mathbf{r}}\hat{S} &= \exp\left(i\frac{\hat{H}}{\hbar}t\right)\hat{\mathbf{r}}\exp\left(-i\frac{\hat{H}}{\hbar}t\right)\\
&= \left(1+\frac{i\hat{H}}{\hbar}t+...\right)\hat{\mathbf{r}}\left(1-\frac{i\hat{H}}{\hbar}t+...\right)\\
&= \hat{\mathbf{r}} + \frac{i}{\hbar}[\hat{H},\hat{\mathbf{r}}]t - \frac{1}{2\hbar^2}[\hat{H},[\hat{H},\hat{\mathbf{r}}]]t^2
\end{align*}
Note that the commutation relation between $\hat{H}$ and $\hat{\mathbf{r}}$ is given as
$$[\hat{H},\hat{\mathbf{r}}] = -\frac{i\hbar}{m}\hat{\mathbf{p}}$$
and
$$[\hat{H},\hat{\mathbf{p}}] = -i\hbar\partiald{U}{d\mathbf{r}}$$
therefore we have
$$\mathbf{\hat{r}}(t) = \hat{\mathbf{r}} + \frac{\hat{\mathbf{p}}}{m}t + \frac{\mathbf{F_0}}{2m}t^2$$
so if we use the relation we find that
\begin{align*}
\left(\hat{\mathbf{r}} - \frac{\mathbf{p}}{m}t + \frac{\mathbf{F_0}}{2m}t^2\right)K(\mathbf{r},t,\mathbf{r'},t_0) &= \mathbf{r'}K(\mathbf{r},t,\mathbf{r'},t_0)
\end{align*}
which if we generalize to a free particle in three dimensions but where $\mathbf{r'}= \mathbf{r}-\mathbf{F_0}t^2/2m$
which yields
$$K(\mathbf{r},t,\mathbf{r'},t_0) = \left(\frac{m}{2\pi{i}\hbar(t-t_0)}\right)^{3/2}\exp\left(\frac{i}{\hbar}\left(\frac{1}{2m(t-t_0)}\left(\mathbf{r}-\mathbf{r'}-\frac{\mathbf{F_0}t^2}{2m}\right)^2+\mathbf{F_0}\cdot\mathbf{r}t-\frac{F_0^2t^3}{6m}\right)\right)$$

\item We follow the same process for the momentum representation noting that
\begin{align*}
\hat{\mathbf{p}}(t) &= \hat{\mathbf{p}} + \frac{i}{\hbar}[\hat{H},\hat{\mathbf{p}}]t - \frac{1}{2\hbar^2}\cancelto{0}{[\hat{H},[\hat{H},\hat{\mathbf{p}}]]}t^2\\
 &= \hat{\mathbf{p}} + \frac{\mathbf{F_0}}{m}t 
\end{align*}
which yields
$$K(\mathbf{p},t,\mathbf{r'},t_0) = \exp\left(-\frac{i(t-t_0)}{2m\hbar}\left(p^2-\mathbf{F_0}\cdot\mathbf{p}(t-t_0)+\frac{1}{3}F_0^2(t-t_0)^2\right)\right)\delta(\mathbf{p}-\mathbf{p'}-\mathbf{F_0}(t-t_0))$$



\end{enumerate}

\end{document}

