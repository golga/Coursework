\documentclass[11pt]{article}

\usepackage{latexsym}
\usepackage{graphicx}
\usepackage{amssymb}
\usepackage{amsthm}
\usepackage{enumerate}
\usepackage{amsmath}
\usepackage{cancel}
\usepackage{mathrsfs}
\numberwithin{equation}{section}

\setlength{\evensidemargin}{.25in}
\setlength{\oddsidemargin}{-.25in}
\setlength{\topmargin}{-.75in}
\setlength{\textwidth}{6.5in}
\setlength{\textheight}{9.5in}
\newcommand{\due}{February 15th, 2016}
\newcommand{\HWnum}{2}
\newcommand{\grad}{\bold\nabla}
\newcommand{\vecE}{\vec{E}}
\newcommand{\scrptR}{\vec{\mathfrak{R}}}
\newcommand{\kapa}{\frac{1}{4\pi\epsilon_0}}
\newcommand{\emf}{\mathcal{E}}
\newcommand{\unit}[1]{\ensuremath{\, \mathrm{#1}}}
\newcommand{\real}{\textnormal{Re}}
\newcommand{\Erf}{\textnormal{Erf}}
\newcommand{\sech}{\textnormal{sech}}
\newcommand{\scrO}{\mathcal{O}}
\newcommand{\levi}{\widetilde{\epsilon}}
\newcommand{\partiald}[2]{\ensuremath{\frac{\partial{#1}}{\partial{#2}}}}
\newcommand{\norm}[2]{\langle{#1}|{#2}\rangle}
\newcommand{\inprod}[2]{\langle{#1}|{#2}\rangle}
\newcommand{\ket}[1]{|{#1}\rangle}
\newcommand{\bra}[1]{\langle{#1}|}





\begin{document}
\begin{titlepage}
\setlength{\topmargin}{1.5in}
\begin{center}
\Huge{Physics 3320} \\
\LARGE{Principles of Electricity and Magnetism II} \\
\Large{Professor Ana Maria Rey} \\[1cm]

\huge{Homework \#\HWnum}\\[0.5cm]

\large{Joe Becker} \\
\large{SID: 810-07-1484} \\
\large{\due} 

\end{center}

\end{titlepage}



\section{Problem \#1}
Given a particle in a uniform time-dependent field with a force $\mathbf{F}(t)\rightarrow{0}$
for $|t|\rightarrow\infty$ we can find the change in the average value of the energy caused
by interaction with the field, $U = -\mathbf{F}(t)\cdot\mathbf{r}(t)$ by noting that the
time evolution of an operator in the Heisenberg representation is given by the commutator 
with the Hamiltonian
$$\frac{d\hat{F}}{dt} =  \frac{1}{i\hbar}[\hat{F},\hat{H}]$$
which allows us to gain the Heisenberg equations of motion
\begin{align*}
\frac{d\hat{\mathbf{p}}(t)}{dt} &= -\grad{U} = \mathbf{F}(t)\\
\frac{d\hat{\mathbf{r}}(t)}{dt} &= \frac{\mathbf{p}}{m}
\end{align*}
Solving for $\hat{\mathbf{p}}(t)$ we integrate to get
\begin{align*}
\hat{\mathbf{p}}(t) + C &= \int_{-\infty}^{t}\mathbf{F}(t')dt'\\
&\Downarrow\\
\hat{\mathbf{p}}(t) &= \int_{-\infty}^{t}\mathbf{F}(t')dt' + \mathbf{p}_0
\end{align*}
Note we use an initial condition to find the integration constant because we assume 
$\mathbf{F}(t=-\infty)=0$ which implies that 
$$\mathbf{p}_{0} \equiv \mathbf{p}(t=-\infty)$$. We also apply this assumption to see that 
at $|t|\rightarrow\infty$ the particle acts like a free particle with the Hamiltonian
$$\lim_{|t|\rightarrow\infty}\hat{H}(t) = \frac{\hat{\mathbf{p}}^2(t=\pm\infty)}{2m}$$
therefore we can find the change in the average value of energy from $t=-\infty$ to $t=\infty$
through the Hamiltonian by noting that
\begin{align*}
\average{E(\pm\infty)} &= \average{H}(\pm\infty) = \frac{\average{\hat{\mathbf{p}}^2(\pm\infty)}}{2m}
\end{align*}
where
$$\hat{\mathbf{p}}^2(t) = \mathbf{p}_0^2 + 2\mathbf{p}_0\int_{-\infty}^{t}\mathbf{F}(t')dt' + \left(\int_{-\infty}^{t}\mathbf{F}(t')dt'\right)^2$$
we note that at $t=-\infty$ we have $\hat{\mathbf{p}}^2(-\infty) = \mathbf{p}_0^2$ as all the
integrals go to zero. So this implies that
$$\average{E(-\infty)} = \frac{\average{\mathbf{p}_0^2}}{2m}$$
and that
\begin{align*}
\average{E(\infty)} &= \frac{\average{\mathbf{p}_0^2}}{2m} + \frac{\average{\mathbf{p}_0}}{m}\int_{-\infty}^{\infty}\mathbf{F}(t')dt' + \left(\int_{-\infty}^{\infty}\mathbf{F}(t')dt'\right)^2\\
\average{E(\infty)} &= \average{E(-\infty)} + \frac{\average{\mathbf{p}_0}}{m}\int_{-\infty}^{\infty}\mathbf{F}(t')dt' + \left(\int_{-\infty}^{\infty}\mathbf{F}(t')dt'\right)^2\\
&\Downarrow\\
\average{E(\infty)} - \average{E(-\infty)} &= \frac{\average{\mathbf{p}_0}}{m}\int_{-\infty}^{\infty}\mathbf{F}(t')dt' + \left(\int_{-\infty}^{\infty}\mathbf{F}(t')dt'\right)^2
\end{align*}

\pagebreak

\section{Problem \#2}
\begin{enumerate}[(a)]
\item For a ground state harmonic oscillator under an applied external force $\mathbf{F}(t)$, 
such that $\mathbf{F}(t)\rightarrow{0}$ for $|t|\rightarrow\infty$ we can find the time 
evolution of the creation and annihilation operators, $\hat{a}$ and $\hat{a}^{\dagger}$, by taking the
Heisenberg representation 
\begin{align*}
\frac{d\hat{\mathbf{p}}(t)}{dt} &= -m\omega^2\mathbf{r} + \mathbf{F}(t)\\
\frac{d\hat{\mathbf{r}}(t)}{dt} &= \frac{\mathbf{p}}{m}
\end{align*}
This allows us to use the definition of the creation and annihilation operators to say
\begin{align*}
\frac{d\hat{a}}{dt} &= \sqrt{\frac{m\omega}{2\hbar}}\left(\frac{d\hat{\mathbf{r}}}{dt}+\frac{i}{m\omega}\frac{d\hat{\mathbf{p}}}{dt}\right)\\
&= \sqrt{\frac{m\omega}{2\hbar}}\left(\frac{\hat{\mathbf{p}}}{m} - i\omega\hat{\mathbf{r}} + \frac{i\mathbf{F}(t)}{m\omega}\right)\\
&= -i\omega\sqrt{\frac{m\omega}{2\hbar}}\left(\frac{i\hat{\mathbf{p}}}{m\omega} + \hat{\mathbf{r}}\right) + \sqrt{\frac{m\omega}{2\hbar}}\frac{i\mathbf{F}(t)}{m\omega}\\
&= -i\omega\hat{a} + i\frac{\mathbf{F}(t)}{\sqrt{2\hbar{m}\omega}}
\end{align*}
And the same follows for $\hat{a}^{\dagger}$ as
$$\frac{d\hat{a}^{\dagger}}{dt} = i\omega\hat{a}^{\dagger} - i\frac{\mathbf{F}(t)}{\sqrt{2\hbar{m}\omega}}$$
Now we can solve the differentials by direct integration using an integrating factor which 
yields
\begin{align*}
\hat{a}(t) &= a(0)e^{-i\omega{t}} + \frac{ie^{-i\omega{t}}}{\sqrt{2\hbar{m}\omega}}\int_{-\infty}^{t}\mathbf{F}(t')e^{i\omega{t'}}dt'\\
\hat{a}^{\dagger}(t) &= a^{\dagger}(0)e^{i\omega{t}} - \frac{ie^{i\omega{t}}}{\sqrt{2\hbar{m}\omega}}\int_{-\infty}^{t}\mathbf{F}(t')e^{-i\omega{t'}}dt'
\end{align*}
Where we have $a(0)$ and $a^{\dagger}(0)$ follow from the integration constants which we take
to be initial conditions. Note that the fact that $\hat{a}$ and $\hat{a}^{\dagger}$ are 
Hermitian conjugates still holds. 

\item To find the average energy as $t\rightarrow+\infty$ we note that in this limit 
$\mathbf(F)(t)=0$ so we have an unforced harmonic oscillator which has a Hamiltonian 
$$\lim_{t\rightarrow\infty}\hat{H}(t) = \hbar{\omega}\left(\hat{a}^{\dagger}(\infty)\hat{a}(\infty)+\frac{1}{2}\right)$$
where we find
\begin{align*}
\hat{a}(t)\hat{a}^{\dagger}(t) &= a(0)a^{\dagger}(0) - \frac{ia(0)}{\sqrt{2\hbar{m}\omega}}\int_{-\infty}^{t}\mathbf{F}(t')e^{-i\omega{t'}}dt' + \frac{ia^{\dagger}(0)}{\sqrt{2\hbar{m}\omega}}\int_{-\infty}^{t}\mathbf{F}(t')e^{i\omega{t'}}dt' + \left|\frac{i}{\sqrt{2\hbar{m}\omega}}\int_{-\infty}^{t}\mathbf{F}(t')e^{i\omega{t'}}dt'\right|^2
\end{align*}
note that in the $-\infty$ limit the integrals disappear which implies that
$$\average{E(-\infty)} = \hbar\omega\average{\hat{a}^{\dagger}(-\infty)\hat{a}(-\infty)+\frac{1}{2}} = \hbar\omega\average{a^{\dagger}(0)a(0)+\frac{1}{2}}$$
which allows us to say that for $t\rightarrow\infty$ we have the average energy
\begin{align*}
\average{E(\infty)} &= \average{E(-\infty)} - \frac{i\hbar{\omega}a(0)}{\sqrt{2\hbar{m}\omega}}\int_{-\infty}^{\infty}\mathbf{F}(t')e^{-i\omega{t'}}dt' + \frac{i\hbar{\omega}a^{\dagger}(0)}{\sqrt{2\hbar{m}\omega}}\int_{-\infty}^{\infty}\mathbf{F}(t')e^{i\omega{t'}}dt' + \left|\frac{i}{\sqrt{2m}}\int_{-\infty}^{\infty}\mathbf{F}(t')e^{i\omega{t'}}dt'\right|^2
\end{align*}

\item Next we can find the excitation probabilities of a stationary state for 
$t\rightarrow\infty$ using the fact that the system is in the ground state for 
$t\rightarrow-\infty$ which implies that
$$\hat{H}(-\infty)\ket{\psi} \left(a^{\dagger}(0)a(0)+\frac{1}{2}\right)\ket{\psi} = \frac{\hbar\omega}{2}\ket{\psi}$$
or that $\ket{\psi}=\ket{0}_{-\infty}$. Next we note that at positive infinity we are again in a 
stationary state as $\mathbf{F}=0$. Therefore we can generate the $n^{th}$ excited state by
acting the creation operator on $\ket{0}$ $n$-times, but due to the fact that we are 
operating in Heisenberg representation we need to use the creation operator at $t=\infty$
\begin{align*}
\ket{n}_{+\infty} &= \frac{\left(\hat{a}^{\dagger}(\infty)\right)^n}{\sqrt{n!}}\ket{0}_{+\infty}
\end{align*}
Note that the set of eigenkets $\ket{n}_{+\infty}$ forms an orthonormal basis, which allows 
us to write the initial state in this basis as
$$\ket{0}_{-\infty} = \sum_{n=0}^{\infty}c_n\ket{n}_{+\infty}$$
where the coefficient $|c_n|^2$ is the transition probability from $\ket{0}_{-\infty}$ to 
$\ket{n}_{+\infty}$. We can determine $c_n$ by noting that if we act the ladder operator
we get
$$\hat{a}(\infty)\ket{n}_{+\infty} = \sqrt{n}\ket{n-1}_{+\infty}$$
and we note that if we act $a(0)$ onto the expansion we get
$$0 = a(0)\sum_{n=0}^{\infty}c_n\ket{n}_{+\infty}$$
noting that $a(0)$ is also contained in $\hat{a}(\infty)$ as found in part (a)
$$\hat{a}(\infty) = e^{-i\omega{t}}\left(a(0) + \frac{i}{\sqrt{2\hbar{m}\omega}}\int_{-\infty}^{\infty}\mathbf{F}(t')e^{i\omega{t'}}dt'\right)$$
. Therefore it must follow that
$$c_n = \frac{c_{n-1}}{\sqrt{n}}\frac{i}{\sqrt{2\hbar{m}\omega}}\int_{-\infty}^{\infty}\mathbf{F}(t')e^{i\omega{t'}}dt'$$
Which implies that the $n^{th}$ state is related to the ground state by
$$c_n = c_0\frac{1}{\sqrt{n!}}\left(\frac{i}{\sqrt{2\hbar{m}\omega}}\int_{-\infty}^{\infty}\mathbf{F}(t')e^{i\omega{t'}}dt'\right)^n$$
therefore we can apply the normalization condition to yeild
\begin{align*}
\sum_{n=0}^{\infty}|c_n|^2 = 1 &= |c_0|^2\sum_{n=0}^{\infty}\frac{1}{n!}\left|\frac{i}{\sqrt{2\hbar{m}\omega}}\int_{-\infty}^{\infty}\mathbf{F}(t')e^{i\omega{t'}}dt'\right|^{2n}\\
&= |c_0|^2\exp\left[\left|\frac{i}{\sqrt{2\hbar{m}\omega}}\int_{-\infty}^{\infty}\mathbf{F}(t')e^{i\omega{t'}}dt'\right|^{2}\right]\\
&\Downarrow\\
|c_0|^2 &= \exp\left[-\frac{1}{2\hbar{m}\omega}\left|\int_{-\infty}^{\infty}\mathbf{F}(t')e^{i\omega{t'}}dt'\right|^{2}\right]
\end{align*}
So the transition probability follows
$$|c_n|^2 = \frac{1}{n!}\left(\frac{i}{\sqrt{2\hbar{m}\omega}}\int_{-\infty}^{\infty}\mathbf{F}(t')e^{i\omega{t'}}dt'\right)^{2n}\exp\left[-\frac{1}{2\hbar{m}\omega}\left|\int_{-\infty}^{\infty}\mathbf{F}(t')e^{i\omega{t'}}dt'\right|^{2}\right]$$
\end{enumerate}

\pagebreak

\section{Problem \#3}
\begin{enumerate}[(a)]
\item For a transformation from a stationary frame of reference to a frame of reference that
uniformly rotates with an angular velocity, $\pmb{\omega}$, we can find a unitary operator, 
$\mathscr{D}(\omega)$, by noting that for a constant angular velocity we move at a
time-dependent angle $\pmb{\phi}(t) = \pmb{\omega}{t}$. Then we treat this like a infinitesimal rotation 
through the angle $-\pmb{\phi}(t)$ which yields
$$\mathscr{D}(\pmb{\omega}) = \lim_{N\rightarrow\infty}\left[1+i\left(\frac{\mathbf{J}}{\hbar}\right)\cdot\left(\frac{\pmb{\omega}{t}}{N}\right)\right]^N = \exp\left(\frac{i\mathbf{J}\cdot\pmb{\omega}{t}}{\hbar}\right)$$
Note we rotate through a negative angle because we are rotating the frame of reference with 
can be though of rotating the vectors in the opposite direction. We also are projecting the angular momentum 
$\mathbf{J}$ to be along the rotation $\pmb{\omega}$. Without loss of generality we assume 
that both $\mathbf{J}$ and $\pmb{\omega}$ point along the $z$ direction.

\item So we can use transformation $\mathscr{D}$ we found in part (a) to transform the 
coordinate operator into the rotating frame. This follows like any unitary transformation 
of an operator where
\begin{align*}
\hat{x}(t) &= \mathscr{D}(\pmb{\omega})x\mathscr{D}^{\dagger}(\pmb{\omega})\\
&= \exp\left(\frac{iJ\omega{t}}{\hbar}\right)x\exp\left(-\frac{iJ\omega{t}}{\hbar}\right)\\
&= x + \frac{i\omega{t}}{\hbar}[J_z,x] - \frac{1}{2}\left(\frac{\omega{t}}{\hbar}\right)^2[J_z,[J_z,x]] + ...\\
&= x + \frac{i\omega{t}}{\hbar}(i\hbar{y}) + \frac{1}{2}\left(\frac{\omega{t}}{\hbar}\right)^2(\hbar^2x) + ...\\
&= x\left(1 + \frac{1}{2}(\omega{t})^2 + ...\right) - y\left(\omega{t} + \frac{1}{3!}(\omega{t})^3\right)\\
&= x\cos(\omega{t}) - y\sin(\omega{t})
\end{align*}
Note this is due to the commutation relation between coordinates and angular momentum. Using
the same commutation relation it follows that
\begin{align*}
\hat{y}(t) &= x\sin(\omega{t}) + y\cos(\omega{t})\\
\hat{z}(t) &= z
\end{align*}
Note the $\hat{z}(t)$ remains untransformed due to the fact that $[J_z,z]=0$.

\item We repeat this process for the momentum noting that 
\begin{align*}
[J_z,p_x] &= i\hbar p_y\\
[J_z,p_y] &= -i\hbar p_x\\
[J_z,p_z] &= 0
\end{align*}
Which yields the same transformation as in part (b), but with momentum. Therefore
\begin{align*}
\hat{p}_x(t) &= p_x\cos(\omega{t}) - p_y\sin(\omega{t})\\
\hat{p}_y(t) &= p_x\sin(\omega{t}) + p_y\cos(\omega{t})\\
\hat{p}_z(t) &= p_z
\end{align*}

\item To find the Hamiltonian of the particle in this frame we note that the unrotated 
Hamiltonian is in the general form 
$$\hat{H} = \frac{\hat{\mathbf{p}}^2}{2m} + U(\mathbf{r},t)$$
note that we require the potential to be dependent on time in order to account for the 
rotation. So we can apply the unitary transformation 
\begin{align*}
\hat{H}(t) &= \mathscr{D}\hat{H}\mathscr{D}^{\dagger}
\end{align*}
Which must still satisfy the Schr\''{o}dinger equation
$$i\hbar\frac{d\psi}{dt} = \hat{H}\psi$$
. So we can transform both sides to yield
\begin{align*}
\mathscr{D}i\hbar\frac{d\psi)}{dt} &= \mathscr{D}\hat{H}\psi\\
&\Downarrow\\
i\hbar\left(\frac{d(\mathscr{D}\psi)}{dt} - \psi\frac{d\mathscr{D}}{dt}\right) &= \hat{H}(\mathscr{D}\psi)\\
&\Downarrow\\
i\hbar\frac{d(\mathscr{D}\psi)}{dt} &=\left(\mathscr{D}\hat{H}\mathscr{D}^{\dagger} + i\hbar\frac{d\mathscr{D}}{dt}\mathscr{D}^{\dagger}\right)(\mathscr{D}\psi)\\
\end{align*}
So our new Hamiltonian operator is the term given in the parenthesis. Which for the rotation is
$$\hat{H}(t) = -\frac{\hat{\mathbf{p}}^2}{2m} + U'(\mathbf{r}) - \frac{\mathbf{J}\cdot\pmb{\omega}}{\hbar}$$


\end{enumerate}

\end{document}

