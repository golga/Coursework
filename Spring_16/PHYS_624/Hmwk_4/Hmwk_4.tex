\documentclass[11pt]{article}

\usepackage{latexsym}
\usepackage{graphicx}
\usepackage{amssymb}
\usepackage{amsthm}
\usepackage{enumerate}
\usepackage{amsmath}
\usepackage{cancel}
\numberwithin{equation}{section}

\setlength{\evensidemargin}{.25in}
\setlength{\oddsidemargin}{-.25in}
\setlength{\topmargin}{-.75in}
\setlength{\textwidth}{6.5in}
\setlength{\textheight}{9.5in}
\newcommand{\due}{March 9th, 2016}
\newcommand{\HWnum}{4}
\newcommand{\grad}{\bold\nabla}
\newcommand{\vecE}{\vec{E}}
\newcommand{\scrptR}{\vec{\mathfrak{R}}}
\newcommand{\kapa}{\frac{1}{4\pi\epsilon_0}}
\newcommand{\emf}{\mathcal{E}}
\newcommand{\unit}[1]{\ensuremath{\, \mathrm{#1}}}
\newcommand{\real}{\textnormal{Re}}
\newcommand{\Erf}{\textnormal{Erf}}
\newcommand{\sech}{\textnormal{sech}}
\newcommand{\scrO}{\mathcal{O}}
\newcommand{\levi}{\widetilde{\epsilon}}
\newcommand{\partiald}[2]{\ensuremath{\frac{\partial{#1}}{\partial{#2}}}}
\newcommand{\norm}[2]{\langle{#1}|{#2}\rangle}
\newcommand{\inprod}[2]{\langle{#1}|{#2}\rangle}
\newcommand{\average}[1]{\left\langle{#1}\right\rangle}
\newcommand{\ket}[1]{|{#1}\rangle}
\newcommand{\bra}[1]{\langle{#1}|}
\newcommand{\Resid}[2]{\ensuremath{\textnormal{Res}\left[{#1},{#2}\right]}}





\begin{document}
\begin{titlepage}
\setlength{\topmargin}{1.5in}
\begin{center}
\Huge{Physics 3310} \\
\LARGE{Principles of Electricity and Magnetism 1} \\
\Large{Professor Thomas R. Schibli} \\[1cm]

\huge{Homework \#\HWnum}\\[0.5cm]

\large{Joe Becker} \\
\large{SID: 810-07-1484} \\
\large{\due} 

\end{center}

\end{titlepage}



\section{Problem \#1}
\begin{enumerate}[(a)]
\item For a Dirac particle we can find the commutator $[\gamma_5,H]$ where
$$\gamma_5 = -\left(\begin{array}{cc}
                   0    &I\\
                   I    &0\\
               \end{array}\right)$$
where $I$ is the $2\times2$ identity matrix and the Hamiltonian is
$$H = c\pmb{\alpha}\cdot\mathbf{p} + mc^2\beta$$
where $\beta$ is the matrix
$$\beta = \left(\begin{array}{cc}
                   I    &0\\
                   0    &-I\\
               \end{array}\right)$$
and each component of $\pmb{\alpha}$ is a matrix given as
$$\alpha^{k} = \left(\begin{array}{cc}
                   0    &\sigma_k\\
                   \sigma_k    &0\\
               \end{array}\right)$$
So we note that we can represent the Hamiltonian as the matrix
$$H = c\left(\begin{array}{cc}
       mcI         &\sigma_kp_k\\
       \sigma_kp_k &-mcI
        \end{array}\right)$$
note we are summing over the index $k$ over the three dimensions. So
\begin{align*}
[\gamma_5,H] &= -c\left(\begin{array}{cc}
                   0    &I\\
                   I    &0\\
               \end{array}\right)
\left(\begin{array}{cc}
       mcI         &\sigma_kp_k\\
       \sigma_kp_k &-mcI
        \end{array}\right)
-c\left(\begin{array}{cc}
       mcI         &\sigma_kp_k\\
       \sigma_kp_k &-mcI
        \end{array}\right)
 \left(\begin{array}{cc}
                   0    &-I\\
                   -I    &0\\
               \end{array}\right)\\
&= -c\left(\begin{array}{cc}
         \sigma_kp_k      &-mcI\\
         mcI              &\sigma_kp_k
     \end{array}\right)
+ c\left(\begin{array}{cc}
         \sigma_kp_k      &mcI\\
         -mcI              &\sigma_kp_k
     \end{array}\right)\\
&= 2mc^2\left(\begin{array}{cc}
         0    &I\\
         -I   &0
     \end{array}\right)
\end{align*}

\item Next we calculate the commutator $[\mathbf{l}^2,H]$ where 
$\mathbf{l} = [\mathbf{r}\times\mathbf{p}]$. So we calculate
\begin{align*}
[\mathbf{l}^2,H] &= [\mathbf{l}^2,c\pmb{\alpha}\cdot\mathbf{p}] + \cancelto{0}{[\mathbf{l}^2,mc^2\beta]}\\
&= [l_il_i,c\alpha_kp_k]\\
&= c\alpha_k(l_i[l_i,p_k] + [l_i,p_k]l_i)\\
&= c\alpha_k(l_ii\hbar\epsilon_{ikl}p_{l} + i\hbar\epsilon_{ikl}p_ll_i)\\
&= i\hbar{c}\epsilon_{ikl}\alpha_k(l_ip_{l} + p_ll_i)
\end{align*}

\item And for the inversion operator $J\psi(\mathbf{r}) = \psi(-\mathbf{r})$ we calculate the
commutation relation
\begin{align*}
[J,H] &= [J,c\pmb{\alpha}\cdot\mathbf{p}] + \cancelto{0}{[J,mc^2\beta]}\\
&= c\pmb{\alpha}\cdot(J\mathbf{p}) - c\pmb{\alpha}\cdot(\mathbf{p}J) \\
&= c\pmb{\alpha}\cdot(-\mathbf{p}J) - c\pmb{\alpha}\cdot(\mathbf{p}J) \\
&= -2c(\pmb{\alpha}\cdot\mathbf{p})J
\end{align*}
\end{enumerate}
\pagebreak

\section{Problem \#2}
\begin{enumerate}[(a)]
\item We can find the non-relativistic limit for the expression for the charge density 
$$\rho = e\psi^*\psi$$
by taking the lower spinor, $\chi$, of the wave function
$$\psi = \left(\begin{array}{c}
        \varphi\\
        \chi
        \end{array}\right)$$
as
$$\chi = \frac{1}{2mc}\pmb{\sigma}\cdot\left(\hat{\mathbf{p}}-\frac{e}{c}\mathbf{A}\right)\varphi$$
Note this follows from the solution to the Hamiltonian for a Dirac particle 
$$i\hbar\partiald{\psi}{t} = (c\pmb{\alpha}\cdot\mathbf{p} + mc^2\beta)\psi$$
which yields the system of equations
\begin{align*}
(mc^2-\varepsilon)\varphi + c\pmb{\sigma}\cdot\mathbf{p}\chi &= 0\\
c\pmb{\sigma}\cdot\mathbf{p}\varphi - (mc^2+\varepsilon)\chi &= 0
\end{align*}
Which if we solve for $\chi$ and taking the non-relativistic limit $\varepsilon<<mc^2$ we 
find
$$\chi = \frac{c\pmb{\sigma}\cdot\mathbf{p}}{2mc^2-\varepsilon}\varphi\approx \frac{1}{2mc}\pmb{\sigma}\cdot\left({\mathbf{p}}-\frac{e}{c}\mathbf{A}\right)\varphi$$
note we take the canonical momentum $\mathbf{p} = \mathbf{p} - e/c\mathbf{A}$. So we
calculate $\rho$ as
\begin{align*}
\rho = e\psi^*\psi &= e\left(\begin{array}{cc}
                        \varphi^*,   &\frac{1}{2mc}\left(\pmb{\sigma}\cdot\left(\hat{\mathbf{p}}-\frac{e}{c}\mathbf{A}\right)\varphi\right)^*
                        \end{array}\right)
\left(\begin{array}{cc}
                        \varphi\\   \frac{1}{2mc}\pmb{\sigma}\cdot\left(\hat{\mathbf{p}}-\frac{e}{c}\mathbf{A}\right)\varphi
                        \end{array}\right)\\
&= e\varphi^*\varphi + \scrO\left(\frac{1}{c^2}\right)
\end{align*}

\item Using the same wave function we found in part (a) we can calculate the current density
\begin{align*}
\mathbf{j} &= ec\psi^*\pmb{\alpha}\psi\\
&= ec\psi^*\left(\begin{array}{cc}
              0             &\pmb{\sigma}\\ 
              \pmb{\sigma}  &0
            \end{array}\right)
            \left(\begin{array}{cc}
              \varphi\\ 
              \frac{1}{2mc}\pmb{\sigma}\cdot\left(\hat{\mathbf{p}}-\frac{e}{c}\mathbf{A}\right)\varphi
            \end{array}\right)\\
&= ec       \left(\begin{array}{cc}
              \varphi^*,   &\frac{1}{2mc}\left(\pmb{\sigma}\cdot\left(\hat{\mathbf{p}}-\frac{e}{c}\mathbf{A}\right)\varphi\right)^*
            \end{array}\right)
            \left(\begin{array}{cc}
              \frac{\pmb{\sigma}}{2mc}\pmb{\sigma}\cdot\left(\hat{\mathbf{p}}-\frac{e}{c}\mathbf{A}\right)\varphi\\
              \pmb{\sigma}\varphi\
            \end{array}\right)\\
&= \frac{ec}{2mc}\left[\varphi^*\pmb{\sigma}\left(\pmb{\sigma}\cdot\left(\hat{\mathbf{p}}-\frac{e}{c}\mathbf{A}\right)\right)\varphi - \left(\left(\hat{\mathbf{p}}+\frac{e}{c}\mathbf{A}\right)\cdot\varphi^*\pmb{\sigma}\right)\pmb{\sigma}\varphi\right]\\
&\Downarrow\\
j_i &= \frac{e}{2m}\left[\varphi^*\sigma_i\left(\sigma_k\left(\hat{p_k}-\frac{e}{c}A_k\right)\right)\varphi - \left(\left(\hat{p_k}+\frac{e}{c}{A_k}\right)\varphi^*\sigma_k\right){\sigma_i}\varphi\right]\\
&= \frac{e}{2m}\left[\varphi^*(\delta_{ik}+i\epsilon_{ikl}\sigma_l)\left(\hat{p_k}-\frac{e}{c}A_k\right)\varphi - \left(\hat{p_k}+\frac{e}{c}{A_k}\right)\varphi^*(\delta_{ik}+i\epsilon_{ikl}\sigma_l)\varphi\right]\\
&= \frac{e}{2m}\left[\varphi^*\hat{p}_i\varphi - \hat{p}_i\varphi^*\varphi - \frac{2e}{c}A_i\varphi^*\varphi + i\epsilon_{ikl}\left(\varphi^*\sigma_l\hat{p}_k\varphi + \hat{p}_k\varphi^*\sigma_l\varphi\right)\right]\\
&\Downarrow\\
\mathbf{j} &= -\frac{i\hbar{e}}{2m}\left[\varphi^*\pmb{\nabla}\varphi - (\pmb{\nabla}\varphi^*)\varphi\right] - \frac{e^2}{mc}\mathbf{A}\varphi^*\varphi + \frac{e\hbar}{2m}\pmb{\nabla}\times(\varphi^*\pmb{\sigma}\varphi)
\end{align*}
\end{enumerate}
\pagebreak

\section{Problem \#3}
Using the Born approximation and the stationary Klein-Gordon equation
$$\left\{-\frac{\hbar^2}{2m}\pmb{\nabla}+U(r)\right\}\psi = \frac{p_0^2}{2m}\psi$$
where $c^2p_0^2 = E^2-m^2c^4$ we can find the dependence of the scattering cross section,
$\sigma(E)$, on the energy, $E$, for a spinless relativistic particle in an external scalar
field $U(r)$ in the ultra-relativistic limiting case $(E\rightarrow\infty)$. In order to 
use the Born approximation we need to write the stationary Klein-Gordon equation in the
form of the Schr\"{o}dinger equation as 
$$\left\{-\frac{\hbar^2}{2m}\pmb{\nabla}+U(r)-\frac{U(r)^2}{2mc^2}+\frac{EU(r)}{mc^2}-\frac{E^2}{2mc^2}\right\}\psi = E\psi$$
Neglecting the constant term this yields an effective potential of the form
$$U_{eff}(r) = \frac{(E+mc^2)U(r)}{mc^2} - \frac{U(r)^2}{2mc^2}$$
using this potential we can find the scattering amplitude by the Born approximation
\begin{align*}
f_B &= -\frac{m}{2\pi\hbar^2}\int{U_{eff}}e^{-i\mathbf{q}\cdot\mathbf{r}}d^3r\\
B &= -\frac{1}{2\pi\hbar^2c^2}\int\left((E+mc^2)U(r) - \frac{1}{2}U(r)^2\right)e^{-i\mathbf{q}\cdot\mathbf{r}}d^3r
\end{align*}
In the ultra-relativistic case we note that the first term is dominant and we take $E\rightarrow{cp_0}$.
This results in the integral becoming a Fourier transformation of $U(r)$ by
\begin{align*}
f_B &= -\frac{p_0}{2\pi\hbar^2c}\int U(r)e^{-i\mathbf{q}\cdot\mathbf{r}}d^3r\\
&= -\frac{p_0}{2\pi\hbar^2c}U(q)
\end{align*}
Using this result we can take the integral over $d\Omega$ where 
$$d\Omega = \frac{\pi\hbar^2}{p_0^2}dq^2$$
which yields
\begin{align*}
\sigma &= \int|f_B|^2d\Omega\\
&= \frac{p_0^2}{4\pi^2\hbar^4c^2}\frac{\pi\hbar^2}{p_0^2}\int|U(q)|^2dq^2\\
&= \frac{1}{4\pi\hbar^2c^2}\int_{0}^{\infty}|U(q)|^2dq^2
\end{align*}
Note that as $E\rightarrow\infty$ we take $p\rightarrow\infty$ which in turn implies that 
$q\rightarrow\infty$. Therefore the integral in $\sigma$ goes to infinity. Note that if $U(r)$
goes to zero as $r\rightarrow\infty$ we have $\sigma$ as a constant.

\end{document}

