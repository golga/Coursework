\documentclass[11pt]{article}

\usepackage{latexsym}
\usepackage{graphicx}
\usepackage{amssymb}
\usepackage{amsthm}
\usepackage{enumerate}
\usepackage{amsmath}
\usepackage{cancel}
\numberwithin{equation}{section}

\setlength{\evensidemargin}{.25in}
\setlength{\oddsidemargin}{-.25in}
\setlength{\topmargin}{-.75in}
\setlength{\textwidth}{6.5in}
\setlength{\textheight}{9.5in}
\newcommand{\due}{February 24th, 2016}
\newcommand{\HWnum}{3}
\newcommand{\grad}{\bold\nabla}
\newcommand{\vecE}{\vec{E}}
\newcommand{\scrptR}{\vec{\mathfrak{R}}}
\newcommand{\kapa}{\frac{1}{4\pi\epsilon_0}}
\newcommand{\emf}{\mathcal{E}}
\newcommand{\unit}[1]{\ensuremath{\, \mathrm{#1}}}
\newcommand{\real}{\textnormal{Re}}
\newcommand{\Erf}{\textnormal{Erf}}
\newcommand{\sech}{\textnormal{sech}}
\newcommand{\scrO}{\mathcal{O}}
\newcommand{\levi}{\widetilde{\epsilon}}
\newcommand{\partiald}[2]{\ensuremath{\frac{\partial{#1}}{\partial{#2}}}}
\newcommand{\norm}[2]{\langle{#1}|{#2}\rangle}
\newcommand{\inprod}[2]{\langle{#1}|{#2}\rangle}
\newcommand{\ket}[1]{|{#1}\rangle}
\newcommand{\bra}[1]{\langle{#1}|}





\begin{document}
\begin{titlepage}
\setlength{\topmargin}{1.5in}
\begin{center}
\Huge{Physics 3320} \\
\LARGE{Principles of Electricity and Magnetism II} \\
\Large{Professor Ana Maria Rey} \\[1cm]

\huge{Homework \#\HWnum}\\[0.5cm]

\large{Joe Becker} \\
\large{SID: 810-07-1484} \\
\large{\due} 

\end{center}

\end{titlepage}



\section{Problem \#1}
For a given potential we can use the \emph{First Born Approximation} to approximate 
the scattering amplitude, $A_{ba}$, by the equation 
\begin{equation}
A_{ba}(\mathbf{q}) = -\frac{\mu}{2\pi\hbar^2}\int{e^{-i\mathbf{q}\cdot\mathbf{r}}}V(\mathbf{r})d^3r
\label{FirstBorn}
\end{equation}
where $\hbar\mathbf{q} =  \hbar(\mathbf{k}_a - \mathbf{k}_b)$. Note that $|A_{ba}|^2$ 
yields the differential cross section.

\begin{enumerate}[(a)]
\item For the given central potential 
$$V(r) = V_0\exp\left(-\frac{r}{R}\right)$$
we note that the potential only depends on the coordinate $r$ which allows us to simplify
equation \ref{FirstBorn} as
\begin{align*}
A_{ba}(\mathbf{q}) &= -\frac{\mu}{2\pi\hbar^2}\int{e^{i\mathbf{q}\cdot\mathbf{r}}}V(\mathbf{r})d^3r\\
&\Downarrow\\
A_{ba}(q) &=  -\frac{\mu}{2\pi\hbar^2}\int_{0}^{\infty}\int_{0}^{\pi}\int_{0}^{2\pi}{e^{-iqr\cos\theta}}V(r)\sin\theta r^2{d\theta}{d\phi}{dr}\\
&=  -\frac{\mu}{\hbar^2}\int_{0}^{\infty}\frac{e^{iqr}-e^{-iqr}}{iqr}V(r)r^2{dr}\\
&=  -\frac{2\mu}{q\hbar^2}\int_{0}^{\infty}\sin(qr)V(r)rdr
\end{align*}
So using the simplified form of equation \ref{FirstBorn} we have
\begin{align*}
A_{ba}(q) &= -\frac{2\mu{V_0}}{q\hbar^2}\int_{0}^{\infty}\sin(qr)e^{-r/R}rdr\\
&= -\frac{2\mu{V_0}}{q\hbar^2}\frac{2qR^3}{(1+(qR)^2)^2}\\
&= -\frac{4\mu{V_0}R^3}{\hbar^2(1+(qR)^2)^2}
\end{align*}
Using this result and noting that $q=2k\sin(\theta/2)$ for a spherically symmetric 
potential we can calculate the total cross section, $\sigma(E)$, as
\begin{align*}
\sigma(k) = \int|A_{ba}|^2d\Omega &= 2\pi\int_{0}^{\pi}|A_{ba}|^2\sin\theta{d\theta}\\
&= 2\pi\int_{0}^{\pi}\left(\frac{4\mu{V_0}R^3}{\hbar^2(1+(qR)^2)^2}\right)^2\sin\theta{d\theta}\\
&= \frac{32\pi\mu^2{V_0^2}R^6}{\hbar^4}\int_{0}^{\pi}\frac{\sin\theta}{(1+(2k\sin(\theta/2)R)^2)^4}{d\theta}\\
&= \frac{16\pi\mu^2{V_0^2}R^4}{3\hbar^4k^2}\left(1-\frac{1}{(1+4k^2R^2)^2}\right)\\
&\Downarrow\\
\sigma(E) &= \frac{8\pi\mu{V_0^2}R^4}{3\hbar^2E}\left(1-\frac{1}{(1+8\mu{E}R^2/\hbar^2)^2}\right)
\end{align*}
Note we changed to a dependence on energy by $k^2=2\mu{E}/\hbar^2$.

\item Now for the potential 
$$V(r) = V_0\exp\left(-\frac{r^2}{R^2}\right)$$
which is also a central potential we repeat the process from part (a) to find $A_{ba}$ as
\begin{align*}
A_{ba}(q) &= -\frac{2\mu{V_0}}{q\hbar^2}\int_{0}^{\infty}\sin(qr)e^{-r^2/R^2}rdr\\
&= -\frac{2\mu{V_0}}{q\hbar^2}\frac{\sqrt{\pi}qR^3e^{-q^2R^2/4}}{4}\\
&= -\frac{\sqrt{\pi}\mu{V_0}R^3}{2\hbar^2}e^{-q^2R^2/4}
\end{align*}
and $\sigma$ as
\begin{align*}
\sigma(k) &= 2\pi\int_{0}^{\pi}\left(\frac{\sqrt{\pi}\mu{V_0}R^3}{2\hbar^2}e^{-q^2R^2/4}\right)^2\sin\theta{d\theta}\\
&= \frac{\pi^2\mu^2V_0^2R^6}{2\hbar^4}\int_{0}^{\pi}e^{-2k^2\sin^2(\theta/2)R^2}\sin\theta{d\theta}\\
&= \frac{\pi^2\mu^2V_0^2R^4}{2\hbar^4k^2}\left(1-e^{-2k^2R^2}\right)\\
&\Downarrow\\
\sigma(E) &= \frac{\pi^2{\mu}V_0^2R^4}{4\hbar^2E}\left(1-e^{-4\mu{E}R^2/\hbar^2}\right)
\end{align*}
Again using the free particle energy $k^2=2\mu{E}/\hbar^2$ to get the total cross section
as a function of the incident energy, $E$.
\end{enumerate}

\pagebreak

\section{Problem \#2}
Given a particle with mass, $M$, and incident wave function $\exp(ikx)$ we can calculate the
wave function after scattering by a potential $V(x)$ is given by 
\begin{equation}
\psi(x) = \exp(ikx) + \frac{2M}{\hbar^2}\int_{-\infty}^{\infty}G(x,x')V(x')\psi(x')dx'
\label{Scatt}
\end{equation}
where the Green's function for the one dimensional Schr\"{o}dinger's equation is given as
$$G(x,x') = \left\{\begin{array}{lc}
                  (2ik)^{-1}\exp(ik(x-x'))   &x\ge{x'}\\
                  (2ik)^{-1}\exp(-ik(x-x'))   &x\le{x'}
            \end{array}\right.$$
Using $G(x,x')$ we can find the explicit form of $\psi(x)$ for an attractive potential 
$$V(x) = -\frac{\gamma\hbar^2}{2M}\delta{x}$$
where $\gamma$ is a positive constant. Therefore equation \ref{Scatt} becomes
\begin{align*}
\psi(x) &= \exp(ikx) + \frac{i\gamma}{2k}\int_{-\infty}^{\infty}e^{ik(x-x')}\delta(x')\psi(x')dx'\\
&= \exp(ikx) + 
\left\{\begin{array}{lc}
\dfrac{i\gamma}{2k}\psi(0)e^{ikx}  &x\ge{0}\\
\\
\dfrac{i\gamma}{2k}\psi(0)e^{-ikx}  &x\le{0}
\end{array}\right.
\end{align*}
Note for $x<0$ the exponential is negative is but so is $x$ this implies that the wave 
function is 
$$\psi(x) = e^{ikx} + \dfrac{i\gamma}{2k}\psi(0)e^{ik|x|}$$
This allows us to solve for $\psi(0)$ by
\begin{align*}
\psi(0) &= e^{ik(0)} + \dfrac{i\gamma}{2k}\psi(0)e^{ik(0)}\\
&= 1 + \frac{i\gamma}{2k}\psi(0)\\
&\Downarrow\\
1 &= \psi(0)\left(1-\frac{i\gamma}{2k}\right)\\
&\Downarrow\\
\psi(0) &= \frac{2k}{2k-i\gamma}
\end{align*}
So we have the wave-function 
$$\psi(x) = e^{ikx} + \frac{i\gamma}{2k-i\gamma}e^{ik|x|}$$
which tells us that in the region left of the potential ($x<0$) we have a scattered 
wave-function moving to the left with an amplitude 
$$|R|^2 = \frac{\gamma^2}{4k^2+\gamma^2}$$
and to the right of the potential ($x>0$) we have a scattered potential moving to the right
with the amplitude 
$$|T|^2 = \frac{4k^2}{4k^2+\gamma^2}$$


\pagebreak

\section{Problem \#3}
We can use the \emph{First Born Approximation}, given by equation \ref{FirstBorn}, to
express the scattering amplitude by $N$ identical scattering centers located along a 
straight line where $b$ is the distance between any two neighboring centers. Note we 
are given the scattering amplitude, $f_0(q)$, of a single scattering center $V_0(r)$.
The total potential, $V(r)$ is given by the sum 
$$V(r) = \sum_{n=0}^{N}V_0(r+nb)$$
where there we apply a periodic condition that
$$V(r) = V(r+b)$$
We note that the scattering potentials only depend on $r$ so we can calculate the total 
scattering amplitude, $f_N(q)$, as
\begin{align*}
f_N(q) &= -\frac{\mu}{2\pi\hbar^2}\int{e^{-i\mathbf{q}\cdot\mathbf{r}}}V(r)d^3r\\
&= -\frac{\mu}{2\pi\hbar^2}\int{e^{-i\mathbf{q}\cdot\mathbf{r}}}\sum_{n=0}^{N}V(r+nb)d^3r\\
&= \sum_{n=0}^{N}-\frac{\mu}{2\pi\hbar^2}\int{e^{-i(r-nb)\mathbf{q}\cdot\hat{r}}}V(r)d^3r\\
&= \left(-\frac{\mu}{2\pi\hbar^2}\int{e^{-i\mathbf{q}\cdot\mathbf{r}}}V(r)d^3r\right)\sum_{n=0}^Ne^{inb\mathbf{q}\cdot{\hat{r}}}\\
&= f_0(q)\sum_{n=0}^Ne^{inbq\cos\theta}\\
&= f_0(q)\left(\frac{1-e^{iNbq\cos\theta}}{1-e^{ibq\cos\theta}}\right)
\end{align*}
where we take $\theta$ as the scattering angle.


\end{document}

