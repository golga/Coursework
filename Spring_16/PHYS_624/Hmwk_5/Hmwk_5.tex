\documentclass[11pt]{article}

\usepackage{latexsym}
\usepackage{graphicx}
\usepackage{amssymb}
\usepackage{amsthm}
\usepackage{enumerate}
\usepackage{amsmath}
\usepackage{cancel}
\numberwithin{equation}{section}

\setlength{\evensidemargin}{.25in}
\setlength{\oddsidemargin}{-.25in}
\setlength{\topmargin}{-.75in}
\setlength{\textwidth}{6.5in}
\setlength{\textheight}{9.5in}
\newcommand{\due}{April 13th, 2016}
\newcommand{\HWnum}{5}
\newcommand{\grad}{\bold\nabla}
\newcommand{\vecE}{\vec{E}}
\newcommand{\scrptR}{\vec{\mathfrak{R}}}
\newcommand{\kapa}{\frac{1}{4\pi\epsilon_0}}
\newcommand{\emf}{\mathcal{E}}
\newcommand{\unit}[1]{\ensuremath{\, \mathrm{#1}}}
\newcommand{\real}{\textnormal{Re}}
\newcommand{\Erf}{\textnormal{Erf}}
\newcommand{\sech}{\textnormal{sech}}
\newcommand{\scrO}{\mathcal{O}}
\newcommand{\levi}{\widetilde{\epsilon}}
\newcommand{\partiald}[2]{\ensuremath{\frac{\partial{#1}}{\partial{#2}}}}
\newcommand{\norm}[2]{\langle{#1}|{#2}\rangle}
\newcommand{\inprod}[2]{\langle{#1}|{#2}\rangle}
\newcommand{\average}[1]{\left\langle{#1}\right\rangle}
\newcommand{\ket}[1]{|{#1}\rangle}
\newcommand{\bra}[1]{\langle{#1}|}
\newcommand{\Resid}[2]{\ensuremath{\textnormal{Res}\left[{#1},{#2}\right]}}





\begin{document}
\begin{titlepage}
\setlength{\topmargin}{1.5in}
\begin{center}
\Huge{Physics 3310} \\
\LARGE{Principles of Electricity and Magnetism 1} \\
\Large{Professor Thomas R. Schibli} \\[1cm]

\huge{Homework \#\HWnum}\\[0.5cm]

\large{Joe Becker} \\
\large{SID: 810-07-1484} \\
\large{\due} 

\end{center}

\end{titlepage}



\section{Problem \#1}
\begin{enumerate}[(i)]
\item For a system of three identical bosons residing in different quantum states with 
quantum numbers $f_1$, $f_2$, and $f_3$ we can use the single particle wave functions 
$\psi_{fi}(\xi)$ which are normalized to unity to find the wave function of the system. We 
note that the requirement for bosons is that the wave function must be symmetric under 
exchange of particles. Which simple allows that all permutations of particles in states are
positive. This implies that
\begin{align*}
\psi &= \frac{1}{\sqrt{6}}\left(\frac{}{}
\psi_{f1}(1)\psi_{f2}(2)\psi_{f3}(3)
+ \psi_{f1}(3)\psi_{f2}(1)\psi_{f3}(2)\right.
+ \psi_{f1}(2)\psi_{f2}(3)\psi_{f3}(1)\\
&\qquad \left.\frac{}{}+ \psi_{f1}(1)\psi_{f2}(3)\psi_{f3}(2)
+ \psi_{f1}(3)\psi_{f2}(2)\psi_{f3}(1)
+ \psi_{f1}(2)\psi_{f2}(1)\psi_{f3}(3)
\right)
\end{align*}
Note the factor of $1/\sqrt{6}$ is due to normalization of the 6 different possible arrangements
of particles as each are equally likely.

\item
The result from part (i) can be applied to the case for fermions except that we must enforce
the requirement that under exchange of particles the wave function is antisymmetric. This 
implies that for even permutations of the particles the wave function remains positive and 
for odd permutations the wave function becomes negative. This yields the result
\begin{align*}
\psi &= \frac{1}{\sqrt{6}}\left(\frac{}{}
\psi_{f1}(1)\psi_{f2}(2)\psi_{f3}(3)
+ \psi_{f1}(3)\psi_{f2}(1)\psi_{f3}(2)\right.
+ \psi_{f1}(2)\psi_{f2}(3)\psi_{f3}(1)\\
&\qquad \left.\frac{}{}- \psi_{f1}(1)\psi_{f2}(3)\psi_{f3}(2)
- \psi_{f1}(3)\psi_{f2}(2)\psi_{f3}(1)
- \psi_{f1}(2)\psi_{f2}(1)\psi_{f3}(3)
\right)
\end{align*}
Note the normalization remains the same.
\end{enumerate}

\section{Problem \#2}
\begin{enumerate}[(i)]
\item To find thee electron momentum probability distribution for the $2s$ state of a Hydrogen
atom we first take the wave function of that state in coordinate representation as
$$\psi_{2s}(r) = \frac{1}{4\sqrt{2\pi{a^3}}} \left(2-\frac{r}{a}\right)e^{-r/2a}$$
where a is the \emph{Bohr radius} given as $a = \hbar^2/me^2$. Now we can calculate the 
wave function in momentum space by
\begin{align*}
\phi_{2s}(p) &= \frac{1}{(2\pi\hbar)^{3/2}}\int\exp\left(i\frac{\mathbf{p}\cdot\mathbf{r}}{\hbar}\right)\psi_{2s}(r)d^3r\\
&= \frac{1}{(4\pi)^2(a\hbar)^{3/2}}\int_{0}^{\infty}\int_{0}^{\pi}\int_{0}^{2\pi}\exp\left(i\frac{pr\cos\theta}{\hbar}\right)\left(2-\frac{r}{a}\right)e^{-r/2a}r^2\sin\theta{dr}{d\theta}{d\phi}\\
&= \frac{\hbar}{4\pi(a\hbar)^{3/2}p}\int_{0}^{\infty}\sin\left(\frac{pr}{\hbar}\right)\left(2-\frac{r}{a}\right)e^{-r/2a}r{dr}\\
&= \frac{\hbar}{4\pi(a\hbar)^{3/2}p}64a^3\hbar^3p\frac{(4a^2p^2-\hbar^2)}{(4a^2p^2+\hbar^2)^3}\\
&= \frac{16\hbar^{5/2}a^{3/2}}{\pi}\frac{(4a^2p^2-\hbar^2)}{(4a^2p^2+\hbar^2)^3}
\end{align*}
So we can find the square of the wave function as the electron momentum probability distribution
for the $2s$ state as
$$|\phi_{2s}(p)|^2 = \frac{256\hbar^{5}a^{3}}{\pi^2}\frac{(4a^2p^2-\hbar^2)^2}{(4a^2p^2+\hbar^2)^6}$$

\item Now for the $2p$ states we take the wave function to be in a superposition of the three
possible $l$ states $(l=-1,0,1)$
$$\psi_{2p}(r,\theta,\phi) = \frac{1}{\sqrt{3}}\left(\frac{}{}\psi_{211}(r,\theta,\phi)+\psi_{210}(r,\theta,\phi)+\psi_{21-1}(r,\theta,\phi)\right)$$
where we take each state as equally likely. Note we have the eigenfunctions as
\begin{align*}
\psi_{210}(r,\theta) &= \frac{1}{4\sqrt{2\pi{a^3}}}\frac{r}{a}e^{-r/2a}\cos\theta\\
\psi_{21\pm1}(r,\theta,\phi) &= \frac{1}{8\sqrt{2\pi{a^3}}}\frac{r}{a}e^{-r/2a}\sin\theta e^{\pm{i\phi}}
\end{align*}
Now we calculate the momentum representation of each of the eigenstates by
\begin{align*}
\phi_{210}(p) &= \frac{1}{(2\pi\hbar)^{3/2}}\int\exp\left(i\frac{\mathbf{p}\cdot\mathbf{r}}{\hbar}\right)\psi_{210}(r,\theta)d^3r\\
&= \frac{1}{(4\pi)^2(a\hbar)^{3/2}}\int_{0}^{\infty}\int_{0}^{\pi}\int_{0}^{2\pi}\exp\left(i\frac{pr\cos\theta}{\hbar}\right)\frac{r}{a}e^{-r/2a}\cos\theta{r^2}\sin\theta{dr}{d\theta}{d\phi}\\
&= \frac{64ia^{5/2}\hbar^{7/2}}{\pi}\frac{p}{(4a^2p^2+\hbar^2)^3}
\end{align*}
and
\begin{align*}
\phi_{21\pm1}(p) &= \frac{1}{(2\pi\hbar)^{3/2}}\int\exp\left(i\frac{\mathbf{p}\cdot\mathbf{r}}{\hbar}\right)\psi_{21\pm1}(r,\theta,\phi)d^3r\\
&= \frac{1}{(4\pi)^2(a\hbar)^{3/2}}\int_{0}^{\infty}\int_{0}^{\pi}\int_{0}^{2\pi}\exp\left(i\frac{pr\cos\theta}{\hbar}\right)\frac{r}{a}e^{-r/2a}\sin{\theta}e^{\pm{i\phi}}r^2\sin\theta{dr}{d\theta}{d\phi}\\
&= \pm\frac{3ia^{3/2}\hbar^{7/2}}{\pi}\frac{1}{(\hbar^2+4a^2p^2)^{5/2}}
\end{align*}
Which allows us to calculate the probability distribution for the $2p$ state as
\begin{align*}
|\phi_{2p}(p)|^2 = \frac{1}{3}\left(\frac{}{}|\phi_{210}|^2 +|\phi_{211}|^2 +|\phi_{21-1}|^2\right)   
&= \frac{64^2a^{5}\hbar^{7}}{3\pi^2}\frac{p^2}{(4a^2p^2+\hbar^2)^6} + \frac{6a^{3}\hbar^{7}}{\pi^2}\frac{1}{(\hbar^2+4a^2p^2)^{5}}\\
&= \frac{2a^3\hbar^7}{3\pi^2}\left(\frac{2048a^2p^2}{(4a^2p^2+\hbar^2)^6} + \frac{9}{(\hbar^2+4a^2p^2)^{5}}\right)\\
&= \frac{2a^3\hbar^7}{3\pi^2}\frac{2084a^2p^2+9\hbar^2}{(4a^2p^2+\hbar^2)^6}
\end{align*}
\end{enumerate}

\pagebreak

\section{Problem \#3}
Consider an artificial helium-like two-electron atom where the Coulomb electron-nucleus 
interaction potential is replaced by a harmonic potential. This has a Hamiltonian of the 
form
$$H = -\frac{1}{2}\grad_1^2-\frac{1}{2}\grad_2^2+\frac{1}{8}\left(\frac{}{}r_1^2+r_2^2\right) + \frac{1}{|r_1-r_2|}$$
where we are working in atomic units and we take the spring constant to be $k=1/4$ which has
an exact ground state solution as
$$\psi(\mathbf{r}_1,\mathbf{r}_2) = \frac{1}{2(8\pi^{5/2}+5\pi^3)^{1/2}}\left(1+\frac{1}{2}|\mathbf{r}_1-\mathbf{r}_2|\right)\exp\left[-\frac{1}{4}(r_1^2+r_2^2)\right]$$
where the energy of this state is given as $E=2\unit{a.u.}$. Using this we can calculate the
electron-electron correlation energy by
$$E_c = E - E_0$$
where we can calculate $E_0$ by 
$$E_0 = \frac{\bra{\psi_0}H\ket{\psi_0}}{\inprod{\psi_0}{\psi_0}}$$
Taking $\psi_0$ as the ground-state wave function of the truncated Hamiltonian
$$H_0 = -\frac{1}{2}\grad_1^2-\frac{1}{2}\grad_2^2+\frac{1}{8}\left(\frac{}{}r_1^2+r_2^2\right)$$
We can see that 
$$\psi_0 = \exp\left[-\frac{1}{4}(r_1^2+r_2^2)\right]$$
note that we neglect the normalization constant as they will not factor into the solution for 
$E_0$. Also we can neglect the integration over the angular component as they too will cancel
in the calculation of $E_0$. So we find that
\begin{align*}
\inprod{\psi_0}{\psi_0} &= \int_{0}^{\infty}\int_{0}^{\infty}\exp\left[-\frac{1}{4}(r_1^2+r_2^2)\right]^2r_1^2r_2^2dr_1dr_2\\
&= \int_{0}^{\infty}\int_{0}^{\infty}\exp\left[-\frac{1}{2}(r_1^2+r_2^2)\right]r_1^2r_2^2dr_1dr_2\\
&= \sqrt{\frac{\pi}{2}}\int_{0}^{\infty}\exp\left[-\frac{1}{2}r_2^2\right]r_2^2dr_2\\
&= \frac{\pi}{2}
\end{align*}
Now we see that 
\begin{align*}
\bra{\psi_0}H\ket{\psi_0} &= -\frac{1}{2}\int\psi_0\grad^2_1\psi_0dr_{12}
-\frac{1}{2}\int\psi_0\grad^2_2\psi_0dr_{12}
+ \frac{1}{8}\int\psi_0\left(\frac{}{}r_1^2+r_2^2\right)\psi_0dr_{12}
+ \int\psi_0\frac{1}{|\mathbf{r_1}-\mathbf{r_2}|}\psi_0dr_{12}\\
\end{align*}
Has four integration terms so we calculate each as
\begin{align*}
-\frac{1}{2}\int\psi_0\grad^2_1\psi_0dr_1dr_2 &= -\frac{1}{2}\int_{0}^{\infty}\int_{0}^{\infty}\exp\left[-\frac{1}{4}(r_1^2+r_2^2)\right]\frac{1}{r_1^2}\partiald{}{r_1}\left(r_1^2\partiald{}{r_1}\exp\left[-\frac{1}{4}(r_1^2+r_2^2)\right]\right)r_1^2r_2^2dr_1dr_2\\
&= -\frac{1}{8}\int_{0}^{\infty}\int_{0}^{\infty}\exp\left[-\frac{1}{2}(r_1^2+r_2^2)\right](r_1^2-6)r_1^2r_2^2dr_1dr_2\\
&= -\frac{1}{8}\left(\frac{3\pi}{2} - 6\frac{\pi}{2}\right) = \frac{3\pi}{16}
\end{align*}
Note due to the symmetry of the $\psi_0$ the integral with the $r_2$ Laplacian has the identical 
result.  Next we calculate 
\begin{align*}
\frac{1}{8}\int\psi_0\left(\frac{}{}r_1^2+r_2^2\right)\psi_0dr_{12} &= \frac{1}{8}\int_{0}^{\infty}\int_{0}^{\infty}\exp\left[-\frac{1}{2}(r_1^2+r_2^2)\right]\left(\frac{}{}r_1^2+r_2^2\right)r_1^2r_2^2dr_1dr_2\\
&= \frac{1}{8}\sqrt{\frac{\pi}{2}}\int_{0}^{\infty}\exp\left[-\frac{1}{2}r_2^2\right]\left(\frac{}{}3+r_2^2\right)r_2^2dr_2\\
&= \frac{3\pi}{8}
\end{align*}
Finally for the interaction potential we expand into spherical harmonics where
$$\frac{1}{|\mathbf{r_1}-\mathbf{r_2}|} = \left\{\begin{array}{cc}
\dfrac{4\pi}{r_1}\sum_{l,m}\dfrac{1}{2l+1}\left(\dfrac{r_2}{r_1}\right)^{l}Y^*_{lm}(\theta_1,\varphi_1)Y_{lm}(\theta_2,\varphi_2),   &r_1>r_2\\
\\
\dfrac{4\pi}{r_2}\sum_{l,m}\dfrac{1}{2l+1}\left(\dfrac{r_1}{r_2}\right)^{l}Y^*_{lm}(\theta_1,\varphi_1)Y_{lm}(\theta_2,\varphi_2),   &r_2>r_1\\
\end{array}\right.$$
We note due to the orthogonality of the spherical harmonics and the fact that $\psi_0$ is in
the $m=l=0$ state we can simplify the integral to just 
$$\frac{1}{|\mathbf{r_1}-\mathbf{r_2}|} = \left\{\begin{array}{cc}
\dfrac{1}{r_1}, &r_1>r_2\\
\\
\dfrac{1}{r_2}, &r_2>r_1\\
\end{array}\right.$$
So we can calculate the interaction term as
\begin{align*}
\int\psi_0\frac{1}{|\mathbf{r_1}-\mathbf{r_2}|}\psi_0dr_{12}
&= \int_{0}^{\infty}\exp\left[-\frac{1}{2}r_1^2\right]\left(\int_{0}^{r_1}\frac{1}{r_1}\exp\left[-\frac{1}{2}r_2^2\right]r_2^2dr_2 + \int_{r_1}^{\infty}\frac{1}{r_2}\exp\left[-\frac{1}{2}r_2^2\right]r_2^2dr_2\right)r_1^2dr_1\\
&= \frac{\sqrt{\pi}}{2}
\end{align*}
So this allows us to calculate $E_0$ by
$$E_0 = \frac{\bra{\psi_0}H\ket{\psi_0}}{\inprod{\psi_0}{\psi_0}} = \frac{2}{\pi}\left(\frac{3\pi}{16} + \frac{3\pi}{16} + \frac{3\pi}{8} + \frac{\sqrt{\pi}}{2}\right) = \frac{3}{2}+\frac{1}{\sqrt{\pi}}\approx2.0642\unit{a.u.}$$
So we can see that the correlation energy is about
$$E_c\approx-0.0642\unit{a.u.}$$



\end{document}

