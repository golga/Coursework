\documentclass[11pt]{article}

\usepackage{latexsym}
\usepackage{graphicx}
\usepackage{amssymb}
\usepackage{amsthm}
\usepackage{enumerate}
\usepackage{amsmath}
\usepackage{cancel}
\numberwithin{equation}{section}

\setlength{\evensidemargin}{.25in}
\setlength{\oddsidemargin}{-.25in}
\setlength{\topmargin}{-.75in}
\setlength{\textwidth}{6.5in}
\setlength{\textheight}{9.5in}
\newcommand{\due}{March 31st, 2016}
\newcommand{\HWnum}{8}
\newcommand{\grad}{\bold\nabla}
\newcommand{\vecE}{\vec{E}}
\newcommand{\scrptR}{\vec{\mathfrak{R}}}
\newcommand{\kapa}{\frac{1}{4\pi\epsilon_0}}
\newcommand{\emf}{\mathcal{E}}
\newcommand{\unit}[1]{\ensuremath{\, \mathrm{#1}}}
\newcommand{\real}{\textnormal{Re}}
\newcommand{\Erf}{\textnormal{Erf}}
\newcommand{\sech}{\textnormal{sech}}
\newcommand{\scrO}{\mathcal{O}}
\newcommand{\levi}{\widetilde{\epsilon}}
\newcommand{\partiald}[2]{\ensuremath{\frac{\partial{#1}}{\partial{#2}}}}
\newcommand{\norm}[2]{\langle{#1}|{#2}\rangle}
\newcommand{\inprod}[2]{\langle{#1}|{#2}\rangle}
\newcommand{\average}[1]{\left\langle{#1}\right\rangle}
\newcommand{\ket}[1]{|{#1}\rangle}
\newcommand{\bra}[1]{\langle{#1}|}
\newcommand{\Resid}[2]{\ensuremath{\textnormal{Res}\left[{#1},{#2}\right]}}





\begin{document}
\begin{titlepage}
\setlength{\topmargin}{1.5in}
\begin{center}
\Huge{Physics 3310} \\
\LARGE{Principles of Electricity and Magnetism 1} \\
\Large{Professor Thomas R. Schibli} \\[1cm]

\huge{Homework \#\HWnum}\\[0.5cm]

\large{Joe Becker} \\
\large{SID: 810-07-1484} \\
\large{\due} 

\end{center}

\end{titlepage}



\section{Problem \#1}
\begin{enumerate}[(1)]
\item To find the Debye screening length, $\lambda_D$, of a semiconductor with a dielectric constant
$\varepsilon=20$ and density of electrons $n_0=10^{18}\unit{cm^{-3}}$ at room temperature, 
$T=300\unit{K}$ we can use the reciprocal length, $\kappa$, given as
$$\kappa^2 = \frac{4\pi{e^2}}{T}\sum_{a}n_{a0}z_a^2$$
Noting that for a semiconductor the ions are fixed so we only consider the electrons in the 
sum. This implies that
$$\lambda_D = \frac{1}{\kappa} = \sqrt{\frac{\varepsilon\varepsilon_0k_BT}{4\pi n_0e^2}}$$
note that we added the constants $k_B$ and $\varepsilon_0$ to get the correct dimensionality.
So for the semiconductor we calculate
\begin{align*}
\lambda_D = \sqrt{\frac{\varepsilon\varepsilon_0k_BT}{4\pi n_0e^2}} &= \sqrt{\frac{(20)(8.85\times10^{-12}\unit{C^2\ N^{-1}\ m^{-2}})(1.38\times10^{-23}\unit{J\ K^{-1}})(300\unit{K})}{4\pi(10^{24}\unit{m^{-3}})(1.6\times10^{-19}\unit{C})^2}}\\
&= 1.51\unit{nm}
\end{align*}

\item For a semiconductor in the form of a rectangular slab with thickness $L$, where $L$ is 
much smaller than the other two dimensions, we attach two electrodes to the faces of the slab
and apply a bias voltage, $V$. To solve for the electric field we solve the \emph{Poisson 
Equation} given by
\begin{equation}
\grad^2\phi-\kappa^2\phi = 0
\end{equation}
where $\kappa$ is defined above. Taking the directions that create the face as infinite we 
assume that $\phi$ is only dependent on $z$ which yields a solution of the form
$$\phi(z) = C_1e^{\kappa{z}} + C_2e^{-\kappa{z}}$$
applying the boundary conditions 
$$\phi(z=0)=0\qquad\phi(z=L)=V$$
we find that 
\begin{align*}
\phi(z=0) = 0 &= C_1e^{\kappa{0}} + C_2e^{-\kappa{0}}\\
&\Downarrow\\
C_1 = -C_2\\
&\Downarrow\\
\phi(z) = 0 &= C\left(e^{\kappa{z}} - e^{-\kappa{z}}\right) = C\sinh(\kappa{z})
\end{align*}
and
\begin{align*}
\phi(z=L) = V &=  C\sinh(\kappa{L})\\
&\Downarrow\\
C &= \frac{V}{\sinh(\kappa{L})}
\end{align*}
So the potential as a function of $z$ is
$$\phi(z) = \frac{V}{\sinh(\kappa{L})}\sinh(\kappa{z})$$

\item Note if we take the limit as the thickness becomes infinite we have the following 
condition $$\kappa{L}\rightarrow\infty$$ we see that in this limit we have
$$\lim_{\kappa{L}\rightarrow\infty}\frac{V}{\sinh(\kappa{L})} = 0$$
So we see that as we take the thickness to be infinite the screening will completely cancel
the bias voltage.

\end{enumerate}

\pagebreak

\section{Problem \#2}
\begin{enumerate}[(a)]
\item Given the physical characterization of the triple point of water found experimentally
as
\begin{align*} 
T_t = 273.16\unit{K}\qquad P_t=612\unit{Pa}\qquad \rho_{L}=1\unit{g\ cm^{-3}}\qquad \rho_{S} = 0.894\unit{g\ cm^{-3}}
\end{align*} 
where $\rho_{L}$ is the density of water in the liquid state and $\rho_{S}$ is the density of
water in the solid state. Noting the data for the freezing point of water at atmospheric 
pressure is 
$$T = 273.15\unit{K}\qquad P_t=101\unit{kPa}\qquad \rho_{L}=1\unit{g\ cm^{-3}}\qquad \rho_{S} = 0.894\unit{g\ cm^{-3}}$$
we can find the heat of ice melting assuming that it does not change between triple point and 
the freezing point at normal pressure by taking the \emph{Clapeyron-Clausius formula}
\begin{equation}
\frac{dP}{dT} = \frac{q}{T(v_2-v_1)}
\label{CCform}
\end{equation}
and solving for the heat, $q$. Note we can approximate the derivative by taking the two 
different data points from the freezing point to the triple point. So equation \ref{CCform}
becomes
\begin{align*}
q &= \frac{dP}{dT}{T(v_2-v_1)}\\
&= \frac{P_t-P}{T_t-T}T\left(\frac{1}{\rho_L}-\frac{1}{\rho_S}\right)\\
&= \frac{612\unit{Pa}-101000\unit{Pa}}{273.16\unit{K}-273.15\unit{K}}(273.15\unit{K})\left(\frac{1}{1\times10^{6}\unit{g\ m^{-3}}}-\frac{1}{0.894\times10^{6}\unit{g\ m^{-3}}}\right)\\
&= 325\unit{J\ g^{-1}}
\end{align*}

\item Using equation \ref{CCform} and the given value for the latent heat of vaporization 
$$q = 2264.76\unit{J\ g^{-1}}$$
we can calculate the derivative of boiling temperature
\begin{align*}
\frac{dT_b}{dP} &= \frac{T(v_2-v_1)}{q}\\
&= \frac{T(\rho_{V}^{-1}-\rho_{L}^{-1})}{q}
\end{align*}
and the same follows for ice vaporization
\begin{align*}
\frac{dT_v}{dP} &= \frac{T(v_2-v_1)}{q}\\
&= \frac{T(\rho_{V}^{-1}-\rho_{S}^{-1})}{q}
\end{align*}
where we take $\rho_{V}$ as the density of water vapor.

\item When we take water at the freezing temperature, $T=273.15\unit{K}$, and atmospheric 
pressure, $P=103\unit{kPa}$, and decrease the pressure until the pressure reaches 
$P'=300\unit{Pa}$ We see that this patch will pass through the triple point described in 
part (a). This implies that the water will skip the liquid phase and sublimate from solid to 
vapor.


\end{enumerate}

\end{document}

