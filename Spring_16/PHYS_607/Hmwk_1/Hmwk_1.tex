\documentclass[11pt]{article}

\usepackage{latexsym}
\usepackage{graphicx}
\usepackage{amssymb}
\usepackage{amsthm}
\usepackage{enumerate}
\usepackage{amsmath}
\usepackage{cancel}
\numberwithin{equation}{section}

\setlength{\evensidemargin}{.25in}
\setlength{\oddsidemargin}{-.25in}
\setlength{\topmargin}{-.75in}
\setlength{\textwidth}{6.5in}
\setlength{\textheight}{9.5in}
\newcommand{\due}{January 21st, 2016}
\newcommand{\HWnum}{1}
\newcommand{\grad}{\bold\nabla}
\newcommand{\vecE}{\vec{E}}
\newcommand{\scrptR}{\vec{\mathfrak{R}}}
\newcommand{\kapa}{\frac{1}{4\pi\epsilon_0}}
\newcommand{\emf}{\mathcal{E}}
\newcommand{\unit}[1]{\ensuremath{\, \mathrm{#1}}}
\newcommand{\real}{\textnormal{Re}}
\newcommand{\Erf}{\textnormal{Erf}}
\newcommand{\sech}{\textnormal{sech}}
\newcommand{\scrO}{\mathcal{O}}
\newcommand{\levi}{\widetilde{\epsilon}}
\newcommand{\partiald}[2]{\ensuremath{\frac{\partial{#1}}{\partial{#2}}}}
\newcommand{\norm}[2]{\langle{#1}|{#2}\rangle}
\newcommand{\inprod}[2]{\langle{#1}|{#2}\rangle}
\newcommand{\ket}[1]{|{#1}\rangle}
\newcommand{\bra}[1]{\langle{#1}|}





\begin{document}
\begin{titlepage}
\setlength{\topmargin}{1.5in}
\begin{center}
\Huge{Physics 3320} \\
\LARGE{Principles of Electricity and Magnetism II} \\
\Large{Professor Ana Maria Rey} \\[1cm]

\huge{Homework \#\HWnum}\\[0.5cm]

\large{Joe Becker} \\
\large{SID: 810-07-1484} \\
\large{\due} 

\end{center}

\end{titlepage}



\section{Problem \#1}
\begin{enumerate}[(a)]
\item Given a gas with a large number, $N$, of non-interacting particles in a box with a 
volume, $V$. We can find the probability that we will find fixed number, $N_1<N$, of particles
in a sub-volume, $V_1<V$, by noting that this probability follows a binomial distribution 
where the probability of finding a single particle in the volume, $V_1$, is given by 
$$p = \frac{V_1}{V}$$
therefore if we expand to $N_1$ particles found we have $N-N_1$ particles not in $V_1$. So it
follows that the total probability of finding $N_1$ particles within $V_1$ is
\begin{equation}
P(N_1) = {{N}\choose{N_1}}\left(1-\frac{V_1}{V}\right)^{N-N_1}\left(\frac{V_1}{V}\right)^{N_1}
\label{Bionom}
\end{equation}
where we note the combinatoric is defined as
$${{N}\choose{N_1}} \equiv \frac{N!}{(N-N_1)!N_1!}.$$
We can find the value $N_1$ that maximizes equation \ref{Bionom} by noting that the $N_1$ 
that gives the mean value which is found by
$$\overline{N_1} = \sum_{N_1=0}^{N}N_1P(N_1)$$
has the maximum probability. For any binomial distribution this value is given by the total 
number of trials multiplied by the probability. So equation \ref{Bionom} is maximized for the
value
$$\overline{N_1} = N\frac{V_1}{V}$$
which makes intuitive sense as the assumption that the particle has equal probability of being
found in any location implies that there is a uniform distribution of particles. We note that
the average square fluctuations, $\overline{(\Delta{N_1})^2}$, is given as
$$\overline{(\Delta{N_1})^2} = N\frac{V_1}{V}\left(1-\frac{V_1}{V}\right)$$
which follows from the standard variance of a binomial distribution. Now if we take the 
number of particles to be large, $N=10^{20}$, we note that we can approximate the 
binomial distribution as a Gaussian distribution where
$$P(N_1) \approx \frac{1}{\sqrt{2\pi{N}p(1-p)}}\exp\left(\frac{(N_1-\overline{N_1})^2}{2\overline{(\Delta{N_1})^2}}\right).$$
Now if we take the size of the sub-volume to be $V_1 = V/2$ we see that $p=1/2$ and 
$\overline{N_1} = N/2$ and $\overline{(\Delta{N_1})^2} = N/4$. So we have the probability 
distribution
$$P(N_1) \approx \sqrt{\frac{2}{\pi{N}}}\exp\left(\frac{(N_1-N/2)^2}{N/2}\right).$$
So using this distribution we can calculate the probability that the number, $N_1$ deviates 
from $N/2$ by $(10^{-4})\%$ by solving the integral
\begin{align*}
\int_{N/2(1-10^{-6})}^{N/2(1+10^{-6})}P(N_1)dN_1
\end{align*}
But we note that the variance goes by the $\sqrt{N/4}$ which is equal to $1/2\times10^{10}$ 
which implies that the distribution does not vary out side of $10^{-8}\%$ with any real 
probability. Therefore we approximate the variance as zero.

\item Now, if we extend the problem to the case where the total volume is divided into an
arbitrary number, $n$, boxes with volumes $V_1,V_2,...,V_n$. We can find the probability and
assuming that the probabilities are independent we can calculate the probability that there
are $N_1$ particles in $V_1$, $N_2$ in $V_2$ and so forth by calculating the product
$$P(N_1,N_2,...,N_n) = \prod_{k=1}^{n}{{N}\choose{N_k}}\left(1-\frac{V_k}{V}\right)^{N-N_k}\left(\frac{V_k}{V}\right)^{N_i}$$

\item The probability found in part (b) is at a maximum when each probability is at it's mean
given by
$$\overline{N_k} = N\frac{V_k}{V}$$
\end{enumerate}

\pagebreak

\section{Problem \#2}
\begin{enumerate}[(a)]
\item For a large number, $N$, of non-interacting identical quantum particles with a mass, 
$m$, confined in a cubic box with the side of length, $L$, we can estimate the density of 
states for a fixed  total energy, $E$. We first assume that $N>>1$ and $kL>>1$ where
$$k = \sqrt{2m\epsilon}/\hbar\qquad\textnormal{with}\qquad\epsilon=E/N.$$
We first consider a single particle in the box with energy $\epsilon$ this implies that
for a wave function of the form
$$\psi_{\mathbf{p}} = e^{i\mathbf{p}\cdot\mathbf{r}}$$
with the periodic boundary condition that $r_i = r_i+L$ for $i=x,y,z$. Which implies that
$$\frac{p_iL}{\hbar} = 2\pi{n_i}$$
which quasiclassically yields the number of states for three degrees of freedom as
$$\frac{dpdq}{(2\pi\hbar)^3}$$
where $dq$ becomes defined by the allowed position space given as $L^3$ and $dp$ is defined
by the allowed energy as the energy is fully kinetic. So 
$$dp = \frac{d\epsilon}{\sqrt{8\pi\epsilon}}$$ 
So the density of states for $N$ particles with $\epsilon=E/N$ we have
$$D(N) = \left(\frac{L}{2\pi\hbar}\right)^{3N}\sqrt{\frac{N}{8\pi{E}}}$$



\end{enumerate}

\end{document}

