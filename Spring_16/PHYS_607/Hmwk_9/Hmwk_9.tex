\documentclass[11pt]{article}

\usepackage{latexsym}
\usepackage{graphicx}
\usepackage{amssymb}
\usepackage{amsthm}
\usepackage{enumerate}
\usepackage{amsmath}
\usepackage{cancel}
\numberwithin{equation}{section}

\setlength{\evensidemargin}{.25in}
\setlength{\oddsidemargin}{-.25in}
\setlength{\topmargin}{-.75in}
\setlength{\textwidth}{6.5in}
\setlength{\textheight}{9.5in}
\newcommand{\due}{April 7th, 2016}
\newcommand{\HWnum}{9}
\newcommand{\grad}{\bold\nabla}
\newcommand{\vecE}{\vec{E}}
\newcommand{\scrptR}{\vec{\mathfrak{R}}}
\newcommand{\kapa}{\frac{1}{4\pi\epsilon_0}}
\newcommand{\emf}{\mathcal{E}}
\newcommand{\unit}[1]{\ensuremath{\, \mathrm{#1}}}
\newcommand{\real}{\textnormal{Re}}
\newcommand{\Erf}{\textnormal{Erf}}
\newcommand{\sech}{\textnormal{sech}}
\newcommand{\scrO}{\mathcal{O}}
\newcommand{\levi}{\widetilde{\epsilon}}
\newcommand{\partiald}[2]{\ensuremath{\frac{\partial{#1}}{\partial{#2}}}}
\newcommand{\norm}[2]{\langle{#1}|{#2}\rangle}
\newcommand{\inprod}[2]{\langle{#1}|{#2}\rangle}
\newcommand{\average}[1]{\left\langle{#1}\right\rangle}
\newcommand{\ket}[1]{|{#1}\rangle}
\newcommand{\bra}[1]{\langle{#1}|}
\newcommand{\Resid}[2]{\ensuremath{\textnormal{Res}\left[{#1},{#2}\right]}}





\begin{document}
\begin{titlepage}
\setlength{\topmargin}{1.5in}
\begin{center}
\Huge{Physics 3310} \\
\LARGE{Principles of Electricity and Magnetism 1} \\
\Large{Professor Thomas R. Schibli} \\[1cm]

\huge{Homework \#\HWnum}\\[0.5cm]

\large{Joe Becker} \\
\large{SID: 810-07-1484} \\
\large{\due} 

\end{center}

\end{titlepage}



\section{Problem \#1}
\begin{enumerate}[(1)]
\item Given the universal equation of state
\begin{equation}
\left(P'+\frac{3}{V'^{2}}\right)\left(3V'-1\right)=8T'
\label{Univer}
\end{equation}
we can find the spinodal curves in the $(T,V)$, $(P,V)$, and $(P,T)$ planes where the
condition is given as
$$\left(\partiald{^2F}{V^2}\right)_{T} = 0 \Rightarrow \left(\partiald{P}{V}\right)_{T} = 0$$
This allows us to solve equation \ref{Univer} and take its derivative with respect to $V$
to find the spinodal curve in the $(T,V)$ plane as
\begin{align*}
P &= \frac{8T}{3V-1}-\frac{3}{V^2}\\
&\Downarrow\\
\left(\partiald{P}{V}\right)_{T} = 0 &= -\frac{24T}{(3V-1)^2} + \frac{6}{V^3}\\
&\Downarrow\\
T &= \frac{(3V-1)^2}{4V^3}\\
\end{align*}
Note we assumed that $T'=P'=V'=1$. Using this we can substitute $T$ into equation
\ref{Univer} to find the $(P,V)$ spinodal curve as
\begin{align*}
\left(P+\frac{3}{V^{2}}\right)\left(3V-1\right) &= 8T\\
&\Downarrow\\
\left(P+\frac{3}{V^{2}}\right)\left(3V-1\right) &= \frac{2(3V-1)^2}{V^3}\\
&\Downarrow\\
P &= \frac{2(3V-1)}{V^3} - \frac{3}{V^2}
\end{align*}
Note we can solve for $V$ in terms of $T$ as 
$$V = \frac{1}{9}\left(2\sqrt{T^2+3T}+2T+3\right)$$
and substitute into the equation for $P$ in terms of $V$ to find the $(P,T)$ spinodal curve.

\item We can find the binodal in the vicinity of the critical point $P-1$, $T-1$, $V-1$. 
Where we define the variables
$$p\equiv{P-1}\qquad t\equiv{T-1}\qquad v\equiv{V-1}$$
where we assume that $p,v,t<<1$. Then we write equation \ref{Univer} in terms of these new
variables and expand about $v$
\begin{align*}
p+1 &= \frac{8(t+1)}{3(v+1)-1}-\frac{3}{(v+1)^2}\\
&= 4(t+1)\left(\frac{3}{2}v+1\right)^{-1} - 3(v+1)^{-2}\\
&\approx 4(t+1)\left(1-\frac{3}{2}v+\frac{9}{4}v^2-\frac{27}{8}v^3\right) - 3\left(1-2v+3v^2-4v^3\right)\\
&\approx 4t + 1 - 6vt + 9tv^2 - \frac{27}{2}tv^3 - \frac{3}{2}v^3\\
&\Downarrow\\
P &= 10(T-1) - 6(T-1)V - \frac{3}{2}(V-1)^3
\end{align*}
Using this approximation we can solve for the spinodal curve by
\begin{align*}
\left(\partiald{P}{V}\right)_{T} = 0 &= -6(T-1) -\frac{9}{2}(V-1)^2\\
&\Downarrow\\
(V-1) &= \pm\sqrt{\frac{4}{3}(1-T)}
\end{align*}
These define the difference between the gas and liquid phases of the binodal curve.

\item Given the free energy of a Van der Waals gas
\begin{equation}
F = F_{id} - NT\ln\left(1-\frac{Nb}{V}\right)-\frac{N^2a}{V}
\label{FreeEn}
\end{equation}
we can find the entropy as
$$S = -\partiald{F}{T} =S_{id} + N\ln\left(1-\frac{Nb}{V}\right)$$
which implies that the change in entropy from the positive to the negative solutions from 
part (2) as
\begin{align*}
\Delta{S} &= N\ln\left(1-\frac{Nb}{V_l}\right) - N\ln\left(1-\frac{Nb}{V_g}\right)\\
&= N\ln\left(\frac{V_l}{V_g}\frac{V_g-Nb}{V_l-Nb}\right)\\
&= N\ln\left(\frac{V_l}{V_g}\right)+N\ln\left(\frac{\sqrt{4/3(1-T/T_c)}+1-Nb}{-\sqrt{4/3(1-T/T_c)}+1-Nb}\right)
\end{align*}
So we can find the latent heat by
$$q = T\Delta{S} = NT\ln\left(\frac{V_l}{V_g}\right)+NT\ln\left(\frac{\sqrt{4/3(1-T/T_c)}+1-Nb}{-\sqrt{4/3(1-T/T_c)}+1-Nb}\right)$$
\end{enumerate}

\pagebreak

\section{Problem \#2}
\begin{enumerate}[(a)]
\item When we add a small amount of salt to boiling water we know that the chemical potential
for both phases must be equal. Assuming that salt is not volatile we can say that the water
vapor is pure so 
$$\mu_{g}^{0}(P_0,T_0) = \mu_{l}^{0}(P_0,T_0) - Tc$$
where $c$ is the concentration of the salt. Now if we take the pressure to be constant we 
expand the chemical potential about a $\Delta{T}$ which yields
$$\left(\partiald{\mu_{g}^{0}}{T}-\partiald{\mu_{l}^{0}}{T}\right)\Delta{T} = -Tc$$
Now we note that the entropy is given as
$$\partiald{\mu_{\nu}}{T} = s_{\nu}$$
and the heat of phase transition is
$$q = T(s_l-s_g)$$
which implies that
\begin{align*}
-\frac{q}{T}\Delta{T} &= -Tc\\
&\Downarrow\\
\Delta{T} &= \frac{T^2c}{q}
\end{align*}
So we see that $c$ and $T$ are positive values and $q$ for boiling water is also positive 
which implies that the temperature of the transition rises.

\item Empirically we know that the concentration of salt in ice is less than in liquid water.
Where the chemical potential of the solvent is given as
$$\mu = T\ln(c) + \psi(P,T)$$
so we have an equilibrium condition
$$T\ln(c_s) + \psi_s(P,T) = T\ln(c_l) + \psi_l(P,T)$$
so we see that $c_{s}<c_{l}$ so for this equality to hold we see that the
$\psi_l(P,T)<\psi_s(P,T)$
this implies that for the same concentration we have the condition
$$\mu_l(P,T,c)<\mu_s(P,T,c)$$

\item We see that due to \emph{Le Chatelier's principle} the system wants to counteract the
increase in concentration from a pure water state. This implies that more water molecules 
go into the liquid water phase when there is salt in the solution as compared to the pure 
water state.
\end{enumerate}

\pagebreak

\section{Problem \#3}
For $0.1\unit{g}$ of NaCl that is thrown into $200\unit{g}$ of water we have to account for
the strong electrolyte solution by adding the energy of interaction
$$E_{int} = -\sqrt{\frac{\pi}{T}}\left(n_{Na+}e^2+n_{Cl-}e^2\right)^{3/2} = -\sqrt{\frac{\pi}{T}}e^3\left(2n\right)^{3/2}$$
Note that we that the concentration of both ions to be equal to $n = 1.03\times10^{21}5$. So for water at 
$300\unit{K}$ we need to take the energy of interaction from the water which results in a 
change in temperature given by
$$\Delta{T} = \frac{e^3}{k_B}\sqrt{\frac{\pi}{T}}\left(2n\right)^{3/2} = 2.84\times10^{-3}\unit{K}$$




\end{document}

