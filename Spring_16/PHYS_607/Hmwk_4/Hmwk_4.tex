\documentclass[11pt]{article}

\usepackage{latexsym}
\usepackage{graphicx}
\usepackage{amssymb}
\usepackage{amsthm}
\usepackage{enumerate}
\usepackage{amsmath}
\usepackage{cancel}
\numberwithin{equation}{section}

\setlength{\evensidemargin}{.25in}
\setlength{\oddsidemargin}{-.25in}
\setlength{\topmargin}{-.75in}
\setlength{\textwidth}{6.5in}
\setlength{\textheight}{9.5in}
\newcommand{\due}{February 17th, 2016}
\newcommand{\HWnum}{4}
\newcommand{\grad}{\bold\nabla}
\newcommand{\vecE}{\vec{E}}
\newcommand{\scrptR}{\vec{\mathfrak{R}}}
\newcommand{\kapa}{\frac{1}{4\pi\epsilon_0}}
\newcommand{\emf}{\mathcal{E}}
\newcommand{\unit}[1]{\ensuremath{\, \mathrm{#1}}}
\newcommand{\real}{\textnormal{Re}}
\newcommand{\Erf}{\textnormal{Erf}}
\newcommand{\sech}{\textnormal{sech}}
\newcommand{\scrO}{\mathcal{O}}
\newcommand{\levi}{\widetilde{\epsilon}}
\newcommand{\partiald}[2]{\ensuremath{\frac{\partial{#1}}{\partial{#2}}}}
\newcommand{\norm}[2]{\langle{#1}|{#2}\rangle}
\newcommand{\inprod}[2]{\langle{#1}|{#2}\rangle}
\newcommand{\average}[1]{\left\langle{#1}\right\rangle}
\newcommand{\ket}[1]{|{#1}\rangle}
\newcommand{\bra}[1]{\langle{#1}|}
\newcommand{\Resid}[2]{\ensuremath{\textnormal{Res}\left[{#1},{#2}\right]}}





\begin{document}
\begin{titlepage}
\setlength{\topmargin}{1.5in}
\begin{center}
\Huge{Physics 3310} \\
\LARGE{Principles of Electricity and Magnetism 1} \\
\Large{Professor Thomas R. Schibli} \\[1cm]

\huge{Homework \#\HWnum}\\[0.5cm]

\large{Joe Becker} \\
\large{SID: 810-07-1484} \\
\large{\due} 

\end{center}

\end{titlepage}



\section{Problem \#1}
\begin{enumerate}[(1)]
\item Given the fact that energy per particle depends only on the intensive variables
$$s = \frac{S}{N},\qquad\textnormal{and}v=\frac{V}{N}$$
we can show that the Gibbs thermodynamic potential per particle given as
$$d\Phi = VdP - SdT + \mu{dN}$$
where $\mu$ is the chemical potential we note that 
$$\mu = \left(\partiald{\Phi}{N}\right)_{P,T}.$$
Now if we write $d\Phi$ in terms of the intensive variables we have
$$d\Phi = NvdP - NsdT + \mu{dN}$$
which implies that we can write $\Phi$ as
$$\Phi(T,P) = N\varphi(T,P)$$
where $T$ and $P$ are intensive variables not dependent on $N$. Then it is easy to see that
\begin{align*}
\mu = \left(\partiald{\Phi}{N}\right)_{P,T} &= \varphi(T,P)\\
&\Downarrow\\
\mu &= \varphi(T,P)
\end{align*}
But by definition we see that
$$\mu = \varphi(T,P) = \frac{\Phi}{N}$$
so the chemical potential is just given by the ratio of $\Phi$ with $N$.

\item We can introduce the density of particles given by
$$n=\frac{N}{V}$$
as an intensive value that is only a function of intensive variables $P$ and $T$ note that
by rearranging to $N = nV$ we see that 
$$\partiald{N}{V} = n.$$ 
Also using the result from part (a) which states that
$$\Phi = N\mu\qquad\Rightarrow\qquad d\Phi = \mu{dN} + Nd\mu$$
which if we combine with our definition of $d\Phi$ yields
\begin{align*}
d\Phi = \cancel{\mu{dN}} + Nd\mu &=  VdP - SdT + \cancel{\mu{dN}}\\
&\Downarrow\\
Nd\mu &=  VdP - SdT 
\end{align*}
It follows that 
$$\left(\partiald{\mu}{P}\right)_{T} = \frac{V}{N} = \frac{1}{n}$$
using the above two identities we see that
\begin{align*}
\left(\partiald{N}{\mu}\right)_{T,V} &= \partiald{N,V}{\mu,V}\\
&= \partiald{N,V}{N,P}\partiald{N,P}{\mu,V}\\
&= \left(\partiald{V}{P}\right)_{T,N}\left(\cancelto{0}{\partiald{N}{\mu}\partiald{P}{V}} - \cancelto{n}{\partiald{N}{V}}\cancelto{n}{\partiald{P}{\mu}}\right)\\
&= -n^2\left(\partiald{V}{P}\right)_{T,N}
\end{align*}
Note we explicitly kept $T$ constant throughout the derivation.
\end{enumerate}

\pagebreak

\section{Problem \#2}
\begin{enumerate}[(1)]
\item For a given gas of Helium atoms expanding adiabatically to a volume ten times its 
initial volume we and immediately can state that the change in entropy, 
$$\Delta{S}=0$$
because this is adiabatic expansion. This fact also implies that the change of energy in the
system is due to the work of expansion. We note for a mono-atomic gas we have a change in
internal energy that is given by
$$dE = \frac{3}{2}NdT$$
where if we use the equation of state for an ideal gas we have
$$d(PV) = NdT$$
where we assume the number of particles is a constant. Using this fact and that the total 
change in energy is given by the work we have
\begin{align*}
dE = \frac{3}{2}d(PV) = \frac{3}{2}(PdV+VdP) &= -PdV\\
&\Downarrow\\
-\left(\frac{3}{2}+1\right)PdV &= \frac{3}{2}VdP\\
&\Downarrow\\
-\gamma\frac{dV}{V} &= \frac{dP}{P}\\
&\Downarrow\\
-\gamma\int_{V_0}^{V}\frac{dV}{V} &= \int_{P_0}^{P}\frac{dP}{P}\\
&\Downarrow\\
P_0V_0^{\gamma} &= PV^{\gamma}
\end{align*}
Where we defined $\gamma$ by the degrees of freedom, $d$, given by
$$\gamma\equiv\frac{d/2 + 1}{d/2}$$
where we take $d=3$ for a mono-atomic gas which yields $\gamma=5/3$. So for $V=10V_0$ we have
$$P = P_0\left(\frac{V_0}{10V_0}\right)^{5/3} = \frac{P_0}{10^{5/3}}$$
This implies that the change in pressure is given by
$$\Delta{P} = P_0\left(10^{-5/3}-1\right)$$
Using this change in pressure and the equation of state for an ideal gas we can find the 
change in temperature by 
\begin{align*}
\Delta{T} = T - T_0 &=  \frac{PV}{N} - \frac{P_0V_0}{N}\\
&=  \frac{P_0}{10^{5/3}}\frac{10V_0}{N} - \frac{P_0V_0}{N} \\
&= \frac{P_0V_0}{N}\left(10^{-2/3} - 1\right)
\end{align*}

\item Repeating this process of a gas of $H_2$ molecules we note that everything remains the
same except for the degrees of freedom change to $d=5$ to account for the two new rotational
degrees of freedom. This implies that $\gamma = 7/5$ so it follows that
\begin{align*}
\Delta{S} &= 0\\
\Delta{P} &= P_0\left(10^{-7/5}-1\right)\\
\Delta{T} &= \frac{P_0V_0}{N}\left(10^{-2/5} - 1\right)
\end{align*}

\item We can calculate the adiabatic coefficient of thermal expansion, $\alpha_S$, for an
ideal gas by noting that $T$ and $V$ are related to a constant, $C$, given by
$$TV^{\gamma-1} = C$$
when we hold $S$ constant which represents an adiabatic process. This allows us to calculate
$\alpha_S$ by
\begin{align*}
\alpha_S = \frac{1}{V}\left(\partiald{V}{T}\right)_{S} &= \frac{1}{V}\partiald{}{T}\left(\frac{C}{T}\right)^{1/\gamma-1}\\
&= \left(\frac{T}{C}\right)^{1/\gamma-1}C^{1/\gamma-1}\left(\frac{-1}{\gamma-1}\right)T^{-\gamma/\gamma-1}\\
&= -\frac{1}{\gamma-1}T^{\frac{-\gamma+1}{\gamma-1}}\\
&= -\frac{1}{\gamma-1}\frac{1}{T}
\end{align*}
We note for the minimal degrees of freedom $d=3$ we have $\gamma>1$ which implies that 
$\alpha_S$ is negative. Therefore an ideal gas undergoing adiabatic expansion decrease in 
temperature at a rate inversely proportional to the temperature.
\end{enumerate}

\pagebreak

\section{Problem \#3}
\begin{enumerate}[(1)]
\item Given an ideal gas with $N$ molecules in a chamber at temperature, $T_1$ under the 
pressure $P_1$ that is penetrating through a porous tube into another chamber with 
controllable pressure $P_2<P_1$ we can find the change in temperature, entropy, and volume 
for this process. Note we assume that $C_P$ is not dependent on temperature. Noting that
the gas is undergoing a \emph{Joule-Thomson} process which in general is governed by the 
equations
\begin{equation}
\left(\partiald{T}{P}\right)_{W} = \frac{1}{C_P}\left[T\left(\partiald{V}{T}\right)_{P}-V\right]
\label{JTEqn1}
\end{equation}
where we apply the equation of state for an ideal gas which states
$$V = \frac{NT}{P}\qquad\Rightarrow\qquad \partiald{V}{T} = \frac{N}{P}$$
Therefore equation \ref{JTEqn1} becomes
\begin{align*}
\left(\partiald{T}{P}\right)_{W} &= \frac{1}{C_P}\left[T\left(\partiald{V}{T}\right)_{P}-V\right]\\
&\Downarrow\\
\left(\partiald{T}{P}\right)_{W} &= \frac{1}{C_P}\left[T\frac{N}{P}-\frac{NT}{P}\right] = 0
\end{align*}
So for any change in pressure we have no change in temperature so 
$$\Delta{T}=0$$
This implies that by the equation of state the change in volume is directly related to the 
change in pressure as $T$ and $N$ are constant. So we see
\begin{align*}
\Delta{V} = V_2-V_1 &= \frac{NT}{P_2} - \frac{NT}{P_1}\\
&= NT\left(\frac{1}{P_2} - \frac{1}{P_1}\right)
\end{align*}
Now, if use the relation
$$dW = TdS+VdP\qquad\Rightarrow\qquad dS = \frac{dW}{T} - \frac{V}{T}dP$$
we can write the change in entropy as
\begin{align*}
\left(\partiald{S}{P}\right)_{W} = -\frac{V}{T} &= -\frac{N}{P}\\
&\Downarrow\\
\int_{S_1}^{S_2}dS &= -N\int_{P_1}^{P_2}\frac{dP}{P}\\
\Delta{S} &= N\ln\left(\frac{P_1}{P_2}\right)
\end{align*}
Note that $P_1>P_2$ so as expected $\Delta{S}>0$. 

\item Now, if we take the gas as slightly non-ideal with an energy given as
$$E = NC_VT + \frac{gN^2}{2V}$$
where $g$ is the strength of interaction between particles. A positive $g$ corresponds to 
repulsion, negative to attraction. We assume that the interaction is weak or $|g|n<<T$. So 
we take the equation of state to be a first order correction given as
$$V = \frac{NT}{P}+\frac{gN^2}{2}.$$
Using this new equation of state we can apply equation \ref{JTEqn1} to find the change in 
temperature
\begin{align*}
\left(\partiald{T}{P}\right)_{W} &= \frac{1}{C_P}\left[T\left(\partiald{V}{T}\right)_{P}-V\right]\\
&\Downarrow\\
\left(\partiald{T}{P}\right)_{W} &= \frac{1}{C_P}\left[T\frac{N}{P}-\frac{NT}{P} + \frac{gN^2}{2}\right] \\
&= \frac{1}{C_P}\frac{gN^2}{2}
\end{align*}
Note the interaction term allows for the change in temperature to be non-zero given by an
integration of a constant. This yields
$$\Delta{T} = \frac{P_2-P_1}{C_P}\frac{gN^2}{2}$$
And we find the change in entropy by
\begin{align*}
\left(\partiald{S}{P}\right)_{W} = -\frac{V}{T} &= -\frac{N}{P} - \frac{gN^2}{2T}\\
&\Downarrow\\
\int_{S_1}^{S_2}dS &= -N\int_{P_1}^{P_2}\frac{dP}{P} - \frac{gN^2}{2T}(P_2-P_1)\\
\Delta{S} &= N\ln\left(\frac{P_1}{P_2}\right) - \frac{gN^2}{2T}(P_2-P_1)
\end{align*}
Note we assume that $|g|n<<T$ so the term we are correcting lowers the change in entropy from
the ideal case but the correction is small so that the total change in entropy remains 
positive.
\end{enumerate}

\pagebreak

\section{Problem \#4}
\begin{enumerate}[(1)]
\item We can find the chemical potential of an ideal gas as a function of variable pairs. 
First we start with the partition function for an ideal gas
$$Z = \frac{(CVT^{3/2})^N}{N!}$$
where we take the log assuming $N$ is large with yields
$$\ln(Z) = N\left(\zeta+1-\ln\left(\frac{V}{N}\right)+\frac{3}{2}\ln(T)\right)$$
So we can calculate $F$ as
$$F = -T\ln(Z) = -NT\left(\zeta+1-\ln\left(\frac{V}{N}\right)+\frac{3}{2}\ln(T)\right)$$
which yields the entropy by
$$S = -\partiald{F}{T} = N\left(\zeta+1+\ln\left(\frac{V}{N}\right) + \frac{3}{2}(1+\ln(T))\right)$$
So we can calculate the chemical potential by
\begin{align*}
\mu = \frac{\Phi}{N} &= \frac{PV+F}{N} = P\frac{V}{N} - T\left(\zeta+1-\ln\left(\frac{V}{N}\right)+\frac{3}{2}\ln(T)\right)\\
&\Downarrow\\
\mu(V,T) &= -T\left(\zeta+\ln(V/N)+\frac{3}{2}\ln(T)\right)
\end{align*}
Using the relation defined by the entropy we have
\begin{align*}
\mu(P,T) &= -T\left(\zeta+\ln(T/P)+\frac{3}{2}\ln(T)\right)\\
\mu(S,T) &= -T\left(\frac{S}{N}-\frac{5}{2}\right)\\
\mu(P,S) &= -\left(\frac{S}{N}-\frac{5}{2}\right)P^{2/5}\exp\left(\frac{2}{5}\left(\frac{S}{N}-\zeta\right)-1\right)
\end{align*}

\item We note that for an ideal gas we have the potential, $\Omega$, given as
$$\Omega = -PV$$
Note the result from part (1) for the chemical potential 
\begin{align*}
\mu(P,T) &= -T\left(\zeta+\ln(T/P)+\frac{3}{2}\ln(T)\right)\\
&\Downarrow\\
P(\mu,T) &= \exp\left(\frac{\mu}{T}+\frac{5}{2}\ln(T)+\zeta\right)
\end{align*}
Replacing we have
$$\Omega = -V\exp\left(\frac{\mu}{T}+\frac{5}{2}\ln(T)+\zeta\right)$$



\end{enumerate}

\end{document}

