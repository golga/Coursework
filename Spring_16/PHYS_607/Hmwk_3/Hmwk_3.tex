\documentclass[11pt]{article}

\usepackage{latexsym}
\usepackage{graphicx}
\usepackage{amssymb}
\usepackage{amsthm}
\usepackage{enumerate}
\usepackage{amsmath}
\usepackage{cancel}
\numberwithin{equation}{section}

\setlength{\evensidemargin}{.25in}
\setlength{\oddsidemargin}{-.25in}
\setlength{\topmargin}{-.75in}
\setlength{\textwidth}{6.5in}
\setlength{\textheight}{9.5in}
\newcommand{\due}{February 12th, 2016}
\newcommand{\HWnum}{3}
\newcommand{\grad}{\bold\nabla}
\newcommand{\vecE}{\vec{E}}
\newcommand{\scrptR}{\vec{\mathfrak{R}}}
\newcommand{\kapa}{\frac{1}{4\pi\epsilon_0}}
\newcommand{\emf}{\mathcal{E}}
\newcommand{\unit}[1]{\ensuremath{\, \mathrm{#1}}}
\newcommand{\real}{\textnormal{Re}}
\newcommand{\Erf}{\textnormal{Erf}}
\newcommand{\sech}{\textnormal{sech}}
\newcommand{\scrO}{\mathcal{O}}
\newcommand{\levi}{\widetilde{\epsilon}}
\newcommand{\partiald}[2]{\ensuremath{\frac{\partial{#1}}{\partial{#2}}}}
\newcommand{\norm}[2]{\langle{#1}|{#2}\rangle}
\newcommand{\inprod}[2]{\langle{#1}|{#2}\rangle}
\newcommand{\average}[1]{\left\langle{#1}\right\rangle}
\newcommand{\ket}[1]{|{#1}\rangle}
\newcommand{\bra}[1]{\langle{#1}|}
\newcommand{\Resid}[2]{\ensuremath{\textnormal{Res}\left[{#1},{#2}\right]}}





\begin{document}
\begin{titlepage}
\setlength{\topmargin}{1.5in}
\begin{center}
\Huge{Physics 3310} \\
\LARGE{Principles of Electricity and Magnetism 1} \\
\Large{Professor Thomas R. Schibli} \\[1cm]

\huge{Homework \#\HWnum}\\[0.5cm]

\large{Joe Becker} \\
\large{SID: 810-07-1484} \\
\large{\due} 

\end{center}

\end{titlepage}



\section{Problem \#1}
\begin{enumerate}[(1)]
\item Given the mass density of the medium, $\rho$, given by the expression
$$\rho = \frac{Nm}{V}$$
Using this relation we can take the partial derivative of pressure with respect to $\rho$ 
keeping the entropy $S$ constant. Using chain rule we can say that
$$\left(\partiald{p}{\rho}\right)_{S} = \frac{\left(\partial{S}/\partial{T}\right)_{p}}{\left(\partial{S}/\partial{T}\right)_{\rho}}\left(\partiald{p}{\rho}\right)_{T}$$
where we apply the cyclic chain rule to state that
\begin{align*}
\left(\partiald{S}{T}\right)_{\rho}\left(\partiald{T}{\rho}\right)_{S}\left(\partiald{\rho}{S}\right)_{T} &= -1\\
&\Downarrow\\
\left(\partiald{S}{T}\right)_{\rho} &= -\left(\partiald{\rho}{T}\right)_{S}\left(\partiald{S}{\rho}\right)_{T}
\end{align*}
We note that $\rho$ is only dependent on $V$ which implies that 
$$\left(\partiald{\rho}{V}\right)_{S} = \left(\partiald{\rho}{V}\right)_{T}$$ 
So, applying chain rule we find that
\begin{align*}
\left(\partiald{S}{T}\right)_{\rho} &= -\left(\partiald{\rho}{V}\right)_{S}\left(\partiald{V}{T}\right)_{S}\left(\partiald{S}{V}\right)_{T}\left(\partiald{V}{\rho}\right)_{T}\\
&= -\left(\partiald{V}{T}\right)_{S}\left(\partiald{S}{V}\right)_{T}\\
&= \left(\partiald{S}{T}\right)_{V}
\end{align*}
Note we used the cyclic chain rule to find that the last partial derivative. We note the 
definitions for specific heat
\begin{align*}
C_p &= T\left(\partiald{S}{T}\right)_{p}\\
C_V &= T\left(\partiald{S}{T}\right)_{V}
\end{align*}
which implies that
$$\left(\partiald{p}{\rho}\right)_{S} = \frac{C_p}{C_V}\left(\partiald{p}{\rho}\right)_{T}$$

\item Starting with the cyclic chain rule for the variables $p$, $V$, and $T$ we have
\begin{align*}
-1 &= \left(\partiald{p}{T}\right)_{V}\left(\partiald{T}{V}\right)_{p}\left(\partiald{V}{p}\right)_{T}\\
&\Downarrow\\
\left(\partiald{p}{T}\right)_{V} &= -\left(\partiald{V}{T}\right)_{p}\left(\partiald{p}{V}\right)_{T}\\
\end{align*}
Therefore if we introduce the thermal expansion coefficient 
$$\alpha \equiv \frac{1}{V}\left(\partiald{V}{T}\right)_{p}$$
and isothermic compressibility
$$\kappa_{T} \equiv -\frac{1}{V}\left(\partiald{V}{p}\right)_{T}$$ 
we see that
$$\left(\partiald{p}{T}\right)_{V} = -V\alpha(-V\kappa_T)^{-1} = \frac{\alpha}{\kappa_T}$$

\item We can find a relation to the quantities we introduced in part (2) to the specific 
heats of the system by noting that $C_V$ is related to the change in energy through heat by
\begin{align*}
C_V &= \left(\partiald{E}{T}\right)_{V}
\end{align*}
Now, if we treat $E$ as a function of $T$ and $V$ by chain rule we have
$$dE = \left(\partiald{E}{T}\right)_{V}dT + \left(\partiald{E}{V}\right)_{T}dV$$
which using the heat function we have
\begin{align*}
\delta{Q} &= \left(\partiald{E}{T}\right)_{V}dT + \left(\partiald{E}{V}\right)_{T}dV + pdV\\
&= C_VdT + \left(\partiald{E}{V}\right)_{T}\left(\partiald{V}{T}\right)_{p}dT + p\left(\partiald{V}{T}\right)_{p}dT\\
&\Downarrow\\
\frac{\delta{Q}}{dT} -C_V = C_p - C_V &= \left(\left(\partiald{E}{V}\right)_{T} + p\right)\left(\partiald{V}{T}\right)_{p}\\
&= \left(T\left(\partiald{P}{T}\right)_{V} - p + p\right)\left(\partiald{V}{T}\right)_{p}\\
&= T\left(\partiald{p}{T}\right)_{V}\left(\partiald{V}{T}\right)_{p}\\
&= T\frac{\alpha}{\kappa_T}V\alpha = \frac{TV\alpha^2}{\kappa_T}
\end{align*}
\end{enumerate}

\pagebreak

\section{Problem \#2}
\begin{enumerate}[(1)]
\item For a glass of water at temperature $T=300\unit{K}$ placed into a heater with 
temperature $T=363\unit{K}$ and reaches thermal equilibrium. Assuming that the volume of the
water remains constant implies that the change in energy is only dependent on the change in 
heat. This allows us to calculate the change in energy for the water in the glass using the
specific heat
\begin{align*}
dQ = dE &= CdT
\end{align*}
Note we assume that $C$ is a constant and independent of $T$ allows us to calculate
\begin{align*}
\Delta{S} &= \int_{300}^{363}\frac{dQ}{T}\\
&= C\int_{300}^{363}\frac{dT}{T}\\
&= \left.C\ln(T)\right|_{300}^{363}\\
&= C\ln\left(\frac{363}{300}\right)
\end{align*}
We note that we assume that the heater reservoir is large enough to have the temperature 
remain constant. This allows us to directly calculate the change in entropy without integrating
by noting that by conservation of energy the heat gained by the glass is the heat lost by the
reservoir. We calculate this as
$$dQ = -C(363-300) = -63C$$
which implies that the change in entropy is calculated as
$$\Delta{S} = \frac{dQ}{T} = -C\frac{63}{363}$$
So the total change in entropy is given by
$$\Delta{S_{tot}} = C\left(\ln\left(\frac{363}{300}\right)-\frac{63}{363}\right) = 0.0171C$$

\item Now if we do the same process except with a two-stage heating with the first heater at
$T=333\unit{K}$ and the second heater at $T= 363\unit{K}$. We note that the change in heat 
remains the same as the assumption that volume remains constant still holds. Which implies 
that the change in entropy of the cup can be calculated as
\begin{align*}
\Delta{S} &= \int_{300}^{333}\frac{dQ}{T} + \int_{333}^{363}\frac{dQ}{T}\\
&= \int_{300}^{363}\frac{dQ}{T}\\
&= C\ln\left(\frac{363}{300}\right)
\end{align*}
Note this verifies that the change in entropy is path independent. So we calculate the change
in entropy of the two reservoirs by the same process
\begin{align*}
\Delta{S_1} &= -C\frac{33}{333}\\
\Delta{S_2} &= -C\frac{30}{363}
\end{align*}
so we calculate the total change in entropy 
$$\Delta{S_{tot}} = C\left(\ln\left(\frac{363}{300}\right)-\frac{33}{333}-\frac{30}{363}\right) = 0.00888C$$

\item We note that the result from parts (1) and (2) that as we increase the number of steps
between the initial and final temperatures, mathematically we write this as
$$\lim_{N\rightarrow\infty}\frac{\Delta{T}}{N}$$
In this limit the sum of the change in entropies of the reservoirs becomes $-C\ln(363/300)$ 
which makes the total change in entropy equal to zero.
\end{enumerate}

\pagebreak

\section{Problem \#3}
\begin{enumerate}[(1)]
\item Given the energy of a monoatomic ideal gas given by $E=\frac{3}{2}NT$ we can calculate
the efficiency of a cycle consisting of two isotherms and two isochores which follows the 
steps
\begin{itemize}
\item 1 to 2: Expansion from $V_1$ to $V_2$ at constant temperature $T_2$
\item 2 to 3: Decrease in temperature from $T_2$ to $T_1$ at constant volume $V_2$
\item 3 to 4: Compression from $V_2$ to $V_1$ at constant temperature $T_1$
\item 4 to 1: Increase in temperature from $T_1$ to $T_2$ at constant volume $V_1$
\end{itemize}
To calculate the
efficiency we use the relation between the work done on the environment versus the amount of
heat used
$$\eta = \frac{-\Delta{W}}{\Delta{Q}}$$
Therefore, we need to calculate the work done by the cycle. We note that along the two 
isotherms $T$ remains constant so all the heat energy is converted to work from $V_1$ to 
$V_2$. So assuming we our gas is ideal we have an expansion from $V_1$ to $V_2$ at $T_2$ 
which has a change in heat as
\begin{align*}
\Delta{Q}_{12} &= NT_2\ln\left(\frac{V_2}{V_1}\right)
\end{align*}
and for the other isotherm we contract from $V_2$ to $V_1$ at a temperature $T_1$ which yields
\begin{align*}
\Delta{Q}_{34} &= NT_1\ln\left(\frac{V_1}{V_2}\right)
\end{align*}
Note it is assumed that $T_2>T_1$. And as we stated both $\Delta{Q}_1$ and $\Delta{Q}_2$ are
fully converted to work so
\begin{align*}
\Delta{W}_{12} &= -NT_2\ln\left(\frac{V_2}{V_1}\right)\\
\Delta{W}_{34} &= -NT_1\ln\left(\frac{V_1}{V_2}\right)
\end{align*}
Now for the isochores we note that $V$ remains constant so there is no work done or
$$\Delta{W}_{23} = \Delta{W}_{41} = 0$$
but there is a change in heat which is the same as the total change in energy given by
\begin{align*}
\Delta{Q}_{23} &= \frac{3}{2}N(T_2-T_1)\\
\Delta{Q}_{41} &= \frac{3}{2}N(T_1-T_2)\\
\end{align*}
We note that $\Delta{Q}_{34}$ and $\Delta{Q}_{41}$ are negative and therefore are not heat added to
the system. Therefore
$$\Delta{Q} = \Delta{Q}_{12} + \Delta{Q}_{23} = N\left(T_2\ln\left(\frac{V_2}{V_1}\right) + \frac{3}{2}(T_2-T_1)\right)$$
And we find the work done by
$$\Delta{W} = \Delta{W}_{12} + \Delta{W}_{34} = N(T_2-T_1)\ln\left(\frac{V_2}{V_1}\right)$$
So we find the efficiency as
\begin{align*}
\eta &= \frac{(T_2-T_1)\ln(V_2/V_1)}{T_2\ln\left({V_2}/{V_1}\right) + 3/2(T_2-T_1)}\\
&= \frac{T_2-T_1}{T_2+3(T_2-T_1)/2\ln(V_2/V_1)}
\end{align*}
We note that the Carnot cycle has an efficiency of
$$\eta_{c} = 1-\frac{T_1}{T_2} = \frac{T_2-T_1}{T_2}$$
we see that there is an extra term in the efficiency of the Sterling cycle given by
$$\frac{3(T_2-T_1)}{2\ln(V_2/V_1)}$$
which for our assumption $T_2>T_1$ and $V_2>V_1$ must be positive which makes $\eta<\eta_c$.

\item 
For a refrigerator that works as a reversed Carnot cycle which goes through the following
\begin{itemize}
\item 1 to 2: Isothermal expansion at $T_1$
\item 2 to 3: Reversible compression from $T_1$ to $T_2$
\item 3 to 4: Isothermal compression at $T_2$ 
\item 4 to 1: Reversible expansion from $T_2$ to $T_1$ 
\end{itemize}
Where we take $T_2>T_1$. By driving the cycle in reverse (doing work onto the system) we take
heat from from the $T_2$ reservoir and absorbed by the low temperature, $T_1$, reservoir. For
the Carnot cycle we coefficient of performance is given by
$$CoF = \frac{Q_{\textnormal{cold}}}{Q_{\textnormal{hot}}-Q_\textnormal{cold}} = \frac{T_1}{T_2-T_1}$$
so the $CoF$ is less than one for $T_2>2T_1$.
\end{enumerate}
\end{document}

