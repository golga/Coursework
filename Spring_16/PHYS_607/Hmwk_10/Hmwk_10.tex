\documentclass[11pt]{article}

\usepackage{latexsym}
\usepackage{graphicx}
\usepackage{amssymb}
\usepackage{amsthm}
\usepackage{enumerate}
\usepackage{amsmath}
\usepackage{cancel}
\numberwithin{equation}{section}

\setlength{\evensidemargin}{.25in}
\setlength{\oddsidemargin}{-.25in}
\setlength{\topmargin}{-.75in}
\setlength{\textwidth}{6.5in}
\setlength{\textheight}{9.5in}
\newcommand{\due}{April 14th, 2016}
\newcommand{\HWnum}{10}
\newcommand{\alphaH}{\hat{\alpha}}
\newcommand{\alphaHD}{\hat{\alpha}^{\dagger}}
\newcommand{\grad}{\bold\nabla}
\newcommand{\vecE}{\vec{E}}
\newcommand{\scrptR}{\vec{\mathfrak{R}}}
\newcommand{\kapa}{\frac{1}{4\pi\epsilon_0}}
\newcommand{\emf}{\mathcal{E}}
\newcommand{\unit}[1]{\ensuremath{\, \mathrm{#1}}}
\newcommand{\real}{\textnormal{Re}}
\newcommand{\Erf}{\textnormal{Erf}}
\newcommand{\sech}{\textnormal{sech}}
\newcommand{\scrO}{\mathcal{O}}
\newcommand{\levi}{\widetilde{\epsilon}}
\newcommand{\partiald}[2]{\ensuremath{\frac{\partial{#1}}{\partial{#2}}}}
\newcommand{\norm}[2]{\langle{#1}|{#2}\rangle}
\newcommand{\inprod}[2]{\langle{#1}|{#2}\rangle}
\newcommand{\ket}[1]{|{#1}\rangle}
\newcommand{\bra}[1]{\langle{#1}|}





\begin{document}
\begin{titlepage}
\setlength{\topmargin}{1.5in}
\begin{center}
\Huge{Physics 3320} \\
\LARGE{Principles of Electricity and Magnetism II} \\
\Large{Professor Ana Maria Rey} \\[1cm]

\huge{Homework \#\HWnum}\\[0.5cm]

\large{Joe Becker} \\
\large{SID: 810-07-1484} \\
\large{\due} 

\end{center}

\end{titlepage}



\section{Problem \#1}
\begin{enumerate}[(a)]
\item For a system where hydrogen penetrates into a solid at a temperature, $T$, and pressure
, $P$, we assume that within the solid the hydrogen can exist either in the molecular state,
$H_2$ or in the atomic state, $H$. We can write this as a chemical reaction
$$H_2 \rightarrow 2H$$
We assume that the system satisfies the condition for chemical equilibrium 
$$\sum_{i}\nu_i\mu_i = 0$$
where we only have two different types of chemicals. So we see that $\nu_{H2} = 1$ and 
$\nu_{H} = -2$. This implies that
$$\mu_{H2} = 2\mu_{H}$$
Now we can take the chemical potential for $H_2$ to be that of an ideal gas
$$\mu_{H2} = T\ln(P) + \chi(T)$$
and assuming that the concentration of atomic hydrogen is small we can take the chemical 
potential of $H$ to be of the form of a solute
$$\mu_{H} = T\ln(c) + \psi(P,T)$$
So we can see that if we satisfy the equilibrium condition
\begin{align*}
\mu_{H2} &= 2\mu_{H}\\
&\Downarrow\\
T\ln(P) + \chi(T) &= 2T\ln(c) + 2\psi(P,T)\\
&\Downarrow\\
\ln(c) &= \frac{1}{2}\ln(P) + \frac{\chi(T) - 2\psi(P,T)}{2T}\\
&\Downarrow\\
c &\propto P^{1/2}
\end{align*}
Note we neglected the $\psi$ dependence on $P$ we have $\psi$ raised a negative exponential 
which we take to be negligible.

\item For a reaction of the type
$$AB \rightarrow A + B$$
we can find the concentrations of the components if the reaction starts with the pure $AB$ 
substance and eventually arrives at the equilibrium state at a fixed temperature and pressure
by applying \emph{the law of mass action}
\begin{equation}
\prod_{i}c_{0i}^{\nu_{i}} = K_c(P,T)
\label{MassAct}
\end{equation}
Now because we started with pure $AB$ we know that the equilibrium concentrations of the 
products, $A$ and $B$, must be the same which yields the condition
$$c_{0A} = c_{0B}$$
Now we also can impose the condition that
$$c_{0AB}+c_{0A}+c_{0B} =1$$
as we only have those three substances in the system. This combined with our first condition
implies
$$c_{0AB} = 1 - 2c_{0A}$$
noting that $\nu_{AB}=1$ and $\nu_{A}=\nu_{B}=-1$ can use equation \ref{MassAct} to find
\begin{align*}
c_{0AB}^{1}c_{0A}^{-1}c_{0B}^{-1} &= K_c(P,T)\\
&\Downarrow\\
\frac{1-2c_{0A}}{c_{0A}^2} &= K_c(P,T)\\
&\Downarrow\\
1-2c_{0A}-K_c(P,T){c_{0A}^2} &= 0\\
&\Downarrow\\
c_{0A} &= \frac{\sqrt{K_c(P,T)+1}-1}{K_c(P,T)} = c_{0B}
\end{align*}
Note we took the positive solution to the quadratic as we assume that $K_c>0$ as we increase
the concentration of the products. So it follows that 
$$c_{0AB} = 1-2c_{0A} =  \frac{K_{c}(P,T) - 2\sqrt{K_c(P,T)+1}+2}{K_c(P,T)}$$
\end{enumerate}

\pagebreak

\section{Problem \#2}
\begin{enumerate}[(1)]
\item We can find the coefficients of the \emph{Bogoliubov transformation} minimizing the 
ground state energy at zero temperature by noting the transformed Hamiltonian is of the form
$$H_{p,-p} = E^{0}_{p,-p} + \varepsilon_{p}\left(\alphaHD_{p}\alphaH_{-p}+\alphaHD_{-p}\alphaH_{p}\right)$$
where we take the ground state energy to be
$$E^{0}_{p,-p} = 2v_p^2\xi_p-2gnu_pv_p$$
and 
$$\varepsilon_p = \sqrt{\xi_p-(gn)^2}$$
Now we use the condition that $u_p^2-v_p^2=1$ so we can write
$$E^{0}_{p,-p} = 2v_p^2\xi_p-2gn\sqrt{1+v_p^2}v_p$$
which allows us to calculate the $v_p$ which minimizes $E^0_{p,-p}$ as
\begin{align*}
\partiald{E_{p,-p}^{0}}{v_p} = 0 &= 4v_p\xi_p - 2gn\frac{2v_p^2+1}{\sqrt{v_p^2+1}}\\
&\Downarrow\\
gn\frac{2v_p^2+1}{\sqrt{v_p^2+1}} &= 2v_p\xi_p \\
\frac{2v_p^2+1}{v_p\sqrt{v_p^2+1}} &= \frac{2\xi_p}{gn} \\
&\Downarrow\\
\frac{v_p^2(v_p^2+1)}{(2v_p^2+1)^2} &= \frac{\xi_p^2-\varepsilon_p^2}{4\xi_p^2} \\
&\Downarrow\\
v_p^4+v_p^2+\frac{1}{4} &= \left(\frac{\xi_p}{2\varepsilon_p}\right)^2\\
&\Downarrow\\
v_p^2 &= \frac{1}{2}\left(\frac{\xi_p}{\varepsilon_p}-1\right)
\end{align*}
or if we solve for $u_p^2 = 1+v_p^2$ we have
$$u_p^2 = \frac{1}{2}\left(\frac{\xi_p}{\varepsilon_p}+1\right)$$
Note these are the results we expected as they follow from the transformation derivation.
\end{enumerate}

\pagebreak

\section{Problem \#3}
\begin{enumerate}[(1)]
\item To find the density of the over-condensate particles of a weakly interacting Bose 
gas at zero temperature by taking the number of particles to be
$$N = N_0 + \sum_{p\ne0}\hat{a}^{\dagger}_p\hat{a}_{p}$$
Where $N_0$ is the number of particles in the condensate. If we take the \emph{Bogoliubov
transformation} where
\begin{align*}
\hat{a}_{p} &= u_p\alphaH_p-v_p\alphaHD_{-p}\\
\hat{a}^{\dagger}_{p} &= -v_p\alphaH_{-p} + u_p\alphaHD_p
\end{align*}
Which allows us to transform 
\begin{align*}
\hat{a}^{\dagger}_p\hat{a}_{p} &=\left(-v_p\alphaH_{-p} + u_p\alphaHD_p\right) \left(u_p\alphaH_p-v_p\alphaHD_{-p}\right)\\
&= u_p^2\alphaHD_p\alphaH_p + v_p^2\alphaH_{-p}\alphaHD_{-p} -u_pv_p\left(\alphaHD_{p}\alphaHD_{-p}+\alphaH_{-p}\alphaH_{p}\right)\\
&= u_p^2\alphaHD_p\alphaH_p + v_p^2\left(\alphaHD_{-p}\alphaH_{-p}+1\right) -u_pv_p\left(\alphaHD_{p}\alphaHD_{-p}+\alphaH_{-p}\alphaH_{p}\right)\\
&= v_p^2 + u_p^2\alphaHD_p\alphaH_p + v_p^2\alphaHD_{-p}\alphaH_{-p} -u_pv_p\left(\alphaHD_{p}\alphaHD_{-p}+\alphaH_{-p}\alphaH_{p}\right)\\
&= v_p^2
\end{align*}
Note that we can neglect all terms that do not conserve momentum which leaves us with 
the $v_p^2$ term. Now we evaluate the sum 
\begin{align*}
N_{e} = \sum_{p\ne{0}}v_p^2 &= V\int_{0}^{\infty}v_p^2p^2\frac{d^3p}{(2\pi\hbar)^3}\\
n_e &= \frac{4\pi}{2(2\pi\hbar)^3}\int_{0}^{\infty}p^2\left(\frac{\xi_p}{\varepsilon_p}-1\right)dp\\
&= \frac{4\pi}{2(2\pi\hbar)^3}\int_{0}^{\infty}p^2\left(\frac{p^2/2m+gn}{\sqrt{p^2/2m(2gn+p^2/2m)}}-1\right)dp\\
&= \frac{4\pi2^{3/2}m^3s^3}{2(2\pi\hbar)^3}\int_{0}^{\infty}y^2\left(\frac{y^2+1}{\sqrt{x^4+2x^2}}-1\right)dy\\
&= \frac{2^{3/2}m^3s^3}{2(2\pi\hbar)^3}\frac{\sqrt{2}}{3}\\
&= \frac{4}{3\pi^2}\left(\frac{ms}{\hbar}\right)^3
\end{align*}
Note we used the change of variables 
$$s^2=\frac{gn}{m}\qquad p=(\sqrt{2}ms)x$$

\item In order to calculate the specific heat of the weakly interacting Bose gas we note 
that the energy spectrum is given as
$$\varepsilon_p = \sqrt{s^2p^2+\left(\frac{p^2}{2m}\right)^2}$$
which we take to the small $p$ limit $\varepsilon_p\approx sp$. This allows us to calculate 
the energy as
\begin{align*}
E &= V\int\frac{\varepsilon_p}{e^{\varepsilon_p/T}-1}\frac{d^3p}{(2\pi\hbar)^3}\\
&= V\frac{s}{2\pi^2\hbar^3}\int\frac{p^3}{e^{sp/T}-1}dp\\
&= \frac{V\pi^2T^4}{30(s\hbar)^3}\\
\end{align*}
Therefore we can find the specific heat as
$$C_V = \left(\partiald{E}{T}\right)_{V} = \frac{2V\pi^2T^3}{15(s\hbar)^3}$$
and the entropy as
$$S = \frac{E}{T} = \frac{V\pi^2T^3}{30(s\hbar)^3}$$
\end{enumerate}

\pagebreak

\section{Problem \#4}
\begin{enumerate}[(1)]
\item We can find the normal density of the Bogoliubov gas by calculating 
\begin{align*}
\rho_n &= -\frac{1}{3}\int p^2\frac{df}{d\varepsilon_p}\frac{d^3p}{(2\pi\hbar)^3}\\
&= \frac{1}{3}\frac{1}{2\pi^2\hbar^3}\int p^4\frac{e^{sp/T}}{e^{sp/T}-1}dp\\
&= \frac{4\zeta(5)T^5}{\pi^2\hbar^3s^5}
\end{align*}
and the superfluid density as
\begin{align*}
\rho_s &= -\frac{1}{3s}\int p^2\frac{df}{dp}\frac{d^3p}{(2\pi\hbar)^3}\\
&= \frac{1}{3}\frac{1}{2\pi^2\hbar^3}\frac{1}{T}\int p^4\frac{e^{sp/T}}{(e^{sp/T}-1)^2}dp\\
&= \frac{2\pi{T^4}}{45\hbar^3s^5}
\end{align*}
We note the difference in the power of $T$ this results in the limited case where $T>>gn=ms^2$ 
it will result in a normal density that goes to one and a super fluid density that goes to
zero. In the other limit we see that both densities becomes smaller than one.

\item We can compare the results for the condensate density found as
\begin{align*}
n &= \int\frac{1}{e^{\varepsilon_p/T}-1}\frac{d^3p}{(2\pi\hbar)^3}\\
&= \frac{1}{2\pi^2\hbar^3}\int\frac{p^2}{e^{sp/T}-1}dp\\
&= \frac{2\zeta(3)T^3}{2\pi^2(s\hbar)^3}
\end{align*}
we see that this differs from the normal density by a factor of $(T/s)^2$.



\end{enumerate}

\end{document}

