\documentclass[11pt]{article}

\usepackage{latexsym}
\usepackage{graphicx}
\usepackage{amssymb}
\usepackage{amsthm}
\usepackage{enumerate}
\usepackage{amsmath}
\usepackage{cancel}
\numberwithin{equation}{section}

\setlength{\evensidemargin}{.25in}
\setlength{\oddsidemargin}{-.25in}
\setlength{\topmargin}{-.75in}
\setlength{\textwidth}{6.5in}
\setlength{\textheight}{9.5in}
\newcommand{\due}{February 22nd, 2016}
\newcommand{\HWnum}{5}
\newcommand{\grad}{\bold\nabla}
\newcommand{\vecE}{\vec{E}}
\newcommand{\scrptR}{\vec{\mathfrak{R}}}
\newcommand{\kapa}{\frac{1}{4\pi\epsilon_0}}
\newcommand{\emf}{\mathcal{E}}
\newcommand{\unit}[1]{\ensuremath{\, \mathrm{#1}}}
\newcommand{\real}{\textnormal{Re}}
\newcommand{\Erf}{\textnormal{Erf}}
\newcommand{\sech}{\textnormal{sech}}
\newcommand{\scrO}{\mathcal{O}}
\newcommand{\levi}{\widetilde{\epsilon}}
\newcommand{\partiald}[2]{\ensuremath{\frac{\partial{#1}}{\partial{#2}}}}
\newcommand{\norm}[2]{\langle{#1}|{#2}\rangle}
\newcommand{\inprod}[2]{\langle{#1}|{#2}\rangle}
\newcommand{\ket}[1]{|{#1}\rangle}
\newcommand{\bra}[1]{\langle{#1}|}





\begin{document}
\begin{titlepage}
\setlength{\topmargin}{1.5in}
\begin{center}
\Huge{Physics 3320} \\
\LARGE{Principles of Electricity and Magnetism II} \\
\Large{Professor Ana Maria Rey} \\[1cm]

\huge{Homework \#\HWnum}\\[0.5cm]

\large{Joe Becker} \\
\large{SID: 810-07-1484} \\
\large{\due} 

\end{center}

\end{titlepage}



\section{Problem \#1}
\begin{enumerate}[(1)]
\item For a degenerate ($T<<\mu$) Fermi gas we have the number of particles given by the 
integral
\begin{equation}
N = V\int_{0}^{\infty}f(\varepsilon)\nu(\varepsilon)d\varepsilon
\label{DFGNum}
\end{equation}
where $\nu(\varepsilon)$ is the density of states given by
\begin{equation}
\nu(\varepsilon) = \frac{g_sm^{3/2}}{\sqrt{2}\pi^2\hbar^3}\varepsilon^{1/2}
\label{DenOStates}
\end{equation}
and $f(\varepsilon)$ is the Fermi-Dirac distribution
\begin{equation}
f(\varepsilon) = \frac{1}{e^{(\varepsilon-\mu)/T}+1}
\label{FDDist}
\end{equation}
This implies that we can calculate the energy $E$ of the system by the integral
\begin{equation}
E = V\int_{0}^{\infty}{\varepsilon}f(\varepsilon)\nu(\varepsilon)d\varepsilon
\label{DFGEnergy}
\end{equation}
We note that both equations \ref{DFGNum} and \ref{DFGEnergy} have integrals of the form
$$I = \int_{0}^{\infty}\frac{\varepsilon^n}{e^{(\varepsilon-\mu)/T}+1}$$
therefore we wish to approximate the solution for $I$ in the limit of small temperatures 
where we take $1<<\mu/T$. So we apply a change of variable by $(\varepsilon-\mu)/T=z$ so that
the integral becomes
\begin{align*}
I &= \int_{-\mu/T}^{\infty}\frac{(\mu+Tz)^{n}}{e^{z}+1}Tdz\\
&= -T\int_{0}^{-\mu/T}\frac{(\mu+Tz)^{n}}{e^{z}+1}dz + T\int_{0}^{\infty}\frac{(\mu+Tz)^{n}}{e^{z}+1}dz\\
&= T\int_{0}^{\mu/T}\frac{(\mu-Tz)^{n}}{e^{-z}+1}dz + T\int_{0}^{\infty}\frac{(\mu+Tz)^{n}}{e^{z}+1}dz
\end{align*}
Note that we convert the first integral by $z\rightarrow-z$ also note that
$$\frac{1}{e^-z+1} = 1 - \frac{1}{e^z+1}$$
So we have
\begin{align*}
I &= \int_{0}^{\mu/T}(\mu-Tz)^{n}Tdz - T\int_{0}^{\mu/T}\frac{(\mu-Tz)^{n}}{e^{z}+1}dz  + T\int_{0}^{\infty}\frac{(\mu+Tz)^{n}}{e^{z}+1}dz\\
&= \int_{0}^{\mu}\varepsilon^{n}d\varepsilon + T\int_{0}^{\infty}\frac{(\mu+Tz)^{n}-(\mu-Tz)^{n}}{e^{z}+1}dz
\end{align*}
Where we take the limit of $\mu/T\rightarrow\infty$ which follows by our low temperature 
assumption. Note the Taylor expansion of a general function of the form $f(a+bx)$ about $x=0$
\begin{align*}
f(a+bx) &\approx f(a) + bf'(a)x + \frac{b^2}{2}f^{(2)}(a)x^2 + \frac{b^3}{3!}f^{(3)}(a)x^3 + \scrO{(x^4)}\\
f(a-bx) &\approx f(a) - bf'(a)x + \frac{b^2}{2}f^{(2)}(a)x^2 - \frac{b^3}{3!}f^{(3)}(a)x^3 + \scrO{(x^4)}
\end{align*}
We note that for $f(a+bx)-f(a-bx)$ all the even powers cancel. So for our integral to leading
order we have in $T$
$$I = \int_{0}^{\mu}\varepsilon^{n}d\varepsilon + 2n\mu^{n-1}T^2\int_{0}^{\infty}\frac{z}{e^{z}+1}dz +\scrO{(T^4)}$$
Note that we can calculate the integral by 
\begin{align*}
\int_{0}^{\infty}\frac{z}{e^{z}+1}dz &= 2\int_{0}^{\infty}\frac{ze^{-z}}{e^{-z}+1}dz \\
&= 2\sum_{k=1}^{\infty}\int_{0}^{\infty}(-1)^kze^{-kz}dz\\
&= 2\sum_{k=1}^{\infty}\frac{(-1)^{k}}{k^2}\int_{0}^{\infty}e^{-u}udu\\
&= \Gamma(2)\sum_{k=1}^{\infty}\frac{1}{k^2}\\
&= \Gamma(2)\zeta(2) = \frac{\pi^2}{6}
\end{align*}
Where we use the \emph{Gamma Function}
$$\Gamma(x) = \int_{0}^{\infty}u^{x-1}e^{-u}du$$
and the \emph{Riemann Zeta Function}
$$\zeta(x) = \sum_{k=1}^{\infty}\frac{1}{k^{x}}$$
So to second order in $T$ we have
$$I = \frac{\mu^{n+1}}{n+1} + \frac{\pi^2}{6}n\mu^{n-1}T^2 +\scrO{(T^4)}$$
So we can approximate $N$ where $n=1/2$ as
\begin{align*}
N &= V\int_{0}^{\infty}f(\varepsilon)\nu(\varepsilon)d\varepsilon = \frac{Vg_sm^{3/2}}{\sqrt{2}\pi^2\hbar^3}I(n=1/2)\\
&= \frac{Vg_sm^{3/2}}{\sqrt{2}\pi^2\hbar^3}\left(\frac{2}{3}\mu^{3/2}+\frac{\pi^2}{12}\frac{T^2}{\mu^{1/2}}\right)
\end{align*}
and for $E$ we approximate using $n=3/2$ as
\begin{align*}
E &= V\int_{0}^{\infty}{\varepsilon}f(\varepsilon)\nu(\varepsilon)d\varepsilon = \frac{Vg_sm^{3/2}}{\sqrt{2}\pi^2\hbar^3}I(n=3/2)\\
&= \frac{Vg_sm^{3/2}}{\sqrt{2}\pi^2\hbar^3}\left(\frac{2}{5}\mu^{5/2}+\frac{\pi^2}{4}T^2\mu^{1/2}\right)
\end{align*}
Given that the thermodynamic potential is related to energy by $\Omega=-2/3E$ we can see that
$$\Omega = -\frac{2}{3}E =  -\frac{\sqrt{2}Vg_sm^{3/2}}{3\pi^2\hbar^3}\left(\frac{2}{5}\mu^{5/2}+\frac{\pi^2}{4}T^2\mu^{1/2}\right)$$
Now we can find the free energy to second order in $T$ by
\begin{align*}
F &= \Omega + \mu{N} = -\frac{\sqrt{2}Vg_sm^{3/2}}{3\pi^2\hbar^3}\left(\frac{2}{5}\mu^{5/2}+\frac{\pi^2}{4}T^2\mu^{1/2}\right) +\mu\left(\frac{Vg_sm^{3/2}}{\sqrt{2}\pi^2\hbar^3}\left(\frac{2}{3}\mu^{3/2}+\frac{\pi^2}{12}\frac{T^2}{\mu^{1/2}}\right)\right)\\
&= \frac{Vg_sm^{3/2}}{\sqrt{2}\pi^2\hbar^3}\left[-\frac{4}{15}\mu^{5/2}-\frac{\pi^2}{6}T^2\mu^{1/2}+\frac{2}{3}\mu^{5/2}+\frac{\pi^2}{12}T^2\mu^{1/2}\right]\\
&= \frac{Vg_sm^{3/2}}{\sqrt{2}\pi^2\hbar^3}\frac{2}{5}\mu^{5/2}\left[1 - \frac{5\pi^2}{24}\left(\frac{T}{\mu}\right)^2\right]
\end{align*}
Now we take $\mu=\varepsilon_F$ where $\varepsilon_F$ is the Fermi energy given by
$$\varepsilon_F = \frac{p_F^2}{2m} = \frac{\hbar^2(6\pi^2N)^{2/3}}{2m(Vg_s)^{2/3}} = \gamma{N^{2/3}}$$
So we have the free energy as a function of the number of particles given by
\begin{align*}
F &= \frac{Vg_sm^{3/2}}{\sqrt{2}\pi^2\hbar^3}\frac{2}{5}\mu^{5/2}\left[1 - \frac{5\pi^2}{24}\left(\frac{T}{\mu}\right)^2\right]\\
&\Downarrow\\
&= \frac{3N}{5\varepsilon_F^{3/2}}\left[\varepsilon_F^{5/2} - \frac{5\pi^2}{12}T^2\varepsilon_F^{1/2}\right]\\
&= \frac{3N}{5}\left[\varepsilon_F - \frac{5\pi^2}{12}\frac{T^2}{\varepsilon_F}\right]\\
&= \frac{3}{5}\left[\gamma{N^{5/3}} - \frac{5\pi^2}{12\gamma}T^2N^{1/3}\right]
\end{align*}
So we can find $\mu$ by the derivative
\begin{align*}
\mu = \left(\partiald{F}{N}\right)_{V,T} &= \frac{3}{5}\left[\frac{5}{3}\gamma{N^{2/3}} - \frac{5\pi^2}{12}\frac{1}{3}\frac{T^2}{\gamma{N^{2/3}}}\right]\\
&\Downarrow\\
&= \left(\varepsilon_F - \frac{\pi^2}{12}\frac{T^2}{\varepsilon_F}\right)\\
&= \varepsilon_F\left(1 - \frac{\pi^2}{12}\left(\frac{T}{\varepsilon_F}\right)^2\right) + \scrO(T^3)
\end{align*}

\item We can use the result from part (1) to find the correction to the energy by taking the
ratio and keeping only terms to second order in $T$
\begin{align*}
\frac{E}{N} &= \left(\frac{2}{5}\mu^{5/2}+\frac{\pi^2}{4}T^2\mu^{1/2}\right)\left(\frac{2}{3}\mu^{3/2}+\frac{\pi^2}{12}\frac{T^2}{\mu^{1/2}}\right)^{-1}\\
&= \frac{3}{5}\mu\left(1+\frac{5\pi^2}{8}\left(\frac{T}{\mu}\right)^2\right)\left(1+\frac{\pi^2}{8}\left(\frac{T}{\mu}\right)^2\right)^{-1}\\
&\approx \frac{3}{5}\varepsilon_F\left(1 - \frac{\pi^2}{12}\left(\frac{T}{\varepsilon_F}\right)^2\right)\left(1+\frac{5\pi^2}{8}\left(\frac{T}{\varepsilon_F}\right)^2\right)\left(1-\frac{\pi^2}{8}\left(\frac{T}{\varepsilon_F}\right)^2\right) + \scrO(T^3)\\
&\approx \frac{3}{5}\varepsilon_F\left(1 + \frac{5\pi^2}{12}\left(\frac{T}{\varepsilon_F}\right)^2\right) + \scrO(T^3)
\end{align*}
Now we see that only the second term depends on $T$ this implies that the specific heat at 
fixed volume to second order in temperature is 
\begin{align*}
\left(\partiald{E}{T}\right)_{N,V} &= \partiald{}{T}\left(\frac{N\pi^2}{4\varepsilon_F}T^2\right) \\
&= \frac{N\pi^2}{2\varepsilon_F}T
\end{align*}

\item Given the isothermal and adiabatic compressibility
$$\kappa_T = -\frac{1}{V}\left(\partiald{V}{p}\right)_T\qquad\kappa_S = -\frac{1}{V}\left(\partiald{V}{p}\right)_S$$
we can calculate $\kappa_T$ and $\kappa_S$ by first noting that 
$$p = \left(\partiald{E}{V}\right)_{N,S} = -\left(\partiald{F}{V}\right)_{N,T}$$
We found the approximations for $F$ and $E$ already where we note that the Fermi energy is
proportional volume by
$$\varepsilon_F = \frac{\hbar^2(6\pi^2N)^{2/3}}{2m(Vg_s)^{2/3}} = \gamma'V^{-2/3}$$
so we have
\begin{align*}
\kappa_S^{-1} = -V\partiald{^2E}{V^2} &= -V\partiald{^2}{V^2}\left[\frac{3}{5}N\varepsilon_F\left(1 + \frac{5\pi^2}{12}\left(\frac{T}{\varepsilon_F}\right)^2\right)\right]\\
&= -\frac{3}{5}NV\partiald{^2}{V^2}\left[\gamma'V^{-2/3} + \frac{5\pi^2}{12}\frac{T^2}{\gamma'}V^{2/3}\right]\\
&= -\frac{3}{5}NV\partiald{}{V}\left[-\frac{2}{3}\gamma'V^{-5/3} + \frac{2}{3}\frac{5\pi^2}{12}\frac{T^2}{\gamma'}V^{-1/3}\right]\\
&= -\frac{2}{3}NV\left(\gamma'V^{-8/3} - \frac{\pi^2}{12}\frac{T^2}{\gamma'}V^{-4/3}\right)\\
&= -\frac{2}{3}N\left(\frac{\varepsilon_F}{V} - \frac{\pi^2}{12}\frac{T^2}{\epsilon_FV}\right)\\
&\Downarrow\\
\kappa_S &= -\frac{3V}{2N\epsilon_F}\left(1 + \frac{\pi^2}{12}\left(\frac{T}{\epsilon_F}\right)^2\right) +\scrO(T^3)
\end{align*}
Now we repeat the process for the isothermal case
\begin{align*}
\kappa_T^{-1} = V\partiald{^2F}{V^2} &= V\partiald{^2}{V^2}\left[\frac{3N}{5}\left(\varepsilon_F - \frac{5\pi^2}{12}\frac{T^2}{\varepsilon_F}\right)\right]\\
&= \frac{3}{5}NV\partiald{^2}{V^2}\left[\gamma'V^{-2/3} - \frac{5\pi^2}{12}\frac{T^2}{\gamma'}V^{2/3}\right]\\
&= \frac{2}{3}\frac{N\varepsilon_F}{V}\partiald{^2}{V^2}\left[1 + \frac{\pi^2}{12}\left(\frac{T}{\varepsilon_F}\right)^2\right)\\
&\Downarrow\\
\kappa_T &= -\frac{3V}{2N\epsilon_F}\left(-1 + \frac{\pi^2}{12}\left(\frac{T}{\epsilon_F}\right)^2\right) +\scrO(T^3)
\end{align*}
We see that for $T=0$ we have equal and opposite expansions for each case.

\item For an electron-proton plasma that has turned into a dense neutron gas at zero 
temperature in which we assume this is an ideal gas. We state that the density of this gas
must be so great that it overcomes the degeneracy pressure given as
$$pV = \frac{2}{5}N\varepsilon_F$$
so that the electrons combine with the protons in the nucleus. This yields the relation
$$n>>\frac{5}{2}\frac{p}{\varepsilon_F}$$
\end{enumerate}


\section{Problem \#2}
\begin{enumerate}[(1)]
\item For a Bose gas of $^{39}K$ atoms whose density is $10^{15}\unit{cm^{-3}}$ we can find 
the condensation temperature by the formula
$$T_{BEC} = 3.31\frac{\hbar^2n^{2/3}}{mk} = 3.31\frac{(1.05\times10^{-34}\unit{m^2\ kg\ s^{-1}})^2(10^{15}\unit{cm^{-3}})^{2/3}}{(6.47\times10^{-26}\unit{kg})(1.38\times10^{-23}\unit{J\ K^{-1})}} = 4.1\times10^{-6}\unit{K}$$

\item For a Bose gas of particles with the mass, $m$, is placed into an anisotropic 
oscillatory potential of the following form
$$V(x,y,z) = \frac{m\omega^2\left(x^2+y^2\right)}{2} + \frac{m\Omega^2z^2}{2}$$
where we assume that $\omega<<\Omega$ and both oscillator lengths $I_{\omega} = \sqrt{\hbar/m\omega}$ and 
$I_{\Omega} = \sqrt{\hbar/m\Omega}$ are much larger than the distance between particles in
the gas, $n^{-1/3}$. We can find the temperature of it's Bose-Einstein condensate by taking
the distribution function
$$f(\varepsilon) = \frac{1}{e^{(\varepsilon(\mathbf{r},\mathbf{p})-\mu)/T} - 1}$$
where the energy is given by the harmonic oscillator
$$\varepsilon = \hbar\omega\left(n_x+\frac{1}{2}\right) + \hbar\omega\left(n_y+\frac{1}{2}\right) + \hbar\Omega\left(n_z+\frac{1}{2}\right)$$
Using this relation we can calculate the density of states $\nu(\varepsilon)$ by integrating 
over the energy space to find that 
$$\nu(\varepsilon) = \frac{\varepsilon^2}{2\hbar^3\omega^2\Omega}$$
This allows us to calculate the number of particles as
\begin{align*}
N = \int_{0}^{\infty}f(\varepsilon)\nu(\varepsilon)d\varepsilon &=  \frac{1}{2\hbar^3\omega^2\Omega}\int_{0}^{\infty}\frac{\varepsilon^2}{e^{(\varepsilon-\mu)/T}-1}d\varepsilon
\end{align*}
We note that we reach condensation at $T=T_{BEC}$ and $\mu=0$ so we have
\begin{align*}
N &=  \frac{1}{2\hbar^3\omega^2\Omega}\int_{0}^{\infty}\frac{\varepsilon^2}{e^{\varepsilon/T_{BEC}}-1}d\varepsilon\\
&= \frac{2T_{BEC}^3\zeta(3)}{2\hbar^3\omega^2\Omega}\\
&= \frac{(1.20)T_{BEC}^3}{\hbar^3\omega^2\Omega}\\
&\Downarrow\\
T_{BEC} &= \left(\frac{N\hbar^3\omega^2\Omega}{1.20}\right)^{1/3}
\end{align*}

\item To find the specific heat near the condensation temperature we calculate the energy by
\begin{align*}
E = \int_{0}^{\infty}\varepsilon f(\varepsilon)\nu(\varepsilon)d\varepsilon &=  \frac{1}{2\hbar^3\omega^2\Omega}\int_{0}^{\infty}\frac{\varepsilon^3}{e^{(\varepsilon-\mu)/T}-1}d\varepsilon\\
&\Downarrow\\
&=  \frac{1}{2\hbar^3\omega^2\Omega}\int_{0}^{\infty}\frac{\varepsilon^3}{e^{\varepsilon/T_{BEC}}-1}d\varepsilon\\
&=  \frac{1}{2\hbar^3\omega^2\Omega}\frac{\pi^4T_{BEC}^4}{15}
\end{align*}
So the specific heat is given by
\begin{align*}
C_V = \partiald{E}{T} &= \frac{2\pi^2T_{BEC}^3}{15\hbar^2\omega^2\Omega}
\end{align*}



\end{enumerate}
\pagebreak

\end{document}
