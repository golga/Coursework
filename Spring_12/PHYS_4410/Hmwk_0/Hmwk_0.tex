\documentclass[11pt]{article}

\usepackage{latexsym}
\usepackage{amssymb}
\usepackage{amsthm}
\usepackage{enumerate}
\usepackage{amsmath}
\usepackage{cancel}
\numberwithin{equation}{section}

\setlength{\evensidemargin}{.25in}
\setlength{\oddsidemargin}{-.25in}
\setlength{\topmargin}{-.75in}
\setlength{\textwidth}{6.5in}
\setlength{\textheight}{9.5in}
\newcommand{\due}{January 24th, 2012}
\newcommand{\HWnum}{0}
\newcommand{\grad}{\bold\nabla}
\newcommand{\vecE}{\vec{E}}
\newcommand{\scrptR}{\vec{\mathfrak{R}}}
\newcommand{\kapa}{\frac{1}{4\pi\epsilon_0}}
\newcommand{\emf}{\mathcal{E}}
\newcommand{\unit}[1]{\ensuremath{\, \mathrm{#1}}}
\newcommand{\real}{\textnormal{Re}}
\newcommand{\Erf}{\textnormal{Erf}}
\newcommand{\sech}{\textnormal{sech}}
\newcommand{\scrO}{\mathcal{O}}
\newcommand{\levi}{\widetilde{\epsilon}}
\newcommand{\partiald}[2]{\ensuremath{\frac{\partial{#1}}{\partial{#2}}}}
\newcommand{\norm}[2]{\langle{#1}|{#2}\rangle}
\newcommand{\inprod}[2]{\langle{#1}|{#2}\rangle}
\newcommand{\ket}[1]{|{#1}\rangle}
\newcommand{\bra}[1]{\langle{#1}|}





\begin{document}
\begin{titlepage}
\setlength{\topmargin}{1.5in}
\begin{center}
\Huge{Physics 3320} \\
\LARGE{Principles of Electricity and Magnetism II} \\
\Large{Professor Ana Maria Rey} \\[1cm]

\huge{Homework \#\HWnum}\\[0.5cm]

\large{Joe Becker} \\
\large{SID: 810-07-1484} \\
\large{\due} 

\end{center}

\end{titlepage}



\section{Problem \#1}
\begin{enumerate}[(a)]
\item
Given the general form of the \emph{Time-independent Schr\"{o}dinger Equation}
\begin{equation}
-\frac{\hbar^2}{2m}\grad^2\psi + V\psi = E\psi
\label{TISE}
\end{equation}
for the two dimensional potential 
$$V(x,y) = \frac{1}{2}m_e\omega^2\left(x^2+y^2\right)$$
we have equation \ref{TISE} in the form
$$-\frac{\hbar^2}{2m_e}\frac{\partial^2}{\partial x^2}\psi(x,y) - \frac{\hbar^2}{2m_e}\frac{\partial^2}{\partial y^2}\psi(x,y) + \frac{1}{2}m_e\omega^2\left(x^2+y^2\right)\psi(x,y) = E\psi(x,y)$$
note that $m_e$ is the mass of the electron and the wave function $\psi(x,y)$ is a function of both $x$ and $y$.

\item
We can solve equation \ref{TISE} for the two dimensional harmonic potential by using polar coordinates given that
$$\grad^2 = \frac{1}{r}\frac{\partial}{\partial r}\left(r\frac{\partial}{\partial r}\right) + \frac{1}{r^2}\frac{\partial}{\partial\theta}$$
in polar coordinates. Also using the fact that $r^2 = x^2+y^2$ we can right the potential as
$$V(r) = \frac{1}{2}m_e\omega^2r^2$$
note that the potential is only a function of $r$. So equation \ref{TISE} becomes
$$-\frac{\hbar^2}{2m_e}\frac{1}{r}\frac{\partial}{\partial r}\left(r\frac{\partial}{\partial r}\psi(r,\theta)\right) - \frac{\hbar^2}{2m_e}\frac{1}{r^2}\frac{\partial^2}{\partial\theta^2}\psi(r,\theta) + \frac{1}{2}m_e\omega^2r^2\psi(r,\theta) = E\psi(r,\theta)$$
now if we assume that the that the solution is in a separable form we have
$$\psi(r,\theta ) = R(r)Q(\theta)$$
and we can now apply the method of separation of variables which yields
\begin{align*}
-\frac{\hbar^2}{2m_e}\frac{1}{r}\frac{\partial}{\partial r}\left(r\frac{\partial}{\partial r}\psi(r,\theta)\right) - \frac{\hbar^2}{2m_e}\frac{1}{r^2}\frac{\partial^2}{\partial\theta^2}\psi(r,\theta) + \frac{1}{2}m_e\omega^2r^2\psi(r,\theta) &= E\psi(r,\theta) \\
&\Downarrow \\
-\frac{\hbar^2}{2m_e}\frac{1}{r}\frac{\partial}{\partial r}\left(r\frac{\partial}{\partial r}R(r)Q(\theta)\right) - \frac{\hbar^2}{2m_e}\frac{1}{r^2}\frac{\partial^2}{\partial\theta^2}R(r)Q(\theta) + \frac{1}{2}m_e\omega^2r^2R(r)Q(\theta) &= ER(r)Q(\theta)\\
-\frac{\hbar^2}{2m_e}\frac{Q(\theta)}{r}\left(r\frac{\partial^2 R(r)}{\partial r^2} + \frac{\partial R(r)}{\partial r}\right) - \frac{\hbar^2}{2m_e}\frac{R(r)}{r^2}\frac{\partial^2 Q(\theta)}{\partial\theta^2} &= \left(E -\frac{1}{2}m_e\omega^2r^2\right)R(r)Q(\theta)
\end{align*}
Now if we divide both sides by $R(r)Q(\theta)$ we find that
\begin{align*}
-\frac{\hbar^2}{2m_e}\frac{\cancel{Q(\theta)}}{R(r)\cancel{Q(\theta)}r}\left(r\frac{\partial^2 R(r)}{\partial r^2} + \frac{\partial R(r)}{\partial r}\right) - \frac{\hbar^2}{2m_e}\frac{\cancel{R(r)}}{\cancel{R(r)}Q(\theta)r^2}\frac{\partial^2 Q(\theta)}{\partial\theta^2} &= E -\frac{1}{2}m_e\omega^2r^2\\
-\frac{\hbar^2}{2m_e}\frac{1}{R(r)r}\left(r\frac{\partial^2 R(r)}{\partial r^2} + \frac{\partial R(r)}{\partial r}\right) - \frac{\hbar^2}{2m_e}\frac{1}{Q(\theta)r^2}\frac{\partial^2 Q(\theta)}{\partial\theta^2} &= E -\frac{1}{2}m_e\omega^2r^2\\
&\Downarrow\\
-\frac{\hbar^2}{2m_e}\frac{1}{R(r)r}\left(r\frac{\partial^2 R(r)}{\partial r^2} + \frac{\partial R(r)}{\partial r}\right) - E + \frac{1}{2}m_e\omega^2r^2 &= \frac{\hbar^2}{2m_e}\frac{1}{Q(\theta)r^2}\frac{\partial^2 Q(\theta)}{\partial\theta^2}
\end{align*}
We can now cancel the $r^{-2}$ with the $\theta$ side of the equation to get
\begin{align*}
-\frac{1}{R(r)}\left(r^2\frac{\partial^2 R(r)}{\partial r^2} + r\frac{\partial R(r)}{\partial r}\right) + \frac{2m_er^2}{\hbar^2}\left(\frac{1}{2}m_e\omega^2r^2 - E\right) &= \frac{1}{Q(\theta)}\frac{\partial^2 Q(\theta)}{\partial\theta^2}
\end{align*}
Now we can note that each side of the equation is only dependent on a single variable. Therefore each side of the equation can only be a constant. So we can separate the radial and angular differential equations. Where the radial equation is
\begin{equation}
-r^2\frac{\partial^2 R(r)}{\partial r^2} - r\frac{\partial R(r)}{\partial r} + \frac{2m_er^2}{\hbar^2}\left(\frac{1}{2}m_e\omega^2r^2 - E\right)R(r) = k^2R(r)
\label{RadDiffEq}
\end{equation}
and the angular equation is
\begin{equation}
\frac{\partial^2 Q(\theta)}{\partial\theta^2} = -k^2Q(\theta)
\label{AngDiffEq}
\end{equation}
Note that we anticipate that the solution to $Q(\theta)$ is a harmonic function therefore we defined the constant as $-k^2$

\item
\begin{enumerate}[(i)]
\item
For the solution to be a physical solution need that the radial component of the wavefunction $\psi(r,\theta) = R(r)Q(\theta)$ go to zero as $r\rightarrow\infty$ so that the wavefunction is normalizable. Therefore we impose the boundary condition
$$\lim_{r\rightarrow\infty} R(r) = 0$$

\item
For the angular component of $\psi(r,\theta)$ we see that the boundary must be periodic with a period of $2\pi$ in order for the solution to be physical. This implies that
$$Q(0) = Q(2\pi)$$
\end{enumerate}

\item
Using the boundary condition we specified in part (c) we can now solve equation \ref{AngDiffEq}
$$\frac{\partial^2 Q(\theta)}{\partial\theta^2} = -k^2Q(\theta)$$
which we can see is a periodic function so we can guess the solution as
$$Q(\theta) = A\sin(c\theta) + B\cos(c\theta)$$
which we can very that
\begin{align*}
Q'(\theta) &= Ac\cos(c\theta) - Bc\sin(c\theta)\\
&\Downarrow\\
Q''(\theta) &= -c^2(A\cos(c\theta) - B\sin(c\theta))
\end{align*}
so we see that $c = k$ therefore the general solution to equation \ref{AngDiffEq} is
$$Q(\theta) = A \sin(k\theta) + B\cos(k\theta)$$
So we can apply the boundary condition from part (c) to find that
\begin{align*}
Q(0) &= Q(2\pi)\\
A\sin(0) + B\cos(0) &= A\sin(k2\pi) + B\cos(k2\pi)\\
B &= A\sin(k2\pi) + B\cos(k2\pi)
\end{align*}
This implies that $\sin(k2\pi) = 0$ and $\cos(k2\pi) = 1$ this is only true if $k$ is an integer so we can see that there are a infinite set of solutions for $Q(\theta)$ where
$$Q_n(\theta) = A\sin(n\theta) + B\cos(n\theta)$$
where $n = 1,2,3,...$. Note that the constant was renamed to $n$ for clarity.

\item
Given that the angular momentum operator is
$$\hat{L} = -i\hbar\partiald{}{\theta}$$
and the Hamiltonian operator 
$$\hat{H} = -\frac{\hbar^2}{2m_e}\frac{1}{r}\partiald{}{r}\left(r\partiald{}{r}\right) - \frac{\hbar^2}{2m_e}\frac{1}{r^2}\partiald{^2}{r^2} + \frac{1}{2}m_e\omega^2r^2$$
we can see if they support simultaneous eigenvalues by testing if they commute so we can see that
\begin{align*}
[\hat{H},\hat{L}] &= \hat{H}\hat{L} - \hat{L}\hat{H}\\
&= -\left(-\frac{\hbar^2}{2m_e}\frac{1}{r}\partiald{}{r}\left(r\partiald{}{r}\right) - \frac{\hbar^2}{2m_e}\frac{1}{r^2}\partiald{^2}{\theta^2} + \frac{1}{2}m_e\omega^2r^2\right)i\hbar\partiald{}{\theta} \\
&\ \ \ \ + i\hbar\partiald{}{\theta}\left(-\frac{\hbar^2}{2m_e}\frac{1}{r}\partiald{}{r}\left(r\partiald{}{r}\right) - \frac{\hbar^2}{2m_e}\frac{1}{r^2}\partiald{^2}{\theta^2} + \frac{1}{2}m_e\omega^2r^2\right)\\
&= \frac{i\hbar^3}{2m_e}\frac{1}{r}\partiald{}{r}\left(r\partiald{^2}{r\partial\theta}\right) + \cancel{\frac{\hbar^2}{2m_e}\frac{1}{r^2}\partiald{^3}{\theta^3} - \frac{i\hbar}{2}m_e\omega^2r^2\partiald{}{\theta}}\\
&\ \ \ \ -\frac{i\hbar^3}{2m_e}\frac{1}{r}\partiald{^2}{\theta\partial r}\left(r\partiald{}{r}\right) - \cancel{\frac{i\hbar^3}{2m_e}\frac{1}{r^2}\partiald{^3}{\theta^3} + \frac{i\hbar}{2}m_e\omega^2r^2\partiald{}{\theta}}\\
&= \frac{i\hbar^3}{2m_e}\frac{1}{r}\left(r\partiald{^3}{r^2\partial\theta} + \partiald{^2}{r\partial\theta} - \partiald{}{\theta}\left(r\partiald{^2}{r^2} + \partiald{}{r}\right)\right)\\
&= \frac{i\hbar^3}{2m_e}\frac{1}{r}\left(r\partiald{^3}{r^2\partial\theta} + \partiald{^2}{r\partial\theta} - r\partiald{^3}{\theta\partial r^2} + \partiald{^2}{\theta\partial r}\right)\\
[\hat{H},\hat{L}] &= 0
\end{align*}
Note that we used the fact that partial derivatives commute that is
$$\partiald{^2}{x\partial y} = \partiald{^2}{y\partial x}$$
therefore we see that $\hat{H}$ and $\hat{L}$ commute which implies that they can have simultaneous eigenvalues.
\end{enumerate}

\section{Problem \#2}
\begin{enumerate}[(a)]

\item
\begin{enumerate}[(i)]

\item
Given the potential 
$$V(x) = \left\{\begin{array}{ll}
                \infty &x<0\\
                +V_0   &0\le x\le a\\
                -V_0   &a\le x\le b\\
                0      &x>b\\
                \end{array}\right.$$
We can see that in the region $x>b$ equation \ref{TISE} for this problem becomes
$$\partiald{^2\psi}{x^2} = -\frac{2mE_0}{\hbar^2}\psi$$
Which is the exact equation for a free particle with energy $E_0$ which has the wave solution
$$\psi_{IV}(x) = Ae^{ik_{IV}x} + B^{-ik_{IV}x}$$
where we define the constant $k_{IV}$ as
$$k_{IV}\equiv \frac{\sqrt{2mE_0}}{\hbar}$$
note that $k_{IV}$ is real since $E_0$ is a positive quantity the subscript denotes that this is in the fourth region left to right.

\item
For the region $0\le x\le a$ we find that equation \ref{TISE} is
$$-\frac{\hbar^2}{2m}\partiald{^2\psi}{x^2} + V_0\psi= E_0\psi$$
which we can rewrite as
\begin{align*}
-\frac{\hbar^2}{2m}\partiald{^2\psi}{x^2} &= E_0\psi - V_0\psi\\
\partiald{^2\psi}{x^2} &= -\frac{2m(E_0 - V_0)}{\hbar^2}\psi\\
\end{align*}
Now we note that $V_0>E_0$ this implies that the solution is not wave like but acts as an exponential decay. So the solution is
$$\psi_{II}(x) = Ae^{-k_{II}x}$$
note that the exponential growth solution was discounted so that the wavefunction could be normalizable. We define $k_{II}$ as
$$k_{II}\equiv\frac{\sqrt{2m(V_0-E_0)}}{\hbar}$$
note that $k_{II}$ is real due to the fact that $V_0>E_0$.

\item
For the region $x<0$ we see that the potential is infinite. Therefore the wave function must be zero in this region
$$\psi_{I} = 0$$
\end{enumerate}

\item
\begin{enumerate}[(i)]

\item
The continuity condition for $\psi(x)$ and $\psi'(x)$ at $x=0$ is
$$\psi_{I}(x=0) = \psi_{II}(x=0)$$
there is no condition on the continuity of $\psi'(x)$ because the potential at $x=0$ is infinite and while the wavefunction has to be continuous everywhere it does not have to change smoothly at an infinite potential.

\item
The continuity condition for $\psi(x)$ and $\psi'(x)$ at $x=a$ is
$$\psi_{II}(x=a) = \psi_{III}(x=a)$$
and
$$\psi_{II}'(x=a) = \psi_{III}'(x=a)$$
Note that both the potentials at $x=a$ are non-infinite therefore both the wavefuction and its first derivative must be continuous.
\end{enumerate}
\end{enumerate}

\section{Problem \#3}
\begin{enumerate}[(a)]
\item
Given the operators
\begin{align*}
\hat{A} &= \left(\begin{array}{cc}
              0     &1+i\\
              1-i   &0  \\
           \end{array}\right)\\
\hat{B} &= \left(\begin{array}{cc}
              1     &0\\
              0     &-1\\
           \end{array}\right)\\
\end{align*}
We can find the eigenvalues for $\hat{A}$ by calculating $\det(A-\lambda1) = 0$. So 
\begin{align*}
0 &= \det\left(\begin{array}{cc}
              -\lambda     &1+i\\
              1-i   &-\lambda  \\
           \end{array}\right)\\
&= \lambda^2 - (1+i)(1-i)\\
&\Downarrow\\
\lambda^2 &= 2\\
&\Downarrow\\
\lambda &= \pm\sqrt{2}
\end{align*}
So if we were to measure $\hat{A}$ we would measure $\sqrt{2}$ or $-\sqrt{2}$.

\item
We can do the same process to find the eigenvalues of $\hat{B}$
\begin{align*}
0 &= \left(\begin{array}{cc}
              1-\lambda     &0\\
              0     &-1-\lambda\\
           \end{array}\right)\\
&=(1-\lambda)(-1-\lambda)\\
&=-(1-\lambda)(1+\lambda)\\
&=(\lambda-1)(\lambda+1)\\
&\Downarrow\\
\lambda &= \pm 1
\end{align*}
So if we were to measure $\hat{B}$ we would measure $1$ or $-1$.

\item
We see that both $\hat{A}$ and $\hat{B}$ are observable operators because they're eigenvalues are all real.

\item
To find out if $\hat{A}$ and $\hat{B}$ are compatible observables we have to test if they commute
\begin{align*}
[\hat{A},\hat{B}] &= \hat{A}\hat{B} - \hat{B}\hat{A}\\
&= \left(\begin{array}{cc}
              0     &1+i\\
              1-i   &0  \\
           \end{array}\right)
   \left(\begin{array}{cc}
              1     &0\\
              0     &-1\\
           \end{array}\right)
-   \left(\begin{array}{cc}
              1     &0\\
              0     &-1\\
           \end{array}\right)
    \left(\begin{array}{cc}
              0     &1+i\\
              1-i   &0  \\
           \end{array}\right)\\
&= \left(\begin{array}{cc}
              0     &-1-i\\
              1-i   &0  \\
           \end{array}\right)
-   \left(\begin{array}{cc}
              0     &1+i\\
              -1+i     &0\\
           \end{array}\right)\\
&= \left(\begin{array}{cc}
              0     &-2-2i\\
              2-2i   &0  \\
           \end{array}\right)
\end{align*}
So we can see that the operators do not commute therefore they are not compatible observables.

\item
The probability of measuring the lowest possible result for $\hat{B}$ after we already measured the lowest possible result for $\hat{A}$ remains at $1/2$. This is due to the fact that the operators are not compatible and once you measure $\hat{A}$ the wavefuction can equally be in either state of $\hat{B}$.
\end{enumerate}

\section{Problem \#4}
\begin{enumerate}[(a)]
\item
We know that the energy level of the state $\psi_{nlm}$ is dependent on the principle quantum number $n$ where
$$E_n = \frac{E_1}{n^2}$$
note that $E_1$ is the ground state energy. So for the wavefunction
$$\psi(r,\theta,\phi,t=0) = \sqrt{1/5}\psi_{100} + \sqrt{2/5}\psi_{210} + \sqrt{2/5}\psi_{211}$$
we can find the expectation energy by
\begin{align*}
\langle E\rangle &= \frac{1}{5}E_1 + \frac{2}{5}E_2 + \frac{2}{5}E_2 \\
&= \frac{1}{5}E_1 + \frac{2}{5}\frac{E_1}{4} + \frac{2}{5}\frac{E_1}{4} \\
&= \left(\frac{1}{5} + \frac{2}{20}+ \frac{2}{20}\right)E_1\\
&= \frac{4}{10}E_1
\end{align*}

\item
Since there is degeneracy for $n=2$ the probability of finding the particle in that energy level is the square sum of both coefficients for states with $n=2$. So the probability is
$$\frac{2}{5} + \frac{2}{5} = \frac{4}{5}$$

\item
After we measured that the energy is $E_2$ we know that the wavefunction collapses into the allowed states so for $t>0$ we wavefunction becomes
$$\psi(r,\theta,\phi,t>0) = \sqrt{1/2}\psi_{210}e^{-iE_2t/\hbar} + \sqrt{1/2}\psi_{211}e^{-iE_2t/\hbar}$$
note that the probabilities are normalized such that it is equally likely to be in the state $m=0$ and $m=1$ and that the time component is included from the solution to the \emph{Time-Dependent Schr\"{o}dinger Equation}.

\item
We can measure $0\hbar$ for the z-component of the angular momentum with a probability of 
$$\frac{1}{5} + \frac{2}{5} = \frac{3}{5}$$
as only the first two states are in the state $m=0$ so only their coefficients are included. Now just after time $t=0$ the normalized wave function becomes
$$\psi(r,\theta,\phi,t>0) = \sqrt{1/3}\psi_{100}e^{-iE_1t/\hbar} + \sqrt{2/3}\psi_{210}e^{-iE_2t/\hbar}$$
Note that it is twice as likely to find the electron in the state $\psi_{210}$ after measuring $m=0$ due to the original weights.

\item
If we measure $L^2$ and get $2\hbar^2$ we can infer from the fact that
$$L^2\ket{nlm} = \hbar l(l+1)\ket{nlm}$$
so this implies that $l(l+1) = 2$ which means that $l = 1$. Note that $l=-2$ is also a solution but we enforce that $l$ is positive. Therefore the probability of finding the electron in the state of $l=1$ is the coefficient squared of $\psi_{210}$ which is $2/3$.
\end{enumerate}

\section{Problem \#5}
\begin{enumerate}[(a)]
\item
The probability density of 
$$\psi(x,t=0) = \frac{1}{\sqrt{2}}\left(i\psi_1(x)+\psi_2(x)\right)$$
can be found by 
\begin{align*}
\psi^*(x,t=0)\psi(x,t=0) &= \frac{1}{2}\left(-i\psi^*_1(x)+\psi^*_2(x)\right)\left(i\psi_1(x)+\psi_2(x)\right)\\
&= \frac{1}{2}\left(\psi^*_1(x)\psi_1(x)+\psi^*_2(x)\psi_2(x)-i\psi^*_1(x)\psi_2(x) + i\psi_1(x)\psi^*_2(x)\right)\\
&= \frac{1}{2}|\psi_1(x)|^2+ \frac{1}{2}|\psi_2(x)|^2
\end{align*}
Note we used the fact that $\psi_n(x) = \psi^*_n(x)$ to cancel the final two terms. Now using the fact that
$$\psi_n(x) = \sqrt{\frac{2}{a}}\sin(n\pi x/a)$$
we find that the probability density is
$$\psi^*(x,t=0)\psi(x,t=0) = \frac{1}{a}\left(\sin^2(\pi x/a)+\sin^2(2\pi x/a)\right)$$
See attached for the sketch of the probability density.


\item
If we make an energy measurement at $t=T$ and find $n=2$ the wave function collapses into the $n=2$ state. So right after the measurement the probability density will look like the probability density for just the $n=2$ state or
$$\psi^*(x,t=T)\psi(x,t=T) = \frac{2}{a}\sin^2(2\pi x/a)$$
See attached for the sketch of the probability density.
\end{enumerate}

\section{Problem \#6}
\begin{enumerate}[(a)]
\item
Given the normalized spin state 
$$\ket{\chi} = \frac{1}{\sqrt{5}}\left(\frac{}{}\ket{\uparrow}_z+2i\ket{\downarrow}_z\right) = \frac{1}{\sqrt{5}}\left(\begin{array}{c}1\\ 2i\end{array}\right)$$
we can see if it is an eigenstate of $\hat{S^2}$ by testing if
$$\hat{S^2}\ket{\chi} = \lambda\ket{\chi}$$
where
$$\hat{S^2} = \frac{3}{4}\hbar^2\left(\begin{array}{cc}
                                       1   &0\\                         
                                       0   &1                         
                                      \end{array}\right)$$
So we find that
\begin{align*}
\hat{S^2}\ket{\chi} &= \frac{3}{4}\hbar^2\left(\begin{array}{cc}
                                       1   &0\\                         
                                       0   &1                         
                                      \end{array}\right)
\frac{1}{\sqrt{5}}\left(\begin{array}{c}1\\ 2i\end{array}\right)\\
&= \frac{3}{4}\hbar^2\frac{1}{\sqrt{5}}\left(\begin{array}{c}1\\ 2i\end{array}\right)\\
&= \frac{3}{4}\hbar^2\ket{\chi}
\end{align*}
so $\ket{\chi}$ is an eigenstate of $\hat{S^2}$ with an eigenvalue of $3/4\hbar^2$
We can see if $\ket{\chi}$ is an eigenstate of $\hat{S}_x$ where
$$\hat{S}_x = \frac{\hbar}{2}\left(\begin{array}{cc}
                                       0   &1\\                        
                                       1   &0                         
                                      \end{array}\right)$$
So we see that
\begin{align*}
\hat{S}_x\ket{\chi} &= \frac{\hbar}{2}\left(\begin{array}{cc}
                                       0   &1\\                       
                                       1   &0                         
                                      \end{array}\right)
\frac{1}{\sqrt{5}}\left(\begin{array}{c}1\\ 2i\end{array}\right)\\
&= \frac{\hbar}{2\sqrt{5}}\left(\begin{array}{c}2i\\ 1\end{array}\right)
\end{align*}
So $\ket{\chi}$ is not an eigenstate of $\hat{S}_x$

\item
If we were to measure the z-component of the angular momentum the possible results are the eigenvalues found by calculating
\begin{align*}
0 &= \det\left(\begin{array}{cc}
                                       \hbar/2-\lambda   &0\\                        
                                       0   &-\hbar/2-\lambda                  
                                      \end{array}\right)\\
&= \left(\lambda + \frac{\hbar}{2}\right)\left(\lambda - \frac{\hbar}{2}\right)
\end{align*}
This implies that the possible results of measuring $\hat{S}_z$ are $\pm\hbar/2$. Now with the given state we can find the probabilities by finding the magnitude of the coefficients. So we would measure $\hbar/2$ with a probability of $1/5$ and we would measure $-\hbar/2$ with a probability of $4/5$.

\item
The expectation is just the sum of the possible outcomes weighted by their probabilities. So we can calculate
\begin{align*}
\langle\hat{S}_z\rangle &= \frac{\hbar}{2}\frac{1}{5} + \frac{-\hbar}{2}\frac{4}{5}\\
&= -\frac{3}{10}\hbar
\end{align*}

\end{enumerate}

\section{Problem \#7}
\begin{enumerate}[(a)]
\item
Given that initially the particle is in the state
$$\psi^i_0(x) = \frac{1}{\pi^{1/4}}\frac{1}{\sqrt{b}}e^{-\frac{x^2}{2b^2}}$$
where 
$$b^2 = \frac{\hbar}{m\omega}$$
once the well is expanded such that $\omega$ decreases to $\Omega$ we can calculate the probability that the particle remains in the ground state by taking the inner product of the original state with the ground state in the new system 
$$\psi^f_0(x) = \frac{1}{\pi^{1/4}}\frac{1}{\sqrt{B}}e^{-\frac{x^2}{2B^2}}$$
where
$$B^2 = \frac{\hbar}{m\Omega}$$
so we can calculate
\begin{align*}
\norm{\psi^i_0(x)}{\psi^f_0(x)} &= \int_{-\infty}^{\infty}\frac{1}{\pi^{1/4}}\frac{1}{\sqrt{b}}e^{-\frac{x^2}{2b^2}}\frac{1}{\pi^{1/4}}\frac{1}{\sqrt{B}}e^{-\frac{x^2}{2B^2}}dx\\
&= \frac{1}{\pi^{1/2}}\frac{1}{\sqrt{b}}\frac{1}{\sqrt{B}}\int_{-\infty}^{\infty}e^{-\frac{x^2}{2b^2}}e^{-\frac{x^2}{2B^2}}dx\\
&= \frac{1}{\pi^{1/2}}\frac{1}{\sqrt{b}}\frac{1}{\sqrt{B}}\int_{-\infty}^{\infty}e^{-\frac{x^2}{2b^2}-\frac{x^2}{2B^2}}dx\\
&= \frac{1}{\pi^{1/2}}\frac{1}{\sqrt{b}}\frac{1}{\sqrt{B}}\int_{-\infty}^{\infty}e^{-\left(\frac{1}{2b^2}-\frac{1}{2B^2}\right)x^2}dx\\
&= \frac{1}{\pi^{1/2}}\frac{1}{\sqrt{b}}\frac{1}{\sqrt{B}}\sqrt{\frac{\pi}{\left(\frac{1}{2b^2}-\frac{1}{2B^2}\right)}}\\
&= \frac{1}{\sqrt{bB\left(\frac{1}{2b^2}-\frac{1}{2B^2}\right)}}
\end{align*}

\item
Given the fact that $\omega = 4\Omega$ we can calculate the probability of the particle being in the ground state of the expanded well by noting that
$$b = \frac{\hbar}{m(4\Omega)} = \frac{1}{4}B$$
so we see that $B = 4b$ so we can calculate the expression we found in part (a)
\begin{align*}
\left(bB\left(\frac{1}{2b^2}-\frac{1}{2B^2}\right)\right)^{-1/2} &= \left(b(4b)\left(\frac{1}{2b^2}-\frac{1}{2(4b)^2}\right)\right)^{-1/2}\\
&= \left(\left(\frac{4b^2}{2b^2}-\frac{4b^2}{32b^2}\right)\right)^{-1/2}\\
&= \left(\left(2-\frac{1}{8}\right)\right)^{-1/2}\\
&= \left(\left(\frac{15}{8}\right)\right)^{-1/2}\\
&= \sqrt{\frac{8}{15}}
\end{align*}

\item
No we would not expect the energy to be conserved due to the fact that the potential changed with time. This implies that the energy is not constant in time so it cannot be conserved.
\end{enumerate}

\end{document}

