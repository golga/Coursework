\documentclass[11pt]{article}

\usepackage{latexsym}
\usepackage{amssymb}
\usepackage{amsthm}
\usepackage{enumerate}
\usepackage{amsmath}
\usepackage{cancel}
\numberwithin{equation}{section}

\setlength{\evensidemargin}{.25in}
\setlength{\oddsidemargin}{-.25in}
\setlength{\topmargin}{-.75in}
\setlength{\textwidth}{6.5in}
\setlength{\textheight}{9.5in}
\newcommand{\due}{September 19th, 2016}
\newcommand{\HWnum}{1}
\newcommand{\grad}{\bold\nabla}
\newcommand{\vecE}{\vec{E}}
\newcommand{\scrptR}{\vec{\mathfrak{R}}}
\newcommand{\kapa}{\frac{1}{4\pi\epsilon_0}}
\newcommand{\emf}{\mathcal{E}}
\newcommand{\unit}[1]{\ensuremath{\, \mathrm{#1}}}
\newcommand{\real}{\textnormal{Re}}
\newcommand{\Erf}{\textnormal{Erf}}
\newcommand{\sech}{\textnormal{sech}}
\newcommand{\scrO}{\mathcal{O}}
\newcommand{\levi}{\widetilde{\epsilon}}
\newcommand{\partiald}[2]{\ensuremath{\frac{\partial{#1}}{\partial{#2}}}}
\newcommand{\norm}[2]{\langle{#1}|{#2}\rangle}
\newcommand{\inprod}[2]{\langle{#1}|{#2}\rangle}
\newcommand{\average}[1]{\left\langle{#1}\right\rangle}
\newcommand{\ket}[1]{|{#1}\rangle}
\newcommand{\bra}[1]{\langle{#1}|}
\newcommand{\Resid}[2]{\ensuremath{\textnormal{Res}\left[{#1},{#2}\right]}}





\begin{document}
\begin{titlepage}
\setlength{\topmargin}{1.5in}
\begin{center}
\Huge{Physics 3310} \\
\LARGE{Principles of Electricity and Magnetism 1} \\
\Large{Professor Thomas R. Schibli} \\[1cm]

\huge{Homework \#\HWnum}\\[0.5cm]

\large{Joe Becker} \\
\large{SID: 810-07-1484} \\
\large{\due} 

\end{center}

\end{titlepage}



\section{Problem \#1}
\begin{enumerate}[(a)]
\item
    For the tensor in three dimensions 
    \begin{equation}
        M_{ij} = \delta_{ij}\cos\alpha + n_{i}n_{j}(1-\cos\alpha) + \epsilon_{ijk}n_k\sin\alpha
        \label{P1Tensor}
    \end{equation}
    where $n_i$ is a unit vector. Given the identity
    \begin{equation}
        \epsilon_{ijm}\epsilon_{klm} = \delta_{ik}\delta_{jl} - \delta_{il}\delta_{jk}
        \label{KronInd}
    \end{equation}
    where $\delta_{ij}$ is the \emph{Kronecker delta} and $\epsilon_{ijk}$ is the \emph{Levi-Civita symbol} we can 
    prove the orthogonality of the tensor given by equation \ref{P1Tensor} by
    \begin{align*}
        M_{ij}M_{ik} &= \left(\delta_{ij}\cos\alpha + n_{i}n_{j}(1-\cos\alpha) + \epsilon_{ijl}n_l\sin\alpha\right)\left(\delta_{ik}\cos\alpha + n_{i}n_{k}(1-\cos\alpha) + \epsilon_{ikm}n_m\sin\alpha\right) \\
                     &= \delta_{ij}\delta_{ik}\cos^2\alpha + n_{i}n_{j}n_{i}n_{k}(1-\cos\alpha)^2 + \epsilon_{ijl}\epsilon_{ikm}n_ln_m\sin^2\alpha \\
                     & \ \ \ \ + \delta_{ij}n_in_k(1-\cos\alpha)\cos\alpha + \delta_{ik}n_in_j(1-\cos\alpha)\cos\alpha \\
                     & \ \ \ \ + \delta_{ij}\epsilon_{ikm}n_m\sin\alpha\cos\alpha + \delta_{ik}\epsilon_{ijl}n_l\sin\alpha\cos\alpha \\
                     & \ \ \ \ +\epsilon_{ikm}n_{i}n_{j}n_m(1-\cos\alpha)\sin\alpha + \epsilon_{ijl}n_in_{k}n_{l}(1-\cos\alpha)\sin\alpha\\
                     &= \delta_{jk}\cos^2\alpha + n_{i}n_{j}n_{i}n_{k}(1-\cos\alpha)^2 + (\delta_{jk}\delta_{lm}-\delta_{jm}\delta_{lk})n_ln_m\sin^2\alpha + 2n_jn_k(1-\cos\alpha)\cos\alpha \\
                     & \ \ \ \ + \epsilon_{jkm}n_m\sin\alpha\cos\alpha + \epsilon_{kjm}n_m\sin\alpha\cos\alpha + \epsilon_{ikm}n_{i}n_{j}n_m(1-\cos\alpha)\sin\alpha + \epsilon_{ijl}n_in_{k}n_{l}(1-\cos\alpha)\sin\alpha\\
                     &= \delta_{jk}\cos^2\alpha + n_{i}n_{j}n_{i}n_{k}(1-\cos\alpha)^2 + (\delta_{jk}n_mn_m-n_jn_k)\sin^2\alpha + 2n_jn_k(1-\cos\alpha)\cos\alpha \\
                     & \ \ \ \ + \cancel{\epsilon_{jkm}n_m\sin\alpha\cos\alpha} - \cancel{\epsilon_{jkm}n_m\sin\alpha\cos\alpha} + \epsilon_{ikm}n_{i}n_{j}n_m(1-\cos\alpha)\sin\alpha + \epsilon_{ijl}n_in_{k}n_{l}(1-\cos\alpha)\sin\alpha \\
                     &= \delta_{jk}\cancel{\cos^2\alpha} + n_{j}n_{k}(1-\cos\alpha)^2 + \delta_{jk}\cancel{\sin^2\alpha}-n_jn_k\sin^2\alpha + 2n_jn_k(1-\cos\alpha)\cos\alpha \\
                     & \ \ \ \ +  \cancelto{0}{\epsilon_{ikm}n_{i}n_{m}}n_j(1-\cos\alpha)\sin\alpha +\cancelto{0}{\epsilon_{ijm}n_in_{m}}n_{k}(1-\cos\alpha)\sin\alpha \\
                     &= \delta_{jk} + n_{j}n_{k} + n_jn_k\cos^2\alpha - n_jn_k\sin^2\alpha - \cancel{2n_jn_k\cos\alpha} +  \cancel{2n_jn_k\cos\alpha} - 2n_jn_k\cos^2\alpha\\
                     &= \delta_{jk} + n_{j}n_{k} - n_jn_k(\cos^2\alpha + \sin^2\alpha)\\
                     &= \delta_{jk}
    \end{align*}
    Note that we used the fact that $n_1$ is a unit vector with implies $n_in_i = 1$ and that 
    $\epsilon_{ijk}n_in_j=0$ by the fact that $\epsilon_{ijk}n_in_j = -\epsilon_{jik}n_in_j$ for all no zero values
    of $\epsilon_{ijk}$ therefore all non-zero terms will cancel within the sum.

\item
    For the special case where the unit vector, $n_i$ points along the $\hat{z}$ direction we note that $n_1=n_2=0$ 
    and $n_3=1$ we  can see the $M_{ij}$ are
    \begin{align*}
        M_{11} &= \delta_{11}\cos\alpha + n_{1}n_{1}(1-\cos\alpha) + \epsilon_{11k}n_k\sin\alpha = \cos\alpha \\
        M_{12} &= \delta_{12}\cos\alpha + n_{1}n_{2}(1-\cos\alpha) + \epsilon_{12k}n_k\sin\alpha = \sin\alpha \\
        M_{13} &= \delta_{13}\cos\alpha + n_{1}n_{3}(1-\cos\alpha) + \epsilon_{13k}n_k\sin\alpha = 0 \\
        M_{21} &= \delta_{21}\cos\alpha + n_{2}n_{1}(1-\cos\alpha) + \epsilon_{21k}n_k\sin\alpha = -\sin\alpha \\
        M_{22} &= \delta_{22}\cos\alpha + n_{2}n_{2}(1-\cos\alpha) + \epsilon_{22k}n_k\sin\alpha = \cos\alpha \\
        M_{23} &= \delta_{23}\cos\alpha + n_{2}n_{3}(1-\cos\alpha) + \epsilon_{23k}n_k\sin\alpha = 0 \\
        M_{31} &= \delta_{31}\cos\alpha + n_{3}n_{1}(1-\cos\alpha) + \epsilon_{31k}n_k\sin\alpha = 0 \\
        M_{32} &= \delta_{32}\cos\alpha + n_{3}n_{2}(1-\cos\alpha) + \epsilon_{32k}n_k\sin\alpha = 0 \\
        M_{33} &= \delta_{33}\cos\alpha + n_{3}n_{3}(1-\cos\alpha) + \epsilon_{33k}n_k\sin\alpha = 1 
    \end{align*}
    Where if we write $M_{ij}$ in matrix form we see that this corresponds to a rotation around the z-axes.
    $$M = \left(\begin{array}{ccc}
            \cos\alpha  &-\sin\alpha    &0\\
            \sin\alpha  &\cos\alpha     &0\\
            0           &0              &1
    \end{array}\right)$$
\end{enumerate}

\pagebreak

\section{Problem \#2}
\begin{enumerate}[(a)]
\item
    Given that any two-index Cartesian 3-tensor, $T_{ij}$ can be viewed as a matrix, $T$, with rows labelled by $i$ 
    columns labelled by $j$, we can rewrite the following equations
    \begin{equation}
        D_{ij} = A_{ik}B_{kl}C_{jl}\qquad D_{ij} = A_{ik}B_{jl}C_{kl}\qquad D_{ij} = A_{ik}(B_{kj}+C_{jk})
        \label{Prob2a}
    \end{equation}
    by noting that $M_{ik}N_{kj}$ represents a standard matrix multiplication and that $M_{ik}N_{jk}$ represents
    matrix multiplication where $N$ is taken as a transpose. This implies that equation \ref{Prob2a} can be written 
    as
    $$D = (A\cdot{B})\cdot{C^T}\qquad D  = (A\cdot{C})\cdot{B^T}\qquad D =   A\cdot({B}+{C^T})$$

\item
    For any $3\times3$ matrix, $W$, with components $W_{ij}$ we can show that 
    \begin{equation}
        W_{il}W_{jm}W_{kn}\epsilon_{lmn} = (\det{W})\epsilon_{ijk}
        \label{Prob2a}
    \end{equation}
    by first proving the antisymmetry of the left hand side. Note if we interchange the indices by
    $i\leftrightarrow{j}$ then we have $W_{jl}W_{im}W_{kn}\epsilon_{lmn}$. Now, now we are free to change the dummy
    indices freely so we can write
    $$W_{jl}W_{im}W_{kn}\epsilon_{lmn} \Rightarrow W_{jm}W_{il}W_{kn}\epsilon_{mln} = -W_{il}W_{jm}W_{kn}\epsilon_{lmn}$$
    Note that this follows for any interchange of $ijk$. For the case where any of the indices $ijk$ are equal we can
    take the sum of all the non-zero values for $lmn$ as
    $$ W_{i1}W_{j2}W_{k3} + W_{i3}W_{j1}W_{k2} + W_{i2}W_{j3}W_{k1} - W_{i3}W_{j2}W_{k1} - W_{i1}W_{j3}W_{k2} - W_{i2}W_{j1}W_{k3}$$
    we can see that if any of the free index $ijk$ are equal then each positive term will have an exact negative term
    which implies that for any $ijk$ equal we have a zero value. Therefore we see that the left hand side is 
    antisymmetric which implies that it must be proportional to $\epsilon_{ijk}$. To find the constant of 
    proportionality we can take a non-zero case where $ijk$ are all different which we can write as
    \begin{align*}
        &W_{11}W_{22}W_{33} + W_{13}W_{21}W_{32} + W_{12}W_{23}W_{31} - W_{13}W_{22}W_{31} - W_{11}W_{23}W_{32} - W_{12}W_{21}W_{33}\\
        &\qquad\qquad\qquad\Downarrow\\
        &W_{11}(W_{22}W_{33}-W_{23}W_{32}) + W_{12}(W_{23}W_{31}-W_{21}W_{33}) + W_{13}(W_{21}W_{32}-W_{22}W_{31}) = \det(W)
    \end{align*}
    and by the antisymmetry we already have shown we see that for any combination of $ijk$ where they are not equal 
    we will have $\pm\det(W)$ therefore we see that equation \ref{Prob2a} is true.

\item
    Given an antisymmetric 4-tensor, $A_{\mu\nu}$ and a symmetric 4-tensor, $S_{\mu\nu}$ we can preform a
    \emph{Lorentz transformation} using $\Lambda^{\mu}_{\ \nu}$ such that
    \begin{align*}
        A_{\mu\nu}' &= \Lambda^{\sigma}_{\ \mu}\Lambda^{\rho}_{\ \nu}A_{\sigma\rho}\\
        S_{\mu\nu}' &= \Lambda^{\sigma}_{\ \mu}\Lambda^{\rho}_{\ \nu}S_{\sigma\rho}
    \end{align*}
    we can see that if we interchange the indices $\mu\leftrightarrow\nu$ we have
    \begin{align*}
        A_{\nu\mu}' &= \Lambda^{\sigma}_{\ \nu}\Lambda^{\rho}_{\ \mu}A_{\sigma\rho}\\
        &= \Lambda^{\rho}_{\ \nu}\Lambda^{\sigma}_{\ \mu}A_{\rho\sigma}\\
        &= \Lambda^{\rho}_{\ \nu}\Lambda^{\sigma}_{\ \mu}(-A_{\sigma\rho})\\
        &= -\Lambda^{\sigma}_{\ \mu}\Lambda^{\rho}_{\ \nu}A_{\sigma\rho} = A_{\mu\nu}'
    \end{align*}
    Therefore the \emph{Lorentz transformation} preserves antisymmetry. The same follows the symmetric tensor, 
    $A_{\mu\nu}$
    \begin{align*}
        S_{\nu\mu}' &= \Lambda^{\sigma}_{\ \nu}\Lambda^{\rho}_{\ \mu}S_{\sigma\rho} = \Lambda^{\rho}_{\ \nu}\Lambda^{\sigma}_{\ \mu}S_{\rho\sigma} = \Lambda^{\rho}_{\ \nu}\Lambda^{\sigma}_{\ \mu}S_{\sigma\rho} = \Lambda^{\sigma}_{\ \mu}\Lambda^{\rho}_{\ \nu}S_{\sigma\rho} = S_{\mu\nu}'
    \end{align*}
    Note that we renamed the dummy indices $\rho$ and $\sigma$.

\end{enumerate}

\pagebreak

\section{Problem \#3}
    Given the constant 4-vector $k_{\mu}$ such that
    \begin{equation}
        \phi\equiv e^{ik_{\mu}x^{\mu}}
        \label{Prob3}
    \end{equation}
    we can find the condition on $k_{\mu}$ that solves the wave equation
    $$\Box\phi=0$$
    where $\Box$ is the \emph{d'Alembertian operator} defined as
    \begin{equation}
        \Box\equiv\partial_{\mu}\partial^{\mu} = -\partial_{0}^{2} + \partial_{i}^{2}
        \label{dAlemb}
    \end{equation}
    So if we apply equation \ref{dAlemb} to equation \ref{Prob3} we find that
    \begin{align*}
        \Box\phi = 0 &= (-\partial_{0}^{2} + \partial_{i}^{2})e^{ik_{\mu}x^{\mu}}\\
                     &= -(ik_0)^2e^{ik_{\mu}} + (ik_1)^2e^{ik_{\mu}}+ (ik_2)^2e^{ik_{\mu}}+ (ik_3)^2e^{ik_{\mu}}\\
                     &= (k_0^2 - k_1^2  - k_2^2 - k_3^2)e^{ik_{\mu}}\\
                     &\Downarrow\\
                   0 &= -k_0^2 + k_1^2  + k_2^2 + k_3^2 \\
                     &\Downarrow\\
        k_{\mu}k^{\mu} &= 0
    \end{align*}
    Therefore the magnitude of $k_{\mu}$ must be zero in order for equation \ref{Prob3} to satisfy the wave equation.

\pagebreak

\section{Problem \#4}
\begin{enumerate}[(a)]
\item
    For the case where we take two successive Lorentz boots along the $x$ axis, one with velocity, $v_1$, and the 
    other with velocity, $v_2$ we use the transformation for a pure boost in $x$ as
    \begin{equation}
        x' = \gamma(x-vt),\qquad y'=y,\qquad z'=z,\qquad t'=\gamma(t-vx)
        \label{PureBoost}
    \end{equation}
   where we define
    \begin{equation} 
        \gamma\equiv\frac{1}{\sqrt{1-v^2}}
        \label{Gamma}
    \end{equation} 
    note that we are using natural units. Using equation \ref{PureBoost} we can see that the first boost at $v_1$ transforms as
    $$x' = \gamma_1(x-v_1t),\qquad y'=y,\qquad z'=z,\qquad t'=\gamma_1(t-v_1x)$$
    Now if we boost again at $v_2$ we have the transformation in $x$
    \begin{align*}
        x'' &= \gamma_2(x'-v_2t') \\
            &= \gamma_2((\gamma_1(x-v_1t)-v_2\gamma_1(t-v_1x)) \\
            &= \gamma_2\gamma_1(x-v_1t-v_2t+v_1v_2x) \\
            &= \gamma_2\gamma_1((1+v_1v_2)x-(v_1+v_2)t) 
    \end{align*}
    and $t$
    \begin{align*}
        t'' &= \gamma_2(t'-v_2x')\\
            &= \gamma_2(\gamma_1(t-v_1x)-v_2\gamma_1(x-v_1t))\\
            &= \gamma_2\gamma_1(t-v_1x-v_2x+v_2v_1t)\\
            &= \gamma_2\gamma_1((1+v_1v_2)t-(v_1+v_2)x)
    \end{align*}
    Now if we repeat the same boost but in the reverse order we first take a boost of $v_2$ which yields 
    $$x' = \gamma_2(x-v_2t),\qquad y'=y,\qquad z'=z,\qquad t'=\gamma_2(t-v_2x)$$
    next we take another boost in $v_1$ which yields a transform in $x$
    \begin{align*}
        x'' &= \gamma_1(x'-v_1t') \\
            &= \gamma_1((\gamma_2(x-v_2t)-v_1\gamma_2(t-v_2x)) \\
            &= \gamma_1\gamma_2(x-v_2t-v_1t+v_1v_2x) \\
            &= \gamma_1\gamma_2((1+v_1v_2)x-(v_1+v_2)t) 
    \end{align*}
    and $t$
    \begin{align*}
        t'' &= \gamma_1(t'-v_2x')\\
            &= \gamma_1(\gamma_2(t-v_2x)-v_2\gamma_2(x-v_2t))\\
            &= \gamma_1\gamma_2(t-v_2x-v_2x+v_2v_1t)\\
            &= \gamma_1\gamma_2((1+v_1v_2)t-(v_1+v_2)x)
    \end{align*}
    We see that two successive boosts of $v_1$ and $v_2$ commute.

\item
    Taking the results from part (a) we see that we can write the total combined boost as
    \begin{align*}
        t'' &= \gamma_1\gamma_2((1+v_1v_2)t-(v_1+v_2)x) \\
            &= \frac{1}{\sqrt{1-v_1^2}}\frac{1}{\sqrt{1-v_2^2}}(1+v_1v_2)\left(t-\frac{v_1+v_2}{1+v_1v_2}x\right) \\
            &= \sqrt{\frac{1+2v_1v_2+(v_1v_2)^2}{1-v_2^2-v_1^2+(v_1v_2)^2}}\left(t-\frac{v_1+v_2}{1+v_1v_2}x\right) \\
            &= \frac{1}{1-((v_1+v_2)/(1+v_1v_2))^2}\left(t-\frac{v_1+v_2}{1+v_1v_2}x\right) 
            &= \frac{1}{1-v_3^2}\left(t-v_3x\right) 
    \end{align*}
    We see that we can write two successive boosts as a single boost with a velocity of $v_3$ where we define $v_3$ with the velocity addition formula given by
    \begin{equation}
        v_3 \equiv \frac{v_1+v_2}{1+v_1v_2}
    \end{equation}

\item
    Now if we take two successive boost, but now we take the first in the $x$ direction, $\vec{v}_1=(v_1,0,0)$, then the second 
    in the $y$ direction, $\vec{v}_2 = (0,v_2,0)$ the first boost transforms as
    $$x' = \gamma_1(x-v_1t),\qquad y'=y,\qquad z'=z,\qquad t'=\gamma_1(t-v_1x)$$
    now the second transformation transforms as 
    $$x'' = \gamma_1(x-v_1t),\qquad y''=\gamma_2(y-v_2t'),\qquad z''=z,\qquad t''=\gamma_1(t'-v_2y)$$
    we note that for $t''$ and $y''$ we need to account for the $x$ boost by
    \begin{align*}
        y'' &= \gamma_2(y-v_2t')  \\
            &= \gamma_2(y-v_2\gamma_1(t-v_1x))  
    \end{align*}
    and for $t''$ we have
    \begin{align*}
        t'' &= \gamma_2(t'-v_2y)  \\
            &= \gamma_2(\gamma_1(t-v_1x)-v_2y)  
    \end{align*}
    Now, if we reverse the boosts we see that the transformation becomes
    $$x' = x,\qquad y'=\gamma_1(y-v_1t),\qquad z'=z,\qquad t'=\gamma_1(t-v_1y)$$
    which in turn makes the second boost into the transformation
    $$x'' = \gamma_2(x-v_2\gamma_1(t-v_1y)),\qquad y''=\gamma_1(y-v_1t),\qquad z''=z,\qquad t''=\gamma_2(\gamma_1(t-v_1y)-v_2x)$$
    We see that these two transformations mix the coordinates $x$ and $y$ into the double boosted frame which 
    depends on the order of the boosts. Therefore the boosts do no commute.

\item
    Given the results from part (c) 
    $$y'' = \gamma_2(y-v_2\gamma_1(t-v_1x)),\qquad t'' = \gamma_2(\gamma_1(t-v_1x)-v_2y)$$
    we see that we cannot write these in form of equation \ref{PureBoost} because we have both $x$ and $y$ coordinates 
    mixed. Therefore no matter what velocity we choose we cannot make place it in the form of a pure boost.
    The only way to make it a pure boost is too mix the $x$ and $y$ coordinates into a new coordinate $y'$ this describes a rotation.

\end{enumerate}

\end{document}

