\documentclass[11pt]{article}

\usepackage{latexsym}
\usepackage{amssymb}
\usepackage{amsthm}
\usepackage{enumerate}
\usepackage{amsmath}
\usepackage{cancel}
\numberwithin{equation}{section}

\setlength{\evensidemargin}{.25in}
\setlength{\oddsidemargin}{-.25in}
\setlength{\topmargin}{-.75in}
\setlength{\textwidth}{6.5in}
\setlength{\textheight}{9.5in}
\newcommand{\due}{November 9th, 2016}
\newcommand{\HWnum}{7}
\newcommand{\grad}{\bold\nabla}
\newcommand{\vecE}{\vec{E}}
\newcommand{\scrptR}{\vec{\mathfrak{R}}}
\newcommand{\kapa}{\frac{1}{4\pi\epsilon_0}}
\newcommand{\emf}{\mathcal{E}}
\newcommand{\unit}[1]{\ensuremath{\, \mathrm{#1}}}
\newcommand{\real}{\textnormal{Re}}
\newcommand{\Erf}{\textnormal{Erf}}
\newcommand{\sech}{\textnormal{sech}}
\newcommand{\scrO}{\mathcal{O}}
\newcommand{\levi}{\widetilde{\epsilon}}
\newcommand{\partiald}[2]{\ensuremath{\frac{\partial{#1}}{\partial{#2}}}}
\newcommand{\norm}[2]{\langle{#1}|{#2}\rangle}
\newcommand{\inprod}[2]{\langle{#1}|{#2}\rangle}
\newcommand{\ket}[1]{|{#1}\rangle}
\newcommand{\bra}[1]{\langle{#1}|}





\begin{document}
\begin{titlepage}
\setlength{\topmargin}{1.5in}
\begin{center}
\Huge{Physics 3320} \\
\LARGE{Principles of Electricity and Magnetism II} \\
\Large{Professor Ana Maria Rey} \\[1cm]

\huge{Homework \#\HWnum}\\[0.5cm]

\large{Joe Becker} \\
\large{SID: 810-07-1484} \\
\large{\due} 

\end{center}

\end{titlepage}



\section{Problem \#1}
\begin{enumerate}[(a)]
\item
    We can solve for the TE modes in a waveguide whose cross section is a rectangle of side $a$ along the $x$ axis
    and side $b$ along the $y$ axis by solving the two dimensional \emph{Helmholtz Equation}
    \begin{equation}
        \partiald{^2\psi}{x^2} + \partiald{^2\psi}{y^2} + \Omega^2\psi = 0.
    \label{Helm}
    \end{equation}
    Note for the TE modes we take $E_z=0$ and $B_z=\psi$ where we apply the boundary condition 
    $$\left.\partiald{\psi}{n}\right|_{S} = 0$$
    where $n$ is normal to the surface $S$ that defines the waveguide cross section. We can solve equation \ref{Helm}
    through the standard separation of variable technique where we take $\psi(x,y) = X(x)Y(y)$ which implies
    \begin{align*}
        \partiald{^2\psi}{x^2} + \partiald{^2\psi}{y^2} + \Omega^2\psi &= 0 \\
                                                                       &\Downarrow\\
        \frac{1}{X(x)}\partiald{^2X}{x^2} + \frac{1}{Y(y)}\partiald{^2Y}{y^2} + \Omega^2 &= 0 
    \end{align*}
    This allows us to take the partial differential equation into two ordinary differential equations
    \begin{align*}
        \partiald{^2X}{x^2} &= k_x^2X(x)\\
        \partiald{^2Y}{y^2} &= k_y^2Y(y)
    \end{align*}
    Where $k_y^2 = \Omega^2-k_x^2$. So we see that we can solve each of these through a combination of sine and 
    cosine.
    \begin{align*}
        X(x) &= A\sin{k_xx} + B\cos{k_xx}\\
        Y(y) &= C\sin{k_yy} + D\cos{k_yy}
    \end{align*}
    Now we apply the boundary conditions, first for $x=0$ we see that we have
    \begin{align*}
        \left.\frac{dX}{dx}\right|_{x=0} = 0 &= Ak_x\cos{0} - Bk_x\sin{0} \Rightarrow A = 0
    \end{align*}
    And that for a non trivial solution for $x=a$ we must have
    \begin{align*}
        \left.\frac{dX}{dx}\right|_{x=a} = 0 &=  -Bk_x\sin{k_xa} \Rightarrow k_x = \frac{m\pi}{a}
    \end{align*}
    The same follows for $y=0$ and $y=b$ which gives us the solution
    $$\psi_{mn}(x,y) = C_{mn}\cos\left(\frac{m\pi}{a}x\right)\cos\left(\frac{n\pi}{b}y\right)$$
    note we combine the constants $B$ and $D$ into $C_{mn}$. This gives us the TE modes
    $$E_{z} = 0\qquad B_z(x,y) = C_{mn}\cos\left(\frac{m\pi}{a}x\right)\cos\left(\frac{n\pi}{b}y\right)$$
    where $\Omega^2_{mn} = m^2\pi^2/a^2 + n^2\pi^2/b^2$.

\item
    Recall that by definition $k^2 = \omega^2-\Omega^2_{mn}$ this relation gives a cutoff frequency in which we have
    a real wave vector. We can see from the solution to part (a) that the smallest value of $\Omega$ that results in
    an oscillating field is when $m=1$ and $n=0$. The $TE_{10}$ mode gives a value for $\Omega$ as
    \begin{align*}
        \Omega &= \frac{\pi}{a}
    \end{align*}
    This implies that $\omega_{\textnormal{min}}^{TE} = \pi/a$
    If we compare this result to the result for the TM mode which is 
    $\omega_{\textnormal{min}}^{TM} = \pi\sqrt{1/a^2+1/b^2}$
    we can see that the ratio of the two cutoff frequencies is
    $$\frac{\omega_{\textnormal{min}}^{TE}}{\omega_{\textnormal{min}}^{TM}} = \left(1+\frac{a^2}{b^2}\right)^{-1/2}$$
\end{enumerate}

\section{Problem \#2}
\begin{enumerate}[(a)]
\item
    For a waveguide made out of an isosceles right-triangle with sides $a$, $a$, and $a\sqrt{2}$ we can use the 
    special case of the rectangular waveguide where $a=b$, but with a linear combination of the eigenfunctions
    of the form $\psi_{mn} + \alpha\psi_{nm}$. This gives the TM solution of the form
    $$E_z = C_{mn}\sin\left(\frac{m\pi}{a}x\right)\sin\left(\frac{n\pi}{a}y\right) + \alpha C_{nm}\sin\left(\frac{n\pi}{a}x\right)\sin\left(\frac{m\pi}{a}y\right)$$
    Note for the legs $x=a$ and $y=0$ we already satisfy the boundary condition $E_z(a,y)=E_z(x,0)=0$. So, we just need
    to satisfy the boundary condition on the hypotenuse which is defined by the line $y=x$. Note for this case we
    have
    \begin{align*}
        C_{mn}\sin\left(\frac{m\pi}{a}x\right)\sin\left(\frac{n\pi}{a}x\right) + \alpha C_{nm}\sin\left(\frac{m\pi}{a}x\right)\sin\left(\frac{n\pi}{a}x\right) &= 0\\
                                                                                                                                                               &\Downarrow\\
        C_{mn} + \alpha C_{nm} &= 0\\
        \alpha &= -\frac{C_{mn}}{C_{nm}}
    \end{align*}
    So we have the solution of the form 
    $$E_z = C_{mn}\left[\sin\left(\frac{m\pi}{a}x\right)\sin\left(\frac{n\pi}{a}x\right) - \sin\left(\frac{m\pi}{a}x\right)\sin\left(\frac{n\pi}{a}x\right)\right]$$

\item
    Note for the general right-triangle we have a solution of the form 
    $$E_z = C_{mn}\sin\left(\frac{m\pi}{a}x\right)\sin\left(\frac{n\pi}{b}y\right) + \alpha C_{nm}\sin\left(\frac{m\pi}{a}y\right)\sin\left(\frac{n\pi}{b}x\right)$$
    Note this follows from the same argument from part (a). We constructed a linear combination of the solutions where
    we take the $E_z(x,y)$ rectangular solution then solve for the case where we swap the coordinates $(E_z(y,x))$. 
    The problem with this approach is that there is no longer the symmetry of the system that is present in the 
    isosceles triangle case. This is why we cannot write the solution as $\psi_{nm}+\alpha\psi_{mn}$.
    This becomes clear when we apply the boundary condition on the line $y=\frac{b}{a}x$ this yields 
    \begin{align*}
        0 &= C_{mn}\sin\left(\frac{m\pi}{a}x\right)\sin\left(\frac{n\pi}{b}y\right) + \alpha C_{nm}\sin\left(\frac{m\pi}{a}y\right)\sin\left(\frac{n\pi}{b}x\right)\\
          &\Downarrow\\
        0 &= C_{mn}\sin\left(\frac{m\pi}{a}x\right)\sin\left(\frac{n\pi}{a}x\right) + \alpha C_{nm}\sin\left(\frac{m\pi b}{a^2}x\right)\sin\left(\frac{n\pi}{b}x\right)
    \end{align*}
    which we cannot for $\alpha$ for any $x$ and $y$.

\item
    Note for the special case where we take a right-triangle with $b=a/\sqrt{3}$ we can verify that the function
    $$\psi_{nm} = \sin\frac{l\pi{x}}{a}\sin\frac{(m-n)\pi{y}}{a\sqrt{3}}
    + \sin\frac{m\pi{x}}{a}\sin\frac{(n-l)\pi{y}}{a\sqrt{3}}
    + \sin\frac{n\pi{x}}{a}\sin\frac{(l-m)\pi{y}}{a\sqrt{3}}$$
    where $l\equiv{-m-n}$ is an eigenfunction of the \emph{Helmholtz Equation}.
    \begin{align*}
        \partiald{^2\psi}{x^2} + \partiald{^2\psi}{y^2} &= -\Omega^2\psi \\
                                                        &\Downarrow\\
        -\Omega^2\psi &= -\frac{\pi^2}{a^2}\left(\left(l^2+\frac{(m-n)^2}{3}\right)\sin\frac{l\pi{x}}{a}\sin\frac{(m-n)\pi{y}}{a\sqrt{3}}\right.\\
                      &\qquad+\left.\left(m^2+\frac{(n-l)^2}{3}\right)\sin\frac{m\pi{x}}{a}\sin\frac{(n-l)\pi{y}}{a\sqrt{3}}\right.\\
                      &\qquad+\left.\left(n^2+\frac{(l-m)^2}{3}\right)\sin\frac{n\pi{x}}{a}\sin\frac{(l-m)\pi{y}}{a\sqrt{3}}\right)\\
                      &= -\frac{\pi^2}{a^2}\left(\left(m^2+n^2+2nm+\frac{m^2+n^2-2mn}{3}\right)\sin\frac{l\pi{x}}{a}\sin\frac{(m-n)\pi{y}}{a\sqrt{3}}\right.\\
                      &\qquad+\left.\left(m^2+\frac{n^2+l^2-2ln}{3}\right)\sin\frac{m\pi{x}}{a}\sin\frac{(n-l)\pi{y}}{a\sqrt{3}}\right.\\
                      &\qquad+\left.\left(n^2+\frac{m^2+l^2-2lm}{3}\right)\sin\frac{n\pi{x}}{a}\sin\frac{(l-m)\pi{y}}{a\sqrt{3}}\right)\\
                      &= -\frac{\pi^2}{a^2}\left(\left(\frac{4m^2+4n^2+4mn}{3}\right)\sin\frac{l\pi{x}}{a}\sin\frac{(m-n)\pi{y}}{a\sqrt{3}}\right.\\
                      &\qquad+\left.\left(\frac{3m^2+n^2+n^2+m^2+2nm+2n(n+m)}{3}\right)\sin\frac{m\pi{x}}{a}\sin\frac{(n-l)\pi{y}}{a\sqrt{3}}\right.\\
                      &\qquad+\left.\left(\frac{3n^2+m^2+n^2+m^2+2nm+2m(n+m)}{3}\right)\sin\frac{n\pi{x}}{a}\sin\frac{(l-m)\pi{y}}{a\sqrt{3}}\right)\\
                      &= -\frac{\pi^2}{a^2}\left(\left(\frac{4m^2+4n^2+4mn}{3}\right)\sin\frac{l\pi{x}}{a}\sin\frac{(m-n)\pi{y}}{a\sqrt{3}}\right.\\
                      &\qquad+\left.\left(\frac{4m^2+4n^2+4nm}{3}\right)\sin\frac{m\pi{x}}{a}\sin\frac{(n-l)\pi{y}}{a\sqrt{3}}\right.\\
                      &\qquad+\left.\left(\frac{4m^2+4n^2+4nm)}{3}\right)\sin\frac{n\pi{x}}{a}\sin\frac{(l-m)\pi{y}}{a\sqrt{3}}\right)\\
                      &= -\frac{4\pi^2}{3a^2}(m^2+n^2+mn)\psi
    \end{align*}
    Therefore we see that $\psi$ is an eigenfunction of the \emph{Helmholtz Equation} with the eigenvalue
    $$\Omega^2_{mn} = -\frac{4\pi^2}{3a^2}(m^2+n^2+mn)$$

\item
    Note using the eigenfunction from part (c) we can verify that the boundary conditions for the TM modes hold for
    each leg of the triangle. For $x=a$ we see that we have
    $$\psi_{nm}(a,y) = \cancelto{0}{\sin{l\pi}}\sin\frac{(m-n)\pi{y}}{a\sqrt{3}}
    + \cancelto{0}{\sin{m\pi}}\sin\frac{(n-l)\pi{y}}{a\sqrt{3}}
    + \cancelto{0}{\sin{n\pi}}\sin\frac{(l-m)\pi{y}}{a\sqrt{3}} = 0$$
    for any integer value of $m$ and $n$. Next for $y = 0$ we see that each term gets a $\sin{0}$ so every term goes
    to zero or $\psi_{nm}(x,0) = 0$. The final boundary condition is on the line $y = x/\sqrt{3}$ this implies that
    \begin{align*}
        \psi_{nm} &= \sin\frac{l\pi{x}}{a}\sin\frac{(m-n)\pi{x}}{3a}
            + \sin\frac{m\pi{x}}{a}\sin\frac{(n-l)\pi{x}}{3a}
            + \sin\frac{n\pi{x}}{a}\sin\frac{(l-m)\pi{x}}{3a}\\
        &= \frac{1}{2}\left[\cos\left(\frac{\pi{x}}{3a}(3l-m+n)\right) - \cos\left(\frac{\pi{x}}{3a}(3l+m-n)\right)\right.\\
        &\qquad+\cos\left(\frac{\pi{x}}{3a}(3m-n+l)\right) - \cos\left(\frac{\pi{x}}{3a}(3m+n-l)\right)\\
        &\qquad+\left.\cos\left(\frac{\pi{x}}{3a}(3n-l+m)\right) - \cos\left(\frac{\pi{x}}{3a}(3n+l-m)\right)\right]\\
        &= \frac{1}{2}\left[\cos\left(\frac{\pi{x}}{3a}(-4m-2n)\right) - \cos\left(\frac{\pi{x}}{3a}(-2m-4n)\right)\right.\\
        &\qquad+\cos\left(\frac{\pi{x}}{3a}(2m-2n)\right) - \cos\left(\frac{\pi{x}}{3a}(4m+2n)\right)\\
        &\qquad+\left.\cos\left(\frac{\pi{x}}{3a}(4n+2m)\right) - \cos\left(\frac{\pi{x}}{3a}(2n-2m)\right)\right]\\
        &= \frac{1}{2}\left[\cos\left(\frac{\pi{x}}{3a}(4m+2n)\right) - \cos\left(\frac{\pi{x}}{3a}(4m+2n)\right)\right.\\
        &\qquad+\cos\left(\frac{\pi{x}}{3a}(2m-2n)\right) - \cos\left(\frac{\pi{x}}{3a}(2m-2n)\right)\\
        &\qquad+\left.\cos\left(\frac{\pi{x}}{3a}(4n+2m)\right) - \cos\left(\frac{\pi{x}}{3a}(4n+2m)\right)\right]\\
        &= 0
    \end{align*}
    So $\psi_{nm}=0$ on all boundaries.
    



\end{enumerate}

\end{document}

