\documentclass[11pt]{article}

\usepackage{latexsym}
\usepackage{amssymb}
\usepackage{amsthm}
\usepackage{enumerate}
\usepackage{amsmath}
\usepackage{cancel}
\numberwithin{equation}{section}

\setlength{\evensidemargin}{.25in}
\setlength{\oddsidemargin}{-.25in}
\setlength{\topmargin}{-.75in}
\setlength{\textwidth}{6.5in}
\setlength{\textheight}{9.5in}
\newcommand{\due}{November 2nd, 2016}
\newcommand{\HWnum}{6}
\newcommand{\grad}{\bold\nabla}
\newcommand{\vecE}{\vec{E}}
\newcommand{\scrptR}{\vec{\mathfrak{R}}}
\newcommand{\kapa}{\frac{1}{4\pi\epsilon_0}}
\newcommand{\emf}{\mathcal{E}}
\newcommand{\unit}[1]{\ensuremath{\, \mathrm{#1}}}
\newcommand{\real}{\textnormal{Re}}
\newcommand{\Erf}{\textnormal{Erf}}
\newcommand{\sech}{\textnormal{sech}}
\newcommand{\scrO}{\mathcal{O}}
\newcommand{\levi}{\widetilde{\epsilon}}
\newcommand{\partiald}[2]{\ensuremath{\frac{\partial{#1}}{\partial{#2}}}}
\newcommand{\norm}[2]{\langle{#1}|{#2}\rangle}
\newcommand{\inprod}[2]{\langle{#1}|{#2}\rangle}
\newcommand{\ket}[1]{|{#1}\rangle}
\newcommand{\bra}[1]{\langle{#1}|}





\begin{document}
\begin{titlepage}
\setlength{\topmargin}{1.5in}
\begin{center}
\Huge{Physics 3320} \\
\LARGE{Principles of Electricity and Magnetism II} \\
\Large{Professor Ana Maria Rey} \\[1cm]

\huge{Homework \#\HWnum}\\[0.5cm]

\large{Joe Becker} \\
\large{SID: 810-07-1484} \\
\large{\due} 

\end{center}

\end{titlepage}



\section{Problem \#1}
For a small test particle with mass, $m$, and positive charge, $q$, which has a \emph{circular} orbit in the $x-y$ 
plane around a fixed positive charge $Q$ at the origin. This is due to the presence of a uniform magnetic field
$B$ oriented along the $z$ direction. Given that the orbit is circular we can say the that the velocity 
of the particle is given by
$$\mathbf{v} = \omega{R}\hat{\theta}$$
where $\omega$ is the angular frequency and $R$ is the radius of the orbit. Note that we are in plane so we can work
within \emph{cylindrical coordinates}. This implies that the \emph{fully relativistic Lorentz force equation}
\begin{equation}
    \frac{d\mathbf{p}}{dt} = q\left(\mathbf{E}+\mathbf{v}\times\mathbf{B}\right)
\label{LorFor}
\end{equation}
can be written in the form 
\begin{align*}
    \frac{d\mathbf{p}}{dt} &= q\left(\mathbf{E}+\mathbf{v}\times\mathbf{B}\right) \\
                           &\Downarrow\\
    \frac{d}{dt}(m\gamma{\mathbf{v}}) &= q\left(\frac{Q}{R^2}\hat{r} + \omega{RB}(\hat{\theta}\times\hat{z})\right)\\
    m\gamma\frac{d\mathbf{v}}{dt} &= q\left(\frac{Q}{R^2} - \omega{RB}\right)\hat{r}
\end{align*}
Note that for circular motion to exist we need a radial acceleration of the form $d\mathbf{v}/dt = -\omega^2R\hat{r}$ this 
implies that
\begin{align*}
    -m\gamma\omega^2R &= q\left(\frac{Q}{R^2} - \omega{RB}\right)\\
                     &\Downarrow\\
    \frac{q}{m} &= \frac{-\gamma\omega^2R}{\left(\frac{Q}{R^2} - \omega{RB}\right)}\\
                &= \frac{-\omega^2R}{\sqrt{1-(\omega{R})^2}}\frac{1}{\frac{Q}{R^2} - \omega{RB}}\\
                &= \frac{\omega^2}{\sqrt{1-(\omega{R})^2}}\frac{1}{\omega{B} - \frac{Q}{R^3}}
\end{align*}
Note that $\gamma = (1-v^2)^{-1/2} = (1-(\omega{R})^2)^{-1/2}$.

\pagebreak

\section{Problem \#2}
\begin{enumerate}[(a)]
\item
    For an electromagnetic wave for which the electric field is given as
    \begin{equation}
        \mathbf{E} = E_0\sin\omega{t}\left(\sin\omega{z},\cos\omega{z},0\right)
    \label{Prob2a}
    \end{equation}
    where $E_0$ and $\omega$ are constant we can solve for $\mathbf{B}$ by noting the maxwell equation
    \begin{align*}
        -\partiald{\mathbf{B}}{t} &= \grad\times\mathbf{E} \\
                                  &= \det\left|\begin{array}{ccc}
                                                \hat{x}                       &\hat{y}                        &\hat{z}\\
                                                \partial_{x}                  &\partial_{y}                   &\partial_{z}\\
                                                E_0\sin\omega{t}\sin\omega{z} &E_0\sin\omega{t}\cos\omega{z} &0
                                          \end{array}\right|\\
                                   &= (E_0\omega\sin\omega{t}\sin\omega{z}, E_0\omega\sin\omega{t}\cos\omega{z}, 0)\\
                                   &\Downarrow\\
                                          \mathbf{B} &= E_0\cos\omega{t}\left(\sin\omega{z}, \cos\omega{z}, 0\right)
    \end{align*}
    Note that this satisfies one of the four source-free Maxwell equation. The remaining three are
    $$\grad\cdot\mathbf{E}=0\qquad \grad\cdot\mathbf{B} = 0\qquad \grad\times\mathbf{B} = \partiald{\mathbf{E}}{t}$$
    We can quickly verify the first two equations by calculating
    \begin{align*}
        \grad\cdot\mathbf{E} &= E_0\sin\omega{t}\left( \cancelto{0}{\partiald{}{x}(\sin\omega{z})} + \cancelto{0}{\partiald{}{y}(\cos\omega{z})} + \partiald{}{z}(0)\right) = 0\\
        \grad\cdot\mathbf{B} &= E_0\cos\omega{t}\left( \cancelto{0}{\partiald{}{x}(\sin\omega{z})} + \cancelto{0}{\partiald{}{y}(\cos\omega{z})} + \partiald{}{z}(0)\right) = 0
    \end{align*}
    and for the last we equation we calculate 
    \begin{align*}
        \grad\times\mathbf{B} - \partiald{\mathbf{E}}{t} &= \det\left|\begin{array}{ccc}
                                                \hat{x}                       &\hat{y}                        &\hat{z}\\
                                                \partial_{x}                  &\partial_{y}                   &\partial_{z}\\
                                                E_0\cos\omega{t}\sin\omega{z} &E_0\cos\omega{t}\cos\omega{z} &0 
    \end{array}\right| - \partiald{}{t}(E_0\sin\omega{t})\left(\sin\omega{z},\cos\omega{z},0\right)\\
    &= E_0\omega\cos\omega{t}(\sin\omega{z},\cos\omega{z},0) - E_0\omega\cos\omega{t}\left(\sin\omega{z},\cos\omega{z},0\right) = 0
    \end{align*}
    Therefore all the source-free Maxwell equations hold.

\item
    For the electromagnetic wave in part (a) we can calculate the energy density, $W$, by
    \begin{align*}
        W = \frac{1}{8\pi}\left(E^2+B^2\right) &= \frac{1}{8\pi}\left(E_0^2\sin^2\omega{t}\cancelto{1}{(\sin^2\omega{z}+\cos^2\omega{z})} + E_0^2\cos^2\omega{t}\cancelto{1}{(\cos^2\omega{z}+\sin^2\omega{z})}\right)\\
                                               &= \frac{1}{8\pi}\left(E_0^2\cancelto{1}{(\sin^2\omega{t}+\cos^2\omega{t})}\right)\\
                                               &= \frac{E_0^2}{8\pi}
    \end{align*}
    We can also calculate the \emph{Poynting vector}, $\mathbf{S}$, as
    \begin{align*}
        \mathbf{S} = \frac{1}{4\pi}\mathbf{E}\times\mathbf{B} &= \frac{1}{4\pi}\det\left|\begin{array}{ccc}
                                                \hat{x}                       &\hat{y}                       &\hat{z}\\
                                                E_0\sin\omega{t}\sin\omega{z} &E_0\sin\omega{t}\cos\omega{z} &0      \\
                                                E_0\cos\omega{t}\sin\omega{z} &E_0\cos\omega{t}\cos\omega{z} &0 
                                                                    \end{array}\right|\\
                                                                    &= 0\hat{x} + 0\hat{y} + \frac{E_0^2}{4\pi}(\sin\omega{t}\cos\omega{t}\sin\omega{z}\cos\omega{z}-\sin\omega{t}\cos\omega{t}\sin\omega{z}\cos\omega{z})\hat{z}\\
                                                &=0
    \end{align*}

\item
    We can repeat the process from parts (a) and (b) for the electric field
    \begin{equation}
        \mathbf{E} = E_0\cos\omega{z}\left(\cos\omega{t},-\sin\omega{t},0\right)
    \label{Prob2c}
    \end{equation}
    which gives us 
    \begin{align*}
        -\partiald{\mathbf{B}}{t} &= \grad\times\mathbf{E} \\
                                  &= \det\left|\begin{array}{ccc}
                                                \hat{x}                       &\hat{y}                        &\hat{z}\\
                                                \partial_{x}                  &\partial_{y}                   &\partial_{z}\\
                                                E_0\cos\omega{t}\cos\omega{z} &-E_0\sin\omega{t}\cos\omega{z} &0
                                          \end{array}\right|\\
                                   &= (-E_0\omega\sin\omega{t}\sin\omega{z}, -E_0\omega\cos\omega{t}\sin\omega{z}, 0)\\
                                   &\Downarrow\\
                                          \mathbf{B} &= E_0\sin\omega{z}\left(\cos\omega{t}, -\sin\omega{t}, 0\right)
    \end{align*}
\end{enumerate}
   which allows us to calculate 
   \begin{align*}
       \grad\cdot\mathbf{E} &= E_0\left(\cos\omega{t}\cancelto{0}{\partiald{}{x}(\cos\omega{z})} - \sin\omega{t}\cancelto{0}{\partiald{}{y}(\cos\omega{z})} + \partiald{}{z}(0)\right) = 0\\
       \grad\cdot\mathbf{B} &= E_0\left( \cos\omega{t}\cancelto{0}{\partiald{}{x}(\sin\omega{z})} - \sin\omega{t}\cancelto{0}{\partiald{}{y}(\sin\omega{z})} + \partiald{}{z}(0)\right) = 0
   \end{align*}
   and
    \begin{align*}
        \grad\times\mathbf{B} - \partiald{\mathbf{E}}{t} &= \det\left|\begin{array}{ccc}
                                                \hat{x}                       &\hat{y}                        &\hat{z}\\
                                                \partial_{x}                  &\partial_{y}                   &\partial_{z}\\
                                                E_0\cos\omega{t}\sin\omega{z} &-E_0\sin\omega{t}\sin\omega{z} &0 
    \end{array}\right| - \partiald{}{t}(E_0\cos\omega{z})\left(\cos\omega{t},-\sin\omega{t},0\right)\\
    &= E_0\omega\cos\omega{z}(-\sin\omega{t},-\cos\omega{t},0) - E_0\omega\cos\omega{z}\left(-\sin\omega{t},-\cos\omega{t},0\right) = 0
    \end{align*}
    We also calculate 
    \begin{align*}
        W = \frac{1}{8\pi}\left(E^2+B^2\right) &= \frac{1}{8\pi}\left(E_0^2\cos^2\omega{z}\cancelto{1}{(\sin^2\omega{t}+\cos^2\omega{t})} + E_0^2\cos^2\omega{z}\cancelto{1}{(\cos^2\omega{t}+\sin^2\omega{t})}\right)\\
                                               &= \frac{1}{8\pi}\left(E_0^2\cancelto{1}{(\sin^2\omega{z}+\cos^2\omega{z})}\right)\\
                                               &= \frac{E_0^2}{8\pi}
    \end{align*}
    and 
    \begin{align*}
        \mathbf{S} = \frac{1}{4\pi}\mathbf{E}\times\mathbf{B} &= \frac{1}{4\pi}\det\left|\begin{array}{ccc}
                                                \hat{x}                       &\hat{y}                       &\hat{z}\\
                                                E_0\cos\omega{t}\cos\omega{z} &-E_0\sin\omega{t}\cos\omega{z} &0 \\
                                                E_0\cos\omega{t}\sin\omega{z} &-E_0\sin\omega{t}\sin\omega{z} &0 
                                                                    \end{array}\right|\\
                                                                    &= 0\hat{x} + 0\hat{y} + \frac{E_0^2}{4\pi}(\sin\omega{t}\cos\omega{t}\sin\omega{z}\cos\omega{z}-\sin\omega{t}\cos\omega{t}\sin\omega{z}\cos\omega{z})\hat{z}\\
                                                &=0
    \end{align*}

\section{Problem \#3}
\begin{enumerate}[(a)]
\item
    For constant $\mathbf{E}$ and $\mathbf{B}$ fields we can show that in general it is possible to find a new
    \emph{Lorentz frame} related to the original by a boost velocity of the form
    $$\mathbf{v} = \lambda\mathbf{E}\times\mathbf{B}$$
    such that in the boosted frame $\mathbf{E}'$ and $\mathbf{B}'$ are parallel. We can do this by taking the boosted
    field equations 
    \begin{align*}
        \mathbf{E}' &= \gamma(\mathbf{E}+\mathbf{v}\times\mathbf{B}) - \frac{\gamma-1}{v^2}(\mathbf{v}\cdot\mathbf{E})\mathbf{v}\\
        \mathbf{B}' &= \gamma(\mathbf{B}-\mathbf{v}\times\mathbf{E}) - \frac{\gamma-1}{v^2}(\mathbf{v}\cdot\mathbf{B})\mathbf{v}
    \end{align*}
    Now if we replace $\mathbf{v}$ with the velocity we were given we find
    \begin{align*}
        \mathbf{E}' &= \gamma(\mathbf{E}+(\lambda\mathbf{E}\times\mathbf{B})\times\mathbf{B}) - \frac{\gamma-1}{v^2}\cancelto{0}{((\lambda\mathbf{E}\times\mathbf{B})\cdot\mathbf{E})}(\lambda\mathbf{E}\times\mathbf{B})\\
                    &= \gamma\left(\mathbf{E} + \lambda\left((\mathbf{E}\cdot\mathbf{B})\mathbf{B}-(\mathbf{B}\cdot\mathbf{B})\mathbf{E}\right)\right)\\
                    &= \gamma\left((1-\lambda{B^2})\mathbf{E} + \lambda(\mathbf{E}\cdot\mathbf{B})\mathbf{B})\right)
    \end{align*}
    And similarly we find
    \begin{align*}
        \mathbf{B}' &= \gamma(\mathbf{B}-(\lambda\mathbf{E}\times\mathbf{B})\times\mathbf{E}) - \frac{\gamma-1}{v^2}\cancelto{0}{((\lambda\mathbf{E}\times\mathbf{B})\cdot\mathbf{B})}(\lambda\mathbf{E}\times\mathbf{B})\\
                    &= \gamma\left(\mathbf{B} - \lambda\left((\mathbf{E}\cdot\mathbf{E})\mathbf{B}-(\mathbf{B}\cdot\mathbf{E})\mathbf{E}\right)\right)\\
                    &= \gamma\left((1-\lambda{E^2})\mathbf{B} + \lambda(\mathbf{E}\cdot\mathbf{B})\mathbf{E})\right)
    \end{align*}
    This allows us to calculate
    \begin{align*}
        \mathbf{E}'\times\mathbf{B} &= \gamma^2\left((1-\lambda{B^2})(1-\lambda{E^2})\mathbf{E}\times\mathbf{B} + \lambda^2(\mathbf{E}\cdot\mathbf{B})^2\mathbf{B}\times\mathbf{E}\right)\\
        &= \gamma^2\left((1-\lambda{B^2})(1-\lambda{E^2}) - \lambda^2(\mathbf{E}\cdot\mathbf{B})^2\right)\mathbf{B}\times\mathbf{E}
    \end{align*}
    Therefore we see that $\mathbf{E}'$ and $\mathbf{B}'$ are parallel if
    \begin{align*}
        (1-\lambda{B^2})(1-\lambda{E^2}) - \lambda^2(\mathbf{E}\cdot\mathbf{B})^2 &= 0\\
                                     &\Downarrow\\
        \lambda = \frac{(E^2+B^2) \pm \sqrt{E^4+B^4+4(\mathbf{E}\cdot\mathbf{B})^2 - 2E^2B^2}}{-2(\mathbf{E}\cdot\mathbf{B})^2-B^2E^2}
    \end{align*}
    Note this exists for any $\mathbf{E}$ and $\mathbf{B}$.

\item
    If $\mathbf{E}\cdot\mathbf{B}=0$ and $|\mathbf{E}|=|\mathbf{B}|$ are true then we can see that $\mathbf{E}$ is perpendicular in all 
    reference frames due to the fact that $\mathbf{E}\cdot\mathbf{B}$ is \emph{Lorentz invariant}. 
    Therefore it is not possible to construct a frame in with $\mathbf{E}$ and $\mathbf{B}$ are parallel. 
    Note this follows from the result from part (a) as $\lambda$ goes to infinity in this case.


\end{enumerate}

\end{document}

