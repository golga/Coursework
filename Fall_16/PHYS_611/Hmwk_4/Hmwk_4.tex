\documentclass[11pt]{article}

\usepackage{latexsym}
\usepackage{amssymb}
\usepackage{amsthm}
\usepackage{enumerate}
\usepackage{amsmath}
\usepackage{cancel}
\usepackage{mathtools}
\numberwithin{equation}{section}

\setlength{\evensidemargin}{.25in}
\setlength{\oddsidemargin}{-.25in}
\setlength{\topmargin}{-.75in}
\setlength{\textwidth}{6.5in}
\setlength{\textheight}{9.5in}
\newcommand{\due}{October 19th, 2016}
\newcommand{\HWnum}{4}
\newcommand{\grad}{\bold\nabla}
\newcommand{\vecE}{\vec{E}}
\newcommand{\scrptR}{\vec{\mathfrak{R}}}
\newcommand{\kapa}{\frac{1}{4\pi\epsilon_0}}
\newcommand{\emf}{\mathcal{E}}
\newcommand{\unit}[1]{\ensuremath{\, \mathrm{#1}}}
\newcommand{\real}{\textnormal{Re}}
\newcommand{\Erf}{\textnormal{Erf}}
\newcommand{\sech}{\textnormal{sech}}
\newcommand{\scrO}{\mathcal{O}}
\newcommand{\levi}{\widetilde{\epsilon}}
\newcommand{\partiald}[2]{\ensuremath{\frac{\partial{#1}}{\partial{#2}}}}
\newcommand{\norm}[2]{\langle{#1}|{#2}\rangle}
\newcommand{\inprod}[2]{\langle{#1}|{#2}\rangle}
\newcommand{\ket}[1]{|{#1}\rangle}
\newcommand{\bra}[1]{\langle{#1}|}





\begin{document}
\begin{titlepage}
\setlength{\topmargin}{1.5in}
\begin{center}
\Huge{Physics 3320} \\
\LARGE{Principles of Electricity and Magnetism II} \\
\Large{Professor Ana Maria Rey} \\[1cm]

\huge{Homework \#\HWnum}\\[0.5cm]

\large{Joe Becker} \\
\large{SID: 810-07-1484} \\
\large{\due} 

\end{center}

\end{titlepage}



\section{Problem \#1}
\begin{enumerate}[(a)]
\item
    Given the energy-momentum tensor for the electromagnetic field
    \begin{equation}
        T_{\mu\nu} = \frac{1}{8\pi}\left(F_{\mu\rho}F^{\ \rho}_{\nu} + \prescript{*}{}{F}_{\mu\rho}\prescript{*}{}{F}^{\ \rho}_{\nu}\right)
    \label{EnMom}
    \end{equation}
    where $\prescript{*}{}{F}_{\mu\nu}=\frac{1}{2}\epsilon_{\mu\nu\rho\sigma}F^{\rho\sigma}$. We can see that 
    equation \ref{EnMom} can be reduced by noting that
    \begin{align*}
        \prescript{*}{}{F}_{\mu\rho}\prescript{*}{}{F}^{\ \rho}_{\nu} &= \left(\frac{1}{2}\epsilon_{\mu\rho\pi\sigma}F^{\pi\sigma}\right)\eta_{\nu\lambda}\prescript{*}{}{F}^{\lambda\rho}\\
            &= \left(\frac{1}{2}\epsilon_{\mu\rho\pi\sigma}F^{\pi\sigma}\right)\eta_{\nu\lambda}\frac{1}{2}\epsilon^{\lambda\rho\phi\omega}F_{\phi\omega}\\
            &= \frac{1}{4}\eta_{\nu\lambda}F^{\pi\sigma}F_{\phi\omega}\left(-\delta^{\lambda}_{\mu}\delta^{\phi}_{\pi}\delta^{\omega}_{\sigma}  - \delta^{\phi}_{\mu}\delta^{\omega}_{\pi}\delta^{\lambda}_{\sigma} - \delta^{\omega}_{\mu}\delta^{\lambda}_{\pi}\delta^{\phi}_{\sigma} + \delta^{\phi}_{\mu}\delta^{\lambda}_{\pi}\delta^{\omega}_{\sigma} + \delta^{\lambda}_{\mu}\delta^{\omega}_{\pi}\delta^{\phi}_{\sigma} + \delta^{\omega}_{\mu}\delta^{\phi}_{\pi}\delta^{\lambda}_{\sigma}\right)\\
            &= \frac{1}{4}\left(-\eta_{\nu\mu}F^{\pi\sigma}F_{\pi\sigma}
                - \eta_{\nu\sigma}F^{\pi\sigma}F_{\mu\pi}
                - \eta_{\nu\pi}F^{\sigma\pi}F_{\mu\sigma}
                + \eta_{\nu\pi}F^{\pi\sigma}F_{\mu\sigma}
                + \eta_{\nu\mu}F^{\pi\sigma}F_{\sigma\pi}
                + \eta_{\nu\sigma}F^{\pi\sigma}F_{\pi\mu} \right)\\
                &= \frac{1}{4}\left(-2\eta_{\mu\nu}F^{\rho\sigma}F_{\rho\sigma} + 4\eta_{\nu\sigma}F^{\sigma\rho}F_{\mu\rho}\right)\\
                &= -\frac{1}{2}\left(\eta_{\mu\nu}F^{\rho\sigma}F_{\rho\sigma} - 2F^{\ \rho}_{\nu}F_{\mu\rho}\right)\\
    \end{align*}
    We see that equation \ref{EnMom} becomes
    \begin{align*}
        T_{\mu\nu} &= \frac{1}{8\pi}\left(F_{\mu\rho}F^{\ \rho}_{\nu} + \prescript{*}{}{F}_{\mu\rho}\prescript{*}{}{F}^{\ \rho}_{\nu}\right)\\
                   &\Downarrow\\
                   &= \frac{1}{8\pi}\left(F_{\mu\rho}F^{\ \rho}_{\nu} - \frac{1}{2}\eta_{\mu\nu}F^{\rho\sigma}F_{\rho\sigma} + F^{\ \rho}_{\nu}F_{\mu\rho}\right)\\
                   &= \frac{1}{4\pi}\left(F_{\mu\rho}F^{\ \rho}_{\nu} - \frac{1}{4}\eta_{\mu\nu}F^{\rho\sigma}F_{\rho\sigma} \right)\\
    \end{align*}
    Note this is the typical form of the electromagnetic field energy momentum tensor.

\item
    We can find the expression for $T_{\mu\rho}T^{\nu\rho}$ by noting that
    \begin{align*}
        T^{\nu\rho} &= \eta^{\nu\sigma}\eta^{\rho\lambda}T_{\sigma\lambda}\\
                    &= \frac{1}{4\pi}\left(\eta^{\nu\sigma}\eta^{\rho\lambda}F_{\sigma\phi}F_{\lambda}^{\ \phi} - \frac{1}{4}\eta^{\nu\sigma}\eta^{\rho\lambda}\eta_{\sigma\lambda}F^{\phi\omega}F_{\phi\omega}\right)\\
                    &= \frac{1}{4\pi}\left(\eta^{\nu\sigma}\eta^{\rho\lambda}F_{\sigma\phi}F_{\lambda}^{\ \phi} - \frac{1}{4}\eta^{\nu\rho}F^{\sigma\lambda}F_{\sigma\lambda}\right)
    \end{align*}
    So we can calculate defining $F^2\equiv F^{\mu\nu}F_{\mu\nu}$
    \begin{align*}
        T_{\mu\rho}T^{\nu\rho} &= \frac{1}{(4\pi)^2}\left(F_{\mu\sigma}F^{\ \sigma}_{\rho} - \frac{1}{4}\eta_{\mu\rho}F^{\sigma\lambda}F_{\sigma\lambda} \right)\left(\eta^{\nu\sigma}\eta^{\rho\lambda}F_{\sigma\phi}F_{\lambda}^{\ \phi} - \frac{1}{4}\eta^{\nu\rho}F^{\sigma\lambda}F_{\sigma\lambda}\right)\\
                               &= \frac{1}{(4\pi)^2}\left(\eta^{\nu\sigma}\eta^{\rho\lambda}F_{\sigma\phi}F_{\lambda}^{\ \phi}F_{\mu\omega}F^{\ \omega}_{\rho} + \frac{1}{16}\eta_{\mu\rho}\eta^{\nu\rho}(F^2)^2 - \frac{1}{4}F^2\eta^{\nu\rho}F_{\mu\sigma}F^{\ \sigma}_{\rho} - \frac{1}{4}F^2\eta_{\mu\rho}\eta^{\nu\sigma}\eta^{\rho\lambda}F_{\sigma\phi}F_{\lambda}^{\ \phi}\right)\\
                               &= \frac{1}{(4\pi)^2}\left(\eta^{\nu\sigma}\eta^{\rho\lambda}F_{\sigma\phi}F_{\lambda}^{\ \phi}F_{\mu\omega}F^{\ \omega}_{\rho} + \frac{1}{16}\delta^{\nu}_{\mu}(F^2)^2 - \frac{1}{4}F^2\eta^{\nu\rho}F_{\mu\sigma}F^{\ \sigma}_{\rho} - \frac{1}{4}F^2\eta^{\nu\rho}F_{\rho\sigma}F_{\mu}^{\ \sigma}\right)
    \end{align*}
    
    Next we consider the term 
    \begin{align*}
        \eta^{\nu\sigma}\eta^{\rho\lambda}F_{\sigma\phi}F_{\lambda}^{\ \phi}F_{\mu\omega}F^{\ \omega}_{\rho} &= \eta^{\nu\sigma}\eta^{\rho\lambda}F_{\sigma\phi}F_{\lambda}^{\ \phi}F_{\mu\omega}F^{\ \omega}_{\rho}\\
    \end{align*}






\end{enumerate}

\pagebreak

\section{Problem \#2}
\begin{enumerate}[(a)]
\item
    Given the Lagrangian density
    \begin{equation}
        \mathcal{L} = -\frac{1}{16}F^{\mu\nu}F_{\mu\nu} - \frac{m^2}{8\pi}A^{\mu}A_{\mu}+J^{\mu}A_{\mu}
        \label{LagDen}
    \end{equation}
    we can derive the equations of motion from the \emph{Euler-Lagrange equations}
    \begin{equation}
        \partiald{\mathcal{L}}{A_{\mu}} - \partial_{\nu}\left(\partiald{\mathcal{L}}{(\partial_{\nu}A_{\mu})}\right) = 0
        \label{EulLag}
    \end{equation}
    Noting that $F_{\mu\nu}\equiv\partial_{\mu}A_{\nu}-\partial_{\nu}A_{\mu}$ we have
    \begin{align*}
        \partiald{\mathcal{L}}{A_{\mu}} &= -\frac{m^2}{8\pi}A^{\mu} + J^{\mu}\\
        \partiald{\mathcal{L}}{(\partial_{\nu}A_{\mu})} &= \partiald{}{(\partial_{\nu}A_{\mu})}\left(-\frac{1}{16}F^{\mu\nu}(\partial_{\mu}A_{\nu}-\partial_{\nu}A_{\mu})\right) = \frac{1}{4}F^{\mu\nu}
    \end{align*}
    So equation \ref{EulLag} yields the equation of motion
    $$-\partial_{\nu}F^{\mu\nu} + m^2A^{\mu} = 4\pi{J^{\mu}}$$

\item
    Using the result from part (a) we can find the solution for the scalar potential $\phi\equiv{A^{0}}$ for a point
    charge $q$ located at the origin. This implies that we have $J^{0} = q\delta^{3}(\mathbf{r})$ which allows us to 
    solve for $\mu=0$
    \begin{align*}
        -\partial_{\nu}F^{0\nu} + m^2A^{0} &= 4\pi q\delta^{3}(\mathbf{r})\\
                                           &\Downarrow\\
        -\grad^2\phi + m^2\phi &= 4\pi q\delta^{3}(\mathbf{r})\\
                                           &\Downarrow\\
        -\frac{1}{r^2}\partiald{}{r}\left(r^2\partiald{\phi}{r}\right) + m^2\phi &= 4\pi q\delta(r)\\
        -\partiald{}{r}\left(r^2\partiald{\phi}{r}\right) + r^2m^2\phi &= 4\pi r^2q\delta(r)\\
                                           &\Downarrow\\
                                  r\phi(r) &= Ce^{-mr}
    \end{align*}
    Where we can solve for the constant $C$ by noting that for \emph{Gauss' law} to hold $C=q$ therefore
    $$\phi(r) = \frac{qe^{-mr}}{r}$$

\end{enumerate}

\pagebreak

\section{Problem \#3}
\begin{enumerate}[(a)]
\item
    Given the potentials describing an electric charge, $e$, moving with constant velocity, $\mathbf{v}$ 
    $$\phi(\mathbf{r},t) = \frac{e\gamma}{r'},\qquad \mathbf{A}(\mathbf{r},t) = \mathbf{v}\phi$$
    we note that the magnetic field is given as
    \begin{align*}
        \mathbf{B} &= \grad\times\mathbf{A}\\
                   &= \grad\times(\mathbf{v}\phi)\\
                   &= \grad\phi\times\mathbf{v} + \phi(\cancelto{0}{\grad\times\mathbf{v}})\\
                   &= -\mathbf{E}\times\mathbf{v}\\
                   &= \mathbf{v}\times\mathbf{E}
    \end{align*}

\item
    Using the expressions we were given in part (a) we can find the electric field by
    \begin{align*}
        \mathbf{E} &= -\grad\phi - \partiald{\mathbf{A}}{t}\\
                   &= -\grad\frac{e\gamma}{r'} - \mathbf{v}\partiald{}{t}\frac{e\gamma}{r'}
    \end{align*}
    we note that we can assume that $\mathbf{v}$ is pointing along the $\hat{x}$ direction this gives us
    $$r' = \gamma\sqrt{(x-vt)^2+y^2+z^2}$$
    which allows us to calculate
    \begin{align*}
        \partiald{r'}{t} &= \gamma\partiald{}{t}\left(\sqrt{(x-vt)^2+y^2+z^2}\right) = \gamma\left((x-vt)^2+y^2+z^2\right)^{-1/2}2(x-vt)(-v)\\
                         &= -\frac{2\gamma^2(x-vt)v}{r'}\\
        \partiald{r'}{x} &= \frac{2\gamma^2(x-vt)}{r'}\\
        \partiald{r'}{y} &= \frac{2\gamma^2y}{r'}\\
        \partiald{r'}{z} &= \frac{2\gamma^2z}{r'}
    \end{align*}
    So this gives the resulting electric field
    \begin{align*}
        \mathbf{E} &= -\grad\frac{e\gamma}{r'} - \mathbf{v}\partiald{}{t}\frac{e\gamma}{r'}\\
        &= \frac{e\gamma}{r'2}\grad{r'} + v\hat{x}\frac{e\gamma}{r'^2}\\
    \end{align*}
    


\end{enumerate}

\end{document}

