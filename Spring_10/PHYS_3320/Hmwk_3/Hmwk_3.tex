\documentclass[11pt]{article}

\usepackage{latexsym}
\usepackage{amssymb}
\usepackage{amsthm}
\usepackage{enumerate}
\usepackage{amsmath}
\usepackage{cancel}
\numberwithin{equation}{section}

\setlength{\evensidemargin}{.25in}
\setlength{\oddsidemargin}{-.25in}
\setlength{\topmargin}{-.75in}
\setlength{\textwidth}{6.5in}
\setlength{\textheight}{9.5in}
\newcommand{\due}{February 5th, 2010}
\newcommand{\HWnum}{3}
\newcommand{\grad}{\bold\nabla}
\newcommand{\vecE}{\vec{E}}
\newcommand{\scrptR}{\vec{\mathfrak{R}}}
\newcommand{\kapa}{\frac{1}{4\pi\epsilon_0}}
\newcommand{\emf}{\mathcal{E}}

\begin{document}
\begin{titlepage}
\setlength{\topmargin}{1.5in}
\begin{center}
\Huge{Physics 3320} \\
\LARGE{Principles of Electricity and Magnetism II} \\
\Large{Professor Ana Maria Rey} \\[1cm]

\huge{Homework \#\HWnum}\\[0.5cm]

\large{Joe Becker} \\
\large{SID: 810-07-1484} \\
\large{\due} 

\end{center}

\end{titlepage}



\section{Problem \#1}
\begin{enumerate}[(a)]
\item 
We start with \emph{Faraday's Law} which states
\begin{equation}
\oint\vec{E}\cdot d\vec{l} = -\frac{d\Phi_B}{dt}
\label{flux}
\end{equation}
where $\Phi_B$ is the magnetic flux which is given as
$$\Phi_B(t) = \alpha t$$
and 
$$\frac{d\Phi_B(t)}{dt} = \alpha$$
So if we take the closed loop integral we can see that
$$\oint\vecE\cdot d\vec{l} = V_1-V_2$$
so now if we apply \emph{Ohm's Law} 
\begin{equation}
V = IR
\label{ohms}
\end{equation}
we see that $V_1 = IR_1$ and $V_2 = -IR_2$. Note that $V_2$ has current flowing in the opposite direction so we have a negative sign. Now we can say that
$$\oint\vecE\cdot d\vec{l} = IR_1+IR_2$$
So now if we apply equation \ref{flux} and solve for the current we get
\begin{align*}
\oint\vec{E}\cdot d\vec{l} &= -\frac{d\Phi_B}{dt}\\
I(R_1+R_2) &= -\alpha\\
I &= -\frac{\alpha}{R_1+R_2}
\end{align*}
So now we replace $I$ in our expressions of $V_1$ and $V_2$ and we get
$$V_1 = -\frac{\alpha R_1}{R_1+R_2}$$
$$V_2 = \frac{\alpha R_2}{R_1+R_2}$$

\item
We get different reading for each of the voltmeters in part (a). This is possible because of equation \ref{flux}. Normally if we go around a closed loop our change in voltage is zero, but we longer have a conservative field and the path of travel matters. So going from point $A$ to point $B$ will change depending of which path you choose.

\item
If we treat the induced electromotive forces as if they were batteries we can analyze the circuit using \emph{Kirchoff's Laws}. So we can say that the left loop is represented by
$$-\emf_1 = I_1R_1+I_1R_2+I_2R_2$$
and the right loop is represented by
$$-\emf_2 = I_2R_3+I_2R_2+I_1R_2$$
Where we know $R_1 = 6$, $R_2=3$, and $R_3=5$ and we can calculate the emfs as
\begin{align*}
\emf_1 &= (100)\pi(0.1)^2\\
&= \pi
\end{align*}
\begin{align*}
\emf_2 &= (100)\pi(0.15)^2\\
&= 2.25\pi
\end{align*}
so our equations become
\begin{align*}
-\emf_1 &= I_1(R_1+R_2)+I_2R_2\\
-\pi &= 9I_1+3I_2
\end{align*}
\begin{align*}
-\emf_2 = I_2R_3+I_2R_2+I_1R_2\\
-2.25\pi = 8I_2+3I_1
\end{align*}
Now we have a system of equations with 2 unknowns which we can write as a matrix
$$\left(\begin{array}{cc|r}
	9	&3	&-\pi\\	
	3	&8	&-2.25\pi	
	\end{array}\right)$$
Now we can perform row reductions to get the solutions for $I_1$ and $I_2$. First $-3r_2+r_1\rightarrow r_1$
$$\left(\begin{array}{cc|r}
	0	&-21	&-\pi-6.75\pi = -5.75\pi\\	
	3	&8	&-2.25\pi	
	\end{array}\right)$$
Next $(-1/21)r_1\rightarrow r_1$
$$\left(\begin{array}{cc|r}
	0	&1	&-(5.75/21)\pi\\	
	3	&8	&-2.25\pi	
	\end{array}\right)$$
then $-8r_1+r_2\rightarrow r_2$
$$\left(\begin{array}{cc|r}
	0	&1	&(5.75/21)\pi\approx0.86\\	
	3	&0	&2.25\pi - (8*5.75/21)\pi\approx0.19
	\end{array}\right)$$
and finally $(1/3)r_2\rightarrow r_2$
$$\left(\begin{array}{cc|r}
	0	&1	&-0.86\\	
	1	&0	&-0.063
	\end{array}\right)$$
Now we see that the current through $R_1$ is $-0.063\ A$ and the current through $R_3$ is $-0.86\ A$ and if we say that the total current though $R_2$ is $-I_1-I_2$ we see that the current through $R_2$ is $0.92\ A$.
\end{enumerate}

\section{Problem \#2}
\begin{enumerate}[(a)]
\item
We know that the force per unit charge, $\vec{f}$ on the plasma is due to the \emph{Lorenz Force} which is given by
\begin{equation}
\vec{f} = \vec{E}+\vec{B}\times\vec{v}
\label{lorenz}
\end{equation}
So in this situation we see that the force from the magnetic field is 
\begin{align*}
\vec{f}_s &= \vec{B}\times\vec{v}\\
&= B_0\hat{x}\times v_z\hat{z}\\
&= B_0\hat{x}v_z\hat{y}
\end{align*}
Now we assume that the force due to the electric field is only smooths out the flow of the plasma. Now we can use the relation 
\begin{equation}
V = \int_a^b\vec{f}_s\cdot d\vec{l}
\label{potdiff}
\end{equation}
So if we apply equation \ref{potdiff} 
\begin{align*}
V &= \int_a^b\vec{f}_s\cdot d\vec{l}\\
&= \int_0^{\Delta y}B_0v_z\hat{y}\cdot dy\hat{y}\\
&= B_0v_z\Delta y
\end{align*}

\item
To find the current driven through the load resistor $R$ we use \emph{Ohm's Law} given by equation \ref{ohms}. So we need to find the resistance through the plasma which is represented by $R$. So we know that resistance is given by
\begin{equation}
R = \frac{l}{\sigma A}
\label{equation}
\end{equation}
where $l$ is the length the current travels, $A$ is the cross-sectional area, and $\sigma$ is the conductivity. So we can find the resistance of the plasma by
\begin{align*}
R &= \frac{l}{\sigma A}\\
&= \frac{\Delta y}{\sigma \Delta x\Delta z}
\end{align*}
Note that the length $l$ is $\Delta y$ because we found that the current is flowing in the $\hat{y}$ direction. So now we can solve equation \ref{ohms} to find the current $I$
\begin{align*}
I &= \frac{V}{R}\\
&= B_0v_z\Delta y\frac{\sigma \Delta x\Delta z}{\Delta y}\\
&= B_0v_z\sigma \Delta x\Delta z
\end{align*}
\end{enumerate}

\section{Problem \#3}
If we look at \emph{Faraday's Law}
\begin{equation}
\grad\times\vec{E} = -\frac{\partial B}{\partial t}
\label{Faraday}
\end{equation}
and \emph{Amp\'{e}re's Law}
\begin{equation}
\grad\times\vec{B} = \mu_0\vec{J}
\label{Faraday}
\end{equation}
we see that they are almost the same. So if assume that
$$\vec{B} \rightarrow \vec{E}$$
$$\mu_0\vec{J}\rightarrow-\frac{\partial B}{\partial t}$$
Now we can see that the toroidal coil is analogous to a finding the magnetic field above the center of a loop of radius $a$ carrying a steady current (example 5.6 on Griffith's page 218). From here the magnetic field is given by
$$B(z) = \frac{\mu_0}{4\pi}I\int\frac{dl'}{\scrptR^2}\cos(\theta)$$
Note that this equation is in terms of $I$ and not $J$, but we know that if we multiply $J$ by the cross-sectional area we get the current. So if we do the same for the magnetic field we get a magnetic flux. This implies that
$$\mu_0I\rightarrow\frac{\partial \Phi_B}{\partial t}$$
So we can say that
$$E(z) = -\frac{1}{4\pi}\frac{\partial \Phi_B}{\partial t}\int\frac{dl'}{\scrptR^2}\cos(\theta)$$
Where we know that the magnetic field from a toroidal coil is
$$\vec{B} = \frac{\mu_0NI(t)}{2\pi a}$$
we see that the only time dependent function in $\vec{B}$ is $I(t)$ and we are given 
$$\frac{\partial I}{\partial t} = k$$
so we can say that
$$\frac{\partial \Phi_B}{\partial t} = \frac{\mu_0Nk}{2\pi a}wh$$
so now our integral becomes
\begin{align*}
E(z) &= -\frac{1}{4\pi}\frac{\mu_0Nk}{2\pi a}wh\int\frac{dl'}{\scrptR^2}\cos(\theta)\\
&= -\frac{1}{4\pi}\frac{\mu_0Nk}{2\pi a}wh\frac{2\pi a}{\scrptR^2}\cos(\theta)\\
&= -\frac{\mu_0Nkwh}{4\pi(a^2+z^2)^2}\cos(\theta)\\
&= -\frac{\mu_0Nkwh}{4\pi(a^2+z^2)^2}\frac{a}{\sqrt{a^2+z^2}}\\
\vec{E} &= -\frac{\mu_0Nhawk}{4\pi(a^2+z^2)^{3/2}}\hat{z}
\end{align*}

\section{Problem \#4}
\begin{enumerate}[(a)]
\item
To find the electromotive force we first need to find the magnetic flux through the loop. Magnetic flux is defined as
$$\Phi_B = \int_S\vec{B}\cdot d\vec{a}$$
We are given the magnetic field as
$$\vec{B} = B\cos(\omega t+\alpha)\hat{x}$$
and now the area vector is changing in time due to the changing angle $\phi(t)$. We see that the components of the area vector $\vec{A}$ can by written as
$$\vec{A} = A\cos(\phi(t)-\pi/2)\hat{x} + A\sin(\phi(t) - \pi/2)\hat{y}$$
Note that we can pull out the $\pi/2$ term by flipping the sines to cosines and vise versa. So
$$\vec{A} = A\sin(\phi(t))\hat{x} + A\cos(\phi(t))\hat{y}$$
So now we can find the magnetic flux
\begin{align*}
\Phi_B &= \int_S\vec{B}\cdot d\vec{a}\\
&= \int_SB\cos(\omega t+\alpha)\hat{x}\cdot\left(da\sin(\phi(t))\hat{x} + da\cos(\phi(t))\hat{y}\right)\\
&= \int_SB\cos(\omega t+\alpha)\sin(\phi(t))da\\
&= B\cos(\omega t+\alpha)\sin(\omega t + \phi_0)\int_Sda\\
&= BA\cos(\omega t+\alpha)\sin(\omega t + \phi_0)
\end{align*}
Note that $A$ is the area of our loop. Now we know that the electromotive force is given by
\begin{equation}
\emf = -\frac{d\Phi_B}{dt}
\label{EMF}
\end{equation}
So we can find the first time derivative of the flux as
\begin{align*}
\frac{d\Phi_B}{dt} &= BA\frac{d}{dt}\left(\cos(\omega t+\alpha)\sin(\omega t + \phi_0)\right)\\
&= BA\left(-\sin(\omega t+\alpha)\omega\sin(\omega t + \phi_0)+\cos(\omega t+\alpha)\cos(\omega t + \phi_0)\omega\right)\\
&= BA\omega\left(\cos(\omega t+\alpha)\cos(\omega t + \phi_0)-\sin(\omega t+\alpha)\sin(\omega t + \phi_0)\right)
\end{align*}
Now if we use the trig identity 
$$\cos(\alpha+\beta) = \cos(\alpha)\cos(\beta)-\sin(\alpha)\sin(\beta)$$
and equation \ref{EMF} we see that our electromotive force is 
\begin{align*}
\emf &= -\frac{d\Phi_B}{dt}\\
&= -BA\omega\left(\cos(\omega t+\alpha)\cos(\omega t + \phi_0)-\sin(\omega t+\alpha)\sin(\omega t + \phi_0)\right)\\
&= -BA\omega\left(\cos(\omega t+\alpha+\omega t + \phi_0)\right)\\
&= -BA\omega\cos(2\omega t+\alpha+\phi_0)
\end{align*}

\item
We see that if we want the electromotive force to be zero for all time $t$ we need to make the $\cos(2\omega t+\alpha+\phi_0)$ term to be zero for all $t$. For this to be true we see that
$$\frac{\pi}{2} = 2\omega t+\alpha+\phi_0$$
would have to be true for all $t$. This is not possible if $\alpha$ and $\phi_0$ are constant. So it is not possible to make the induced emf to always be zero.
\end{enumerate}

\end{document}

