\documentclass[11pt]{article}

\usepackage{latexsym}
\usepackage{amssymb}
\usepackage{amsthm}
\usepackage{enumerate}
\usepackage{amsmath}
\usepackage{cancel}
\numberwithin{equation}{section}

\setlength{\evensidemargin}{.25in}
\setlength{\oddsidemargin}{-.25in}
\setlength{\topmargin}{-.75in}
\setlength{\textwidth}{6.5in}
\setlength{\textheight}{9.5in}
\newcommand{\due}{January 26th, 2010}
\newcommand{\HWnum}{2}
\newcommand{\grad}{\bold\nabla}
\newcommand{\vecE}{\vec{E}}
\newcommand{\scrptR}{\vec{\mathfrak{R}}}
\newcommand{\kapa}{\frac{1}{4\pi\epsilon_0}}

\begin{document}
\begin{titlepage}
\setlength{\topmargin}{1.5in}
\begin{center}
\Huge{Physics 3320} \\
\LARGE{Principles of Electricity and Magnetism II} \\
\Large{Professor Ana Maria Rey} \\[1cm]

\huge{Homework \#\HWnum}\\[0.5cm]

\large{Joe Becker} \\
\large{SID: 810-07-1484} \\
\large{\due} 

\end{center}

\end{titlepage}



\section{Problem \#1}
To find the resistance of the two cylinders $R$ we will find it from \emph{Ohm's Law}
\begin{equation}
V = IR
\label{ohm}
\end{equation}
First need to find the electric field in between the two cylinders. To do this we need to use \emph{Gauss' Law}
\begin{equation}
\oint\vecE\cdot d\vec{a} = \frac{q_{enc}}{\epsilon_0}
\label{gauss}
\end{equation}
with a Gaussian cylinder with radius $s$ where $a<s<b$. So equation \ref{gauss} is calculated as
\begin{align*}
\oint\vecE\cdot d\vec{a} &= \frac{q_{enc}}{\epsilon_0}\\
E\oint da &= \frac{\lambda l}{\epsilon_0}\\
E(2\pi sl) &= \frac{\lambda l}{\epsilon_0}\\
E &= \frac{\lambda l}{2\pi sl\epsilon_0}\\
\vecE &= \frac{\lambda}{2\pi s\epsilon_0}\hat{s}
\end{align*}
Note that $\lambda$ is the charge per length of the inner cylinder. Now we can find the current using the equation
$$I = \int \vec{J}\cdot d\vec{a}$$
note that we know $\vec{J}$ from \emph{Ohm's Law}, but not equation \ref{ohm} but \emph{Ohm's Law} in the form
\begin{equation}
\vec{J} = \frac{1}{\rho}\vec{E}
\label{ohm2}
\end{equation}
So we can find the current as
\begin{align*}
I &= \int \vec{J}\cdot d\vec{a}\\
&= \int \frac{1}{\rho}\vec{E}\cdot d\vec{a}\\
&= \frac{1}{\rho}E\int da\\
&= \frac{1}{\rho}\frac{\lambda}{2\pi s\epsilon_0}(2\pi sl)\\
&= \frac{1}{\rho}\frac{\lambda l}{\epsilon_0}
\end{align*}
where $\rho$ is the resistivity. Now we can calculate the difference in the potential at point $a$ and $b$ by saying 
\begin{align*}
V &= -\int_a^b\vecE\cdot d\vec{l}\\
&= -\int_a^b\frac{\lambda}{2\pi s\epsilon_0}\hat{s}\cdot ds\hat{s}\\
&= \frac{\lambda}{2\pi\epsilon_0}\int_a^b\frac{1}{s} ds\\
&= \frac{\lambda}{2\pi\epsilon_0}\left(\ln(s)\right|_a^b\\
&= \frac{\lambda}{2\pi\epsilon_0}\left(\ln(b)-\ln(a)\right)\\
&= \frac{\lambda}{2\pi\epsilon_0}\ln\left(\frac{b}{a}\right)
\end{align*}
So now we can use equation \ref{ohm} to find the resistance $R$
\begin{align*}
V &= IR\\
R &= \frac{I}{V}\\
&= \frac{1}{\rho}\frac{\lambda l}{\epsilon_0}\frac{2\pi\epsilon_0}{\lambda\ln(b/a}\\
&= \frac{1}{\rho}\frac{2\pi l}{\ln(b/a)}
\end{align*}

\section{Problem \#2}
\begin{enumerate}[(a)]
\item
We need to find the potential from an infinite line of charge with charge distribution $\lambda$. So we use \emph{Gauss' Law} or equation \ref{gauss} to find the electric field due to the line.
\begin{align*}
\oint\vecE\cdot d\vec{a} &= \frac{Q_{enc}}{\epsilon_0}\\
E\oint da &= \frac{\lambda l}{\epsilon_0}\\
E(2\pi sl) &= \frac{\lambda l}{\epsilon_0}\\
E &= \frac{\lambda l}{2\pi sl\epsilon_0}\\
\vecE &= \frac{\lambda}{2\pi \epsilon_0 s}\hat{s}
\end{align*}
Note that the charge distribution for the second line of charge is $-\lambda$ so the electric field is
$$\vecE = \frac{-\lambda}{2\pi \epsilon_0 s'}\hat{s'}$$
Now we can find the potential from these electric fields by
\begin{align*}
V &= -\int\vecE\cdot d\vec{l}\\
\cancelto{0}{V(1)}-V(s) &= -\int_s^1\frac{\lambda}{2\pi \epsilon_0 s}\hat{s}\cdot ds\hat{s}\\
V(s) &= \int_s^1\frac{\lambda}{2\pi \epsilon_0 s} ds\\
&= \int_s^1\frac{\lambda}{2\pi \epsilon_0 }\ln(s)|_s^1 \\
&= \frac{\lambda}{2\pi \epsilon_0 }(\ln(1)-\ln(s)) \\
V(s) &= -\frac{\lambda}{2\pi \epsilon_0 }\ln(s) \\
\end{align*}
And for the other line charge we have the potential 
$$V(s') = \frac{\lambda}{2\pi \epsilon_0 }\ln(s')$$
So now we need to superimpose the potentials where we can say that
$$s = \sqrt{(x+d/2)^2+y^2}$$
and
$$s' = \sqrt{(x-d/2)^2+y^2}$$
so our total potential is
\begin{align*}
V_{tot}(s,\phi) &= V(s) + V(s')\\
&= \frac{\lambda}{2\pi \epsilon_0 }\ln\left(\sqrt{(x+d/2)^2+y^2}\right) -\frac{\lambda}{2\pi \epsilon_0 }\ln\left(\sqrt{(x-d/2)^2+y^2}\right)\\
&= \frac{\lambda}{2\pi \epsilon_0 }\ln\left(\frac{\sqrt{(x+d/2)^2+y^2}}{\sqrt{(x-d/2)^2+y^2}}\right)\\
&= \frac{\lambda}{4\pi \epsilon_0 }\ln\left(\frac{(x+d/2)^2+y^2}{(x-d/2)^2+y^2}\right)
\end{align*}
Now we want to remove the log from our equation so we say that
\begin{align*}
V &= \frac{\lambda}{4\pi \epsilon_0 }\ln\left(\frac{(x+d/2)^2+y^2}{(x-d/2)^2+y^2}\right) \\
\frac{4\pi\epsilon_0V}{\lambda} &= \ln\left(\frac{(x+d/2)^2+y^2}{(x-d/2)^2+y^2}\right) \\
e^{\frac{4\pi\epsilon_0V}{\lambda}} &= \frac{(x+d/2)^2+y^2}{(x-d/2)^2+y^2}
\end{align*}
now for simplicity we can say that
$$\alpha\equiv e^{\frac{4\pi\epsilon_0V}{\lambda}}$$ 
So now 
\begin{align*}
\alpha &= \frac{(x+d/2)^2+y^2}{(x-d/2)^2+y^2}\\
\alpha\left((x-d/2)^2+y^2\right) &= (x+d/2)^2+y^2\\
\alpha\left(x^2-dx+d^2/4+y^2\right) &= x^2+dx+d^2/4+y^2\\
\alpha x^2-\alpha dx+\alpha d^2/4+\alpha y^2 - x^2-dx-d^2/4-y^2 &= 0\\
x^2(\alpha-1) -dx(\alpha+1)+d^2/4(\alpha-1)+y^2(\alpha-1) &= 0\\
x^2 -dx\frac{\alpha+1}{\alpha-1} +d^2/4+y^2 &= 0
\end{align*}
Now we can define 
$$\beta \equiv \frac{\alpha+1}{\alpha-1}$$
So now we can say that
$$x^2 -\beta dx+d^2/4+y^2 = 0$$
Now we need to complete the square so we add $(\beta d/2)^2$ to both sides to get
\begin{align*}
x^2 -\beta dx+d^2/4+y^2 + \left(\frac{\beta d}{2}\right)^2 &= \left(\frac{\beta d}{2}\right)^2 \\ 
\left(x-\frac{\beta d}{2}\right)^2 +\frac{d^2}{4} +y^2  &= \left(\frac{\beta d}{2}\right)^2 \\ 
\left(x-\frac{\beta d}{2}\right)^2 +y^2  &= \left(\frac{\beta d}{2}\right)^2-\frac{d^2}{4}  \\ 
\left(x-\frac{\beta d}{2}\right)^2 +y^2  &= \frac{d^2}{4}\left(\beta^2-1\right) 
\end{align*}
Now we can see that this is the same as an equation for a circle where
$$\frac{h}{2} = \frac{\beta d}{2}$$
and
$$a^2 = \frac{d^2}{4}\left(\beta^2-1\right)$$
So if we calculate $h/2a$ we see that
\begin{align*}
\frac{h}{2a} &= \frac{\dfrac{\beta d}{2}}{\frac{d}{2}\sqrt{\beta^2-1}}\\
&= \frac{\beta}{\sqrt{\beta^2-1}}\\
&= \frac{\alpha+1}{\alpha-1}\left(\frac{(\alpha+1)^2}{(\alpha-1)^2}-\frac{(\alpha-1)^2}{(\alpha-1)^2}\right)^{-1/2}\\
&= \frac{\alpha+1}{\alpha-1}\left(\frac{\alpha^2+2\alpha+1 - \alpha^2 + 2\alpha - 1}{(\alpha-1)^2}\right)^{-1/2}\\
&= \frac{\alpha+1}{\alpha-1}\left(\frac{4\alpha}{(\alpha-1)^2}\right)^{-1/2}\\
&= \frac{\alpha+1}{2\sqrt{\alpha}}
\end{align*}
Now we can solve for $\alpha$ by saying
\begin{align*}
\frac{h}{2a} &= \frac{\alpha+1}{2\sqrt{\alpha}}\\
\frac{h}{2a}2\sqrt{\alpha} &= {\alpha+1}\\
\frac{h^2}{a^2}\alpha &= (\alpha+1)^2\\
\frac{h^2}{a^2}\alpha &= \alpha^2+2\alpha+1\\
0 &= -\frac{h^2}{a^2}\alpha + \alpha^2+2\alpha+1\\
0 &=  \alpha^2+\left(2-\frac{h^2}{a^2}\right)\alpha+1
\end{align*}
Now we can use the \emph{Quadratic Formula} 
\begin{equation}
x = \frac{-b\pm\sqrt{b^2-4ac}}{2a}
\label{quad}
\end{equation}
to find a solution for $\alpha$. So equation \ref{quad} yields
\begin{align*}
\alpha &= \frac{-\left(2-\frac{h^2}{a^2}\right)\pm\sqrt{\left(2-\frac{h^2}{a^2}\right)^2 - 4}}{2}\\
&= \frac{-2+\frac{h^2}{a^2}\pm\sqrt{4-2\frac{h^2}{a^2}+\frac{h^4}{a^4} - 4}}{2}\\
&= \frac{-2+\frac{h^2}{a^2}\pm\frac{h}{a}\sqrt{2+\frac{h^2}{a^2}}}{2}\\
&= -1+\frac{h^2}{2a^2}+\frac{h}{2a}\sqrt{2+\frac{h^2}{a^2}}
\end{align*}
So know we see that from our definitions that
\begin{align*}
\alpha &= e^{\frac{4\pi\epsilon_0V}{\lambda}}\\ 
\ln(\alpha) &= \frac{4\pi\epsilon_0V}{\lambda}\\ 
\lambda &= \frac{4\pi\epsilon_0V}{\ln(\alpha)} 
\end{align*}
and we now we know that 
\begin{align*}
\frac{\beta d}{2} &= \frac{h}{2}\\
d &= \frac{h}{\beta}\\
&= h\frac{\alpha+1}{\alpha-1}
\end{align*}
Where
$$\alpha = -1+\frac{h^2}{2a^2}+\frac{h}{2a}\sqrt{2+\frac{h^2}{a^2}}$$
Now we can check our units, we expect that $<\lambda> = C\ m^{-1}$ and we know that
$$<V_0> = N\ m\ C^{-1};\ <\epsilon_0> = C^2\ N^{-1}\ m^{-2}$$
So we calculate
\begin{align*}
\left<-\frac{2V_0\pi \epsilon_0}{\ln(\alpha)}\right> &= N\ m\ C^{-1} C^2\ N^{-1}\ m^{-2}\\ 
&=C\ m^{-1}
\end{align*}
Good our units agree.

\item
To find the current per length we use the fact that
$$I = \int\vec{J}\cdot d\vec{a}$$
and we apply equation \ref{ohm2} to say
$$I = \frac{1}{\rho}\int\vec{E}\cdot d\vec{a}$$
where $\rho$ is the resistivity of the material. So now we can apply equation \ref{gauss} to say that
\begin{align*}
I &= \frac{1}{\rho}\int\vec{E}\cdot d\vec{a}\\
&= \frac{1}{\rho}\frac{q_{enc}}{\epsilon_0}\\
&= \frac{\lambda l}{\rho\epsilon_0}
\end{align*}
So the current per length is given by
$$\frac{I}{l} = \frac{\lambda}{\rho\epsilon_0}$$
where $\lambda$ is found in part (a).

\end{enumerate}

\section{Problem \#3}
First we can start by finding the magnetic field from the line by using \emph{Amp\'{e}re's law}
\begin{equation}
\oint\vec{B}\cdot d\vec{l} = \mu_0 I_{enc}
\label{amp}
\end{equation}
So we can calculate the magnetic field using equation \ref{amp}
\begin{align*}
\oint\vec{B}\cdot d\vec{l} &= \mu_0 I_{enc}\\
B\oint d{l} &= \mu_0 I\\
B(2\pi s) &= \mu_0 I\\
\vec{B} = \frac{\mu_0I}{2\pi s}\hat{\phi}
\end{align*}
and we can say that the velocity $v$ is
$$\vec{v} = v_0\hat{z}$$
So now we can say the force per unit charge, $\vec{f}$, is given by
$$\vec{f} = \vec{v}\times\vec{B}$$
this is from the \emph{Lorentz force law} so we calculate
\begin{align*}
\vec{f} &= \vec{v}\times\vec{B}\\
&= v_0\hat{z}\times\frac{\mu_0I}{2\pi s}\hat{\phi}\\
&= -v_0\frac{\mu_0I}{2\pi s}\hat{s}
\end{align*}
So we can find the electromotive force by 
\begin{align*}
\varepsilon &= -\frac{d\Phi}{dt} \\
&= \oint\vec{f}\cdot d\vec{l}\\
&= \oint-v_0\frac{\mu_0I}{2\pi s}\hat{s}\cdot ds\hat{s}\\
&= -\int_{r-l/2}^{r+l/2} v_0\frac{\mu_0I}{2\pi s}ds\\
&= -v_0\frac{\mu_0I}{2\pi}\int_{r-l/2}^{r+l/2}\frac{1}{s} ds\\
&= -v_0\frac{\mu_0I}{2\pi}\ln(s)|_{r-l/2}^{r+l/2}\\
&= -v_0\frac{\mu_0I}{2\pi}(\ln({r+l/2}) - \ln({r-l/2}))\\
&= -v_0\frac{\mu_0I}{2\pi}\ln\left(\frac{r+l/2}{r-l/2}\right)
\end{align*}
But we need to note that the total electromotive force is zero because there will be a build up of charge and the induced electric field will cancel this force out perfectly.

\section{Problem \#4}
We can find the electromotive force by using
$$\varepsilon = -\frac{d\Phi_m}{dt}$$
Where we can say that the magnetic flux is 
$$\Phi_m = Bxl$$
So we can can say that
\begin{align*}
-\frac{d\Phi_m}{dt} &= -Bl\frac{d}{dt}x\\
&= -Blv
\end{align*}
So we can calculate the current this EMF induces by using \emph{Ohm's Law} or equation \ref{ohm}
\begin{align*}
\varepsilon &= IR\\
I &= \frac{\varepsilon}{R}\\
&= \frac{-Blv}{R}\\
\end{align*}
Now we can find the force that this EMF creates by saying 
\begin{align*}
\vec{F}_m &= I\int d\vec{l}\times\vec{B}\\
&= IB\int dl\\
&= IBl\\
&= \frac{B^2l^2v}{R}
\end{align*}
Note that this is opposite of the direction of the tension from the string. So now we have the differential equation 
$$m\frac{dv}{dt} = Mg - \frac{B^2l^2}{R}v$$
First we need to solve the homogeneous solution of
$$m\frac{dv}{dt} = -\frac{B^2l^2}{R}v$$
which we can solve by using separation of variables 
\begin{align*}
m\frac{dv}{dt} &= -\frac{B^2l^2}{R}v\\
\int\frac{1}{v}dv &= -\int\frac{B^2l^2}{Rm}dt\\
\ln(v) &= -\frac{B^2l^2}{Rm}t + C\\
v(t) &= e^{-\frac{B^2l^2}{Rm}t + C}\\
&= e^{C}e^{-\frac{B^2l^2}{Rm}t}\\
&= Ce^{-\frac{B^2l^2}{Rm}t}\\
\end{align*}
So now we can solve the particular solution which we know will be constant so we see that
\begin{align*}
0 &= Mg - \frac{B^2l^2}{R}v\\
Mg &= \frac{B^2l^2}{R}v\\
v_p &= \frac{MgR}{B^2l^2}
\end{align*}
So the general solution is 
$$v(t) = Ce^{-\frac{B^2l^2}{Rm}t} + \frac{MgR}{B^2l^2}$$
So we apply the initial condition
$$v(0) = 0$$
So 
\begin{align*}
v(0) &= 0 = Ce^{-\frac{B^2l^2}{Rm}0}  + \frac{MgR}{B^2l^2}\\
&= 0 = C  + \frac{MgR}{B^2l^2}\\
A  &= -\frac{MgR}{B^2l^2}
\end{align*}
So our final function of $v$ is
$$v(t) = \frac{MgR}{B^2l^2}\left(1 - \exp\left({-\frac{B^2l^2}{Rm}t}\right)\right)$$
So if we check our units we expect that $<v(t)> = m\ s^{-1}$ and we know that
$$<M> = <m> = kg;\ <g> = m\ s^{-2};\ <t> = s;\ <l> = m;\ <B> = kg\ C^{-1}\ s^{-1};\ <R> = kg\ m^2\ s^{-1}\ C^{-2}$$
So we can calculate 
\begin{align*}
\left<\frac{MgR}{B^2l^2}\left(1 - \exp\left({-\frac{B^2l^2}{Rm}t}\right)\right)\right> &= \frac{kg\ m\ s^{-2}\ kg\ m^2\ s^{-1}\ C^{-2}}{kg^2\ C^{-2}\ s^{-2}\ m^2}\left(\exp\left[\frac{kg^2\ C^{-2}\ s^{-2}\ m^2}{kg\ m^2\ s^{-1}\ C^{-2}\ kg}s\right]\right)\\
&= \frac{kg^2\ m^3\ s^{-3}\ C^{-2}}{kg^2\ m^2\ s^{-2}\ C^{-2}}\left(\exp\left[\frac{kg^2\ m^2\ s^{-1}\ C^{-2}}{kg^2\ m^2\ s^{-1}\ C^{-2}}\right]\right)\\
&= m\ s^{-1}
\end{align*}
Good our units agree.

\end{document}

