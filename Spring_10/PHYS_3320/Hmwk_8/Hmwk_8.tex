\documentclass[11pt]{article}

\usepackage{latexsym}
\usepackage{amssymb}
\usepackage{amsthm}
\usepackage{enumerate}
\usepackage{amsmath}
\usepackage{cancel}
\numberwithin{equation}{section}

\setlength{\evensidemargin}{.25in}
\setlength{\oddsidemargin}{-.25in}
\setlength{\topmargin}{-.75in}
\setlength{\textwidth}{6.5in}
\setlength{\textheight}{9.5in}
\newcommand{\due}{March 19th, 2010}
\newcommand{\HWnum}{8}
\newcommand{\grad}{\bold\nabla}
\newcommand{\vecE}{\vec{E}}
\newcommand{\scrptR}{\vec{\mathfrak{R}}}
\newcommand{\kapa}{\frac{1}{4\pi\epsilon_0}}
\newcommand{\emf}{\mathcal{E}}

\begin{document}
\begin{titlepage}
\setlength{\topmargin}{1.5in}
\begin{center}
\Huge{Physics 3320} \\
\LARGE{Principles of Electricity and Magnetism II} \\
\Large{Professor Ana Maria Rey} \\[1cm]

\huge{Homework \#\HWnum}\\[0.5cm]

\large{Joe Becker} \\
\large{SID: 810-07-1484} \\
\large{\due} 

\end{center}

\end{titlepage}



\section{Problem \#1}
\begin{enumerate}[(a)]
\item
Given the plane wave solution in a linear homogeneous dielectric
$$\vec{E}(z,t) = \vec{E_1}\cos\left(\sqrt{6}z-6\times10^{10}t\right)$$
we can see that the angular frequency is $\omega=6\times10^{10}$. So this wave falls under the classification of \emph{microwave} in the electromagnetic spectrum.

\item
We know that the index of refraction $n$ is the ratio between the speed of the wave in free space with the speed of the wave in the medium or
\begin{equation}
n = \frac{c}{v}
\label{Index}
\end{equation}
We can find the speed of this wave in the medium using the relation
\begin{equation}
v = \frac{\omega}{k}
\label{MedSpe}
\end{equation}
So if we combine equations \ref{Index} and \ref{MedSpe} we get
\begin{align*}
n &= \frac{ck}{\omega}\\
&= \frac{(3\times10^{8})(\sqrt{6}\times10^{2})}{6\times10^{10}}\\
&= 1.22
\end{align*}

\item
Since we know that $\omega=6\times10^{10}$ and in free space we know the wave travels at a velocity of $c$ so we can find the wavelength from
\begin{equation}
c = \frac{\omega}{k}
\label{Speed}
\end{equation}
where $k$ is the wave number which is related to wavelength by
\begin{equation}
k = \frac{2\pi}{\lambda}
\label{Numb}
\end{equation}
so if we combine equations \ref{Speed} and \ref{Numb} we get
\begin{align*}
c &= \frac{\omega}{k}\\
c &= \frac{\omega\lambda}{2\pi}
\end{align*}
So if we solve for $\lambda$ we get
\begin{align*}
\lambda &= \frac{2\pi c}{\omega} \\
&= \frac{2\pi (3\times10^{8})}{6\times10^{10}} \\
&= 3.14\times10^{-2}\ {m}
\end{align*}

\item
To find the magnetic field induced by this field we say
\begin{equation}
\vec{B} = \frac{1}{c}(\hat{k}\times\vec{E})
\label{Mag}
\end{equation}
So if we assume that $\vec{E}$ is only in $\hat{x}$ and $\hat{y}$ or that $\vec{E_1}=E_1(\hat{x}+\hat{y})$ we can see equation \ref{Mag} becomes
\begin{align*}
\vec{B} &= \frac{1}{c}(\hat{k}\times\vec{E})\\
&= \frac{1}{c}\left(\hat{k}\times\vec{E_1}\cos\left(\sqrt{6}z-6\times10^{10}t\right)\right)\\
&= \frac{1}{c}E_1\cos\left(\sqrt{6}z-6\times10^{10}t\right)\left(\hat{z}\times(\hat{x}+\hat{y})\right)\\
&= \frac{E_1}{c}\cos\left(\sqrt{6}z-6\times10^{10}t\right)\det\left(\begin{array}{ccc}
\hat{x}		&\hat{y}	&\hat{z}\\
0		&0		&1	\\
1		&1		&0	\\
\end{array}\right)\\
&= \frac{E_1}{c}\cos\left(\sqrt{6}z-6\times10^{10}t\right)(-\hat{x}+\hat{y})\\
&= \frac{\vec{E_2}}{c}\cos\left(\sqrt{6}z-6\times10^{10}t\right)
\end{align*}
Note we define 
$$\vec{E_2}\equiv E_1(\hat{y}-\hat{x})$$
\end{enumerate}

\section{Problem \#2}
To prove
\begin{equation}
\grad\cdot\vec{S}+\frac{\partial}{\partial t}U = -\vec{J}_{free}\cdot\vec{E}
\label{Poyn}
\end{equation}
We can start with \emph{Amp\'{e}re's Law}
\begin{equation}
\grad\times\vec{H} = \vec{J}_{free}+\frac{\partial\vec{D}}{\partial t}
\label{Amp}
\end{equation}
and solve for $\vec{J}_{free}$ and dot that with $\vec{E}$ we get
\begin{align*}
\vec{E}\cdot\vec{J}_{free} &= \vec{E}\cdot(\grad\times\vec{H}) - \vec{E}\cdot\frac{\partial\vec{D}}{\partial t}
\end{align*}
Now we know that
\begin{align*}
\grad\cdot(\vec{E}\times\vec{H}) &= \vec{E}\cdot(\grad\times\vec{H})-\vec{H}\cdot(\grad\times\vec{E})\\
\vec{E}\cdot(\grad\times\vec{H}) &=  \grad\cdot\vec{S} + \vec{H}\cdot(\grad\times\vec{E})\\
&=  \grad\cdot\vec{S} - \vec{H}\cdot\left(\frac{\partial\vec{B}}{\partial t}\right)
\end{align*}
from the triple product formula and \emph{Maxwell's equations}. Note that we said that
$$\vec{S} = \vec{E}\times\vec{H}$$
So now we can say that
\begin{align*}
\vec{E}\cdot\vec{J}_{free} &= \grad\cdot\vec{S} - \vec{H}\cdot\frac{\partial\vec{B}}{\partial t} - \vec{E}\cdot\frac{\partial\vec{D}}{\partial t}
\end{align*}
Now we can see that from the product rule of derivatives that
$$\vec{E}\cdot\frac{\partial\vec{D}}{\partial t} = \frac{1}{2}\frac{\partial}{\partial t}(\vec{E}\cdot\vec{D})$$
and
$$\vec{H}\cdot\frac{\partial\vec{B}}{\partial t} = \frac{1}{2}\frac{\partial}{\partial t}(\vec{B}\cdot\vec{H})$$
So
\begin{align*}
\vec{E}\cdot\vec{J}_{free} &= \grad\cdot\vec{S} - \vec{H}\cdot\frac{\partial\vec{B}}{\partial t} - \vec{E}\cdot\frac{\partial\vec{D}}{\partial t}\\
&= \grad\cdot\vec{S} - \frac{1}{2}\frac{\partial}{\partial t}(\vec{B}\cdot\vec{H}) - \frac{1}{2}\frac{\partial}{\partial t}(\vec{E}\cdot\vec{D})\\
&= \grad\cdot\vec{S} - \frac{1}{2}\frac{\partial}{\partial t}(\vec{B}\cdot\vec{H} + \vec{E}\cdot\vec{D})\\
&= \grad\cdot\vec{S} - \frac{\partial}{\partial t}U
\end{align*}
where we say that
$$U = \frac{1}{2}(\vec{B}\cdot\vec{H} + \vec{E}\cdot\vec{D})$$

\section{Problem \#3}
\begin{enumerate}[(a)]
\item
If we assume that the charge density $\rho$ is zero then we see the \emph{Maxwell equation}
\begin{equation}
\grad\cdot\vec{E} = 0
\label{Max1}
\end{equation}
Now from the \emph{Maxwell equation}
\begin{equation}
\grad\times\vec{E} = -\frac{\partial\vec{B}}{\partial t}
\label{Max2}
\end{equation}
we we take the curl of both sides we see that
$$\grad\times(\grad\times\vec{E}) = -\frac{\partial}{\partial t}(\grad\times\vec{B})$$
and if we apply the fact that
$$\grad\times(\grad\times\vec{A}) = \grad(\grad\cdot\vec{A})-\grad^2\vec{A}$$
we see that
\begin{align*}
\grad(\grad\cdot\vec{E})-\grad^2\vec{E} &= -\frac{\partial}{\partial t}(\grad\times\vec{B})\\
\grad^2\vec{E} &= \frac{\partial}{\partial t}(\grad\times\vec{B})
\end{align*}
Now from the \emph{Maxwell equation}
\begin{equation}
\grad\times\vec{B} = \mu_0\vec{J}+\frac{1}{c^2}\frac{\partial\vec{E}}{\partial t}
\label{Max3}
\end{equation}
Now we get
\begin{align*}
\grad^2\vec{E} &= \frac{\partial}{\partial t}\left(\mu_0\vec{J}+\frac{1}{c^2}\frac{\partial\vec{E}}{\partial t}\right)\\
\grad^2\vec{E} &= \mu_0\frac{\partial}{\partial t}\vec{J}+\frac{1}{c^2}\frac{\partial^2}{\partial t^2}\vec{E}\\
\grad^2\vec{E} - \frac{1}{c^2}\frac{\partial^2}{\partial t^2}\vec{E} &= \mu_0\frac{\partial}{\partial t}\vec{J}
\end{align*}

\item
If we start with 
$$\vec{F} = m\vec{a}$$
we can relate this to $\vec{J}$ by saying that
$$\vec{J} = \rho\vec{v}$$
so if we take a time derivative of $\vec{J}$ we see
$$\frac{\partial\vec{J}}{\partial t} = \rho\vec{a}$$
So we see that 
$$\vec{F} = \frac{m_e}{\rho}\frac{\partial\vec{J}}{\partial t}$$
now we know that 
$$\vec{F} = e\vec{E}$$
where $e$ is the charge of an electron. So we can see that
\begin{align*}
e\vec{E} &= \frac{m_e}{\rho}\frac{\partial\vec{J}}{\partial t}\\
\frac{e\rho}{m_e}\vec{E} &= \frac{\partial\vec{J}}{\partial t}\\
\frac{e^2N_e}{m_e}\vec{E} &= \frac{\partial\vec{J}}{\partial t}
\end{align*}
Note that we used the fact that $\rho = N_e e$ where $N_e$ is the electron density. So now if we look at the equation we found in part (a) we get
\begin{align*}
\grad^2\vec{E} - \frac{1}{c^2}\frac{\partial^2}{\partial t^2}\vec{E} &= \mu_0\frac{\partial}{\partial t}\vec{J}\\
\grad^2\vec{E} - \frac{1}{c^2}\frac{\partial^2}{\partial t^2}\vec{E} &= \mu_0\frac{e^2N_e}{m_e}\vec{E}\\
\grad^2\vec{E} - \frac{1}{c^2}\frac{\partial^2}{\partial t^2}\vec{E} &= \epsilon_0\mu_0\frac{e^2N_e}{\epsilon_0m_e}\vec{E}
\end{align*}
If we define the plasma frequency $\omega_p$ as
$$\omega_p^2\equiv\frac{N_ee^2}{\epsilon_0m_e}$$ 
we see that 
$$\grad^2\vec{E} - \frac{1}{c^2}\frac{\partial^2}{\partial t^2}\vec{E} = \frac{\omega_p^2}{c^2}\vec{E}$$
now if we use the fact that $\vec{E} = \vec{\widetilde{E}}e^{-i\omega t}$ we see that 
\begin{align*}
\grad^2\vec{\widetilde{E}}e^{-i\omega t} - \frac{1}{c^2}\frac{\partial^2}{\partial t^2}\vec{\widetilde{E}}e^{-i\omega t} &= \frac{\omega_p^2}{c^2}\vec{\widetilde{E}}e^{-i\omega t}\\
\grad^2\vec{\widetilde{E}}e^{-i\omega t} - \frac{1}{c^2}i^2\omega^2\vec{\widetilde{E}}e^{-i\omega t} &= \frac{\omega_p^2}{c^2}\vec{\widetilde{E}}e^{-i\omega t}\\
\grad^2\vec{\widetilde{E}} + \frac{\omega^2}{c^2}\vec{\widetilde{E}} &= \frac{\omega_p^2}{c^2}\vec{\widetilde{E}}\\
\left(\grad^2 + \frac{\omega^2}{c^2}\right)\vec{\widetilde{E}} &= \frac{\omega_p^2}{c^2}\vec{\widetilde{E}}
\end{align*}

\item
Now if we assume that $\vec{\widetilde{E}} = E_0\hat{x}e^{ikz}$ then we can apply the equation we found in part (c) where
\begin{align*}
\grad^2\vec{\widetilde{E}} &= \frac{\partial^2}{\partial z^2}E_0\hat{x}e^{ikz}\\
&= i^2k^2E_0\hat{x}e^{ikz} =  -k^2\vec{\widetilde{E}}
\end{align*}
So
\begin{align*}
\left(-k^2 + \frac{\omega^2}{c^2}\right)\vec{\widetilde{E}} &= \frac{\omega_p^2}{c^2}\vec{\widetilde{E}}\\
-k^2 + \frac{\omega^2}{c^2} &= \frac{\omega_p^2}{c^2}\\
k^2 &= \frac{\omega^2}{c^2} - \frac{\omega_p^2}{c^2}\\
k &= \frac{\sqrt{\omega^2-\omega_p^2}}{c}
\end{align*}

\item
We see that when $\omega<\omega_p$ then $k'=ik$ where $k$ is real so 
$$\vec{\widetilde{E}} = E_0\hat{x}e^{kz}$$
so the real part of $\vec{E}$ is given by
$$\textnormal{Re}[\vec{E}] = E_0\hat{x}e^{kz}\cos(\omega t)$$

\item
If we say that the reflection coefficient $R$ is given by
$$R = \frac{(n_1-n_2)^2}{(n_1+n_2)^2}$$
where $n_1$ and $n_2$ are the respective indexes of refraction we can say that for $\omega<\omega_p$ $R\rightarrow0$.

\end{enumerate}

\end{document}

