\documentclass[11pt]{article}

\usepackage{latexsym}
\usepackage{amssymb}
\usepackage{amsthm}
\usepackage{enumerate}
\usepackage{amsmath}
\usepackage{cancel}
\numberwithin{equation}{section}

\setlength{\evensidemargin}{.25in}
\setlength{\oddsidemargin}{-.25in}
\setlength{\topmargin}{-.75in}
\setlength{\textwidth}{6.5in}
\setlength{\textheight}{9.5in}
\newcommand{\due}{April 12th, 2010}
\newcommand{\HWnum}{10}
\newcommand{\grad}{\bold\nabla}
\newcommand{\vecE}{\vec{E}}
\newcommand{\scrptR}{\vec{\mathfrak{R}}}
\newcommand{\kapa}{\frac{1}{4\pi\epsilon_0}}
\newcommand{\emf}{\mathcal{E}}

\begin{document}
\begin{titlepage}
\setlength{\topmargin}{1.5in}
\begin{center}
\Huge{Physics 3310} \\
\LARGE{Principles of Electricity and Magnetism 1} \\
\Large{Professor Thomas R. Schibli} \\[1cm]

\huge{Homework \#\HWnum}\\[0.5cm]

\large{Joe Becker} \\
\large{SID: 810-07-1484} \\
\large{\due} 

\end{center}

\end{titlepage}



\section{Problem \#1}
\begin{enumerate}[(a)]
\item
We know that the wavenumber $k$ is related to the index of refraction $n$ through
\begin{equation}
k = \frac{\omega}{c}n
\label{wavenum}
\end{equation}
where $\omega$ is the frequency of the wave and $c$ is the speed of light. Now if we assume that $k$ and $n$ are functions of $\omega$ equation \ref{wavenum} becomes
$$k(\omega) = \frac{\omega}{c}n(\omega)$$
To find the group velocity $v_g$ we use the relation
\begin{equation}
v_g = \frac{1}{\frac{dk(\omega)}{d\omega}}
\label{groupvel}
\end{equation}
Where
\begin{align*}
\frac{dk(\omega)}{d\omega} &= \frac{d}{d\omega}\frac{\omega}{c}n(\omega)\\
&= \frac{1}{c}\left(n(\omega)+\omega\frac{dn(\omega)}{d\omega}\right)
\end{align*}
by chain rule. So equation \ref{groupvel} yields
$$v_g = \frac{c}{n(\omega)+\omega\frac{dn(\omega)}{d\omega}}$$

\item
For a single resonance frequency we can say that $n(\omega)$ is
$$n(\omega) = 1+\frac{Nq^2}{2m\epsilon_0}\frac{(\omega_0^2-\omega^2)}{(\omega_0^2-\omega^2)^2+\gamma^2\omega^2}$$
we can define a quantity $\Delta\equiv\omega-\omega_0$ we see that
\begin{align*}
\omega_0^2-\omega^2 &= (\omega_0+\omega)(\omega_0-\omega)\\
&= -(\omega_0+\omega)\Delta
\end{align*}
So we see that 
\begin{align*}
n(\omega) &= 1+\frac{Nq^2}{2m\epsilon_0}\frac{(\omega_0^2-\omega^2)}{(\omega_0^2-\omega^2)^2+\gamma^2\omega^2}\\
&= 1+\frac{Nq^2}{2m\epsilon_0}\frac{-(\omega_0+\omega)\Delta}{(-(\omega_0+\omega)\Delta)^2+\gamma^2\omega^2}\\
&= 1-\frac{Nq^2}{2m\epsilon_0}\frac{(\omega_0+\omega)\Delta}{(\omega_0+\omega)^2\Delta^2+\gamma^2\omega^2}\\
&= 1-\frac{Nq^2}{2m\epsilon_0}\frac{(\omega_0+\omega)\Delta}{\omega^2\left(\frac{(\omega_0+\omega)^2}{\omega^2}\Delta^2+\gamma^2\right)}\\
&= 1-\frac{Nq^2}{2m\epsilon_0}\frac{\frac{\omega_0+\omega}{\omega}\Delta}{\omega\left(\left(\frac{\omega_0+\omega}{\omega}\right)^2\Delta^2+\gamma^2\right)}
\end{align*}
Now we assume that $\omega$ is very close to $\omega_0$ this implies that
$$\frac{\omega_0}{\omega} \approx 1$$
so we see that
\begin{align*}
\frac{\omega_0+\omega}{\omega} &= \frac{\omega_0}{\omega}+1\\
&= \frac{\omega_0}{\omega}+1\\
&= 2
\end{align*}
So we see that $n(\omega)$ becomes 
\begin{align*}
n(\omega) &= 1-\frac{Nq^2}{2m\epsilon_0}\frac{\frac{\omega_0+\omega}{\omega}\Delta}{\omega\left(\left(\frac{\omega_0+\omega}{\omega}\right)^2\Delta^2+\gamma^2\right)}\\
&= 1-\frac{Nq^2}{2m\epsilon_0}\frac{2\Delta}{\omega\left(2^2\Delta^2+\gamma^2\right)}\\
&= 1-\frac{Nq^2}{4m\epsilon_0\omega}\frac{\Delta}{\Delta^2+\gamma^2/4}\\
&= 1-A\frac{\Delta}{\Delta^2+\gamma^2/4}
\end{align*}
Where
$$A\equiv\frac{Nq^2}{4m\epsilon_0\omega}$$

\item
\end{enumerate}

\section{Problem \#2}
\begin{enumerate}
\item
If we assume that the velocity of the wave is proportional to the square of the wavelength or
$$v = C\sqrt{\lambda}$$
Now we know that the wavelength is related to the wave number by
$$\lambda = \frac{2\pi}{k}$$
and the velocity of a wave is given by
$$v = \frac{\omega}{k}$$
So we find that
\begin{align*}
v &= C\sqrt{\lambda}\\
\frac{\omega}{k} &= C\sqrt{\frac{2\pi}{k}}\\
\omega &= Ck\sqrt{\frac{2\pi}{k}}\\
\omega(k) &= C\sqrt{2\pi k}
\end{align*}
Now we know that the group velocity of the wave is given by
$$v_g = \frac{d\omega}{dk}$$ 
Note that we have $\omega$ as a function of $k$ so we just take a derivative with respect to $k$ to find $v_g$
\begin{align*}
v_g &= \frac{d\omega}{dk}\\
&= \frac{d}{dk}C\sqrt{2\pi k}\\
&= C\sqrt{2\pi}\frac{d}{dk}\sqrt{k}\\
&= C\sqrt{2\pi}\frac{1}{2}\frac{1}{\sqrt{k}}\\
&= \frac{1}{2}C\sqrt{\frac{2\pi}{k}}\\
&= \frac{1}{2}C\sqrt{\lambda}\\
v_g &= \frac{1}{2}v
\end{align*}
So the group velocity is twice the velocity $v$.

\item
We see that we can relate the wave number and frequency for the wave function
$$\Psi(x,t) = Ae^{i(px-Et)/\hbar}$$
by seeing that we normally write a plane wave as
$$f(x,t) = Ae^{i(kx-\omega t)}$$
so we can say
\begin{align*}
\frac{i(px-Et)}{\hbar} &= i(kx-\omega t)\\
\frac{p}{\hbar}x-\frac{E}{\hbar}t &= kx-\omega t
\end{align*}
So we can see that $k = p/\hbar$ and $\omega = E/\hbar$. Now we can use the relation 
$$E = \frac{p^2}{2m}$$
to find the phase velocity, $v$, by
\begin{align*}
v &= \frac{\omega}{k}\\
&= \frac{E}{p}\\
&= \frac{p}{2m}
\end{align*}
Now to find the group velocity we first need to find $\omega$ as a function of $k$ by
\begin{align*}
\omega &= \frac{E}{\hbar}\\
&= \frac{p^2}{2m\hbar}\\
&= \frac{\hbar k^2}{2m}
\end{align*}
So we can find the group velocity as
\begin{align*}
v_g &= \frac{d\omega}{dk}\\
&= \frac{d}{dk}\frac{\hbar k^2}{2m}\\
&= \frac{\hbar k}{m}\\
&= \frac{p}{m}
\end{align*}
Note that the group velocity is again twice that of the phase velocity. We also see that classically the velocity is given by $p = mv_c$. So we can say that the group velocity is the classical velocity.
\end{enumerate}

\section{Problem \#3}
\begin{enumerate}[(a)]
\item
To find the \emph{Fourier Transform}
\begin{equation}
f(\omega) = \int_{-\infty}^{\infty}f(t)e^{-i\omega t}dt
\label{four}
\end{equation}
for 
$$f(t) = Ae^{-t^2/2\Delta^2}$$
equation \ref{four} yields
\begin{align*}
f(\omega) &= \int_{-\infty}^{\infty}Ae^{-t^2/2\Delta^2}e^{-i\omega t}dt\\
&= \int_{-\infty}^{\infty}Ae^{-t^2/2\Delta^2-i\omega t}dt
\end{align*}
Note that we can reduce the exponent by completing the square
\begin{align*}
-t^2/2\Delta^2-i\omega t &= -\frac{1}{2\Delta^2}\left(t^2+2i\Delta^2\omega t\right)\\
&= -\frac{1}{2\Delta^2}\left(t^2+2i\Delta^2\omega t + i^2\Delta^4\omega^2 - i^2\Delta^4\omega^2\right)\\
&= -\frac{1}{2\Delta^2}\left((t+i\Delta^2\omega)^2 + \Delta^4\omega^2\right)\\
&= -\frac{1}{2\Delta^2}(t+i\Delta^2\omega)^2 - \frac{1}{2}\Delta^2\omega^2
\end{align*}
So the transform becomes
\begin{align*}
\int_{-\infty}^{\infty}Ae^{-t^2/2\Delta^2-i\omega t}dt &= \int_{-\infty}^{\infty}Ae^{-\frac{1}{2\Delta^2}(t+i\Delta^2\omega)^2  \frac{1}{2}\Delta^2\omega^2}dt\\
&= e^{-\omega^2\Delta^2/2}\int_{-\infty}^{\infty}Ae^{-\frac{1}{2\Delta^2}(t+i\Delta^2\omega)^2}dt
\end{align*}
Now we can use a $u$ substitution where
$$u = \frac{1}{\sqrt{2}\Delta}(t+i\Delta^2\omega)$$
where
$$du = \frac{1}{\sqrt{2}\Delta}dt$$
\begin{align*}
e^{-\omega^2\Delta^2/2}\int_{-\infty}^{\infty}Ae^{-\frac{1}{2\Delta^2}(t+i\Delta^2\omega)^2}dt &= e^{-\omega^2\Delta^2/2}\sqrt{2}{\Delta}\int_{-\infty}^{\infty}Ae^{-u^2}du\\
&= Ae^{-\omega^2\Delta^2/2}\sqrt{2}{\Delta}\sqrt{\pi}\\
&= A\sqrt{2\pi\Delta^2}e^{-\omega^2\Delta^2/2}
\end{align*}
Note that the characteristic bandwidth is when
\begin{align*}
\omega^2\frac{\Delta^2}{2} &= 1\\
\omega &= \frac{\sqrt{2}}{\Delta}
\end{align*}

\item
Now to find the \emph{Fourier Transform} of
$$E(t) = E_0e^{-t^2/(2\Delta^2)}\cos(\omega_0t)$$
we first need to say that 
$$\cos(\omega_0t) = \frac{1}{2}(e^{i\omega_0t}+e^{-i\omega_0t})$$
So equation \ref{four} yields 
\begin{align*}
E(\omega) &= \frac{E_0}{2}\int_{-\infty}^{\infty}e^{-t^2/(2\Delta^2)}(e^{i\omega_0t}+e^{-i\omega_0t})e^{-i\omega t}dt\\
&= \frac{E_0}{2}\int_{-\infty}^{\infty}e^{-t^2/(2\Delta^2)}(e^{-i(\omega-\omega_0)t}+e^{-i(\omega+\omega_0)t})dt\\
&= \frac{E_0}{2}\left(\int_{-\infty}^{\infty}e^{-t^2/(2\Delta^2)}e^{-i(\omega-\omega_0)t}dt+\int_{-\infty}^{\infty}e^{-t^2/(2\Delta^2)}e^{-i(\omega+\omega_0)t}dt\right)
\end{align*}
Note that both of these integrals are the same as in part (a) except we replace $\omega$ with $\omega\pm\omega_0$ so we can just use the result from part (a) to find that
\begin{align*}
E(\omega) &= \frac{E_0}{2}\left(\int_{-\infty}^{\infty}e^{-t^2/(2\Delta^2)}e^{-i(\omega-\omega_0)t}dt+\int_{-\infty}^{\infty}e^{-t^2/(2\Delta^2)}e^{-i(\omega+\omega_0)t}dt\right)\\
&= \frac{E_0}{2}\left(\sqrt{2\pi\Delta^2}e^{-(\omega-\omega_0)\Delta^2/2}+\sqrt{2\pi\Delta^2}e^{-(\omega+\omega_0)\Delta^2/2}\right)\\
&= E_0\sqrt{\frac{\pi\Delta^2}{2}}\left(e^{-(\omega-\omega_0)\Delta^2/2}+e^{-(\omega+\omega_0)\Delta^2/2}\right)
\end{align*}
See attached for a sketch of $E(\omega)$
0
\item
Now if we add the $z$ dependence to $E(\omega)$ we get
$$E(\omega,z) = E_0\sqrt{\frac{\pi\Delta^2}{2}}e^{-ikz}\left(e^{-(\omega-\omega_0)\Delta^2/2}+e^{-(\omega+\omega_0)\Delta^2/2}\right)$$
where $k$ is dependent on $\omega$ by $k(\omega) = k_0+(\omega-\omega_0)k_0'+1/2(\omega-\omega_0)^2k_0''$. Now we apply the \emph{Fourier Transform} to find $E(z,t)$
\begin{align*}
E(z,t) &= \int_{-\infty}^{\infty}E(\omega,t)e^{i\omega t}d\omega\\
&= E_0\sqrt{\frac{\pi\Delta^2}{2}}\int_{-\infty}^{\infty}\left(e^{-(\omega-\omega_0)\Delta^2/2}+e^{-(\omega+\omega_0)\Delta^2/2}\right)e^{-ik(\omega)z}e^{i\omega t}d\omega\\
&= E_0\sqrt{\frac{\pi\Delta^2}{2}}\int_{-\infty}^{\infty}\left(e^{-(\omega-\omega_0)\Delta^2/2}+e^{-(\omega+\omega_0)\Delta^2/2}\right)e^{-i(k_0+(\omega-\omega_0)k_0'+1/2(\omega-\omega_0)^2k_0'')z}e^{i\omega t}d\omega
\end{align*}
Using mathematica we see that
$$E(t,z) = \Re\left[E_0\frac{\Delta}{\sqrt{\Delta^2-ik_0''z}}e^{\frac{-(t-z/v_g)}{2(\Delta^2-ik_0''z)}}e^{i(k_0z-\omega_0t)}\right]$$

\item
\end{enumerate}

\end{document}

