\documentclass[11pt]{article}

\usepackage{latexsym}
\usepackage{amssymb}
\usepackage{amsthm}
\usepackage{enumerate}
\usepackage{amsmath}
\usepackage{cancel}
\numberwithin{equation}{section}

\setlength{\evensidemargin}{.25in}
\setlength{\oddsidemargin}{-.25in}
\setlength{\topmargin}{-.75in}
\setlength{\textwidth}{6.5in}
\setlength{\textheight}{9.5in}
\newcommand{\due}{March 31st, 2010}
\newcommand{\HWnum}{10}
\newcommand{\grad}{\bold\nabla}
\newcommand{\vecE}{\vec{E}}
\newcommand{\scrptR}{\vec{\mathfrak{R}}}
\newcommand{\kapa}{\frac{1}{4\pi\epsilon_0}}
\newcommand{\expt}[1]{\langle{#1}\rangle}
\newcommand{\norm}[2]{\left\langle{#1}\right|\left.{#2}\right\rangle}
\newcommand{\ket}[1]{\left|{#1}\right\rangle}
\newcommand{\bra}[1]{\left\langle{#1}\right|}


\begin{document}
\begin{titlepage}
\setlength{\topmargin}{1.5in}
\begin{center}
\Huge{Physics 3310} \\
\LARGE{Principles of Electricity and Magnetism 1} \\
\Large{Professor Thomas R. Schibli} \\[1cm]

\huge{Homework \#\HWnum}\\[0.5cm]

\large{Joe Becker} \\
\large{SID: 810-07-1484} \\
\large{\due} 

\end{center}

\end{titlepage}



\section{Problem \#1}
\begin{enumerate}[(a)]
\item
Given a particle in the ground state of the simple harmonic oscillator with a potential of $V(x) = 1/2m\omega^2x^2$. We say that the particle's wavefunction is
$$\Psi(x,0) = \left(\frac{m\omega}{\pi\hbar}\right)^{1/4}e^{-\frac{m\omega}{2\hbar}x^2}$$
now if we want to find the probability that a measurement of the momentum will be $p\pm dp$ we have to find the wavefunction in momentum space. To do this we need to find the \emph{Fourier transform} of $\Psi(x,0)$ from
\begin{equation}
\phi(k) = \frac{1}{\sqrt{2\pi\hbar}}\int_{-\infty}^{\infty}\Psi(x,0)e^{-ikx}dx
\label{FourTran}
\end{equation}
So if we apply equation \ref{FourTran} to the wavefunction for the ground state of the harmonic oscillator we can use the result from problem 1 of homework \#6 which states that for 
$$\Psi(x,0) = \left(\frac{2a}{\pi}\right)^{1/4}e^{-ax^2}$$
the momentum space wave function is 
$$\phi(k) = \left(\frac{1}{2\pi a}\right)^{1/4}e^{-k^2/4a}$$
Note that for this case 
$$a = \frac{m\omega}{2\hbar}$$
So $\phi(k)$ for the ground state of the harmonic oscillator is
$$\phi(k) = \left(\frac{\hbar}{m\pi\omega}\right)^{1/4}e^{-\frac{\hbar}{2m\omega}k^2}$$
Now we want $\phi(p)$ not $\phi(k)$ so we use the relation
\begin{equation} 
\phi(p) = \frac{1}{\sqrt{\hbar}}\phi(k)
\label{MomSpace} 
\end{equation} 
So equation \ref{MomSpace} yields
$$\phi(p) = \left(\frac{1}{m\pi\hbar\omega}\right)^{1/4}e^{-\frac{1}{2m\hbar\omega}p^2}$$
Now we can say that the probability of measuring a momentum in the range $p\pm dp$ is given by the probability density or
\begin{align*}
\phi^*(p)\phi(p) &= \left(\frac{1}{m\pi\hbar\omega}\right)^{1/4}e^{-\frac{1}{2m\hbar\omega}p^2}\left(\frac{1}{m\pi\hbar\omega}\right)^{1/4}e^{-\frac{1}{2m\hbar\omega}p^2}\\
&= \left(\frac{1}{m\pi\hbar\omega}\right)^{1/2}e^{-\frac{1}{2m\hbar\omega}p^2-\frac{1}{2m\hbar\omega}p^2}\\
&= \left(\frac{1}{m\pi\hbar\omega}\right)^{1/2}e^{-p^2/{m\hbar\omega}}
\end{align*}

\item
Now to make sure this is a correct probability density we integrate $\phi^*(p)\phi(p)$ over all $p$ 
\begin{align*}
\int_{-\infty}^{\infty}\phi^*(p)\phi(p)dp &= \int_{-\infty}^{\infty}\left(\frac{1}{m\pi\hbar\omega}\right)^{1/2}e^{-p^2/{m\hbar\omega}}dp
\end{align*}
Now we can use the identity 
\begin{equation}
\int_{-\infty}^{\infty}e^{-az^2}dz = \sqrt{\frac{\pi}{a}}
\label{GaussInt}
\end{equation}
So for our integral we apply equation \ref{GaussInt} where
$$a = \frac{1}{m\hbar\omega}$$
so
\begin{align*}
\left(\frac{1}{m\pi\hbar\omega}\right)^{1/2}\int_{-\infty}^{\infty}e^{-p^2/{m\hbar\omega}}dp &= \frac{1}{\sqrt{m\pi\hbar\omega}}\sqrt{m\pi\hbar\omega} = 1
\end{align*}
So we have a reasonable answer.

\item
If we measure the momentum to be $p$ we know that the state of the particle will be in an eigenfunction of momentum with an eigenvalue of $p$ so
$$\frac{\hbar}{i}\frac{\partial}{\partial x}f_p(x) = pf_p(x)$$
Which we can find the general solution by
\begin{align*}
\frac{\hbar}{i}\frac{\partial f_p}{\partial x} &= pf_p\\
\int\frac{1}{f_p}{\partial f_p} &= \int\frac{ip}{\hbar}{\partial x}\\
\ln(f_p) &= \frac{ip}{\hbar}x+a\\
f_p(x) &= e^{ipx/\hbar+a}
\end{align*}
Note that $e^{a}$ is a constant which we will call $A$. So the general solution is
$$f_p(x) = Ae^{ipx/\hbar}$$
So we can say that the particle is in the state given by $f_p$.

\item
Now after this measurement if we wait for some time $T$ the state will revert back to the ground state of the harmonic oscillator. So if we measure the energy we will find that the energy is at the ground state with a probability of 1.
\end{enumerate}

\section{Problem \#2}
\begin{enumerate}[(a)]
\item
Since we know that the energy is equally likely to have an energy of $E_1$ or $EE_2$ we can say the state is given by
$$\Psi(x,0) = \frac{1}{2}\psi_0(x)+\frac{1}{2}\psi_1(x)$$
where $\psi_n(x)$ is the $n$th eigenfunction for the energy. So now we can find the expectation of momentum where we use the momentum operator
$$\hat{p} = i\sqrt{\frac{\hbar m\omega}{2}}(\hat{a}_+-\hat{a}_-)$$
Note that $\hat{a}_+$ and $\hat{a}_-$ are the raising and lowering operators that act of $\psi_n$ by 
\begin{equation}
\hat{a}_+\psi_n = \sqrt{n+1}\psi_{n+1}
\label{Raise}
\end{equation}
\begin{equation}
\hat{a}_-\psi_n = \sqrt{n}\psi_{n-1}
\label{Raise}
\end{equation}
So we can find the expectation of $\hat{p}$ by 
\begin{align*}
\expt{\hat{p}} &= \norm{\Psi(x,0)}{\hat{p}\Psi(x,0)}\\
&= \norm{\frac{1}{2}\psi_0(x)+\frac{1}{2}\psi_1(x)}{i\sqrt{\frac{\hbar m\omega}{2}}(\hat{a}_+-\hat{a}_-)\left(\frac{1}{2}\psi_0(x)+\frac{1}{2}\psi_1(x)\right)}\\
&= \frac{i}{4}\sqrt{\frac{\hbar m\omega}{2}}\norm{\psi_0(x)+\psi_1(x)}{(\hat{a}_+-\hat{a}_-)\left(\psi_0(x)+\psi_1(x)\right)}\\
&= \frac{i}{4}\sqrt{\frac{\hbar m\omega}{2}}\norm{\psi_0(x)+\psi_1(x)}{\hat{a}_+\psi_0(x)+\hat{a}_+\psi_1(x)-\hat{a}_-\psi_0(x)-\hat{a}_-\psi_1(x)}\\
&= \frac{i}{4}\sqrt{\frac{\hbar m\omega}{2}}\norm{\psi_0(x)+\psi_1(x)}{\psi_1(x)+\sqrt{2}\psi_2(x)-\psi_0(x)}\\
&= \frac{i}{4}\sqrt{\frac{\hbar m\omega}{2}}\left(\norm{\psi_1(x)}{\psi_1(x)}+\norm{\psi_0(x)}{-\psi_0(x)}\right)\\
&= \frac{i}{4}\sqrt{\frac{\hbar m\omega}{2}}\left(\norm{\psi_1(x)}{\psi_1(x)}-\norm{\psi_0(x)}{\psi_0(x)}\right)\\
&= \frac{i}{4}\sqrt{\frac{\hbar m\omega}{2}}(1-1)\\
\expt{\hat{p}} &=0
\end{align*}

\item
We assumed in part (a) that $t=0$ so for times later times the state is in the form of
$$\Psi(x,t) = \frac{1}{2}\left(\psi_0e^{iE_0t/\hbar}+\psi_1e^{iE_1t/\hbar}\right)$$
where the energy is given by 
$$E_n = \left(n+\frac{1}{2}\right)\hbar\omega$$
\end{enumerate}

\section{Problem \#3}
\begin{enumerate}[(a)]
\item
Given the relation 
\begin{equation}
\frac{d}{dt}\expt{Q} = \frac{i}{\hbar}\expt{[\hat{H},\hat{Q}]}+\left\langle\frac{\partial\hat{Q}}{\partial t}\right\rangle
\label{Comm}
\end{equation}
we can find the rate of change of the expectation of an operator $\hat{Q}$.
\begin{itemize}
\item For $\hat{Q} = 1$ we find the how $\hat{Q}$ commutes with $\hat{H}$ 
\begin{align*}
[\hat{H},\hat{Q}] &= \hat{H}\hat{Q} - \hat{Q}\hat{H}\\
&= -\frac{\hbar^2}{2m}\frac{\partial^2}{\partial x^2}1 +V(x)1 + 1\frac{\hbar^2}{2m}\frac{\partial^2}{\partial x^2} - 1V(x)\\
&= 0
\end{align*}
So equation \ref{Comm} yields
\begin{align*}
\frac{d}{dt}\expt{Q} &= \frac{i}{\hbar}\expt{[\hat{H},\hat{Q}]}+\left\langle\frac{\partial\hat{Q}}{\partial t}\right\rangle\\
&= \frac{i}{\hbar}\expt{0}+\left\langle\frac{\partial}{\partial t}1\right\rangle\\
&= \frac{i}{\hbar}\expt{0}+\expt{0}\\
&= 0
\end{align*}
This implies that the expectation of the operator $1$ is constant in time. This makes sense because we can see that 
$$\expt{1} = \int_{-\infty}^{\infty}|\psi(x)|^2dx$$
which is the normalization of the wavefunction $\psi$. Note that a wavefunction is always constant in time this implies that
$$\frac{d}{dt}\expt{1} = \frac{d}{dt}\int_{-\infty}^{\infty}|\psi(x)|^2dx=0$$
which is in agreement which the result from equation \ref{Comm}.
\item For $\hat{Q}=\hat{H}$ we can find how $\hat{H}$ commutes 
\begin{align*}
[\hat{H},\hat{H}] &= \hat{H}\hat{H} - \hat{H}\hat{H}\\
&= \left(-\frac{\hbar^2}{2m}\frac{\partial^2}{\partial x^2}+V(x)\right)\left(-\frac{\hbar^2}{2m}\frac{\partial^2}{\partial x^2}+V(x)\right) - \left(-\frac{\hbar^2}{2m}\frac{\partial^2}{\partial x^2}+V(x)\right)\left(-\frac{\hbar^2}{2m}\frac{\partial^2}{\partial x^2}+V(x)\right)\\
&= 0
\end{align*}
So equation \ref{Comm} yields
\begin{align*}
\frac{d}{dt}\expt{H} &= \frac{i}{\hbar}\expt{[\hat{H},\hat{H}]}+\left\langle\frac{\partial\hat{H}}{\partial t}\right\rangle\\
&= \frac{i}{\hbar}\expt{0}+\left\langle\frac{\partial\hat{H}}{\partial t}\right\rangle\\
&= \left\langle\frac{\partial\hat{H}}{\partial t}\right\rangle
\end{align*}
Note that $\hat{H}$ is independent of time so 
$$\left\langle\frac{\partial\hat{H}}{\partial t}\right\rangle = 0$$
so we see that
$$\frac{d}{dt}\expt{\hat{H}} = 0$$
This is what we expected. When we use the operator $\hat{H}$ we assume that the potential is independent of time therefore the total energy in the system remains constant in time. Note that $\hat{H}$ represents a measurement of the total energy of the system. So it makes sense that the expectation of $\hat{H}$ is constant in time. 
\item For $\hat{Q}=\hat{p}$ we see that the commuter is
\begin{align*}
[\hat{H},\hat{p}]f(x) &= \hat{H}\hat{p}f(x) - \hat{p}\hat{H}f(x)\\
&= \left(-\frac{\hbar^2}{2m}\frac{\partial^2}{\partial x^2}+V(x)\right)\frac{\hbar}{i}\frac{\partial}{\partial x}f(x) - \frac{\hbar}{i}\frac{\partial}{\partial x}\left(-\frac{\hbar^2}{2m}\frac{\partial^2}{\partial x^2}+V(x)\right)f(x)\\
&= -\frac{\hbar^3}{2im}\frac{\partial^2}{\partial x^2}\frac{\partial f(x)}{\partial x}+V(x)\frac{\hbar}{i}\frac{\partial f(x)}{\partial x} - \frac{\hbar}{i}\frac{\partial}{\partial x}\left(-\frac{\hbar^2}{2m}\frac{\partial^2f(x)}{\partial x^2}+V(x)f(x)\right)\\
&= -\frac{\hbar^3}{2im}\frac{\partial^3 f(x)}{\partial x^3}+V(x)\frac{\hbar}{i}\frac{\partial f(x)}{\partial x} + \frac{\hbar^3}{2im}\frac{\partial^3f(x)}{\partial x^3}- \frac{\hbar}{i}\frac{\partial}{\partial x}V(x)f(x)\\
&= V(x)\frac{\hbar}{i}\frac{\partial f(x)}{\partial x} - \frac{\hbar}{i}V(x)\frac{\partial f(x)}{\partial x} - \frac{\hbar}{i}\frac{\partial V(x)}{\partial x}f(x)\\
[\hat{H},\hat{p}] &=  -\frac{\hbar}{i}\frac{\partial V(x)}{\partial x}
\end{align*}
So equation \ref{Comm} yields
\begin{align*}
\frac{d}{dt}\expt{p} &= \frac{i}{\hbar}\expt{[\hat{H},\hat{p}]}+\left\langle\frac{\partial\hat{p}}{\partial t}\right\rangle\\
&= -\frac{i}{\hbar}\frac{\hbar}{i}\left\langle\frac{\partial V(x)}{\partial x}\right\rangle+\left\langle\frac{\partial\hat{p}}{\partial t}\right\rangle\\
&= \left\langle\frac{-\partial V(x)}{\partial x}\right\rangle
\end{align*}
This is the same relation classically that the force is equal to the change in potential with respect to position. This follows from the quantum \emph{Ehrenfest's theorem}.
\item Recall from homework \#9 that
$$\frac{d}{dt}\expt{x} = \frac{1}{m}\expt{\hat{p}}$$
we see that this is just the classical relation of velocity and the change in position with respect to time.
\end{itemize}

\item
Given a particle in the state 
$$\Psi(x,0) = A(u_1(x)+u_2(x))$$
where $u_n(x)$ is the $n$th eigenfunction of $\hat{H}$ for the infinite square well given by
$$u_n(x) = \sqrt{2}{a}\sin\left(\frac{n\pi}{a}x\right)$$
note that $u_n(x)$ forms an orthonormal basis. Therefore
\begin{align*}
\norm{\Psi(x,0)}{\Psi(x,0)} = 1 &= A\norm{u_1(x)+u_2(x)}{u_1(x)+u_2(x)}\\
&= A\left(\norm{u_1(x)}{u_1(x)} + \norm{u_2(x)}{u_2(x)}\right)\\
&= A(1+1)\\
A &= \frac{1}{2}
\end{align*}
With the time dependence the state is
$$\Psi(x,t) = \frac{1}{2}\left(u_1(x)e^{iE_1t/\hbar}+u_2(x)e^{iE_2t/\hbar}\right)$$
Where $E_n$ is the eigenvalue of $u_n(x)$. We can find the expectation value of the operator $\hat{H}$.
\begin{align*}
\norm{\Psi(x,0)}{\hat{H}\Psi(x,0)} &= \frac{1}{4}\norm{u_1(x)+u_2(x)}{\hat{H}\left(u_1(x)+u_2(x)\right)} \\
&= \frac{1}{4}\norm{u_1(x)+u_2(x)}{\hat{H}u_1(x)+\hat{H}u_2(x)} \\
&= \frac{1}{4}\norm{u_1(x)+u_2(x)}{E_1u_1(x)+E_2\hat{H}u_2(x)} \\
&= \frac{1}{4}\left(E_1\norm{u_1(x)}{u_1(x)}+E_2\norm{u_2(x)}{u_2(x)}\right)\\
\expt{H} &= \frac{1}{4}\left(E_1+E_2\right)
\end{align*}
And for $\expt{\hat{H^2}}$ we get
\begin{align*}
\norm{\Psi(x,0)}{\hat{H^2}\Psi(x,0)} &= \frac{1}{4}\norm{u_1(x)+u_2(x)}{\hat{H^2}\left(u_1(x)+u_2(x)\right)} \\
&= \frac{1}{4}\norm{u_1(x)+u_2(x)}{\hat{H}\left(\hat{H}u_1(x)+\hat{H}u_2(x)\right)} \\
&= \frac{1}{4}\norm{u_1(x)+u_2(x)}{\left(E_1\hat{H}u_1(x)+E_2\hat{H}u_2(x)\right)} \\
&= \frac{1}{4}\norm{u_1(x)+u_2(x)}{E_1^2u_1(x)+E_2^2u_2(x)} \\
&= \frac{1}{4}\left(E_1^2\norm{u_1(x)}{u_1(x)}+E_2^2\norm{u_2(x)}{u_2(x)}\right) \\
\expt{H^2} &= \frac{1}{4}\left(E_1^2+E_2^2\right)
\end{align*}
Now we can find the standard deviation $\sigma_H$
\begin{align*}
\sigma_H &= \sqrt{\expt{H^2} - \expt{H}^2}\\
&= \sqrt{\frac{1}{4}\left(E_1^2+E_2^2\right) - \frac{1}{16}\left(E_1+E_2\right)^2}\\
&= \frac{1}{2}\sqrt{E_1^2+E_2^2 - \frac{1}{4}\left(E_1^2+E_2^2+2E_1E_2\right)}\\
&= \frac{1}{4}\sqrt{3E_1^2+3E_2^2 + 2E_1E_2}\\
\end{align*}
\end{enumerate}

\section{Problem \#4}
Given an oil droplet of diameter $1.0\pm0.1\ \mu m$. We can apply the \emph{Uncertainty Principle} which we well estimate as
$$\Delta x\Delta p\approx\hbar$$
Note that we dropped the factor of two and we assume that we are at the minimum allowed uncertainty. So we say that $\Delta x$ is the uncertainty in the diameter of the oil drop so $\Delta x=0.1\ \mu m$. This implies
\begin{align*}
\Delta x\Delta p &= \hbar\\
\Delta p &= \frac{\hbar}{\Delta x}\\
&= \frac{1.05\times10^{-34}\ Js}{1\times10^{-7}\ m}\\
&= 1\times10^{-27}\ kg\ m\ s^{-1}
\end{align*}
Now to find the mass of the droplet we assume that volume of the droplet is about the radius cubed or $1\times10^{-19}\ m$ and we know that oil generally floats on top of water. So we assume that they have about the same density or about $1000\ kg\ m^{-3}$. So the mass of the droplet is $1\times10^{-16}\ kg$ so if we divide the mass out of $\Delta p$ to find the uncertainty in the velocity so
\begin{align*}
\Delta v &= \frac{\Delta p}{m}\\
&= \frac{1\times10^{-27}\ kg\ m\ s^{-1}}{1\times10^{-16}\ kg}\\
&= 1\times10^{-11}\ m\ s^{-1}
\end{align*}
This is a very small value and it would be very difficult to observe this uncertainty. Note that at this scale the uncertainty principle is negligible explaining such small values.

\section{Problem \#5}
\begin{enumerate}[(a)]
\item
Given the hermitian operators $\hat{A}$ and $\hat{B}$ which are a discrete spectrum. Note that each operator creates a separate complete orthonormal set of eigenfunctions where
$$\hat{A}\psi_n=a_n\psi_n$$
$$\hat{B}\phi_n=b_n\phi_n$$
So we can write any wavefunction as
$$\Psi(x,t) = \sum_nc_n(t)\psi_n(x) = \sum_nd_n(t)\phi_n(x)$$
We can find the expectation value for $\hat{A}$ by saying
\begin{align*}
\expt{\hat{A}} &= \int_{-\infty}^{\infty}\Psi(x,t)^*\hat{A}\Psi(x,t)dx\\
&= \int_{-\infty}^{\infty}\left(\sum_mc_m(t)\psi_m(x)\right)^*\sum_nc_n(t)\hat{A}\psi_n(x)dx\\
&= \int_{-\infty}^{\infty}\left(\sum_mc_m(t)\psi_m(x)\right)^*\sum_nc_n(t)a_n\psi_n(x)dx
\end{align*}
Note that $\psi_n$ forms an orthogonal basis, this implies that the sum is zero expect when $n=m$ so we see that
\begin{align*}
\int_{-\infty}^{\infty}\left(\sum_mc_m(t)\psi_m(x)\right)^*\sum_nc_n(t)a_n\psi_n(x)dx &= \sum_n\int_{-\infty}^{\infty}a_nc_n^*c_n\psi_n(x)^*\psi_n(x)dx\\
&= \sum_na_n|c_n|^2\cancelto{1}{\int_{-\infty}^{\infty}|\psi_n(x)|^2dx}\\
\expt{\hat{A}} &= \sum_na_n|c_n|^2
\end{align*}

\item
To find a relation between $c_n$ and $d_n$ we use the fact that $\Psi(x,t)$ is a normalizable function so we can say
\begin{align*}
1 &= \norm{\Psi(x,t)}{\Psi(x,t)}\\
&= \norm{\sum_nd_n\phi_n}{\sum_mc_m\psi_m}\\
&= \sum_n\sum_mc_md_n^*\norm{\phi_n}{\psi_m}
\end{align*}
Now we see if we want to find a specific $d_n$ we need to sum over $m$ so
\begin{align*}
d_n^{-1} &= \left(\sum_mc_m\norm{\phi_n}{\psi_m}\right)^*
\end{align*}

\item
To show that that $\hat{A}$'s \emph{spectral decomposition} is
$$\hat{A} = \sum_na_n\ket{\psi_n}\bra{\psi_n}$$
we can define a vector $\ket{f}$ which we can write in the basis formed by $\hat{A}$ as
$$\ket{f} = \sum_nh_n\ket{\psi_n}$$
So if we apply $\hat{A}$ to $\ket{f}$ we get
\begin{align*}
\hat{A}\ket{f} &= \sum_nh_n\hat{A}\ket{\psi_n}\\
&= \sum_nh_na_n\ket{\psi_n}
\end{align*}
Note that the pre-factor $h_n$ is found by taking the inner product 
$$h_n = \norm{\psi_n}{f}$$
So we see that
\begin{align*}
\sum_nh_na_n\ket{\psi_n} &= \sum_n\norm{\psi_n}{f}a_n\ket{\psi_n}\\
&= \sum_na_n\ket{\psi_n}\norm{\psi_n}{f}\\
\hat{A}\ket{f} &= \sum_na_n\ket{\psi_n}\bra{\psi_n}\ket{f}\\
\hat{A} &= \sum_na_n\ket{\psi_n}\bra{\psi_n}
\end{align*}
\end{enumerate}
\end{document}

