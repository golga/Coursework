\documentclass[11pt]{article}

\usepackage{latexsym}
\usepackage{amssymb}
\usepackage{amsthm}
\usepackage{enumerate}
\usepackage{amsmath}
\usepackage{cancel}
\numberwithin{equation}{section}

\setlength{\evensidemargin}{.25in}
\setlength{\oddsidemargin}{-.25in}
\setlength{\topmargin}{-.75in}
\setlength{\textwidth}{6.5in}
\setlength{\textheight}{9.5in}
\newcommand{\due}{April 14th, 2010}
\newcommand{\HWnum}{11}
\newcommand{\grad}{\bold\nabla}
\newcommand{\vecE}{\vec{E}}
\newcommand{\scrptR}{\vec{\mathfrak{R}}}
\newcommand{\kapa}{\frac{1}{4\pi\epsilon_0}}
\newcommand{\expt}[1]{\langle{#1}\rangle}
\newcommand{\norm}[2]{\langle{#1}|{#2}\rangle}
\newcommand{\ket}[1]{|{#1}\rangle}
\newcommand{\bra}[1]{\langle{#1}|}


\begin{document}
\begin{titlepage}
\setlength{\topmargin}{1.5in}
\begin{center}
\Huge{Physics 3310} \\
\LARGE{Principles of Electricity and Magnetism 1} \\
\Large{Professor Thomas R. Schibli} \\[1cm]

\huge{Homework \#\HWnum}\\[0.5cm]

\large{Joe Becker} \\
\large{SID: 810-07-1484} \\
\large{\due} 

\end{center}

\end{titlepage}



\section{Problem \#1}
\begin{enumerate}[(a)]
\item
For a particle of mass $m$ in an cubical box of edge-length $a$ we assume the solution to \emph{the Time-independent Schr\"{o}dinger's Equation}
\begin{equation}
-\frac{\hbar^2}{2m}\grad^2\psi+V\psi=E\psi
\label{Schro}
\end{equation}
for a potential 
$$V(\vec{r}) = \left\{\begin{array}{cc}
0	&0<x,y,z<a\\
\infty	&\textnormal{o.w.}
\end{array}\right.$$
is 
$$\psi(\vec{r}) = \left(\frac{2}{a}\right)^{3/2}\sin\left(\frac{n_x\pi}{a}x\right)\sin\left(\frac{n_y\pi}{a}y\right)\sin\left(\frac{n_z\pi}{a}z\right)$$
with a total energy given by
$$E = \frac{\hbar^2\pi^2}{2ma^2}(n_x^2+n_y^2+n_z^2)$$
we see that the first energy level $E_1$ is when $n_x=n_y=n_z=1$ or 
$$E_1 = 3\frac{\hbar^2\pi^2}{2ma^2} = 3\epsilon$$
where 
$$\epsilon\equiv\frac{\hbar^2\pi^2}{2ma^2}$$
The next energy level up $E_2$ is when $n_x=2$ $n_y=n_z=1$ or 
$$E_2 = 6\epsilon$$
note that this energy level has 3 degeneracies. Given by
$$\begin{array}{ccc}
n_x	&n_y	&n_z	\\
2	&1	&1	\\
1	&2	&1	\\
1	&1	&2	\\
\end{array}$$
The next four energy levels and their degeneracies are
$$E_3 = 9\epsilon$$
$$\begin{array}{ccc}
n_x	&n_y	&n_z	\\
2	&2	&1	\\
2	&1	&2	\\
1	&2	&2	\\
\end{array}$$
$$E_4 = 11\epsilon$$
$$\begin{array}{ccc}
n_x	&n_y	&n_z	\\
3	&1	&1	\\
1	&3	&1	\\
1	&1	&3	\\
\end{array}$$
$$E_5 = 12\epsilon$$
$$\begin{array}{ccc}
n_x	&n_y	&n_z	\\
2	&2	&2	\\
\end{array}$$
$$E_6 = 14\epsilon$$
$$\begin{array}{ccc}
n_x	&n_y	&n_z	\\
1	&2	&3	\\
1	&3	&2	\\
2	&1	&3	\\
2	&3	&1	\\
3	&1	&2	\\
3	&2	&1	\\
\end{array}$$
We see that degeneracies increase as the energy levels increase, but we see that the max number of degeneracies is 6 since that is the maximum number of ways to permute three values. So we assume that the values will cycle through degeneracies at different levels.

\item
For the potential
$$V(\vec{r}) = \left\{\begin{array}{cc}
0	&0<x<a, 0<y<b, 0<z<c\\
\infty	&\textnormal{o.w.}
\end{array}\right.$$
We see that the energy is given by
$$E = \frac{\hbar^2\pi^2}{2m}\left(\frac{n_x^2}{a^2}+\frac{n_y^2}{b^2}+\frac{n_z^2}{c^2}\right)$$

\item
If we assume that $b=2a$ and $c=3a$ we see that the energy is given by
$$E = \frac{\hbar^2\pi^2}{2ma^2}\left(n_x^2+\frac{n_y^2}{4}+\frac{n_z^2}{9}\right)$$
we see that the first energy level is still $n_x=n_y=n_z=1$ or
$$E_1 = \frac{49}{36}\epsilon$$
note that we only have one degeneracy. The next energy level is when $n_x=n_y=1$ $n_z=2$ or
$$E_2 = \frac{53}{36}\epsilon$$
again this level only has one degeneracy. The next energy level is when $n_x=n_y=1$ and $n_z=3$ or
$$E_3 = \frac{57}{36}\epsilon$$
again there is only one degeneracy. 
For $n_x=n_z=1$ $n_y=2$
$$E_4 = \frac{58}{36}\epsilon$$
For $n_x=1$ $n_y=n_z=2$
$$E_5 = \frac{62}{36}\epsilon$$
Note that non of these energy levels have degeneracies. We see that in a asymmetric system we remove the degeneracies of the energies.
\end{enumerate}

\section{Problem \#2}
\begin{enumerate}[(a)]
\item
If we assume that the electron in a metal is like an electron of particle in a box. So if we have a block of metal of side $a=0.1\ mm$ we see that the energy is given by 
$$E = n^2\epsilon$$
where 
$$n^2 = n_x^2+n_y^2+n_z^3$$
we say that the value of $\epsilon$ for $a=0.1\ mm$ is 
\begin{align*}
\epsilon &= \frac{\hbar^2\pi^2}{2m_ea^2}\\
&= \frac{(1.05\times10^{-34})^2(3.14)^2}{2(9.11\times10^{-31})(0.0001)^2}\\
&= 5.97\times10^{-30}\ J\ \frac{1\ eV}{1.6\times10^{-19}\ J} = 3.73\times10^{-11}\ eV
\end{align*}
So we can find the value of $n$ as
\begin{align*}
E &= n^2\epsilon\\
n &= \sqrt{\frac{E}{\epsilon}}\\
&= \sqrt{\frac{3\ eV}{3.73\times10^{-11}\ eV}}\\
&= 2.84
\end{align*}

\item
\item
\item
\end{enumerate}

\section{Problem \#3}
\begin{enumerate}[(a)]
\item
To show that we can represent the operator $L_z$ by using spherical coordinates as
$$L_z = -i\hbar\frac{\partial}{\partial\varphi}$$
we apply the \emph{Chain Rule} which states
$$\frac{\partial}{\partial\varphi} = \frac{\partial x}{\partial\phi}\frac{\partial}{\partial x} + \frac{\partial y}{\partial\phi}\frac{\partial}{\partial y} + \frac{\partial z}{\partial\phi}\frac{\partial}{\partial z}$$
Now with the identities 
$$x = r\sin\theta\cos\varphi$$
$$y = r\sin\theta\sin\varphi$$
$$z = r\cos\theta$$
we see that 
$$\frac{\partial x}{\partial\varphi} = -r\sin\theta\sin\varphi = -y$$
$$\frac{\partial y}{\partial\varphi} = r\sin\theta\cos\varphi= x$$
$$\frac{\partial z}{\partial\varphi} = 0$$
So we see that 
\begin{align*}
\frac{\partial}{\partial\varphi} &= \frac{\partial x}{\partial\phi}\frac{\partial}{\partial x} + \frac{\partial y}{\partial\phi}\frac{\partial}{\partial y} + \cancelto{0}{\frac{\partial z}{\partial\phi}\frac{\partial}{\partial z}}\\
&= -y\frac{\partial}{\partial x} + x\frac{\partial}{\partial y}\\
&=  x\frac{\partial}{\partial y}-y\frac{\partial}{\partial x}\\
&= L_z
\end{align*}

\item
To find the uncertainty relation $\sigma_{Lz}\sigma_{\varphi}$ we need to use 
\begin{equation}
\sigma_A\sigma_B \ge \frac{1}{2i}\langle[\hat{A},\hat{B}]\rangle
\label{Uncert}
\end{equation}
So we need to find how $\varphi$ commutes with $L_z$ so
\begin{align*}
[L_z,\varphi]f(\varphi) &= (L_z\varphi - \varphi L_z)f(\varphi)\\
&= -i\hbar\frac{\partial}{\partial\varphi}\varphi f(\varphi) +  i\hbar\varphi\frac{\partial f(\varphi)}{\partial\varphi}\\
&= -i\hbar f(\varphi)- i\hbar\varphi\frac{\partial f(\varphi)}{\partial\varphi} +  i\hbar\varphi\frac{\partial f(\varphi)}{\partial\varphi}\\
[\varphi, L_z] &= i\hbar
\end{align*}
So we see that equation \ref{Uncert} yields
\begin{align*}
\sigma_{L_z}\sigma_{\varphi} &\ge \frac{1}{2i}\langle[\hat{\varphi},\hat{L_z}]\rangle\\
&\ge \frac{1}{2i}i\hbar\\
\sigma_{L_z}\sigma_{\varphi} &\ge \frac{\hbar}{2}
\end{align*}
This is very simular to the uncertianty principle in one dimension. Note that we dropped the negative because equation \ref{Uncert} is derived from a square and we always take the postive value.

\item
Given the definition of the operator $L_{\pm}$
$$\hat{L}_{\pm} = \hat{L}_x\pm i\hat{L}_y$$
we can convert $\hat{L}_{\pm}$ into a function of $\theta$ and $\varphi$ by using the relations
$$L_{x} = \frac{\hbar}{i}\left(-\sin\varphi\frac{\partial}{\partial\theta}-\cos\varphi\cot\theta\frac{\partial}{\partial\varphi}\right)$$
$$L_{x} = \frac{\hbar}{i}\left(\cos\varphi\frac{\partial}{\partial\theta}-\sin\varphi\cot\theta\frac{\partial}{\partial\varphi}\right)$$
So we can say that $\hat{L}_{\pm}$ is given by
\begin{align*}
\hat{L}_{\pm} &= \hat{L}_x \pm i\hat{L}_y\\
&= \frac{\hbar}{i}\left(-\sin\varphi\frac{\partial}{\partial\theta}-\cos\varphi\cot\theta\frac{\partial}{\partial\varphi}\right) \pm i\frac{\hbar}{i}\left(\cos\varphi\frac{\partial}{\partial\theta}-\sin\varphi\cot\theta\frac{\partial}{\partial\varphi}\right)\\
&= \frac{\hbar}{i}\left(-\sin\varphi\frac{\partial}{\partial\theta}-\cos\varphi\cot\theta\frac{\partial}{\partial\varphi} \pm i\cos\varphi\frac{\partial}{\partial\theta}\mp i\sin\varphi\cot\theta\frac{\partial}{\partial\varphi}\right)\\
&= \frac{\hbar}{i}\left((-\sin\varphi\pm i\cos\varphi)\frac{\partial}{\partial\theta}-(\cos\varphi \pm i\sin\varphi)\cot\theta\frac{\partial}{\partial\varphi}\right)\\
\end{align*}




\item
\end{enumerate}

\section{Problem \#4}
\begin{enumerate}[(a)]
\item
\item
\item
\end{enumerate}

\end{document}

