\documentclass[11pt]{article}

\usepackage{latexsym}
\usepackage{amssymb}
\usepackage{amsthm}
\usepackage{enumerate}
\usepackage{amsmath}
\usepackage{cancel}
\numberwithin{equation}{section}

\setlength{\evensidemargin}{.25in}
\setlength{\oddsidemargin}{-.25in}
\setlength{\topmargin}{-.75in}
\setlength{\textwidth}{6.5in}
\setlength{\textheight}{9.5in}
\newcommand{\due}{April 21st, 2010}
\newcommand{\HWnum}{12}
\newcommand{\grad}{\bold\nabla}
\newcommand{\vecE}{\vec{E}}
\newcommand{\scrptR}{\vec{\mathfrak{R}}}
\newcommand{\kapa}{\frac{1}{4\pi\epsilon_0}}
\newcommand{\expt}[1]{\langle{#1}\rangle}
\newcommand{\norm}[2]{\langle{#1}|{#2}\rangle}
\newcommand{\ket}[1]{|{#1}\rangle}
\newcommand{\bra}[1]{\langle{#1}|}


\begin{document}
\begin{titlepage}
\setlength{\topmargin}{1.5in}
\begin{center}
\Huge{Physics 3310} \\
\LARGE{Principles of Electricity and Magnetism 1} \\
\Large{Professor Thomas R. Schibli} \\[1cm]

\huge{Homework \#\HWnum}\\[0.5cm]

\large{Joe Becker} \\
\large{SID: 810-07-1484} \\
\large{\due} 

\end{center}

\end{titlepage}



\section{Problem \#1}
\begin{enumerate}[(a)]
\item
We see that the radial part of the \emph{Scr\"{o}dinger Equation} is given by
\begin{equation}
\frac{d}{dr}\left(r^2\frac{dR}{dr}\right)-\frac{2mr^2}{\hbar^2}[V(r)-E]R = l(l+1)R
\label{RadSchro}
\end{equation}
Now if we define a new function $u(r)$ where
$$u(r)\equiv rR(r)$$
note that
$$\frac{du}{dr} = R+r\frac{dR}{dr}$$
and
$$\frac{d^2u}{dr^2} = 2\frac{dR}{dr}+r\frac{d^2R}{dr^2}$$
now we can say that
\begin{align*}
r\frac{d^2u}{dr^2} &= 2r\frac{dR}{dr}+r^2\frac{d^2R}{dr^2}\\
&= \frac{d}{dr}\left(r^2\frac{dR}{dr}\right)
\end{align*}
from the \emph{Chain rule}. So we see that equation \ref{RadSchro} becomes
\begin{align*}
r\frac{d^2u}{dr^2}-\frac{2mr}{\hbar^2}[V(r)-E]u &= \frac{l(l+1)}{r}u\\
\frac{d^2u}{dr^2}-\frac{2m}{\hbar^2}[V(r)-E]u &= \frac{l(l+1)}{r^2}u\\
\frac{d^2u}{dr^2}-\frac{2m}{\hbar^2}V(r)u - \frac{l(l+1)}{r^2}u &=Eu\\
-\frac{\hbar^2}{2m}\frac{d^2u}{dr^2}+\left[V(r) + \frac{\hbar^2}{2m}\frac{l(l+1)}{r^2}\right]u &=Eu
\end{align*}
Now for the infinite spherical well potential
$$V(r) = \left\{\begin{array}{lc}
0	&r\le a\\
\infty	&r> a
	\end{array}\right.$$
We see that for $r\le a$
\begin{align*}
-\frac{\hbar^2}{2m}\frac{d^2u}{dr^2}+\left[\frac{\hbar^2}{2m}\frac{l(l+1)}{r^2}\right]u &=Eu\\
\frac{\hbar^2}{2m}\frac{d^2u}{dr^2} &= \frac{\hbar^2}{2m}\frac{l(l+1)}{r^2}u - Eu\\
\frac{d^2u}{dr^2} &= \frac{l(l+1)}{r^2}u - \frac{2mE}{\hbar^2}u
\end{align*}
So we have the differential equation 
$$\frac{d^2u}{dr^2} = \left[\frac{l(l+1)}{r^2} - k^2\right]u$$
where
$$k\equiv\frac{\sqrt{2mE}}{\hbar}$$
now if we assume that the angular momentum is zero or $l=0$ we get
$$\frac{d^2u}{dr^2} = -k^2u$$
this equation implies that the solution is a combination of sines and cosines or
$$u(r) = A\sin(kr)+B\cos(kr)$$
Note that we are concerned with $R(r)$ so the solution to equation \ref{RadSchro} is
$$R(r) = \frac{A\sin(kr)}{r} + \frac{B\cos(kr)}{r}$$
we see that for $r=0$ this equation goes to infinity if $B\ne0$ so we can say that
$$R(r) = \frac{A\sin(kr)}{r}$$
now if we apply the next boundary condition $R(r=a) = 0$ we see that 
$$\sin(ka) = 0$$
this implies that
$$ka = n\pi$$
where $n=1,2,3,...$ so we see that
$$k = \frac{n\pi}{a}$$
Now to normalize $R(r)$ we note that $u(r)$ shares the same constant $A$ so if we normalize $u(r)$ we have normalized $R(r)$. So
\begin{align*}
1 = \int_{0}^{\infty}|u(r)|^2dr &= A^2\int_0^{a}\sin^2(kr)dr\\
&= \frac{A^2}{2}\int_0^{a}1-\cos(2kr)dr\\
&= \frac{A^2}{2}\left(a-\right.\frac{1}{2k}\sin(2kr)\left|_0^a\right)\\
&= A^2\frac{a}{2}\\
A &= \sqrt{\frac{2}{a}}
\end{align*}
So the radial part of the wave function is
$$R(r) = \sqrt{\frac{2}{a}}\frac{\sin(n\pi r/a)}{r}$$
to find the whole wave function $\psi$ we recall that 
$$\psi(\vec{r}) = R(r)Y_l^m(\theta,\phi)$$
where $Y_l^m(\theta,\phi)$ is the spherical harmonic given by
$$Y_l^m(\theta,\phi) = (-1)^m\sqrt{\frac{(2l+1)}{4\pi}\frac{(l-|m|)!}{(l+|m|)!}}e^{im\phi}P_l^m(\cos\theta)$$
note that $P_0^0=1$, so we can see that for $l=0$ (this requires $m=0$) we get that
$$Y_0^0 = \frac{1}{\sqrt{4\pi}}$$
So we can see that
$$\psi_{n00} = \frac{1}{\sqrt{2\pi a}}\frac{\sin(n\pi r/a)}{r}$$
Note that the eigenvalue $E$ is given by 
\begin{align*}
k^2 &= \frac{2mE}{\hbar^2}\\
\frac{n^2\pi^2}{a^2} &= \frac{2mE}{\hbar^2}\\
E &= \frac{n^2\pi^2\hbar^2}{2ma^2}
\end{align*}

\item
Given the radial probability density $P(r)$ where 
$$P(r)dr = 4\pi r^2|\psi(r)|^2dr$$
where 
$$\psi_n(r) = \sqrt{\frac{2}{\pi}}\frac{\sin(n\pi r/a)}{r}$$
So 
$$P_n(r) = 8\sin^2(n\pi r/a)$$
See attached for sketches of $P_1(r)$, $P_2(r)$, and $P_3(r)$.
\end{enumerate}

\section{Problem \#2}
\begin{enumerate}[(a)]
\item
Given \emph{Time-Independent Schr\"{o}dinger's Equation} in 3 dimensions
$$-\frac{\hbar^2}{2m}\grad^2\psi+V\psi = E\psi$$
but if we assume that a spherically-symmetric wavefunction 
$$\psi(r,\theta,\phi) = Ae^{-r/a}$$
we only need to take into account the radial equation to solve for the constant $a$. 
$$\frac{d}{dr}\left(r^2\frac{dR}{dr}\right)-\frac{2mr^2}{\hbar^2}[V(r)-E]R = l(l+1)R$$

\item
\item
\item
\end{enumerate}

\section{Problem \#3}
\begin{enumerate}[(a)]
\item
\item
\item
\item
\item
\item
\end{enumerate}

\section{Problem \#4}
\begin{enumerate}[(a)]
\item
\item
\item
\end{enumerate}

\section{Problem \#5}
\begin{enumerate}[(a)]
\item
\item
\item
\end{enumerate}

\end{document}

