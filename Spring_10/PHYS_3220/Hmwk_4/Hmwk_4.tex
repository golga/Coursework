\documentclass[11pt]{article}

\usepackage{latexsym}
\usepackage{amssymb}
\usepackage{amsthm}
\usepackage{enumerate}
\usepackage{amsmath}
\usepackage{cancel}
\numberwithin{equation}{section}

\setlength{\evensidemargin}{.25in}
\setlength{\oddsidemargin}{-.25in}
\setlength{\topmargin}{-.75in}
\setlength{\textwidth}{6.5in}
\setlength{\textheight}{9.5in}
\newcommand{\due}{February 3rd, 2010}
\newcommand{\HWnum}{4}
\newcommand{\grad}{\bold\nabla}
\newcommand{\vecE}{\vec{E}}
\newcommand{\scrptR}{\vec{\mathfrak{R}}}
\newcommand{\kapa}{\frac{1}{4\pi\epsilon_0}}
\newcommand{\expt}[1]{\langle{#1}\rangle}

\begin{document}
\begin{titlepage}
\setlength{\topmargin}{1.5in}
\begin{center}
\Huge{Physics 3310} \\
\LARGE{Principles of Electricity and Magnetism 1} \\
\Large{Professor Thomas R. Schibli} \\[1cm]

\huge{Homework \#\HWnum}\\[0.5cm]

\large{Joe Becker} \\
\large{SID: 810-07-1484} \\
\large{\due} 

\end{center}

\end{titlepage}



\section{Problem \#1}
\begin{enumerate}[(a)]
\item
We assume that the wave function of an infinite square well has a solution that is separable that is it is in the form $\Psi(x,t) = \psi(x)\varphi(t)$. Where $\varphi(t)$ is the solution to the differential equation 
$$i\hbar\frac{d\varphi}{dt} = E\varphi$$
where $E$ is a constant. So we see that this solution is an exponential or
$$\varphi(t) = e^{-iE_nt/\hbar}$$
So the wave function for all time $t$ is given as
$$\Psi(x,t) = \sqrt{\frac{2}{a}}\sin\left(\frac{n\pi x}{a}\right)e^{-iE_nt/\hbar}$$
where the energy $E_n$ is given by
$$E_n = \frac{n^2\pi^2\hbar^2}{2ma^2}$$

\item
To find the expectation of x $\langle x\rangle$ we calculate
$$\langle x\rangle = \int_{-\infty}^{\infty}\Psi^*x\Psi dx$$
where we use $\Psi(x,t)$ from part (a). So we calculate $\langle x\rangle$ as
\begin{align*}
\langle x\rangle &= \int_{-\infty}^{\infty}\Psi^*x\Psi dx\\
&= \int_{0}^{a}\sqrt{\frac{2}{a}}\sin\left(\frac{n\pi x}{a}\right)e^{iE_nt/\hbar}x\sqrt{\frac{2}{a}}\sin\left(\frac{n\pi x}{a}\right)e^{-iE_nt/\hbar}dx\\
&= {\frac{2}{a}}\int_{0}^{a}\sin^2\left(\frac{n\pi x}{a}\right)e^{iE_nt/\hbar-iE_nt/\hbar}xdx\\
&= {\frac{2}{a}}\int_{0}^{a}\sin^2\left(\frac{n\pi x}{a}\right)xdx
\end{align*}
To evaluate this integral we need to use integration by parts where
$$u = x\ du = 1$$
and using the half angle relation 
$$dv = \sin^2\left(\frac{n\pi x}{a}\right) = \frac{1}{2} - \frac{1}{2}\cos\left(\frac{2n\pi x}{a}\right)$$
and we can calculate 
$$v = \frac{1}{2}x - \frac{a}{4n\pi}\sin\left(\frac{2n\pi x}{a}\right)$$
so using
\begin{equation}
\int udv = uv -\int vdu
\label{intparts}
\end{equation}
So applying equation \ref{intparts} to our integral becomes 
\begin{align*}
\frac{2}{a}\int_{0}^{a}\sin^2\left(\frac{n\pi x}{a}\right)xdx &= \frac{2}{a}\left(\frac{1}{2}x^2 - \frac{a}{4n\pi}\sin\left(\frac{2n\pi x}{a}\right)\right|_0^a  - \frac{1}{a}\int_{0}^{a}x-\frac{a}{4n\pi}\sin\left(\frac{2n\pi x}{a}\right)dx\\
&= \frac{2}{a}\left(\frac{1}{2}a^2\right) - \frac{1}{a}\left(\frac{1}{2}x^2+\frac{a^2}{8n^2\pi^2}\cos\left(\frac{2n\pi x}{a}\right)\right|_0^a\\
&= a - \frac{1}{a}\frac{1}{2}a^2\\
&= a - \frac{a}{2}\\
\langle x\rangle &= \frac{a}{2}
\end{align*}
So we see that the expectation is halfway down the well. This means that the most probable location for the particle is in the middle of the wave. To find $\langle x^2\rangle$ we need to find the integral 
\begin{align*}
\langle x^2\rangle &= \int_{-\infty}^{\infty}\Psi^*x^2\Psi dx\\
&= {\frac{2}{a}}\int_{0}^{a}\sin^2\left(\frac{n\pi x}{a}\right)x^2dx
\end{align*}
Again we use integration by parts where
$$u = x^2\ du = 2x$$
and
$$dv = \sin^2\left(\frac{n\pi x}{a}\right) = \frac{1}{2} - \frac{1}{2}\cos\left(\frac{2n\pi x}{a}\right)$$
$$v = \frac{1}{2}x - \frac{a}{4n\pi}\sin\left(\frac{2n\pi x}{a}\right)$$
again we apply equation \ref{intparts} to get
\begin{align*}
{\frac{2}{a}}\int_{0}^{a}\sin^2\left(\frac{n\pi x}{a}\right)x^2dx &= \frac{2}{a}\left(\frac{1}{2}x^3 - \cancelto{0}{\frac{ax^2}{4n\pi}\sin\left(\frac{2n\pi x}{a}\right)}\right|_0^a - \frac{2}{a}\int_{0}^{a} x^2 - \frac{ax}{2n\pi}\sin\left(\frac{2n\pi x}{a}\right) dx\\
&= \frac{2}{a}\frac{a^3}{2} - \left(\frac{2}{a}\frac{1}{3}x^3\right|_0^a - \frac{2}{a}\int_0^a\frac{ax}{2n\pi}\sin\left(\frac{2n\pi x}{a}\right)dx\\
&= a^2 - \frac{2}{a}\frac{a^3}{3} + \frac{1}{n\pi}\int_0^a x\sin\left(\frac{2n\pi x}{a}\right)dx\\
&= a^2 - \frac{2a^2}{3} + \frac{1}{n\pi}\int_0^a x\sin\left(\frac{2n\pi x}{a}\right)dx\\
&= \frac{a^2}{3} + \frac{1}{n\pi}\int_0^a x\sin\left(\frac{2n\pi x}{a}\right)dx
\end{align*}
Now we use integration by parts again where
$$u = x\ du = 1$$
and 
$$dv = \sin\left(\frac{2n\pi x}{a}\right)$$
where
$$v = -\frac{a}{2n\pi}\cos\left(\frac{2n\pi x}{a}\right)$$
so equation \ref{intparts} gives us
\begin{align*}
\frac{1}{n\pi}\int_0^a x\sin\left(\frac{2n\pi x}{a}\right)dx &= \frac{1}{n\pi}\left(-\frac{ax}{2n\pi}\cos\left(\frac{2n\pi x}{a}\right)\right|_0^a + \frac{1}{n\pi}\frac{a}{2n\pi}\int_0^a \cos\left(\frac{2n\pi x}{a}\right)dx\\
&= \frac{-a}{2n^2\pi^2}\left(a\cos\left(\frac{2n\pi a}{a}\right) - 0\right) + \cancelto{0}{\frac{a}{2n^2\pi^2}\frac{a}{2n\pi}\left(\sin\left(\frac{2n\pi x}{a}\right)\right|_0^a}\\
&= \frac{-a^2}{2n^2\pi^2}
\end{align*}
So we can say that
$$\langle x^2\rangle = \frac{a^2}{3} - \frac{a^2}{2n^2\pi^2}$$
To find the expectation of the momentum we use the momentum operator $\hat{p}$ where
\begin{equation}
\hat{p} = \frac{\hbar}{i}\frac{\partial}{\partial x}
\label{momop}
\end{equation}
So to calculate $\expt{p}$ we say that
\begin{align*}
\expt{p} &= \int_{-\infty}^{\infty}\Psi^*\frac{\hbar}{i}\frac{\partial}{\partial x}\Psi dx\\
&= \frac{2}{a}\int_{0}^{a}\sin\left(\frac{n\pi x}{a}\right)e^{iE_nt/\hbar}\frac{\hbar}{i}\frac{\partial}{\partial x}\sin\left(\frac{n\pi x}{a}\right)e^{-iE_nt/\hbar}dx\\
&= \frac{2}{a}\int_{0}^{a}\sin\left(\frac{n\pi x}{a}\right)\frac{\hbar}{i}\frac{n\pi}{a}\cos\left(\frac{n\pi x}{a}\right)dx\\
&= \frac{2\hbar n\pi}{ia^2}\int_{0}^{a}\sin\left(\frac{n\pi x}{a}\right)\cos\left(\frac{n\pi x}{a}\right)dx
\end{align*}
Now we can use the trig identity 
\begin{equation}
\sin(2\theta) = 2\cos(\theta)\sin(\theta)
\end{equation}
So our integral becomes 
\begin{align*}
\frac{2\hbar n\pi}{ia^2}\int_{0}^{a}\sin\left(\frac{n\pi x}{a}\right)\cos\left(\frac{n\pi x}{a}\right)dx &= \frac{2\hbar n\pi}{ia^2}\frac{1}{2}\int_{0}^{a}\sin\left(\frac{2n\pi x}{a}\right)dx\\
&= \frac{\hbar n\pi}{ia^2}\frac{a}{2n\pi}\left(\cos\left(\frac{2n\pi x}{a}\right)\right|_0^a\\
&= \frac{\hbar n\pi}{ia^2}\frac{a}{2n\pi}\left(\cos\left(\frac{2n\pi a}{a}\right) - \cos\left(\frac{2n\pi 0}{a}\right)\right)\\
&= \frac{\hbar n\pi}{ia^2}\frac{a}{2n\pi}\left(\cos\left(2n\pi\right) - \cos\left(0\right)\right)\\
&= \frac{\hbar n\pi}{ia^2}\frac{a}{2n\pi}\left(1-1\right)\\
\expt{p} &= 0
\end{align*}
So expectation of the momentum is zero. This makes sense because we are in a stationary state so the probability does not change in time. Now to calculate $\expt{p^2}$ we square the momentum operator (equation \ref{momop}) or
$$\hat{p}^2 = -\hbar^2\frac{\partial^2}{\partial x^2}$$
so we calculate $\expt{p^2}$. Note that we already canceled the time part because the is no $x$ dependence of we can pull it out of the derivative.
\begin{align*}
\expt{p^2} &= -\int_{-\infty}^{\infty}\Psi^*{\hbar^2}\frac{\partial^2}{\partial x^2}\Psi dx\\
&= -\frac{2\hbar^2}{a}\int_{0}^{a}\sin\left(\frac{n\pi x}{a}\right)\frac{\partial^2}{\partial x^2}\sin\left(\frac{n\pi x}{a}\right)dx\\
&= \frac{2\hbar^2}{a}\frac{n^2\pi^2}{a^2}\int_{0}^{a}\sin\left(\frac{n\pi x}{a}\right)\sin\left(\frac{n\pi x}{a}\right)dx\\
&= \frac{2\hbar^2 n^2\pi^2}{a^3}\int_{0}^{a}\sin^2\left(\frac{n\pi x}{a}\right)dx\\
&= \frac{2\hbar^2 n^2\pi^2}{a^3}\frac{1}{2}\int_{0}^{a}1-\cos\left(\frac{2n\pi x}{a}\right)dx\\
&= \frac{\hbar^2 n^2\pi^2}{a^3}\left(x-\frac{a}{2n\pi}\sin\left(\frac{2n\pi x}{a}\right)\right|_{0}^{a}\\
&= \frac{\hbar^2 n^2\pi^2}{a^3}\left(a-\cancelto{0}{\frac{a}{2n\pi}\sin\left(\frac{2n\pi a}{a} \right)}-0\right)\\
&= \frac{\hbar^2 n^2\pi^2}{a^3}a\\
&= \frac{\hbar^2 n^2\pi^2}{a^2}
\end{align*}

\item
To find the standard deviations of $x$ and $p$ we use 
\begin{equation}
\sigma_{Q} = \sqrt{\expt{Q^2}-\expt{Q}^2}
\label{stdev}
\end{equation}
So we calculate $\sigma_x$ as
\begin{align*}
\sigma_{x} &= \sqrt{\expt{x^2}-\expt{x}^2}\\
&= \sqrt{\frac{a^2}{3} - \frac{a^2}{2n^2\pi^2} - \frac{a^2}{2^2}}\\
&= \sqrt{\frac{1}{3} - \frac{1}{2n^2\pi^2} - \frac{1}{2^2}}\\
&= a\sqrt{\frac{8n^2\pi^2-12-6n^2\pi^2}{24n^2\pi^2}}\\
&= \frac{a}{n\pi}\sqrt{\frac{8n^2\pi^2-12-6n^2\pi^2}{24}}
\end{align*}
And we calculate $\sigma_p$ as
\begin{align*}
\sigma_{p} &= \sqrt{\expt{p^2}-\expt{p}^2}\\
&= \sqrt{\frac{\hbar^2 n^2\pi^2}{a^2}-0^2}\\
&= \sqrt{\frac{\hbar^2 n^2\pi^2}{a^2}}\\
&= \frac{\hbar n\pi}{a}
\end{align*}

\item
Now if we apply \emph{Heisenberg's Uncertainty Principle}
\begin{equation}
\sigma_x\sigma_p \ge\frac{\hbar}{2}
\label{HeisUP}
\end{equation}
to this system we see that
\begin{align*}
\sigma_x\sigma_p &= \frac{a}{n\pi}\sqrt{\frac{8n^2\pi^2-12-6n^2\pi^2}{24}} \frac{\hbar n\pi}{a}\\
\sigma_x\sigma_p &= \hbar\sqrt{\frac{8n^2\pi^2-12-6n^2\pi^2}{24}}
\end{align*}
So if we apply equation \ref{HeisUP} 
$$\hbar\sqrt{\frac{8n^2\pi^2-12-6n^2\pi^2}{24}} \ge \frac{\hbar}{2}$$
we see that the $\hbar$s cancel so we are left with
$$\sqrt{\frac{8n^2\pi^2-12-6n^2\pi^2}{24}} \ge \frac{1}{2}$$
Note that if $n=0$ we see that we are at about $1/2$, but $n=0$ is non-physical not to mention that it is imaginary value. So the stationary state that is closest to the lower bound is $n=1$.
\end{enumerate}

\section{Problem \#2}
\begin{enumerate}[(a)]
\item
Recall from problem 1 that
$$\psi_n(x) = \sqrt{\frac{2}{a}}\sin\left(\frac{n\pi x}{a}\right)$$
So if we have a wavefunction of the form 
$$\Psi(x,0) = A(\psi_2(x)+\psi_3(x))$$
we can normalize the function by saying
$$1 = \int_{-\infty}^{\infty}\Psi^*\Psi dx$$
where
\begin{align*}
\Psi^*\Psi &= A\left(\sqrt{\frac{2}{a}}\sin\left(\frac{2\pi x}{a}\right) + \sqrt{\frac{2}{a}}\sin\left(\frac{3\pi x}{a}\right)\right)A\left(\sqrt{\frac{2}{a}}\sin\left(\frac{2\pi x}{a}\right) + \sqrt{\frac{2}{a}}\sin\left(\frac{3\pi x}{a}\right)\right)\\
&= A^2\frac{2}{a}\left(\sin\left(\frac{2\pi x}{a}\right) + \sin\left(\frac{3\pi x}{a}\right)\right)^2\\
&= A^2\frac{2}{a}\left(\sin^2\left(\frac{2\pi x}{a}\right) + \sin^2\left(\frac{3\pi x}{a}\right) + 2\sin\left(\frac{2\pi x}{a}\right)\sin\left(\frac{3\pi x}{a}\right)\right)
\end{align*}
So we can take the integral over all space
\begin{align*}
1 &= \int_{-\infty}^{\infty}\Psi^*\Psi dx\\
&= A^2\frac{2}{a}\int_{0}^{a}\sin^2\left(\frac{2\pi x}{a}\right)dx + \int_{0}^{a}\sin^2\left(\frac{3\pi x}{a}\right)dx + \int_{0}^{a}2\sin\left(\frac{2\pi x}{a}\right)\sin\left(\frac{3\pi x}{a}\right)dx
\end{align*}
Note that 
$$\int_{0}^{a}2\sin\left(\frac{2\pi x}{a}\right)\sin\left(\frac{3\pi x}{a}\right)dx = 0$$
because sine is an orthogonal function and over the interval $[0,a]$ (for this $k_n$) the integral is zero due to the fact that we have different indexes. So
\begin{align*}
&= A^2\frac{2}{a}\int_{0}^{a}\sin^2\left(\frac{2\pi x}{a}\right)dx + \int_{0}^{a}\sin^2\left(\frac{3\pi x}{a}\right)dx + \cancelto{0}{\int_{0}^{a}2\sin\left(\frac{2\pi x}{a}\right)\sin\left(\frac{3\pi x}{a}\right)dx}\\
&= A^2\frac{2}{a}\left(\frac{1}{2}\int_{0}^{a}1-\cos\left(\frac{4\pi x}{a}\right)dx + \frac{1}{2}\int_{0}^{a}1-\cos\left(\frac{6\pi x}{a}\right)dx\right) \\
&= A^2\frac{2}{a}\frac{1}{2}\left(x-\frac{a}{4\pi}\sin\left(\frac{4\pi x}{a}\right)\right|_0^a + \left(x-\frac{a}{6\pi}\sin\left(\frac{6\pi x}{a}\right)\right|_0^a \\
&= \frac{A^2}{a}\left(a-\frac{a}{4\pi}\sin\left(\frac{4\pi a}{a}\right)-0\right) + \left(a-\frac{a}{6\pi}\sin\left(\frac{6\pi a}{a}\right)-0\right) \\
&= \frac{A^2}{a}(a + a) \\
&= \frac{A^2}{a}2a \\
1 &= 2A^2 \\
A &= \frac{\sqrt{2}}{2} 
\end{align*}

\item
So from problem one we know that the time dependent part of $\Psi_n(x,t)$ is given by
$$\varphi(t) = e^{-iE_nt/\hbar}$$
So we can say that our wavefunction is
$$\Psi(x,t) = \frac{1}{\sqrt{a}}\left(\sin\left(\frac{2\pi x}{a}\right)e^{-iE_2t/\hbar} + \sin\left(\frac{3\pi x}{a}\right)e^{-iE_3t/\hbar}\right)$$
Now we can find $|\Psi(x,t)|^2$ by using the relation 
\begin{equation}
|\Psi(x,t)|^2 = \Psi^*(x,t)\Psi(x,t)
\end{equation}
So we calculate
\begin{align*}
\Psi^*(x,t)\Psi(x,t) &= \frac{1}{a}\left(\sin\left(\frac{2\pi x}{a}\right)e^{iE_2t/\hbar} + \sin\left(\frac{3\pi x}{a}\right)e^{iE_3t/\hbar}\right)\left(\sin\left(\frac{2\pi x}{a}\right)e^{-iE_2t/\hbar} + \sin\left(\frac{3\pi x}{a}\right)e^{-iE_3t/\hbar}\right)\\
&= \frac{1}{a}\left(\sin\left(\frac{2\pi x}{a}\right)e^{iE_2t/\hbar}\sin\left(\frac{2\pi x}{a}\right)e^{-iE_2t/\hbar} + \sin\left(\frac{2\pi x}{a}\right)e^{iE_2t/\hbar}\sin\left(\frac{3\pi x}{a}\right)e^{-iE_3t/\hbar} \right.\\
&\left.+ \sin\left(\frac{3\pi x}{a}\right)e^{iE_3t/\hbar}\sin\left(\frac{2\pi x}{a}\right)e^{-iE_2t/\hbar} + \sin\left(\frac{3\pi x}{a}\right)e^{iE_3t/\hbar}\sin\left(\frac{3\pi x}{a}\right)e^{-iE_3t/\hbar}\right)\\
&= \frac{1}{a}\left(\sin^2\left(\frac{2\pi x}{a}\right) + \sin^2\left(\frac{3\pi x}{a}\right) \right.\\
&\left.+\sin\left(\frac{2\pi x}{a}\right)e^{iE_2t/\hbar}\sin\left(\frac{3\pi x}{a}\right)e^{-iE_3t/\hbar} + \sin\left(\frac{3\pi x}{a}\right)e^{iE_3t/\hbar}\sin\left(\frac{2\pi x}{a}\right)e^{-iE_2t/\hbar}\right)\\
&= \frac{1}{a}\left(\sin^2\left(\frac{2\pi x}{a}\right) + \sin^2\left(\frac{3\pi x}{a}\right) + \sin\left(\frac{2\pi x}{a}\right)\sin\left(\frac{3\pi x}{a}\right)\left(e^{iE_2t/\hbar-iE_3t/\hbar} + e^{iE_3t/\hbar-iE_2t/\hbar}\right)\right)\\
&= \frac{1}{a}\left(\sin^2\left(\frac{2\pi x}{a}\right) + \sin^2\left(\frac{3\pi x}{a}\right) + \sin\left(\frac{2\pi x}{a}\right)\sin\left(\frac{3\pi x}{a}\right)\left(e^{it/\hbar(E_2-E_3)} + e^{it/\hbar(E_3-E_2)}\right)\right)
\end{align*}
\begin{align*}
&= \frac{1}{a}\left(\sin^2\left(\frac{2\pi x}{a}\right) + \sin^2\left(\frac{3\pi x}{a}\right) + \sin\left(\frac{2\pi x}{a}\right)\sin\left(\frac{3\pi x}{a}\right)\left(e^{it/\hbar(E_2-E_3)} + e^{-it/\hbar(E_2-E_3)}\right)\right)\\
&= \frac{1}{a}\left(\sin^2\left(\frac{2\pi x}{a}\right) + \sin^2\left(\frac{3\pi x}{a}\right) + 2\sin\left(\frac{2\pi x}{a}\right)\sin\left(\frac{3\pi x}{a}\right)\cos\left(\frac{E_3-E_2}{\hbar}t\right)\right)
\end{align*}
Note that we flipped $E_3$ and $E_2$ this is trivial as the negative can be absorbed into the cosine. So we see that the angular frequency of oscillation $\omega$ is
$$\omega = \frac{E_3-E_2}{\hbar}$$
where
$$E_n = \frac{n^2\pi^2\hbar^2}{2ma^2}$$
so we see that $\omega$ is calculated as
\begin{align*}
\omega &= \frac{E_3-E_2}{\hbar}\\
&= \frac{1}{\hbar}\frac{\pi^2\hbar^2}{2ma^2}(3^2-2^2)\\
&= \frac{5\pi^2\hbar}{2ma^2}
\end{align*}

\item
To find the $\expt{x}$ we need to find
$$\int_{-\infty}^{\infty}\Psi^*x\Psi dx$$
For the sake of clarity we will split the integrals into each of the separate parts first we calculate
\begin{align*}
\frac{1}{a}\int_0^ax\sin^2\left(\frac{2\pi x}{a}\right)dx &= \frac{1}{a}\frac{1}{2}\int_0^ax-x\cos\left(\frac{4\pi x}{a}\right)dx\\
&= \frac{1}{2a}\left(\frac{1}{2}x^2\right|_0^a - \int_0^ax\cos\left(\frac{4\pi x}{a}\right)dx\\
&= \frac{1}{2a}\frac{a^2}{2} - \frac{1}{2a}\int_0^ax\cos\left(\frac{4\pi x}{a}\right)dx
\end{align*}
Now we use integration by parts to calculate the remaining integral where
$$u = x\ du = 1$$
$$dv = \cos\left(\frac{4\pi x}{a}\right)$$
and
$$v = \frac{a}{4\pi}\sin\left(\frac{4\pi x}{a}\right)$$
so equation \ref{intparts} yields 
\begin{align*}
&= \frac{a}{4} - \frac{1}{2a}\int_0^ax\cos\left(\frac{4\pi x}{a}\right)dx\\
&= \frac{a}{4} - \frac{1}{2a}\left(x\frac{a}{4\pi}\sin\left(\frac{4\pi x}{a}\right)\right|_0^a - \int_0^a\frac{a}{4\pi}\sin\left(\frac{4\pi x}{a}\right)dx\\
&= \frac{a}{4} - \frac{1}{8\pi}\frac{a}{4\pi}\left(\cos\left(\frac{4\pi x}{a}\right)\right|_0^a\\
&= \frac{a}{4} - \frac{1}{8\pi}\frac{a}{4\pi}\left(1-1\right)\\
&= \frac{a}{4}
\end{align*}
Note that 
$$\frac{1}{a}\int_0^ax\sin^2\left(\frac{3\pi x}{a}\right)dx = \frac{a}{4}$$
as the calculation is exactly the same and as you see from above the terms inside of the sine function play no role in the final integral. Now to calculate 
$$\frac{2}{a}\int_0^a x\sin\left(\frac{2\pi x}{a}\right)\sin\left(\frac{3\pi x}{a}\right)\cos\left(\frac{E_3-E_2}{\hbar}t\right)dx$$
Using the two trig identies 
$$\cos(\alpha+\beta) = \cos(\alpha)\cos(\beta)+\sin(\alpha)\sin(\beta)$$
and
$$\cos(\alpha-\beta) = \cos(\alpha)\cos(\beta)-\sin(\alpha)\sin(\beta)$$
we see that 
\begin{align*}
\cos(\alpha+\beta) - \cos(\alpha-\beta) &= \cos(\alpha)\cos(\beta)+\sin(\alpha)\sin(\beta) - \cos(\alpha)\cos(\beta) + \sin(\alpha)\sin(\beta)\\
&= 2\sin(\alpha)\sin(\beta) 
\end{align*}
So we can rewrite our integral as
\begin{align*}
\frac{2}{a}\int_0^a x\sin\left(\frac{2\pi x}{a}\right)\sin\left(\frac{3\pi x}{a}\right)\cos\left(\frac{E_3-E_2}{\hbar}t\right)dx &= \frac{2}{a}\frac{1}{2}\int_0^a x\left(\cos\left(\frac{(3+2)\pi x}{a}\right) - \cos\left(\frac{\pi x}{a}\right)\right)\cos\left(\omega t\right)dx\\
&= \frac{1}{a}\cos\left(\omega t\right)\int_0^a x\cos\left(\frac{5\pi x}{a}\right)dx - \int_0^ax\cos\left(\frac{\pi x}{a}\right)dx
\end{align*}
Now for each of these integrals we need to apply integration by parts so for the first integral we have
$$u = x\ du = 1$$
and
$$dv = \cos\left(\frac{5\pi x}{a}\right)$$
where 
$$v = \frac{a}{5\pi}\sin\left(\frac{5\pi x}{a}\right)$$
So we can apply equation \ref{intparts} to get
\begin{align*}
\int_0^a x\cos\left(\frac{5\pi x}{a}\right)dx &= \cancelto{0}{\left.x\frac{a}{5\pi}\sin\left(\frac{5\pi x}{a}\right)\right|_0^a} - \int_0^a\frac{a}{5\pi}\sin\left(\frac{5\pi x}{a}\right)dx\\
&= \frac{a}{5\pi}\frac{a}{5\pi}\left(\cos\left(\frac{5\pi x}{a}\right)\right|_0^a\\
&= \frac{a^2}{25\pi^2}\left(\cos\left(\frac{5\pi a}{a}\right) - \cos\left(\frac{5\pi 0}{a}\right)\right)\\
&= \frac{a^2}{25\pi^2}\left(-1 - 1\right)\\
&= \frac{-2a^2}{25\pi^2}
\end{align*}
Now we can see that the only difference between this integral and the other integral is the factor in the cosine and that only changes the constant in front so
$$\int_0^ax\cos\left(\frac{\pi x}{a}\right)dx = \frac{-2a^2}{\pi^2}$$
So now we just sum the terms to find $\expt{x}$
\begin{align*}
\expt{x} &= \frac{a}{4}+\frac{a}{4}+\frac{\cos(\omega t)}{a}\left(\frac{-2a^2}{25\pi^2} - \frac{-2a^2}{\pi^2}\right)\\
&= \frac{a}{2}+\frac{\cos(\omega t)}{a}2a^2\left(\frac{1}{\pi^2} - \frac{1}{25\pi^2}\right)\\
&= \frac{a}{2}+2\cos(\omega t)a\left(\frac{25}{25\pi^2} - \frac{1}{25\pi^2}\right)\\
&= \frac{a}{2}+\frac{48\cos(\omega t) a}{25\pi^2}
\end{align*}
Now to find $\expt{p}$ we can use the relation
\begin{equation}
\expt{p} = m\frac{d\expt{x}}{dt}
\label{exptmom}
\end{equation}
So we calculate equation \ref{exptmom}
\begin{align*}
\expt{p} &= m\frac{d\expt{x}}{dt}\\
&= m\frac{d}{dt}\frac{a}{2}+\frac{48 a}{25\pi^2}\cos(\omega t)\\
&= -m\frac{48\omega a}{25\pi^2}\sin(\omega t)
\end{align*}
Where 
$$\omega = \frac{E_3-E_2}{\hbar}$$

\item
We can find the expectation of the Hamiltonian using the relation
\begin{equation}
\expt{H} = \sum_{n=1}^{\infty}|c_n|^2E_n
\label{exptH}
\end{equation}
Now we see that 
$$|c_n|^2 = \frac{1}{a}$$
for all $n$ so we can find $\expt{H}$ from equation \ref{exptH}
\begin{align*}
\expt{H} &= \frac{1}{a}E_2+\frac{1}{a}E_3\\
&= \frac{1}{a}(E_2+E_3)\\
&= \frac{1}{a}\left(\frac{2^2\pi^2\hbar^2}{2ma^2}+\frac{3^2\pi^2\hbar^2}{2ma^2}\right)\\
&= \frac{13\pi^2\hbar^2}{2ma^3}
\end{align*}
We can apply the fact that
\begin{align*}
\hat{H}^2\psi &= \hat{H}(\hat{H}\psi)\\
&= \hat{H}(E\psi)\\
&= E(\hat{H}\psi)\\
&= E(E\psi)\\
&= E^2\psi
\end{align*}
This implies the fact that $\expt{H^2} = E^2 = \expt{H}^2$. So we can see that the variance of the energy is
\begin{align*}
\sigma_E &= \sqrt{\expt{H^2} -\expt{H}^2}\\
&= \sqrt{\expt{H}^2 -\expt{H}^2}\\
&= 0
\end{align*}
So we see that we have no uncertainty in the value for the energy.
\end{enumerate}

\section{Problem \#3}
\begin{enumerate}[(a)]
\item
At $t=0$ the wavefunction is the same as an infinite square well with length $a=1/2$. So we can say that
$$\Psi(x,t) = 2\sin\left(2\pi x\right)e^{-iE_1t/\hbar}$$
for $0<x<1/2$. And for the range $1/2<x<1$ the wave function is zero so the total wavefunction is
$$\Psi(x,t) = \left\{\begin{array}{cc}
	2\sin\left(2\pi x\right)e^{-iE_1t/\hbar}	&0<x<1/2\\
	0						&1/2<x<1
	\end{array}\right.$$

\item
So now we see that we need to find the coefficients $c_n$ we can use the equation
\begin{equation}
c_n = \int\psi_n(x)^*f(x)dx
\label{coeff}
\end{equation}
where $f(x)$ is the function we have initially which we found in part (a). And $\psi_n(x)$ are the stationary states of the well with width 1 given by
$$\psi_n(x) = \sqrt{2}\sin(n\pi x)$$
so now we can apply equation \ref{coeff} to get
\begin{align*}
c_n &= \int\psi_n(x)^*f(x)dx\\
&= 2\sqrt{2}\int_0^{1/2}\sin(n\pi x)\sin(2\pi x)dx + \cancelto{0}{\int_{1/2}^1(0)dx}\\
&= 2\sqrt{2}\int_0^{1/2}\sin(n\pi x)\sin(2\pi x)dx\\
&= 2\sqrt{2}\frac{1}{2}\int_0^{1/2}\cos(n\pi x-2\pi x)-\cos(n\pi x + 2\pi x)dx\\
&= \sqrt{2}\int_0^{1/2}\cos((n-2)\pi x)dx-\int_{0}^{1/2}\cos((n+2)\pi x)dx\\
&= \sqrt{2}\left(\left.\frac{1}{(n-2)\pi}\sin((n-2)\pi x)\right|_0^{1/2} - \left.\frac{1}{(n+2)\pi}\sin((n+2)\pi x)\right|_0^{1/2}\right)
\end{align*}
\begin{align*}
&= \sqrt{2}\left(\frac{1}{(n-2)\pi}\left(\sin((n-2)\pi/2)-0\right) - \frac{1}{(n+2)\pi}\left(\sin((n+2)\pi/2)-0\right)\right)\\
&= \sqrt{2}\left(\frac{1}{(n-2)\pi}\sin\left(\frac{(n-2)\pi}{2}\right) - \frac{1}{(n+2)\pi}\sin\left(\frac{(n+2)\pi}{2}\right)\right)\\
&= \sqrt{2}\left(\frac{1}{(n-2)\pi}\sin\left(\frac{n\pi}{2}-\pi\right) - \frac{1}{(n+2)\pi}\sin\left(\frac{n\pi}{2}+\pi\right)\right)\\
&= \frac{\sqrt{2}}{\pi}\left(-\frac{1}{(n-2)}\sin\left(\frac{n\pi}{2}\right) + \frac{1}{(n+2)}\sin\left(\frac{n\pi}{2}\right)\right)\\
c_n &= \frac{\sqrt{2}}{\pi}\sin\left(\frac{n\pi}{2}\right)\left(\frac{1}{n+2}-\frac{1}{n-2}\right)\\
\end{align*}
Note that for $n=2$ we have a discontinuity so we have to plug in $n=2$ before taking the integral. Doing this yields
\begin{align*}
2\sqrt{2}\int_0^{1/2}\sin(2\pi x)\sin(2\pi x)dx &= 2\sqrt{2}\int_0^{1/2}\sin^2(2\pi x)dx \\
&= 2\sqrt{2}\frac{1}{2}\int_0^{1/2}1-\cos(4\pi x)dx \\
&= \sqrt{2}\left(x-\frac{1}{4\pi}\sin(4\pi x)\right|_0^{1/2} \\
&= \sqrt{2}\left(\frac{1}{2}-\cancelto{0}{\frac{1}{4\pi}\sin(2\pi)} - 0\right) \\
c_2 &= \frac{\sqrt{2}}{2}
\end{align*}
See attached for the Mathematica notebook that shows plots of $\Psi$ for $n=5,\ 50,\ 500$.

\item
Given the time constant $t_0$ which is defined as
$$t_0 \equiv \frac{2ma^2}{\hbar \pi^2}$$
We see that the units of $t_0$ are 
\begin{align*}
\left[t_0\right] &= \frac{kg\ m^2}{J\ s}\\
&= \frac{kg\ m^2}{kg\ m^s\ s^{-2}\ s}\\
&= \frac{1}{s^{-1}}\\
&= s
\end{align*}
Now we know that the time dependent part of the wavefunction is given by
$$\varphi(t) = e^{-iE_nt\hbar}$$
but if we want to replace $t$ here with $\tau\equiv t/t_0$ we have to say that
\begin{align*}
\varphi(t) &= e^{-iE_nt\hbar}\\
&= \exp{\frac{-iE_nt_0\tau}{\hbar}}\\
&= \exp{\frac{-i\tau}{\cancel{\hbar}}\frac{n^2\cancel{\pi^2\hbar^2}}{\cancel{2ma^2}}\cancel{\frac{2ma^2}{\hbar \pi^2}}}\\
&= e^{-in^2\tau}
\end{align*}
See attached for Mathematica notebook with plots.
\end{enumerate}

\section{Problem \#4}
\begin{enumerate}[(a)]
\item
To calculate the energy of an electron at a ground state in an infinite square well with $a=10^{-15}\ m$ we say
\begin{align*}
E_1 &= \frac{\pi^2\hbar^2}{2m_ea^2}\\
&= \frac{\pi^2\hbar^2}{2(9.1\times10^{-31})(10^{-15})^2}\\
&= 6.02\times10^{-10}\ J\frac{1\ eV}{1.6\times10^{-19}\ J}\\
&= 3.8\times10^9\ eV
\end{align*}
So we can calculate the mass of this energy using
\begin{equation}
E = mc^2
\label{ensti}
\end{equation}
Equation \ref{ensti} gives us
\begin{align*}
m &= \frac{E}{c^2}\\
&= \frac{3.8\times10^9}{c^2}\\
&= 4.2\times10^{-8}\ eV\ c^{-2}
\end{align*}
Now we know that the mass of a proton is $m_p = 9.4\times10^{8}\ eV\ c^{-2}$. So we can say that the ratio between the two is 
$$\frac{m_e}{m_p} = 4.5\times10^{-17}$$
this is a very small amount. This fact implies that this is not a good model for the nucleus because we will not be able to account for the increased mass with this small of an amount.

\item
Now if we repeat part (a), but this time we pick $a$ such that it models the electron orbital say $a=10^{-10}$ we can calculate the energy of a ground state electron
\begin{align*}
E_1 &= \frac{\pi^2\hbar^2}{2m_ea^2}\\
&= \frac{\pi^2\hbar^2}{2(9.1\times10^{-31})(10^{-10})^2}\\
&= 6.03\times10^{-20}\ J\frac{1\ eV}{1.6\times10^{-19}\ J}\\
&= 3.8\times10^{-1}\ eV
\end{align*}
Wow this is a very small energy! The mass of this binding energy from equation \ref{ensti} is
$$m = 4.1\times10^{-18}\ eV\ c^{-2}$$
so the ratio of this with the mass of the proton is
$$\frac{m_e}{m_p} = 4.5\times10^{-27}$$
The binding energy from an electron orbital contributes next to nothing in the corrections of the mass of the atom.
\end{enumerate}

\end{document}

