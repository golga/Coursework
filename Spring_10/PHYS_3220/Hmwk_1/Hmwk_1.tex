\documentclass[11pt]{article}

\usepackage{latexsym}
\usepackage{amssymb}
\usepackage{amsthm}
\usepackage{enumerate}
\usepackage{amsmath}
\usepackage{cancel}
\numberwithin{equation}{section}

\setlength{\evensidemargin}{.25in}
\setlength{\oddsidemargin}{-.25in}
\setlength{\topmargin}{-.75in}
\setlength{\textwidth}{6.5in}
\setlength{\textheight}{9.5in}
\newcommand{\due}{January 13th, 2010}
\newcommand{\HWnum}{1}
\newcommand{\grad}{\bold\nabla}
\newcommand{\vecE}{\vec{E}}
\newcommand{\scrptR}{\vec{\mathfrak{R}}}
\newcommand{\kapa}{\frac{1}{4\pi\epsilon_0}}

\begin{document}
\begin{titlepage}
\setlength{\topmargin}{1.5in}
\begin{center}
\Huge{Physics 3320} \\
\LARGE{Principles of Electricity and Magnetism II} \\
\Large{Professor Ana Maria Rey} \\[1cm]

\huge{Homework \#\HWnum}\\[0.5cm]

\large{Joe Becker} \\
\large{SID: 810-07-1484} \\
\large{\due} 

\end{center}

\end{titlepage}



\begin{enumerate}
\item Problem 1)\\
Let $z = -3+3i$ we calculate
\begin{align*}
|z|^2 &= z^*z\\
&=(-3+3i)(-3-3i)\\
&= 9 + 9i - 9i - 9i^2\\
&= 9 + 9 = 18
\end{align*}
If we write $z$ in the form $z=Ae^{i\theta}$ we see that 
$$A = |z| = \sqrt{18}$$
and
$$\theta = \frac{3\pi}{4}$$
note that the plot is attached. \fbox{i}

\item Problem 2)\\
Let $$A = \left(\begin{array}{cc}
	1	&0\\
	0	&2\\
		\end{array}\right)$$
and
$$B = \left(\begin{array}{cc}
	1	&2\\
	3	&4\\
		\end{array}\right)$$
We calculate
\begin{align*}
\bold{A}\bold{B} &= 
\left(\begin{array}{cc}
	1	&0\\
	0	&2\\
		\end{array}\right)
\left(\begin{array}{cc}
	1	&2\\
	3	&4\\
		\end{array}\right)\\
&= \left(\begin{array}{cc}
	1(1)+3(0)	&1(2)+4(0)\\
	0(1)+2(3)	&0(2)+2(4)\\
		\end{array}\right)\\
&= \left(\begin{array}{cc}
	1	&2\\
	6	&8\\
		\end{array}\right)
\end{align*}
\fbox{i}

\item Problem 3)\\
To calculate 
$$\int_{-7}^{7}(x-8)\delta(x-6)dx$$
we use the definition of the \emph{delta function}
$$\int f(x)\delta(x-a)dx = f(a)$$
So it follows that
$$\int_{-7}^{7}(x-8)\delta(x-6)dx = 6-8 = -2$$
\fbox{i}

\item Problem 4)\\
See attached \fbox{ii}

\item Problem 5)\\
The relation between energy, $E$, and the frequency, $f$, of a photon is given by
$$E = hf$$
where $h$ is \emph{Plank's constant} given by $h = 6.62\times10^{-34}$. To approximate the energy of a photon of yellow light we assume that the wavelength of the photon is given by $\lambda = 575\ nm$. Using the relation 
$$c = \lambda f$$
where $c$ is the speed of light in a vacuum, we calculate the energy of the photon as
\begin{align*}
E &= hf\\
&= h\frac{c}{\lambda}\\
&= (6.62\times10^{-34})\frac{3\times10^8}{575\times10^{-9}}\\
&= 3.45\times10^{-19}\ J
\end{align*}
\fbox{iii}

\item Problem 6)\\
The relation between the wavelength, $\lambda$, and the momentum, $p$, is 
\begin{equation}
p = \frac{h}{\lambda}
\label{momen}
\end{equation}
To find the kinetic energy of an electron with $\lambda = 1\times10^{10}$ we use the identity 
\begin{equation}
T = \frac{p^2}{2m}
\label{kinen}
\end{equation}
Combining equations \ref{momen} and \ref{kinen} we see
$$T = \frac{h^2}{2m\lambda^2}$$
so we calculate the kinetic energy of the electron as
\begin{align*}
T &= \frac{h^2}{2m_e\lambda^2}\\
&= \frac{(6.62\times10^{-34})^2}{2(9.1\times10^{-31})(1\times10^{-10})^2}\\
&= 2.4\times10^{-17}\ J
\end{align*}
or in electron volts 
$$2.4\times10^{-17}\ J\frac{1\ eV}{1.6\times10^{-19}\ J} = 150\ eV$$
This is not a high voltage device. \fbox{ii}

\item Problem 7)\\
The equation describing a sinusoidal traveling wave is given by
$$x(t) = A\sin(\omega t+\phi)$$
the speed of the wave is given by the frequency $\omega$ and the direction is given by amplitude factor $A$. \fbox{iii}

\item Problem 8)\\
Two functions are orthogonal when 
$$\int f_n(x)g_m(x)dx = \left\{\begin{array}{cc}
			n\ne m	&0\\
			n=m	&A\\
			\end{array}\right.$$
An example of orthogonal functions are
$$\sin(k_nx)$$
for $n = 1,2,3...$\fbox{ii}
\end{enumerate}

\end{document}

