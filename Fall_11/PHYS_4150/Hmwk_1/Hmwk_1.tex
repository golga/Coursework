\documentclass[11pt]{article}

\usepackage{latexsym}
\usepackage{amssymb}
\usepackage{amsthm}
\usepackage{enumerate}
\usepackage{amsmath}
\usepackage{cancel}
\numberwithin{equation}{section}

\setlength{\evensidemargin}{.25in}
\setlength{\oddsidemargin}{-.25in}
\setlength{\topmargin}{-.75in}
\setlength{\textwidth}{6.5in}
\setlength{\textheight}{9.5in}
\newcommand{\due}{August 26th, 2011}
\newcommand{\HWnum}{1}
\newcommand{\grad}{\bold\nabla}
\newcommand{\vecE}{\vec{E}}
\newcommand{\scrptR}{\vec{\mathfrak{R}}}
\newcommand{\kapa}{\frac{1}{4\pi\epsilon_0}}
\newcommand{\emf}{\mathcal{E}}
\newcommand{\unit}[1]{\ensuremath{\, \mathrm{#1}}}
\newcommand{\real}{\textnormal{Re}}
\newcommand{\Erf}{\textnormal{Erf}}
\newcommand{\sech}{\textnormal{sech}}
\newcommand{\scrO}{\mathcal{O}}
\newcommand{\levi}{\widetilde{\epsilon}}
\newcommand{\partiald}[2]{\ensuremath{\frac{\partial{#1}}{\partial{#2}}}}
\newcommand{\norm}[2]{\langle{#1}|{#2}\rangle}
\newcommand{\inprod}[2]{\langle{#1}|{#2}\rangle}
\newcommand{\ket}[1]{|{#1}\rangle}
\newcommand{\bra}[1]{\langle{#1}|}





\begin{document}
\begin{titlepage}
\setlength{\topmargin}{1.5in}
\begin{center}
\Huge{Physics 3320} \\
\LARGE{Principles of Electricity and Magnetism II} \\
\Large{Professor Ana Maria Rey} \\[1cm]

\huge{Homework \#\HWnum}\\[0.5cm]

\large{Joe Becker} \\
\large{SID: 810-07-1484} \\
\large{\due} 

\end{center}

\end{titlepage}



\section{Problem \#1}
Given the Maxwellian distribution
$$\hat{f}(u) = A\exp(-mu^2/2KT)$$
we can derive the constant $A$ by using the normalization condition
$$\int_{-\infty}^{\infty}\hat{f}(u)du = 1$$
so we calculate
\begin{align*}
1 &= \int_{-\infty}^{\infty}\hat{f}(u)du \\
&= \int_{-\infty}^{\infty} A\exp(-mu^2/2KT)du 
\end{align*}
To solve this integral we use the \emph{Gaussian Integral Identity}
\begin{equation}
\int_{-\infty}^{\infty}e^{-ax^2} = \sqrt{\frac{\pi}{a}}
\label{Gauss}
\end{equation}
So we see that for our distribution $a = m/2KT$. Now by equation \ref{Gauss} we have
\begin{align*}
1 &= A\int_{-\infty}^{\infty} \exp(-mu^2/2KT)du\\
&= A\sqrt{2\pi KT}{m}\\
\Rightarrow A &= \sqrt{\frac{m}{2\pi KT}}
\end{align*}

\section{Problem \#2}
For a plasma with electron and ion temperature $KT_{e} = KT_{i} = 20\unit{keV}$ and particle density $n = 10^{21}\unit{m^{-3}}$ we can calculate the pressure of the plasma on its container by using
\begin{equation}
P = nKT
\label{Press}
\end{equation}
where $T = T_e+T_i$. So equation \ref{Press} yields
\begin{align*}
P &= nKT\\
&= nK(T_e+T_i)\\
&= n(KT_e+KT_i)\\
&= 10^{21}\unit{m^{-3}}(20\unit{keV}+20\unit{keV})\\
&= \left(\frac{4.0\times10^{22}\unit{keV}}{\unit{m^{-3}}}\right)\left(\frac{1000\unit{eV}}{1\unit{keV}}\right)\left(\frac{1.6\times10^{-19}\unit{J}}{1\unit{eV}}\right)\\
&= 6.4\times10^{6}\unit{N\ m^{-2}}\left(\frac{1\unit{atm}}{10^5\unit{N\ m^{-2}}}\right) = 64\unit{atm}
\end{align*}

\section{Problem \#3}
\begin{enumerate}[(i)]
\item
Given the \emph{Maxwell Equations}
\begin{align}
\grad\cdot\vec{E} &= 0 \\
\grad\cdot\vec{B} &= 0 \\
\grad\times\vec{E} &= -\frac{\partial\vec{B}}{\partial t} \\   
\grad\times\vec{B} &= \mu_0\epsilon_0\frac{\partial\vec{E}}{\partial t}
\end{align}
Note that these equations are in a vacuum. Now we can calculate the wave equation 
$$\grad^2\vec{E} = \mu_0\epsilon_0\frac{\partial^2\vec{E}}{\partial t^2}$$
so by taking the curl of equation 3.3 and using the vector identity 
\begin{equation}
\grad\times(\grad\times\vec{A}) = \grad(\grad\cdot\vec{A}) - (\grad\cdot\grad)\vec{A}
\label{ident}
\end{equation}
we can find that
\begin{align*}
\grad\times(\grad\times\vec{E}) &= \grad\cancelto{0}{(\grad\cdot\vec{E})} - (\grad\cdot\grad)\vec{E}\\
\Rightarrow -\grad^2\vec{E} &= \grad\times\left(-\frac{\partial\vec{B}}{\partial t}\right)\\
\grad^2\vec{E} &= \frac{\partial}{\partial t}(\grad\times\vec{B})\\
&= \frac{\partial}{\partial t}\left(\mu_0\epsilon_0\frac{\partial\vec{E}}{\partial t}\right)\\
\Rightarrow \grad^2\vec{E} &= \mu_0\epsilon_0\frac{\partial^2\vec{E}}{\partial t^2}
\end{align*}

\item
Given the electric field
$$\vec{E} = \vec{E}_0\exp(i(2\pi/\lambda)z-i(2\pi f)t)$$
we can calculate show that this electric field is a solution to the wave equation we found in part i. First we see that the right hand side yields
\begin{align*}
\grad^2\vec{E} &= \frac{\partial^2}{\partial z^2}\vec{E}_0\exp(i(2\pi/\lambda)z-i(2\pi f)t)\\
&= \vec{E}_0\exp(i(2\pi/\lambda)z-i(2\pi f)t)(i(2\pi/\lambda))^2\\
&= -\vec{E}_0\exp(i(2\pi/\lambda)z-i(2\pi f)t)(2\pi/\lambda)^2
\end{align*}
And the left hand side yields
\begin{align*}
\mu_0\epsilon_0\frac{\partial^2\vec{E}}{\partial t^2} &= \mu_0\epsilon_0\frac{\partial^2}{\partial t^2}(\vec{E}_0\exp(i(2\pi/\lambda)z-i(2\pi f)t))\\
&= \mu_0\epsilon_0\vec{E}_0\exp(i(2\pi/\lambda)z-i(2\pi f)t)(i(2\pi f))^2\\
&= -\mu_0\epsilon_0\vec{E}_0\exp(i(2\pi/\lambda)z-i(2\pi f)t)(2\pi f)^2
\end{align*}
Now if we combine both sides we have
\begin{align*}
-\vec{E}_0\exp(i(2\pi/\lambda)z-i(2\pi f)t)(2\pi/\lambda)^2 &= -\mu_0\epsilon_0\vec{E}_0\exp(i(2\pi/\lambda)z-i(2\pi f)t)(2\pi f)^2 \\
&\Downarrow\\
\lambda^{-1} &= \sqrt{\mu_0\epsilon_0} f \\
\lambda^{-1} &=  c^{-1}f \\
&\Downarrow\\
f = c/\lambda
\end{align*}
So the given electric field only satisfies the wave equation if $f = c/\lambda$
\end{enumerate}

\section{Problem \#4}
If we assume a Maxwellian distribution we can take Maxwell equation to say that
\begin{align*}
\grad\cdot\vec{E} &= \frac{\rho}{\epsilon_0}\\
&\Downarrow\\
-\epsilon_0\grad^2\phi &= q_in_i + q_en_e\\
&= en_0 - en_0\exp\left(\frac{e\phi}{KT_e}\right)\\
&= en_0\left(1 - \exp\left(\frac{e\phi}{KT_e}\right)\right)
\end{align*}
Now if we use the fact that $e\phi<<KT_e$ we can approximate the exponential such that
\begin{align*}
&= en_0\left(1 - 1 - \frac{e\phi}{KT_e}\right)\\
&\Downarrow\\
\grad^2\phi &= \frac{e^2n_0}{\epsilon_0KT_e}\phi
\end{align*}
We see that this has a solution 
$$\phi(x) = Ae^{-x/\lambda_D}+Be^{x/\lambda_D}$$
But we want the potential to be zero far away so we take $B=0$ now we see that the potential potential at $x=0$ is given by $e$ the charge of the proton. So outside of the shielding the proton charge is canceled out.
\end{document}

