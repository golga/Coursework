\documentclass[11pt]{article}

\usepackage{latexsym}
\usepackage{amssymb}
\usepackage{amsthm}
\usepackage{enumerate}
\usepackage{amsmath}
\usepackage{cancel}
\numberwithin{equation}{section}

\setlength{\evensidemargin}{.25in}
\setlength{\oddsidemargin}{-.25in}
\setlength{\topmargin}{-.75in}
\setlength{\textwidth}{6.5in}
\setlength{\textheight}{9.5in}
\newcommand{\due}{September 23rd, 2010}
\newcommand{\HWnum}{4}
\newcommand{\grad}{\bold\nabla}
\newcommand{\vecE}{\vec{E}}
\newcommand{\scrptR}{\vec{\mathfrak{R}}}
\newcommand{\kapa}{\frac{1}{4\pi\epsilon_0}}
\newcommand{\emf}{\mathcal{E}}
\newcommand{\unit}[1]{\ensuremath{\, \mathrm{#1}}}

\begin{document}
\begin{titlepage}
\setlength{\topmargin}{1.5in}
\begin{center}
\Huge{Physics 3310} \\
\LARGE{Principles of Electricity and Magnetism 1} \\
\Large{Professor Thomas R. Schibli} \\[1cm]

\huge{Homework \#\HWnum}\\[0.5cm]

\large{Joe Becker} \\
\large{SID: 810-07-1484} \\
\large{\due} 

\end{center}

\end{titlepage}



\section{Problem \#1}
If we assume that the deuterium atoms have a charge $q=2e$, where $e$ is the charge of an electron, and mass $m=2m_p$ where $m_p$ is the mass of a proton. Now if we assume that the energy of the deuterium atoms are $200\unit{keV}$. We see that the velocity of the atom is related to the energy by
$$\frac{1}{2}mv^2 = \frac{3}{2}KT$$
where $KT$ is the energy in eV. So
\begin{align*}
v &= \sqrt{\frac{3KT}{2m_p}}\\
&= \sqrt{\frac{3(200\unit{keV})}{2(1.67\times10^{-27}\unit{kg})}}\\
&= \sqrt{\frac{9.6\times10^{-14}\unit{J}}{3.32\times10^{-27}\unit{kg}}}\\
&= 5.36\times10^{6}\unit{m\ s^{-1}}
\end{align*}
So if we assume that the ion if moving only perpendicular to the magnetic field so that $v_{\perp}$ is at a maximum. We see that in a $5\unit{T}$ B-field the \emph{Larmor radius} is
\begin{align*} 
r_L &= \frac{mv_{\perp}}{|q|B}\\
&= \frac{2m_pv_{\perp}}{|2e|B}\\
&= \frac{({4.32\times10^{-27}\unit{kg})(5.36\times10^{6}\unit{m\ s^{-1})}}}{2(1.6\times10^{-19}\unit{C})(5\unit{T})}\\
&= 0.011\unit{m}
\end{align*} 
So we see that the \emph{Larmor radius} is smaller than the minor radius of the toroidal plasma, $a=0.6\unit{m}$.

\section{Problem \#2}
To begin this problem we first need to find the magnitude of the electric field outside of the electron beam. To do this we will use \emph{Gauss' Law} 
\begin{equation}
\oint\vec{E}\cdot d\vec{a} = \frac{q_{enc}}{\epsilon_0}
\label{Gauss}
\end{equation}
with a Gaussian cylinder with $r>a$. We assume that the density of the electron beam in $n_e = 10^{14}\unit{m^{-3}}$. So we see that for equation \ref{Gauss} the change enclosed is given by
$$q_{enc} = \pi a^2 l n_e e$$
where $a$ is the radius of the electron beam, $e$ is the charge of an electron, and $l$ is the length of the Gaussian cylinder. So equation \ref{Guass} yields
\begin{align*}
\oint\vec{E}\cdot d\vec{a} &= \frac{q_{enc}}{\epsilon_0}\\
E\oint da &= \frac{\pi a^2 l n_ee}{\epsilon_0}\\
E(2\pi r l)&= \frac{\pi a^2 l n_ee}{\epsilon_0}\\
\vec{E} &= \frac{a^2 n_ee}{2\epsilon_0}\frac{1}{r}\hat{r}
\end{align*}
Note that we used the fact that both $\vec{E}$ and $d\vec{a}$ point in the $\hat{r}$ direction. So we can see for $r=a=0.01\unit{m}$ we get
\begin{align*} 
\vec{E} &= \frac{a^2 n_ee}{2\epsilon_0}\frac{1}{a}\hat{r}\\
&= \frac{(0.01\unit{m})(10^{14}\unit{m^{-3}})(-1.6\times10^{-19}\unit{C})}{2(8.85\times10^{-12}\unit{C^2\ N^{-1}\ m^{-2}})}\hat{r}\\
&= -9.04\times10^{4}\unit{N\ C^{-1}}\hat{r}
\end{align*} 
Now we can find the $\vec{E}\times\vec{B}$ drift at $r=a$ by
\begin{align*}
\vec{v}_E &= \frac{\vec{E}\times\vec{B}}{B^2}\\
&= \frac{(-9.04\times10^4\unit{N\ C^{-1}})(2\unit{T})(\hat{r}\times\hat{z})}{2^2\unit{T^2}}\\
&= -4.5\times10^{4}\unit{m\ s^{-1}}(\hat{r}\times\hat{z})\\
&= 4.5\times10^{4}\unit{m\ s^{-1}}\hat{\theta}
\end{align*}



\section{Problem \#3}
\begin{enumerate}[(a)]
\item
If we assume that the magnetic field that goes like a dipole looks like
$$\vec{B} = C\frac{1}{r^3}\hat{\phi}$$
where $C$ is a constant of proportionality with the value of $C = 3\times10^{-5}\unit{(earth\ radii)^3\ T}$. We can find the drift velocity of the electrons at $30\unit{keV}$ by
\begin{align*}
v &= \sqrt{\frac{3KT}{m_e}}\\
&= \sqrt{\frac{3(30\unit{keV})}{9.11\times10^{-31}\unit{kg}}}\\
&= \sqrt{\frac{1.44\times10^{-14}\unit{J}}{9.11\times10^{-31}\unit{kg}}}\\
&= 1.25\times10^{8}\unit{m\ s^{-1}}
\end{align*}
Now if we assume that $v$ is the tangential velocity we can then calculate the grad-B drift velocity by first finding $\grad B$ or
\begin{align*}
\grad B &= C\grad\frac{1}{r^3}\\
&= C\frac{\partial}{\partial r}\frac{1}{r^3}\hat{r}\\
&= -3C\frac{1}{r^4}\hat{r}\\
\end{align*}
So we can find $v_{\grad B}$ at $r=5\unit{earth\ radii}$ by
\begin{align*}
\vec{v}_{\grad B} &= -\frac{1}{2}v_{\perp}r_L \frac{\vec{B}\times\grad B}{B^2}\\
\vec{v}_{\grad B} &= -\frac{1}{2}v_{\perp}\frac{m_ev_{\perp}}{eB}\frac{\vec{B}\times\grad B}{B^3}\\
&= -\frac{1}{2}v_{\perp}\frac{m_ev_{\perp}}{e}\frac{C}{r^3}\frac{-3C}{r^4}\frac{r^9}{C^3}(\hat{\phi}\times\hat{r})\\
&= \frac{3}{2}\frac{m_ev_{\perp}^2}{e}\frac{r^2}{C}\hat{\theta}\\
&= \frac{3}{2}\frac{(9.11\times10^{-31}\unit{kg})(1.25\times10^{8}\unit{m\ s^{-1}})^2}{1.6\times10^{-19}\unit{C}}\frac{(5\unit{earth\ radii})^2}{3\times10^{-5}\unit{(earth radii)^3\ T}}\hat{\theta}\\
&= \frac{3}{2}\frac{(9.11\times10^{-31}\unit{kg})(1.25\times10^{8}\unit{m\ s^{-1}})^2}{1.6\times10^{-19}\unit{C}}(0.13\unit{m^{-1}\ T})\hat{\theta}\\
&= 1.16\times10^{4}\unit{m\ s^{-1}}\hat{\theta}
\end{align*}
Now for the $1\unit{keV}$ protons we can find the velocity of the protons like we did for the electrons
\begin{align*}
v &= \sqrt{\frac{3KT}{m_p}}\\
&= \sqrt{\frac{3(1\unit{keV})}{1.67\times10^{-27}\unit{kg}}}\\
&= \sqrt{\frac{4.8\times10^{-16}\unit{J}}{1.67\times10^{-27}\unit{kg}}}\\
&= 5.36\times10^{5}\unit{m\ s^{-1}}
\end{align*}
Now we can find the grad-B drift velocity the same as way we did for the electron 
\begin{align*}
\vec{v}_{\grad B} &= -\frac{3}{2}\frac{m_pv_{\perp}^2}{e}\frac{r^2}{C}\hat{\theta}\\
&= -\frac{3}{2}\frac{(1.67\times10^{-27}\unit{kg})(5.36\times10^{5}\unit{m\ s^{-1}})^2}{1.6\times10^{-19}\unit{C}}(0.13\unit{m^{-1}\ T})\hat{\theta}\\
&= -585\unit{m\ s^{-1}}\hat{\theta}
\end{align*}

\item
The electron moves in the positive $\hat{\theta}$ direction. So the electron drifts eastward.

\item
If we assume that the electron is at $5\unit{earth\ radii}$ or $3.19\times10^{7}\unit{m}$ we see that for the electron to complete one orbit the electron must travel
$$2\pi(3.19\times10^{7}\unit{m}) = 2.00\times10^{8}\unit{m}$$
so if the electron is traveling at $5.36\times10^{5}\unit{m\ s^{-1}}$ as we found in part (a) we see that the time for the electron the make one for revolution is 
$$\frac{2.00\times10^{8}\unit{m}}{5.36\times10^{5}\unit{m\ s^{-1}}} = 3.74\times10^{2}\unit{s}$$

\item
The ring current density for the electron is
\begin{align*}
j_{e} &= nev_{\grad B}\\
&= (10^{7}\unit{m^{-3}})(-1.6\times10^{-19}\unit{C})(5.36\times10^{5}\unit{m\ s^{-1}})\\
&= -8.58\times10^{-6}\unit{A\ s^{-2}}
\end{align*}
and for the proton we get
\begin{align*}
j_{p} &= nev_{\grad B}\\
&= (10^{7}\unit{m^{-3}})(1.6\times10^{-19}\unit{C})(-585\unit{m\ s^{-1}})\\
&= -9.36\times10^{-9}\unit{A\ s^{-2}}
\end{align*}
Now we can just add the two separate ring current densities to get
$$j = -8.60\times10^{-6}\unit{A\ s^{-2}}$$
\end{enumerate}

\section{Problem \#4}
\begin{enumerate}[(a)]
\item
To find the total drift velocity of the electron outside an infinite wire carrying current $I$. First we see from \emph{Amp\'{e}re's Law} that the magnetic field if given by
\begin{align*}
\oint \vec{B}\cdot d\vec{l} &= \mu_0I_{enc}\\
B\oint dl &= \mu_0I\\
B(2\pi r) &= \mu_0I\\
\vec{B} &= \frac{\mu_0I}{2\pi r}\hat{\theta}
\end{align*}
Now we can use the equation
$$\vec{v}_R+\vec{v}_{\grad B} = \frac{m}{q}\frac{\vec{R}_c\times\vec{B}}{R_c^2B^2}\left(v_{\parallel}^2+\frac{1}{2}v_{\perp}^2\right)$$
to find the total drift velocity. Where $\vec{R}_c = r_0\hat{r}$ and $v_{\parallel} = v_{\perp}$. So we can calculate
\begin{align*}
\vec{v}_R+\vec{v}_{\grad B} &= \frac{m}{q}\frac{\vec{R}_c\times\vec{B}}{R_c^2B^2}\left(v_{\parallel}^2+\frac{1}{2}v_{\perp}^2\right)\\
&= \frac{m_e}{e} \frac{R_cB}{R_c^2B^2}(\hat{r}\times\hat{\theta})\left(v_{\perp}^2+\frac{1}{2}v_{\perp}^2\right)\\
&= \frac{m_e}{e} \frac{1}{R_cB}\frac{3}{2}v_{\perp}^2\hat{z}\\
&= \frac{3}{2}\frac{m_e}{e} \frac{2\pi r_0}{r_0\mu_0I}v_{\perp}^2\hat{z}\\
&= \frac{3}{2}\frac{m_e}{e}\frac{2\pi}{\mu_0I}v_{\perp}^2\hat{z}
\end{align*}
\end{enumerate}
\end{document}

