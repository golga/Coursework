\documentclass[11pt]{article}

\usepackage{latexsym}
\usepackage{amssymb}
\usepackage{amsthm}
\usepackage{enumerate}
\usepackage{amsmath}
\usepackage{cancel}
\numberwithin{equation}{section}
\usepackage{url,graphicx,tabularx,array,geometry}

\setlength{\parskip}{1ex} %--skip lines between paragraphs
%\setlength{\parindent}{0pt} %--don't indent paragraphs

\setlength{\evensidemargin}{.25in}
\setlength{\oddsidemargin}{-.25in}
\setlength{\topmargin}{-.75in}
\setlength{\textwidth}{6.5in}
\setlength{\textheight}{9.5in}
\newcommand{\grad}{\bold\nabla}
\newcommand{\vecE}{\vec{E}}
\newcommand{\scrptR}{\vec{\mathfrak{R}}}
\newcommand{\kapa}{\frac{1}{4\pi\epsilon_0}}
\newcommand{\emf}{\mathcal{E}}
\newcommand{\unit}[1]{\ensuremath{\, \mathrm{#1}}}
\newcommand{\real}{\textnormal{Re}}
\newcommand{\Erf}{\textnormal{Erf}}
\newcommand{\sech}{\textnormal{sech}}
\newcommand{\scrO}{\mathcal{O}}
\newcommand{\levi}{\widetilde{\epsilon}}
\newcommand{\partiald}[2]{\ensuremath{\frac{\partial{#1}}{\partial{#2}}}}
\newcommand{\norm}[2]{\langle{#1}|{#2}\rangle}
\newcommand{\inprod}[2]{\langle{#1}|{#2}\rangle}
\newcommand{\average}[1]{\left\langle{#1}\right\rangle}
\newcommand{\ket}[1]{|{#1}\rangle}
\newcommand{\bra}[1]{\langle{#1}|}
\newcommand{\Resid}[2]{\ensuremath{\textnormal{Res}\left[{#1},{#2}\right]}}





%-- Commands for header
\renewcommand{\title}[1]{\textbf{#1}\\}
\renewcommand{\line}{\begin{tabularx}{\textwidth}{X>{\raggedleft}X}\hline\\\end{tabularx}\\[-0.5cm]}
\newcommand{\leftright}[2]{\begin{tabularx}{\textwidth}{X>{\raggedleft}X}#1%
& #2\\\end{tabularx}\\[-0.5cm]}

\begin{document}
\title{Study Questions \#2}
\line
\leftright{August 30, 2011}{Joe Becker} %-- left and right positions in the header

\textbf{Question \#1}\\
\indent \textit{\indent What did you find most interesting or most useful for you as a writer in the authors’ discussion of audience?}

I found the idea that your technical writing's main purpose is to instill a trust in what you have to say the most interesting. It is very useful to have this goal in mind when writing. This allows the author to constantly think about what they are saying and if what they are saying is credible.

\textbf{Question \#2}\\
\indent \textit{\indent What common principles do you see in all types of reviews (peer, technical, editorial, and managerial)? Think about this from the perspective of:}
\begin{enumerate}[(a)]
\item \textit{the author}\\
The number one thing the author needs to find out from any reviewer is if the content and clarity of the paper comes through. If the reviewer fails to find clarity of the problem or purpose of the writing then the author needs to rework what they are trying to state.

\item \textit{the reviewer}\\
All types of reviewers are trying to understand the paper first and foremost. This goes back to the clarity of the problem. While different types of reviewers can focus on specific parts of the writing. Whether it be the technical content or the organization all reviewers are looking for understanding of the material presented to them.
\end{enumerate}

\textbf{Question \#3}\\
\indent \textit{\indent Perhaps the number one feature of scientific papers that makes them hard on the audience is the lack of coherence. The ‘road map’ is one feature that can provide coherence to your papers. Explain what you think the author means by ‘road map’.}

By 'road map' the author means a clear statement of what the paper will cover. This statement is in the beginning of the paper an gives the reader a feel of where the paper is going and what the paper is trying to say.

\textbf{Question \#4}\\
\indent \textit{\indent How do you think a ‘road map’ in a scientific paper contributes to coherence? From whose perspective, reader or writer?}

A 'road map' sets the foundation of the paper coherence. To the writer the 'road map' allows them to lay out their ideas and gives them a test to see if their points are relevant to the topic on a whole. For the reader the 'road map' gives them a feel of what is to come. This allows the reader to contextualize the information they are given in the following sections or paragraphs.



\end{document}

