\documentclass[11pt]{article}

\usepackage{latexsym}
\usepackage{amssymb}
\usepackage{amsthm}
\usepackage{enumerate}
\usepackage{amsmath}
\usepackage{cancel}
\numberwithin{equation}{section}
\usepackage{url,graphicx,tabularx,array,geometry}

\setlength{\parskip}{1ex} %--skip lines between paragraphs
%\setlength{\parindent}{0pt} %--don't indent paragraphs

\setlength{\evensidemargin}{.25in}
\setlength{\oddsidemargin}{-.25in}
\setlength{\topmargin}{-.75in}
\setlength{\textwidth}{6.5in}
\setlength{\textheight}{9.5in}
\newcommand{\grad}{\bold\nabla}
\newcommand{\vecE}{\vec{E}}
\newcommand{\scrptR}{\vec{\mathfrak{R}}}
\newcommand{\kapa}{\frac{1}{4\pi\epsilon_0}}
\newcommand{\emf}{\mathcal{E}}
\newcommand{\unit}[1]{\ensuremath{\, \mathrm{#1}}}
\newcommand{\real}{\textnormal{Re}}
\newcommand{\Erf}{\textnormal{Erf}}
\newcommand{\sech}{\textnormal{sech}}
\newcommand{\scrO}{\mathcal{O}}
\newcommand{\levi}{\widetilde{\epsilon}}
\newcommand{\partiald}[2]{\ensuremath{\frac{\partial{#1}}{\partial{#2}}}}
\newcommand{\norm}[2]{\langle{#1}|{#2}\rangle}
\newcommand{\inprod}[2]{\langle{#1}|{#2}\rangle}
\newcommand{\average}[1]{\left\langle{#1}\right\rangle}
\newcommand{\ket}[1]{|{#1}\rangle}
\newcommand{\bra}[1]{\langle{#1}|}
\newcommand{\Resid}[2]{\ensuremath{\textnormal{Res}\left[{#1},{#2}\right]}}





%-- Commands for header
\renewcommand{\title}[1]{\textbf{#1}\\}
\renewcommand{\line}{\begin{tabularx}{\textwidth}{X>{\raggedleft}X}\hline\\\end{tabularx}\\[-0.5cm]}
\newcommand{\leftright}[2]{\begin{tabularx}{\textwidth}{X>{\raggedleft}X}#1%
& #2\\\end{tabularx}\\[-0.5cm]}

\begin{document}
\title{Study Questions \#1}
\line
\leftright{August 25, 2011}{Joe Becker} %-- left and right positions in the header

\textbf{Question \#1}\\
\indent \textit{\indent What is research?}\\
\indent \textit{\indent Write an example of a research topic you might use for this class.}

Research, according to The Craft of Research, is the gathering of “information to answer an question that solves a problem” (Booth p.10). While most people believe that the question we are looking to answer is one that has not been asked before. This is not the case for most forms of research. We preform the act of research every time we look a simple trivia fact on the Internet or when we look up an equation in a textbook. The seeking of knowledge from a source that we have deem trustworthy is research. In this class, for example, one could research the history and current state of quantum computing. Including the current limitation and problems of the this field.

\textbf{Question \#2}\\
\indent \textit{\indent Why write?}\\
\indent \textit{\indent Note as many reasons as you can for why you have written in the past, for example, in classes, labs, work, or social situations.}

	Booth, Colomb, and Williams give three reasons for writing they are: to remember, to understand, and to test your thinking. They argue that the intention of writing is not just for the reader, but also for the writer themselves. By writing the author gains an insight into the area that they are writing about. This allows them to gain a better understanding of their own reasoning as well as revealing any holes in their logic. I have written for may reasons all of which revolve around communicating ideas. I write in my problem sets to explain my process behind my problem solving. I write in a lab report to explain what I did in the experiment. 

\textbf{Question \#3}\\
\indent \textit{\indent The authors repeatedly mention a ‘conversation.’ Who is taking part in this conversation?  Why is it important to think of your writing as part of this conversation? How might that change how you write?}

The conversation the authors are referring to the greater conversation through history about your topic. Many ideas and facts have been presented through the years and by doing research and writing it you are adding your opinion to the conversation. This makes you, as a writer, focus on the fact that what you say is going to be looked at critically by others and you have to account for the fact that people may and will disagree.

\textbf{Question \#4}\\
\indent \textit{\indent What happens when you miscast your audience?}

If you miss cast your audience you will not present the information you have in a way that they want to receive it. You can be too detailed and not entertaining enough or you can not be rigorous enough, and the audience will not trust your findings.

\textbf{Question \#5}\\
\indent \textit{\indent Has the Internet changed any of this? How/why/why not?}

The Internet has not changed the conversation itself, but has changed the way the conversation is held and who is allowed to talk. Before, only few people could join the community to voice their opinions, but now everyone can see and contribute to the greater conversation with out leaving their home.

\textbf{Question \#6}\\
\indent \textit{\indent How can you offer more than a collection of facts when you write about your work?}

You can offer more than a collection of fact by critically thinking about the facts that you present. By doing this you draw connections between the facts that you have, you can infer things that may not be black and white. This increases the robustness of the conversation and can create a disagreement that can be resolved through open discussion.
\end{document}

