\documentclass[11pt]{article}

\usepackage{latexsym}
\usepackage{amssymb}
\usepackage{amsthm}
\usepackage{enumerate}
\usepackage{amsmath}
\usepackage{cancel}
\numberwithin{equation}{section}

\setlength{\evensidemargin}{.25in}
\setlength{\oddsidemargin}{-.25in}
\setlength{\topmargin}{-.75in}
\setlength{\textwidth}{6.5in}
\setlength{\textheight}{9.5in}
\newcommand{\due}{September 11th, 2017}
\newcommand{\HWnum}{1}
\newcommand{\grad}{\bold\nabla}
\newcommand{\vecE}{\vec{E}}
\newcommand{\scrptR}{\vec{\mathfrak{R}}}
\newcommand{\kapa}{\frac{1}{4\pi\epsilon_0}}
\newcommand{\emf}{\mathcal{E}}
\newcommand{\unit}[1]{\ensuremath{\, \mathrm{#1}}}
\newcommand{\real}{\textnormal{Re}}
\newcommand{\Erf}{\textnormal{Erf}}
\newcommand{\sech}{\textnormal{sech}}
\newcommand{\scrO}{\mathcal{O}}
\newcommand{\levi}{\widetilde{\epsilon}}
\newcommand{\partiald}[2]{\ensuremath{\frac{\partial{#1}}{\partial{#2}}}}
\newcommand{\norm}[2]{\langle{#1}|{#2}\rangle}
\newcommand{\inprod}[2]{\langle{#1}|{#2}\rangle}
\newcommand{\ket}[1]{|{#1}\rangle}
\newcommand{\bra}[1]{\langle{#1}|}





\begin{document}
\begin{titlepage}
\setlength{\topmargin}{1.5in}
\begin{center}
\Huge{Physics 3320} \\
\LARGE{Principles of Electricity and Magnetism II} \\
\Large{Professor Ana Maria Rey} \\[1cm]

\huge{Homework \#\HWnum}\\[0.5cm]

\large{Joe Becker} \\
\large{SID: 810-07-1484} \\
\large{\due} 

\end{center}

\end{titlepage}



\section{Problem \#1}
For two noncommuting operators $\hat{A}$ and $\hat{B}$ which satisfy the conditions
\begin{equation}
    [[\hat{A},\hat{B}],\hat{A}] = [[\hat{A},\hat{B}],\hat{B}] = 0
    \label{Prob1}
\end{equation}
we can find an expression for $e^{\hat{A}+\hat{B}}$ we define a function
    $$f(x) \equiv e^{\hat{A}x}e^{\hat{B}x}$$
and take the derivate with respect to $x$ to find that
\begin{align*}
    \frac{df(x)}{dx} &=  \hat{A}e^{\hat{A}x}e^{\hat{B}x} + e^{\hat{A}x}e^{\hat{B}x}\hat{B}\\
                     &=  e^{\hat{A}x}e^{\hat{B}x}\left(e^{-\hat{B}x}\hat{A}e^{\hat{B}x} + \hat{B}\right)\\
                     &=  f(x)\left(\left(\hat{A} + [\hat{A},\hat{B}]x + \cancelto{0}{\frac{1}{2!}[[\hat{A},\hat{B}],\hat{B}] +...}\right) + \hat{B}\right)\\
                     &\Downarrow\\
    \frac{df}{f}     &= \left(\hat{A} + [\hat{A},\hat{B}]x + \hat{B}\right)dx\\
    \ln(f)           &= \hat{A}x + \frac{1}{2}[\hat{A},\hat{B}]x^2 + \hat{B}x\\
                     &\Downarrow\\
    f(x)           &= \exp\left(\hat{A}x + \frac{1}{2}[\hat{A},\hat{B}]x^2 + \hat{B}x\right)
\end{align*}
Note that we used the \emph{Baker-Hausdorff formula}
\begin{equation}
    e^{-\hat{T}}\hat{A}e^{\hat{T}} = \hat{A} + [\hat{A},\hat{B}]x + \frac{1}{2!}[[\hat{A},\hat{B}],\hat{B}] + \frac{1}{3!}[[[\hat{A},\hat{T}],\hat{T},\hat{T}]] + ...
    \label{Baker}
\end{equation}
which due to equation \ref{Prob1} is zero for all terms except the first two. Now for $x=1$ we find that
\begin{align*}
    e^{\hat{A}}e^{\hat{B}} &= e^{\hat{A}+\hat{B}}e^{\frac{1}{2}[\hat{A},\hat{B}]}\\
                           &\Downarrow\\
    e^{\hat{A}+\hat{B}}    &= e^{-\frac{1}{2}[\hat{A},\hat{B}]}e^{\hat{A}}e^{\hat{B}} 
\end{align*}
Note that we can factor $f(x)$ differently which yields
    $$\frac{df(x)}{dx} = e^{\hat{B}x}e^{\hat{A}x}\left(\hat{A} + e^{-\hat{A}x}\hat{B}e^{\hat{A}x}\right)$$
which results in 
    $$e^{\hat{A}+\hat{B}} = e^{\frac{1}{2}[\hat{A},\hat{B}]}e^{\hat{B}}e^{\hat{A}}$$
by following the same steps as above.
\pagebreak

\section{Problem \#2}
For two noncommuting operators $\hat{A}$ and $\hat{B}$ and the parameter $\alpha$ we can see find a generalized 
result for equation \ref{Baker} by taking $e^{-\alpha\hat{A}}\hat{B}e^{\alpha\hat{A}}$ and expanding each exponential
to yield
\begin{align*}
    e^{-\alpha\hat{A}}\hat{B}e^{\alpha\hat{A}} &= \left(1-\alpha\hat{A}+\frac{(\alpha\hat{A})^2}{2!}+...\right)\hat{B}\left(1+\alpha\hat{A}+\frac{(\alpha\hat{A})^2}{2}+...\right)\\
                                               &= \hat{B} - \alpha[\hat{A},\hat{B}] + \frac{\alpha^2}{2!}[\hat{A},[\hat{A},\hat{B}]] + ...
\end{align*}
Note that the result follows from simply expanding the products and grouping powers of $\alpha$.

\section{Problem \#3}
Given a function, $f(a,a^{\dagger})$ which can be expanded in a power series of $a$ and $a^{\dagger}$
\begin{equation}
    f(a,a^{\dagger}) = 1 + \partiald{f}{a}a + \partiald{f}{a^{\dagger}}a^{\dagger} + \partiald{f}{aa^{\dagger}}aa^{\dagger} + \partiald{f}{a^{\dagger}a}a^{\dagger}a + ...
    \label{Prob3}
\end{equation}
while noting that $[a,a^{\dagger}] = 1$ we can find the following relations
\begin{enumerate}[(a)]
\item
    \begin{align*}
        [a,f(a,a^{\dagger})] &= \partiald{f}{a}\cancel{[a,a]} + \partiald{f}{a^{\dagger}}[a,a^{\dagger}] + \partiald{f}{aa^{\dagger}}\cancel[{a,aa^{\dagger}}] + \partiald{f}{a^{\dagger}a}\cancel{[a,a^{\dagger}a]} + ...\\
                             &= \partiald{f}{a^{\dagger}}
    \end{align*}

\item
    \begin{align*}
        [a^{\dagger},f(a,a^{\dagger})] &= \partiald{f}{a}[a^{\dagger},a] + \partiald{f}{a^{\dagger}}\cancel{[a^{\dagger},a^{\dagger}]} + \partiald{f}{aa^{\dagger}}\cancel[{a^{\dagger},aa^{\dagger}]} + \partiald{f}{a^{\dagger}a}\cancel{[a^{\dagger},a^{\dagger}a]} + ...\\
                             &= -\partiald{f}{a}
    \end{align*}

\item
    \begin{align*}
        e^{-\alpha{aa^{\dagger}}}f(a,a^{\dagger})e^{\alpha{aa^{\dagger}}} &= f(a,a^{\dagger}) - \alpha[aa^{\dagger},f(a,a^{\dagger})] + ...\\
                                                                          &= f(a,a^{\dagger}) - \alpha a^{\dagger}[a,f(a,a^{\dagger}] - \alpha [a^{\dagger},f(a,a^{\dagger}]a+...
                                                                          &= f(a,a^{\dagger}) - \alpha a^{\dagger}\partiald{f}{a^{\dagger}} + \alpha a\partiald{f}{a}+...\\
                                                                          &= f(ae^{\alpha},a^{\dagger}e^{-\alpha})
    \end{align*}
\end{enumerate}
\pagebreak

\section{Problem \#4}
Expanding the exponential we can show that
\begin{align*}
    [a,e^{-\alpha{a^{\dagger}a}}] &= [a, 1 - \alpha{a^{\dagger}a} + (\alpha{a^{\dagger}a})^2+...]\\
                                  &=  - \alpha[a,a^{\dagger}a] + \alpha^2[a,(a^{\dagger}a)^2]+...\\
                                  &=  - \alpha{a} + \alpha^2(a-2a^{\dagger}a)+...\\
                                  &= \left(-\alpha + \frac{1}{2}\alpha^2 + ...\right)\left(1-\alpha{a^{\dagger}a}+...\right)a\\
                                  &= \left(e^{-\alpha}-1\right)e^{-\alpha{a^{\dagger}a}}a
\end{align*}
and
\begin{align*}
    [a^{\dagger},e^{-\alpha{a^{\dagger}a}}] &= [a^{\dagger}, 1 - \alpha{a^{\dagger}a} + (\alpha{a^{\dagger}a})^2+...]\\
                                  &=  - \alpha[a^{\dagger},a^{\dagger}a] + \alpha^2[a^{\dagger},(a^{\dagger}a)^2]+...\\
                                  &=  \alpha{a^{\dagger}} + \alpha^2(a-2a^{\dagger}a^{\dagger})+...\\
                                  &= \left(\alpha + \frac{1}{2}\alpha^2 + ...\right)\left(1-\alpha{a^{\dagger}a}+...\right)a^{\dagger}\\
                                  &= \left(e^{\alpha}-1\right)e^{-\alpha{a^{\dagger}a}}a^{\dagger}
\end{align*}

\section{Problem \#5}
We can verify that $\sum_i\hat{\mathbf{e}}_i\cdot\hat{\mathbf{e}}_i = \mathbf{1}$ by taking the dot product with any vector 
$\mathbf{v}$ which we define as
\begin{equation}
    \mathbf{v} = \sum_{i}v_i\hat{\mathbf{e}}_i
    \label{Prob5}
\end{equation}
we note that by equation \ref{Prob5} we find that
    $$\mathbf{v}\cdot\mathbf{v} = \sum_iv_i^2\hat{\mathbf{e}}_i\cdot\hat{\mathbf{e}}_i = \sum_iv_i^2\sum_i\hat{\mathbf{e}}_i\cdot\hat{\mathbf{e}}_i = \sum_{i}v_i^2$$
which only holds true if 
    $$\sum_i\hat{\mathbf{e}}_i\cdot\hat{\mathbf{e}}_i = \mathbf{1}$$
So, if we define the direction of our basis to be along the wavevetor $\mathbf{k}$ we have
    $$\hat{\mathbf{e}}_1 = \hat{\epsilon}_{\mathbf{k}}^{(1)},\qquad
        \hat{\mathbf{e}}_2 = \hat{\epsilon}_{\mathbf{k}}^{(2)},\qquad
        \hat{\mathbf{e}}_3 = \mathbf{k}/k$$
which if we take to be in polar coordinates we can find that 
\begin{align*}
    \hat{\epsilon}_{\mathbf{k}}^{(1)} &\equiv (\sin\phi,-\cos\phi,0)\\
    \hat{\epsilon}_{\mathbf{k}}^{(2)} &\equiv (\cos\theta\cos\phi,\cos\theta\sin\phi,-\sin\theta)
\end{align*}
Which if we take the dot product we find that
\begin{align*}
    \mathbf{1} &= \hat{\epsilon}_{\mathbf{k}}^{(1)}\cdot\hat{\epsilon}_{\mathbf{k}}^{(1)} + \hat{\epsilon}_{\mathbf{k}}^{(2)}\cdot\hat{\epsilon}_{\mathbf{k}}^{(2)} + k^2/k^2\\
      &= \sin^2\phi+\cos^2\phi + \cos^2\theta\cos^2\phi + \cos^2\theta\sin^2\phi + \sin^2\theta + 1\\
      &= 3
\end{align*}
So it follows that
$$\epsilon_{\mathbf{k}i}^{(1)}\epsilon_{\mathbf{k}j}^{(1)} + \epsilon_{\mathbf{k}i}^{(2)}\epsilon_{\mathbf{k}j}^{(2)} = \delta_{ij} - \frac{k_ik_j}{k^2}$$

\pagebreak

\section{Problem \#6}
For a one dimensional system where two conducting reflecting mirrors are placed at a distance $L$ apart we see that
\begin{align*}
    H_0^{box} &= \sum_{l}\frac{1}{2}\hbar\nu_l = \sum_{l}\frac{1}{2}\hbar{c}\frac{l\pi}{L}
\end{align*}
and for outside the box we have the vacuum energy
    $$H_0^{vac} = \frac{c\hbar}{2}\int_{0}^{\infty}\frac{l\pi}{L'}dl$$
this allows us to calculate the difference in energy as
\begin{align*}
    H_0^{box} - H_0^{vac}  &= \sum_{l}\frac{1}{2}\hbar{c}\frac{l\pi}{L} - \frac{c\hbar}{2}\int_{0}^{\infty}\frac{l\pi}{L'}dl\\
                           &= -\frac{c\hbar}{2L}\frac{1}{180}
\end{align*}

\end{document}

