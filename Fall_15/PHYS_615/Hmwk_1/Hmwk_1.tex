\documentclass[11pt]{article}

\usepackage{latexsym}
\usepackage{amssymb}
\usepackage{amsthm}
\usepackage{enumerate}
\usepackage{amsmath}
\usepackage{cancel}
\numberwithin{equation}{section}

\setlength{\evensidemargin}{.25in}
\setlength{\oddsidemargin}{-.25in}
\setlength{\topmargin}{-.75in}
\setlength{\textwidth}{6.5in}
\setlength{\textheight}{9.5in}
\newcommand{\due}{September 9th, 2015}
\newcommand{\HWnum}{1}
\newcommand{\grad}{\bold\nabla}
\newcommand{\vecE}{\vec{E}}
\newcommand{\scrptR}{\vec{\mathfrak{R}}}
\newcommand{\kapa}{\frac{1}{4\pi\epsilon_0}}
\newcommand{\emf}{\mathcal{E}}
\newcommand{\unit}[1]{\ensuremath{\, \mathrm{#1}}}
\newcommand{\real}{\textnormal{Re}}
\newcommand{\Erf}{\textnormal{Erf}}
\newcommand{\sech}{\textnormal{sech}}
\newcommand{\scrO}{\mathcal{O}}
\newcommand{\levi}{\widetilde{\epsilon}}
\newcommand{\partiald}[2]{\ensuremath{\frac{\partial{#1}}{\partial{#2}}}}
\newcommand{\norm}[2]{\langle{#1}|{#2}\rangle}
\newcommand{\inprod}[2]{\langle{#1}|{#2}\rangle}
\newcommand{\ket}[1]{|{#1}\rangle}
\newcommand{\bra}[1]{\langle{#1}|}





\begin{document}
\begin{titlepage}
\setlength{\topmargin}{1.5in}
\begin{center}
\Huge{Physics 3320} \\
\LARGE{Principles of Electricity and Magnetism II} \\
\Large{Professor Ana Maria Rey} \\[1cm]

\huge{Homework \#\HWnum}\\[0.5cm]

\large{Joe Becker} \\
\large{SID: 810-07-1484} \\
\large{\due} 

\end{center}

\end{titlepage}



\section{Problem \#1}
We can solve the ordinary differential equation
\begin{equation}
\frac{y'}{y} = x\log(y) + 2x
\label{Prob1}
\end{equation}
through the method of separation of variables where we place equation \ref{Prob1} in the form
$$A(y)dy = B(x)dx.$$
We can do this by
\begin{align*}
\frac{y'}{y} &= x\log(y) + 2x\\
&\Downarrow\\
\frac{dy}{dx}\frac{1}{y} &= x(\log(y) + 2)\\
&\Downarrow\\
\frac{dy}{y} &= xdx(\log(y) + 2)\\
\frac{dy}{y(\log(y)+2)} &= xdx\\
&\Downarrow\\
\int\frac{dy}{y(\log(y)+2)} &= \int xdx = \frac{1}{2}x^2 + C
\end{align*}
Note we can solve the integral with by substituting
\begin{align*}
u &= \log(y) + 2\\
du &= \frac{1}{y}dy.
\end{align*}
Therefore we can solve the integral by
\begin{align*}
\int\frac{dy}{y(\log(y)+2)} &\Rightarrow \int\frac{du}{u}\\
&= \log(u) \Rightarrow \log(\log(y)+2)
\end{align*}
Now we can solve for $y(x)$ as
\begin{align*}
\log(\log(y)+2) &= \frac{1}{2}x^2+C\\
&\Downarrow\\
\log(y)+2 &= Ce^{x^2/2}\\
\log(y) &= Ce^{x^2/2}-2\\
&\Downarrow\\
y(x) &= \exp(C\exp(x^2/2)-2)\\
&= \exp(\exp(x^2/2)-2/C)\\
&= C\exp(\exp(x^2/2))
\end{align*}
Note that we grouped the constants $\exp(-C/2)$ into another constant $C$ without loss of 
generality.

\pagebreak

\section{Problem \#2}
For the differential equation 
\begin{equation}
y' = \sin(x+y) -1,
\label{Prob2}
\end{equation}
we change variables using the equality
$$z = \alpha x + \beta y + \gamma$$
where we choose $\alpha=\beta = 1$ and $\gamma = 0$ so that we can have
$$z = x + y$$
with
$$\frac{dz}{dx} = 1 + \frac{dy}{dx}$$
solving for $$y'$$ we see that we have
$$\frac{dy}{dx} = \frac{dz}{dx} - 1$$
Which implies that equation \ref{Prob2} becomes 
$$\frac{dy}{dx} = \sin(x+y) - 1 \Rightarrow \sin(z) -1 = \frac{dz}{dx} - 1.$$
Therefore by changing to the variable $z$ we have a separable differential equation
$$\frac{dz}{dx} = \sin(z).$$
Which we separate as
\begin{align*}
\frac{dz}{dx} &= \sin(z)\\
\frac{dz}{\sin(z)} &= dx\\
&\Downarrow\\
\int dx &= \int\frac{dz}{\sin(z)}\\
x+C &= \int\csc(z)dz\\
&= -\ln(\cos(z/2))+\ln(\sin(z/2))\\
&= -\ln\left(\frac{\sin(z/2)}{\cos(z/2)}\right)\\
&= -\ln(\tan(z/2))
\end{align*}
Next we solve for $z$ by 
\begin{align*}
x+C &= -\ln(\tan(z/2))+\\
&\Downarrow\\
Ae^{-x} &= \tan(z/2)\\
&\Downarrow\\
z &= 2\arctan\left(Ae^{-x}\right)
\end{align*}
Now we replace $z$ in the solution for $y$ that is given by $y=z-x$. Therefore,
$$y = 2\arctan\left(Ae^{-x}\right) - x$$

\pagebreak


\section{Problem \#3}
To solve the ordinary differential equation
\begin{equation}
3x^2y^2 + 2x^3yy' + 10y^4y' = 0
\label{Prob3}
\end{equation}
we note that equation \ref{Prob3} can be rewritten as
\begin{align*}
3x^2y^2 + 2x^3yy' + 10y^4y' &= 0\\
&\Downarrow\\
3x^2y^2 &=  -(2x^3y + 10y^4)y'\\
&\Downarrow\\
(3x^2y^2)dx + (2x^3y + 10y^4)dy &= 0
\end{align*}
which is in the form
$$A(x,y)dx + B(x,y)dy = 0$$
where
\begin{align*}
A(x,y) &= 3x^2y^2 \\
B(x,y) &= 2x^3y + 10y^4 
\end{align*}
We test to see if this equation is \emph{exact} by the condition
\begin{equation}
\frac{\partial A}{\partial y} = \frac{\partial B}{\partial x}.
\label{Exact}
\end{equation}
We calculate 
\begin{align*}
\frac{\partial A}{\partial y} &= \frac{\partial}{\partial y}(3x^2y^2)\\
&= 6x^2y
\end{align*}
and
\begin{align*}
\frac{\partial B}{\partial x} &= \frac{\partial}{\partial x}(2x^3y + 10y^4)\\
&= 6x^2y
\end{align*}
So, we see that equation \ref{Exact} holds true which implies that equation \ref{Prob3} is 
exact. Therefore, there exists a function $u$ such that
\begin{align*}
\frac{\partial u}{\partial x} &= A(x,y)\\
\frac{\partial u}{\partial y} &= B(x,y).
\end{align*}
We note that we can solve for $u$ using the integral
\begin{align*}
u(x,y) &= \int A(x,y)dx + \int B(x,y)dy = C = \textnormal{const.}\\
&\Downarrow\\
u(x,y) &= \int 3x^2y^2dx + \int 2x^3y+10y^4dy\\
&= x^3y^2 + x^3y^2+2y^5dy + C\\
&=  2x^3y^2+2y^5 + A
\end{align*}
So we can see that the solution of equation \ref{Prob3} is 
$$2x^3y^2+2y^5 = C$$
Note we combined constants $A$ and $C$ without loss of generality.

\pagebreak

\section{Problem \#4}
For the given second order ODE
\begin{equation}
y'' - 3y' + 2y = e^{3x}(x+x^2)
\label{Prob4}
\end{equation}
we note that this is a nonhomogeneous linear ODE with constant coefficients. Therefore we 
first need to solve for the homogeneous solution, 
$$0 = y_0'' - 3y_0' + 2y_0,$$ 
by the ansatz 
$$y_0 = e^{mx}.$$
We see that our homogeneous version of equation \ref{Prob4} becomes
\begin{align*}
0 &= y_0'' - 3y_0' + 2y_0\\
&= m^2\cancel{e^{mx}} - 3m\cancel{e^{mx}} + 2\cancel{e^{mx}}\\
&= m^2 - 3m + 2\\
&= (m-1)(m-2).
\end{align*}
Solving for $m$ gives us $m =1,2$ so $y_0$ is given by a linear combination 
$$y_0(x) = C_1e^{x} + C_2e^{2x}$$
Next we need to construct the particular solution, $y_p$, by the ansatz
$$y_p = e^{3x}(ax^2+bx+c)$$
where
\begin{align*}
y_p' &= \frac{d}{dx}\left(e^{3x}(ax^2+bx+c)\right)\\
&= e^{3x}(3)(ax^2+bx+c) + e^{3x}(2ax+b)\\
&= e^{3x}(3ax^2+3bx+3c+2ax+b)\\
&= e^{3x}(3ax^2+(3b+2a)x+3c+b)
\end{align*}
and
\begin{align*}
y_p'' &= \frac{d}{dx}\left(e^{3x}(3ax^2+(3b+2a)x+3c+b)\right)\\
&= e^{3x}(3)(3ax^2+(3b+2a)x+3c+b) + e^{3x}(6ax+(3b+2a))\\
&= e^{3x}(9ax^2+(9b+6a)x+9c+3b + 6ax+(3b+2a))\\
&= e^{3x}(9ax^2+(12a+9b)x+9c+6b+2a))
\end{align*}
Now we can replace $y_p$ into equation \ref{Prob4} by first noting the common $e^{3x}$ factor
that we can cancel yielding the resulting equation 
\begin{align*}
9ax^2+(12a+9b)x+9c+6b+2a - 3(3ax^2+(2a+3b)x+3c+b) + 2(ax^2+bx+c) &= x+x^2
\end{align*}
Grouping like terms gives us
\begin{align*}
\cancel{9ax^2}+(12a+9b)x+9c+6b+2a - \cancel{9ax^2} - (9b+6a)x - 9c - 3b + 2ax^2 + 2bx + 2c &= x+x^2\\
2ax^2 + (12a+9b-9b-6a+2b)x + 9c+6b+2a-9c-3b+2c &= x+x^2\\
2ax^2 + (6a+2b)x + 2a+3b+2c &= x+x^2
\end{align*}
Which yields the system of equations
\begin{align*}
2a &= 1\\
6a + 2b &= 1\\
2a + 3b + 2c &= 0
\end{align*}
Which implies that
\begin{align*}
2 &= \frac{1}{2}\\
b &= -1\\
c &= 1
\end{align*}
$$\Rightarrow y_p(x) = e^{3x}\left(\frac{1}{2}x^2-x+1\right)$$
Therefore our solution is a linear combination of both $y_0$ and $y_p$ given by
$$y(x) = e^{3x}\left(\frac{1}{2}x^2-x+1\right) + y_0(x) = C_1e^{x} + C_2e^{2x}$$

\end{document}

