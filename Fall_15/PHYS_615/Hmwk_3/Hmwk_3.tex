\documentclass[11pt]{article}

\usepackage{latexsym}
\usepackage{amssymb}
\usepackage{amsthm}
\usepackage{enumerate}
\usepackage{amsmath}
\usepackage{cancel}
\numberwithin{equation}{section}

\setlength{\evensidemargin}{.25in}
\setlength{\oddsidemargin}{-.25in}
\setlength{\topmargin}{-.75in}
\setlength{\textwidth}{6.5in}
\setlength{\textheight}{9.5in}
\newcommand{\due}{September 23rd, 2015}
\newcommand{\HWnum}{3}
\newcommand{\grad}{\bold\nabla}
\newcommand{\vecE}{\vec{E}}
\newcommand{\scrptR}{\vec{\mathfrak{R}}}
\newcommand{\kapa}{\frac{1}{4\pi\epsilon_0}}
\newcommand{\emf}{\mathcal{E}}
\newcommand{\unit}[1]{\ensuremath{\, \mathrm{#1}}}
\newcommand{\real}{\textnormal{Re}}
\newcommand{\Erf}{\textnormal{Erf}}
\newcommand{\sech}{\textnormal{sech}}
\newcommand{\scrO}{\mathcal{O}}
\newcommand{\levi}{\widetilde{\epsilon}}
\newcommand{\partiald}[2]{\ensuremath{\frac{\partial{#1}}{\partial{#2}}}}
\newcommand{\norm}[2]{\langle{#1}|{#2}\rangle}
\newcommand{\inprod}[2]{\langle{#1}|{#2}\rangle}
\newcommand{\average}[1]{\left\langle{#1}\right\rangle}
\newcommand{\ket}[1]{|{#1}\rangle}
\newcommand{\bra}[1]{\langle{#1}|}
\newcommand{\Resid}[2]{\ensuremath{\textnormal{Res}\left[{#1},{#2}\right]}}





\begin{document}
\begin{titlepage}
\setlength{\topmargin}{1.5in}
\begin{center}
\Huge{Physics 3310} \\
\LARGE{Principles of Electricity and Magnetism 1} \\
\Large{Professor Thomas R. Schibli} \\[1cm]

\huge{Homework \#\HWnum}\\[0.5cm]

\large{Joe Becker} \\
\large{SID: 810-07-1484} \\
\large{\due} 

\end{center}

\end{titlepage}



\section{Problem \#1}
For the differential equation given by
\begin{equation}
(1-x^3)y''-6x^2y'-6xy=0
\label{Prob1}
\end{equation}
with the boundary conditions
$$y(0)=1,\qquad y'(0)=0$$
can be solved by finding a series for $y$ in the form
$$y(x) = \sum_{n=0}^{\infty}a_nx^n.$$
Where we note that
\begin{align*}
y'(x) &= \sum_{n=0}^{\infty}a_nnx^{n-1}\\
y''(x) &= \sum_{n=0}^{\infty}a_nn(n-1)x^{n-2}
\end{align*}
and we replace these sums into equation \ref{Prob1} to get
\begin{align*}
(1-x^3)\sum_{n=0}^{\infty}a_nn(n-1)x^{n-2} - 6x^2\sum_{n=0}^{\infty}a_nnx^{n-1} - 6x\sum_{n=0}^{\infty}a_nx^n &= 0\\
\sum_{n=0}^{\infty}a_nn(n-1)x^{n-2}-x^3\sum_{n=0}^{\infty}a_nn(n-1)x^{n-2} - 6x^2\sum_{n=0}^{\infty}a_nnx^{n-1} - 6x\sum_{n=0}^{\infty}a_nx^n &= 0\\
\sum_{n=-3}^{\infty}a_{n+3}(n+2)(n+3)x^{n+1}-\sum_{n=0}^{\infty}a_nn(n-1)x^{n+1} - \sum_{n=0}^{\infty}6a_nnx^{n+1} - \sum_{n=0}^{\infty}6a_nx^{n+1} &= 0\\
a_2 + \sum_{n=0}^{\infty}\left(\frac{}{}a_{n+3}(n+2)(n+3)-a_nn(n-1)-6a_nn-6a_n\right)x^{n+1} &= 0
\end{align*}
Note that we changed the dummy index over the first summation to $n\rightarrow n+3$. We note 
that the $n=-3,-2$ terms are zero and the $n=-1$ term is $a_2$ which is the pre-factor. We
now see that for this equation to be true for all $x$ the following needs to be true. This 
implies that $a_2=0$.
\begin{align*}
0 &= a_{n+3}(n+2)(n+3)-a_nn(n-1)-6a_nn-6a_n\\
0 &= a_{n+3}(n+2)(n+3)-a_n(n(n-1)+6n+6)\\
0 &= a_{n+3}(n+2)(n+3)-a_n(n^2+5n+6)\\
0 &= a_{n+3}(n+2)(n+3)-a_n((n+2)(n+3))\\
&\Downarrow\\
a_{n+3} &= \frac{(n+2)(n+3)}{(n+2)(n+3)}a_n\\
a_{n+3} &= a_n
\end{align*}
Now that we have a recursion relation for the coefficients $a_n$ we can apply the boundary
conditions to determine the values for $a_n$. Applying $y(0) = 1$ gives
\begin{align*}
y(0) = 1 &=  \sum_{n=0}^{\infty}a_n(0)^n\\
&= a_0 + \cancelto{0}{\sum_{n=1}^{\infty}a_n(0)^n}\\
&\Downarrow\\
1 &= a_0
\end{align*}
and
\begin{align*}
y'(0) = 0 &=   \sum_{n=0}^{\infty}a_nn(0)^{n-1}\\
&\Downarrow\\
0 &= a_1
\end{align*}
Therefore we see only the $a_0,a_3,a_6,...$ terms are non-zero and are equal to 1. This 
makes the series for $y$ become
$$y(x) = 1 + x^3 + x^6 + x^9 +... =\sum_{n=0}^{\infty}x^{3n} = \frac{1}{1-x^3}$$
we can verify this is a solution by first noting that
\begin{align*}
y(x) &= \frac{1}{1-x^3}\\
y'(x) &= \frac{3x^2}{(1-x^3)^2}\\
y''(x) &= 3\left(\frac{2x}{(1-x^3)^2} + \frac{6x^4}{(1-x^3)^3}\right) = \frac{6x(2x^3+1)}{(1-x^3)^3}
\end{align*}
and plugging back into equation \ref{Prob1} which yields
\begin{align*}
(1-x^3)y''-6x^2y'-6xy &= (1-x^3)\frac{6x(2x^3+1)}{(1-x^3)^3} - 6x^2\frac{3x^2}{(1-x^3)^2} - 6x\frac{1}{1-x^3}\\
&= \frac{6x(2x^3+1)}{(1-x^3)^2} - \frac{18x^4}{(1-x^3)^2} - \frac{6x(1-x^3)}{(1-x^3)^2}\\ 
&= \frac{12x^4+6x-18x^4-6x+6x^4)}{(1-x^3)^2}\\
&=0
\end{align*}
Therefore, 
$$y(x) = \frac{1}{1-x^3}$$
is a solution for equation \ref{Prob1}.

\pagebreak

\section{Problem \#2}
For the differential equation 
\begin{equation}
y'' - 2xy' - 2y =0
\label{Prob2}
\end{equation}
we can find a solution by a power series like in Problem 1. With
\begin{align*}
y(x) &= \sum_{n=0}^{\infty}a_nx^{n}\\
y'(x) &= \sum_{n=0}^{\infty}a_nnx^{n-1}\\
y''(x) &= \sum_{n=0}^{\infty}a_nn(n-1)x^{n-2}
\end{align*}
which makes equation \ref{Prob2} become
\begin{align*}
0 &= \sum_{n=0}^{\infty}a_nn(n-1)x^{n-2} - 2x\sum_{n=0}^{\infty}a_nnx^{n-1} - 2\sum_{n=0}^{\infty}a_nx^{n}\\
&= \sum_{n=-2}^{\infty}a_{n+2}(n+1)(n+2)x^{n} - 2\sum_{n=0}^{\infty}a_nnx^{n} - 2\sum_{n=0}^{\infty}a_nx^{n}\\
&= \sum_{n=0}^{\infty}\left(\frac{}{}a_{n+2}(n+1)(n+2)- 2a_nn- 2a_n\right)x^{n}
\end{align*}
This leads us to the condition that for all $x$
\begin{align*}
0 &= a_{n+2}(n+1)(n+2)- 2a_nn- 2a_n\\
&\Downarrow\\
a_{n+2} &= \frac{2n+2}{(n+1)(n+2)}a_n\\
&= \frac{2}{(n+2)}a_n
\end{align*}
We see that even $n$ $a_n$ are proportional to $a_0$ and odd $n$, $a_n$ are proportional to 
$a_1$. This gives us two solutions to equation \ref{Prob2} given by
\begin{align*}
y_1(x) &= a_0 + a_2x^2 + a_4x^4 + a_6x^6...\\
&= a_0 + a_0x^2 + \frac{1}{2}a_2x^4 + \frac{1}{3}a_4x^6...\\
&= a_0 + a_0x^2 + \frac{1}{2}a_0x^4 + \frac{1}{6}a_0x^6...\\
&= a_0\sum_{n=0}^{\infty}\frac{x^{2n}}{n!}\\
&= a_0e^{x^2}
\end{align*}
and the odd terms yield
\begin{align*}
y_2(x) &= a_1x + a_3x^3 + a_5x^5 + a_7x^7 + ...\\
&= a_1x + \frac{2}{3}a_1x^3 + \frac{2}{5}a_3x^5 + \frac{2}{7}a_5x^7 + ...\\
&= a_1\left(x + \frac{2}{3}x^3 + \frac{4}{5\times3}x^5 + \frac{8}{7\times5\times3}x^7 + ...\right)\\
&= a_1\sum_{n=0}^{\infty}\frac{2^nx^{2n+1}}{(2n+1)!!} \\
&= a_1\frac{\sqrt{\pi}}{2}e^{x^2}\textnormal{erf}(x)
\end{align*}
We can test to see if these two solutions are linearly independent by calculating the 
\emph{Wronskian} and if the result is nonzero then we can say the two solutions are linearly 
independent. We note that for a second order ODE the \emph{Wronskian} is given by
\begin{align*}
W &= y_1y_2' - y_2y_1'
\end{align*}
Which we can calculate $y_1'(x)$ as
\begin{align*}
y_1'(x) &= \frac{d}{dx}\left(a_0e^{x^2}\right)\\
&= a_0(2x)e^{x^2}
\end{align*}
and we calculate $y_2'(x)$ as
\begin{align*}
y_2'(x) &= \frac{d}{dx}\left(a_1\frac{\sqrt{\pi}}{2}e^{x^2}\textnormal{erf}(x)\right)\\
&= a_1\frac{\sqrt{\pi}}{2}\left(2xe^{x^2}\textnormal{erf}(x) + e^{x^2}\frac{2e^{-x^2}}{\pi}\right)\\
&= a_1\frac{\sqrt{\pi}}{2}\left(2xe^{x^2}\textnormal{erf}(x) + \frac{2}{\pi}\right)
\end{align*}
Now we can calculate $W$
\begin{align*}
W &= y_1y_2' - y_2y_1'\\
&= a_0e^{x^2}a_1\frac{\sqrt{\pi}}{2}\left(2xe^{x^2}\textnormal{erf}(x) + \frac{2}{\pi}\right) - a_1\frac{\sqrt{\pi}}{2}e^{x^2}\textnormal{erf}(x)a_0(2x)e^{x^2}\\
&= a_0a_1e^{x^2}\frac{\sqrt{\pi}}{2}\left(2xe^{x^2}\textnormal{erf}(x) + \frac{2}{\pi} - 2xe^{x^2}\textnormal{erf}(x)\right)\\
&= a_0a_1e^{x^2}\frac{1}{\sqrt{\pi}}
\end{align*}
as nonzero for any value of $x$ and nonzero values of $a_0$ and $a_1$. Therefore, the 
solutions we found are linearly independent.

\pagebreak

\section{Problem \#3}
For the differential equation
\begin{equation}
xy'' + \frac{3}{x}y = 1 + x^2
\label{Prob3}
\end{equation}
We first solve the homogeneous equation
$$xy''+\frac{3}{x}y=0$$
with a solution of the form
$$y(x) = x^{s}\sum_{n=0}^{\infty}a_nx^n$$
due to the regular singularity at $x=0$. We note that 
\begin{align*}
y'(x) &= \sum_{n=0}^{\infty}a_n(n+s)x^{n+s-1}\\
y''(x) &= \sum_{n=0}^{\infty}a_n(n+s)(n+s-1)x^{n+s-2}
\end{align*}
which we plug into equation \ref{Prob3} to yield
\begin{align*}
0 &= x\sum_{n=0}^{\infty}a_n(n+s)(n+s-1)x^{n+s-2} + \frac{3}{x}\sum_{n=0}^{\infty}a_nx^{n+s}\\
&= \sum_{n=0}^{\infty}a_n(n+s)(n+s-1)x^{n+s-1} + 3\sum_{n=0}^{\infty}a_nx^{n+s-1}\\
&= \sum_{n=0}^{\infty}a_n\left(\frac{}{}(n+s)(n+s-1) + 3\right)x^{n+s-1}\\
&= a_0\left(\frac{}{}s(s-1) + 3\right)x^{s-1} + \sum_{n=1}^{\infty}a_n\left(\frac{}{}(n+s)(n+s-1) + 3\right)x^{n+s-1}
\end{align*}
Note that we pulled the $n=0$ term out because we want the $x^{s-1}$ term to vanish for all
$x$. This implies
\begin{align*} 
0 &= a_0\left(\frac{}{}s(s-1) + 3\right)\\
&\Downarrow\\
0 &= s^2 - s + 3
\end{align*} 
We see that this has a complex solution where the value of $s$ becomes 
$s=1/2(1\pm i\sqrt{11}))$. So we can pick the positive solution to find an indicial equation
by
\begin{align*}
0 &= a_n\left(\frac{}{}(n+s)(n+s-1) + 3\right)\\
&\Downarrow\\
0 &= (n+1/2+i/2\sqrt{11}))(n+1/2+i/2\sqrt{11}-1) + 3\\
&= (n+1/2(1+i\sqrt{11}))(n+1/2(i\sqrt{11}-1)) + 3\\ 
&= n^2 + \frac{1}{4}(1+i\sqrt{11})(i\sqrt{11}-1) + n\frac{1}{2}(i\sqrt{11}-1) + n\frac{1}{2}(1+i\sqrt{11}) + 3\\
&= n^2 - \frac{12}{4} + in\sqrt{11} + 3\\
&= n^2  + in\sqrt{11} = n(n+i\sqrt{11})
\end{align*}
We have two solutions for $n$ where $n=0,-i\sqrt{11}$. We note that there is no recursion 
relation so in general our solution is of the form $x^(s+n)$ for both values of $n$. 
Therefore our series becomes
\begin{align*}
y_0(x) = a_0x^{1/2+i/2\sqrt{11}+0} + a_1x^{1/2+i/2\sqrt{11}-i\sqrt{11}}&= a_0\sqrt{x}x^{i/2\sqrt{11}} + a_1\sqrt{x}x^{-i/2\sqrt{11}}\\
&= a_0\sqrt{x}\exp\left(\log(x^{i/2\sqrt{11}})\right) + a_1\sqrt{x}\exp\left(\log(x^{-i/2\sqrt{11}})\right)\\
&= a_0\sqrt{x}\exp\left(\frac{i\sqrt{11}}{2}\log x\right) + a_1\sqrt{x}\exp\left(\frac{-i\sqrt{11}}{2}\log x\right)\\
&= a_0\sqrt{x}\cos\left(\frac{\sqrt{11}}{2}\log x\right) + a_1\sqrt{x}\sin\left(\frac{\sqrt{11}}{2}\log x\right)
\end{align*}
Now we can find the particular solution of equation \ref{Prob3} by first calculating the
\emph{Wronskian} by
\begin{align*}
W &= y_1y_2' - y_1'y_2\\
&= a_0\sqrt{x}\cos\left(\frac{\sqrt{11}}{2}\log x\right)\frac{a_1}{2\sqrt{x}}\left(\sin\left(\frac{\sqrt{11}}{2}\log x\right) + \sqrt{11}\cos\left(\frac{\sqrt{11}}{2}\log x\right)\right) \\
&\qquad - a_1\sqrt{x}\sin\left(\frac{\sqrt{11}}{2}\log x\right)\frac{a_0}{2\sqrt{x}}\left(\cos\left(\frac{\sqrt{11}}{2}\log x\right) - \sqrt{11}\sin\left(\frac{\sqrt{11}}{2}\log x\right)\right)\\
&= \frac{a_0a_1\sqrt{11}}{2}\left(\sin^2\left(\frac{\sqrt{11}}{2}\log x\right) + \cos^2\left(\frac{\sqrt{11}}{2}\log x\right)\right)\\
&= \frac{a_0a_1\sqrt{11}}{2}
\end{align*}
We note that this value is nonzero for all $x$ which confirms that $y_1$ and $y_2$ are 
linearly independent. So we can find $y_p$ by
\begin{equation}
y_p = y_2\int\frac{y_1f}{W} - y_1\int\frac{y_2f}{W}
\label{PartSoln}
\end{equation}
Because $W$ is shown to be a constant we can just remove it from the integral in equation
\ref{PartSoln}. So for $f=(1+x^2)/x$ we calculate
\begin{align*}
\int y_1\frac{(1+x^2)}{x}dx &= \frac{a_0}{18}x^{1/2}\left((3+5x^2)\cos(\sqrt{11}/2\log x) + \sqrt{11}(3+x^2)\sin(\sqrt{11}/2\log x)\right)\\
\int y_2\frac{(1+x^2)}{x}dx &= \frac{a_1}{18}x^{1/2}\left((3+5x^2)\sin(\sqrt{11}/2\log x) - \sqrt{11}(3+x^2)\cos(\sqrt{11}/2\log x)\right)
\end{align*}
using Mathematica. Which gives the particular solution by equation \ref{PartSoln}
\begin{align*}
y_p(x) &= \frac{x}{9\sqrt{11}}\left(\cancel{(3+5x^2)\cos(\sqrt{11}/2\log x)\sin(\sqrt{11}/\log x)} + \sqrt{11}(3+x^2)\sin^2(\sqrt{11}/2\log x)\right)\\
&\qquad + \frac{x}{9\sqrt{11}}\left(\cancel{-(3+5x^2)\cos(\sqrt{11}/2\log x)\sin(\sqrt{11}/\log x)} + \sqrt{11}(3+x^2)\cos^2(\sqrt{11}/2\log x)\right)\\
&=  \frac{x}{9}(3+x^2)
\end{align*}
Which gives the general solution
$$y(x) = a_0\sqrt{x}\cos(\sqrt{11}/2\log x) + a_1\sqrt{x}\sin(\sqrt{11}/2\log x) +  \frac{x}{9}(3+x^2)$$

\pagebreak

\section{Problem \#4}
For the differential equation
\begin{equation}
2xy'' + y' + xy = 0
\label{Prob4}
\end{equation}
which can be rewritten as
$$y'' + \frac{1}{2x}y' + \frac{1}{2}y = 0$$
we note that we have a \emph{regular singular point} at $x=0$ this implies that we try a 
solution in the form of a power series of the form 
\begin{align*}
y(x) &= x^{s}\sum_{n=0}^{\infty}a_nx^{n} = \sum_{n=0}^{\infty}a_nx^{n+s}\\
y'(x) &= \sum_{n=0}^{\infty}a_n(n+s)x^{n+s-1}\\
y''(x) &= \sum_{n=0}^{\infty}a_n(n+s)(n+s-1)x^{n+s-2}
\end{align*}
which
\begin{align*}
0 &= \sum_{n=0}^{\infty}a_n(n+s)(n+s-1)x^{n+s-2} + \frac{1}{2x}\sum_{n=0}^{\infty}a_n(n+s)x^{n+s-1} + \frac{1}{2}\sum_{n=0}^{\infty}a_nx^{n+s}\\
&= \sum_{n=-2}^{\infty}a_{n+2}(n+s+2)(n+s+1)x^{n+s} + \frac{1}{2}\sum_{n=0}^{\infty}a_n(n+s)x^{n+s-2} + \frac{1}{2}\sum_{n=0}^{\infty}a_nx^{n+s}\\
&= \sum_{n=-2}^{\infty}a_{n+2}(n+s+2)(n+s+1)x^{n+s} + \frac{1}{2}\sum_{n=-2}^{\infty}a_{n+2}(n+s+2)x^{n+s} + \frac{1}{2}\sum_{n=0}^{\infty}a_nx^{n+s}\\
&=\left(a_0(s)(s-1)+\frac{1}{2}a_0s\right)x^{s-2} + \left(a_1s(s+1)+\frac{1}{2}a_1(s+1)\right)x^{s-1}\\
&\qquad  + \sum_{n=-0}^{\infty}\left(a_{n+2}(n+s+2)(n+s+1)+ \frac{1}{2}a_{n+2}(n+s+2) + \frac{1}{2}a_n\right)x^{n+s}
\end{align*}
Now we take the coefficients the lowest powers of $x$ to vanish, namely $x^{s-2}$ which is 
given by the first term
\begin{align*}
a_0(s)(s-1) + \frac{1}{2}a_0s &= 0\\
a_0\left(s^2-s + \frac{1}{2}s\right) &= 0\\
&\Downarrow\\
s\left(s-\frac{1}{2}\right) &= 0
\end{align*}
assuming $a_0$ is nonzero then we have 
$$s = 0,\frac{1}{2}$$ 
next we take the coefficients with the term $x^{s-1}$ which are from the second term this 
yields
\begin{align*}
\left(a_1s(s+1)+\frac{1}{2}a_1(s+1)\right) &= 0\\
a_1\left(s^2+s+\frac{1}{2}s+\frac{1}{2}\right) &= 0\\
a_1\left(s^2+\frac{3}{2}s+\frac{1}{2}\right) &= 0\\
a_1(s+1)\left(s+\frac{1}{2}\right) &= 0
\end{align*}
We note for both solutions $s=0,1/2$ the only way for the above to  hold true is if $a_1=0$. 
This is okay as the values of $s$ are separated by a non-integer value, which implies that 
the solutions for each $s$ are linearly independent. Next we take the $x^{n+s}$ coefficients
which includes all term to get
\begin{align*}
0 &= a_{n+2}(n+s+2)(n+s+1) + \frac{1}{2}a_{n+2}(n+s+2) + \frac{1}{2}a_n\\
0 &= a_{n+2}\left((n+s+2)(n+s+1) + \frac{1}{2}(n+s+2)\right) + \frac{1}{2}a_n\\
&\Downarrow\\
a_{n+2} &= -\frac{a_n}{2(n+s+2)(n+s+1) + (n+s+2)}\\
\end{align*}
We note that due to this recursion relation we have $a_1=a_{\textnormal{odd}}=0$. We can 
choose $s=0$ so that the relation becomes 
$$a_{n+2} = -\frac{a_n}{2(n+1)(n+2)+(n+2)}$$
which allows us to calculate coefficients in terms of $a_0$ by
\begin{align*}
a_2 &= -\frac{1}{2(1)(2)+2}a_0 = -\frac{1}{6}a_0
a_4 &= -\frac{1}{2(3)(4)+2}a_0 = \frac{1}{168}a_0
a_6 &= -\frac{1}{2(5)(6)+2}a_0 = -\frac{1}{10416}a_0
&\vdots
\end{align*}
So we can say that our first solution is 
$$y_1(x) = a_0\left(-\frac{1}{6} + \frac{1}{168}x^2 - \frac{1}{10416}x^4+...\right)$$
Now we can find the other solution by choosing $s=1/2$ while keeping $a_1=0$ to give the 
recursion relation 
$$a_{n+2} =  -\frac{a_n}{2(n+5/2)(n+3/2)+(n+5/2)} = -\frac{2a_n}{(2n+5)(2n+3)+(2n+5)}$$
Calculating our coefficients yields
\begin{align*}
a_2 &= -\frac{1}{10}a_0\\
a_4 &= -\frac{1}{36}a_2 = \frac{1}{360}a_0\\
a_6 &= -\frac{1}{55}a_4 = -\frac{1}{19800}a_0
&\vdots
\end{align*}
which yields the second solution
$$y_2(x) = a_0\left(-\frac{1}{10}x^{1/2} + \frac{1}{360}x^{5/2} - \frac{1}{19800}x^{7/2}+...\right)$$
\end{document}
