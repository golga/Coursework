\documentclass[11pt]{article}

\usepackage{latexsym}
\usepackage{amssymb}
\usepackage{amsthm}
\usepackage{enumerate}
\usepackage{amsmath}
\usepackage{cancel}
\numberwithin{equation}{section}

\setlength{\evensidemargin}{.25in}
\setlength{\oddsidemargin}{-.25in}
\setlength{\topmargin}{-.75in}
\setlength{\textwidth}{6.5in}
\setlength{\textheight}{9.5in}
\newcommand{\due}{September 12th, 2015}
\newcommand{\HWnum}{1}
\newcommand{\grad}{\bold\nabla}
\newcommand{\vecE}{\vec{E}}
\newcommand{\scrptR}{\vec{\mathfrak{R}}}
\newcommand{\kapa}{\frac{1}{4\pi\epsilon_0}}
\newcommand{\emf}{\mathcal{E}}
\newcommand{\unit}[1]{\ensuremath{\, \mathrm{#1}}}
\newcommand{\real}{\textnormal{Re}}
\newcommand{\Erf}{\textnormal{Erf}}
\newcommand{\sech}{\textnormal{sech}}
\newcommand{\scrO}{\mathcal{O}}
\newcommand{\levi}{\widetilde{\epsilon}}
\newcommand{\partiald}[2]{\ensuremath{\frac{\partial{#1}}{\partial{#2}}}}
\newcommand{\norm}[2]{\langle{#1}|{#2}\rangle}
\newcommand{\inprod}[2]{\langle{#1}|{#2}\rangle}
\newcommand{\average}[1]{\left\langle{#1}\right\rangle}
\newcommand{\ket}[1]{|{#1}\rangle}
\newcommand{\bra}[1]{\langle{#1}|}
\newcommand{\Resid}[2]{\ensuremath{\textnormal{Res}\left[{#1},{#2}\right]}}





\begin{document}
\begin{titlepage}
\setlength{\topmargin}{1.5in}
\begin{center}
\Huge{Physics 3310} \\
\LARGE{Principles of Electricity and Magnetism 1} \\
\Large{Professor Thomas R. Schibli} \\[1cm]

\huge{Homework \#\HWnum}\\[0.5cm]

\large{Joe Becker} \\
\large{SID: 810-07-1484} \\
\large{\due} 

\end{center}

\end{titlepage}



\section{Problem \#1}
To evaluate the integral
$$I(k) = \int_{0}^{5}\sin{t}e^{-k(\sinh{t})^4}dt$$
to leading order in $k$ we can use an asymptotic approach by taking $t$ to be small. This 
implies that
$$I(k) \approx \int_{0}^{5}te^{-kt^4}dt$$
where we apply a change in variables as $t'=t^2$ which implies that
$$I(k) \approx \frac{1}{2}\int_{0}^{\sqrt{5}}e^{-kt'^2}dt'$$
Now we change variable again where $\tau=\sqrt{k}t'^2$ so 
\begin{align*}
d\tau &= 2\sqrt{k}t'dt'
&\Downarrow\\
dt' &= \frac{1}{2}\frac{1}{\sqrt{k\tau}}d\tau
\end{align*}
So our integral becomes
\begin{align*}
I(k) &\approx \int_{0}^{5}te^{-kt^4}dt\\
&\Downarrow\\
&\approx \lim_{k\rightarrow\infty}\frac{1}{4\sqrt{k}}\int_{0}^{f(k)}\tau^{-1/2}e^{-\tau}d\tau\\
&\approx \frac{1}{4\sqrt{k}}\int_{0}^{\infty}\tau^{-1/2}e^{-\tau}d\tau\\
&\approx \frac{1}{4\sqrt{k}}\Gamma(1/2)\\
&\approx \frac{\sqrt{\pi}}{4\sqrt{k}}
\end{align*}

\pagebreak

\section{Problem \#2}
For the higher order expansion of the Gamma function given as
$$\Gamma(x) \approx \sqrt{2\pi}x^{x-1/2}e^{-x}\left(1+\frac{A}{x}+\frac{B}{x^2}\right)$$
we can find $A$ and $B$ using the relation
$$\Gamma(x+1) = x\Gamma(x)$$
this implies that for large $x$ we have
\begin{align*}
\Gamma(x+1) &= x\Gamma(x)\\
\sqrt{2\pi}(x+1)^{x+1/2}e^{-(x+1)}\left(1+\frac{A}{x+1}+\frac{B}{(x+1)^2}\right) &= x\sqrt{2\pi}x^{x-1/2}e^{-x}\left(1+\frac{A}{x}+\frac{B}{x^2}\right)\\
\sqrt{2\pi}(x+1)^{x+1/2}e^{-(x+1)}\left(1+\frac{A}{x+1}+\frac{B}{(x+1)^2}\right) &= \sqrt{2\pi}x^{x+1/2}e^{-x}\left(1+\frac{A}{x}+\frac{B}{x^2}\right)\\
\Downarrow\\
\left(\frac{x+1}{x}\right)^{x+1/2}e^{-1}\left(1+\frac{A}{x+1}+\frac{B}{(x+1)^2}\right) &= 1+\frac{A}{x}+\frac{B}{x^2}\\
\left(1+\frac{1}{x}\right)^{x+1/2}e^{-1}\left(1+\frac{A}{x+1}+\frac{B}{(x+1)^2}\right) &= 1+\frac{A}{x}+\frac{B}{x^2}
\end{align*}
This allows us to expand 
$$\left(1+\frac{1}{x}\right)^{x+1/2} = e\left(1+\frac{1}{12x^2}-\frac{1}{12x^3}+\frac{113}{1440x^4}+\scrO{(x^{-5})}\right)$$
Which makes our equality become which allows us to solve for $A$ and $B$ by grouping terms of
equal order.
\begin{align*}
\left(1+\frac{1}{12x^2}-\frac{1}{12x^3}\right)\left(1+\frac{A}{x+1}+\frac{B}{(x+1)^2} \right) &= 1+\frac{A}{x}+\frac{B}{x^2} \\
&\Downarrow\\
\left(1+\frac{1}{12x^2}-\frac{1}{12x^3}\right)\left(1+\frac{A}{x}-\frac{A}{x^2}+\frac{B}{x^2} - \frac{2B}{x^3}\right) &= 1+\frac{A}{x}+\frac{B}{x^2}\\
1 + \frac{A}{x} + \frac{B-A+1/12}{x^2} + \frac{13/12A-2B-1/12}{x^3} &= 1+\frac{A}{x}+\frac{B}{x^2}\\
&\Downarrow\\
\frac{B-A+1/12}{x^2} + \frac{13/12A-2B-1/12}{x^3} &= \frac{B}{x^2}
\end{align*}
So we can say that
$$A = \frac{1}{12}$$
which leads to 
\begin{align*}
\frac{13}{144}-\frac{1}{12} -2B &= 0\\
2B = \frac{1}{144}\\ 
B = \frac{1}{288}
\end{align*}


\pagebreak

\section{Problem \#3}
We can find the leading term of the integral
$$I = \int_{0}^{\infty}dte^{kt-e^t}$$
by changing variables from $x = e^{t}/k$ which is small for large $k$. This implies that
$dx = e^{t}/kdt$ or $dt = x^{-1}dx$. This allows us to change variables to
\begin{align*}
I &= \int_{1/k}^{\infty}e^{k\log(kx)-kx}\frac{1}{x}dx\\
&= \int_{1/k}^{\infty}e^{k\log(k)+k\log(x)-kx}\frac{1}{x}dx\\
&= e^{k\log(k)}\int_{1/k}^{\infty}e^{-k(x-\log(x))}\frac{1}{x}dx
\end{align*}
We can take $\phi(x) = x-\log(x)$ which has a minimum at $x=1$ where $\phi''(x)>0$ therefore
we can take the asymptotic solution
$$I(k) \approx f(c)e^{-k\phi(c)}\sqrt{\frac{2\pi}{k\phi''(c)}}$$
which for $c=1$ and $f(c) = 1$ we have
$$I(k) \approx k^ke^{-k}\sqrt{\frac{2\pi}{k}}$$

\pagebreak

\section{Problem \#4}
To find the sum 
$$\sum_{n=-\infty}^{\infty}\frac{1}{(a+n)^2}$$
we can consider the complex integral
$$I = \oint_{C}\frac{\pi\cot(\pi{z})}{(a+z)^2}$$
Note that $a$ is not an integer. We note that $\pi\cot(\pi{z})$ has a simple pole at integer 
values of $z$. They have a residue of value one. So if we take a contour that covers the 
positive half of the complex plane. This allows us to solve the integral as
$$I = 2\pi{i}\sum_{n=-N}^{N}\frac{1}{(z+a)^2} + 2\pi{i}\sum[\Resid{\pi\cot(\pi{z})}{a}]$$
This implies that 
$$\sum_{n=-\infty}^{\infty}\frac{1}{(a+n)^2} = -\Resid{\pi\cot(\pi{z})}{a} = -\pi^2\csc^2(\pi{a})$$


\end{document}

