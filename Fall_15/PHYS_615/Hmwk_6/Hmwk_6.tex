\documentclass[11pt]{article}

\usepackage{latexsym}
\usepackage{amssymb}
\usepackage{amsthm}
\usepackage{enumerate}
\usepackage{amsmath}
\usepackage{cancel}
\numberwithin{equation}{section}

\setlength{\evensidemargin}{.25in}
\setlength{\oddsidemargin}{-.25in}
\setlength{\topmargin}{-.75in}
\setlength{\textwidth}{6.5in}
\setlength{\textheight}{9.5in}
\newcommand{\due}{October 14th, 2015}
\newcommand{\HWnum}{6}
\newcommand{\grad}{\bold\nabla}
\newcommand{\vecE}{\vec{E}}
\newcommand{\scrptR}{\vec{\mathfrak{R}}}
\newcommand{\kapa}{\frac{1}{4\pi\epsilon_0}}
\newcommand{\emf}{\mathcal{E}}
\newcommand{\unit}[1]{\ensuremath{\, \mathrm{#1}}}
\newcommand{\real}{\textnormal{Re}}
\newcommand{\Erf}{\textnormal{Erf}}
\newcommand{\sech}{\textnormal{sech}}
\newcommand{\scrO}{\mathcal{O}}
\newcommand{\levi}{\widetilde{\epsilon}}
\newcommand{\partiald}[2]{\ensuremath{\frac{\partial{#1}}{\partial{#2}}}}
\newcommand{\norm}[2]{\langle{#1}|{#2}\rangle}
\newcommand{\inprod}[2]{\langle{#1}|{#2}\rangle}
\newcommand{\ket}[1]{|{#1}\rangle}
\newcommand{\bra}[1]{\langle{#1}|}





\begin{document}
\begin{titlepage}
\setlength{\topmargin}{1.5in}
\begin{center}
\Huge{Physics 3320} \\
\LARGE{Principles of Electricity and Magnetism II} \\
\Large{Professor Ana Maria Rey} \\[1cm]

\huge{Homework \#\HWnum}\\[0.5cm]

\large{Joe Becker} \\
\large{SID: 810-07-1484} \\
\large{\due} 

\end{center}

\end{titlepage}



\section{Problem \#1}
For the function 
$$f(z) = \frac{e^z}{z(z^2-16)}$$
we can see that $f(z)$ is analytic everywhere except $z=0$ and $|z|=4$. So we can calculate 
the integral 
$$I = \oint_{C}\frac{e^z}{z(z^2-16)}$$
where the contour $C$ is defined as the boundary of the annulus between the circles $|z|=1$
and $|z|=3$. We note that $f(z)$ is analytic everywhere along $C$ which implies that 
$$I = \oint_{C}\frac{e^z}{z(z^2-16)}=0$$
due to the fact that $C$ is a closed contour.

\section{Problem \#2}
\begin{enumerate}[1)]
\item For the integral
$$\oint_{C}e^{x^2}\left(\frac{1}{z^2}-\frac{1}{z^3}\right)$$
where $C$ is the unit circle centered at the origin. We first expand $e^{z^2}$ by
$$e^{z^2} = 1 + z^2 + \frac{z^4}{2} +...$$
Which makes our integral become
\begin{align*}
\oint_{C}e^{x^2}\left(\frac{1}{z^2}-\frac{1}{z^3}\right) &= \oint_{C}(1 + z^2 + \frac{z^4}{2} +...)\left(\frac{1}{z^2}-\frac{1}{z^3}\right)\\
&= \oint_{C}\frac{1}{z^2} - \frac{1}{z^3} + 1 - \frac{1}{z} + ...
\end{align*}
We note that the only integral in this sum that is nonzero is the $z^{-1}$. We can use the
fact that $z$ on the unit circle is given by $z=e^{i\phi}$ which allows us to integrate over
the circle by
\begin{align*}
-\oint_{C}\frac{1}{z}dz &= -\oint_{C}\frac{1}{e^{i\phi}}d(e^{i\phi}) \\
&= -\oint_{0}^{2\pi}\frac{1}{e^{i\phi}}ie^{i\phi}d\phi \\
&= -\oint_{0}^{2\pi}id\phi \\
&= -2\pi{i}
\end{align*}
So the solution to the integral is
$$\oint_{C}e^{x^2}\left(\frac{1}{z^2}-\frac{1}{z^3}\right) = -2\pi i$$

\item For the integral 
$$\oint_{C} \frac{(\sin{z})^6}{(z-\frac{\pi}{6})^3}$$ 
we note the derivative formula
$$f^{(n)}(z) = \frac{n!}{2\pi{i}}\oint_{C}\frac{f(w)}{(w-z)^{n+1}}$$
where we can rearrange
\begin{align*}
f^{(n)}(z) &= \frac{n!}{2\pi{i}}\oint_{C}\frac{f(w)}{(w-z)^{n+1}}\\
&\Downarrow\\
\oint_{C}\frac{(\sin{z})^6}{(z-\pi/6)^{3}} &= \frac{n!}{2\pi{i}}f^{(2)}(z)\\
&= \left.\frac{2\pi{i}}{2!}\frac{d^2}{dz^2}(\sin{z})^6\right|_{\pi/6}\\
&= \left.\frac{2\pi{i}}{2}\frac{d}{dz}6(\sin{z})^5\cos{z}\right|_{\pi/6}\\
&= i\pi\left(\frac{}{}30(\sin{z})^4(\cos{z})^2-6(\sin{z})^6\right|_{\pi/6}\\
&= i\pi\left(30(\sin{\pi/6})^4(\cos{\pi/6})^2-6(\sin{\pi/6})^6\right)\\
&= i\pi\left(30\frac{1}{2^4}\frac{3}{4}-6\frac{1}{2^6}\right)\\
&= i\pi\left(\frac{90}{64}-\frac{6}{64}\right)\\
&= i\pi\left(\frac{84}{64}\right)\\
&= \frac{21}{16}\pi{i}
\end{align*}

\end{enumerate}

\pagebreak

\section{Problem \#3}
Given \emph{Rodrigues' formula} 
\begin{equation}
P_n(t) = \frac{1}{2^nn!}\left(\frac{d}{dx}\right)^n[(x^2-1)^n]
\label{RodForm}
\end{equation}
we can write the derivative as a complex integral by the formula
$$f^{(n)}(z) = \frac{n!}{2\pi{i}}\oint_{C}\frac{f(w)}{(w-z)^{n+1}}$$
where we note that
$$f^{(n)}(z) = \left(\frac{d}{dz}\right)^n[(z^2-1)^n]$$
which implies that $f(z) = (z^2-1)^n$. We can take the nth derivative of $f(z)$ by the 
integral
$$f^{(n)}(z) = \frac{n!}{2\pi{i}}\oint_{C}\frac{(z^2-1)^n}{(z-t)^{n+1}}$$
And by replacing into equation \ref{RodForm} we get
\begin{align*}
P_n(t) &= \frac{1}{2^nn!}\left(\frac{d}{dx}\right)^n[(x^2-1)^n]\\
&\Downarrow\\
P_n(t) &= \frac{1}{2^nn!}\frac{n!}{2\pi{i}}\oint_{C}\frac{(z^2-1)^n}{(z-t)^{n+1}}\\
&= \frac{1}{{2\pi{i}}\frac{1}{2^n}}\oint_{C}\frac{(z^2-1)^n}{(z-t)^{n+1}}
\end{align*}

\section{Problem \#4}
For the function 
$$f(z) = \frac{1}{e^z-1}$$
we can calculate the first three non-zero terms in the \emph{Laurent expansion} where we 
expand about the point $z=0$. We expand 
$$e^{z} = 1+z+z^2+z^3+...$$
which makes $f(z)$ become
$$f(z) = \frac{1}{z+z^2/2+...}=\frac{1}{z}\frac{1}{1+z/2+...}$$
Now we can expand 
$$\frac{1}{1+z/2} = 1 - \frac{z}{2} + \frac{z^2}{4} - \frac{z^3}{8}$$
so the first four terms of the Laurent expansion are
$$f(z) = \frac{1}{e^z-1} = \frac{1}{z}\frac{1}{1+z/2} = z^{-1} - \frac{1}{2}z + \frac{1}{4}z^2 - \frac{1}{8}z^3$$
\end{document}

