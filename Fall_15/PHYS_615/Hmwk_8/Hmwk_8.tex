\documentclass[11pt]{article}

\usepackage{latexsym}
\usepackage{amssymb}
\usepackage{amsthm}
\usepackage{enumerate}
\usepackage{amsmath}
\usepackage{cancel}
\numberwithin{equation}{section}

\setlength{\evensidemargin}{.25in}
\setlength{\oddsidemargin}{-.25in}
\setlength{\topmargin}{-.75in}
\setlength{\textwidth}{6.5in}
\setlength{\textheight}{9.5in}
\newcommand{\due}{October 28th, 2015}
\newcommand{\HWnum}{8}
\newcommand{\grad}{\bold\nabla}
\newcommand{\vecE}{\vec{E}}
\newcommand{\scrptR}{\vec{\mathfrak{R}}}
\newcommand{\kapa}{\frac{1}{4\pi\epsilon_0}}
\newcommand{\emf}{\mathcal{E}}
\newcommand{\unit}[1]{\ensuremath{\, \mathrm{#1}}}
\newcommand{\real}{\textnormal{Re}}
\newcommand{\Erf}{\textnormal{Erf}}
\newcommand{\sech}{\textnormal{sech}}
\newcommand{\scrO}{\mathcal{O}}
\newcommand{\levi}{\widetilde{\epsilon}}
\newcommand{\partiald}[2]{\ensuremath{\frac{\partial{#1}}{\partial{#2}}}}
\newcommand{\norm}[2]{\langle{#1}|{#2}\rangle}
\newcommand{\inprod}[2]{\langle{#1}|{#2}\rangle}
\newcommand{\ket}[1]{|{#1}\rangle}
\newcommand{\bra}[1]{\langle{#1}|}





\begin{document}
\begin{titlepage}
\setlength{\topmargin}{1.5in}
\begin{center}
\Huge{Physics 3320} \\
\LARGE{Principles of Electricity and Magnetism II} \\
\Large{Professor Ana Maria Rey} \\[1cm]

\huge{Homework \#\HWnum}\\[0.5cm]

\large{Joe Becker} \\
\large{SID: 810-07-1484} \\
\large{\due} 

\end{center}

\end{titlepage}



\section{Problem \#1}
To evaluate the integral
$$I = \int_{-\infty}^{\infty}\frac{x^2}{x^4+1}dx$$
We can complexify the integral and integrate over a contour from $-R$ to $R$ along the real
axis and then over a semicircle of radius $R$. We then take $R$ to infinity. We note the 
integral along the semicircle $z$ becomes $Re^{i\phi}$ and we integrate from $0$ to $\pi$ so
as we take the limit as $R\rightarrow\infty$ we see
\begin{align*}
\lim_{R\rightarrow\infty}\int_{C_R}\frac{z^2}{z^4+1}dz &= \lim_{R\rightarrow\infty}\int_{0}^{\pi}d\phi\frac{R^2e^{i2\phi}}{R^4e^{i4\phi}+1} = 0
\end{align*}
Due to the fact that our integrand goes by $1/R^2$. So we can say that the integral over the
reals is also
$$I = \oint_{C_{\infty}}\frac{z^2}{z^4+1}dz = \oint_{C_{\infty}}\frac{z^2}{(z-e^{i\pi/4})(z-e^{i3\pi/4})(z-e^{i5\pi/4})(z-e^{i7\pi/4})}dz$$
where the contour is over the positive complex plane. We can solve this integral by noting 
that there are two residues within this contour $z_0=e^{i\pi/4}$ and $z_0=e^{i3\pi/4}$ which are
both simple poles. This means we can calculate the residues by noting that we are in the form
of 
$$f(z) = \frac{g(z)}{h(z)}$$
\begin{align*}
\Resid{f(z)}{e^{i\pi/4}} &= \frac{g(e^{i\pi/4})}{h'(e^{i\pi/4})}\\
&= \frac{(e^{i\pi/4})^2}{4(e^{i\pi/4})^3}\\
&= \frac{1}{4}e^{-i\pi/4} 
\end{align*}
and
\begin{align*}
\Resid{f(z)}{e^{i3\pi/4}} &= \frac{g(e^{i3\pi/4})}{h'(e^{i3\pi/4})}\\
&= \frac{(e^{i3\pi/4})^2}{4(e^{i3\pi/4})^3}\\
&= \frac{1}{4}e^{-i3\pi/4}
\end{align*}
So the integral is the sum of the residues 
\begin{align*}
I &= 2\pi{i}\frac{1}{4}\left(e^{-i\pi/4}+e^{-i3\pi/4}\right)\\
&= \pi{i}\frac{1}{2}\left(\cos(\pi/4)-i\sin(\pi/4) + \cos(3\pi/4) - i\sin(3\pi/4)\right)\\
&= \pi{i}\frac{1}{2}\left(-i\sqrt{2}\right)\\
&= \frac{\sqrt{2}}{2}\pi
\end{align*}

\pagebreak

\section{Problem \#2}
For the integral 
$$I = \int_{-\infty}^{\infty}\frac{1}{x^6+1}dx$$ 
we see that we can use the same contour as before and know that the semicircle contribution
is zero again. So we note that we have poles in the contour of $z_0=e^{i\pi/6}$, 
$z_0=e^{i3\pi/6}=i$, and $z_0=e^{i5\pi/6}$. So like before we can calculate
\begin{align*}
\Resid{f(z)}{e^{i\pi/6}} &= \frac{1}{h'(e^{i\pi/6})}\\
&= \frac{1}{6(e^{i\pi/6})^5}\\
&= \frac{1}{6}e^{-i5\pi/6}
\end{align*}
and for $z_0=e^{i5\pi/6}$
\begin{align*}
\Resid{f(z)}{e^{i5\pi/6}} &= \frac{1}{h'(e^{i\pi/6})}\\
&= \frac{1}{6(e^{i5\pi/6})^5}\\
&= \frac{1}{6}e^{-i25\pi/6}
\end{align*}
and finally for $z_0=i$ we have
\begin{align*}
\Resid{f(z)}{i} &= \frac{1}{h'(i)}\\
&= \frac{1}{6(i)^5}\\
&= -\frac{1}{6}i
\end{align*}
So we can solve for $I$ by summing the residues 
\begin{align*}
I &= 2\pi{i}\frac{1}{6}\left(e^{-i5\pi/6}+e^{-i25\pi/6}-i\right)\\
&= \frac{\pi{i}}{3}\left(-\frac{\sqrt{3}}{2} - i\frac{1}{2} + \frac{\sqrt{3}}{2} -i\frac{1}{2} - i\right) \\
&= \frac{2\pi}{3}
\end{align*}

\pagebreak

\section{Problem \#3}
For the given integral 
$$I = \int_{-\infty}^{\infty}\frac{x\sin\pi{x}}{x^2+2x+5}dx$$
we can complexify the integral by noting that
$$\sin(\pi{z}) = \Im[e^{i\pi{z}}]$$
we note that along semicircle contour we used in problems one and two the integrand only goes
by $1/R$. Therefore we need to see that the integrand is of the form $g(z)e^{iaz}$ which
allows us to apply \emph{Jordan's lemma} to say that the semicircle contribution goes to zero
as the radius is taken to infinity. Therefore 
$$I = \Im\left[\int_{C}\frac{z(e^{i\pi{z}})}{z^2+2z+5}dz\right]$$
which implies that we can calculate $I$ by residue theorem where we only need the residue of
the simple pole at $z_0=-1+2i$ which allows us to calculate
\begin{align*}
\Resid{f(z)}{-1+2i} &= \lim_{z\rightarrow{-1+2i}}(z-(-1+2i))\frac{z(e^{i\pi{z}})}{z^2+2z+5}\\
&= \lim_{z\rightarrow{-1+2i}}\frac{z(e^{i\pi{z}}}{(z+(1+2i))}\\
&= \frac{(-1+2i)(e^{i\pi{(-1+2i)}})}{-1+2i+1+2i}\\
&= \frac{(-1+2i)(-e^{-2\pi})}{4i}\\
&= \frac{-2-i}{4}e^{-2\pi}
\end{align*}
So we can calculate $I$ by
\begin{align*}
I &= \Im[2\pi{i}\frac{-2-i}{4}e^{-2\pi}]\\
 &= 2\pi e^{-2\pi}\Im[\frac{-2i+1}{4}]\\
 &= \pi e^{-2\pi}
\end{align*}

\pagebreak

\section{Problem \#4}
We can evaluate the integral 
$$I = \int_{0}^{\infty}\frac{\log{x}}{(x+a)(x+b)}dx,\qquad a,b>0,a\ne{b}$$
by considering the integral of the complex function
$$f(z) = \frac{(\log{z})^2}{(z+a)(z+b)}$$
over a keyhole contour, $C$, around the branch cut of $\log{z}$. Where we define the branch cut of
the $\log(z)$ function to be the positive real axis. We note that as we take the circle of 
radius $\epsilon$ to zero the integral becomes zero. The same holds true when we take $R$ to
infinity
\begin{align*}
\oint_{C}f(z)dz &= \lim_{\epsilon\rightarrow{0}}\lim_{R\rightarrow\infty}\cancelto{0}{\oint_{C_{R}}f(z)dz} + \cancelto{0}{\oint_{C_{\epsilon}}f(z)dz} \int_{\epsilon}^{R}f(z)dz + \int^{\epsilon}_{R}f(ze^{2\pi{i}})dz\\
&= \int_{0}^{\infty}f(z)dz + \int_{\infty}^{0}f(ze^{2\pi{i}})dz
\end{align*}
We note that due to the branch cut this integral does not equal zero. So we can see that
\begin{align*}
\oint_{C}f(z)dz = \int_{0}^{\infty}f(z)dz + \int_{\infty}^{0}f(ze^{2\pi{i}})dz &= \int_{0}^{\infty}\frac{(\log{x})^2}{(x+a)(x+b)}dx + \int_{\infty}^{0}\frac{(\log{xe^{2\pi{i}}})^2}{(x+a)(x+b)}dx\\
&= \int_{0}^{\infty}\frac{(\log{x})^2}{(x+a)(x+b)}dx - \int^{\infty}_{0}\frac{(\log{x}+2\pi{i})^2}{(x+a)(x+b)}dx\\
&= \int_{0}^{\infty}\frac{(\log{x})^2}{(x+a)(x+b)}dx - \int^{\infty}_{0}\frac{(\log{x})^2-4\pi+4\pi{i}\log{x}}{(x+a)(x+b)}dx\\
&= 4\pi\int_{0}^{\infty}\frac{1}{(x+a)(x+b)}dx - 4\pi{i}\int^{\infty}_{0}\frac{\log{x}}{(x+a)(x+b)}dx\\
&\Downarrow\\
\int^{\infty}_{0}\frac{\log{x}}{(x+a)(x+b)}dx &= -i\int_{0}^{\infty}\frac{1}{(x+a)(x+b)}dx - \frac{1}{4\pi{i}}\oint_{C}f(z)dz
\end{align*}
We note that the integral solution to the first integral is
$$\int_{0}^{\infty}\frac{1}{(x+a)(x+b)}dx = \frac{\log{(a/b)}}{b-a}$$
which implies that we just need to find the solution to the complex integral which we do using
residue theorem. Where we can say that
\begin{align*}
\Resid{f(z)}{-a} &= \frac{(\log{-a})^2}{b-a} = -\frac{(\log{a})^2-\pi+2\pi i\log{a}}{a-b}\\
\Resid{f(z)}{-b} &= \frac{(\log{-b})^2}{a-b} = \frac{(\log{b})^2-\pi+2\pi{i}\log{b}}{a-b}
\end{align*}
So we have the integral as
\begin{align*}
I &= -i\frac{\log{(a/b)}}{b-a} - \frac{2\pi{i}}{4\pi{i}}\left(\frac{(\log{b})^2-\pi+2\pi{i}\log{b}}{a-b} - \frac{(\log{a})^2-\pi+2\pi i\log{a}}{a-b}\right)\\
&= i\frac{\log{(a/b)}}{a-b} - \frac{(\log{b})^2-(\log{a})^2}{2(a-b)} - {i}\frac{\log{b}-\log{a}}{a-b}\\
&= \frac{(\log{a})^2-(\log{b})^2}{2(a-b)} 
\end{align*}



\end{document}

