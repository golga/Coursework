\documentclass[11pt]{article}

\usepackage{latexsym}
\usepackage{amssymb}
\usepackage{amsthm}
\usepackage{enumerate}
\usepackage{amsmath}
\usepackage{cancel}
\numberwithin{equation}{section}

\setlength{\evensidemargin}{.25in}
\setlength{\oddsidemargin}{-.25in}
\setlength{\topmargin}{-.75in}
\setlength{\textwidth}{6.5in}
\setlength{\textheight}{9.5in}
\newcommand{\due}{November 11th, 2015}
\newcommand{\HWnum}{10}
\newcommand{\grad}{\bold\nabla}
\newcommand{\vecE}{\vec{E}}
\newcommand{\scrptR}{\vec{\mathfrak{R}}}
\newcommand{\kapa}{\frac{1}{4\pi\epsilon_0}}
\newcommand{\emf}{\mathcal{E}}
\newcommand{\unit}[1]{\ensuremath{\, \mathrm{#1}}}
\newcommand{\real}{\textnormal{Re}}
\newcommand{\Erf}{\textnormal{Erf}}
\newcommand{\sech}{\textnormal{sech}}
\newcommand{\scrO}{\mathcal{O}}
\newcommand{\levi}{\widetilde{\epsilon}}
\newcommand{\partiald}[2]{\ensuremath{\frac{\partial{#1}}{\partial{#2}}}}
\newcommand{\norm}[2]{\langle{#1}|{#2}\rangle}
\newcommand{\inprod}[2]{\langle{#1}|{#2}\rangle}
\newcommand{\average}[1]{\left\langle{#1}\right\rangle}
\newcommand{\ket}[1]{|{#1}\rangle}
\newcommand{\bra}[1]{\langle{#1}|}
\newcommand{\Resid}[2]{\ensuremath{\textnormal{Res}\left[{#1},{#2}\right]}}





\begin{document}
\begin{titlepage}
\setlength{\topmargin}{1.5in}
\begin{center}
\Huge{Physics 3310} \\
\LARGE{Principles of Electricity and Magnetism 1} \\
\Large{Professor Thomas R. Schibli} \\[1cm]

\huge{Homework \#\HWnum}\\[0.5cm]

\large{Joe Becker} \\
\large{SID: 810-07-1484} \\
\large{\due} 

\end{center}

\end{titlepage}



\section{Problem \#1}
For the integral
$$I(x) = \int_{0}^{\infty}e^{xt-e^{2t}}dt$$
we can find the leading order behavior for $x\rightarrow\infty$. To do this we change 
variables by $u = e^{2t}/x$ which implies that 
$$du = 2\frac{e^{2t}}{x}dt = 2udt$$
So we can change variables to
\begin{align*}
I(x) &= \int_{0}^{\infty}e^{xt-e^{2t}}dt\\
&\Downarrow\\
I(x) &= \int_{1/x}^{\infty}\exp\left(\frac{1}{2}x\log(xu)-xu\right)\frac{2}{u}du\\
&= e^{x/2\log(x)}\int_{1/x}^{\infty}\exp\left(\frac{1}{2}x\log(u)-xu\right)\frac{2}{u}du\\
&= x^{x/2}\int_{1/x}^{\infty}\exp\left(-x(u-\log(u^{1/2})\right)\frac{2}{u}du
\end{align*}
We note that this is a Laplace type integral where $\phi(u) = u - \log(u^{1/2})$. We note 
that $\phi(u)$ has an extrema, $c$, which we find at
\begin{align*}
\phi'(u) = 0 &= 1 - \frac{1}{2u}\\
&\Downarrow\\
c &= \frac{1}{2}
\end{align*}
We can test to see that the extrema at $c$ is a minimum by noting 
\begin{align*}
\phi''(u) = \frac{1}{2u^2} \qquad \Rightarrow \qquad \phi''(c) = 2
\end{align*}
So we use that asymptotic solution given by
$$I(x) \approx f(c)e^{-x\phi(c)}\sqrt{\frac{2\pi}{x\phi''(c)}}$$
where $f(c) = 4x^{x/2}$ and $\phi(c) = 1/2(1-\log(2))$ so
$$I(x) \approx 4x^{x/2}e^{-x/2(1-\log(2))}\sqrt{\frac{\pi}{x}} = 4\sqrt{\frac{\pi}{x}}\left(\frac{2x}{e}\right)^{x/2}$$

\pagebreak

\section{Problem \#2}
We can find the leading behavior of the integral
$$I(x) = \int_{0}^{\infty}\cos\left(xt^2-t\right)dt$$
by complexifying this integral and writing it in the form $e^{x\phi(z)}$ by
$$I(x) = \frac{1}{2}\int_{0}^{\infty}\left(e^{x(iz^2-iz/x)}+e^{-x(iz^2-iz/x)}\right)dz$$
which allows us to approximate this integral by the method of steepest descent which yields
$$I(x) \approx f(z_0)e^{i\theta}e^{x\phi(z_0)}\sqrt{\frac{2\pi}{xa}}$$
Where for the first term we have
$$\phi(z) = iz^2-\frac{i}{x}z$$
which has a critical point
$$\phi'(z) = 0 = 2iz - \frac{i}{x} \qquad\Rightarrow\qquad z_0 = \frac{1}{2x}$$
Using this critical point we can calculate $\alpha$ setting
$$\phi''(z_0) = ae^{i\alpha}$$
so we have $\phi''(z) = 2i$ which implies that $a=2$ and $\alpha=\pi/2$. Using $\alpha$ we 
can determine $\theta$ by 
$$\theta = -\frac{\alpha}{2}\pm\frac{\pi}{2}$$
so $\theta = \pi/4,-3\pi/4$ where we can pick either solution as still be able to deform the
contour back to the real axis. So this first integral has the solution
\begin{align*}
I_1(x) &\approx e^{i\pi/4}e^{x(i/4x^2-i/2x^2)}\sqrt{\frac{2\pi}{2x}}\\
&\approx e^{i\pi/4}e^{-i/2x}\sqrt{\frac{\pi}{x}}\\
\end{align*}
Now we can solve the negative integral by noting that
$$\phi(z) = -iz^2+\frac{i}{x}z$$
which has a critical point
$$\phi'(z) = 0 = -2iz + \frac{i}{x} \qquad\Rightarrow\qquad z_0 = \frac{1}{2x}$$
Now we can find $\alpha$ by seeing $\phi''(z) = -2i$ which implies that $a=2$ and $\alpha=-\pi/2$.
So we can find $\theta = -\pi/4,3\pi/2$ So we have an approximate solution by picking 
$\theta=-\pi/4$
\begin{align*}
I_2(x) &\approx e^{-i\pi/4}e^{x(-i/4x^2+i/2x^2)}\sqrt{\frac{2\pi}{2x}}\\
&\approx e^{-i\pi/4}e^{+i/2x}\sqrt{\frac{\pi}{x}}
\end{align*}
So we can find the total integral by
\begin{align*}
I(x) &\approx \frac{1}{2}\left(e^{i\pi/4}e^{-i/2x}\sqrt{\frac{\pi}{x}} + e^{-i\pi/4}e^{+i/2x}\sqrt{\frac{\pi}{x}}\right)\\
&\approx \frac{\sqrt{\pi}}{2\sqrt{x}}\left(e^{-i(1/2x-\pi/4)} + e^{i(1/2x-\pi/4)}\right)\\
&\approx \sqrt{\frac{\pi}{x}}\cos\left(\frac{1}{2x}-\frac{\pi}{4}\right)
\end{align*}



\section{Problem \#3}
We can take the Fourier transform, $F(\omega)$, of the Gaussian function given by
$$f(t) = e^{-a^2t^2},\qquad a\in\mathbb{R}$$
by calculating the integral
\begin{align*}
F(\omega) &= \frac{1}{2\pi}\int_{-\infty}^{\infty}f(t)e^{i\omega{t}}\\
&= \frac{1}{\sqrt{2\pi}}\int_{-\infty}^{\infty}e^{-a^2t^2}e^{i\omega{t}}\\
&= \frac{1}{\sqrt{2\pi}}\int_{-\infty}^{\infty}e^{-a^2t^2+i\omega{t}}\\
&= \frac{1}{\sqrt{2\pi}}\int_{-\infty}^{\infty}e^{-a^2t^2 + i\omega{t} + \omega^2/4a^2 - \omega^2/4a^2}\\
&= \frac{1}{\sqrt{2\pi}}e^{-\omega^2/4a^2}\int_{-\infty}^{\infty}e^{-(at + i\omega/2a)^2} \\
&= \frac{1}{\sqrt{2\pi}}e^{-\omega^2/4a^2}\frac{\sqrt{\pi}}{a}\\
&= \frac{1}{\sqrt{2}a}e^{-\omega^2/(2a)^2}
\end{align*}

\end{document}

