\documentclass[11pt]{article}

\usepackage{latexsym}
\usepackage{amssymb}
\usepackage{amsthm}
\usepackage{enumerate}
\usepackage{amsmath}
\usepackage{cancel}
\numberwithin{equation}{section}

\setlength{\evensidemargin}{.25in}
\setlength{\oddsidemargin}{-.25in}
\setlength{\topmargin}{-.75in}
\setlength{\textwidth}{6.5in}
\setlength{\textheight}{9.5in}
\newcommand{\due}{October 21st, 2015}
\newcommand{\HWnum}{7}
\newcommand{\grad}{\bold\nabla}
\newcommand{\vecE}{\vec{E}}
\newcommand{\scrptR}{\vec{\mathfrak{R}}}
\newcommand{\kapa}{\frac{1}{4\pi\epsilon_0}}
\newcommand{\emf}{\mathcal{E}}
\newcommand{\unit}[1]{\ensuremath{\, \mathrm{#1}}}
\newcommand{\real}{\textnormal{Re}}
\newcommand{\Erf}{\textnormal{Erf}}
\newcommand{\sech}{\textnormal{sech}}
\newcommand{\scrO}{\mathcal{O}}
\newcommand{\levi}{\widetilde{\epsilon}}
\newcommand{\partiald}[2]{\ensuremath{\frac{\partial{#1}}{\partial{#2}}}}
\newcommand{\norm}[2]{\langle{#1}|{#2}\rangle}
\newcommand{\inprod}[2]{\langle{#1}|{#2}\rangle}
\newcommand{\ket}[1]{|{#1}\rangle}
\newcommand{\bra}[1]{\langle{#1}|}





\begin{document}
\begin{titlepage}
\setlength{\topmargin}{1.5in}
\begin{center}
\Huge{Physics 3320} \\
\LARGE{Principles of Electricity and Magnetism II} \\
\Large{Professor Ana Maria Rey} \\[1cm]

\huge{Homework \#\HWnum}\\[0.5cm]

\large{Joe Becker} \\
\large{SID: 810-07-1484} \\
\large{\due} 

\end{center}

\end{titlepage}



\section{Problem \#1}
For the integral 
$$I = \int_{0}^{2\pi}\frac{\cos3\theta}{5-4\cos\theta}d\theta$$
we can calculate $I$ by changing to the variable $z=e^{i\theta}$ and integrating over the 
unit circle in the complex plane. We note that for $z$ we have
\begin{align*}
dz &= ie^{i\theta}d\theta\\
\cos{\theta} &= \frac{1}{2}\left(z+\frac{1}{z}\right)\\
\cos{3\theta} &= \frac{1}{2}\left(z^3+\frac{1}{z^3}\right)
\end{align*}
So our integral, $I$, becomes
\begin{align*}
I = \int_{0}^{2\pi}\frac{\cos3\theta}{5-4\cos\theta}d\theta &= -i\oint\frac{dz}{z}\frac{\frac{1}{2}\left(z^3+\frac{1}{z^3}\right)}{5-2\left(z+\frac{1}{z}\right)}\\
&= -\frac{i}{2}\oint\frac{dz}{z^4}\frac{z^6+1}{5-2\left(z+\frac{1}{z}\right)}\\
&= -\frac{i}{2}\oint dz\frac{z^6+1}{5z^4-2z^5-2z^3}\\
&= -\frac{i}{2}\oint dz\frac{z^6+1}{-z^3(2z^2-5z+2)}\\
&= -\frac{i}{2}\oint dz\frac{z^6+1}{-z^3(z-2)(2z-1)}
\end{align*}
So, we have three poles one at $z=0$ of order three and two of first order at $z=2$ and 
$z=1/2$.  We note that the pole located at $z=2$ is outside the unit circle so we do not 
need to calculate the residue at $z=2$. So the integral becomes
$$I = -\frac{i}{2}\oint dz\frac{z^6+1}{-z^3(z-2)(2z-1)} = -\frac{i}{2}2\pi{i}\left(\frac{}{}\Resid{f(z)}{0}+\Resid{f(z)}{1/2}\right)$$
Where $\Resid{f(z)}{z_0}$ is the Residue at the pole, $z_0$, and $f(z)$ is the integrand
which is 
$$f(z) = \frac{z^6+1}{-z^3(z-2)(2z-1)}$$
for this problem. We can calculate the residue of a pole, $z_0$, of order $N$ by the formula
\begin{equation}
\Resid{f(z)}{z_0} = \frac{1}{(N-1)!}\lim_{z\rightarrow{z_0}}\frac{d^{N-1}}{dz^{N-1}}\left(\frac{}{}(z-z_0)^Nf(z)\right)
\label{ResOrdN}
\end{equation}
So for $z_0=1/2$ we have a pole of $N=1$ which makes equation \ref{ResOrdN} become
\begin{align*}
\Resid{f(z)}{1/2} &= \lim_{z\rightarrow{1/2}}(z-1/2)\frac{z^6+1}{-z^3(z-2)(2z-1)}\\
&= \lim_{z\rightarrow{1/2}}\frac{1}{2}\cancel{(2z-1)}\frac{z^6+1}{-z^3(z-2)\cancel{(2z-1)}}\\
&= \lim_{z\rightarrow{1/2}}\frac{1}{2}\frac{z^6+1}{-z^3(z-2)}\\
&= \frac{1}{2}\frac{\frac{1}{64}+1}{-\frac{1}{8}\left(\frac{1}{2}-2\right)}= \frac{65}{64}\frac{16}{6} = \frac{65}{24}
\end{align*}
For the pole at $z=0$ we can use equation \ref{ResOrdN} again but we will need to have $N=3$
\begin{align*}
\Resid{f(z)}{0} &= \frac{1}{2}\lim_{z\rightarrow{0}}\frac{d^{2}}{dz^{2}}\left(z^3\frac{z^6+1}{-z^3(z-2)(2z-1)}\right)\\
&= -\frac{1}{2}\lim_{z\rightarrow{0}}\frac{d}{dz}\left(\frac{8z^7-25z^6+12z^5-4z+5}{(z-2)^2(2z-1)^2}\right)\\
&= -\frac{1}{2}\lim_{z\rightarrow{0}}\frac{48z^8-300z^7+636z^6-480z^5+120z^4+24z^2+42}{(z-2)^3(2z-1)^3}\\
&= -\frac{1}{2}\frac{42}{(-2)^3(-1)^3}\\
&= -\frac{1}{2}\frac{42}{8} = -\frac{21}{8}
\end{align*}
So our integral can be calculated as
\begin{align*}
I = -\frac{i}{2}\oint dz\frac{z^6+1}{-z^3(z-2)(2z-1)} &= -\frac{i}{2}2\pi{i}\left(\frac{}{}\Resid{f(z)}{0}+\Resid{f(z)}{1/2}\right)\\
&= \pi\left(-\frac{21}{8}+\frac{65}{24}\right)\\
&= \pi\frac{65-63}{24}\\
&= \frac{\pi}{12}
\end{align*}

\pagebreak

\section{Problem \#2}
For the integral given by
$$\frac{1}{2\pi{i}}\oint_{C} \frac{e^{zt}}{z^2(z^2+2z+2)}dz.$$
Where the contour, $C$, is a circle of radius $3$ centered at the origin of the complex plane.
Note that for any value of $t$ we have $e^{zt}$ as analytic in the and on the contour $C$.
By the fundamental theorem of algebra we are guaranteed the existence of two poles of order 
one from the quadratic term in the denominator. We note that these poles are given by 
$$z^2+2z+2 = (z+(1-i))(z+(1+i))$$
This implies that we have a pole of order one at $z=-1+i$ and $z=-1-i$, as well as a pole of
order two at $z=0$. We can calculate the residues at each of these poles by equation 
\ref{ResOrdN}. For $z=0$ we have
\begin{align*}
\Resid{f(z)}{0} &= \frac{1}{(1)!}\lim_{z\rightarrow{0}}\frac{d}{dz}\left(\frac{}{}(z)^2\frac{e^{zt}}{z^2(z^2+2z+2)}\right)\\
&= \lim_{z\rightarrow{0}}\frac{d}{dz}\left(\frac{e^{zt}}{(z^2+2z+2)}\right)\\
&= \lim_{z\rightarrow{0}}\frac{e^{zt}t}{(z^2+2z+2)}-\frac{e^{zt}(2z+2)}{(z^2+2z+2)^2}\\
&= \frac{e^{0t}t}{2}-\frac{2}{(2)^2} =\frac{t-1}{2}
\end{align*}
for $z=-1+i$ we have the limit
\begin{align*}
\Resid{f(z)}{-1+i} &= \lim_{z\rightarrow{-1+i}}(z+(1-i))\frac{e^{zt}}{z^2(z+(1+i))(z+(1-i))}\\
&= \lim_{z\rightarrow{-1+i}}\frac{e^{zt}}{z^2(z+(1+i))}\\
&= \frac{e^{(-1+i)t}}{(-1+i)^2(-1+i+(1+i))}\\
&= \frac{e^{-t+it}}{(1-1-2i)(2i)}\\
&= \frac{e^{-t+it}}{4}
\end{align*}
and for $z=-1-i$ we have
\begin{align*}
\Resid{f(z)}{-1-i} &= \lim_{z\rightarrow{-1-i}}(z+(1+i))\frac{e^{zt}}{z^2(z+(1+i))(z+(1-i))}\\
&= \lim_{z\rightarrow{-1-i}}\frac{e^{zt}}{z^2(z+(1-i))}\\
&= \frac{e^{-t-it}}{(-1-i)^2(-1-i+(1-i))}\\
&= \frac{e^{-t-it}}{(2i)(-2i)}\\
&= \frac{e^{-t-it}}{4}
\end{align*}
So now we can calculate the integral by the sum of the residues
\begin{align*}
\frac{1}{2\pi{i}}\oint_{C} \frac{e^{zt}}{z^2(z^2+2z+2)}dz &= \frac{1}{2\pi{i}}2\pi{i}\left(\frac{}{}\Resid{f(z)}{0} + \Resid{f(z)}{-1+i} + \Resid{f(z)}{-1-i}\right)\\
&= \frac{t-1}{2} + \frac{e^{-t+it}}{4} + \frac{e^{-t-it}}{4}\\
&= \frac{t-1}{2} + \frac{e^{-t}\left(e^{+it}+e^{-it}\right)}{4}\\
&= \frac{t-1}{2} + \frac{e^{-t}\cos{t}}{2}\\
&= \frac{t-1+e^{-t}\cos{t}}{2}
\end{align*}

\pagebreak

\section{Problem \#3}
Given the function 
$$f(z) = \frac{z^2-2z}{(z+1)^2(z^2+4)}$$
we can calculate the residues of $f(z)$ by noting we can factor the quadratic term into
$$f(z) = \frac{z^2-2z}{(z+1)^2(z+2i)(z-2i)}$$
Therefore we have two poles of order one at $z=\pm2i$ and a second order pole at $z=-1$. We 
note the numerator of $f(z)$ is entire. So we calculate the residues by equation 
\ref{ResOrdN}. For $z=-1$ we have
\begin{align*}
\Resid{f(z)}{-1} &= \lim_{z\rightarrow{-1}}\frac{d}{dz}\left(\frac{}{}(z+1)^2\frac{z^2-2z}{(z+1)^2(z+2i)(z-2i)}\right)\\
&= \lim_{z\rightarrow{-1}}\frac{d}{dz}\left(\frac{}{}\frac{z^2-2z}{z^2+4}\right)\\
&= \lim_{z\rightarrow{-1}}\frac{2z-2}{z^2+4} - \frac{2z(z^2-2z)}{(z^2+4)^2}\\
&= \frac{-4}{5} - \frac{-6}{(5)^2}\\
&= -\frac{20}{25} + \frac{6}{25} = -\frac{14}{25}
\end{align*}
For $z=2i$ we have
\begin{align*}
\Resid{f(z)}{2i} &= \lim_{z\rightarrow{2i}}(z-2i)\frac{z^2-2z}{(z+1)^2(z+2i)(z-2i)}\\
&= \lim_{z\rightarrow{2i}}\frac{z^2-2z}{(z+1)^2(z+2i)}\\
&= \frac{(2i)^2-2(2i)}{(2i+1)^2(2i+2i)}\\
&= \frac{-4-4i}{(-4+1+4i)4i}\\
&= \frac{-4-4i}{-12i-16}\\
&= \frac{1+i}{3i+4}\frac{-3i+4}{-3i+4}\\
&= \frac{-3i+4+3+4i}{25} = \frac{7+i}{25}
\end{align*}
For $z=-2i$ we have
\begin{align*}
\Resid{f(z)}{-2i} &= \lim_{z\rightarrow{-2i}}(z+2i)\frac{z^2-2z}{(z+1)^2(z+2i)(z-2i)}\\
&= \lim_{z\rightarrow{-2i}}\frac{z^2-2z}{(z+1)^2(z-2i)}\\
&= \frac{(-2i)^2-2(-2i)}{(-2i+1)^2(-2i-2i)}\\
&= \frac{-4+4i}{(-4+1-4i)(-4i)}\\
&= \frac{-4+4i}{12i-16} = \frac{i-1}{3i-4}\frac{-3i-4}{-3i-4} = \frac{3-4i+3i+4}{25} = \frac{7-i}{25}
\end{align*}

\pagebreak

\section{Problem \#4}
For the given integral 
$$I = \int_{-\pi}^{\pi}\frac{d\theta}{1+\sin^2\theta}$$
we can complexify the problem by the change of variable
\begin{align*}
dz &= ie^{i\theta}d\theta\\
\sin{\theta} &= \frac{1}{2i}\left(z-\frac{1}{z}\right)
\end{align*}
So that $I$ becomes
\begin{align*}
I &= \int_{-\pi}^{\pi}\frac{d\theta}{1+\sin^2\theta}\\
&\Downarrow\\
&= -i\oint_{C}\frac{dz}{z}\frac{1}{1+\frac{1}{(2i)^2}\left(z-\frac{1}{z}\right)^2}\\
&= -i\oint_{C}\frac{dz}{z}\frac{1}{1-\frac{1}{4}\left(z^2+\frac{1}{z^2}-2\right)}\\
&= -i\oint_{C}\frac{dz}{z-\frac{1}{4}z^3-\frac{1}{4z}+\frac{1}{2}z}\\
&= -i\oint_{C}\frac{4zdz}{6z^2-z^4-1}\\
&= -i\oint_{C}\frac{4zdz}{-(z-(\sqrt{2}-1))(z-(\sqrt{2}+1))(z+(\sqrt{2}-1))(z+\sqrt{2}+1)}\\
&= i\oint_{C}\frac{4zdz}{(z-(\sqrt{2}-1))(z-(\sqrt{2}+1))(z+(\sqrt{2}-1))(z+\sqrt{2}+1)}
\end{align*}
Note the contour $C$ is the unit circle in the complex plane. So we have poles at
\begin{align*}
z &= \sqrt{2}-1\\
z &= \sqrt{2}+1\\
z &= -\sqrt{2}+1\\
z &= -\sqrt{2}-1
\end{align*}
Note the only poles within the contour $C$ are the poles valued less than one these are
$z = \sqrt{2}-1$ and $z=-\sqrt{2}+1$ so we can calculate the residues for these points by
equation \ref{ResOrdN}
\begin{align*}
\Resid{f(z)}{\sqrt{2}-1} &= \lim_{z\rightarrow{\sqrt{2}-1}}(z-(\sqrt{2}-1))\frac{4z}{(z-(\sqrt{2}-1))(z-(\sqrt{2}+1))(z+(\sqrt{2}-1))(z+\sqrt{2}+1)}\\
&= \lim_{z\rightarrow{\sqrt{2}-1}}\frac{4z}{(z-(\sqrt{2}+1))(z+(\sqrt{2}-1))(z+\sqrt{2}+1)}\\
&= \frac{4(\sqrt{2}-1)}{(\sqrt{2}-1-\sqrt{2}-1)(\sqrt{2}-1+\sqrt{2}-1)(\sqrt{2}-1+\sqrt{2}+1)}\\
&= \frac{4(\sqrt{2}-1)}{(-2)2(\sqrt{2}-1)(2\sqrt{2})} = -\frac{1}{2\sqrt{2}}\\
\end{align*}
and
\begin{align*}
\Resid{f(z)}{-\sqrt{2}+1} &= \lim_{z\rightarrow{-\sqrt{2}+1}}(z-(-\sqrt{2}+1))\frac{4z}{(z-(\sqrt{2}-1))(z-(\sqrt{2}+1))(z+(\sqrt{2}-1))(z+\sqrt{2}+1)}\\
&= \lim_{z\rightarrow{-\sqrt{2}+1}}\frac{4z}{(z-(\sqrt{2}-1))(z-(\sqrt{2}+1))(z+\sqrt{2}+1)}\\
&= \frac{4(-\sqrt{2}+1)}{(-\sqrt{2}+1-\sqrt{2}+1)(-\sqrt{2}+1-\sqrt{2}-1)(-\sqrt{2}+1+\sqrt{2}+1)}\\
&= \frac{4(-\sqrt{2}+1)}{2(-\sqrt{2}+1)(-2\sqrt{2})(2)}\\
&= \frac{4}{4(-2\sqrt{2})} = -\frac{1}{2\sqrt{2}}
\end{align*}
So we can find $I$ by the summation of residues 
\begin{align*}
I &= i\oint_{C}\frac{4zdz}{(z-(\sqrt{2}-1))(z-(\sqrt{2}+1))(z+(\sqrt{2}-1))(z+\sqrt{2}+1)}\\
&= i(2\pi{i}\left(\frac{}{}\Resid{f(z)}{\sqrt{2}+1} + \Resid{f(z)}{-\sqrt{2}+1}\right)\\
&= -2\pi\left(-\frac{1}{2\sqrt{2}}-\frac{1}{2\sqrt{2}}\right)\\
&= 2\pi\frac{1}{\sqrt{2}} = \sqrt{2}\pi
\end{align*}

\end{document}

