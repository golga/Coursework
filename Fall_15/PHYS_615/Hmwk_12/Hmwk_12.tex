\documentclass[11pt]{article}

\usepackage{latexsym}
\usepackage{amssymb}
\usepackage{amsthm}
\usepackage{enumerate}
\usepackage{amsmath}
\usepackage{cancel}
\numberwithin{equation}{section}

\setlength{\evensidemargin}{.25in}
\setlength{\oddsidemargin}{-.25in}
\setlength{\topmargin}{-.75in}
\setlength{\textwidth}{6.5in}
\setlength{\textheight}{9.5in}
\newcommand{\due}{December 2nd, 2015}
\newcommand{\HWnum}{12}
\newcommand{\grad}{\bold\nabla}
\newcommand{\vecE}{\vec{E}}
\newcommand{\scrptR}{\vec{\mathfrak{R}}}
\newcommand{\kapa}{\frac{1}{4\pi\epsilon_0}}
\newcommand{\emf}{\mathcal{E}}
\newcommand{\unit}[1]{\ensuremath{\, \mathrm{#1}}}
\newcommand{\real}{\textnormal{Re}}
\newcommand{\Erf}{\textnormal{Erf}}
\newcommand{\sech}{\textnormal{sech}}
\newcommand{\scrO}{\mathcal{O}}
\newcommand{\levi}{\widetilde{\epsilon}}
\newcommand{\partiald}[2]{\ensuremath{\frac{\partial{#1}}{\partial{#2}}}}
\newcommand{\norm}[2]{\langle{#1}|{#2}\rangle}
\newcommand{\inprod}[2]{\langle{#1}|{#2}\rangle}
\newcommand{\ket}[1]{|{#1}\rangle}
\newcommand{\bra}[1]{\langle{#1}|}





\begin{document}
\begin{titlepage}
\setlength{\topmargin}{1.5in}
\begin{center}
\Huge{Physics 3320} \\
\LARGE{Principles of Electricity and Magnetism II} \\
\Large{Professor Ana Maria Rey} \\[1cm]

\huge{Homework \#\HWnum}\\[0.5cm]

\large{Joe Becker} \\
\large{SID: 810-07-1484} \\
\large{\due} 

\end{center}

\end{titlepage}



\section{Problem \#1}
To prove the identity
\begin{equation}
\exp(i\vec{\sigma}\cdot\vec{n}\omega)=\cos\omega\cdot\mathbf{1}+i\vec{\sigma}\vec{n}\sin\omega
\label{Prob1}
\end{equation}
where $\sigma_i$ are the Pauli matrices, $\mathbf{1}$ is the $2\times2$ identity matrix, and 
$\vec{n}$ is a unit vector in $\mathbb{R}^3$, we take the exponential of a matrix, $A$, to be 
defined as
\begin{equation}
e^{A} = \sum_{n=0}^{\infty}\frac{1}{n!}A^n
\label{ExpMat}
\end{equation}
Therefore using equation \ref{ExpMat} on equation \ref{Prob1} where $\vec{\sigma}\cdot\vec{n}$
is taken to be the matrix, $A$
\begin{align*}
\exp(i\vec{\sigma}\cdot\vec{n}\omega) &= \sum_{n=0}^{\infty}\frac{1}{n!}(i\omega\vec{\sigma}\cdot\vec{n})^n\\
&= \sum_{n=0}^{\infty}\frac{i^n}{n!}(\vec{\sigma}\cdot\vec{n})^n\\
&= \sum_{n=0}^{\infty}\frac{(-1)^{2n}\omega^{2n}}{(2n)!}(\vec{\sigma}\cdot\vec{n})^{2n} + i\sum_{n=0}^{\infty}\frac{(-1)^{2n+1}\omega^{2n+1}}{(2n+1)!}(\vec{\sigma}\cdot\vec{n})^{2n+1}
\end{align*}
We note that the Pauli matrices are given as
$$\sigma_x = \left(\begin{array}{cc}
               0     &1\\
               1     &0\\
             \end{array}\right)\qquad
\sigma_y = \left(\begin{array}{cc}
               0     &-i\\
               i     &0\\
             \end{array}\right)\qquad
\sigma_z = \left(\begin{array}{cc}
               1     &0\\
               0     &-1\\
             \end{array}\right)$$
Which implies that
$$\vec{\sigma}\cdot\vec{n} = \frac{1}{\sqrt{3}}\left(\begin{array}{cc}
                                   1     &1-i\\
                                   1+i  &-1\\
                                 \end{array}\right)$$
Note the factor of $\sqrt{3}$ comes from the normalization of $n$ in $\mathbb{R}^3$.  Using
this we can see that
\begin{align*}
(\vec{\sigma}\cdot\vec{n})^2 &= \frac{1}{3}\left(\begin{array}{cc}
                                   1     &1-i\\
                                   1+i  &-1\\
                                 \end{array}\right)
 \left(\begin{array}{cc}
                                   1     &1-i\\
                                   1+i  &-1\\
                                 \end{array}\right)\\
&=\frac{1}{3}\left(\begin{array}{cc}
                                   1+(1-i)(1+i)     &(1-i)-(1-i)\\
                                   (1+i)-(1+i)  &(1+i)(1-i)+1\\
                                 \end{array}\right)\\
&=\left(\begin{array}{cc}
                                    1   &0\\
                                    0   &1\\
                                 \end{array}\right) = \mathbf{1}
\end{align*}
And we can generalize to all even powers to say
$$(\vec{\sigma}\cdot\vec{n})^{2n} = \mathbf{1}$$ 
Therefore
\begin{align*}
\exp(i\vec{\sigma}\cdot\vec{n}\omega) &= \sum_{n=0}^{\infty}\frac{(-1)^{2n}\omega^{2n}}{(2n)!}\mathbf{1} + i\sum_{n=0}^{\infty}\frac{(-1)^{2n+1}\omega^{2n+1}}{(2n+1)!}(\vec{\sigma}\cdot\vec{n})\mathbf{1}\\
&=\cos\omega\cdot\mathbf{1}+i\vec{\sigma}\vec{n}\sin\omega
\end{align*}


\pagebreak

\section{Problem \#2}
Given Hermitian matrices $A$, $B$ and unitary matrices $C$, $D$. Which implies that
$$A=A^{\dagger},\qquad B=B^{\dagger},\qquad C^{\dagger}C=CC^{\dagger}=1 \qquad, D^{\dagger}D=DD^{\dagger} = 1$$
\begin{enumerate}[1)]
\item We can show that
\begin{align*}
(C^{-1}AC)^{\dagger} &= C^{\dagger}A^{\dagger}\left(C^{-1}\right)^{\dagger}\\
&= C^{-1}A^{\dagger}C\\
&= C^{-1}AC
\end{align*}
Therefore $C^{-1}AC$ is Hermitian.

\item We can show that
\begin{align*}
C^{-1}DC(C^{-1}DC)^{\dagger} &= C^{-1}D\cancelto{1}{CC^{\dagger}}D^{\dagger}(C^{-1})^{\dagger}\\
&= C^{-1}\cancelto{1}{DD^{\dagger}}(C^{-1})^{\dagger}\\
&= C^{-1}(C^{-1})^{\dagger}\\
&= C^{\dagger}C = 1
\end{align*}
Therefore $C^{1}DC$ is a unitary matrix.

\item We can show that
\begin{align*}
(i(AB-BA))^{\dagger} &= -i((AB)^{\dagger}-(BA)^{\dagger})\\
&= -i(B^{\dagger}A^{\dagger}-A^{\dagger}B^{\dagger})\\
&= -i(BA-AB) = i(AB-BA)
\end{align*}
Therefore $i(AB-BA)$ is Hermitian.
\end{enumerate}

\section{Problem \#3}
Given the unitary matrix $\mathbf{1}$ we can find the eigenvalues $\lambda$ of this matrix 
by enforcing the condition
$$\det(\mathbf{A}-\lambda\mathbf{1}) = 0$$
where we take $\mathbf{A}$ to be the unitary matrix. This implies that
\begin{align*}
\det(\mathbf{1}-\lambda\mathbf{1}) = \det((1-\lambda)\mathbf{1}) = (1-\lambda)^n
\end{align*}
where $n$ is the rank of the unitary matrix. So the equation 
$$(1-\lambda)^n=0$$
implies that $|\lambda|=1$ or that $\lambda$ is unimodular.

\pagebreak

\section{Problem \#4}
For the center, $Z$, of a group, $G$, defined as the set of elements $z\in{G}$ which commute
with all elements within $G$
$$Z = \{z\in{G}| \ zg=gz, \ \forall{g\in{G}}\}$$
We note that the group $Z$ is an Abelian group by definition. This is due to the fact for
elements $f,g\in{Z}$ by definition we have $fg=gf$. Therefore we have a commutative operator 
(multiplication).


\end{document}

