\documentclass[11pt]{article}

\usepackage{latexsym}
\usepackage{amssymb}
\usepackage{amsthm}
\usepackage{enumerate}
\usepackage{amsmath}
\usepackage{cancel}
\numberwithin{equation}{section}

\setlength{\evensidemargin}{.25in}
\setlength{\oddsidemargin}{-.25in}
\setlength{\topmargin}{-.75in}
\setlength{\textwidth}{6.5in}
\setlength{\textheight}{9.5in}
\newcommand{\due}{September 16th, 2015}
\newcommand{\HWnum}{2}
\newcommand{\grad}{\bold\nabla}
\newcommand{\vecE}{\vec{E}}
\newcommand{\scrptR}{\vec{\mathfrak{R}}}
\newcommand{\kapa}{\frac{1}{4\pi\epsilon_0}}
\newcommand{\emf}{\mathcal{E}}
\newcommand{\unit}[1]{\ensuremath{\, \mathrm{#1}}}
\newcommand{\real}{\textnormal{Re}}
\newcommand{\Erf}{\textnormal{Erf}}
\newcommand{\sech}{\textnormal{sech}}
\newcommand{\scrO}{\mathcal{O}}
\newcommand{\levi}{\widetilde{\epsilon}}
\newcommand{\partiald}[2]{\ensuremath{\frac{\partial{#1}}{\partial{#2}}}}
\newcommand{\norm}[2]{\langle{#1}|{#2}\rangle}
\newcommand{\inprod}[2]{\langle{#1}|{#2}\rangle}
\newcommand{\ket}[1]{|{#1}\rangle}
\newcommand{\bra}[1]{\langle{#1}|}





\begin{document}
\begin{titlepage}
\setlength{\topmargin}{1.5in}
\begin{center}
\Huge{Physics 3320} \\
\LARGE{Principles of Electricity and Magnetism II} \\
\Large{Professor Ana Maria Rey} \\[1cm]

\huge{Homework \#\HWnum}\\[0.5cm]

\large{Joe Becker} \\
\large{SID: 810-07-1484} \\
\large{\due} 

\end{center}

\end{titlepage}



\section{Problem \#1}
We can derive the \emph{Rodrigues' formula} given by
\begin{equation}
P_l(x) = \frac{1}{2^ll!}\left(\frac{d}{dx}\right)^l(x^2-1)^l
\label{RodForm}
\end{equation}
where $P_l(x)$ represent \emph{Legendre's polynomials}. We first take note of the derivative
\begin{align*}
\frac{d}{dx}(x^2-1)^{l} &= l(x^2-1)^{l-1}(2x)
\end{align*}
Which leads to the following identity 
\begin{equation}
(x^2-1)\frac{d}{dx}(x^2-1)^l = 2lx(x^2-1)^l
\label{Ident}
\end{equation}
Now we need to calculate the $l+1$th derivative of both sides of equation \ref{Ident} by 
\emph{Leibnitz rule} which is given by
\begin{equation}
\frac{d^{l+1}}{dx^{l+1}}\left[\frac{}{}u(x)v(x)\right] = \sum_{k=0}^{l+1}{l+1\choose{k}}\frac{d^{l+1-k}u}{dx^{l+1-k}}\frac{d^kv}{dx^k}.
\label{Leibnitz}
\end{equation}
So if we can see that using equation \ref{Leibnitz} on the left hand side of equation 
\ref{Ident} to get
\begin{align*}
\frac{d^{l+1}}{dx^{l+1}}\left[(x^2-1)\frac{d}{dx}(x^2-1)^l\right] &= \sum_{k=0}^{l+1}{l+1\choose{k}}\frac{d^{l+1-k}}{dx^{l+1-k}}\left[\frac{d}{dx}(x^2-1)^l\right]\frac{d^k}{dx^k}\left[x^2-1\right]\\
\end{align*}
We see that the term 
$$\frac{d^k}{dx^k}\left[x^2-1\right]$$
goes to zero for $k\ge0$, therefore we our sum only goes to $k=2$. Which makes the sum become
\begin{align*}
\Rightarrow &= {l+1\choose{0}}\frac{d^{l+1}}{dx^{l+1}}\left[\frac{d}{dx}(x^2-1)^l\right]\left(x^2-1\right)
        + {l+1\choose{1}}\frac{d^{l}}{dx^{l}}\left[\frac{d}{dx}(x^2-1)^l\right](2x) 
        + {l+1\choose{2}}\frac{d^{l-1}}{dx^{l-1}}\left[\frac{d}{dx}(x^2-1)^l\right](2)\\
&= \frac{d^{2}}{dx^{2}}\left[\frac{d^l}{dx^l}(x^2-1)^l\right]\left(x^2-1\right)
        + (l+1)2x\frac{d}{dx}\left[\frac{d^l}{dx^l}(x^2-1)^l\right]
        + l(l+1)\left[\frac{d^{l}}{dx^{l}}(x^2-1)^l\right]
\end{align*}
Then we apply equation \ref{Leibnitz} to the right hand side of equation \ref{Ident} to get
\begin{align*}
\frac{d^{l+1}}{dx^{l+1}}\left[2lx(x^2-1)^l\right] &= \sum_{k=0}^{l+1}{l+1\choose{k}}\frac{d^{l+1-k}}{dx^{l+1-k}}\left[(x^2-1)^l\right]\frac{d^k}{dx^k}\left[2lx\right]\\
&= {l+1\choose{0}}\frac{d^{l+1}}{dx^{l+1}}\left[(x^2-1)^l\right](2lx)
 + {l+1\choose{1}}\frac{d^{l}}{dx^{l}}\left[(x^2-1)^l\right](2l)\\
&= 2lx\frac{d}{dx}\left[\frac{d^l}{dx^l}(x^2-1)^l\right]
 + 2l(l+1)\left[\frac{d^{l}}{dx^{l}}(x^2-1)^l\right]
\end{align*}
We can define
$$y \equiv \left(\frac{d}{dx}\right)^l(x^2-1)^l$$
which when we combine both sides of \ref{Ident}
\begin{align*}
\left(x^2-1\right)\frac{d^{2}}{dx^{2}}y+(l+1)2x\frac{d}{dx}y + l(l+1)y &= 2lx\frac{d}{dx}y + 2l(l+1)y\\
&\Downarrow\\
\left(x^2-1\right)\frac{d^{2}}{dx^{2}}y+(l+1)2x\frac{d}{dx}y + l(l+1)y - 2lx\frac{d}{dx}y - 2l(l+1)y &=0\\
\left(x^2-1\right)\frac{d^{2}y}{dx^{2}} + 2x\frac{dy}{dx} - l(l+1)y &=0\\
\left(1-x^2\right)\frac{d^{2}y}{dx^{2}} - 2x\frac{dy}{dx} + l(l+1)y &=0
\end{align*}
We see that this is the \emph{Legendre Differential Equation} which we know has the 
solutions $P_l(x)$. This implies that $y$ are the Legendre Polynomials with a normalization 
factor. To find the normalization factor we impose the condition $P_n(1) = 1$. We note that 
this condition makes every term with $x^2-1$ go to zero. We can infer that the only terms 
that do not have this term is the term that is derived $l$ times this gives us a factor of
$2^ll!$. So for $x=1$ we have
\begin{align*}
y = {2^ll!}
\end{align*}
which implies that 
$$P_l(x) = \frac{1}{2^ll!}y = \frac{1}{2^ll!}\left(\frac{d}{dx}\right)^l(x^2-1)^l$$
Which is in agreement with equation \ref{RodForm}.


\pagebreak

\section{Problem \#2}
Given the differential equation 
\begin{equation}
xy^2y' - \frac{1}{3}(x^3+y^3) = 0
\label{Prob2}
\end{equation}
we can rearrange equation \ref{Prob2} into the form
$$A(x,y)dx + B(x,y)dy = 0$$
which we get as
$$y^2dy - \frac{1}{3}\left(x^2+\frac{y^3}{x}\right)dx = 0$$
where $A(x,y) = \frac{1}{3}\left(x^2+\frac{y^3}{x}\right)$ and $B(x,y) = y^2$. We verify 
that the conditions
\begin{align*}
A(ax,ay) &= a^rA(x,y)\\
B(ax,ay) &= a^rB(x,y)
\end{align*}
by
\begin{align*}
A(ax,ay) &= \frac{1}{3}\left((ax)^2+\frac{(ay)^3}{ax}\right) \\
&= \frac{1}{3}\left((ax)^2+\frac{(ay)^3}{ax}\right)\\
&= \frac{1}{3}\left(a^2x^2+\frac{a^3y^3}{ax}\right)\\
&= \frac{1}{3}\left(a^2x^2+a^2\frac{y^3}{x}\right)\\
&= a^2\frac{1}{3}\left(2x^2+\frac{y^3}{x}\right)\\
&= a^2A(x,y)
\end{align*}
and
\begin{align*}
B(ax,ay) &= (ay)^2\\
&= a^2y^2\\
&= a^2B(x,y).
\end{align*}
This implies that we can set a change of variables
$$x,y\rightarrow x,v=\frac{y}{x}$$
which transforms equation \ref{Prob2} into
\begin{align*}
0 &= y^2dy - \frac{1}{3}\left(x^2+\frac{y^3}{x}\right)dx\\
&\Downarrow\\
0 &= (vx)^2(vdx+xdv) - \frac{1}{3}\left(x^2+\frac{(vx)^3}{x}\right)dx
\end{align*}
Which allows us to use separation of variables by
\begin{align*}
0 &= v^2x^3dv + \left(v^3x^2 - \frac{1}{3}x^2-\frac{1}{3}v^3x^2\right)dx\\
0 &= v^2x^3dv + \left(-\frac{1}{3}x^2+\frac{2}{3}v^3x^2\right)dx\\
0 &= v^2x^3dv - \frac{1}{3}x^2\left(1-2v^3\right)dx\\
&\Downarrow\\
\frac{3v^2}{1-2v^3}dv &= \frac{1}{x}dx
\end{align*}
Now we can solve by integrating both sides 
\begin{align*}
\int\frac{1}{x}dx &= \int\frac{3v^2}{1-2v^3}dv\\
&\Downarrow\\
\log(x) &= \int\frac{3v^2}{1-2v^3}dv
\end{align*}
where we use a substitution $u=1-2v^3$ and $du = -6v^2dv$ to get
\begin{align*}
\log(x) &= \int\frac{3v^2}{1-2v^3}dv\\
&\Downarrow\\
\log(x) &= \frac{1}{2}\int\frac{du}{u}\\
\log(x) &= -\frac{1}{2}\log(1-2v^3)+c\\
&\Downarrow\\
x &= C\exp\left[\log((1-2v^3)^{-1/2})\right]\\
x &= C(1-2v^3)^{-1/2}\\
&\Downarrow\\
x &= C\left(1-2\left(\frac{y}{x}\right)^3\right)^{-1/2}\\
&\Downarrow\\
-\frac{Cx^{-2}-1}{2} &= \frac{y}{x}\\
&\Downarrow\\
y^3 &= \frac{Cx+x^3}{2} \\
&\Downarrow\\
y(x) &= \left(\frac{x(x^2+C)}{2}\right)^{1/3}
\end{align*}

\pagebreak

\section{Problem \#3}
Given the second order differential equation
\begin{equation}
y''+3xy'-y=0
\label{Prob3}
\end{equation}
we can solve for the general solution by using a power series given by
\begin{align*}
y(x) &= \sum_{n=0}^{\infty}c_nx^n\\
y'(x) &= \sum_{n=0}^{\infty}c_nnx^{n-1}\\
y''(x) &= \sum_{n=0}^{\infty}c_nn(n-1)x^{n-2}\\
\end{align*}
Note we can shift the indices of $y''(x)$ such that
$$y''(x) = \sum_{n=0}^{\infty}c_{n+2}(n+1)(n+2)x^{n}.$$
So, we replace the power series into equation \ref{Prob3} and get
\begin{align*}
0 &= y''+3xy'-y\\
&\Downarrow\\
0 &= \sum_{n=0}^{\infty}\left(c_{n+2}(n+1)(n+2)x^{n} + 3c_nnx^{n}-c_nx^n\right)\\
0 &= \sum_{n=0}^{\infty}\left(c_{n+2}(n+1)(n+2) + 3c_nn-c_n\right)x^n
\end{align*}
Which implies that for all $x$ 
$$0 = c_{n+2}(n+1)(n+2) + c_n(3n-1)$$
must hold true. This leads to a recursion relation
\begin{equation}
c_{n+2} = -\frac{3n-1}{(n+1)(n+2)}c_n
\label{RecCN}
\end{equation}
We expect to have two free constants due to the fact that this is a second order equation.
We note that two constants define the power series where $c_0$ defines the even indices and
$c_1$ defines the odd indices by equation \ref{RecCN}. These are our two free constants. 
So, we can write the general solution by
$$y(x) = c_0\left(1 + \frac{1}{2}x^2 - \frac{5}{24}x^4 + \frac{55}{720}x^6+...\right)
         +c_1\left(x - \frac{1}{3}x^3 + \frac{2}{15}x^5 + ...\right)$$


\pagebreak

\section{Problem \#4}
Given the differential equation
\begin{equation}
y'' - 4y' + 3y = e^{2x}+3x^2
\label{Prob4}
\end{equation}
we can solve for $y(x)$ by first solving the homogeneous version of equation \ref{Prob4} 
with the ansatz $y_0(x) = e^{ax}$. This yields the equation
\begin{align*}
a^2 - 4a + 3 &= 0\\
(a-3)(a-1) &= 0\\
&\Downarrow\\
a &= 3,1
\end{align*}
So we can say that 
$$y_0(x) = c_1e^{3x} + c_2e^{x}.$$
Now we need to find the particular solution by using the ansatz 
$$y_p(x) = ae^{2x} + bx^2 + cx + d$$
which we can calculate
\begin{align*}
y_p'(x) &= 2ae^{2x} + 2bx + c\\
y_p''(x) &= 4ae^{2x} + 2b.
\end{align*}
Then we plug $y_p(x)$, $y_p'(x)$, and $y_p''(x)$ in equation \ref{Prob4} 
\begin{align*}
y_p'' - 4y_p' + 3y_p &= e^{2x}+3x^2\\
&\Downarrow\\
4ae^{2x} + 2b - 4(2ae^{2x} + 2bx + c) + 3(ae^{2x} + bx^2 + cx + d) &= e^{2x}+3x^2\\
4ae^{2x}-8ae^{2x}+3ae^{2x} + 3bx^2 - 8bx+3cx + 2b-4c+3d &= e^{2x}+3x^2\\
-ae^{2x} + 3bx^2 + (3c-8b)x + 2b-4c+3d &= e^{2x}+3x^2
\end{align*}
and solve for the coefficients by the system of equations
\begin{align*}
-a &= 1\\
3b &= 3\\
3c - 8c &= 0\\
2b - 4c +3d &= 0
\end{align*}
which gives the solution
\begin{align*}
a &= -1\\
b &= 1\\
c &= \frac{8}{3}\\
d &= \frac{26}{9}
\end{align*}
Which results in the particular solution
$$y_p(x) = -e^{2x} + x^2 + \frac{8}{3}x + \frac{26}{9}$$ 
and the total solution
$$y(x) =c_1e^{3x} + c_2e^{x} -e^{2x} + x^2 + \frac{8}{3}x + \frac{26}{9}$$ 



\end{document}

