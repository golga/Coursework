\documentclass[11pt]{article}

\usepackage{latexsym}
\usepackage{amssymb}
\usepackage{amsthm}
\usepackage{enumerate}
\usepackage{amsmath}
\usepackage{cancel}
\numberwithin{equation}{section}

\setlength{\evensidemargin}{.25in}
\setlength{\oddsidemargin}{-.25in}
\setlength{\topmargin}{-.75in}
\setlength{\textwidth}{6.5in}
\setlength{\textheight}{9.5in}
\newcommand{\due}{October 7th, 2015}
\newcommand{\HWnum}{4}
\newcommand{\grad}{\bold\nabla}
\newcommand{\vecE}{\vec{E}}
\newcommand{\scrptR}{\vec{\mathfrak{R}}}
\newcommand{\kapa}{\frac{1}{4\pi\epsilon_0}}
\newcommand{\emf}{\mathcal{E}}
\newcommand{\unit}[1]{\ensuremath{\, \mathrm{#1}}}
\newcommand{\real}{\textnormal{Re}}
\newcommand{\Erf}{\textnormal{Erf}}
\newcommand{\sech}{\textnormal{sech}}
\newcommand{\scrO}{\mathcal{O}}
\newcommand{\levi}{\widetilde{\epsilon}}
\newcommand{\partiald}[2]{\ensuremath{\frac{\partial{#1}}{\partial{#2}}}}
\newcommand{\norm}[2]{\langle{#1}|{#2}\rangle}
\newcommand{\inprod}[2]{\langle{#1}|{#2}\rangle}
\newcommand{\ket}[1]{|{#1}\rangle}
\newcommand{\bra}[1]{\langle{#1}|}





\begin{document}
\begin{titlepage}
\setlength{\topmargin}{1.5in}
\begin{center}
\Huge{Physics 3320} \\
\LARGE{Principles of Electricity and Magnetism II} \\
\Large{Professor Ana Maria Rey} \\[1cm]

\huge{Homework \#\HWnum}\\[0.5cm]

\large{Joe Becker} \\
\large{SID: 810-07-1484} \\
\large{\due} 

\end{center}

\end{titlepage}



\section{Problem \#1}
For a step potential given by
$$U(x) = \left\{\begin{array}{cl}
                  0      &x<0\\
                  U_0    &x>0
         \end{array}\right.$$
with an energy, $E$, in the range: $0<E<U_0$. We see that we have two regions, region $I$ is
$x<0$ and region $II$ is given by $x>0$. In the region $I$ we have a free particle with a 
wave-function 
$$\psi_I(x) = Ae^{ikx} + Be^{-ikx}$$
where the positive exponent is related to the right moving incident wave and the negative 
exponent is the left moving reflection wave. We note that $k$ is given by
$$k = \frac{\sqrt{2\mu E}}{\hbar}.$$
Then for region $II$ we note that $U_0>E$ which
yields an exponential decay given by
$$\psi_{II}(x) = Ce^{-\kappa x}$$
where $\kappa$ is a real positive constant given by
$$\kappa = \frac{\sqrt{2\mu(U_0-E)}}{\hbar}.$$
We need to use the continuity of the wave-function and it's derivative to find the 
coefficients $A$, $B$, and $C$. First we see that the wave-function at $x=0$ 
\begin{align*}
\psi_I(x=0) &= \psi_{II}(x=0)\\
&\Downarrow\\
Ae^{ik0} + Be^{-ik0} &= Ce^{-\kappa 0}\\
A + B = C
\end{align*}
then we take the first spacial derivative of the wave-function at $x=0$ as
\begin{align*}
\psi_I'(x=0) &= \psi_{II}'(x=0)\\
&\Downarrow\\
A(ik)e^{ik0} + B(-ik)e^{-ik0} &= C(-\kappa)e^{-\kappa 0}\\
ik(A - B) &= -\kappa C
\end{align*}
Using these two continuity conditions we can calculate the reflection by noting that $|B|^2$
is the magnitude of the reflected wave which implies that
$$\textnormal{Reflection Coefficient}\rightarrow R\equiv\left|\frac{B}{A}\right|^2$$
So we can solve for the ratio by
\begin{align*}
ik(A - B) &= -\kappa C\\
ik(A - B) &= -\kappa (A+B)\\
\frac{ik}{\kappa}A - \frac{ik}{\kappa}B &= -A - B\\
\frac{ik}{\kappa}A  + A &= \frac{ik}{\kappa}B- B\\
\left(\frac{ik}{\kappa}+1\right)A&= \left(\frac{ik}{\kappa}-1\right) B\\
&\Downarrow\\
\frac{B}{A} &= \frac{\frac{ik}{\kappa}+1}{\frac{ik}{\kappa}-1}
\end{align*}
So we can calculate the magnitude squared by
\begin{align*}
R = \left|\frac{B}{A}\right|^2 &= \left(\frac{\dfrac{ik}{\kappa}+1}{\dfrac{ik}{\kappa}-1}\right)\left(\frac{\dfrac{ik}{\kappa}+1}{\dfrac{ik}{\kappa}-1}\right)^*\\
&= \left(\frac{\dfrac{ik}{\kappa}+1}{\dfrac{ik}{\kappa}-1}\right)\left(\frac{-\dfrac{ik}{\kappa}+1}{-\dfrac{ik}{\kappa}-1}\right)\\
&= \frac{\dfrac{k^2}{\kappa^2}+1}{\dfrac{k^2}{\kappa^2}+1} = 1
\end{align*}
We note that for an infinitely deep potential barrier where we have an energy less that the 
barrier potential we have complete reflection. We can also think about limiting cases the 
first being when $E\rightarrow\infty$ we see that in this limit we have $E>>U_0$ which makes
$$\lim_{E\rightarrow\infty}\kappa = \lim_{E\rightarrow\infty}\sqrt{2\mu(U_0-E)}/\hbar = i\sqrt{2\mu E}/\hbar = ik$$
so we see that in the limit
\begin{align*}
\lim_{E\rightarrow\infty} R =\lim_{E\rightarrow\infty} \left|\frac{B}{A}\right|^2 &= \lim_{E\rightarrow\infty}\left(\frac{\dfrac{ik}{\kappa}+1}{\dfrac{ik}{\kappa}-1}\right)\left(\frac{\dfrac{ik}{\kappa}+1}{\dfrac{ik}{\kappa}-1}\right)^*\\
&= \left(\frac{\dfrac{ik}{ik}+1}{\dfrac{ik}{ik}-1}\right)\left(\frac{\dfrac{ik}{ik}+1}{\dfrac{ik}{ik}-1}\right)^*\\
&= \left(\frac{2}{1-1}\right)\left(\frac{2}{1-1}\right)^* = 0
\end{align*}
So we see that as we go to infinite energy the barrier becomes infinitely small. This implies
that the reflection coefficient becomes zero or we have no reflected wave. In the other limit
where $E\rightarrow{U_0}$ we see that the wave-function in region $II$ becomes constant 
because 
$$\kappa = \frac{\sqrt{2\mu(U_0-U_0)}}{\hbar} = 0$$
so $\psi_{II}$ becomes
$$\psi_{II} = Ce^{0x} = C$$
So the continuity conditions become
\begin{align*}
A + B &= C\\
A - B &= 0
\end{align*}
So we see that $A=B$ which implies that for $E\rightarrow{U_0}$ we have a reflection 
coefficient, $R$, that is still unity. So for up to $E=U_0$ we have full reflection if our 
step potential is infinitely deep.

\pagebreak

\section{Problem \#2}
For a two-dimensional potential well with absolutely impermeable walls we have a potential 
given by
$$U(x,y) = \left\{\begin{array}{ccc}
                 0       &0<x<a\\
                 \infty  &x>a \textnormal{\ and\ } x<0\\
                 0       &0<y<b\\
                 \infty  &y>b \textnormal{\ and\ } y<0\\
            \end{array}\right.$$
We can take the overall wave function as the product of the two individual solutions as
$$\Psi(x,y) = \psi_n(x)\psi_m(y)$$
where $\psi_n$ are the normalized eigenfunction of the one-dimensional infinite square well given by
\begin{align*}
\psi_n(x) &= \sqrt{\frac{2}{a}}\sin\left(\frac{n\pi}{a}x\right)\\
\psi_m(y) &= \sqrt{\frac{2}{b}}\sin\left(\frac{m\pi}{b}y\right)
\end{align*}
So we can write the overall wave function as
$$\Psi(x,y) = \sqrt{\frac{2}{a}}\sqrt{\frac{2}{b}}\sin\left(\frac{n\pi}{a}x\right)\sin\left(\frac{m\pi}{b}y\right)$$
in the domain where $0<x<a$ and $0<y<b$. We note that $\Psi(x,y)=0$ everywhere else. This
allows us to find the probability of finding a particle in the region $a/3\le{x}\le{2a/3}, 
b/3\le{y}\le{2b/3}$ by integrating the probability density function given by $\Psi^*(x,y)\Psi(x,y)$ 
over this domain
\begin{align*}
\int_{a/3}^{2a/3}\int_{b/3}^{2b/3}\Psi^*(x,y)\Psi(x,y)dydx &= \int_{a/3}^{2a/3}\int_{b/3}^{2b/3}\left(\sqrt{\frac{2}{a}}\sqrt{\frac{2}{b}}\sin\left(\frac{n\pi}{a}x\right)\sin\left(\frac{m\pi}{b}y\right)\right)^2dydx\\
&= \frac{4}{ab}\int_{a/3}^{2a/3}\sin^2\left(\frac{n\pi}{a}x\right)dx\int_{b/3}^{2b/3}\sin^2\left(\frac{m\pi}{b}y\right)dy\\
&= \frac{4}{ab}\int_{a/3}^{2a/3}\frac{1}{2}\left(1-\cos\left(2\frac{n\pi}{a}x\right)\right)dx\int_{b/3}^{2b/3}\frac{1}{2}\left(1-\cos\left(2\frac{m\pi}{b}y\right)\right)dy\\
&= \frac{1}{ab}\left(x-\frac{a}{2n\pi}\sin\left(2\frac{n\pi}{a}x\right)\right|_{a/3}^{2a/3}\left(y-\frac{2m\pi}{b}\sin\left(2\frac{m\pi}{b}y\right)\right|_{b/3}^{2b/3}\\
&= \frac{1}{ab}\left(\frac{2a}{3}-\frac{a}{2n\pi}\sin\left(2\frac{n\pi}{a}\frac{2a}{3}\right)-\frac{a}{3}+\frac{a}{2n\pi}\sin\left(2\frac{n\pi}{a}\frac{a}{3}\right)\right)\\
&\qquad \times\left(\frac{2b}{3}-\frac{b}{2m\pi}\sin\left(2\frac{m\pi}{b}\frac{2b}{3}\right) - \frac{b}{3}+\frac{b}{2m\pi}\sin\left(2\frac{m\pi}{b}\frac{b}{3}\right)\right)\\
&= \left(\frac{1}{3}-\frac{1}{2n\pi}\sin\left(\frac{4n\pi}{3}\right)+\frac{1}{2n\pi}\sin\left(\frac{2n\pi}{3}\right)\right)\\
&\qquad\times\left(\frac{1}{3}-\frac{1}{2m\pi}\sin\left(\frac{4m\pi}{3}\right) + \frac{1}{2m\pi}\sin\left(\frac{2m\pi}{3}\right)\right)
\end{align*}
We note that the probability depends on which state we are in. If we take the particle to be
in the ground state which implies that $n=m=1$ so we calculate
\begin{align*}
&= \left(\frac{1}{3}-\frac{1}{2n\pi}\sin\left(\frac{4n\pi}{3}\right)+\frac{1}{2n\pi}\sin\left(\frac{2n\pi}{3}\right)\right)\left(\frac{1}{3}-\frac{1}{2m\pi}\sin\left(\frac{4m\pi}{3}\right) + \frac{1}{2m\pi}\sin\left(\frac{2m\pi}{3}\right)\right)\\
&\Downarrow\\
&= \left(\frac{1}{3}-\frac{1}{2\pi}\sin\left(\frac{4\pi}{3}\right)+\frac{1}{2\pi}\sin\left(\frac{2\pi}{3}\right)\right)\left(\frac{1}{3}-\frac{1}{2\pi}\sin\left(\frac{4\pi}{3}\right) + \frac{1}{2\pi}\sin\left(\frac{2\pi}{3}\right)\right)\\
&= \frac{1}{9}+\frac{3}{4\pi^2}+\frac{\sqrt{3}}{3\pi}\approx 0.37
\end{align*}

\pagebreak

\section{Problem \#3}
For a particle of mass $m_0$ in a potential well given by
$$U(x)=\left\{\begin{array}{cl}
               \infty,     &x<0\\
               0,          &0<x<a\\
               U_0,        &x>a
        \end{array}\right.$$
with an energy, $E<U_0$. We see that there are three distinct regions for this particle we
note that for an infinite potential $\psi(x)=0$ with a discontinuous first spacial derivative.
We then see that for the zero potential we assume we have an oscillating wave-function and
in the region $x>a$ we have an exponential decay. So our wave function is given by
$$\psi(x)=\left\{\begin{array}{ll}
               0,                        &x<0\\
               A\sin(kx+\delta),          &0<x<a\\
               Be^{-\kappa{x}},           &x>a
        \end{array}\right.$$
Where $k$ and $\kappa$ are real and positive valued given by
\begin{align*}
k &= \frac{\sqrt{2m_0E}}{\hbar}\\
\kappa &= \frac{\sqrt{2m_0(U_0-E)}}{\hbar} = \sqrt{\frac{2m_0U_0}{\hbar^2}-k^2}
\end{align*}
Next, we apply the continuity condition at $x=0$ to see
\begin{align*}
\psi_{I}(x=0) &= \psi_{II}(x=0)\\
&\Downarrow\\
0 &= A\sin(k0+\delta)\\
0 &= \sin(\delta)
&\Downarrow\\
\delta = 0
\end{align*}
Next we have the continuity of the wave-function and its derivative at $x=a$ to get
\begin{align*}
\psi_{II}(x=a) &= \psi_{III}(x=a)\\
&\Downarrow\\
A\sin(ka) &= Be^{-\kappa{a}}
\end{align*}
and 
\begin{align*}
\psi_{II}'(x=a) &= \psi_{III}'(x=a)\\
&\Downarrow\\
Ak\cos(ka) &= -B\kappa e^{-\kappa{a}}
\end{align*}
We can divide the two continuity relations to yield
\begin{align*}
\frac{Ak\cos(ka)}{A\sin(ka)} &= \frac{-B\kappa e^{-\kappa{a}}}{Be^{-\kappa{a}}}\\
k\cot(ka) &= -\kappa\\
&\Downarrow\\
ka &= -\textnormal{arccot}\left(\sqrt{\frac{2m_0U_0}{\hbar^2k^2}-1}\right)
\end{align*}
So we note the relation between $\textnormal{arccot}(x)$ and $\arcsin(x)$ is
$$\textnormal{arccot}(x) = n\pi - \arcsin\left(\sqrt{\frac{1}{x^2+1}}\right)$$
which uses the periodicity of the $\cot$ function. This identity yields
\begin{align*}
ka &= n\pi-\arcsin\left(\sqrt{\frac{1}{\frac{2m_0U_0}{\hbar^2k^2}-1+1}}\right)\\
ka &= n\pi-\arcsin\left(\frac{\hbar{k}}{\sqrt{2m_0U_0}}\right)
\end{align*}
We next note that for a discrete spectrum to have a condition on $k$ such that
$$k<\sqrt{\frac{2m_0{U_0}}{\hbar}}.$$
Therefore, we can see in the limiting case where $k=\sqrt{2m_0U_0}/\hbar$ we can find the 
total number of discrete states $n$ by
\begin{align*}
a\frac{\sqrt{2m_0U_0}}{\hbar} &= n\pi-\arcsin\left(\frac{\hbar{k}}{\sqrt{2m_0U_0}}\frac{\sqrt{2m_0U_0}}{\hbar}\right)\\
a\frac{\sqrt{2m_0U_0}}{\hbar} &= n\pi-\arcsin(1)\\
a\frac{\sqrt{2m_0U_0}}{\hbar} &= n\pi-\frac{\pi}{2}\\
&\Downarrow\\
n &= \frac{\sqrt{2m_0a^2U_0}}{\hbar\pi}
\end{align*}
And for for $a^2U_0=110\hbar^2/m_0$ we can find the number of states as
\begin{align*}
n &= \frac{\sqrt{2m_0a^2U_0}}{\hbar\pi}\\
&\Downarrow\\
n &= \sqrt{\frac{2m_0}{\hbar^2\pi^2}\frac{110\hbar^2}{m_0}}\\
&= \frac{\sqrt{220}}{\pi}\approx 4.7
\end{align*}
This implies that for the given energy there are four discrete states.

\pagebreak

\section{Problem \#4}
For the triangle potential barrier given by
$$U(x) = \left\{\begin{array}{ll}
               0                    &x<-a\\
               \frac{U_0}{a}x+U_0   &-a<x<0\\
               -\frac{U_0}{a}x+U_0  &0<x<a\\
               0                    &x>a\\
         \end{array}\right.$$
we can calculate the transmission coefficient, $D$, for a particle of mass $m_0$ by the 
equation given by the quasiclassical approximation.
\begin{equation}
D = \exp\left(-\frac{2}{\hbar}\int_{x_1}^{x_2}\sqrt{2m_0(U(x)-E)}dx\right)
\label{Trans}
\end{equation}
where $x_1$ and $x_2$ are the classical turning points given when $E=U(x)$. Which we can
find as
\begin{align*}
E &= \frac{U_0}{a}x_1+U_0\\
&\Downarrow\\
x_1 &= \frac{a}{U_0}(E-U_0)\\
&= a\left(\frac{E}{U_0}-1\right)
\end{align*}
and
\begin{align*}
E &= -\frac{U_0}{a}x_2+U_0\\
&\Downarrow\\
x_2 &= -\frac{a}{U_0}(E-U_0)\\
&= a\left(1-\frac{E}{U_0}\right)
\end{align*}
Now, we can calculate the integral in equation \ref{Trans} by
\begin{align*}
\int_{x_1}^{x_2}\sqrt{2m_0(U(x)-E)}dx &= \sqrt{2m_0}\int_{x_1}^{0}\sqrt{\frac{U_0}{a}x+U_0-E}dx + \sqrt{2m_0}\int_{0}^{x_2}\sqrt{-\frac{U_0}{a}x+U_0-E}dx\\
&= \sqrt{2m_0}\left(\frac{2a}{3U_0}\left(\frac{U_0}{a}x+U_0-E\right)^{3/2}\right|_{x_1}^{0} + \sqrt{2m_0}\left(-\frac{2a}{3U_0}\left(-\frac{U_0}{a}x+U_0-E\right)^{3/2}\right|_{0}^{x_2}\\
&= \sqrt{2m_0}\frac{2a}{3U_0}\left((U_0-E)^{3/2} + (U_0-E)^{3/2}\frac{}{}\right)\\
&= \frac{4a\sqrt{2m_0}}{3U_0}(U_0-E)^{3/2}
\end{align*}
So the transmission coefficient as found by equation \ref{Trans} is
$$D = \exp\left(-\frac{8a\sqrt{2m_0}}{3\hbar{U_0}}(U_0-E)^{3/2}\right)$$


\end{document}

