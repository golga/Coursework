\documentclass[11pt]{article}

\usepackage{latexsym}
\usepackage{amssymb}
\usepackage{amsthm}
\usepackage{enumerate}
\usepackage{amsmath}
\usepackage{cancel}
\numberwithin{equation}{section}

\setlength{\evensidemargin}{.25in}
\setlength{\oddsidemargin}{-.25in}
\setlength{\topmargin}{-.75in}
\setlength{\textwidth}{6.5in}
\setlength{\textheight}{9.5in}
\newcommand{\due}{November 9th, 2015}
\newcommand{\HWnum}{7}
\newcommand{\grad}{\bold\nabla}
\newcommand{\vecE}{\vec{E}}
\newcommand{\scrptR}{\vec{\mathfrak{R}}}
\newcommand{\kapa}{\frac{1}{4\pi\epsilon_0}}
\newcommand{\emf}{\mathcal{E}}
\newcommand{\unit}[1]{\ensuremath{\, \mathrm{#1}}}
\newcommand{\real}{\textnormal{Re}}
\newcommand{\Erf}{\textnormal{Erf}}
\newcommand{\sech}{\textnormal{sech}}
\newcommand{\scrO}{\mathcal{O}}
\newcommand{\levi}{\widetilde{\epsilon}}
\newcommand{\partiald}[2]{\ensuremath{\frac{\partial{#1}}{\partial{#2}}}}
\newcommand{\norm}[2]{\langle{#1}|{#2}\rangle}
\newcommand{\inprod}[2]{\langle{#1}|{#2}\rangle}
\newcommand{\average}[1]{\left\langle{#1}\right\rangle}
\newcommand{\ket}[1]{|{#1}\rangle}
\newcommand{\bra}[1]{\langle{#1}|}
\newcommand{\Resid}[2]{\ensuremath{\textnormal{Res}\left[{#1},{#2}\right]}}





\begin{document}
\begin{titlepage}
\setlength{\topmargin}{1.5in}
\begin{center}
\Huge{Physics 3310} \\
\LARGE{Principles of Electricity and Magnetism 1} \\
\Large{Professor Thomas R. Schibli} \\[1cm]

\huge{Homework \#\HWnum}\\[0.5cm]

\large{Joe Becker} \\
\large{SID: 810-07-1484} \\
\large{\due} 

\end{center}

\end{titlepage}



\section{Problem \#1}
To find the expectation value, $<\hat{L}^2>$, in the state with a wave function
$$\psi(\theta,\varphi) = A\sin\theta\cos\varphi$$
we note that $\psi(\theta,\varphi)$ is a spherical harmonic with $l=1$. This implies that
\begin{align*}  
<\hat{L}^2> &= \int_{d\Omega}\psi^*(\theta,\varphi)\hat{L}^2\psi(\theta,\varphi)d\Omega\\
&= \int_{d\Omega}\psi^*(\theta,\varphi)\hbar^2(1)(1+1)\psi(\theta,\varphi)d\Omega\\
&= 2\hbar^2
\end{align*}  

\section{Problem \#2}
To find the eigenvalue of the $\hat{L}^2$ operator corresponding to the eigenfunction
$$Y(\theta,\varphi) = A(3\cos^2\theta-1+\sin(2\theta)\cos\varphi)$$
we note that $Y(\theta,\varphi)$ is the eigenfunction that corresponds to the sum of two 
spherical harmonics 
$$Y(\theta,\varphi) = A(Y_{20} + 2Y_{21})$$
note both are in the $l=2$ state. This implies
\begin{align*}
\hat{L}^2Y(\theta,\varphi) &= A(\hat{L}^2Y_{20} + 2\hat{L}^2Y_{21})\\
&= A(\hbar^2(2)(2+1)Y_{20} + \hbar^2(2)(2+1)2Y_{21})\\
&= 6\hbar^2Y(\theta,\varphi)
\end{align*}

\pagebreak

\section{Problem \#3}
For a $s$-state ($l=0$) particle with at mass, $m_0$, in an spherically symmetric infinitely
deep rectangular potential well of radius $a$ is described by the wave function
$$\psi_{klm} = Aj_{l}(kr)Y_{lm}(\theta,\varphi)$$
for $l=0$ we have the state
$$\psi_{k00} = A\frac{\sin(kr)}{kr}$$
Where we can find the normalization constant by noting the continuity condition that 
$\psi_{k00}(r=0) = \psi_{k00}(r=a) = 0$ which implies that $k= n\pi/a$. Where $n$ becomes
the principle quantum number
\begin{align*}
1 &= \int_{V}\psi^*_{k00}\psi_{k00}dV\\
&= 4\pi\int_{0}^{a}A^2\left(\frac{\sin(kr)}{kr}\right)^2r^2dr\\
&= \frac{4\pi{a^2}}{n^2\pi^2}A^2\int_{0}^{a}\sin^2(kr)dr\\
&= \frac{4a^2}{n^2\pi}A^2\frac{a}{2}\\
&\Downarrow\\
A &= n\sqrt{\frac{\pi}{2a^3}}
\end{align*}
we can find the expectation value of $r$ of a particle in this state by
\begin{align*}
<r> &= \int_{V}\psi_{k00}^*r\psi_{k00}dV\\
&= 4\pi\int_{0}^{a}A\frac{\sin(kr)}{kr}rA\frac{\sin(kr)}{kr}r^2dr\\
&= 4\pi\frac{n^2\pi}{2a^3}\frac{a^2}{n^2\pi^2}\int_{0}^{a}r\sin^2(kr)dr\\
&= \frac{2}{a}\frac{a^2}{4} = \frac{a}{2}
\end{align*}
which is the same for one dimensional infinite square well which is what we expect. The same
follows for $<r^2>$
\begin{align*}
<r^2> &= \int_{V}\psi_{k00}^*r\psi_{k00}dV\\
&= \frac{2}{a}\int_{0}^{a}r^2\sin^2(kr)dr\\
&= \frac{a^2}{6}\left(2-\frac{3}{(n\pi)^2}\right)
\end{align*}
Which means we can find the variance as
$$<(\Delta{r})^2> = <r>^2 - <r^2> = \frac{a^2}{12}\left(1-\frac{6}{(n\pi)^2}\right)$$

\pagebreak

\section{Problem \#4}
For a particle of mass, $m_0$, in the ground state of a spherically symmetric infinitely 
deep square potential we a wave function given by
$$\psi_{100}(r) = \frac{1}{\sqrt{2a\pi}}\frac{\sin(\pi{r}/a)}{r}$$
which in order to find the momentum probability distribution we find $\psi$ in momentum 
representation as
\begin{align*}
\phi_{100}(p) &= \frac{1}{\sqrt{2\pi\hbar}}\int_{V}\psi_{100}(r)\exp\left(\frac{-ipr}{\hbar}\right)dV\\
&= \frac{4\pi}{2\pi\sqrt{\hbar}}\int_{0}^a\sin\left(\frac{\pi{r}}{a}\right)\exp\left(\frac{-ipr}{\hbar}\right)dr\\
&= \frac{2}{\sqrt{\hbar{a}}}a\pi\frac{1+e^{-ipa/\hbar}}{\pi^2-(pa/\hbar)^2}\\
&= 2\pi\sqrt{\frac{a}{\hbar}}\frac{1+e^{-ipa/\hbar}}{\pi^2-(pa/\hbar)^2}
\end{align*}
So the momentum probability distribution is given by
\begin{align*}
|\phi(p)|^2 &= \frac{4\pi^2{a}}{\hbar}\frac{1+e^{ipa/\hbar}}{\pi^2-(pa/\hbar)^2}\frac{1+e^{-ipa/\hbar}}{\pi^2-(pa/\hbar)^2}\\
&= \frac{4\pi^2{a}}{\hbar}\frac{2 + e^{-ipa/\hbar} + e^{ipa/\hbar}}{(\pi^2-(pa/\hbar)^2)^2}\\
&= \frac{8\pi^2{a}}{\hbar}\frac{1 + \cos(pa/\hbar)}{(\pi^2-(pa/\hbar)^2)^2}
\end{align*}
\end{document}

