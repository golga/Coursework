\documentclass[11pt]{article}

\usepackage{latexsym}
\usepackage{amssymb}
\usepackage{amsthm}
\usepackage{enumerate}
\usepackage{amsmath}
\usepackage{cancel}
\numberwithin{equation}{section}

\setlength{\evensidemargin}{.25in}
\setlength{\oddsidemargin}{-.25in}
\setlength{\topmargin}{-.75in}
\setlength{\textwidth}{6.5in}
\setlength{\textheight}{9.5in}
\newcommand{\due}{November 30th, 2015}
\newcommand{\HWnum}{9}
\newcommand{\grad}{\bold\nabla}
\newcommand{\vecE}{\vec{E}}
\newcommand{\scrptR}{\vec{\mathfrak{R}}}
\newcommand{\kapa}{\frac{1}{4\pi\epsilon_0}}
\newcommand{\emf}{\mathcal{E}}
\newcommand{\unit}[1]{\ensuremath{\, \mathrm{#1}}}
\newcommand{\real}{\textnormal{Re}}
\newcommand{\Erf}{\textnormal{Erf}}
\newcommand{\sech}{\textnormal{sech}}
\newcommand{\scrO}{\mathcal{O}}
\newcommand{\levi}{\widetilde{\epsilon}}
\newcommand{\partiald}[2]{\ensuremath{\frac{\partial{#1}}{\partial{#2}}}}
\newcommand{\norm}[2]{\langle{#1}|{#2}\rangle}
\newcommand{\inprod}[2]{\langle{#1}|{#2}\rangle}
\newcommand{\ket}[1]{|{#1}\rangle}
\newcommand{\bra}[1]{\langle{#1}|}





\begin{document}
\begin{titlepage}
\setlength{\topmargin}{1.5in}
\begin{center}
\Huge{Physics 3320} \\
\LARGE{Principles of Electricity and Magnetism II} \\
\Large{Professor Ana Maria Rey} \\[1cm]

\huge{Homework \#\HWnum}\\[0.5cm]

\large{Joe Becker} \\
\large{SID: 810-07-1484} \\
\large{\due} 

\end{center}

\end{titlepage}



\section{Problem \#1}
For a particle of mass $m_0$ in a infinitely deep potential well of width $a$ we add a small 
perturbation given as
$$V(x) = V_0\cos\left(\frac{2\pi{x}}{a}\right)$$
we can calculate the first order correction to the energy due to this perturbation by 
calculating
$$E_n^{(1)} = \bra{\psi_n^{(0)}}V(x)\ket{\psi_n^{(0)}}$$
where $\ket{\psi_n^{(0)}}$ is the eigenfunction of the unperturbed Hamiltonian which is
$$\ket{\psi_n^{(0)}} = \sqrt{\frac{2}{a}}\sin\left(\frac{n\pi{x}}{a}\right)$$
So we can calculate the first order correction as
\begin{align*}
E_n^{(1)} = \bra{\psi_n^{(0)}}V(x)\ket{\psi_n^{(0)}} &= \frac{2V_0}{a}\int_{0}^{a}\sin^2\left(\frac{n\pi{x}}{a}\right)\cos\left(\frac{2\pi{x}}{a}\right)\\
&= \left\{\begin{array}{cl}
-V_0/2           &\textnormal{for}\ n=1\\
0                &\textnormal{for}\ n>1
        \end{array}\right.
\end{align*}
Now to calculate the second order correction to the energy we use the sum
$$E_{n}^{(2)} = \sum_{n\ne{l}}\frac{|V_{nl}|^2}{E_{l}^{(0)}-E_{n}^{(0)}}$$
where $V_{nl}$ are the matrix elements of the perturbation which we can calculate as
\begin{align*}
V_{nl} = \bra{\psi_n^{(0)}}V(x)\ket{\psi_l^{(0)}} &= \frac{2V_0}{a}\int_{0}^{a}\sin\left(\frac{n\pi{x}}{a}\right)\cos\left(\frac{2\pi{x}}{a}\right)\sin\left(\frac{l\pi{x}}{a}\right)\\
&= 0\qquad \textnormal{for}\ n\ne{l}
\end{align*}
This implies that the second order correction $E_n^{(2)}=0$.

\pagebreak

\section{Problem \#2}
\begin{enumerate}[(a)]
\item
For rigid rotor with a moment of inertia, $I$, and an electric dipole, $\mathbf{D}$, rotating
in the $xy$ plane in the presence of a uniform electric field, $\mathbf{E}$. This system adds
a perturbation of the form
$$V = -\mathbf{D}\cdot\mathbf{E} = -DE\cos\theta$$
to the rigid rotor Hamiltonian which has eigenfunctions that are the spherical harmonics, 
$Y_l^m(\theta,\phi)$. Note that the perturbation is proportional to a spherical harmonic 
given as
$$V = -2DE\sqrt{\frac{\pi}{3}}Y_{1}^{0}(\theta,\phi).$$
We note that for the first order correction to the energy is zero due to the fact that $V$ 
has odd parity and $|Y_l^{m}|^2$ has even parity therefore the integral over the solid angle
is zero. So the leading order correction is of the second order in order to calculate this
correction we find the matrix element of $V$ by
$$V_{ll'}^{mm'} = -2DE\sqrt{\frac{\pi}{3}}\int{\left(Y_{l'}^{m'}(\theta,\phi)\right)^*}Y_{1}^{0}(\theta,\phi){Y_{l}^{m}(\theta,\phi)}d\Omega$$
where we note that the integral of three spherical harmonics is related to the 
\emph{Clebsch-Gordan coefficients}, $C(l_i,m_i)$, by
$$\int{\left(Y_{l_3}^{m_3}(\theta,\phi)\right)^*}{Y_{l_1}^{m_1}(\theta,\phi)}{Y_{l_2}^{m_2}(\theta,\phi)}d\Omega = \sqrt{\frac{(2l_1+1)(2l_2+1)}{4\pi(2l_3+1)}}C(l_1,l_2,l_3|0,0,0)C(l_1,l_2,l_3|m_1,m_2,m_3)$$
For the matrix elements we have
\begin{align*}
V_{ll'}^{mm'} &= -2DE\sqrt{\frac{\pi}{3}}\int{\left(Y_{l'}^{m'}(\theta,\phi)\right)^*}Y_{1}^{0}(\theta,\phi){Y_{l}^{m}(\theta,\phi)}d\Omega\\
&= -DE\sqrt{\frac{4\pi}{3}}\sqrt{\frac{3(2l+1)}{4\pi(2l'+1)}}C(l,1,l'|0,0,0)C(l,1,l'|m,0,m')\\
&= -DE\sqrt{\frac{2l+1}{2l'+1}}C(l_1,1,l_3|0,0,0)C(l,1,l'|m,0,m')
\end{align*}
Now we note the relationship between the Clebsch-Gordan coefficients which imply that this 
integral is nonzero only when $m'=m$ and $l'=l\pm{1}$. This implies that we have the relation
\begin{align*}
V_{ll'}^{mm'} &= -DE\sqrt{\frac{2l+1}{2(l\pm1)+1}}C(l_1,1,l_3|0,0,0)C(l,1,l'|m,0,m)\\
&= -DE\sqrt{\frac{l^2-m^2}{4l^2-1}}
\end{align*}
This allows us to calculate the second order correction by using the unperturbed energy levels
given by
$$E_{l}^{(0)} = \frac{\hbar^2l(l+1)}{2I}$$ 
which implies that the nonzero terms in the sum are for $n=l\pm1$ which yields
\begin{align*}
E_{l}^{(2)} &= \sum_{l\ne{n}}\frac{|V_{ln}|^2}{E_{l}^{(0)}-E_{n}^{(0)}}\\
&= \frac{(DE)^2}{E_{l}^{(0)}-E_{l+1}^{(0)}}\frac{l^2-m^2}{4l^2-1} + \frac{(DE)^2}{E_{l}^{(0)}-E_{l-1}^{(0)}}\frac{l^2-m^2}{4l^2-1}\\
&= (DE)^2\frac{l^2-m^2}{4l^2-1}\left(\frac{1}{E_{l}^{(0)}-E_{l+1}^{(0)}}+\frac{1}{E_{l}^{(0)}-E_{l-1}^{(0)}}\right)\\
&= (DE)^2\frac{l^2-m^2}{4l^2-1}\left(-\frac{1}{E_{l}^{(0)}}\frac{l}{2}+\frac{1}{E_{l}^{(0)}}\frac{l+1}{2}\right)\\
&= \frac{(DE)^2}{2E_{l}^{(0)}}\frac{l^2-m^2}{4l^2-1}
\end{align*}
We note that the dipole added a $m$ dependence to the energy lifting the $l$ degeneracy.

\item Using the result from part (a) we can find the first order correction to the 
wave function by taking 
\begin{align*}
\ket{\psi^{(1)}} &= \sum_{l\ne{n}}\frac{|V_{ln}|^2}{(E_{l}^{(0)}-E_{n}^{(0)})^2}Y_{l}^{m}(\theta,\phi)\\
&= \frac{(DE)^2}{(E_{l}^{(0)}-E_{l+1}^{(0)})^2}\frac{l^2-m^2}{4l^2-1}Y_{l+1}^{m}(\theta,\phi) + \frac{(DE)^2}{E_{l}^{(0)}-E_{l-1}^{(0)}}\frac{l^2-m^2}{4l^2-1}Y_{l-1}^{m}(\theta,\phi)\\
&= \frac{(DE)^2}{(2E_{l}^{(0)})^2}\frac{l^2-m^2}{4l^2-1}\left(-l^2Y_{l+1}^{m}(\theta,\phi) + (l+1)^2Y_{l-1}^{m}(\theta,\phi)\right)
\end{align*}
Note that the perturbation mixed the nearest two states into the wave function.


\end{enumerate}

\pagebreak

\section{Problem \#3}
For the Hamiltonian with a potential field given by $U(x)=U_0x^4$ we have 
$$\hat{H} = \frac{\hat{p}^2}{2m} + U_0 x^4$$
we can apply \emph{Variational Principle} to estimate the ground state energy. Variational principle states that the ground state energy has an upper bound that is set by
\begin{equation}
E_0 \le \bra{\psi}\hat{H}\ket{\psi}
\label{VarPrin}
\end{equation}
where $\ket{\psi}$ is a trial wavefunction. We can pick $\ket{\psi}$ as a Gaussian of the form
$$\ket{\psi(\beta)} = Ae^{-x^2/2\beta^2}$$
where $\beta$ is the varied parameter and $A$ is the normalization factor which we calculate as
\begin{align*}
1 = \int_{-\infty}^{\infty}\inprod{\psi(\beta)}{\psi(\beta)} &= A^2\int_{-\infty}^{\infty}e^{-x^2/\beta^2}\\
&= A^2\beta\sqrt{\pi}\\
&\Downarrow\\
A &= \left(\beta^2{\pi}\right)^{-1/4}
\end{align*}
So now we can set an upper bound on the ground state energy by equation \ref{VarPrin}
\begin{align*}
E(\beta) = \bra{\psi}\hat{H}\ket{\psi} &= \left(\beta^2{\pi}\right)^{-1/2}\int_{-\infty}^{\infty}dxe^{- x^2/2\beta^2}\left(-\frac{\hbar^2}{2m}\frac{\partial^2}{\partial x^2} + U_0 x^4\right)e^{-x^2/2\beta^2}\\
&= \left(\beta^2{\pi}\right)^{-1/2}\int_{-\infty}^{\infty}dxe^{-x^2/2\beta^2}\left(-\frac{\hbar^2}{2m}\frac{\partial^2}{\partial x^2}e^{-x^2/2\beta^2} + U_0 x^4e^{-x^2/2\beta^2}\right)\\
&= \left(\beta^2{\pi}\right)^{-1/2}\int_{-\infty}^{\infty}dxe^{-x^2/2\beta^2}\left(\frac{\hbar^2}{2m}\frac{\partial}{\partial x}\left(\frac{x}{\beta^2}\right)e^{-x^2/2\beta^2} + U_0 x^4e^{-x^2/2\beta^2}\right)\\
&= \left(\beta^2{\pi}\right)^{-1/2}\int_{-\infty}^{\infty}dxe^{-x^2/2\beta^2}\left(\frac{\hbar^2}{2m}\left(\frac{1}{\beta^2}\right)e^{-x^2/2\beta^2} - \frac{\hbar^2}{2m}\left(\frac{x}{\beta^2}\right)^2 e^{-x^2/2\beta^2} + U_0 x^4e^{-x^2/2\beta^2}\right)\\
&= \left(\beta^2{\pi}\right)^{-1/2}\int_{-\infty}^{\infty}dxe^{-x^2/\beta^2}\left(\frac{\hbar^2}{2m\beta^2} - \frac{\hbar^2}{2m\beta^4}x^2 + U_0 x^4\right)
\end{align*}
Now we have three integrals involving the Gaussian $e^{-x^2/\beta^2}$. This allows us to use the fact that for even powers of $x$ we have
\begin{equation} 
\left(\beta^2{\pi}\right)^{-1/2}\int_{-\infty}^{\infty} x^{2n}e^{-x^2/\beta^2} = \left(\frac{\beta^2}{2}\right)^{n}(2n-1)!!
\label{Gauss}
\end{equation} 
Note that $n!! = n(n-2)(n-4)...$. Note for $x^2$ we have $n=1$ and for $x^4$ we have $n=2$ so equation \ref{Gauss} yields
\begin{align*}
\left(\beta^2{\pi}\right)^{-1/2}\int_{-\infty}^{\infty} x^{2}e^{-x^2/\beta^2} &= \frac{\beta^2}{2}\\
\left(\beta^2{\pi}\right)^{-1/2}\int_{-\infty}^{\infty} x^{4}e^{-x^2/\beta^2} &= \frac{3\beta^4}{4}
\end{align*}
Note the integral with the constant $\hbar^2/2m\beta^2$ is just the constant due to normalization. So the integral becomes
\begin{align*}
E(\beta) &= \frac{\hbar^2}{2m\beta^2} - \frac{\hbar^2}{2m\beta^4}\frac{\beta^2}{2} + U_0\frac{3\beta^4}{4}\\
&= \frac{\hbar^2}{2m\beta^2} - \frac{\hbar^2}{4m\beta^2} + U_0\frac{3\beta^4}{4}\\
&= \frac{\hbar^2}{4m\beta^2} + U_0\frac{3\beta^4}{4}
\end{align*}
Now we just need to find $\beta_0$ that minimizes $E(\beta)$ by
\begin{align*}
0 = \frac{dE(\beta)}{d\beta} &= -\frac{\hbar^2}{2m\beta_0^3} + 3U_0\beta_0^3\\
&\Downarrow\\
\frac{\hbar^2}{2m\beta_0^3} &= 3U_0\beta_0^3\\
&\Downarrow\\
\beta_0 &= \left(\frac{\hbar^2}{6mU_0}\right)^{1/6}
\end{align*}
Now we replace we can find $E(\beta_0)$ by
\begin{align*}
E(\beta_0) &= \frac{\hbar^2}{4m}\frac{1}{\beta_0^2} + U_0\frac{3\beta_0^4}{4}\\
&= \frac{\hbar^2}{4m}\left(\frac{6mU_0}{\hbar^2}\right)^{1/3} + U_0\frac{3}{4} \left(\frac{\hbar^2}{6mU_0}\right)^{2/3}\\
&= U_0^{1/3}\left(\left(\frac{3\hbar^4}{32m^2}\right)^{1/3} + \frac{3}{4}\left(\frac{\hbar^2}{6m}\right)^{2/3}\right)\\
&= U_0^{1/3}\left(\frac{\hbar^2}{2m}\right)^{2/3}\left(\left(\frac{3}{8}\right)^{1/3} + \frac{3}{4}\left(\frac{1}{3}\right)^{2/3}\right)\\
\end{align*}
So we have set the upper bound on the ground state energy as
$$E_0\le(1.082)U_0^{1/3}\left(\frac{\hbar^2}{2m}\right)^{2/3}$$

\end{document}

