\documentclass[11pt]{article}

\usepackage{latexsym}
\usepackage{amssymb}
\usepackage{amsthm}
\usepackage{enumerate}
\usepackage{amsmath}
\usepackage{cancel}
\numberwithin{equation}{section}

\setlength{\evensidemargin}{.25in}
\setlength{\oddsidemargin}{-.25in}
\setlength{\topmargin}{-.75in}
\setlength{\textwidth}{6.5in}
\setlength{\textheight}{9.5in}
\newcommand{\due}{November 18th, 2015}
\newcommand{\HWnum}{8}
\newcommand{\grad}{\bold\nabla}
\newcommand{\vecE}{\vec{E}}
\newcommand{\scrptR}{\vec{\mathfrak{R}}}
\newcommand{\kapa}{\frac{1}{4\pi\epsilon_0}}
\newcommand{\emf}{\mathcal{E}}
\newcommand{\unit}[1]{\ensuremath{\, \mathrm{#1}}}
\newcommand{\real}{\textnormal{Re}}
\newcommand{\Erf}{\textnormal{Erf}}
\newcommand{\sech}{\textnormal{sech}}
\newcommand{\scrO}{\mathcal{O}}
\newcommand{\levi}{\widetilde{\epsilon}}
\newcommand{\partiald}[2]{\ensuremath{\frac{\partial{#1}}{\partial{#2}}}}
\newcommand{\norm}[2]{\langle{#1}|{#2}\rangle}
\newcommand{\inprod}[2]{\langle{#1}|{#2}\rangle}
\newcommand{\ket}[1]{|{#1}\rangle}
\newcommand{\bra}[1]{\langle{#1}|}





\begin{document}
\begin{titlepage}
\setlength{\topmargin}{1.5in}
\begin{center}
\Huge{Physics 3320} \\
\LARGE{Principles of Electricity and Magnetism II} \\
\Large{Professor Ana Maria Rey} \\[1cm]

\huge{Homework \#\HWnum}\\[0.5cm]

\large{Joe Becker} \\
\large{SID: 810-07-1484} \\
\large{\due} 

\end{center}

\end{titlepage}



\section{Problem \#1}
For the addition of the angular momenta for two particles with the quantum numbers 
$l_1=l_2=1$, $m_1=0,\pm1$, and $m_2=0, \pm1$. For the net states $\ket{l=2,m=0}$ and
$\ket{l=2,m=-1}$ we can 
write it as a linear combination of $\ket{m_1,m_2}$ by using the ladder operator given by
$$L_{\pm}\ket{lm} = \sqrt{(l\mp{m})(l\pm{m}+1)}\ket{lm-1}$$ 
where we take the total angular momentum ladder operator is 
$$L_{\pm} = L_{1\pm}+L_{2\pm}$$
This allows us to act on the state $\ket{l=2,m=-2} = \ket{m_1=-1,m_2=-1}$ which yields
\begin{align*}
L_{+}\ket{l=2,m=-2} &= (L_{1+}+L_{2+})\ket{m_1=-1,m_2=-1}\\
&\Downarrow\\
\sqrt{(2+2)(2-2+1)}\ket{l=2,m=-1} &= \sqrt{(1+1)(1-1+1)}\ket{m_1=0,m_2=-1} + \sqrt{(1+1)(1-1+1)}\ket{m_1=-1,m_2=0}\\
2\ket{l=2,m=-1} &= \sqrt{2}\ket{m_1=0,m_2=-1} + \sqrt{2}\ket{m_1=-1,m_2=0}\\
\ket{l=2,m=-1} &= \frac{\sqrt{2}}{2}\left(\frac{}{}\ket{m_1=0,m_2=-1} + \ket{m_1=-1,m_2=0}\right)
\end{align*}
Which allows us to act $L_{+}$ again as
\begin{align*}
L_{+}\ket{l=2,m=-1} &= (L_{1+}+L_{2+})\frac{\sqrt{2}}{2}\left(\ket{m_1=0,m_2=-1} + \ket{m_1=-1,m_2=0}\right)\\
&\Downarrow\\
\sqrt{(2+1)(2-1+1)}\ket{l=2,m=0} &= \frac{\sqrt{2}}{2}L_{1+}\left(\ket{m_1=0,m_2=-1} + \ket{m_1=-1,m_2=0}\right)\\ 
&\qquad\qquad + \frac{\sqrt{2}}{2}L_{2+}\left(\ket{m_1=0,m_2=-1} + \ket{m_1=-1,m_2=0}\right)\\
\sqrt{6}\ket{l=2,m=0} &= \frac{\sqrt{2}}{2}\left(\frac{}{}\sqrt{2}\ket{m_1=1,m_2=-1} + \sqrt{2}\ket{m_1=0,m_2=0}\right)\\ 
&\qquad\qquad + \frac{\sqrt{2}}{2}\left(\frac{}{}\ket{m_1=0,m_2=-1} + \sqrt{2}\ket{m_1=-1,m_2=0}\right)\\
\ket{l=2,m=0} &= \frac{\sqrt{6}}{6}\left(\frac{}{}\ket{m_1=1,m_2=-1} + \ket{m_1=-1,m_2=1} + 2\ket{m_1=0,m_2=0}\right)\\
&= \frac{\sqrt{6}}{6}\ket{m_1=1,m_2=-1} + \frac{\sqrt{6}}{6}\ket{m_1=-1,m_2=1} + \frac{\sqrt{6}}{3}\ket{m_1=0,m_2=0}
\end{align*}

\pagebreak

\section{Problem \#2}
\begin{enumerate}[(a)]
\item To calculate the matrix elements for a $l=1$ state of $\hat{l}_x$, $\hat{l}_y$, and
$\hat{l}_{\pm}$ which are the ladder operators defined as
$$\hat{l}_{\pm} \equiv \hat{l}_x \pm i\hat{l}_y$$
which allows us to say that
\begin{align*}
\hat{l}_x &= \frac{1}{2}\left(\hat{l}_{+}+\hat{l}_{-}\right)\\
\hat{l}_y &= \frac{1}{2i}\left(\hat{l}_{+}-\hat{l}_{-}\right)
\end{align*}
So the matrix elements for $l=1$ we have
\begin{align*}
\bra{1m'}\hat{l}_x\ket{1m} &= \sum_{m'=-1}^{1}\sum_{m=-1}^{1}\frac{1}{2}\bra{1m'}\left(\hat{l}_{+}+\hat{l}_{-}\right)\ket{1m}\\
&= \sum_{m'=-1}^{1}\sum_{m=-1}^{1}\frac{\hbar}{2}\bra{1m'}\left(\sqrt{(1-m)(1+m+1)}\ket{1m+1}\right.\left.+\sqrt{(1+m)(1-m+1)}\ket{1m-1}\right)\\
&= \sum_{m'=-1}^{1}\sum_{m=-1}^{1}\frac{\hbar}{2}\left(\sqrt{(1-m)(2+m)}\inprod{1m'}{1m+1}+\sqrt{(1+m)(2-m)}\inprod{1m'}{1m-1}\right)
\end{align*}
Which if we manually calculate each entry we have
$$\hat{l}_x = \hbar\frac{\sqrt{2}}{2}\left(\begin{array}{ccc}
              0        &1     &0\\
              1        &0     &1\\
              0        &1     &0
              \end{array}\right)$$
Now we repeat the process with $\hat{l}_y$
\begin{align*}
\bra{1m'}\hat{l}_y\ket{1m} &= \sum_{m'=-1}^{1}\sum_{m=-1}^{1}\frac{1}{2i}\bra{1m'}\left(\hat{l}_{+}-\hat{l}_{-}\right)\ket{1m}\\
&= \sum_{m'=-1}^{1}\sum_{m=-1}^{1}\frac{\hbar}{2i}\left(\sqrt{(1-m)(2+m)}\inprod{1m'}{1m+1}-\sqrt{(1+m)(2-m)}\inprod{1m'}{1m-1}\right)
\end{align*}
Which gives 
$$\hat{l}_y = \hbar\frac{\sqrt{2}}{2i}\left(\begin{array}{ccc}
              0        &-1    &0\\
              1        &0     &-1\\
              0        &1     &0
              \end{array}\right)$$
Which allows us to calculate the ladder operators by definition
\begin{align*}
\hat{l}_{+} &= \hbar\sqrt{2}\left(\begin{array}{ccc}
              0        &0     &0\\
              1        &0     &0\\
              0        &1     &0
              \end{array}\right)\\
\hat{l}_{-} &= \hbar\sqrt{2}\left(\begin{array}{ccc}
              0        &1     &0\\
              0        &0     &1\\
              0        &0     &0
              \end{array}\right)
\end{align*}

\item We can use the above result to find the state, $\psi_{lx}$ with $l_x=0$ which implies that
$$\hat{l}_x\psi_{lx} = 0$$
which we can calculate as
\begin{align*}
\hat{l}_x\psi_{lx} &= \hbar\frac{\sqrt{2}}{2}\left(\begin{array}{ccc}
              0        &1     &0\\
              1        &0     &1\\
              0        &1     &0
              \end{array}\right)
              \left(\begin{array}{c}a\\b\\c\end{array}\right)\\
&\Downarrow\\
0 &= \hbar\frac{\sqrt{2}}{2}\left(\begin{array}{c}b\\a+c\\b\end{array}\right)
\end{align*}
This implies that $b=0$ and $c=-a$ so our state is 
$$\psi_{lx} = a\left(\begin{array}{c}1\\0\\-1\end{array}\right)$$
where we can find $a$ by normalizing 
\begin{align*}
|\psi_{lx}|^2 = 1 &= a^2(1+1)\\
&\Downarrow\\
a &= \frac{\sqrt{2}}{2}
\end{align*}
So we have the state
$$\psi_{lx} = \frac{\sqrt{2}}{2}\left(\begin{array}{c}1\\0\\-1\end{array}\right)$$
\end{enumerate}

\pagebreak

\section{Problem \#3}
\begin{enumerate}[(a)]
\item For an eigenfunction, $\ket{m}$, of the orbital angular momentum operator $\hat{l}_z$.
We can find expectation values of the operators $\hat{l}_x$ and $\hat{l}_y$ by using the 
which allows us to calculate 
\begin{align*}
\left\langle\hat{l}_{y}\right\rangle &= \bra{m}\frac{1}{2i}\left(\hat{l}_{+}-\hat{l}_{-}\right)\ket{m}\\
&= \frac{1}{2i}\left(\bra{m}\hat{l}_{+}\ket{m}-\bra{m}\hat{l}_{-}\ket{m}\right)\\
&= \frac{1}{2i}\left(\hbar{A}\inprod{m}{m+1}-\hbar{B}\inprod{m}{m-1}\right)\\
&= 0
\end{align*}
and
\begin{align*}
\left\langle\hat{l}_{x}\right\rangle &= \bra{m}\frac{1}{2}\left(\hat{l}_{+}+\hat{l}_{-}\right)\ket{m}\\
&= \frac{1}{2}\left(\bra{m}\hat{l}_{+}\ket{m}+\bra{m}\hat{l}_{-}\ket{m}\right)\\
&= \frac{1}{2}\left(\hbar{A}\inprod{m}{m+1}+\hbar{B}\inprod{m}{m-1}\right)\\
&= 0
\end{align*}

\item
To calculate the expectation value of the operator $\hat{l}_{x}\hat{l}_{y}+\hat{l}_{y}\hat{l}_{x}$ 
in the state $\ket{m}$ we use ladder operators to find
\begin{align*}
\hat{l}_{x}\hat{l}_{y}+\hat{l}_{y}\hat{l}_{x} &= \frac{1}{4i}\left(\frac{}{}\left(\hat{l}_{+}+\hat{l}_{-}\right)\left(\hat{l}_{+}-\hat{l}_{-}\right)+\left(\hat{l}_{+}-\hat{l}_{-}\right)\left(\hat{l}_{+}+\hat{l}_{-}\right)\right)\\
&= \frac{1}{4i}\left(\hat{l}_{+}\hat{l}_{+} -\hat{l}_{+}\hat{l}_{-} +\hat{l}_{-}\hat{l}_{+} -\hat{l}_{-}\hat{l}_{-} +\hat{l}_{+}\hat{l}_{+} +\hat{l}_{+}\hat{l}_{-} -\hat{l}_{-}\hat{l}_{+} -\hat{l}_{-}\hat{l}_{-}\right)\\
&= \frac{1}{2i}\left(\hat{l}_{+}\hat{l}_{+} - \hat{l}_{-}\hat{l}_{-}\right)
\end{align*}
Note the cross terms cancel so neither operator will bring the state back to $\ket{m}$ 
which implies that
$$\left\langle\hat{l}_{x}\hat{l}_{y}+\hat{l}_{y}\hat{l}_{x}\right\rangle = 0$$

\item Note for the operators $\hat{l}^2_{x}$ and $\hat{l}^2_{y}$ we have
\begin{align*}
\hat{l}_x^2 &= \frac{1}{4}\left(\hat{l}_{+}+\hat{l}_{-}\right)^2\\
&= \frac{1}{4}\left(\hat{l}_{+}\hat{l}_{+} + \hat{l}_{-}\hat{l}_{-} + \hat{l}_{+}\hat{l}_{-} +\hat{l}_{-}\hat{l}_{+}\right)
\end{align*}
and
\begin{align*}
\hat{l}_y^2 &= -\frac{1}{4}\left(\hat{l}_{+}-\hat{l}_{-}\right)^2\\
&= -\frac{1}{4}\left(\hat{l}_{+}\hat{l}_{+} + \hat{l}_{-}\hat{l}_{-} - \hat{l}_{+}\hat{l}_{-} - \hat{l}_{-}\hat{l}_{+}\right)
\end{align*}+
We note that only the cross terms have nonzero contributions to the expectation values of
these operators. So for a state in $\ket{lm}$ we have
\begin{align*}
\left\langle\hat{l}_x^2\right\rangle &= \frac{1}{4}\bra{lm}\left(\hat{l}_{+}\hat{l}_{+} + \hat{l}_{-}\hat{l}_{-} + \hat{l}_{+}\hat{l}_{-} + \hat{l}_{-}\hat{l}_{+}\right)\ket{lm}\\
&= \frac{1}{4}\left(\bra{lm}\hat{l}_{+}\hat{l}_{-}\ket{lm} + \bra{lm}\hat{l}_{-}\hat{l}_{+}\ket{lm}\right)\\
&= \frac{\hbar}{4}\left(\sqrt{(l+m)(l-m+1)}\bra{lm}\hat{l}_{+}\ket{lm-1} + \sqrt{(l-m)(l+m+1)}\bra{lm}\hat{l}_{-}\ket{lm+1}\right)\\
&= \frac{\hbar^2}{4}\left(\sqrt{(l+m)(l-m+1)}\sqrt{(l-m+1)(l+m+1-1)}\bra{lm}\ket{lm}\right.\\
&\qquad\left. + \sqrt{(l-m)(l+m+1)}\sqrt{(l+m+1)(l-m-1+1)}\bra{lm}\ket{lm}\right)\\
&= \frac{\hbar^2}{2}\left(l(l+1)-m^2\right)
\end{align*}
and
\begin{align*}
\left\langle\hat{l}_y^2\right\rangle &= -\frac{1}{4}\bra{lm}\left(\hat{l}_{+}\hat{l}_{+} + \hat{l}_{-}\hat{l}_{-} - \hat{l}_{+}\hat{l}_{-} - \hat{l}_{-}\hat{l}_{+}\right)\ket{lm}\\
&= \frac{1}{4}\left(\bra{lm}\hat{l}_{+}\hat{l}_{-}\ket{lm} + \bra{lm}\hat{l}_{-}\hat{l}_{+}\ket{lm}\right)\\
&= \frac{\hbar^2}{2}\left(l(l+1)-m^2\right)
\end{align*}
\end{enumerate}

\pagebreak

\section{Problem \#4}
For a wave function that describes a planer rotor 
$$\psi(\varphi) = A\sin^2\varphi$$ 
we can write it as eigenfunctions of the $\hat{l}_z$ operator which are of the form 
$e^{im\varphi}$ by
\begin{align*}
\psi(\varphi) = A\sin^2\varphi &= A\left(\frac{}{}e^{i\phi}-e^{i\phi}\right)^2\\
&= A\left(\frac{}{}e^{i2\phi}+e^{i2\phi}-1\right)\\
&= A\left(\frac{}{}\ket{m=2}+\ket{m=-2}-\ket{m=0}\right)
\end{align*}
Where we can find $A$ by normalizing
\begin{align*}
|\psi(\varphi)|^2 = 1 &= A^2\left(\frac{}{}\inprod{m=2}{m=2}+\inprod{m=-2}{m=-2}-\inprod{m=0}{m=0}\right)\\
&= A^2(1+1-1)\\
&\Downarrow\\
A &= 1
\end{align*}
So we have the expectation values
\begin{align*}
\left\langle\hat{L}_z\right\rangle &= \bra{m=2}\hat{L}_z\ket{m=2}+\bra{m=-2}\hat{L}_z\ket{m=-2}-\bra{m=0}\hat{L}_z\ket{m=0}\\
&= 2\hbar - 2\hbar - 1\hbar = -\hbar
\end{align*}
and
\begin{align*}
\left\langle\hat{L}_z^2\right\rangle &= \bra{m=2}\hat{L}^2_z\ket{m=2}+\bra{m=-2}\hat{L}^2_z\ket{m=-2}-\bra{m=0}\hat{L}^2_z\ket{m=0}\\
&= 4\hbar^2 + 4\hbar^2 + 1\hbar^2 = 9\hbar
\end{align*}



\end{document}

