\documentclass[11pt]{article}

\usepackage{latexsym}
\usepackage{amssymb}
\usepackage{amsthm}
\usepackage{enumerate}
\usepackage{amsmath}
\usepackage{cancel}
\numberwithin{equation}{section}

\setlength{\evensidemargin}{.25in}
\setlength{\oddsidemargin}{-.25in}
\setlength{\topmargin}{-.75in}
\setlength{\textwidth}{6.5in}
\setlength{\textheight}{9.5in}
\newcommand{\due}{September 30th, 2015}
\newcommand{\HWnum}{3}
\newcommand{\grad}{\bold\nabla}
\newcommand{\vecE}{\vec{E}}
\newcommand{\scrptR}{\vec{\mathfrak{R}}}
\newcommand{\kapa}{\frac{1}{4\pi\epsilon_0}}
\newcommand{\emf}{\mathcal{E}}
\newcommand{\unit}[1]{\ensuremath{\, \mathrm{#1}}}
\newcommand{\real}{\textnormal{Re}}
\newcommand{\Erf}{\textnormal{Erf}}
\newcommand{\sech}{\textnormal{sech}}
\newcommand{\scrO}{\mathcal{O}}
\newcommand{\levi}{\widetilde{\epsilon}}
\newcommand{\partiald}[2]{\ensuremath{\frac{\partial{#1}}{\partial{#2}}}}
\newcommand{\norm}[2]{\langle{#1}|{#2}\rangle}
\newcommand{\inprod}[2]{\langle{#1}|{#2}\rangle}
\newcommand{\average}[1]{\left\langle{#1}\right\rangle}
\newcommand{\ket}[1]{|{#1}\rangle}
\newcommand{\bra}[1]{\langle{#1}|}
\newcommand{\Resid}[2]{\ensuremath{\textnormal{Res}\left[{#1},{#2}\right]}}





\begin{document}
\begin{titlepage}
\setlength{\topmargin}{1.5in}
\begin{center}
\Huge{Physics 3310} \\
\LARGE{Principles of Electricity and Magnetism 1} \\
\Large{Professor Thomas R. Schibli} \\[1cm]

\huge{Homework \#\HWnum}\\[0.5cm]

\large{Joe Becker} \\
\large{SID: 810-07-1484} \\
\large{\due} 

\end{center}

\end{titlepage}



\section{Problem \#1}
For the operator given by
$$\hat{A} = -i\hbar\partiald{}{\varphi} + a\sin\varphi$$
we assume there exists an eigenfunction which is a solution to the equation
$$(\hat{A} - A)\psi(\varphi) = 0$$
where $A$ is the eigenvalue of $\hat{A}$. We can see that for the given $\hat{A}$ we have
\begin{align*}
\left(-i\hbar\partiald{}{\varphi} + a\sin\varphi\right)\psi(\varphi) - A\psi(\varphi) &= 0\\
-i\hbar\partiald{}{\varphi}\psi(\varphi) + a\sin\varphi\psi(\varphi) - A\psi(\varphi) &= 0\\
&\Downarrow\\
-i\hbar\partiald{}{\varphi}\psi(\varphi) &= - a\sin\varphi\psi(\varphi) + A\psi(\varphi)\\
\partiald{}{\varphi}\psi(\varphi) &=i\frac{A-a\sin\varphi}{\hbar}\psi(\varphi)
\end{align*}
This equation implies that
$$\psi(\varphi) = C\exp\left(i\frac{A}{\hbar}\varphi\right)\exp\left(i\frac{a\cos\varphi}{\hbar}\right)$$
We can find the normalization constant, $C$, by normalizing
\begin{align*}
1 = \int_{0}^{2\pi}\psi^*(\varphi)\psi(\varphi)d\phi &= C^2\int_{0}^{2\pi}\exp\left(-i\frac{A}{\hbar}\varphi\right)\exp\left(i\frac{A}{\hbar}\varphi\right)\exp\left(-i\frac{a\cos\varphi}{\hbar}\right)\exp\left(i\frac{a\cos\varphi}{\hbar}\right)d\varphi\\
&= C^2\int_{0}^{2\pi}d\varphi = C^22\pi\\
&\Downarrow\\
C &= \frac{1}{\sqrt{2\pi}}
\end{align*}
Therefore our normalized eigenfunction is 
$$\psi(\varphi) = \frac{1}{\sqrt{2\pi}}\exp\left(i\frac{A}{\hbar}\varphi\right)\exp\left(i\frac{a\cos\varphi}{\hbar}\right)$$
where we apply the periodic boundary condition that $\psi(\varphi+2\pi)=\psi(\varphi)$ we 
note that the second term satisfies this requirement and the first term is periodic for
$$A = n\hbar\qquad n=1,2,3,...$$
so we finally have the eigenfunction 
$$\psi(\varphi) = \frac{1}{\sqrt{2\pi}}\exp\left(in\varphi\right)\exp\left(i\frac{a\cos\varphi}{\hbar}\right)$$
with the eigenvalue 
$$A = n\hbar\qquad n=1,2,3...$$
\pagebreak

\section{Problem \#2}
\begin{enumerate}[(a)]
\item For a general potential $U(x)$ the Hamiltonian operator in three-dimensions, $\hat{H}$,
becomes
$$\hat{H} = \frac{\mathbf{p}^2}{2m}+U(x)$$
where in three-dimensions 
$$\mathbf{p}^2 = \hat{p}^2_x+\hat{p}^2_y+\hat{p}^2_z$$
For this Hamiltonian we can calculate the commutator $[\hat{H},y]$ by
\begin{align*}
\left[\hat{H},y\right] &= \frac{1}{2m}\left(\cancelto{0}{[\hat{p}^2_x,y]} + [\hat{p}^2_y,y] + \cancelto{0}{[\hat{p}^2_z,y]}\right)+\cancelto{0}{[U(x),y]}\\
&= \frac{1}{2m}[\hat{p}^2_y,y]\\
&= \frac{1}{2m}\left(\hat{p}_y[\hat{p}_y,y] +[\hat{p}_y,y]\hat{p}_y\right)\\
&= \frac{1}{2m}\left(\hat{p}_y(-i\hbar) + (-i\hbar)\hat{p}_y\right)\\
&= \frac{-i\hbar}{m}\hat{p}_y
\end{align*}

\item For the same Hamiltonian we calculate
\begin{align*}
[\hat{H},\hat{p}_x] &= \frac{1}{2m}\left(\cancelto{0}{[\hat{p}^2_x,\hat{p}_x]} + \cancelto{0}{[\hat{p}^2_y,\hat{p}_x]} + \cancelto{0}{[\hat{p}^2_z,\hat{p}_x]}\right)+[U(x),\hat{p}_x]\\
&= [U(x),\hat{p}_x]
\end{align*}
We can calculate this commutation by multiplying by a function $f$
\begin{align*}
[U(x),\hat{p}_x]f(x) &= U(x)\hat{p}_xf(x) - \hat{p}_x(U(x)f(x))\\
&= U(x)(-i\hbar)\partiald{}{d}f(x) - (-i\hbar)\partiald{}{x}(U(x)f(x))\\
&= -i\hbar U(x)\partiald{f(x)}{d} + i\hbar\partiald{f(x)}{x}U(x) +i\hbar\partiald{U(x)}{x}f(x)\\
&\Downarrow\\
[U(x),\hat{p}_x]\cancel{f(x)} &= i\hbar\partiald{U(x)}{x}\cancel{f(x)}\\
[U(x),\hat{p}_x] &= i\hbar\partiald{U(x)}{x}
\end{align*}
So we have
$$[\hat{H},\hat{p}_x] = i\hbar\partiald{U(x)}{x}$$

\item Next we calculate how $\hat{p}_x^2$ commutes with the Hamiltonian
\begin{align*}
[\hat{H},\hat{p}_x^2] &= \frac{1}{2m}\cancel{([\hat{p}^2_x,\hat{p}_x^2] + [\hat{p}^2_y,\hat{p}_x^2] + [\hat{p}^2_z,\hat{p}_x^2])}+[U(x),\hat{p}_x^2]\\
&= [U(x),\hat{p}_x^2] = [U(x),\hat{p}_x]\hat{p}_x + \hat{p}_x[U(x),\hat{p}_x]\\
&= i\hbar\partiald{U(x)}{x}\hat{p}_x + \hat{p}_xi\hbar\partiald{U(x)}{x}\\
&= i\hbar\partiald{U(x)}{x}\hat{p}_x + i\hbar(-i\hbar)\partiald{}{x}\partiald{U(x)}{x}\\
&= i\hbar\partiald{U(x)}{x}\hat{p}_x + \hbar^2\partiald{^2U(x)}{x^2}
\end{align*}



\end{enumerate}

\pagebreak

\section{Problem \#3}
In order to express the translation operator $\hat{T}_{\mathbf{a}}$ which is defined as a 
spacial translation $\mathbf{a}$ given by
$$\hat{T}_{\mathbf{a}}\psi(\mathbf{r}) = \psi(\mathbf{r}+\mathbf{a})$$
in terms of the momentum operator $\mathbf{\hat{p}}$ by noting that the expansion
$$\psi(\mathbf{r}+\mathbf{a}) = \sum_{n=0}^{\infty}\frac{1}{n!}\grad^n(\psi(\mathbf{r}))\cdot\mathbf{a}^n$$
We can place this into terms of $\mathbf{P}$ by noting that
$$\mathbf{p} = -i\hbar\grad$$
which implies that
\begin{align*}
\psi(\mathbf{r}+\mathbf{a}) &= \sum_{n=0}^{\infty}\frac{1}{n!}\grad^n(\psi(\mathbf{r}))\cdot\mathbf{a}^n\\
&= \sum_{n=0}^{\infty}\frac{1}{n!}\mathbf{a}^n\cdot\left(\frac{i}{\hbar}\mathbf{p}\right)^n\psi(\mathbf{r})\\
&= \sum_{n=0}^{\infty}\frac{1}{n!}\left(\frac{i}{\hbar}\mathbf{a}\cdot\mathbf{p}\right)^n\psi(\mathbf{r})
\end{align*}
We note that we have a Taylor expansion of the exponential
$$\sum_{n=0}^{\infty}\frac{1}{n!}\left(\frac{i}{\hbar}\mathbf{a}\cdot\mathbf{p}\right)^n = \exp\left(i\frac{\mathbf{a}\cdot\mathbf{p}}{\hbar}\right)$$
So we see that we have found that
$$\psi(\mathbf{r}+\mathbf{a}) = \exp\left(i\frac{\mathbf{a}\cdot\mathbf{p}}{\hbar}\right)\psi(\mathbf{r})$$
So the translation operator in terms of the momentum operator is given by
$$\hat{T}_{\mathbf{a}}= \exp\left(i\frac{\mathbf{a}\cdot\mathbf{p}}{\hbar}\right)$$

\pagebreak

\section{Problem \#4}
For a free particle in one-dimension in the presence of an impenetrable wall (or barrier) 
which is given by the potential
$$U(x) = \left\{\begin{array}{cl}
                  0       &x>0\\ 
                  \infty  &x<0
          \end{array}\right.$$
We note that in the region $x<0$ there is no wave-function because we have an infinite 
potential therefore we have $\psi(x)=0$ for $x\le0$. And away from the barrier we have a free
particle given by
$$\psi(x) = Ae^{ikx} + Be^{-ikx}$$
We note that in general the term $kx$ is found by the integral
$$k(x) = -\int_{x_0}^{x}\frac{1}{\hbar}\sqrt{2\mu(E-U(x'))}dx'$$
but for $U(x)=0$ we have a constant integral where 
$$k = \frac{\sqrt{2\mu{E}}}{\hbar}$$
end
Next we apply the continuity condition where $\psi(0) = 0$. We note that due to the infinte
change in potential we cannot have a continuous derivative. So we see that
\begin{align*}
\psi(0) = 0 &= Ae^{ik0} + Be^{-ik0}\\
\psi(0) = 0 &= A + B\\
&\Downarrow\\
A &= -B
\end{align*}
So our solution becomes
$$\psi(x) = A\left(\exp\left(\frac{\sqrt{2\mu{E}}}{\hbar}x\right) - \exp\left(-\frac{\sqrt{2\mu{E}}}{\hbar}x\right)\right)$$
Now we can normalize our $\psi(x)$ to a delta function in $E$. By calculating 
\begin{align*}
\delta(E-E') &= A^2\int_{0}^{E}\left(\exp\left(-\frac{\sqrt{2\mu{E'}}}{\hbar}x\right) - \exp\left(\frac{\sqrt{2\mu{E'}}}{\hbar}x\right)\right)\left(\exp\left(\frac{\sqrt{2\mu{E'}}}{\hbar}x\right) - \exp\left(-\frac{\sqrt{2\mu{E'}}}{\hbar}x\right)\right)dE'\\
&= A^2\int_{0}^{E}2\exp\left(\frac{\sqrt{2\mu{E'}}}{\hbar}x\right)\exp\left(-\frac{\sqrt{2\mu{E'}}}{\hbar}x\right)dE\\
&= A^2\int_{0}^{E}2dE\\
&= 2A^2E\\
&\Downarrow\\
A &= \frac{1}{2E\delta(E-E')}
\end{align*}
So we have
$$\psi(x) = \frac{1}{2E\delta(E)}\left(\exp\left(\frac{\sqrt{2\mu{E}}}{\hbar}x\right) - \exp\left(-\frac{\sqrt{2\mu{E}}}{\hbar}x\right)\right)$$

\end{document}

