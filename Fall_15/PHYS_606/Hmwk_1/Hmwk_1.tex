\documentclass[11pt]{article}

\usepackage{latexsym}
\usepackage{amssymb}
\usepackage{amsthm}
\usepackage{enumerate}
\usepackage{amsmath}
\usepackage{cancel}
\numberwithin{equation}{section}

\setlength{\evensidemargin}{.25in}
\setlength{\oddsidemargin}{-.25in}
\setlength{\topmargin}{-.75in}
\setlength{\textwidth}{6.5in}
\setlength{\textheight}{9.5in}
\newcommand{\due}{September 16th, 2015}
\newcommand{\HWnum}{1}
\newcommand{\grad}{\bold\nabla}
\newcommand{\vecE}{\vec{E}}
\newcommand{\scrptR}{\vec{\mathfrak{R}}}
\newcommand{\kapa}{\frac{1}{4\pi\epsilon_0}}
\newcommand{\emf}{\mathcal{E}}
\newcommand{\unit}[1]{\ensuremath{\, \mathrm{#1}}}
\newcommand{\real}{\textnormal{Re}}
\newcommand{\Erf}{\textnormal{Erf}}
\newcommand{\sech}{\textnormal{sech}}
\newcommand{\scrO}{\mathcal{O}}
\newcommand{\levi}{\widetilde{\epsilon}}
\newcommand{\partiald}[2]{\ensuremath{\frac{\partial{#1}}{\partial{#2}}}}
\newcommand{\norm}[2]{\langle{#1}|{#2}\rangle}
\newcommand{\inprod}[2]{\langle{#1}|{#2}\rangle}
\newcommand{\ket}[1]{|{#1}\rangle}
\newcommand{\bra}[1]{\langle{#1}|}





\begin{document}
\begin{titlepage}
\setlength{\topmargin}{1.5in}
\begin{center}
\Huge{Physics 3320} \\
\LARGE{Principles of Electricity and Magnetism II} \\
\Large{Professor Ana Maria Rey} \\[1cm]

\huge{Homework \#\HWnum}\\[0.5cm]

\large{Joe Becker} \\
\large{SID: 810-07-1484} \\
\large{\due} 

\end{center}

\end{titlepage}



\section{Problem \#1}
Given the \emph{translation operator}, $\hat{T}_a$ defined as
$$\hat{T}_a\Psi(x) = \Psi(x+a)$$
we can find the Hermitian conjugate of $\hat{T}_a$ by using the definition of a Hermitian 
conjugate
\begin{equation}
\int\psi^*\hat{A}\varphi d\tau = \int\varphi\hat{A}^{\dagger}\psi^* d\tau
\label{Hermi}
\end{equation}
where $\psi$ and $\varphi$ are arbitrary functions. So we can go about finding
$\hat{T}_a^{\dagger}$ by solving equation \ref{Hermi} in one dimension for $\psi^*(x)$ and
$\varphi(x)$. Which yields
\begin{align*}
\int\psi^*\hat{A}\varphi d\tau &= \int\varphi\hat{A}^{\dagger}\psi^* d\tau\\
&\Downarrow\\
\int\psi^*(x)\hat{T}_a\varphi(x)dx &= \int\varphi(x)\hat{T}_a^{\dagger}\psi^*(x) dx\\
&\Downarrow\\
\int\psi^*(x)\varphi(x+a)dx &= \int\varphi(x)\hat{T}_a^{\dagger}\psi^*(x) dx
\end{align*}
We note that we need to get the left hand side of the above equation to be equal to the right
hand side. This implies that we need to get $\varphi(x+a)$ to equal $\varphi(x)$. So we can 
set a change of variables on the left hand side such that $x\rightarrow x-a$ which yields
\begin{align*}
\int\psi^*(x)\varphi(x+a)dx &= \int\varphi(x)\hat{T}_a^{\dagger}\psi^*(x) dx\\
&\Downarrow\\
\int\psi^*(x-a)\varphi(x-a+a)dx &= \int\varphi(x)\hat{T}_a^{\dagger}\psi^*(x) dx\\
\int\psi^*(x-a)\varphi(x)dx &= \int\varphi(x)\hat{T}_a^{\dagger}\psi^*(x) dx\\
&\Downarrow\\
\hat{T}_a^{\dagger}\psi^*(x) &= \psi^*(x-a)
\end{align*}
Therefore, the Hermitian conjugate of the translation operator, $\hat{T}_a$ is the 
translation in the opposite direction or
$$\hat{T}_a^{\dagger} = \hat{T}_{-a}$$ 
where
$$\hat{T}_a^{\dagger}\Psi(x) = \Psi(x-a)$$

\pagebreak

\section{Problem \#2}
For an operator of the form $\hat{F} = F\left(\hat{f}\right)$ given that $F(z)$ is expandable
by 
$$F(z) = \sum_{n=0}^{\infty}c_nz^n$$
we can expand the operator as
\begin{equation}
\hat{F} = \sum_{n=0}^{\infty}c_n\hat{z}^n
\label{OperExpand}
\end{equation}

\begin{enumerate}[(a)]
\item For the given operator 
$$\hat{G}_a = \exp\left(a\frac{d}{dx}\right)$$
we can apply equation \ref{OperExpand} by noting that $e^{z}$ can be expanded into the power
series
$$e^z = \sum_{n=0}^{\infty}\frac{1}{n!}z^n$$
by a Taylor expansion. This implies that
$$\hat{G}_a =  \sum_{n=0}^{\infty}\frac{1}{n!}\left(a\frac{d}{dx}\right)^n.$$
So we can apply this expansion to the function
$$\Phi(x) = \hat{G}_a\Psi(x)$$
which yields
\begin{align*}
\Phi(x) &= \hat{G}_a\Psi(x)\\
&= \sum_{n=0}^{\infty}\frac{1}{n!}\left(a\frac{d}{dx}\right)^n\Psi(x)\\
&= \sum_{n=0}^{\infty}\frac{1}{n!}\frac{d^n\Psi(x)}{dx^n}a^n\\
&= \Psi(x+a)
\end{align*}
Therefore we can see that $\hat{G}_a$ is the translation operator with distance $a$. This 
makes sense because we know $\exp(d/dx)$ is related to the wave vector which is related to 
the velocity. We can therefore infer that this operator would be related to some sort of 
translation.

\item For a new operator we are given
$$\hat{G}_a = \exp\left(ax\frac{d}{dx}\right)$$
which we can expand using the expansion of $e^{z}$ from before this yields the result
$$\hat{G}_a =  \sum_{n=0}^{\infty}\frac{1}{n!}\left(ax\frac{d}{dx}\right)^n.$$
which we can use to find the function $\Phi(x)$ by
\begin{align*}
\Phi(x) &= \hat{G}_a\Psi(x)\\
&= \sum_{n=0}^{\infty}\frac{1}{n!}\left(ax\frac{d}{dx}\right)^n\Psi(x)
\end{align*}
Next we note that we can expand $\Psi(x)$ by
$$\Psi(x) = \sum_{m=0}^{\infty}\frac{1}{m!}\frac{d^m\Psi(0)}{dx^m}x^m$$
replacing this into our series over $n$ we have
\begin{align*}
\Rightarrow &= \sum_{n=0}^{\infty}\frac{1}{n!}\left(ax\frac{d}{dx}\right)^n\sum_{m=0}^{\infty}\frac{1}{m!}\frac{d^m\Psi(0)}{dx^m}x^m\\
&= \sum_{n=0}^{\infty}\sum_{m=0}^{\infty}\frac{a^n}{n!}\frac{1}{m!}\frac{d^m\Psi(0)}{dx^m}x^n\frac{d^n}{dx^n}x^m\\
&= \sum_{n=0}^{\infty}\sum_{m=0}^{\infty}\frac{a^n}{n!}\frac{1}{m!}\frac{d^m\Psi(0)}{dx^m}x^n\frac{m!}{(m-n)!}x^{m-n}\\
&= \sum_{n=0}^{\infty}\sum_{m=0}^{\infty}\frac{a^n}{m!}\frac{d^m\Psi(0)}{dx^m}\frac{m!}{(m-n)!n!}x^{m}\\
&= \sum_{n=0}^{\infty}\sum_{m=0}^{\infty}\frac{a^n}{m!}\frac{d^m\Psi(0)}{dx^m}{m\choose{n}}x^{m}\\
&= \sum_{m=0}^{\infty}\frac{1}{m!}\frac{d^m\Psi(0)}{dx^m}x^{m}\sum_{n=0}^{\infty}{m\choose n}a^n\\
&= \sum_{m=0}^{\infty}\frac{1}{m!}\frac{d^m\Psi(0)}{dx^m}x^{m}(1+a)^m\\
&= \sum_{m=0}^{\infty}\frac{1}{m!}\frac{d^m\Psi(0)}{dx^m}(x(1+a))^m\\
\Phi(x) &= \Psi(x(1+a))
\end{align*}


\end{enumerate}

\pagebreak

\section{Problem \#3}
Given the Hermitian operator $\hat{f}$ which follows the relation 
\begin{equation}
\hat{f}^3 = c^2\hat{f}
\label{fEqn}
\end{equation}
we can determine the eigenvalues of $\hat{f}$ by noting that because $\hat{f}$ is Hermitian
it follows that
$$\hat{f}\ket{\psi} = f\ket{\psi}$$
where $f$ is a real valued eigenvalue. So we need to determine the value for $f$. We are 
given the equality from equation \ref{fEqn} we can act both sides of a wave function 
$\ket{\psi}$
\begin{align*}
\hat{f}^3 &= c^2\hat{f}\\
&\Downarrow\\
\hat{f}^3\ket{\psi} &= c^2\hat{f}\ket{\psi}
\end{align*}
We can calculate the right hand side to get
\begin{align*}
\hat{f}^3\ket{\psi} &= \hat{f}^2\hat{f}\ket{\psi}\\
&= f\hat{f}^2\ket{\psi}\\
&= f^2\hat{f}\ket{\psi}\\
&= f^3\ket{\psi}
\end{align*}
and the left hand side gives
\begin{align*}
c^2\hat{f}\ket{\psi} &= c^2f\ket{\psi}
\end{align*}
So by equation \ref{fEqn} we get
\begin{align*}
\hat{f}^3\ket{\psi} &= c^2\hat{f}\ket{\psi}\\
&\Downarrow\\
f^3\ket{\psi} &= c^2f\ket{\psi}
\end{align*}
This implies that
\begin{align*}
\left[f^3 - c^2f\right]\ket{\psi} &= 0\\
&\Downarrow\\
f^3 - c^2f &= 0\\
&\Downarrow\\
f &= \pm c, 0
\end{align*}
Therefore the eigenvalues for $\hat{f}$ are $\pm c$ and $0$.

\end{document}

