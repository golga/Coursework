\documentclass[11pt]{article}

\usepackage{latexsym}
\usepackage{amssymb}
\usepackage{amsthm}
\usepackage{enumerate}
\usepackage{amsmath}
\usepackage{cancel}
\numberwithin{equation}{section}

\setlength{\evensidemargin}{.25in}
\setlength{\oddsidemargin}{-.25in}
\setlength{\topmargin}{-.75in}
\setlength{\textwidth}{6.5in}
\setlength{\textheight}{9.5in}
\newcommand{\due}{October 26th, 2015}
\newcommand{\HWnum}{5}
\newcommand{\grad}{\bold\nabla}
\newcommand{\vecE}{\vec{E}}
\newcommand{\scrptR}{\vec{\mathfrak{R}}}
\newcommand{\kapa}{\frac{1}{4\pi\epsilon_0}}
\newcommand{\emf}{\mathcal{E}}
\newcommand{\unit}[1]{\ensuremath{\, \mathrm{#1}}}
\newcommand{\real}{\textnormal{Re}}
\newcommand{\Erf}{\textnormal{Erf}}
\newcommand{\sech}{\textnormal{sech}}
\newcommand{\scrO}{\mathcal{O}}
\newcommand{\levi}{\widetilde{\epsilon}}
\newcommand{\partiald}[2]{\ensuremath{\frac{\partial{#1}}{\partial{#2}}}}
\newcommand{\norm}[2]{\langle{#1}|{#2}\rangle}
\newcommand{\inprod}[2]{\langle{#1}|{#2}\rangle}
\newcommand{\average}[1]{\left\langle{#1}\right\rangle}
\newcommand{\ket}[1]{|{#1}\rangle}
\newcommand{\bra}[1]{\langle{#1}|}
\newcommand{\Resid}[2]{\ensuremath{\textnormal{Res}\left[{#1},{#2}\right]}}





\begin{document}
\begin{titlepage}
\setlength{\topmargin}{1.5in}
\begin{center}
\Huge{Physics 3310} \\
\LARGE{Principles of Electricity and Magnetism 1} \\
\Large{Professor Thomas R. Schibli} \\[1cm]

\huge{Homework \#\HWnum}\\[0.5cm]

\large{Joe Becker} \\
\large{SID: 810-07-1484} \\
\large{\due} 

\end{center}

\end{titlepage}



\section{Problem \#1}
\begin{enumerate}[(a)]
\item Given the translation operator defined as
$$\hat{T}_{a}\Psi(x) = \Psi(x+a)$$
which using coordinate representation we can write it as
$$\hat{T}_{a}\inprod{x}{a} = \inprod{x}{b}$$
We can find the momentum representation of $\hat{T}_a$ by first finding the matrix elements 
by
$$\bra{p'}\hat{T}_a\ket{p} = \int{dx}\inprod{p'}{x}\hat{T}_{a}\inprod{x}{p}$$
We note that the eigenfunctions of the momentum operator in coordinate representation is 
given by
$$\inprod{x}{p} = (2\pi\hbar)^{-1/2}\exp\left(ip\frac{x}{\hbar}\right)$$
Acting $\hat{T}_a$ on $\inprod{x}{p}$ yields
\begin{align*}
\hat{T}_{a}\inprod{x}{p} &= \hat{T}_{a}(2\pi\hbar)^{-1/2}\exp\left(ip\frac{x}{\hbar}\right)\\
&= (2\pi\hbar)^{-1/2}\exp\left(ip\frac{x+a}{\hbar}\right)\\
&= \exp\left(ip\frac{a}{\hbar}\right)(2\pi\hbar)^{-1/2}\exp\left(ip\frac{x}{\hbar}\right)\\
&= \exp\left(ip\frac{a}{\hbar}\right)\inprod{x}{p}
\end{align*}
Therefore we can find the matrix elements by
\begin{align*}
\bra{p'}\hat{T}_a\ket{p} &= \int{dx}\inprod{p'}{x}\hat{T}_{a}\inprod{x}{p}\\
&= \exp\left(ip\frac{a}{\hbar}\right)\int{dx}\inprod{p'}{x}\inprod{x}{p}\\
&= \exp\left(ip\frac{a}{\hbar}\right)\delta(p'-p)
\end{align*}
Note this forms a matrix of continuous indices. Using the matrix elements we can find how
this operator acts on a wave function in momentum space $\inprod{p}{a}$ as
\begin{align*}
\inprod{p'}{b} &= \int{dp}\bra{p'}\hat{T}_{a}\ket{p}\inprod{p}{a}\\
&= \exp\left(ip\frac{a}{\hbar}\right)\int{dp}\delta(p'-p)\inprod{p}{a}\\
&= \exp\left(ip\frac{a}{\hbar}\right)\inprod{p'}{a}\\
\end{align*}
So in momentum space 
$$\hat{T}_{a} = \exp\left(ip\frac{a}{\hbar}\right)$$

\item For the inversion operator, $\hat{I}$, defined as
$$\hat{I}\psi(x) = \psi(-x)$$
which in coordinate representation is
$$\hat{I}\inprod{x}{a} = \inprod{x}{b}$$
We calculate the momentum representation of $\hat{I}$ by finding the matrix elements like in
part (a) noting that
\begin{align*}
\hat{I}\inprod{x}{p} &= \hat{I}(2\pi\hbar)^{-1/2}\exp\left(ip\frac{x}{\hbar}\right)\\
&= (2\pi\hbar)^{-1/2}\exp\left(ip\frac{-x}{\hbar}\right)\\
&= \inprod{x}{-p}
\end{align*}
So this allows us to find the matrix elements of $\hat{I}$ by
\begin{align*}
\bra{p'}\hat{I}\ket{p} &= \int{dx}\inprod{p'}{x}\hat{I}\inprod{x}{p}\\
&= \int{dx}\inprod{p'}{x}\inprod{x}{-p}\\
&= \delta(p'+p)
\end{align*}
Note that this is a matrix that represents a momentum in the opposite direction as we would 
expect. Using this we can act the $\hat{I}$ operator on a wave function in momentum space.
\begin{align*}
\inprod{p'}{b} &= \int{dp}\bra{p'}\hat{I}\ket{p}\inprod{p}{a}\\
&= \int{dp}\delta(p'+p)\inprod{p}{a}\\
&= \inprod{-p'}{a}\\
\end{align*}
So we can say the inversion operator in momentum space acts like
$$\hat{I}\psi(p) = \psi(-p)$$
\end{enumerate}

\pagebreak

\section{Problem \#2}
\begin{enumerate}[(a)]
\item For the translation operator $\hat{T}_{a}$ we note that the inverse of $\hat{T}_a$ is
$\hat{T}_{-a}$ which implies that these represent a unitary transformation by
$$\hat{T}_{a}\hat{T}_{-a} = 1$$
This allows us to do a unitary transformation by the general transformation of an operator 
$\hat{F}$
$$\hat{F}' = S\hat{F}S^{-1}$$
So the transformation of the position operator is given by
\begin{align*}
\hat{x'} &= \hat{T}_{a}\hat{x}\hat{T}_{-a}\\
&= (x+a)\hat{T}_{-a}
\end{align*}
Then we can find the transformation of the operator $\hat{p}$ by using $\hat{T}_{a}$ in 
momentum space which we found in problem one.
\begin{align*}
\hat{p}' &= \hat{T}_{a}\hat{p}\hat{T}_{-a}\\
&= \exp\left(ip\frac{a}{\hbar}\right)\hat{p}\exp\left(ip\frac{-a}{\hbar}\right)\\
&= \hat{p}\exp\left(ip\frac{a}{\hbar}\right)\exp\left(ip\frac{-a}{\hbar}\right)\\
&= \hat{p}
\end{align*}
So the operator $\hat{p}$ is invariant under the unitary transformation $\hat{T}_{a}$.

\item We note that the inversion operator's inverse operator is itself $\hat{I}^{-1}=\hat{I}$.
Which we note $\hat{I}$ represents an unitary transformation. Therefore we can calculate the
transformation of $\hat{x}$ as
\begin{align*}
\hat{x}' &= \hat{I}\hat{x}\hat{I}^{-1}\\
&= -\hat{x}\hat{I}
\end{align*}
And for the momentum operator we have a analogous result
\begin{align*}
\hat{p}' &= \hat{I}\hat{p}\hat{I}^{-1}\\
&= -\hat{p}\hat{I}
\end{align*}



\end{enumerate}

\pagebreak

\section{Problem \#3}
For the wave function in coordinate representation
$$\psi(x) = \inprod{x}{\psi} = \left\{\begin{array}{cc}
                  a^{-1/2}\exp\left(\dfrac{i}{\hbar}xp_0\right)     &-a/2\le{x}\le{a/2}\\
                  0                                                &|x|>a/2
            \end{array}\right.$$
This allows us the find the momentum representation, $\inprod{p}{\psi}$, by
\begin{align*}
\inprod{p}{\psi} &= \int{dx}\inprod{p}{x}\inprod{x}{\psi}\\
&= \int_{-a/2}^{a/2}dx(2\pi\hbar)^{-1/2}\exp\left(-\frac{i}{\hbar}xp\right)a^{-1/2}\exp\left(\dfrac{i}{\hbar}xp_0\right)\\
&= \sqrt{\frac{1}{2a\pi\hbar}}\int_{-a/2}^{a/2}dx\exp\left(\frac{i}{\hbar}x(p_0-p)\right)\\
&= -\sqrt{\frac{1}{2a\pi\hbar}}\frac{i\hbar}{p_0-p}\left(\exp\left(\frac{i}{\hbar}x(p_0-p)\right)\right|_{-a/2}^{a/2}\\
&= -\sqrt{\frac{1}{2a\pi\hbar}}\frac{i\hbar}{p_0-p}\left(\exp\left(i\frac{a}{2\hbar}(p_0-p)\right)-\exp\left(-i\frac{a}{2\hbar}(p_0-p)\right)\right)\\
&= -\sqrt{\frac{1}{2a\pi\hbar}}\frac{i\hbar}{p_0-p}2i\sin\left(\frac{a}{2\hbar}(p_0-p)\right)\\
&= \sqrt{\frac{1}{2a\pi\hbar}}\frac{2\hbar}{p_0-p}\sin\left(\frac{a}{2\hbar}(p_0-p)\right)
\end{align*}
\pagebreak

\section{Problem \#4}
For a particle in a potential $U(x) = \alpha{x}$ we can find the eigenvalues and 
eigenfunctions of the energy operator, $\hat{H}$, in the momentum representation. Where
in coordinate representation we have
$$\hat{H}_x = -\frac{\hbar^2}{2\mu}\partiald{^2}{x^2} + \alpha{x}$$
by writing the Hamiltonian in momentum representation as
$$\hat{H}_p = \frac{p^2}{2\mu} + i\alpha\hbar\partiald{}{p}$$
So by solving the eigenvalue problem we can find the eigenvalues, $E_p$, and eigenfunctions,
$\phi(p)$.
\begin{align*}
\hat{H}_p\phi(p) &= E_p\phi(p)\\
&\Downarrow\\
\left[\frac{p^2}{2\mu} + i\alpha\hbar\partiald{}{p}\right]\phi(p) &= E_p\phi(p)\\
&\Downarrow\\
i\alpha\hbar\partiald{\phi}{p} &= E_p\phi(p) - \frac{p^2}{2\mu}\phi(p)\\
&\Downarrow\\
\int\frac{\partial(\phi(p))}{\phi(p)} &= -\int\frac{i}{\alpha\hbar}\left(E_p - \frac{p^2}{2\mu}\right)\partial{p}\\
\log(\phi(p)) &= -\frac{i}{\alpha\hbar}\left(E_pp - \frac{p^3}{6\mu}\right) + C\\
\phi(p) &= A\exp\left(-\frac{iE_p}{\alpha\hbar}p + \frac{i}{6\alpha\hbar\mu}p^3\right)
\end{align*}
We note that this is a free particle so $E_p$ represents a continuous spectrum of eigenvalues.

\end{document}

