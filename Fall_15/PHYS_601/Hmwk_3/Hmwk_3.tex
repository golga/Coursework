\documentclass[11pt]{article}

\usepackage{latexsym}
\usepackage{amssymb}
\usepackage{amsthm}
\usepackage{enumerate}
\usepackage{amsmath}
\usepackage{cancel}
\numberwithin{equation}{section}

\setlength{\evensidemargin}{.25in}
\setlength{\oddsidemargin}{-.25in}
\setlength{\topmargin}{-.75in}
\setlength{\textwidth}{6.5in}
\setlength{\textheight}{9.5in}
\newcommand{\due}{September 23rd, 2015}
\newcommand{\HWnum}{3}
\newcommand{\grad}{\bold\nabla}
\newcommand{\vecE}{\vec{E}}
\newcommand{\scrptR}{\vec{\mathfrak{R}}}
\newcommand{\kapa}{\frac{1}{4\pi\epsilon_0}}
\newcommand{\emf}{\mathcal{E}}
\newcommand{\unit}[1]{\ensuremath{\, \mathrm{#1}}}
\newcommand{\real}{\textnormal{Re}}
\newcommand{\Erf}{\textnormal{Erf}}
\newcommand{\sech}{\textnormal{sech}}
\newcommand{\scrO}{\mathcal{O}}
\newcommand{\levi}{\widetilde{\epsilon}}
\newcommand{\partiald}[2]{\ensuremath{\frac{\partial{#1}}{\partial{#2}}}}
\newcommand{\norm}[2]{\langle{#1}|{#2}\rangle}
\newcommand{\inprod}[2]{\langle{#1}|{#2}\rangle}
\newcommand{\ket}[1]{|{#1}\rangle}
\newcommand{\bra}[1]{\langle{#1}|}





\begin{document}
\begin{titlepage}
\setlength{\topmargin}{1.5in}
\begin{center}
\Huge{Physics 3320} \\
\LARGE{Principles of Electricity and Magnetism II} \\
\Large{Professor Ana Maria Rey} \\[1cm]

\huge{Homework \#\HWnum}\\[0.5cm]

\large{Joe Becker} \\
\large{SID: 810-07-1484} \\
\large{\due} 

\end{center}

\end{titlepage}



\section{Problem \#1}
Given the \emph{Laplace-Runge-Lenz vector} (LRL-vector) defined as
\begin{equation}
\mathbf{A} = \frac{1}{mk}\mathbf{p}\times\mathbf{L}-\mathbf{\hat{r}}.
\label{LRLVector}
\end{equation}
Note, for the central potential $V(r) = -k/r$ the LRL-vector is conserved.

\begin{enumerate}[(a)]
\item We can show that the momentum vector, $\mathbf{p}$, traces out a circle in momentum 
space by first calculating $\mathbf{L}\times\mathbf{A}$, which yields
\begin{align*}
\mathbf{L}\times\mathbf{A} &= \frac{1}{mk}\mathbf{L}\times\left(\mathbf{p}\times\mathbf{L}\right) - \mathbf{L}\times\mathbf{\hat{r}}
\end{align*}
Where we calculate
\begin{align*}
\mathbf{L}\times\left(\mathbf{p}\times\mathbf{L}\right) &= \mathbf{p}\left(\mathbf{L}\cdot\mathbf{L}\right) - \mathbf{L}\left(\mathbf{p}\cdot\mathbf{L}\right)\\
&= L^2\mathbf{p} - \mathbf{L}\left(\mathbf{p}\cdot\left(\mathbf{r}\times\mathbf{p}\right)\right)\\
&= L^2\mathbf{p} - \mathbf{L}\left(\mathbf{r}\cdot\left(\cancelto{0}{\mathbf{p}\times\mathbf{p}}\right)\right)\\
&= L^2\mathbf{p}
\end{align*}
Next, we assume that $\mathbf{A}$ points in the $\hat{x}$ direction. This coupled with the
knowledge that our motion is constrained to $xy$-plane implies that $\mathbf{L}$ points in 
the $\hat{z}$ direction. Therefore
\begin{align*}
\mathbf{L}\times\mathbf{A} = LA(\mathbf{\hat{z}}\times\mathbf{\hat{x}}) = LA\mathbf{\hat{y}}
\end{align*}
Using this same assumption that $\mathbf{L} = L\mathbf{\hat{z}}$ we find
$$\mathbf{L}\times\mathbf{\hat{r}} = L(\mathbf{\hat{z}}\times\mathbf{\hat{r}}) = -L\mathbf{\hat{\theta}}$$
This allows us to rearrange our equation into the form
\begin{align*}
\mathbf{L}\times\mathbf{A} &= \frac{1}{mk}\mathbf{L}\times\left(\mathbf{p}\times\mathbf{L}\right) - \mathbf{L}\times\mathbf{\hat{r}}\\
\mathbf{L}\times\mathbf{A} - \frac{1}{mk}\mathbf{L}\times\left(\mathbf{p}\times\mathbf{L}\right) &= - \mathbf{L}\times\mathbf{\hat{r}}\\
&\Downarrow\\
L\mathbf{\hat{\theta}} &= LA\mathbf{\hat{y}} - \frac{L^2}{mk}\mathbf{p}\\
L\mathbf{\hat{\theta}} &= LA\mathbf{\hat{y}} - \frac{L^2}{mk}(p_x\mathbf{\hat{x}}+p_y\mathbf{\hat{y}})\\
L\mathbf{\hat{\theta}} &= \left(LA - \frac{L^2}{mk}p_y\right)\mathbf{\hat{y}} - \frac{L^2}{mk}p_x\mathbf{\hat{x}}
\end{align*}
Now we calculate the magnitude of both sides to get
\begin{align*}
L^2 &= \left(LA - \frac{L^2}{mk}p_y\right)^2 - \left(\frac{L^2}{mk}p_x\right)^2\\
&= (LA)^2 + \left(\frac{L^2}{mk}\right)^2p_y^2 - 2\frac{AL^3}{mk}p_y + \left(\frac{L^2}{mk}\right)^2p_x^2\\
\end{align*}
Which we can rearrange by grouping together the $p_x$ and $p_y$ terms
\begin{align*}
\Rightarrow\left(\frac{mk}{L}\right)^2 &= p_x^2 + p_y^2 - 2\frac{m^2k^2}{L^4}\frac{AL^3}{mk}p_y + \frac{m^2k^2}{L^4}(AL)^2\\
\left(\frac{mk}{L}\right)^2 &= p_x^2 + p_y^2 - 2\frac{Amk}{L}p_y + \left(\frac{mkA}{L}\right)^2\\
&\Downarrow\\
\left(\frac{mk}{L}\right)^2 &= p_x^2 + \left(p_y - \frac{Amk}{L}\right)^2
\end{align*}
We recall that the magnitude of the LRL-vector is the eccentricity, $e$, which we replace as
\begin{equation}
p_x^2 + \left(p_y - \frac{mke}{L}\right)^2 = \left(\frac{mk}{L}\right)^2
\label{MomCirc}
\end{equation}
which is the equation for a circle centered at $(0,mke/L)$ with a radius of $mk/L$ in 
momentum space.

\item
We can take the result from part (a) given by equation \ref{MomCirc} and find where it 
crosses the $p_x$ axis by setting $p_y=0$. This yields
\begin{align*}
p_x^2 + \left(0 - \frac{mke}{L}\right)^2 &= \left(\frac{mk}{L}\right)^2\\
&\Downarrow\\
p_x^2 &= \left(\frac{mke}{L}\right)^2 - \left(\frac{mke}{L}\right)^2\\
&= \left(\frac{mk}{L}\right)^2\left(1-e^2\right)\\
&\Downarrow\\
p_0 &\equiv \pm\frac{mk}{L}\sqrt{1-e^2}
\end{align*}
Where we defined the momentum at this point as $p_0$. We note that for the condition of bound orbits, $|e|<1$, this gives a real result. Note the 
result that eccentricity is given by
\begin{equation}
e = \sqrt{\frac{2L^2E}{mk^2}+1}
\label{eccen}
\end{equation}
for elliptical orbits. We can replace this into the result from above to get
\begin{align*}
p_0 = \pm\frac{mk}{L}\sqrt{1-e^2} &= \pm\frac{mk}{L}\sqrt{1-\frac{2L^2E}{mk^2}+1} \\
&= \pm\sqrt{\frac{m^2k^2}{L^2}\frac{2L^2E}{mk^2}} \\
&= \pm\sqrt{2mE}
\end{align*}
This result can be written in the familiar form as 
$$E = \frac{p_0^2}{2m}$$
which implies when we are at $p_0$ the energy is entirely kinetic. This corresponds to the 
two turning points of our system.

\item
Given the $p_0$ we found in part (b) we can rescale the momentum to a dimensionless quantity
$\mathbf{R} = \mathbf{p}/p_0$. Which we can transform equation \ref{MomCirc} by dividing 
both sides by $p_0^2$
\begin{align*}
\left(\frac{p_x}{p_0}\right)^2 + \left(\frac{p_y}{p_0} - \frac{mke}{Lp_0}\right)^2 &= \left(\frac{mk}{Lp_0}\right)^2\\
&\Downarrow\\
R_x^2 + \left(R_y - \frac{e}{\sqrt{1-e^2}}\right)^2 &= \frac{1}{1-e^2}
\end{align*}
Note that for we have the intersections $\pm1$ on the $X$ axis. Which is consistent with 
the result from part (b). Now, we can find the intersections of the $Y$ axis by
\begin{align*}
\cancelto{0}{R_x^2} + \left(R_y - \frac{e}{\sqrt{1-e^2}}\right)^2 &= \frac{1}{1-e^2}\\
&\Downarrow\\
R_y - \frac{e}{\sqrt{1-e^2}} &= \pm\frac{1}{\sqrt{1-e^2}}\\
R_y &= \pm\frac{1}{\sqrt{1-e^2}} + \frac{e}{\sqrt{1-e^2}} \\
&= \frac{e\pm1}{\sqrt{1-e^2}}
\end{align*}
So our two intersection points on $Y$ are $e-1/\sqrt{1-e^2}$ and $e+1/\sqrt{1-e^2}$.

\item In part (c) we created the $X-Y$ plane. We can take a unit sphere and intersect this 
plane at the equator of the unit sphere. We can calculate the points that pierce the unit 
sphere by a line drawn from the north pole of the unit sphere at location $(0,0,1)$ to the
point on the $X-Y$ plane by
$$x=\frac{2X}{1+X^2+Y^2},\qquad \; y=\frac{2Y}{1+X^2+Y^2},\qquad z=\frac{-1+X^2+Y^2}{1+X^2+Y^2}$$
For the points that intersect the $X$ axis we have the points $(1,0)$ and $(-1,0)$ and it is
easy to see that for the point $(1,0)$
\begin{align*}
x &= \frac{2(1)}{1+1} = 1\\
y &= \frac{0}{1+1} = 0\\
z &= \frac{-1+1+0}{1+1} = 0
\end{align*}
and for $(-1,0)$ the piercing point is
\begin{align*}
x &= \frac{2(-1)}{1+1} = -1\\
y &= \frac{0}{1+1} = 0\\
z &= \frac{-1+1+0}{1+1} = 0
\end{align*}
Now, we can calculate the piercing coordinates for the position $(0,(e+1)/\sqrt{1-e^2})$ by 
finding $y$ and $z$ with $x=0$.
\begin{align*}
y &= \frac{2\frac{e+1}{\sqrt{1-e^2}}}{1+\frac{(e+1)^2}{1-e^2}}\\
&= \frac{2(e+1)}{\sqrt{1-e^2}}\frac{1-e^2}{1-e^2+(e+1)^2}\\
&= \frac{2(e+1)}{\sqrt{1-e^2}}\frac{1-e^2}{1-e^2+e^2+1+2e}\\
&= \frac{2(e+1)}{\sqrt{1-e^2}}\frac{1-e^2}{2(e+1)}\\
&= \sqrt{1-e^2}
\end{align*}
and
\begin{align*}
z &= \frac{-1 + \frac{(e+1)^2}{1-e^2}}{1+\frac{(e+1)^2}{1-e^2}}\\
&= \frac{(e+1)^2-1+e^2}{1-e^2}\frac{1-e^2}{(e+1)^2+1-e^2}\\
&= \frac{e^2+2e+1-1+e^2}{e^2+2e+1+1-e^2}\\
&= \frac{2e(e+1)}{2(e+1)}\\
&= e
\end{align*}
And we can repeat this calculation for $(0,(e-1)/\sqrt{1-e^2})$ by
\begin{align*}
y &= \frac{2\frac{e-1}{\sqrt{1-e^2}}}{1+\frac{(e-1)^2}{1-e^2}}\\
&= \frac{2(e-1)}{\sqrt{1-e^2}}\frac{1-e^2}{1-e^2+(e-1)^2}\\
&= \frac{2(e-1)}{\sqrt{1-e^2}}\frac{1-e^2}{1-e^2+e^2+1-2e}\\
&= \frac{2(e-1)}{\sqrt{1-e^2}}\frac{1-e^2}{2(1-e)}\\
&= -\sqrt{1-e^2}
\end{align*}
and
\begin{align*}
z &= \frac{-1 + \frac{(e-1)^2}{1-e^2}}{1+\frac{(e-1)^2}{1-e^2}}\\
&= \frac{(e-1)^2-1+e^2}{1-e^2}\frac{1-e^2}{(e-1)^2+1-e^2}\\
&= \frac{e^2-2e+1-1+e^2}{e^2-2e+1+1-e^2}\\
&= \frac{2e(e-1)}{-2(e-1)}\\
&= -e
\end{align*}
So we can see the four points found in part (c) correspond to the pierced points
\begin{align*}
(1,0) &\Rightarrow (1,0,0)\\
(-1,0) &\Rightarrow (-1,0,0)\\
\left(0,\frac{e+1}{\sqrt{1-e^2}}\right) &\Rightarrow (0,\sqrt{1-e^2},e)\\
\left(0,\frac{e-1}{\sqrt{1-e^2}}\right) &\Rightarrow (0,-\sqrt{1-e^2},-e)
\end{align*}
We can calculate the volume made by three of these points by
\begin{align*}
(-1,0,0)\cdot((1,0,0)\times(0,\sqrt{1-e^2},e) &= 0\\
&\textnormal{or}\\
(-1,0,0)\cdot((0,-\sqrt{1-e^2},-e)\times(0,\sqrt{1-e^2},e) &= 0\\
(-1,0,0)\cdot((0,-\sqrt{1-e^2},-e)\times(0,\sqrt{1-e^2},e) &= 0\\
\end{align*}
that these points lie on the same plane, and we know they intersect the unit sphere. This
implies that these points must make a greater circle on the unit sphere. We can test to see
that this holds true for a general point on the normalized $Y$ circle by seeing that for 
$X=R_x$ and $Y=R_y-e/\sqrt{1-e^2}$ where the equation of our circle becomes
$$X^2+Y^2 = \frac{1}{1-e^2}$$
which implies that
$$1+X^2+Y^2 = \frac{2-e^2}{1-e^2}$$
so our piecing points become
\begin{align*}
x &= 2\frac{1-e^2}{2-e^2}R_x\\
y &= 2\frac{1-e^2}{2-e^2}R_y - 2\frac{e\sqrt{1-e^2}}{2-e^2}\\
z &= -\frac{e^2}{1-e^2}
\end{align*}
we note that the above $z$ does not depend on $R_x$ or $R_y$ so we can infer that all $R_x$ 
and $R_y$ lie on a plane. Therefore for all $R_x$ and $R_y$ we project onto a great circle
on the unit sphere.

\item We see that the hidden symmetry of the Kepler problem arises from the forcing of the
momentum space to fit onto a circle in the $X$-$Y$ plane.
\end{enumerate}

\pagebreak

\section{Problem \#2}
Given Kepler's equation
\begin{equation}
\psi - e\sin\psi = \frac{2\pi t}{T}\equiv\omega t
\label{Keps}
\end{equation}
we can expand over a small $e$ using the recursion relation 
\begin{equation}
\psi_{n+1} = \omega t + e\sin\psi_n
\label{Recur}
\end{equation}
with $\psi_0=\omega t$. Calculating to first order in $e$ is easy to see
$$\psi_1 = \omega t + e\sin\psi_0 = \omega t +e \sin\omega t$$
Now to calculate to second order in $e$ we need to use the fact that $e$ is small.
\begin{align*}
\psi_2 &= \omega t + e\sin\psi_1 \\
&= \omega t +e \sin(\omega t + e\sin\omega t)\\
&= \omega t +e(\sin\omega t \cos(e\sin\omega t)+ \sin(e\sin\omega t)\cos\omega t)\\
&= \omega t +e\sin\omega t + e^2\sin\omega t\cos\omega t
\end{align*}
And finally to third order we get
\begin{align*}
\psi_3 &= \omega t + e\sin\psi_2\\
&= \omega t + e\sin(\omega t +e\sin\omega t + e^2\sin\omega t\cos\omega t)\\
&= \omega t + e\sin\left(\frac{}{}\omega t +e\sin\omega t(1 + e\cos\omega t)\right)\\
&= \omega t + e\left(\frac{}{}\sin\omega t\cancelto{1}{\cos(e\sin\omega t(1 + e\cos\omega t))} + \sin(e\sin\omega t(1 + e\cos\omega t))\cos\omega t\right)\\
&= \omega t + e\left(\frac{}{}\sin\omega t + e\sin\omega t(1 + e\cos\omega t))\cos\omega t\right)\\
&= \omega t + e\sin\omega t + e^2\sin\omega t\cos\omega t + e^3\sin\omega t\cos^2\omega t)
\end{align*}
\pagebreak

\section{Problem \#3}
For a gravitational potential with a perturbation given as
$$V(r) = -\frac{k}{r} + \frac{\beta}{r^n}$$
we can calculate the resulting precession by the integral
$$\Delta\theta = \int_{0}^{T}\Omega(t)dt$$
where $\Omega(t)t$ is the angle of rotation of the LRL-vector about a vector 
$\mathbf{\Omega}$. Where we know that $\Omega$ is given by
$$\Omega = \frac{-f(r)}{mk}\frac{\cos\theta}{e}L$$
which allows us to change the integral over time into an integral over angle $\theta$.
\begin{align*}
\Delta\theta &= \int_{0}^{T}\Omega(t)dt\\
&= \frac{1}{mke}\int_{0}^{T}(-f(r))\cos\theta Ldt\\
&= \frac{1}{mke}\int_{0}^{2\pi}(-f(r))\cos\theta\left(mr^2\frac{d\theta}{dt}\right)dt\\
&= \frac{1}{ke}\int_{0}^{2\pi}(-f(r)r^2)\cos\theta d\theta
\end{align*}
Where $f(r)$ is the perturbative force which for our case we can calculate as
\begin{align*}
f(r) &= -V_p'(r) = -\frac{d}{dr}\frac{\beta}{r^n}\\
&= \frac{n\beta}{r^{n+1}}
\end{align*}
Which gives the equation for $\Delta\theta$
$$\Delta\theta = \frac{n\beta}{ke}\int_{0}^{2\pi}r^{1-n}(\theta)\cos\theta d\theta$$
We note that $r$ is a function of $\theta$ and taking the perturbation to be small we can 
assume that $r$ is the unperturbed result given by
$$r = \frac{a(1-e^2)}{1+e\cos\theta}$$
which replacing into the integral yields
$$\Delta\theta_n = \frac{n\beta (a(1-e^2))^{1-n}}{ke}\int_{0}^{2\pi}\frac{\cos\theta}{(1+e\cos\theta)^{1-n}} d\theta$$
Now we can calculate the period for $n=2,3,4$ by taking the integrals. First we have
\begin{align*}
\Delta\theta_{2} &= \frac{2\beta (a(1-e^2))^{-1}}{ke}\int_{0}^{2\pi}\frac{\cos\theta}{(1+e\cos\theta)^{-1}} d\theta\\
&= \frac{2\beta }{kae(1-e^2)}\int_{0}^{2\pi}(1+e\cos\theta)\cos\theta d\theta\\
&= \frac{2\beta \pi}{kae(1-e^2)}e\pi = \frac{2\beta \pi}{ka(1-e^2)}
\end{align*}
Now for $n=3$ we can calculate the integral for $\Delta\theta_3$.
\begin{align*}
\Delta\theta_{3} &= \frac{3\beta a(1-e^2)^{-2}}{ke}\int_{0}^{2\pi}\frac{\cos\theta}{(1+e\cos\theta)^{-2}} d\theta\\
&= \frac{3\beta }{ke(a(1-e^2))^{2}}\int_{0}^{2\pi}(1+e\cos\theta)^{2}\cos\theta d\theta\\
&= \frac{3\beta }{ke(a(1-e^2))^{2}}\int_{0}^{2\pi}(1+2e\cos\theta+e^2\cos^2\theta)\cos\theta d\theta\\
&= \frac{3\beta }{ke(a(1-e^2))^{2}}\int_{0}^{2\pi}\cos\theta+2e\cos^2\theta+e^2\cos^3\theta d\theta\\
&= \frac{3\beta }{ke(a(1-e^2))^{2}}2e\pi = \frac{6\beta a\pi}{k(a(1-e^2))^{2}}
\end{align*}
And finally for $\Delta\theta_4$ we have
\begin{align*}
\Delta\theta_{4} &= \frac{4\beta }{ke(a(1-e^2))^{3}}\int_{0}^{2\pi}(1+e\cos\theta)^{3}\cos\theta d\theta\\
&= \frac{4\beta a}{ke(a(1-e^2))^{3}}\int_{0}^{2\pi}(1+3e\cos\theta+3e^2\cos^2\theta+e^3\cos^3\theta)\cos\theta d\theta\\
&= \frac{4\beta a}{ke(a(1-e^2))^{3}}\int_{0}^{2\pi}\cos\theta+3e\cos^2\theta+3e^2\cos^3\theta+e^3\cos^4\theta d\theta\\
&= \frac{4\beta a}{ke(a(1-e^2))^{3}}\left(3e\pi + \frac{3}{4}e^3\pi\right) = \frac{3\beta \pi(4+e^2)}{k(a(1-e^2))^{3}}
\end{align*}
\pagebreak

\section{Problem \#4}
For the precession of the perihelion of the orbit of Mercury we take a perturbation on the
potential as
$$V(r) = \frac{-mMG}{r}\left(1+\alpha\frac{GM}{rc^2}\right)$$ 
we see that the perturbation is on the order of $1/r^2$. So we can use $\Delta\theta_2$ 
given by
$$\Delta\theta_2 = \frac{2\beta \pi}{ka(1-e^2)}$$
where
$$k = mMG,\qquad \beta = k\alpha\frac{GM}{c^2}$$
So we can solve for $\alpha$ by taking $\Delta\theta_2$ for Mercury.
\begin{align*}
\Delta\theta_2 &= \frac{2\beta \pi}{ka(1-e^2)}\\
&= \frac{2 k\pi}{ka(1-e^2)}\frac{GM}{c^2}\alpha\\
&= \frac{2 \pi}{a(1-e^2)}\frac{GM}{c^2}\alpha\\
&\Downarrow\\
\alpha &= \Delta\theta_2\frac{c^2}{GM}\frac{a(1-e^2)}{2\pi}
\end{align*}
So in order calculate $\alpha$ we can calculate $\Delta\theta_2$ by using the fact that
the period of the orbit is $T=0.241\unit{yr}$ and the rate of the perihelion's advancement 
is $\Omega = 43\unit{arcsec/century}$ which we can convert to radians per year
$$\Omega = \frac{43\unit{arcsec}}{1\unit{century}}\frac{1\unit{century}}{100\unit{yr}}\frac{4.85\times10^{-6}\unit{radians}}{1\unit{arcsec}}=2.08\times10^{-6}\unit{yr^{-1}}$$
So we can calculate $\Delta\theta_2$ by
$$\Delta\theta_2 = \Omega T = 2.08\times10^{-6}\unit{yr^{-1}}(0.241\unit{yr}) = 5.03\times10^{-7}$$
and we use the given values 
\begin{align*}
a &= 57900000\unit{km}\\
e &= 0.206\\
GM/c^2 &= 1.475\unit{km}
\end{align*}
to calculate
$$\alpha = 3.01$$
We note that this perturbation on the potential does not act as an addition to the barrier 
from the angular momentum. This is because the motion of the orbit is completely constrained
in the plane. This means any extra terms results in precession.


\end{document}

