\documentclass[11pt]{article}

\usepackage{latexsym}
\usepackage{amssymb}
\usepackage{amsthm}
\usepackage{enumerate}
\usepackage{amsmath}
\usepackage{cancel}
\numberwithin{equation}{section}

\setlength{\evensidemargin}{.25in}
\setlength{\oddsidemargin}{-.25in}
\setlength{\topmargin}{-.75in}
\setlength{\textwidth}{6.5in}
\setlength{\textheight}{9.5in}
\newcommand{\due}{November 18th, 2015}
\newcommand{\HWnum}{10}
\newcommand{\grad}{\bold\nabla}
\newcommand{\vecE}{\vec{E}}
\newcommand{\scrptR}{\vec{\mathfrak{R}}}
\newcommand{\kapa}{\frac{1}{4\pi\epsilon_0}}
\newcommand{\emf}{\mathcal{E}}
\newcommand{\unit}[1]{\ensuremath{\, \mathrm{#1}}}
\newcommand{\real}{\textnormal{Re}}
\newcommand{\Erf}{\textnormal{Erf}}
\newcommand{\sech}{\textnormal{sech}}
\newcommand{\scrO}{\mathcal{O}}
\newcommand{\levi}{\widetilde{\epsilon}}
\newcommand{\partiald}[2]{\ensuremath{\frac{\partial{#1}}{\partial{#2}}}}
\newcommand{\norm}[2]{\langle{#1}|{#2}\rangle}
\newcommand{\inprod}[2]{\langle{#1}|{#2}\rangle}
\newcommand{\ket}[1]{|{#1}\rangle}
\newcommand{\bra}[1]{\langle{#1}|}





\begin{document}
\begin{titlepage}
\setlength{\topmargin}{1.5in}
\begin{center}
\Huge{Physics 3320} \\
\LARGE{Principles of Electricity and Magnetism II} \\
\Large{Professor Ana Maria Rey} \\[1cm]

\huge{Homework \#\HWnum}\\[0.5cm]

\large{Joe Becker} \\
\large{SID: 810-07-1484} \\
\large{\due} 

\end{center}

\end{titlepage}



\section{Problem \#1}
If we preform a canonical transformation from $(q,p)\rightarrow(Q,P)$ by
$$F_2(q,P) = qP + \frac{1}{2}\epsilon H(q,P)$$
where the Hamiltonian is given as
$$H(q,p) = \frac{p^2}{2m}+V(q)$$
we can solve for the transformation variables by
\begin{align*}
p = \partiald{F_2}{q} &= P + \epsilon\frac{1}{2}\frac{dV}{dq}\\
&\Downarrow\\
P &= p - \epsilon\frac{1}{2}\frac{dV}{dq}
\end{align*}
and
\begin{align*}
Q = \partiald{F_2}{P} &= q + \epsilon\frac{1}{2}\frac{P}{m}\\
&\Downarrow\\
Q &= q + \epsilon\frac{p}{2m} - \epsilon^2\frac{1}{4m}\frac{dV}{dq}
\end{align*}
This allows us to preform a new canonical transformation from $(Q,P)\rightarrow(q',p')$ by
$$F_3(P,q') = -Pq' + \frac{1}{2}\epsilon H(q',P)$$
Which yields
\begin{align*}
Q = -\partiald{F_3}{P} &= q' - \epsilon\frac{1}{2}\frac{P}{m}\\
&\Downarrow\\
q' &= Q + \epsilon\frac{1}{2}\frac{P}{m}
\end{align*}
and
\begin{align*}
p' = -\partiald{F_3}{q'} &= P - \epsilon\frac{1}{2}\frac{dV}{dq'}\\
&= P - \epsilon\frac{1}{2}\frac{dV}{dQ}\cancelto{1}{\frac{dQ}{dq'}}\\
&\Downarrow\\
p' &= P - \epsilon\frac{1}{2}\frac{dV}{dQ}
\end{align*}
Now we can transform $(Q,P)$ back to $(q,p)$ by the transformations we first found to get the
transformation from $(q,p)\rightarrow(q',p')$. So first for $q'$ we have
\begin{align*}
q' &= Q + \epsilon\frac{1}{2}\frac{P}{m}\\
&= q + \epsilon\frac{p}{2m} - \epsilon^2\frac{1}{4m}\frac{dV}{dq} + \epsilon\frac{1}{2m}\left( p - \epsilon\frac{1}{2}\frac{dV}{dq}\right)\\
&= q + \epsilon\frac{p}{m} - \epsilon^2\frac{1}{2m}\frac{dV}{dq} 
\end{align*}
We note the expansion of $q(t+\epsilon)$ about epsilon yields
$$q(t+\epsilon) = q + \epsilon\frac{dq}{dt} + \epsilon^2\frac{1}{2}\frac{d^2q}{dt^2}$$
we note that this is the same as the transformed result we found, due to the fact that 
$$\frac{dq}{dt} = \frac{p}{m},\qquad \frac{d^2q}{dt^2} = -\frac{1}{m}\frac{dV}{dq}$$
Which implies that $q' = q(t+\epsilon)$. Now we can do the same with $p'$ as
\begin{align*}
p' &= P - \epsilon\frac{1}{2}\frac{dV(Q)}{dQ}\frac{dQ}{dq}\\
&= p - \epsilon\frac{1}{2}\frac{dV}{dq} - \epsilon\frac{1}{2}\frac{dV}{dq}\frac{d}{dq}\left(q + \epsilon\frac{p}{2m} - \epsilon^2\frac{1}{4m}\frac{dV}{dq}\right)\\
&= p - \epsilon\frac{1}{2}\frac{dV}{dq} - \epsilon\frac{1}{2}\frac{dV}{dq}\left(1 - \epsilon^2\frac{1}{4m}\frac{d^2V}{dq^2}\right)\\
&= p - \epsilon\frac{dV}{dq} + \scrO(\epsilon^3)
\end{align*}
We see that this agrees to second order due to the fact that $\ddot{p}=0$.

\pagebreak

\section{Problem \#2}
\begin{enumerate}[(a)]
\item For the free fall Hamiltonian
$$H = \frac{p^2}{2m} + mgq$$
where we are given the principle function
$$S(q,\alpha,t) = W(q,\alpha) - Et$$
Using this we can apply the Hamilton-Jacobi equation 
$$H + \frac{dS}{dt} = 0$$
where we note that $p$ is given by 
$$p = \frac{dS}{dq} = \frac{dW}{dq}$$
So we have the equation
$$0 = \frac{1}{2m}\left(\frac{dW}{dq}\right)^2 + mgq - E$$

\item Using the equation if part (a) we can solve for $W(q,\alpha)$ by 
\begin{align*}
0 &= \frac{1}{2m}\left(\frac{dW}{dq}\right)^2 + mgq - E\\
\Downarrow\\
\frac{dW}{dq} &= \left(\frac{}{}2m(E-mgq)\right)^{1/2}\\
\Downarrow\\
W &= C + \int\left(\frac{}{}2m(E-mgq)\right)^{1/2}dq\\
&= C - \frac{1}{3m^2g}\left(\frac{}{}2m(E-mgq)\right)^{3/2}
\end{align*}
Note we take $\alpha$ as a constant of motion which we see is $E$ so
$$W(q,\alpha) = C - \frac{1}{3m^2g}\left(\frac{}{}2m(\alpha-mgq)\right)^{3/2}$$

\item
We can solve for the equation of motion, $q(t)$, by taking the transformation
\begin{align*}
\beta &= \partiald{S}{\alpha} = \partiald{W}{\alpha} - t\\
&\Downarrow\\
\beta +t &= - \frac{1}{3m^2g}\frac{3}{2}\left(\frac{}{}2m(\alpha-mgq)\right)^{1/2}2m\\
&= - \frac{1}{mg}\left(\frac{}{}2m(\alpha-mgq)\right)^{1/2}\\
&\Downarrow\\
(mg(\beta+t))^2-2m\alpha &= -2m^2gq\\
&\Downarrow\\
q(t) &= \frac{\alpha}{mg} - \frac{1}{2}g(\beta+t)^2 =  \frac{\alpha}{mg} - \frac{1}{2}g\beta^2 - g\beta{t} - \frac{1}{2}gt^2
\end{align*}

\item Now we can apply the initial conditions 
$$p(0) = mv_0,\qquad q(0)=q_0$$
to find
\begin{align*}
q(0) = q_0 &= \frac{\alpha}{mg} - \frac{1}{2}g\beta^2
\end{align*}
and we take the momentum initial condition as
\begin{align*}
p(0) = mv_0 &= \left(\frac{}{}2m(\alpha-mgq(0))\right)^{1/2}\\
&= \left(\frac{}{}2m(\alpha-\alpha+1/2mg^2\beta^2)\right)^{1/2}\\
&= \pm mg\beta\\
&\Downarrow\\
\beta &= \pm\frac{v_0}{g}
\end{align*}
So replacing in terms of $q_0$ and $v_0$ we find
\begin{align*}
q(t) &=  \frac{\alpha}{mg} - \frac{1}{2}g\beta^2 - g\beta{t} - \frac{1}{2}gt^2\\
&\Downarrow\\
q(t) &= q_0 \pm v_0{t} - \frac{1}{2}gt^2
\end{align*}
\end{enumerate}

\pagebreak

\section{Problem \#3}
\begin{enumerate}[(a)]
\item For the simple harmonic oscillator we have the Hamiltonian
$$H = \frac{p^2}{2m}+\frac{1}{2}kq^2$$
which gives us the Hamilton-Jacobi equation using the same principle functional form as in
problem (2) as
$$0 = \frac{1}{2m}\left(\frac{dW}{dq}\right)^2 + \frac{1}{2}kq^2 - E$$

\item
We can solve for $W(q,\alpha)$ where we take $\alpha=E$ as
\begin{align*}
\frac{dW}{dq} &= \left(2m(\alpha-1/2kq^2)\right)^{1/2}\\
&\Downarrow\\
W &= C + \int\left(2m(\alpha-1/2kq^2)\right)^{1/2}dq
\end{align*}
Note we will solve the integral when solving for $q$

\item We solve for $q(t)$ by
\begin{align*}
\beta = \partiald{S}{\alpha} &= \partiald{W}{\alpha} - t\\
&\Downarrow\\
\beta - t &= \partiald{}{\alpha}\int\left(2m(\alpha-1/2kq^2)\right)^{1/2}dq\\
&= \int\frac{1}{2}\left(2m(\alpha-1/2kq^2)\right)^{-1/2}2mdq\\
&= \sqrt{\frac{m}{2\alpha}}\int\left(1-\frac{kq^2}{2\alpha}\right)^{-1/2}dq\\
&= \sqrt{\frac{m}{2\alpha}}\arcsin\left(\sqrt{\frac{k}{2\alpha}}q\right)\sqrt{\frac{2\alpha}{k}}\\
&\Downarrow\\
q(t) &= \sqrt{\frac{2\alpha}{k}}\sin\left(\sqrt{\frac{k}{m}}(\beta-t)\right)
\end{align*}

\item We can apply the initial conditions $q(0) = q_0$ and $p(0) = p_0$. This implies that
$$q_0 = \sqrt{\frac{2\alpha}{k}}\sin\left(\sqrt{\frac{k}{m}}\beta\right)$$
and we can find 
\begin{align*}
p_0 &= \left(2m(\alpha-1/2kq_0^2)\right)^{1/2}\\
&= \left(2m\alpha-2m\alpha\sin^2\left(\sqrt{\frac{k}{m}}\beta\right)\right)^{1/2}\\
&= \sqrt{2m\alpha}\cos\left(\sqrt{\frac{k}{m}}\beta\right)
\end{align*}
We note that we can expand $q(t)$ using trigonometric addition
\begin{align*}
q(t) &= \sqrt{\frac{2\alpha}{k}}\sin\left(\sqrt{\frac{k}{m}}(\beta-t)\right)\\
&\Downarrow\\
q(t) &= \sqrt{\frac{2\alpha}{k}}\sin\left(\sqrt{\frac{k}{m}}\beta\right)\cos\left(\sqrt{\frac{k}{m}}t\right) - \sqrt{\frac{2\alpha}{k}}\sin\left(\sqrt{\frac{k}{m}}\beta\right)\sin\left(\sqrt{\frac{k}{m}}t\right)\\
q(t) &= q_0\cos\left(\omega{t}\right) - \frac{p_0}{m\omega}\sin\left(\omega{t}\right)
\end{align*}
Where we defined $\omega^2=k/m$. We can use this result to get $p(t)$ by
\begin{align*}
p(t) &= \left(2m(\alpha-1/2kq(t)^2)\right)^{1/2}\\
&\Downarrow\\
p(t) &= \left(2m\alpha-2m\alpha{k}\frac{2\alpha}{k}\sin^2(\omega(\beta-t))\right)^{1/2}\\
&= \sqrt{2m\alpha}\cos(\omega\beta)\cos(\omega{t}) + \sqrt{2m\alpha}\sin(\omega\beta)\sin(\omega{t})\\
&= p_0\cos(\omega{t}) + q_0m\omega\sin(\omega{t})
\end{align*}
Note as we expect $p_0 = m\dot{q}$.
\end{enumerate}

\pagebreak

\section{Problem \#4}
For the system with the following kinetic energy, $T$, and potential energy, $V$
$$T= \frac{1}{2}(\dot{q_1}^2+\dot{q_2}^2)(q_1^2+q_2^2),\qquad V = (q_1^2+q_2^2)^{-1}$$
We have the Lagrangian
$$\frac{1}{2}(\dot{q_1}^2+\dot{q_2}^2)(q_1^2+q_2^2) - (q_1^2+q_2^2)^{-1}$$
which allows us to calculate the generalized momenta as
\begin{align*}
p_1 = \partiald{L}{\dot{q}_1} &= \dot{q}_1(q_1^2+q_2^2)\\
p_2 = \partiald{L}{\dot{q}_2} &= \dot{q}_2(q_1^2+q_2^2)
\end{align*}
Using these we can calculate the Hamiltonian by
\begin{align*}
H &= p_1\dot{q}_1 + p_2\dot{q}_2 - L\\
&\Downarrow\\
&= \frac{p_1^2}{q_1^2+q_2^2} + \frac{p_2^2}{q_1^2+q_2^2} - \frac{1}{2}\left(\frac{p_1^2}{(q_1^2+q_2^2)^2}\frac{p_2^2}{(q_1^2+q_2^2)^2}\right) + (q_1^2+q_2^2)^{-1}\\
&= \frac{p_1^2+p_2^2+2}{2(q_1^2+q_2^2)} 
\end{align*}
This yields the Hamilton-Jacobi equation 
$$(q_1^2+q_2^2)^{-1}\left(\frac{1}{2\dfrac{}{}}\left(\partiald{S}{q_1}\right)^2 + \frac{1}{2}\left(\partiald{S}{q_2}\right)^2 + 1\right) + \partiald{S}{t} = 0$$
where we define the principle function as
$$S(q_1,q_2,\alpha_1,\alpha_2,t) = W_1(q_1,\alpha_1,\alpha_2) + W(q_2,\alpha_1,\alpha_2) - Et$$
This makes the Hamilton-Jacobi equation become
\begin{align*}
0 &= (q_1^2+q_2^2)^{-1}\left(\frac{1}{2\dfrac{}{}}\left(\partiald{S}{q_1}\right)^2 + \frac{1}{2}\left(\partiald{S}{q_2}\right)^2 + 1\right) + \partiald{S}{t}\\
&\Downarrow\\
E(q_1^2+q_2^2) &= \frac{1}{2}\left(\partiald{W_1}{q_1}\right)^2 + \frac{1}{2}\left(\partiald{W_2}{q_2}\right)^2 + 1\\
&\Downarrow\\
Eq_1^2 - \frac{1}{2}\left(\partiald{W_1}{q_1}\right)^2 &= \frac{1}{2}\left(\partiald{W_2}{q_2}\right)^2 + 1 - Eq_2^2
\end{align*}
We note that we have a function of $q_1$ on the left and a function of $q_2$ on the right. 
This implies that both are constant and equal. This allows us to solve for $W_1$ and $W_2$.
\begin{align*}
C &= Eq_1^2 - \frac{1}{2}\left(\partiald{W_1}{q_1}\right)^2\\
&\Downarrow\\
\partiald{W_1}{q_1} &= \sqrt{2(Eq_1^2 - C)}\\
&\Downarrow\\
W_1(q_1,\alpha_1,\alpha_2) &= \int\sqrt{2(Eq_1^2 - C)}dq_1
\end{align*}
By the same process we can find $W_2$ as
\begin{align*}
C &= \frac{1}{2}\left(\partiald{W_2}{q_2}\right)^2 + 1 - Eq_2^2\\
&\Downarrow\\
W_2(q_2,\alpha_1,\alpha_2) &= \int\sqrt{2(C + Eq_2^2-1)}dq_2
\end{align*}
This gives us the principle function where we pick $\alpha_1=E$ and $\alpha_2=C$ as 
they are the constants of motion this makes our principle function become
$$S(q_1,q_2,\alpha_1,\alpha_2) = \int\sqrt{2(\alpha_1q_1^2 - \alpha_2)}dq_1 + \int\sqrt{2(\alpha_2 + \alpha_1q_2^2-1)}dq_2 - \alpha_1t$$
Where we can solve for the dynamics by
\begin{align*}
\beta_1 = \partiald{S}{\alpha_1} &= \int\frac{q_1^2}{\sqrt{2(\alpha_1q_1^2 - \alpha_2)}}dq_1 + \int\frac{q_2^2}{\sqrt{2(\alpha_2 + \alpha_1q_2^2-1)}}dq_2 - t\\
\beta_2 = \partiald{S}{\alpha_2} &= \int\frac{1}{\sqrt{2(\alpha_1q_1^2 - \alpha_2)}}dq_1 + \int\frac{1}{\sqrt{2(\alpha_2 + \alpha_1q_2^2-1)}}dq_2 
\end{align*}



\end{document}

