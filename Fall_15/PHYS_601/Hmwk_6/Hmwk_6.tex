\documentclass[11pt]{article}

\usepackage{latexsym}
\usepackage{amssymb}
\usepackage{amsthm}
\usepackage{enumerate}
\usepackage{amsmath}
\usepackage{cancel}
\numberwithin{equation}{section}

\setlength{\evensidemargin}{.25in}
\setlength{\oddsidemargin}{-.25in}
\setlength{\topmargin}{-.75in}
\setlength{\textwidth}{6.5in}
\setlength{\textheight}{9.5in}
\newcommand{\due}{October 14th, 2015}
\newcommand{\HWnum}{6}
\newcommand{\grad}{\bold\nabla}
\newcommand{\vecE}{\vec{E}}
\newcommand{\scrptR}{\vec{\mathfrak{R}}}
\newcommand{\kapa}{\frac{1}{4\pi\epsilon_0}}
\newcommand{\emf}{\mathcal{E}}
\newcommand{\unit}[1]{\ensuremath{\, \mathrm{#1}}}
\newcommand{\real}{\textnormal{Re}}
\newcommand{\Erf}{\textnormal{Erf}}
\newcommand{\sech}{\textnormal{sech}}
\newcommand{\scrO}{\mathcal{O}}
\newcommand{\levi}{\widetilde{\epsilon}}
\newcommand{\partiald}[2]{\ensuremath{\frac{\partial{#1}}{\partial{#2}}}}
\newcommand{\norm}[2]{\langle{#1}|{#2}\rangle}
\newcommand{\inprod}[2]{\langle{#1}|{#2}\rangle}
\newcommand{\average}[1]{\left\langle{#1}\right\rangle}
\newcommand{\ket}[1]{|{#1}\rangle}
\newcommand{\bra}[1]{\langle{#1}|}
\newcommand{\Resid}[2]{\ensuremath{\textnormal{Res}\left[{#1},{#2}\right]}}





\begin{document}
\begin{titlepage}
\setlength{\topmargin}{1.5in}
\begin{center}
\Huge{Physics 3310} \\
\LARGE{Principles of Electricity and Magnetism 1} \\
\Large{Professor Thomas R. Schibli} \\[1cm]

\huge{Homework \#\HWnum}\\[0.5cm]

\large{Joe Becker} \\
\large{SID: 810-07-1484} \\
\large{\due} 

\end{center}

\end{titlepage}



\section{Problem \#1}
For a particle that is projected vertically upward from the surface of the rotating earth at
a co-latitude, $\theta$. Given that we define the earth-fixed coordinates as
\begin{align*}
x&\rightarrow\textnormal{East}\\
y&\rightarrow\textnormal{North}\\
z&\rightarrow\textnormal{Vertically Up}
\end{align*}
In these coordinates the rotation vector, $\mathbf{\omega}$, is given by
$$\pmb{\omega} = \omega\cos\theta\hat{z}+\omega\sin\theta\hat{y}.$$
We note that for the rotation of earth we can use the approximation that $\omega<<1$ which 
allows us to write the \emph{Coriolis Force} to first order in $\omega$ as
$$\mathbf{F}_{cor} = -2m\pmb{\omega}\times\mathbf{v}.$$
Which allows us to calculate the force with a vertical velocity $\mathbf{v} = v\hat{z}$
\begin{align*}
\mathbf{F}_{cor} &= -2m\boldsymbol{\omega}\times\mathbf{v}\\
&\Downarrow\\
\mathbf{F}_{cor} &= -2m(\omega\cos\theta\hat{z}+\omega\sin\theta\hat{y})\times v\hat{z}\\
&= -2m(\cancelto{0}{\omega\cos\theta\hat{z}\times v\hat{z}}+\omega\sin\theta\hat{y}\times v\hat{z})\\
&= -2mv\omega\sin\theta(\hat{y}\times\hat{z})\\
&= -2mv\omega\sin\theta\hat{x}
\end{align*}
Now we have reduced this problem to a simple kinematics problem with two accelerations, one 
in the $x$ direction found above, and one in the $z$ as a constant gravitational acceleration. 
\begin{align*}
\mathbf{a}_{cor} &= -2v\omega\sin\theta\hat{x}\\
\mathbf{a}_{g} &= -g\hat{z}
\end{align*}
We note that the acceleration in the $x$ direction is dependent on the velocity that is in 
the $z$ direction. This motivates us to solve the dynamics in the $z$ direction first. Which
is simply
$$v(t) = \sqrt{2gh} - gt$$
we note that the time required to reach the peak is the time for $v(t)=0$ which yields 
$$t_{peak} = \sqrt{\frac{2h}{g}}$$
So we can find the dynamics of the $x$ position by noting the equation of motion
\begin{align*}
a(t) &= \mathbf{a}_{cor}\\
&\Downarrow\\
a(t) &= -2\omega\sin\theta\left(\sqrt{2gh}-gt\right)
\end{align*}
So we need to integrate twice to find the expression for $x(t)$ where we assume that the 
initial position and velocity is zero.
\begin{align*}
v(t) &= \int_{0}^{t}a_{cor}(t')dt'\\
&\Downarrow\\
v(t) &= \int_{0}^{t}-2\omega\sin\theta\left(\sqrt{2gh}-gt'\right)dt'\\
&= -2\omega\sin\theta\left(\sqrt{2gh}t'-\frac{g}{2}t'^2\right)
\end{align*}
And for $x$
\begin{align*}
x(t) &= \int_{0}^{t}v(t')dt'\\
&\Downarrow\\
x(t) &= \int_{0}^{t}-2\omega\sin\theta\left(\sqrt{2gh}t'-\frac{g}{2}t'^2\right)dt'\\
&= -2\omega\sin\theta\left(\frac{\sqrt{2gh}}{2}t^2-\frac{g}{6}t^3\right)
\end{align*}
So for the displacement at the peak we plug in the time.
\begin{align*}
x(t_{peak}) &= -2\omega\sin\theta\left(\frac{\sqrt{2gh}}{2}\left(\sqrt{\frac{2h}{g}}\right)^2-\frac{g}{6}\left(\sqrt{\frac{2h}{g}}\right)^3\right)\\
&= -2\omega\sin\theta\left(\frac{\sqrt{2gh}}{2}\frac{2h}{g}-\frac{g}{6}\frac{2h}{g}\sqrt{\frac{2h}{g}}\right)\\
&= -2\omega\sin\theta\left(\sqrt{\frac{2h^3}{g}}-\frac{1}{3}\sqrt{\frac{2h^3}{g}}\right)\\
&= -\frac{4}{3}\omega\sin\theta\sqrt{\frac{2h^3}{g}}
\end{align*}
And for the displacement for when it strikes the ground we note that $t=2t_{peak}$ so we 
find
\begin{align*}
x(2t_{peak}) &= -2\omega\sin\theta\left(\frac{\sqrt{2gh}}{2}\left(2\sqrt{\frac{2h}{g}}\right)^2-\frac{g}{6}\left(2\sqrt{\frac{2h}{g}}\right)^3\right)\\
&= -2\omega\sin\theta\left(4\frac{\sqrt{2gh}}{2}\frac{2h}{g}-8\frac{g}{6}\frac{2h}{g}\sqrt{\frac{2h}{g}}\right)\\
&= -2\omega\sin\theta\left(4\sqrt{\frac{2h^3}{g}}-\frac{8}{3}\sqrt{\frac{2h^3}{g}}\right)\\
&= -\frac{8}{3}\omega\sin\theta\sqrt{\frac{2h^3}{g}}
\end{align*}


\pagebreak

\section{Problem \#2}
\begin{enumerate}[(a)]
\item For a cannon that makes an angle, $\alpha$, with the horizontal that shoots due east
we note that the projectile has an initial velocity out of the barrel $V_0$ which is in
the coordinates defined in question 1 is
$$\mathbf{v} = V_0\sin\alpha\hat{z} + V_0\cos\alpha\hat{x}$$
this allows us to calculate the Coriolis Force as
\begin{align*}
\mathbf{F}_{cor} &= -2m\boldsymbol{\omega}\times\mathbf{v}\\
&\Downarrow\\
\mathbf{F}_{cor} &= -2m(\omega\cos\theta\hat{z}+\omega\sin\theta\hat{y})\times\left(V_0\sin\alpha\hat{z} + V_0\cos\alpha\hat{x}\right)\\
&= -2m\det\left(\begin{array}{ccc}
                 \hat{x}         &\hat{y}             &\hat{z}\\
                 0               &\omega\sin\theta    &\omega\cos\theta\\
                 V_0\cos\alpha   &0                   &V_0\sin\alpha-gt
           \end{array}\right)\\
&= -2m\left(\frac{}{}V_0\omega\sin\theta\sin\alpha - \omega\sin\theta{gt}\hat{x} + V_0\omega\cos\theta\cos\alpha\hat{y} - V_0\omega\sin\theta\cos\alpha\hat{z}\right)
\end{align*}
We note that included the constant acceleration due to gravity in the $z$ direction. 
We can solve for the time it reaches the peak of $z$ by
\begin{align*}
0 &= V_0\sin\alpha + (2V_0\omega\sin\theta\cos\alpha-g)t\\
&\Downarrow\\
t_{peak} &= \frac{V_0\sin\alpha}{g-2V_0\omega\sin\theta\cos\alpha} \approx \frac{V_0\sin\alpha}{g}
\end{align*}
Note in the approximation we use the fact that $\omega<<g$. So, the deflection is given by 
the $y$ component for the time, $t=2t_{peak}$, which is 
\begin{align*}
y(2t_{peak}) &= \frac{1}{2}a_yt^2\\
&= -\frac{1}{2}2V_0\omega\cos\theta\cos\alpha\left(2\frac{V_0\sin\alpha}{g}\right)^2\\
&= -\frac{4V_0^3}{g^2}\omega\cos\theta\sin^2\alpha\cos\alpha
\end{align*}

\item If the range neglecting the Coriolis Force is given by $R$ where we note that 
$$R = \frac{V_0^2\sin\alpha\cos\alpha}{g}.$$
Using the $x$ component of the force we found in part (a) we can write 
$$x(t) = V_0\cos\alpha{t} - V_0\omega\sin\theta\sin\alpha{t^2} + \frac{1}{3}\omega\sin\theta{g}t^3$$
Note the expansion of $2t_{peak}$ to first order in $\omega$
$$\frac{V_0\sin\alpha}{g-2V_0\omega\sin\theta\cos\alpha} = \frac{V_0\sin\alpha}{g}\frac{1}{1-2V_0\omega\sin\theta\cos\alpha/g} = \frac{V_0\sin\alpha}{g}\left(1 + 2\frac{2V_0\omega\sin\theta\cos\alpha}{g} +\scrO(\omega^2)\right)$$
We note that the zeroth order term is just $R$ which implies that the change in range is
given by
\begin{align*}
\delta{R} &= V_0\cos\alpha\frac{V_0\sin\alpha}{g}\frac{2V_0\omega\sin\theta\cos\alpha}{g} - V_0\omega\sin\theta\sin\alpha\left(\frac{V_0\sin\alpha}{g}\right)^2 + \frac{1}{3}\omega\sin\theta\left(\frac{V_0\sin\alpha}{g}\right)^3 + \scrO(\omega^2)\\
&= \frac{2V_0^3}{g^2}\sin\alpha\cos^2\alpha\omega\sin\theta - \frac{V_0^3}{g^2}\sin^3\alpha\omega\sin\theta + \frac{1}{3}\omega\sin\theta\frac{V_0^3\sin^3\alpha}{g^3}\\
&= \frac{V_0^3}{g^2}\sin\alpha\omega\sin\theta\left(2\cos^2\alpha-\sin^2\alpha + \frac{1}{3}\sin^2\alpha\right)\\
&= \frac{2V_0^3}{g^2}\sin\alpha\omega\sin\theta\left(\cos^2\alpha-\frac{1}{3}\sin^2\alpha\right)\\
&= \frac{2V_0^2\sin\alpha\cos\alpha}{g}\frac{V_0}{g}\omega\sin\theta\left(\cos\alpha-\frac{1}{3}\tan\alpha\sin\alpha\right)\\
&= \frac{2R}{g}V_0\omega\sin\theta\left(\cos\alpha-\frac{1}{3}\tan\alpha\sin\alpha\right)\\
&= \frac{2R}{g}\sqrt{\frac{gR}{2\sin\alpha\cos\alpha}}\omega\sin\theta\left(\cos\alpha-\frac{1}{3}\tan\alpha\sin\alpha\right)\\
&= \left(\frac{2R^3}{g}\right)^{1/2}\omega\sin\theta\left(\frac{\cos\alpha}{\sqrt{\sin\alpha\cos\alpha}}-\frac{1}{3}\frac{\tan\alpha\sin\alpha}{\sqrt{\sin\alpha\cos\alpha}}\right)\\
&= \left(\frac{2R^3}{g}\right)^{1/2}\omega\sin\theta\left((\cot\alpha)^{1/2}-\frac{1}{3}(\tan\alpha)^{3/2}\right)
\end{align*}
\end{enumerate}

\pagebreak

\section{Problem \#3}
Given that the rate precession of a Foucault Pendulum is 
\begin{equation}
\Omega_p = -\omega\cos\theta
\label{Fouc}
\end{equation}
where $\omega$ is the angular frequency of the rotation of earth and $\theta$ is the
angle of co-latitude. Using equation \ref{Fouc} we can find the precession period of a
Foucault Pendulum at a given co-latitude, $\theta$, by 
$$T = \frac{2\pi}{\Omega_p} = \frac{2\pi}{-\omega\cos\theta} =  \frac{2\pi}{-2\pi/24\unit{hrs}\cos\theta} = \frac{24\unit{hrs}}{-\cos\theta}$$
We note that for $\theta =0$ we have a period of $24\unit{hrs}$ and for $\theta=\pi/2$
which corresponds to the pendulum sitting on the equator we have an infinite period
This follows with what we is what we expect. So for various locations on earth we 
calculate

\begin{center}
\begin{tabular}{c|c|c|c}
&Location     &Co-latitude($\theta$)   &Precession Period($T$)\\
\hline
(a)      &Moscow, Russia         &$34.25^{\circ}$     &$1\unit{day}\ 5\unit{hrs}\ 2\unit{min}$\\
(b)      &Rome, Italy            &$47.17^{\circ}$     &$1\unit{day}\ 11\unit{hrs}\ 18\unit{min}$\\
(c)      &Beijing, China         &$50.07^{\circ}$     &$1\unit{day}\ 13\unit{hrs}\ 23\unit{min}$\\
(d)      &Washington DC, USA     &$51.10^{\circ}$     &$1\unit{day}\ 14\unit{hrs}\ 13\unit{min}$\\
(e)      &Seoul, South Korea     &$52.43^{\circ}$     &$1\unit{day}\ 15\unit{hr}\ 21\unit{min}$\\
(f)      &Tokyo, Japan           &$54.33^{\circ}$     &$1\unit{day}\ 15\unit{hrs}\ 46\unit{min}$\\
(g)      &College Station, Texas &$59.41^{\circ}$     &$1\unit{day}\ 23\unit{hrs}\ 9\unit{min}$\\
(h)      &Singapore              &$88.63^{\circ}$     &$41\unit{days}\ 20\unit{hrs}$
\end{tabular}
\end{center}

\end{document}

