\documentclass[11pt]{article}

\usepackage{latexsym}
\usepackage{amssymb}
\usepackage{amsthm}
\usepackage{enumerate}
\usepackage{amsmath}
\usepackage{cancel}
\numberwithin{equation}{section}

\setlength{\evensidemargin}{.25in}
\setlength{\oddsidemargin}{-.25in}
\setlength{\topmargin}{-.75in}
\setlength{\textwidth}{6.5in}
\setlength{\textheight}{9.5in}
\newcommand{\due}{November 25th, 2015}
\newcommand{\HWnum}{11}
\newcommand{\grad}{\bold\nabla}
\newcommand{\vecE}{\vec{E}}
\newcommand{\scrptR}{\vec{\mathfrak{R}}}
\newcommand{\kapa}{\frac{1}{4\pi\epsilon_0}}
\newcommand{\emf}{\mathcal{E}}
\newcommand{\unit}[1]{\ensuremath{\, \mathrm{#1}}}
\newcommand{\real}{\textnormal{Re}}
\newcommand{\Erf}{\textnormal{Erf}}
\newcommand{\sech}{\textnormal{sech}}
\newcommand{\scrO}{\mathcal{O}}
\newcommand{\levi}{\widetilde{\epsilon}}
\newcommand{\partiald}[2]{\ensuremath{\frac{\partial{#1}}{\partial{#2}}}}
\newcommand{\norm}[2]{\langle{#1}|{#2}\rangle}
\newcommand{\inprod}[2]{\langle{#1}|{#2}\rangle}
\newcommand{\average}[1]{\left\langle{#1}\right\rangle}
\newcommand{\ket}[1]{|{#1}\rangle}
\newcommand{\bra}[1]{\langle{#1}|}
\newcommand{\Resid}[2]{\ensuremath{\textnormal{Res}\left[{#1},{#2}\right]}}





\begin{document}
\begin{titlepage}
\setlength{\topmargin}{1.5in}
\begin{center}
\Huge{Physics 3310} \\
\LARGE{Principles of Electricity and Magnetism 1} \\
\Large{Professor Thomas R. Schibli} \\[1cm]

\huge{Homework \#\HWnum}\\[0.5cm]

\large{Joe Becker} \\
\large{SID: 810-07-1484} \\
\large{\due} 

\end{center}

\end{titlepage}



\section{Problem \#1}
\begin{enumerate}[(a)]
\item For a free falling object with the Hamiltonian
$$H = \frac{p^2}{2m} + mgq$$
with initial conditions $q_0$ and $p_0=0$. We know from kinematics that the time from the
initial height to the ground is given by
$$t = \sqrt{\frac{2q_0}{g}}$$
this implies the period of oscillation is twice this time or
$$T =  2\sqrt{\frac{2q_0}{g}}$$

\item We can compute the action variable, $J$, for this system by noting that the 
Hamiltonian is equal to a constant energy, $E$, and solving for $p$ as
\begin{align*}
E &= \frac{p^2}{2m} + mgq\\
&\Downarrow\\
p &= \sqrt{2mE - 2m^2gq}
\end{align*}
This allows us to integrate $p$ over a cycle of $q$ by
\begin{align*}
J &= \frac{1}{2\pi}\oint{pdq}\\
&= \frac{1}{2\pi}2\int_{0}^{q_0}\sqrt{2mE - 2m^2gq}dq\\
&= -\frac{2}{3\pi}\left.\frac{(2mE - 2m^2gq)^{3/2}}{2m^2g}\right|_0^{q_0}\\
&= -\frac{1}{3\pi{m^2g}}\left(\frac{}{}(2mE - 2m^2gq_0)^{3/2} - (2mE)^{3/2}\right)
\end{align*}
We note that at $q_0$ we have $E = mgq_0$ so $q_0=E/mg$ which replacing yields
$$J = \frac{(2mE)^{3/2}}{3\pi{m^2g}}$$

\item Using the result from part (b) we can calculate the angular frequency by
$$\omega = \partiald{E}{J}$$
where we first solve for $E(J)$ as
\begin{align*}
J &= \frac{(2mE)^{3/2}}{3\pi{m^2g}}\\
&\Downarrow\\
E &= \frac{(3\pi{m^2g}J)^{2/3}}{2m}
\end{align*}
So we can calculate the derivative with respect to $J$ as
$$\omega = \partiald{E}{J} = \frac{(3\pi{m^2g})^{2/3}}{3m}J^{-1/3}$$
but if we replace $J$ in terms of $q_0$ we have
\begin{align*}
\omega &= \frac{(3\pi{m^2g})^{2/3}}{3m}\left(\frac{3\pi{m^2g}}{(2m^2gq_0)^{3/2}}\right)^{1/3}\\
&= \frac{\pi{mg}}{(2m^2gq_0)^{1/2}}\\
&= \pi\sqrt{\frac{g}{2q_0}}
\end{align*}

\item Now we know that period is given by
$$T = \frac{2\pi}{\omega} = \frac{2\pi}{\pi}\sqrt{\frac{2q_0}{g}} = 2\sqrt{\frac{2q_0}{g}}$$
which agrees with the result from part (a).

\end{enumerate}

\pagebreak

\section{Problem \#2}
We can calculate the integral
$$I = \int_{a}^{b} \frac{dx}{x}\sqrt{(x-a)(b-x)}$$
by expanding the product and bringing in the $x^{-1}$ term to yield
$$I = \int_{a}^{b}dx\sqrt{-1+(a+b)/x-ab/x^2}$$
which allows us to use integration by parts which states 
$$\int udv = uv - \int vdu$$
where we take
\begin{align*}
dv = dx &\qquad u = \sqrt{-1+(a+b)/x-ab/x^2}\\
v = x &\qquad du = \frac{1}{2}\left(-1+\frac{a+b}{x}-\frac{ab}{x^2}\right)^{-1/2}\left(-\frac{a+b}{x^2}+\frac{2ab}{x^3}\right)
\end{align*}
Which implies that
\begin{align*}
I &= \left.\sqrt{(x-a)(b-x)}\right|_{a}^{b} - \frac{1}{2}\int_{a}^{b}x\left(-1+\frac{a+b}{x}-\frac{ab}{x^2}\right)^{-1/2}\left(-\frac{a+b}{x^2}+\frac{2ab}{x^3}\right)\\
&= 0 - \frac{1}{2}\int_{a}^{b}x\left(-1+\frac{a+b}{x}-\frac{ab}{x^2}\right)^{-1/2}\left(-\frac{a+b}{x^2}+\frac{2ab}{x^3}\right)\\
&= - \frac{1}{2}\int_{a}^{b}\frac{dx}{\sqrt{-1+(a+b)/{x}-{ab}/{x^2}}}\left(-\frac{a+b}{x}+\frac{2ab}{x^2}\right)\\
&= \frac{a+b}{2}\int_{a}^{b}\frac{dx}{\sqrt{-x^2+(a+b)x-ab}} - ab\int_{a}^{b}\frac{dx}{\sqrt{-x^4+(a+b)x^3-abx^2}}\\
&= \frac{a+b}{2}\int_{a}^{b}\frac{dx}{\sqrt{-(x-(a+b)/2)^2 + ((a-b)/2)^2}} - ab\int_{a}^{b}\frac{dx}{\sqrt{-x^4+(a+b)x^3-abx^2}}
\end{align*}
We can calculate the first integral using a substitution 
$$\frac{a-b}{2}\sin{u} = x - \frac{a+b}{2}$$
which has the infinitesimal of
$$\frac{a-b}{2}\cos{u}du = dx$$
So the first term becomes
\begin{align*}
\frac{a+b}{2}\int_{a}^{b}\frac{dx}{\sqrt{-(x-(a+b)/2)^2 + ((a-b)/2)^2}} &= (a+b)\int_{u(a)}^{u(b)}\frac{(a-b)/2\cos{u}}{\sqrt{((a-b)/2)^2(1-\sin^2u)}}du\\
&= \frac{a+b}{2}\int_{u(a)}^{u(b)}\frac{(a-b)/2\cos{u}}{((a-b)/2\cos{u}}du\\
&= \frac{a+b}{2}\int_{u(a)}^{u(b)}du\\
&= \frac{a+b}{2}\left.\arcsin\left(\frac{2x-a-b}{a-b}\right)\right|_{a}^{b} = \frac{a+b}{2}\pi
\end{align*}
Now for the second integral we use the same substitution so that we are left with
\begin{align*}
-ab\int_{a}^{b}\frac{dx}{\sqrt{-x^4+(a+b)x^3-abx^2}} &= -ab\int_{u(a)}^{u(b)}\frac{du}{x(u)}\\
&= -ab\int_{u(a)}^{u(b)}\frac{2}{(a-b)\sin(u)+a+b}\\
&= -2\frac{ab}{\sqrt{ab}}\left.\arctan\left(\frac{(a+b)\tan(u/2)+a-b}{2\sqrt{ab}}\right)\right|_{-\pi}^{\pi}\\
&= -2\frac{ab}{\sqrt{ab}}\frac{\pi}{2} = -\pi\sqrt{ab}
\end{align*}
Therefore we combine terms to find $I$ as
$$I = \pi\left(\frac{a+b}{2}-\sqrt{ab}\right)$$

\pagebreak

\section{Problem \#3}
\begin{enumerate}[(a)]
\item For the isotropic two dimensional harmonic oscillator we have the Hamiltonian
$$H = \frac{p_r^2}{2m} + \frac{p_{\theta}^2}{2mr^2} + \frac{1}{2}m\omega_0^2r^2$$
We can calculate $J_{\theta}$ by noting for a central force problem $p_{\theta}=L$ where $L$
is a constant. Therefore
$$J_{\theta} = \frac{1}{2\pi}\oint{p_\theta}d\theta = \frac{L}{2\pi}\int_{0}^{2\pi}d\theta = L$$
Using this result and solving for $p_r$ as
$$p_r = \sqrt{2m(E-L^2/2mr^2-1/2m\omega_0^2r^2)}$$ 
we can calculate by using the result from problem (2)
\begin{align*}
J_{\theta} = \frac{1}{2\pi}\oint{p_r}dr &= \frac{1}{2\pi}\oint{dr}\sqrt{2m(E-L^2/2mr^2-1/2m\omega_0^2r^2)}\\
&= \frac{1}{2\pi}\oint{dr}\frac{1}{r}\sqrt{2mEr^2-L^2-m^2\omega_0^2r^4}\\
&= \frac{m\omega_0}{4\pi}\oint{dr'}\frac{1}{r'}\sqrt{-m^2\omega_0^2r'^2+(2mE/m^2\omega_0^2)r'-L^2/m^2\omega_0^2}\\
&= \frac{m\omega_0}{4\pi}\pi\left(\frac{mE}{m^2\omega_0^2}-\frac{L}{m\omega_0}\right)\\
&= \frac{E}{4\omega_0}-\frac{L}{4}
\end{align*}
Note that we used a change of variables $r'=r^2$ with $dr' = dr/2r$

\item Solving the result from part (b) we have
$$E = 4\omega_0J_r + \omega_0L$$
So we can calculate the angular frequencies as
$$\omega_r = \partiald{E}{J_r} = 4\omega_0,\qquad \omega_{\theta}=\partiald{E}{L} = \omega_0$$
We can see that this results in a closed orbit due to the fact that
$$\frac{\omega_r}{\omega_{\theta}} = 4$$
which implies that we have an integer multiple of $\omega_{\theta}$ in radial oscillations.

\item The correct period is given by $\omega_{\theta}/2\pi$ because the frequency in the $\theta$ 
direction is the motion in which the radial motion precesses about.
\end{enumerate}

\pagebreak

\section{Problem \#4}
\begin{enumerate}[(a)]
\item We can calculate a canonical perturbation given by the form
$$H(p,q,\epsilon) = H_0(p,q) + \epsilon H_1(p,q),\qquad \textnormal{where}\qquad H_0(p,q) = \frac{p^2}{2m}+\frac{1}{2}m\omega^2q^2$$
for a perturbation of the form
$$H_1(p,q) = \frac{1}{2}m\omega_1^2q^2$$
we can calculate the first order term to the corrected Kamiltonian given by
$$K(J) = \omega_0J + \epsilon K_1(J)$$
where $K_1(J)$ is given by the average over $\phi_0$ or
$$K_1(J) = \left\langle{H_1}\frac{}{}\right\rangle_{\phi_0}$$
So we use the solution of the unperturbed state
$$q=\sqrt{\frac{2J}{m\omega_0}}\sin\phi_0$$ 
So we calculate
\begin{align*}
\left\langle{H_1}\frac{}{}\right\rangle_{\phi_0} &= \frac{1}{2\pi}\int_{0}^{2\pi}\frac{1}{2}m\omega_1^2\frac{2J}{m\omega_0}\sin^2\phi_0d\phi_0\\
&= \frac{1}{2\pi}\frac{J\omega_1^2}{\omega_0}\int_{0}^{2\pi}\sin^2\phi_0d\phi_0\\
&= \frac{J\omega_1^2}{2\omega_0}
\end{align*}
Which yields
$$K(J) = \omega_0J + \epsilon \frac{J\omega_1^2}{2\omega_0}$$
which gives us the corrected angular frequency
$$\omega = \partiald{K}{J} = \omega_0 + \epsilon\frac{\omega_1^2}{2\omega_0}$$
We can compare this to the exact solution $\omega = \sqrt{\omega_0^2+\omega_1^2}$ by
expanding to first order
\begin{align*}
\sqrt{\omega_0^2+\omega_1^2} &= \omega_0\sqrt{1+\frac{\omega_1^2}{\omega_0^2}}\\
&= \omega_0\left(1+\frac{1}{2}\frac{\omega_1^2}{\omega_0^2}\right)\\
&= \omega_0+\frac{1}{2}\frac{\omega_1^2}{\omega_0}
\end{align*}
So we see that it agrees to first order for $\omega_1/\omega_0$ small.

\item We can repeat this process for the perturbation
$$H_1(p,q) = \frac{1}{6}mq^6$$
which yields
\begin{align*}
\left\langle{H_1}\frac{}{}\right\rangle_{\phi_0} &= \frac{1}{2\pi}\int_{0}^{2\pi}H_1d\phi_0\\
&= \frac{1}{2\pi}m\frac{1}{6}\left(\frac{2J}{m\omega_0}\right)^{3}\int_{0}^{2\pi}\sin^6\phi_0d\phi_0\\
&= \frac{1}{2\pi}m\frac{1}{6}\left(\frac{2J}{m\omega_0}\right)^{3}\frac{5}{8}\pi\\
&= \frac{5}{12}\frac{J^3}{m^2\omega_0^3}
\end{align*}
So we have the corrected Kamiltonian as
$$K(J) = \omega_0J + \epsilon \frac{5}{12}\frac{J^3}{m^2\omega_0^3}$$
which yields
$$\omega = \omega_0 + \epsilon\frac{5}{4}\frac{J^2}{m^2\omega_0^3}$$
where we note that 
$$J = \frac{E}{\omega_0} = \frac{1}{2}m\omega_0q_0^2$$
which allows us to say that
$$\omega = \omega_0 + \epsilon\frac{5}{16}\frac{q_0^4}{\omega_0^2}$$


\end{enumerate}

\pagebreak

\end{document}

