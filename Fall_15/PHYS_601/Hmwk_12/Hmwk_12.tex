\documentclass[11pt]{article}

\usepackage{latexsym}
\usepackage{amssymb}
\usepackage{amsthm}
\usepackage{enumerate}
\usepackage{amsmath}
\usepackage{cancel}
\numberwithin{equation}{section}

\setlength{\evensidemargin}{.25in}
\setlength{\oddsidemargin}{-.25in}
\setlength{\topmargin}{-.75in}
\setlength{\textwidth}{6.5in}
\setlength{\textheight}{9.5in}
\newcommand{\due}{December 4th, 2015}
\newcommand{\HWnum}{12}
\newcommand{\grad}{\bold\nabla}
\newcommand{\vecE}{\vec{E}}
\newcommand{\scrptR}{\vec{\mathfrak{R}}}
\newcommand{\kapa}{\frac{1}{4\pi\epsilon_0}}
\newcommand{\emf}{\mathcal{E}}
\newcommand{\unit}[1]{\ensuremath{\, \mathrm{#1}}}
\newcommand{\real}{\textnormal{Re}}
\newcommand{\Erf}{\textnormal{Erf}}
\newcommand{\sech}{\textnormal{sech}}
\newcommand{\scrO}{\mathcal{O}}
\newcommand{\levi}{\widetilde{\epsilon}}
\newcommand{\partiald}[2]{\ensuremath{\frac{\partial{#1}}{\partial{#2}}}}
\newcommand{\norm}[2]{\langle{#1}|{#2}\rangle}
\newcommand{\inprod}[2]{\langle{#1}|{#2}\rangle}
\newcommand{\average}[1]{\left\langle{#1}\right\rangle}
\newcommand{\ket}[1]{|{#1}\rangle}
\newcommand{\bra}[1]{\langle{#1}|}
\newcommand{\Resid}[2]{\ensuremath{\textnormal{Res}\left[{#1},{#2}\right]}}





\begin{document}
\begin{titlepage}
\setlength{\topmargin}{1.5in}
\begin{center}
\Huge{Physics 3310} \\
\LARGE{Principles of Electricity and Magnetism 1} \\
\Large{Professor Thomas R. Schibli} \\[1cm]

\huge{Homework \#\HWnum}\\[0.5cm]

\large{Joe Becker} \\
\large{SID: 810-07-1484} \\
\large{\due} 

\end{center}

\end{titlepage}



\section{Problem \#1}
\begin{enumerate}[(a)]
\item For the generator
$$F_2(q,P) = qP + \epsilon{H(q,P)}$$
where the Hamiltonian is separable and of the form
$$H(q,p) = \frac{p^2}{2m}+V(q)$$
we have the Canonical Transformations given by
\begin{align*}
Q &= q + \epsilon\frac{p}{m} - \frac{\epsilon^2}{m}\frac{dV}{dq}\\
P &= p - \epsilon\frac{dV}{dq}
\end{align*}
This allows us to calculate the \emph{Jacobian matrix of transformation}, $M$, by
$$M = \left(\begin{array}{cc}
      \dfrac{dQ}{dq}      &\dfrac{dQ}{dp}\\
\\
      \dfrac{dP}{dq}      &\dfrac{dP}{dp}
      \end{array}\right)
= \left(\begin{array}{cc}
      1 - \dfrac{\epsilon^2}{m}\dfrac{d^2V}{dq^2}  &\dfrac{\epsilon}{m}\\
\\
       -\epsilon\dfrac{d^2V}{dq^2}      &1
      \end{array}\right)$$
This allows us to calculate the determinant of $M$ as
\begin{align*}
\det(M) &= \left(1 - \dfrac{\epsilon^2}{m}\dfrac{d^2V}{dq^2}\right) - \frac{\epsilon}{m}\left(-\epsilon\dfrac{d^2V}{dq^2}\right)\\
&= 1 - \dfrac{\epsilon^2}{m}\dfrac{d^2V}{dq^2} + \frac{\epsilon^2}{m}\dfrac{d^2V}{dq^2} = 1
\end{align*}
Therefore $\det(M)=1$ as we expect for a canonical transformation.

\item We can repeat this for the generator
$$F_3(Q,P) = -pQ + \epsilon{H(Q,p)}$$
which has the canonical transformation given by
\begin{align*}
Q &= q + \epsilon\frac{p}{m} - \frac{\epsilon^2}{m}\frac{dV}{dq}\\
P &= p - \epsilon\frac{dV}{dq}
\end{align*}
We note that the \emph{Poisson Brackets} give the $\det(M)$ so that we can calculate
\begin{align*}
\det(M) = \{Q,P\} &= \partiald{Q}{q}\partiald{P}{p} - \partiald{Q}{p}\partiald{P}{q}\\
&= 1 - \frac{\epsilon^2}{m}\frac{d^2V}{dq^2}  + \frac{\epsilon^2}{m}\frac{d^2V}{dq^2}\\
&= 1
\end{align*}
\end{enumerate}

\pagebreak

\section{Problem \#2}
\begin{enumerate}[(a)]
\item For a Lie operator $\hat{S}$ we are given the identity
$$\hat{S}(fg) = (\hat{S}f)g + f(\hat{S}g)$$
which we can use to show that
$$e^{\epsilon\hat{S}}(fg) = (e^{\epsilon\hat{S}}f)(e^{\epsilon\hat{S}}g).$$
First we expand the exponential to get
$$e^{\epsilon\hat{S}} = 1 + \epsilon\hat{S} + \frac{1}{2}(\epsilon\hat{S})^2 + \frac{1}{3!}(\epsilon\hat{S})^3 + ...$$
So we act the expansion on the product $(fg)$ to get
\begin{align*}
e^{\epsilon\hat{S}}(fg) &= \left(1 + \epsilon\hat{S} + \frac{1}{2}(\epsilon\hat{S})^2 + \frac{1}{3!}(\epsilon\hat{S})^3 + ...\right)(fg)\\
&= fg + \epsilon\hat{S}(fg) + \frac{1}{2}(\epsilon\hat{S})^2(fg) + \frac{1}{3!}(\epsilon\hat{S})^3(fg) + ...\\
&= fg + \epsilon(\hat{S}f)g + \epsilon{f}(\hat{S}g) + \frac{1}{2}\epsilon^2\hat{S}\left((\hat{S}f)g + f(\hat{S}g)\right) + ...\\
&= fg + \epsilon(\hat{S}f)g + \epsilon{f}(\hat{S}g) + \frac{1}{2}\epsilon^2\left((\hat{S}^2f)g + 2\hat{S}f(\hat{S}g) + f(\hat{S}^2g)\right) + ...\\
&= fg + \epsilon(\hat{S}f)g + \epsilon{f}(\hat{S}g) +  \epsilon^2\hat{S}f(\hat{S}g) + \frac{1}{2}\epsilon^2(\hat{S}^2f)g  + \frac{1}{2}\epsilon^2f(\hat{S}^2g) + ...\\
&= \left(f + \epsilon\hat{S}f + \frac{1}{2}\epsilon^2\hat{S}^2f+...\right)\left(g + \epsilon\hat{S}g + \frac{1}{2}\epsilon^2\hat{S}^2g+...\right)\\
&=  (e^{\epsilon\hat{S}}f)(e^{\epsilon\hat{S}}g)
\end{align*}

\item Given 
$$\hat{S}\{f,g\} = \{\hat{S}f,g\} + \{f,\hat{S}g\}$$
we can show that 
\begin{align*}
e^{\epsilon\hat{S}}\{f,g\} &= \left(1 + \epsilon\hat{S} + \frac{1}{2}(\epsilon\hat{S})^2 + \frac{1}{3!}(\epsilon\hat{S})^3 + ...\right)\{f,g\}\\
&= \left(\{f,g\} + \epsilon\hat{S}\{f,g\} + \frac{1}{2}(\epsilon\hat{S})^2\{f,g\} + \frac{1}{3!}(\epsilon\hat{S})^3\{f,g\} + ...\right)\\
&= \{f,g\} + \epsilon\{\hat{S}f,g\} + \epsilon\{f,\hat{S}g\} + \frac{1}{2}\epsilon^2\{f,\hat{S}^2g\} + \frac{1}{2}\epsilon^2\{\hat{S}^2f,g\} + ...\\
&= \left\{\left(f+\epsilon\hat{S}f+\frac{\epsilon^2}{2}\hat{S}^2f+...\right),\left(g+\epsilon\hat{S}g+\frac{\epsilon^2}{2}\hat{S}^2g+...\right)\right\}\\
&= \{e^{\epsilon\hat{S}}f, e^{\epsilon\hat{S}}g\}
\end{align*}
\end{enumerate}

\pagebreak

\section{Problem \#3}
For the harmonic oscillator with Hamiltonian
$$H = \frac{p^2}{2m}+\frac{1}{2}kq^2$$
we note that the Lie operator is given as
$$\hat{H} = \{\cdot,H\} = \frac{p}{m}\partiald{}{q} - kq\partiald{}{p}$$
This allows us to calculate the transformations
\begin{align*}
q(t) = e^{t\hat{H}}q &= \left(1 + t\hat{H} + \frac{1}{2}t^2\hat{H}^2 + \frac{1}{3!}t^2\hat{H}^3 +...\right)q\\
p(t) = e^{t\hat{H}}p &= \left(1 + t\hat{H} + \frac{1}{2}t^2\hat{H}^2 + \frac{1}{3!}t^2\hat{H}^3 +...\right)p\\
\end{align*}
Where we take $q$ and $p$ to be the initial position and momentum. So we can calculate $q(t)$
as
\begin{align*}
q(t) &= q + t\hat{H}q + \frac{1}{2}t^2\hat{H}^2q + \frac{1}{3!}t^3\hat{H}^3q +...\\
&= q + t\frac{p}{m} + \frac{1}{2}t^2\hat{H}\frac{p}{m} + \frac{1}{3!}t^3\hat{H}^2\frac{p}{m} +...\\
&= q + t\frac{p}{m} - \frac{1}{2}t^2\frac{k}{m}q - \frac{1}{3!}t^3\hat{H}\frac{k}{m}q +...\\
&= q + t\frac{p}{m} - \frac{1}{2}t^2\frac{k}{m}q - \frac{1}{3!}t^3\frac{k}{m}\frac{p}{m} +...\\
&= q\left(1 - \frac{1}{2}\left(\sqrt{\frac{k}{m}}t\right)^2 + \frac{1}{4!}\left(\sqrt{\frac{k}{m}}t\right)^4 + ...\right) + \frac{p}{m}\sqrt{\frac{m}{k}}\left(t - \frac{1}{3!}\left(\sqrt{\frac{k}{m}}t\right)^3+...\right)\\
&= q\cos(\omega{t}) + \frac{p}{m\omega}\sin(\omega{t})
\end{align*}
Where we define $\omega^2=k/m$. We repeat for $p(t)$ as
\begin{align*}
p(t) &= p + t\hat{H}p + \frac{1}{2}t^2\hat{H}^2p + \frac{1}{3!}t^3\hat{H}^3p +...\\
&= p - tkq - \frac{1}{2}t^2\hat{H}kq - \frac{1}{3!}t^3\hat{H}^2kq +...\\
&= p - tkq - \frac{1}{2}t^2\omega^2p - \frac{1}{3!}t^3\hat{H}\omega^2p +...\\
&= p - tkq - \frac{1}{2}t^2\omega^2p + \frac{1}{3!}t^3\omega^2kq +...\\
&=p\left(1-\frac{1}{2}(\omega{t})^2+...\right) - \frac{qk}{\omega}\left(t-\frac{1}{3!}(\omega{t})^2+...\right)\\
&=p\cos(\omega{t}) - qm\omega\sin(\omega{t})
\end{align*}
So we note that the transformation given by $e^{t\hat{H}}$ yields the exact solution.

\pagebreak

\section{Problem \#4}
For the same harmonic oscillator in problem \#3 we can use the transformation
\begin{align*}
q(t) &= \mathcal{T}q\\
p(t) &= \mathcal{T}p
\end{align*}
Where we take
$$\mathcal{T} =  \exp\left(\frac{1}{2}t\hat{T}\right) \exp\left(t\hat{V}\right)\exp\left(\frac{1}{2}t\hat{T}\right)$$
we note that the transformation $e^{t\hat{H}}$ yields the exact solution. We can write this 
to second order in $t$ as
\begin{align*}
e^{t\hat{H}} = e^{t(\hat{V}+\hat{T})} &= 1 + t(\hat{V}+\hat{T}) + \frac{1}{2}t^2(\hat{V}+\hat{T})^2 + \scrO(t^3)\\ 
&= 1 + t(\hat{V}+\hat{T}) + \frac{1}{2}t^2(\hat{V}^2 + \hat{V}\hat{T} + \hat{T}\hat{V} +\hat{T}^2) + \scrO(t^3)\\ 
\end{align*}
We can see that if we expand $\mathcal{T}$ to second order we see that
\begin{align*}
\mathcal{T} &=  \left(1 + t\frac{1}{2}\hat{T} + \frac{1}{2}t^2\frac{1}{2}\hat{T}^2\right)\left(1 + t\hat{V} + \frac{1}{2}t^2\hat{V}^2\right)\left(1 + t\frac{1}{2}\hat{T} + \frac{1}{2}t^2\frac{1}{2}\hat{T}^2\right) + \scrO(t^3)\\
&=  \left(1 + t\frac{1}{2}\hat{T} + \frac{1}{2}t^2\frac{1}{2}\hat{T}^2\right)\left(1 + \frac{1}{2}t\hat{T} + \frac{1}{4}t^2\hat{T}^2 + t\hat{V} + \frac{1}{2}t^2\hat{V}\hat{T} + \frac{1}{2}t^2\hat{V}^2\right) + \scrO(t^3)\\
&=  1 + t\frac{1}{2}\hat{T} + \frac{1}{2}t^2\frac{1}{2}\hat{T}^2 + t\hat{V} + \frac{1}{2}t^2\hat{V}\hat{T} + \frac{1}{2}t^2\hat{V}^2 + \frac{1}{2}t\hat{T} + \frac{1}{4}t\hat{T}^2 + \frac{1}{2}t^2\hat{T}\hat{V} + \scrO(t^3)\\
&= 1 + t(\hat{T}+\hat{V}) + \frac{1}{2}t^2\left(\frac{3}{2}\hat{T}^2 + \hat{V}\hat{T} + \hat{T}\hat{V} + \hat{V}^2\right) + \scrO(t^3)
\end{align*}
We note that the $\hat{T}^2$ term has an additional factor, but this term is zero when acting
on $q$ or $p$. Therefore we can say that
\begin{align*}
e^{t\hat{H}}q &= \mathcal{T}q + \scrO(t^3)\\
e^{t\hat{H}}p &= \mathcal{T}p + \scrO(t^3)
\end{align*}
which implies that the transformation $\mathcal{T}$ solves the simple harmonic motion to 
second order in $t$.

\pagebreak

\section*{Bonus Problem}
For the transformation given by
$$\mathcal{T}\left(\frac{t}{2-s}\right)\mathcal{T}\left(\frac{-st}{2-s}\right)\mathcal{T}\left(\frac{t}{2-s}\right)$$
we can show that this solves the harmonic oscillator to fourth order in $t$ by noting that 
the operators $\hat{T}^n$ and $\hat{V}^n$ are zero when acting on $p$ or $q$ for this system.
Therefore we can expand neglecting those terms to get
\begin{align*}
e^{t\hat{H}} &= 1 + t(\hat{V}+\hat{T}) + \frac{1}{2}t^2(\hat{V}\hat{T} + \hat{T}\hat{V}) + \frac{1}{3!}t^3(\hat{V}\hat{T} + \hat{T}\hat{V})(\hat{T}+\hat{V}) +  \frac{1}{4!}t^4(\hat{V}\hat{T} + \hat{T}\hat{V})(\hat{T}+\hat{V})^2 + \scrO(t^5)\\
&= 1 + t(\hat{V}+\hat{T}) + \frac{1}{2}t^2(\hat{V}\hat{T} + \hat{T}\hat{V}) + \frac{1}{3!}t^3(\hat{V}\hat{T}\hat{V} + \hat{T}\hat{V}\hat{T}) +  \frac{1}{4!}t^4(\hat{V}\hat{T}\hat{V} + \hat{T}\hat{V}\hat{T})(\hat{T}+\hat{V}) + \scrO(t^5)\\
&= 1 + t(\hat{V}+\hat{T}) + \frac{1}{2}t^2(\hat{V}\hat{T} + \hat{T}\hat{V}) + \frac{1}{3!}t^3(\hat{V}\hat{T}\hat{V} + \hat{T}\hat{V}\hat{T}) +  \frac{1}{4!}t^4(\hat{V}\hat{T}\hat{V}\hat{T} + \hat{T}\hat{V}\hat{T}\hat{V}) + \scrO(t^5)\\
&= 1 + t\hat{V} + \frac{1}{2}t^2\hat{T}\hat{V} + \frac{1}{3!}t^3\hat{V}\hat{T}\hat{V} +  \frac{1}{4!}t^4\hat{T}\hat{V}\hat{T}\hat{V} + \scrO(t^5)
\end{align*}
Note we neglected terms with $\hat{T}$ on the right as those terms go to zero for the $q$ 
transformation.
Now we can expand our new transformation again neglecting the higher power terms to get
\begin{align*}
\mathcal{T}\left(\frac{t}{2-s}\right) & = 1 + \frac{t}{2-s}(\hat{T}+\hat{V}) + \frac{t^2}{2(2-s)^2}(\hat{V}\hat{T}+\hat{T}\hat{V}) + \frac{t^3}{4(2-s)^3}\hat{T}\hat{V}\hat{T}\\
\mathcal{T}\left(\frac{-st}{2-s}\right) & = 1 - \frac{st}{2-s}(\hat{T}+\hat{V}) + \frac{s^2t^2}{2(2-s)^2}(\hat{V}\hat{T}+\hat{T}\hat{V}) - \frac{s^3t^3}{4(2-s)^3}\hat{T}\hat{V}\hat{T}
\end{align*}
So the product of these to fourth order in $t$ yields
\begin{align*}
&\mathcal{T}\left(\frac{-st}{2-s}\right)\mathcal{T}\left(\frac{t}{2-s}\right)\\
&= \left( 1 - \frac{st}{2-s}(\hat{T}+\hat{V}) + \frac{s^2t^2}{2(2-s)^2}(\hat{V}\hat{T}+\hat{T}\hat{V}) - \frac{s^3t^3}{4(2-s)^3}\hat{T}\hat{V}\hat{T}\right)\left(1 + \frac{t}{2-s}\hat{V} + \frac{t^2}{2(2-s)^2}\hat{T}\hat{V}\right)\\
&= 1 + t\left(\frac{1}{(2-s)}\hat{V}-\frac{s}{(2-s)}\hat{V}\right) + t^2\left(\frac{s^2}{2(2-s)^2}\hat{T}\hat{V} + \frac{1}{2(2-s)^2}\hat{T}\hat{V} - \frac{2s}{2(2-s)^2}\hat{T}\hat{V}\right)\\
&\qquad + t^3\left(\frac{s^2}{2(2-s)^3}\hat{V}\hat{T}\hat{V} - \frac{s}{2(2-s)^3}\hat{V}\hat{T}\hat{V}\right)
+ t^4\left(\frac{s^2}{4(2-s)^4}\hat{T}\hat{V}\hat{T}\hat{V} - \frac{s^3}{4(2-s)^4}\hat{T}\hat{V}\hat{T}\hat{V}\right) + \scrO(t^5)\\
&= 1 + \frac{(1-s)t}{2-s}\hat{V} + \frac{(1-s)^2t^2}{(2-s)^2}\hat{T}\hat{V} + \frac{(s^2-s)t^3}{2(2-s)^3}\hat{V}\hat{T}\hat{V} - \frac{(s^3+s^2)t^4}{4(2-s)^4}\hat{T}\hat{V}\hat{T}\hat{V + \scrO(t^5)}
\end{align*}
Again we neglected the terms with $\hat{T}$ on the right.  Next we forward multiply 
$\mathcal{T}(t/(2-s))$ to yield
\begin{align*}
&\mathcal{T}\left(\frac{t}{2-s}\right)\mathcal{T}\left(\frac{-st}{2-s}\right)\mathcal{T}\left(\frac{t}{2-s}\right)\\
&=  \left(1 + \frac{t}{2-s}(\hat{T}+\hat{V}) + \frac{t^2}{2(2-s)^2}(\hat{V}\hat{T}+\hat{T}\hat{V}) + \frac{t^3}{4(2-s)^3}\hat{T}\hat{V}\hat{T}\right)\\
&\qquad\times \left(1 + \frac{t(1-s)}{2-s}\hat{V} + \frac{(1-s)^2t^2}{(2-s)^2}\hat{T}\hat{V}+ \frac{(s^2-s)t^3}{2(2-s)^3}\hat{V}\hat{T}\hat{V} - \frac{(s^3+s^2)t^4}{4(2-s)^4}\hat{T}\hat{V}\hat{T}\hat{V}\right)+\scrO(t^5)\\
&= 1 + t\left(\frac{1-s}{2-s}\hat{V}+\frac{1}{2-s}\hat{V}\right) + t^2\left(\frac{2(1-s)^2}{2(2-s)^2}\hat{T}\hat{V} + \frac{1}{2(2-s)^2}\hat{T}\hat{V} + \frac{2(1-s)}{(2-s)^2}\hat{T}\hat{V}\right)\\
&\qquad + t^3\left(\frac{(1-s)^2}{(2-s)^4}\hat{V}\hat{T}\hat{V} + \frac{1-s}{4(2-s)^4}\hat{V}\hat{T}\hat{V} + \frac{(1-s)}{(2-s)^2}\hat{V}\hat{T}\hat{V}\right) + t^4\left(\frac{(s^2-s)}{2(2-s)^4}\hat{T}\hat{V}\hat{T}\hat{V}-\frac{(s^3+s^2)}{4(2-s)^4}\hat{T}\hat{V}\hat{T}\hat{V}\right)+\scrO(t^5)\\
&= 1 + t\hat{V} + \frac{1}{2}t^2\hat{T}\hat{V} + \frac{1}{3!}t^3\hat{V}\hat{T}\hat{V} +  \frac{1}{4!}t^4\hat{T}\hat{V}\hat{T}\hat{V} + \scrO(t^5) \qquad\textnormal{ \ for \ } s=2^{1/3}
\end{align*}



\end{document}

