\documentclass[11pt]{article}

\usepackage{latexsym}
\usepackage{amssymb}
\usepackage{amsthm}
\usepackage{enumerate}
\usepackage{amsmath}
\usepackage{cancel}
\numberwithin{equation}{section}

\setlength{\evensidemargin}{.25in}
\setlength{\oddsidemargin}{-.25in}
\setlength{\topmargin}{-.75in}
\setlength{\textwidth}{6.5in}
\setlength{\textheight}{9.5in}
\newcommand{\due}{November 11th, 2015}
\newcommand{\HWnum}{9}
\newcommand{\grad}{\bold\nabla}
\newcommand{\vecE}{\vec{E}}
\newcommand{\scrptR}{\vec{\mathfrak{R}}}
\newcommand{\kapa}{\frac{1}{4\pi\epsilon_0}}
\newcommand{\emf}{\mathcal{E}}
\newcommand{\unit}[1]{\ensuremath{\, \mathrm{#1}}}
\newcommand{\real}{\textnormal{Re}}
\newcommand{\Erf}{\textnormal{Erf}}
\newcommand{\sech}{\textnormal{sech}}
\newcommand{\scrO}{\mathcal{O}}
\newcommand{\levi}{\widetilde{\epsilon}}
\newcommand{\partiald}[2]{\ensuremath{\frac{\partial{#1}}{\partial{#2}}}}
\newcommand{\norm}[2]{\langle{#1}|{#2}\rangle}
\newcommand{\inprod}[2]{\langle{#1}|{#2}\rangle}
\newcommand{\average}[1]{\left\langle{#1}\right\rangle}
\newcommand{\ket}[1]{|{#1}\rangle}
\newcommand{\bra}[1]{\langle{#1}|}
\newcommand{\Resid}[2]{\ensuremath{\textnormal{Res}\left[{#1},{#2}\right]}}





\begin{document}
\begin{titlepage}
\setlength{\topmargin}{1.5in}
\begin{center}
\Huge{Physics 3310} \\
\LARGE{Principles of Electricity and Magnetism 1} \\
\Large{Professor Thomas R. Schibli} \\[1cm]

\huge{Homework \#\HWnum}\\[0.5cm]

\large{Joe Becker} \\
\large{SID: 810-07-1484} \\
\large{\due} 

\end{center}

\end{titlepage}



\section{Problem \#1}
For a one-dimensional harmonic oscillator of mass, $m$, and spring constant $k$ we can 
construct a Hamiltonian by finding the Lagrangian as
$$L = \frac{1}{2}m\dot{q}^2 - \frac{1}{2}m\omega^2q^2$$
where $\omega = \sqrt{k/m}$. We find the canonical momentum of the system as
$$p = \partiald{L}{\dot{q}} = m\dot{q}$$
which allows us to construct the Hamiltonian by
\begin{align*}
H = p\dot{q} - L  &= \frac{p^2}{m} - \frac{p^2}{2m} + \frac{1}{2}m\omega^2q^2\\
&= \frac{p^2}{2m} + \frac{1}{2}m\omega^2q^2\\
&= \frac{1}{2}m\left(\left(\frac{p}{m}\right)^2 + (\omega{q})^2\right)
\end{align*}
For the \emph{Canonical Transformation} of a one-dimensional harmonic oscillator given by
$$Q = C(p+im\omega{q}), \qquad P = C(p-im\omega{q})$$
We can find the constant $C$ by using the fact that the determinant of the Jacobian is unity. 
This implies that
\begin{align*}
1 &= \partiald{Q}{q}\partiald{P}{p} - \partiald{Q}{p}\partiald{P}{q}\\
&= (Cim\omega)C - C(-Cim\omega)\\
& 2C^2im\omega\\
&\Downarrow\\
C &= \frac{1}{\sqrt{2im\omega}}
\end{align*}
So our transformations become
$$Q = \frac{1}{\sqrt{2im\omega}}(p+im\omega{q}), \qquad P = \frac{1}{\sqrt{2im\omega}}(p-im\omega{q})$$
Now we can find the generating function $F_2(q,P)$ for this transformation by noting that
$$\partiald{F_2}{q} = p, \qquad \partiald{F_2}{P} = Q$$
Where we can solve the transformations such that
$$p(q,P) = \sqrt{2im\omega}P+im\omega{q}$$
and
$$Q(q,P) = \frac{1}{\sqrt{2im\omega}}(p(q,P)+im\omega{q}) = P +\sqrt{2im\omega}{q}$$
Which allows us to find 
\begin{align*}
\partiald{F_2}{P} = Q  &= P +\sqrt{2im\omega}{q}\\
&\Downarrow\\
F_2(q,P) &= \frac{1}{2}P^2 +\sqrt{2im\omega}{q}P + f(q)
\end{align*}
Now we can find the function of $q$ by
\begin{align*}
\partiald{F_2}{q} = P  &= \sqrt{2im\omega}P + f'(q)\\
&\Downarrow\\
\sqrt{2im\omega}P+im\omega{q} &= \sqrt{2im\omega}P + f'(q)\\
&\Downarrow\\
f(q) = \frac{1}{2}im\omega{q^2}
\end{align*}
So our generator function is
$$F_2(q,P) = \frac{1}{2}P^2 +\sqrt{2im\omega}{q}P + \frac{1}{2}im\omega{q^2}$$
which allows us see that
\begin{align*}
QP &= \frac{1}{2im\omega}(p+im\omega{q})(p-im\omega{q})\\
&= -\frac{i}{2m\omega}(p^2+(m\omega{q})^2)\\
&= -\frac{i}{2}m\left(\frac{1}{\omega}\left(\frac{p}{m}\right)^2+\omega{q}^2\right)
\end{align*}
Which implies that the Hamiltonian is 
$$H = i\omega{QP}$$
so the equations of motion are given by
$$\dot{P} = -\partiald{H}{Q} = -i\omega{P},\qquad \dot{Q} = \partiald{H}{P} = i\omega{Q}$$
Which easily allows us to solve the equations of motion as
$$P(t) = Ae^{-i\omega{t}},\qquad Q(t) = Be^{i\omega{t}}$$

\pagebreak

\section{Problem \#2}
We note that the generator $F(q,Q) = F_1(q,Q)$ follows from the Legendre transform that states
$$\delta\int_{t_1}^{t_2}(p\dot{q}-H)-(P\dot{Q}-K)dt = 0$$
which implies that the difference in the integrand can only differ by a total time derivative
$$(p\dot{q}-H)-(P\dot{Q}-K) = \frac{dF}{dt}$$
which implies that 
$$dF = (K-H)dt + pdq - PdQ$$
We note that $F$ is a function of $q$ and $Q$. We can use this to find $F_3(p,Q)$ by noting
that
$$d(pq) = pdq + qdp$$
which allows us to write
\begin{align*}
dF &= (K-H)dt + pdq - PdQ\\
&\Downarrow\\
dF &= (K-H)dt + d(pq) - qdp - PdQ\\
&\Downarrow\\
d(F-pq) &= (K-H)dt - qdp - PdQ
\end{align*}
So $F_3(p,Q) = F-pq$. Next we repeat the process to find $F_4(p,P)$ by noting that
$$d(PQ) = PdQ + QdP$$
which implies
\begin{align*}
dF &= (K-H)dt + pdq - PdQ\\
&\Downarrow\\
dF &= (K-H)dt + pdq - (d(PQ) - QdP)\\
&\Downarrow\\
d(F+PQ) &= (K-H)dt + pdq + QdP
\end{align*}
which implies that $F_4(p,P) = F+PQ$.

\pagebreak

\section{Problem \#3}
\begin{enumerate}[(a)]
\item
For the infinitesimal transformation with the generator given by
$$F_2(q,P) = qP + \epsilon{H(q,P)}$$
where $H$ is the Hamiltonian given by
$$H(q,p) = \frac{p^2}{2m} + V(q)$$
we can find the canonical transformations associated with $F_2(q,P)$ by 
\begin{align*}
Q = \partiald{F_2}{P} &= q + \epsilon\partiald{}{P}\left(\frac{P^2}{2m} + V(q)\right)\\
&= q + \epsilon\frac{P}{m}
\end{align*}
Then we can find $p$ and invert by
\begin{align*}
p = \partiald{F_2}{q} &= P + \epsilon\frac{dV}{dq}\\
&\Downarrow\\
P &= p - \epsilon\frac{dV}{dq}
\end{align*}
So replacing $P$ in the $Q$ term we get the canonical transformations to first order in $\epsilon$
\begin{align*}
P &= p - \epsilon\frac{dV}{dq}\\
Q &= q + \epsilon\frac{p}{m}
\end{align*}

\item If we take $\epsilon$ to be a small time step we can see that for $q=q(t)$ and $p=p(t)$
we have the transformations
\begin{align*}
q(t+\epsilon) = Q &= q(t) + \epsilon\frac{p(t)}{m}\\
p(t+\epsilon) = P &= p(t) - \epsilon\frac{dV}{dq}
\end{align*}
If we Taylor expand $q(t+\epsilon)$ about $\epsilon$ to first order we see that
\begin{align*}
q(t+\epsilon) &= q(t) + \dot{q}\epsilon\\
p(t+\epsilon) &= p(t) + \dot{p}\epsilon
\end{align*}
We note that the velocity $\dot{q}$ is related to momentum $p$ by $\dot{q}=p/m$ which implies
that The $q$ expansion is the same as the canonical transformation. We also note that 
$\dot{p} = F = -dV/dq$ so the expansion of $p$ also is the same to first order.

\item Using the results from above we can find the transformed Hamiltonian $K(P,Q)$ by noting
\begin{align*}
H(q,p) &= \frac{p^2}{2m} + V(q)\\
&\Downarrow\\
K(Q,P) &= \frac{1}{2m}\left(P+\epsilon\frac{dV}{dq}\right)^2 + V\left(Q-\epsilon\frac{P}{m}\right)\\
&= \frac{1}{2m}\left(P^2+P\epsilon\frac{dV}{dq}+\epsilon^2\left(\frac{dV}{dq}\right)^2\right) + V(Q) - \epsilon\frac{P}{m}\frac{dV}{dQ} + \epsilon^2\frac{p^2}{m^2}\frac{d^2V}{dQ^2}\\
&= \frac{P^2}{2m} + V(Q) + \epsilon\left(P\frac{dV}{dq}-\frac{P}{m}\frac{dV}{dQ}\right) + \epsilon^2\left(\left(\frac{dV}{dq}\right)^2 + \frac{p^2}{m^2}\frac{d^2V}{dQ^2}\right)
\end{align*}
\end{enumerate}

\pagebreak

\section{Problem \#4}
\begin{enumerate}[(a)]
\item For the generator 
$$F_3(p,Q) = -pQ + \epsilon{H(Q,p)}$$
where $H$ is the same Hamiltonian from problem 3 this allows up to find the transformations 
by
\begin{align*}
q = -\partiald{F_3}{p} &= Q - \epsilon\frac{p}{m}
\end{align*}
and 
$$P = -\partiald{F_3}{Q} = p - \epsilon\frac{dV}{dQ}$$
So solving for $Q$ we have the canonical transformations by noting that
$$\partiald{V}{q} = \partiald{V}{Q}\partiald{V}{q}$$
but as we can see $dQ/dq =1$ so
\begin{align*}
Q &= q + \epsilon\frac{p}{m}\\
P &= p - \epsilon\frac{dV}{dq}
\end{align*}
which are the same canonical transformations as problem 3.

\item For $\epsilon$ as a time step we can find the same result that
\begin{align*}
q(t+\epsilon) = Q &= q(t) + \epsilon\frac{p(t)}{m}\\
p(t+\epsilon) = P &= p(t) - \epsilon\frac{dV}{dq}
\end{align*}
This again is the same as the Taylor expansion about $\epsilon$ to first order. Where 
$\dot{p} = -dV/dq$ and $\dot{q} = p/m$.

\item We note that the canonical transformation found in part (a) is the same as the 
transformation from problem 3. Therefore the transformed Hamiltonian to second order is the
same given by
$$K(Q,P) = \frac{P^2}{2m} + V(Q) + \epsilon\left(P\frac{dV}{dq}-\frac{P}{m}\frac{dV}{dQ}\right) + \epsilon^2\left(\left(\frac{dV}{dq}\right)^2 + \frac{p^2}{m^2}\frac{d^2V}{dQ^2}\right)$$

\item
\end{enumerate}

\pagebreak

\section{Problem \#5}
For the Kepler problem with a three dimensional Hamiltonian given by
$$H = \frac{\mathbf{p}^2}{2m} - \frac{k}{|\mathbf{r}|}.$$
Using this we can show that 
$$\mathbf{L} = \mathbf{r}\times\mathbf{p}$$
is a constant of motion by taking the Poisson bracket with $H$ and noting that the $l$ 
component of $\mathbf{L}$ is
$$\left(\frac{}{}\mathbf{r}\times\mathbf{p}\right)_{l} = \epsilon_{lmn}r_mp_n$$
\begin{align*}
{\mathbf{L},H} &= 
\end{align*}


\end{document}

