\documentclass[11pt]{article}

\usepackage{latexsym}
\usepackage{amssymb}
\usepackage{amsthm}
\usepackage{enumerate}
\usepackage{amsmath}
\usepackage{cancel}
\numberwithin{equation}{section}

\setlength{\evensidemargin}{.25in}
\setlength{\oddsidemargin}{-.25in}
\setlength{\topmargin}{-.75in}
\setlength{\textwidth}{6.5in}
\setlength{\textheight}{9.5in}
\newcommand{\due}{September 26th, 2012}
\newcommand{\HWnum}{3}
\newcommand{\grad}{\bold\nabla}
\newcommand{\vecE}{\vec{E}}
\newcommand{\scrptR}{\vec{\mathfrak{R}}}
\newcommand{\kapa}{\frac{1}{4\pi\epsilon_0}}
\newcommand{\emf}{\mathcal{E}}
\newcommand{\unit}[1]{\ensuremath{\, \mathrm{#1}}}
\newcommand{\real}{\textnormal{Re}}
\newcommand{\Erf}{\textnormal{Erf}}
\newcommand{\sech}{\textnormal{sech}}
\newcommand{\scrO}{\mathcal{O}}
\newcommand{\levi}{\widetilde{\epsilon}}
\newcommand{\partiald}[2]{\ensuremath{\frac{\partial{#1}}{\partial{#2}}}}
\newcommand{\norm}[2]{\langle{#1}|{#2}\rangle}
\newcommand{\inprod}[2]{\langle{#1}|{#2}\rangle}
\newcommand{\ket}[1]{|{#1}\rangle}
\newcommand{\bra}[1]{\langle{#1}|}





\begin{document}
\begin{titlepage}
\setlength{\topmargin}{1.5in}
\begin{center}
\Huge{Physics 3320} \\
\LARGE{Principles of Electricity and Magnetism II} \\
\Large{Professor Ana Maria Rey} \\[1cm]

\huge{Homework \#\HWnum}\\[0.5cm]

\large{Joe Becker} \\
\large{SID: 810-07-1484} \\
\large{\due} 

\end{center}

\end{titlepage}



\section{Problem \#1}
For two non-interacting electrons in an infinite square potential with a width $a$ we first note that the steady-state eigenfunctions are given as
$$\psi_n(x) = \sqrt{\frac{2}{a}}\sin(k_n x)$$
with a quantized energy
$$E_n = \frac{\hbar^2k_n^2}{2m_e}$$
where $k_n = n\pi/a$. So for two non-interacting electrons of the same spin we have to construct a wave function that is symmetric in swapping particles and states. This results in the state
$$\psi_{nm}(x_1,x_2) = A[\psi_n(x_1)\psi_m(x_2) - \psi_n(x_2)\psi_m(x_1)]$$
where the subtraction follows from the fact that electrons are fermions and $x_1$ and $x_2$ represent the position of each electron. Note that for $\psi_{nm}(x_1,x_2)\ne0$ the condition $n\ne m$ must be true for the quantum numbers. This implies that the electrons cannot occupy the same state. So for the ground state we have $n=1$ and $m=2$ so the ground state yields
$$\psi_{12}(x_1,x_2) = \frac{\sqrt{2}}{a}\left[\psi_1(x_1)\psi_2(x_2) - \psi_1(x_2)\psi_2(x_1)\right]$$
Note the factor $\sqrt{2}/a$ follows from the normalization constraint.

\section{Problem \#2}
By combining the angular momentums $S = 1/2$ and $L = 3$ we can see that the allowed values for $S$ and $L$ are
\begin{align*}
s &= \frac{1}{2}, -\frac{1}{2}\\
l &= 3,2,1,0
\end{align*}
the total spectroscopic states are
\begin{align*}
^2F_{7/2}, ^0F_{5/2}, ^2D_{5/2}, ^0D_{3/2}, ^2P_{3/2}, ^0P_{1/2}, ^2S_{1/2}, ^0S_{1/2}
\end{align*}


\section{Problem \#3}

\section{Problem \#4}

\end{document}

