\documentclass[11pt]{article}

\usepackage{latexsym}
\usepackage{amssymb}
\usepackage{amsthm}
\usepackage{enumerate}
\usepackage{amsmath}
\usepackage{cancel}
\usepackage{eufrak}
\numberwithin{equation}{section}

\setlength{\evensidemargin}{.25in}
\setlength{\oddsidemargin}{-.25in}
\setlength{\topmargin}{-.75in}
\setlength{\textwidth}{6.5in}
\setlength{\textheight}{9.5in}
\newcommand{\due}{September 12th, 2012}
\newcommand{\HWnum}{1}
\newcommand{\grad}{\bold\nabla}
\newcommand{\vecE}{\vec{E}}
\newcommand{\scrptR}{\vec{\mathfrak{R}}}
\newcommand{\kapa}{\frac{1}{4\pi\epsilon_0}}
\newcommand{\emf}{\mathcal{E}}
\newcommand{\unit}[1]{\ensuremath{\, \mathrm{#1}}}
\newcommand{\real}{\textnormal{Re}}
\newcommand{\Erf}{\textnormal{Erf}}
\newcommand{\sech}{\textnormal{sech}}
\newcommand{\scrO}{\mathcal{O}}
\newcommand{\levi}{\widetilde{\epsilon}}
\newcommand{\partiald}[2]{\ensuremath{\frac{\partial{#1}}{\partial{#2}}}}
\newcommand{\norm}[2]{\langle{#1}|{#2}\rangle}
\newcommand{\inprod}[2]{\langle{#1}|{#2}\rangle}
\newcommand{\ket}[1]{|{#1}\rangle}
\newcommand{\bra}[1]{\langle{#1}|}





\begin{document}
\begin{titlepage}
\setlength{\topmargin}{1.5in}
\begin{center}
\Huge{Physics 3320} \\
\LARGE{Principles of Electricity and Magnetism II} \\
\Large{Professor Ana Maria Rey} \\[1cm]

\huge{Homework \#\HWnum}\\[0.5cm]

\large{Joe Becker} \\
\large{SID: 810-07-1484} \\
\large{\due} 

\end{center}

\end{titlepage}



\section{Problem \#1}
\begin{enumerate}[(a)]
\item
For a three level system represented by the Hamilitonian
$$\mathbf{H} = \left(\begin{array}{ccc}
               a    &0    &b\\
               0    &c    &0\\
               b    &0    &a
\end{array}\right)$$
where $a$, $b$, and $c$ are real numbers with the assumption $a-c = \pm b$. We can find the time evolving state $\ket{\mathcal{S}(t)}$ given that the system starts out in the state
$$\ket{\mathcal{S}(0)} = \left(\begin{array}{c}
               0    \\
               1    \\
               0    
\end{array}\right).$$
First we solve the \emph{time-independent Schr\"{o}dinger equation}
\begin{equation}
\mathbf{H}\ket{\mathcal{S}} = E\ket{\mathcal{S}}
\end{equation}
by finding the eigenvectors and eigenvalues of $\mathbf{H}$. Note that we can find the eigenvalues, $E$, by finding the determinate of the matrix
\begin{align*}
\left(\begin{array}{ccc}
               a-E    &0    &b\\
               0    &c-E    &0\\
               b    &0    &a-E
\end{array}\right)
\end{align*}
and setting it to zero. So
\begin{align*}
\det\left(\begin{array}{ccc}
               a-E    &0    &b\\
               0    &c-E    &0\\
               b    &0    &a-E
\end{array}\right) &= 0\\
&\Downarrow\\
(a-E)\left[(c-E)(a-E)\right] - b\left[(c-E)b\right] &= 0 \\
\left[(a-E)^2-b^2\right](c-E) &= 0
\end{align*}
We see that the equation 
$$\left[(a-E)^2-b^2\right](c-E) = 0$$
has three solutions for $E$. The easiest to see is that $E = c$ for the other two we first must solve
\begin{align*}
\left[(a-E)^2-b^2\right] &= 0\\
&\Downarrow\\
(a-E)^2 &= b^2\\
a-E &= \mp b\\
&\Downarrow\\
E &= a\pm b
\end{align*}
So we have found the allowed energies as $E = a\pm b, c$. Note that we assumed that $a\pm b \ne c$ so each allowed value for $E$ is unique. Now once we have found the eigenvectors for $E$ we will have a complete basis to represent the state $\ket{\mathcal{S}(0)}$. So we calculate for the eigenvalue $a+b$ that
\begin{align*}
\left(\begin{array}{ccc}
               a    &0    &b\\
               0    &c    &0\\
               b    &0    &a
\end{array}\right)\left(\begin{array}{c}\alpha\\ \beta\\ \gamma\end{array}\right) &= a+b\left(\begin{array}{c}\alpha\\ \beta\\ \gamma\end{array}\right)\\
&\Downarrow\\
\left(\begin{array}{c}a\alpha+b\gamma\\ 
                      c\beta\\ 
                      b\alpha+a\gamma\end{array}\right) 
&= \left(\begin{array}{c}(a+b)\alpha\\ 
                         (a+b)\beta\\ 
                         (a+b)\gamma\end{array}\right)\\
\left(\begin{array}{c}a\alpha+b\gamma\\ 
                      c\beta\\ 
                      b\alpha+a\gamma\end{array}\right) 
- \left(\begin{array}{c}(a+b)\alpha\\ 
                         (a+b)\beta\\ 
                         (a+b)\gamma\end{array}\right) &= 0\\
\left(\begin{array}{c}a\alpha+b\gamma-(a+b)\alpha\\ 
                      c\beta-(a+b)\beta\\ 
                      b\alpha+a\gamma-(a+b)\gamma\end{array}\right) &=0\\
&\Downarrow\\ 
\left(\begin{array}{c}b(\gamma-\alpha)=0\\ 
                      (c-a-b)\beta = 0\\ 
                      b(\alpha-\gamma)=0\end{array}\right)&
\end{align*}
Note that by assumption $c-a-b\ne 0$ and $b\ne 0$ this implies that for this eigenvalue the eigenvector must have $\beta=0$ and $\gamma = \alpha$. This gives us the normalized eigenvector 
$$\ket{\mathcal{S}_{(a+b)}} = \frac{1}{\sqrt{2}}\left(\begin{array}{c}
                                     1\\ 0\\ 1
                              \end{array}\right)$$
Note for the eigenvalue $a-b$ we get the same result except that $\gamma = -\alpha$ so we have
$$\ket{\mathcal{S}_{(a-b)}} = \frac{1}{\sqrt{2}}\left(\begin{array}{c}
                                     1\\ 0\\ -1
                              \end{array}\right)$$
For the final eigenvalue $E=c$ we find the eigenvector as
\begin{align*}
\left(\begin{array}{c}a\alpha+b\gamma-c\alpha\\ 
                      c\beta-c\beta\\ 
                      b\alpha+a\gamma-c\gamma\end{array}\right) &=0\\
&\Downarrow\\
\left(\begin{array}{c}(a-c)\alpha+b\gamma=0\\ 
                      0 \\ 
                      b\alpha+(a-c)\gamma=0\end{array}\right)&
\end{align*}
solving the system of equations 
\begin{align*}
(a-c)\alpha+b\gamma &= 0\\ 
 b\alpha+(a-c)\gamma &= 0
\end{align*}
we find that
$$\left((a-c)^2-b^2\right)\gamma = 0$$
but by the assumption $a-c\ne\pm b$ we see that $\gamma = 0$ which implies that $\alpha = 0$ as well. So we have the eigenvector
$$\ket{\mathcal{S}_{c}} = \left(\begin{array}{c}
                                     0\\ 1\\ 0
                              \end{array}\right)$$
Note that $\beta$ was a free parameter an we picked it to be one. So now we can write the initial state $\ket{\mathcal{S}(0)}$ in this basis and find that
$$\ket{\mathcal{S}(0)} = \ket{\mathcal{S}_c}$$
Now if we add the time dependence exponent we get the time evolving state
$$\ket{\mathcal{S}(t)} = e^{-ict/\hbar}\ket{\mathcal{S}_c}$$

\item
Now for the system that starts in that state
$$\ket{\mathcal{S}(0)} = \left(\begin{array}{c}
               0    \\
               1    \\
               0    
\end{array}\right)$$
we can write this in the basis found in part (a) to yield
$$\ket{\mathcal{S}(0)} = \frac{1}{\sqrt{2}}\left[\ket{\mathcal{S}_{(a+b)}} - \ket{\mathcal{S}_{(a-b)}}\frac{}{}\right]$$
and when we add the time-dependence we get
$$\ket{\mathcal{S}(t)} = \frac{1}{\sqrt{2}}\left[e^{-i(a+b)t/\hbar}\ket{\mathcal{S}_{(a+b)}} - e^{-i(a-b)t/\hbar}\ket{\mathcal{S}_{(a-b)}}\frac{}{}\right]$$

\end{enumerate}

\section{Problem \#2}
\begin{enumerate}[(a)]
\item
Using the \emph{canonical commutation relations}
\begin{align}
[r_i,p_j] &= -[p_i,r_j] = i\hbar\delta_{ij}\\
[r_i,r_j] &= [p_i,p_j] = 0
\end{align}
and the commutator identities
\begin{align}
[\hat{A}+\hat{B},\hat{C}] &= [\hat{A},\hat{C}]+[\hat{B},\hat{C}]\\
[\hat{A}\hat{B},\hat{C}] &= [\hat{A},\hat{C}]\hat{B}+\hat{A}[\hat{B},\hat{C}]
\end{align}
we can work out the angular momentum with position commutators using the fact that
$$L_z = xp_y-yp_x$$
\begin{align*}
[L_z,x] = [xp_y-yp_x, x] &= [xp_y,x] - [yp_x,x]\\
&= \cancelto{0}{[x,x]p_y} + \cancelto{0}{x[p_y,x]} - \cancelto{0}{[y,x]p_x} - y[p_x,x]\\
[L_z,x] &= i\hbar y
\end{align*}
and
\begin{align*}
[L_z,y] = [xp_y-yp_x, y] &= [xp_y,y] - [yp_x,y]\\
&= \cancelto{0}{[x,y]p_y} + x[p_y,y] - \cancelto{0}{[y,y]p_x} - \cancelto{0}{y[p_x,y]}\\
[L_z,y] &= -i\hbar x
\end{align*}
and
\begin{align*}
[L_z,z] = [xp_y-yp_x, z] &= [xp_y,z] - [yp_x,z]\\
&= \cancelto{0}{[x,z]p_y} + \cancelto{0}{x[p_y,z]} - \cancelto{0}{[y,z]p_x} - \cancelto{0}{y[p_x,z]}\\
[L_z,y] &= 0
\end{align*}
We can also find the angular momentum with momentum commutators by
\begin{align*}
[L_z,p_x] = [xp_y-yp_x, p_x] &= [xp_y,p_x] - [yp_x,p_x]\\
&= [x,p_x]p_y + \cancelto{0}{x[p_y,p_x]} - \cancelto{0}{[y,p_x]p_x} - \cancelto{0}{y[p_x,p_x]}\\
&= i\hbar p_y
\end{align*}
and
\begin{align*}
[L_z,p_y] = [xp_y-yp_x, p_y] &= [xp_y,p_y] - [yp_x,p_y]\\
&= \cancelto{0}{[x,p_y]p_y} + \cancelto{0}{x[p_y,p_y]} - [y,p_y]p_x - \cancelto{0}{y[p_x,p_y]}\\
&= -i\hbar p_x
\end{align*}
and
\begin{align*}
[L_z,p_z] = [xp_y-yp_x, p_z] &= [xp_y,p_z] - [yp_x,p_z]\\
&= \cancelto{0}{[x,p_z]p_y} + \cancelto{0}{x[p_y,p_z]} - \cancelto{0}{[y,p_z]p_x} - \cancelto{0}{y[p_x,p_z]}\\
&= 0
\end{align*}

\item
Using the results from part (a) and the equation
$$L_x = yp_z - zp_y$$
we can find the commutation $[L_z,L_x]$ by
\begin{align*}
[L_z,L_x] = [L_z,yp_z - zp_y] &= [L_z,yp_z] - [L_z,zp_y]\\
&= [L_z,y]p_z + \cancelto{0}{y[L_z,p_z]} - \cancelto{0}{[L_z,z]p_y} - z[L_z,p_y]\\
&=-i\hbar xp_z + i\hbar zp_x \\
&= i\hbar(zp_x-xp_z) \\
[L_z,L_x] &= i\hbar L_y
\end{align*}

\item
We can evaluate the commutator $[L_z,r^2]$ where $r^2 = x^2+y^2+z^2$ by the same method as parts (a) and (b) which yields
\begin{align*}
[L_z,r^2] = [L_z, x^2+y^2+z^2] &= [L_z,x^2]+[L_z,y^2]+[L_z,z^2]\\
&= [L_z,x]x + x[L_z,x] +[L_z,y]y + y[L_z,y] + \cancelto{0}{[L_z,z]z} + \cancelto{0}{z[L_z,z]}\\
&= i\hbar yx + i\hbar yx - i\hbar xy -i\hbar xy\\
&= 2i\hbar yx - 2i\hbar yx = 0
\end{align*}
We can do the same for the operator $p^2 = p_x^2 +p_y^2 +p_z^2$ by
\begin{align*}
[L_z,p^2] = [L_z, p_x^2+p_y^2+p_z^2] &= [L_z,p_x^2]+[L_z,p_y^2]+[L_z,p_z^2]\\
&= [L_z,p_x]p_x + p_x[L_z,p_x] +[L_z,p_y]p_y + p_y[L_z,p_y] + \cancelto{0}{[L_z,p_z]p_z} + \cancelto{0}{p_z[L_z,p_z]}\\
&= i\hbar p_yp_x + i\hbar p_yp_x - i\hbar p_xp_y -i\hbar p_xp_y\\
&= 2i\hbar p_yp_x - 2i\hbar p_yp_x = 0
\end{align*}

\item
Using the result from part (c) we can show that the Hamiltonian given by
$$H = \frac{p^2}{2m} + V$$
commutes with all three components of the angular momentum operator $\hat{L}$. Assuming that $V$ is only dependent of position, $r$. So we find that
$$[L_z,H] = [L_z, p^2/2m+V(r)] = [L_z,p^2/2m]+[L_z,V(r)]$$
We see that from part (c) that both commutators are zero. Therefore we know that the Hamiltonian, $H$, commutes with all components of $\hat{L}$.
\end{enumerate}

\section{Problem \#3}
To construct the angular momentum matrices representing the angular momentum operators $L_x$, $L_y$, and $L_z$ in the $z$-basis for $l=1$ we use the fact that in this basis there exists a eigenvector $\ket{lm}$ for $L_z$ such that
\begin{equation}
L_z\ket{lm} = \hbar m\ket{lm}
\label{Lz}
\end{equation}
note that the eigenvalue for $\ket{lm}$ is $\hbar m$ where $m$ is defined by the $l$ quantum number $m = l,...,1,0,-1...,-l$. Also $\ket{lm}$ is an eigenvector of the operator $L^2$ with eigenvalue
\begin{equation}
L^2\ket{lm} = \hbar^2 l(l+1)\ket{lm}
\end{equation}
Now we can construct the angular momentum matrix of $L_z$ by taking an innerproduct equation \ref{Lz} with $\bra{lm'}$. Note that $m'\ne m$ so we have
\begin{align*}
\bra{lm'}L_z\ket{lm} &= \hbar m\inprod{lm'}{lm}\\
&= \hbar m\delta_{mm'}
\end{align*}
where $\delta_{mm'}$ is a \emph{Kronecker Delta} and for $l=1$ the quantum number $m$ can have the values $m = -1,0,1$. This implies that the operator $L_z$ can be represented by a $3\times3$ matrix with the only nonzero components are $m=m'=1,-1$. So
$$L_z = \hbar\left(\begin{array}{ccc}
        1   &0   &0\\
        0   &0   &0\\
        0   &0   &-1\\
        \end{array}\right)$$
Note for a general $l$ the matrix will be $2l+1$ by $2l+1$ and still be diagonal. Now to construct the $L_x$ and $L_y$ matrices we need to use the \emph{Ladder Operator} $L_{\pm}$ defined by
\begin{equation}
L_{\pm}\equiv L_x\pm iL_y
\end{equation}
Using the $L_{\pm}$ allows us to use the fact that it commutes with $L^2$ which implies that $L_{\pm}\ket{lm}$ is an eigenvector of $L_z$ which leads to the relation
\begin{equation}
L_{\pm}\ket{lm} = \hbar\sqrt{(l\mp m)(l\pm m+1)}\ket{lm\pm1}
\label{pm}
\end{equation}
Note the quantum number $m$ is raised or lowered depending on the operator. Using equation \ref{pm} we can find the matrix representation of $L_{\pm}$ using the same method as before
\begin{align*}
\bra{lm'}L_{\pm}\ket{lm} &= \hbar\sqrt{(l\mp m)(l\pm m+1)}\inprod{lm'}{lm\pm1}\\
&= \hbar\sqrt{(1\mp m)(2\pm m)}\delta_{m'm\pm1}
\end{align*}
Note that this matrix is zero for all $m'\ne m\pm1$. So for $L_{+}$ with the restriction on $m$ and $m'$ imposed by $l$ the only non-zero components are when $m'=1,m=0$ with $\hbar\sqrt{2}$ and $m'=0,m=-1$ with $\hbar\sqrt{2}$ this yields
\begin{equation}
L_+ = \hbar\left(\begin{array}{ccc}
                 0    &\sqrt{2}   &0\\
                 0    &0   &\sqrt{2}\\
                 0    &0   &0
           \end{array}\right)
\label{Lp}
\end{equation}
For $L_-$ the only non-zero components are when $m'=-1,m=0$ and $m'=0,m=1$. Note both have values of $\hbar\sqrt{2}$. So
\begin{equation}
L_- = \hbar\left(\begin{array}{ccc}
                 0    &0   &0 \\
                 \sqrt{2}    &0   &0\\
                 0    &\sqrt{2}   &0
           \end{array}\right)
\label{Lp}
\end{equation}
From here we can find $L_x$ and $L_y$ through equation \ref{pm} just in terms of $L_x$ and $L_y$ such that
\begin{align*}
L_x &= \frac{1}{2}(L_++L_-)\\
L_y &= \frac{1}{2i}(L_+-L_-)
\end{align*}
So we calculate
\begin{align*}
L_x &= \frac{\hbar}{2}\left(
   \left(\begin{array}{ccc}
         0    &\sqrt{2}   &0\\
         0    &0   &\sqrt{2}\\
         0    &0   &0
   \end{array}\right)
+  \left(\begin{array}{ccc}
         0    &0   &0 \\
         \sqrt{2}    &0   &0\\
         0    &\sqrt{2}   &0
         \end{array}\right)\right)\\
&= \frac{\hbar\sqrt{2}}{2}
   \left(\begin{array}{ccc}
         0    &1   &0\\
         1    &0   &1\\
         0    &1   &0
   \end{array}\right)
\end{align*}
and
\begin{align*}
L_y &= \frac{\hbar}{2}\left(
   \left(\begin{array}{ccc}
         0    &\sqrt{2}   &0\\
         0    &0   &\sqrt{2}\\
         0    &0   &0
   \end{array}\right)
-  \left(\begin{array}{ccc}
         0    &0   &0 \\
         \sqrt{2}    &0   &0\\
         0    &\sqrt{2}   &0
         \end{array}\right)\right)\\
&= \frac{\hbar\sqrt{2}}{2}
   \left(\begin{array}{ccc}
         0    &1    &0\\
         -1   &0    &1\\
         0    &-1   &0
   \end{array}\right)
\end{align*}

\section{Problem \#4}
To construct the matrix $\mathbf{S}_r$ where the direction $\hat{r}$ defined in spherical coordinates by
$$\hat{r} = \sin(\theta)\cos(\phi)\hat{i} + \sin(\theta)\sin(\phi)\hat{j}+\cos(\theta)\hat{k}$$
we use the spin matrices in rectangular coordinates such that
$$\mathbf{S}_r = \sin(\theta)\cos(\phi)\mathbf{S}_x + \sin(\theta)\sin(\phi)\mathbf{S}_y+\cos(\theta)\mathbf{S}_z$$
which we can calculate to be
\begin{align*}
\mathbf{S}_r &= \sin(\theta)\cos(\phi)\mathbf{S}_x + \sin(\theta)\sin(\phi)\mathbf{S}_y+\cos(\theta)\mathbf{S}_z\\
&=\sin(\theta)\cos(\phi)\frac{\hbar}{2}\left(\begin{array}{cc}
                                              0    &1\\  
                                              1    &0
                                             \end{array}\right) 
+ \sin(\theta)\sin(\phi)\frac{\hbar}{2}\left(\begin{array}{cc} 
                                              0    &-i\\  
                                              i    &0
                                             \end{array}\right)
+           \cos(\theta)\frac{\hbar}{2}\left(\begin{array}{cc} 
                                              1    &0\\  
                                              0    &-1
                                             \end{array}\right)\\
&= \frac{\hbar}{2}\left(\begin{array}{cc}
                         \cos(\theta)    &\sin(\theta)(\cos(\phi)-i\sin(\phi))\\
                         \sin(\theta)(\cos(\phi)+i\sin(\phi)) &-\cos(\theta)
                        \end{array}\right)\\
&= \frac{\hbar}{2}\left(\begin{array}{cc}
                         \cos(\theta)    &e^{-i\phi}\sin(\theta)\\
                         e^{i\phi}\sin(\theta) &-\cos(\theta)
                        \end{array}\right)
\end{align*}
Now we can find the eigenvalues of the problem by finding the determinant of the matrix $\mathbf{S}_r-\mathbf{I}\lambda$ set to zero. So
\begin{align*}
\det\left(\begin{array}{cc}
 \frac{\hbar}{2}\cos(\theta)-\lambda    &\frac{\hbar}{2}e^{-i\phi}\sin(\theta)\\
 \frac{\hbar}{2}e^{i\phi}\sin(\theta) &-\frac{\hbar}{2}\cos(\theta)-\lambda
                         \end{array}\right) &= 0\\
&\Downarrow\\
\frac{\hbar^2}{4}\cancelto{1}{e^{-i\phi}e^{i\phi}}\sin^2(\theta) - \left(\frac{\hbar}{2}\cos(\theta)-\lambda\right)\left(-\frac{\hbar}{2}\cos(\theta)-\lambda\right)&=0\\
\frac{\hbar^2}{4}\sin^2(\theta) + \frac{\hbar^2}{4}\cos^2(\theta)-\lambda^2 &= 0\\
\frac{\hbar^2}{4}\cancelto{1}{(\sin^2(\theta) + \cos^2(\theta))}-\lambda^2 &= 0\\
\lambda^2 &= \frac{\hbar^2}{4}\\
&\Downarrow\\
\lambda &= \pm\frac{\hbar}{2}
\end{align*}
So now that we have determined the eigenvalues we just need to solve the eigenvalue problems
\begin{align}
\mathbf{S}_r\chi^{(r)}_+ &= \frac{\hbar}{2}\chi^{(r)}_+\\
\mathbf{S}_r\chi^{(r)}_- &= -\frac{\hbar}{2}\chi^{(r)}_-
\end{align}
to find the eigenspinors of $\mathbf{S}_r$. Let $\chi^{(r)}_+$ be a general eigenvector
$$\chi^{(r)}_+ = \left(\begin{array}{c}a\\ b\end{array}\right)$$
we can now solve for the components $a$ and $b$ by
\begin{align*}
\frac{\hbar}{2}\left(\begin{array}{cc}
       \cos(\theta)    &e^{-i\phi}\sin(\theta)\\
       e^{i\phi}\sin(\theta) &-\cos(\theta)
      \end{array}\right)\left(\begin{array}{c}a\\ b\end{array}\right) &= \frac{\hbar}{2}\left(\begin{array}{c}a\\ b\end{array}\right)\\
&\Downarrow\\
\left(\begin{array}{c}
       a\cos(\theta) + be^{-i\phi}\sin(\theta)\\
       ae^{i\phi}\sin(\theta) - b\cos(\theta)\end{array}\right)
      &= \left(\begin{array}{c}a\\ b\end{array}\right)
\end{align*}
This yields a system of equations which we can write as
\begin{align}
b &= \frac{a(1-\cos(\theta))}{e^{-i\phi}\sin(\theta)}\label{eq:1}\\
a &= \frac{b(1+\cos(\theta))}{e^{i\phi}\sin(\theta)}\label{eq:2}
\end{align}
Note we can reduce these equations using trig identities to reduce
\begin{align*}
b &= \frac{1-\cos(\theta)}{e^{-i\phi}\sin(\theta)}a\\
&= \frac{2e^{i\phi}\sin^2(\theta/2)}{\sin(\theta)}a\\
&= \frac{2e^{i\phi}\sin^2(\theta/2)}{2\sin(\theta/2)\cos(\theta/2)}a\\
&= \frac{e^{i\phi}\sin(\theta/2)}{\cos(\theta/2)}a
\end{align*}
Now we must apply the normalization condition onto $a$ and $b$ to solve for the value of $a$
\begin{align*}
|a|^2+|b|^2 = 1 &= |a|^2 + \frac{\cancelto{1}{|e^{i\phi}|^2}\sin^2(\theta/2)}{\cos^2(\theta/2)}|a|^2\\
&= |a|^2\left(1 + \frac{\sin^2(\theta/2)}{\cos^2(\theta/2)}\right)\\
&= |a|^2\frac{\cos^2(\theta/2)+\sin^2(\theta/2)}{\cos^2(\theta/2)}\\
&= |a|^2\frac{1}{\cos^2(\theta/2)}\\
&\Downarrow\\
a &= \cos(\theta/2)
\end{align*}
From this we can see that $b = e^{i\phi}\sin(\theta/2)$ or
$$\chi^{(r)}_+ = \left(\begin{array}{c}
                 \cos(\theta/2)\\
                 e^{i\phi}\sin(\theta/2)
\end{array}\right)$$
Now we repeat the process for the next eigenvalue to get
\begin{align*}
\frac{\hbar}{2}\left(\begin{array}{cc}
       \cos(\theta)    &e^{-i\phi}\sin(\theta)\\
       e^{i\phi}\sin(\theta) &-\cos(\theta)
      \end{array}\right)\left(\begin{array}{c}a\\ b\end{array}\right) &= -\frac{\hbar}{2}\left(\begin{array}{c}a\\ b\end{array}\right)\\
&\Downarrow\\
\left(\begin{array}{c}
       a\cos(\theta) + be^{-i\phi}\sin(\theta)\\
       ae^{i\phi}\sin(\theta) - b\cos(\theta)\end{array}\right)
      &= \left(\begin{array}{c}-a\\ -b\end{array}\right)
\end{align*}
This again yields a system of equations
\begin{align}
b &= \frac{-a(1+\cos(\theta))}{e^{-i\phi}\sin(\theta)}\label{eq:3}\\
a &= \frac{b(-1+\cos(\theta))}{e^{i\phi}\sin(\theta)}\label{eq:4}
\end{align}
Again we reduce
\begin{align*}
b &= -\frac{1+\cos(\theta)}{e^{-i\phi}\sin(\theta)}a\\
&= -\frac{2e^{i\phi}\cos^2(\theta/2)}{\sin(\theta)}a\\
&= -\frac{2e^{i\phi}\cos^2(\theta/2)}{2\sin(\theta/2)\cos(\theta/2)}a\\
&= -\frac{e^{i\phi}\cos(\theta/2)}{\sin(\theta/2)}a
\end{align*}
Again we apply the normalization condition
\begin{align*}
|a|^2 + |b|^2 = 1 &= |a|^2 + \frac{\cancelto{1}{|e^{i\phi}|^2}\cos^2(\theta/2)}{\sin^2(\theta/2)}|a|^2\\
&= |a|^2\left(1 + \frac{\cos^2(\theta/2)}{\sin^2(\theta/2)}\right)\\
&= |a|^2\frac{\sin^2(\theta/2)+\cos^2(\theta/2)}{\sin^2(\theta/2)}\\
&= |a|^2\frac{1}{\sin^2(\theta/2)}\\
&\Downarrow\\
a &= \sin(\theta/2)
\end{align*}
Which implies that
$$\chi^{(r)}_- = \left(\begin{array}{c}
                 \sin(\theta/2)\\
                 -e^{i\phi}\cos(\theta/2)
\end{array}\right)$$

\section{Problem \#5}
\begin{enumerate}[(a)]
\item
To find the normalized eigenspinors of 
$$S_y = \frac{\hbar}{2}\left(\begin{array}{cc}
                              0   &-i\\
                              i   &0 \\
                       \end{array}\right)$$
we first need to find the eigenvalues by
\begin{align*}
\det\left(\begin{array}{cc}
          -\lambda   &-i\frac{\hbar}{2}\\
          i\frac{\hbar}{2}   &-\lambda \\
     \end{array}\right) &= 0\\
&\Downarrow\\
\lambda^2 - \frac{\hbar^2}{4} &= 0\\
&\Downarrow\\
\lambda &= \pm\frac{\hbar}{2} 
\end{align*}
Now we can find the eigenspinors, $\chi^{(y)}_{\pm}$ by the same method as problem 4. First we let $\chi^{(y)}_{\pm}$ be a general eigenvector
$$\chi^{(y)}_{\pm} = \left(\begin{array}{c}a\\ b\end{array}\right)$$
and we solve the eigenvalue problem
\begin{align*}
S_y\chi^{(y)}_{\pm} &= \pm\frac{\hbar}{2}\chi^{(y)}_{\pm}\\
&\Downarrow\\
\frac{\hbar}{2}\left(\begin{array}{cc}
                      0   &-i\\  
                      i   &0 \\
               \end{array}\right)\left(\begin{array}{c}a\\ b\end{array}\right)
     &= \pm\frac{\hbar}{2}\left(\begin{array}{c}a\\ b\end{array}\right)\\
\left(\begin{array}{c}-ib\\ ia\end{array}\right)
     &= \left(\begin{array}{c}\pm a\\ \pm b\end{array}\right)\\
\end{align*}
And again we must normalize first for the positive eigenvalue we have
\begin{align*}
|a|^2+|b|^2 = 1 &= |-ib|^2 + |b|^2\\
&=  2|b|^2\\
&\Downarrow\\
|b|^2 &= \frac{1}{2}\\
b &= \frac{1}{\sqrt{2}}
\end{align*}
By the fact that $a = -ib$ we see that the top spinor is
$$\chi^{(y)}_+ = \left(\begin{array}{c}
                   -i/\sqrt{2}\\
                    1/\sqrt{2}
                 \end{array}\right)$$
Now for the negative eigenvalue we still have
\begin{align*}
|a|^2+|b|^2 = 1 &= |ib|^2 + |b|^2\\
&=  2|b|^2\\
&\Downarrow\\
|b|^2 &= \frac{1}{2}\\
b &= \frac{1}{\sqrt{2}}
\end{align*}
Which implies by the equation $a = ib$ that
$$\chi^{(y)}_- = \left(\begin{array}{c}
                    i/\sqrt{2}\\
                    1/\sqrt{2}
                 \end{array}\right)$$

\item
To find the probability of measuring $-\hbar/2$ for the system in the state
$$\chi = \frac{1}{\sqrt{65}}\left(\begin{array}{c}4\\ 7\end{array}\right)$$
We need to construct this state with the eigenspinors for $S_y$ found in part (a)
\begin{align}
\chi^{(y)}_+ &= \left(\begin{array}{c}
                   -i/\sqrt{2}\\
                    1/\sqrt{2}
                 \end{array}\right)\\
\chi^{(y)}_- &= \left(\begin{array}{c}
                    i/\sqrt{2}\\
                    1/\sqrt{2}
                 \end{array}\right)
\end{align}
We can express $\chi$ as a linear combination of $\chi^{(y)}_{\pm}$ but first we need to generalize
\begin{align*}
\chi &= c\chi^{(y)}_{+} + d\chi^{(y)}_{-}\\
\left(\begin{array}{c}
   a\\
   b
 \end{array}\right) &=
\frac{c}{\sqrt{2}}\left(\begin{array}{c}
   -i\\
    1
 \end{array}\right) + 
\frac{d}{\sqrt{2}}\left(\begin{array}{c}
   i\\
    1
 \end{array}\right)\\
&= \frac{1}{\sqrt{2}}\left(\begin{array}{c}
   i(d-c)\\
    c+d
 \end{array}\right)
\end{align*}
This gives us the system of equations
\begin{align}
a &= \frac{i(d-c)}{\sqrt{2}} \label{eq:a}\\
b &= \frac{c+d}{\sqrt{2}} \label{eq:b}
\end{align}
Note here $|d|^2$ is the probability of finding $-\hbar/2$ when measuring $S_y$. So now we just need to get $d$ in terms of $a$ and $b$ which are known. First we solve equation \ref{eq:b} in terms of $c$ to get
$$c = \sqrt{2}b -d$$
then we replace $c$ in equation \ref{eq:a} and solve for $d$
\begin{align*}
a &= \frac{i(d-c)}{\sqrt{2}}\\
&= \frac{i(d - \sqrt{2}b + d)}{\sqrt{2}}\\
&= \frac{i(2d - \sqrt{2}b)}{\sqrt{2}}\\
&\Downarrow\\
2id &= \sqrt{2}a + \sqrt{2}ib\\
d &= \frac{1}{\sqrt{2}}(b-ia)
\end{align*}
Now we can use the fact that $a = 4/\sqrt{65}$ and $b = 7/\sqrt{65}$ to calculate $|d|^2$
\begin{align*}
|d|^2 &= \left|\frac{1}{\sqrt{2}}\left(\frac{7}{\sqrt{65}}-i\frac{4}{\sqrt{65}}\right)\right|^2\\
&= \frac{1}{130}|7-4i|^2\\
&= \frac{1}{130}(7-4i)(7+4i)\\
&= \frac{1}{130}(49+16) = \frac{1}{2}
\end{align*}
\end{enumerate}

\section{Problem \#6}
The time evolution of a spin 1/2 particle in a magnetic field is given by the Hamiltonian 
$$\hat{H} = -\gamma \mathbf{B}\cdot\mathbf{S}$$
where $\gamma$ is the gyromagnetic ratio. For $\mathbf{B} = B\hat{z}$ we have
$$\hat{H} = -\frac{\gamma B_0\hbar}{2}\left(\begin{array}{cc}
                                            1    &0 \\
                                            0    &-1
                                      \end{array}\right)$$
Note $\hat{H}$ has the eigenvectors $\chi^{(z)}_{\pm}$ which are the same eigenvectors for $S_z$ with eigenvalues 
$$E_{\pm} = \mp\frac{\gamma B_0 \hbar}{2}$$
which if we add the time dependent component we get
$$\chi(t) = a\chi^{(z)}_{+}e^{-iE_+t/\hbar} + b\chi^{(z)}_{-}e^{-iE_-t/\hbar}$$ 
where $a$ and $b$ are given by the initial condition 
$$\chi(0) = \frac{1}{\sqrt{2}}\left(\begin{array}{c}
                                    1\\ 1
                               \end{array}\right)$$
this implies that
$$\chi(t) = \frac{1}{\sqrt{2}}\left(\begin{array}{c}
                                    e^{i\gamma B_0t/2}\\ 
                                    e^{-i\gamma B_0t/2}
                               \end{array}\right)$$
Now at time $t=T$ we change the direction of the magnetic field such that $\mathbf{B} = B_0\hat{y}$. This changes the Hamiltonian to
$$\hat{H} = -\frac{\gamma B_0\hbar}{2}\left(\begin{array}{cc}
                                            0    &-i\\
                                            i    &0 
                                      \end{array}\right)$$
again we have the same eigenvalues except this time the eigenvectors are $\chi^{(y)}_{\pm}$ this implies that for $\chi(t>T)$ we have
$$\chi(t>T) = a'\chi^{(y)}_{+}e^{-iE_+t/\hbar} + b'\chi^{(y)}_{-}e^{-iE_-t/\hbar}$$ 
Using the results from problem 5 we can account for the change in basis. This yields
$$\chi(t>T) = \frac{1}{\sqrt{2}}\left(\begin{array}{c}
                       -ia'e^{i\gamma B_0t/2} + ib'e^{-i\gamma B_0t/2}\\
                       a'e^{i\gamma B_0t/2} + b'e^{-i\gamma B_0t/2}\\
                                \end{array}\right)$$
Now if we apply the initial condition found for the initial magnetic field for $t=T$ we find
$$\frac{1}{\sqrt{2}}\left(\begin{array}{c}
                       -ia'e^{i\gamma B_0T/2} + ib'e^{-i\gamma B_0T/2}\\
                       a'e^{i\gamma B_0T/2} + b'e^{-i\gamma B_0T/2}\\
                                \end{array}\right)
= \frac{1}{\sqrt{2}}\left(\begin{array}{c}
                                    e^{i\gamma B_0T/2}\\ 
                                    e^{-i\gamma B_0T/2}
                               \end{array}\right)$$

\end{document}

