\documentclass[11pt]{article}

\usepackage{latexsym}
\usepackage{amssymb}
\usepackage{amsthm}
\usepackage{enumerate}
\usepackage{amsmath}
\usepackage{cancel}
\numberwithin{equation}{section}

\setlength{\evensidemargin}{.25in}
\setlength{\oddsidemargin}{-.25in}
\setlength{\topmargin}{-.75in}
\setlength{\textwidth}{6.5in}
\setlength{\textheight}{9.5in}
\newcommand{\due}{November 7th, 2012}
\newcommand{\HWnum}{8}
\newcommand{\Brf}{B_{\textnormal{rf}}}
\newcommand{\grad}{\bold\nabla}
\newcommand{\vecE}{\vec{E}}
\newcommand{\scrptR}{\vec{\mathfrak{R}}}
\newcommand{\kapa}{\frac{1}{4\pi\epsilon_0}}
\newcommand{\emf}{\mathcal{E}}
\newcommand{\unit}[1]{\ensuremath{\, \mathrm{#1}}}
\newcommand{\real}{\textnormal{Re}}
\newcommand{\Erf}{\textnormal{Erf}}
\newcommand{\sech}{\textnormal{sech}}
\newcommand{\scrO}{\mathcal{O}}
\newcommand{\levi}{\widetilde{\epsilon}}
\newcommand{\partiald}[2]{\ensuremath{\frac{\partial{#1}}{\partial{#2}}}}
\newcommand{\norm}[2]{\langle{#1}|{#2}\rangle}
\newcommand{\inprod}[2]{\langle{#1}|{#2}\rangle}
\newcommand{\ket}[1]{|{#1}\rangle}
\newcommand{\bra}[1]{\langle{#1}|}





\begin{document}
\begin{titlepage}
\setlength{\topmargin}{1.5in}
\begin{center}
\Huge{Physics 3320} \\
\LARGE{Principles of Electricity and Magnetism II} \\
\Large{Professor Ana Maria Rey} \\[1cm]

\huge{Homework \#\HWnum}\\[0.5cm]

\large{Joe Becker} \\
\large{SID: 810-07-1484} \\
\large{\due} 

\end{center}

\end{titlepage}



\section{Problem \#1}
\begin{enumerate}[(a)]
\item
Given a particle in an one-dimensional box of length $a$ the particle can be represented as the wavefunction
$$\ket{n} = \sqrt{\frac{2}{a}}\sin\left(\frac{n\pi}{a}x\right)$$
Now if we add a time-dependent potential 
$$V_1(x,t) = \lambda\left(x-\frac{a}{2}\right)\sin(\omega t)$$
in the region $0<x<a$. We can calculate the probability that this particle will transition from the ground state ($n=1$) to the first excited state ($n=2$) by finding the coefficient
\begin{equation}
c_f(t) = -\frac{i}{\hbar}\int_{0}^{t}\bra{f}V_1(x,t')\ket{i}e^{i/\hbar(E_f-E_i)t'}dt'
\label{Cf}
\end{equation}
Where the probability of transitioning from state $\ket{i}$ to $\ket{f}$ is given by 
$$P_{fi} = |c_f|^2$$
Note we will operate under the assumption that 
$$\omega\approx \frac{E_f-E_i}{\hbar}$$
which lets us calculate equation \ref{Cf} as
\begin{align*}
c_f(t) &= -\frac{i}{\hbar}\int_{0}^{t}\bra{2}V_1(x,t')\ket{1}e^{i/\hbar(E_f-E_i)t'}dt'\\
&= -\frac{i}{\hbar}\frac{2}{a}\int_{0}^{t}\sin\left(\frac{2\pi}{a}x\right)\lambda\left(x-\frac{a}{2}\right)\sin(\omega t)\sin\left(\frac{\pi}{a}x\right)e^{i\omega t'}dt'\\
&= -\frac{i}{\hbar}\frac{2\lambda}{a}\sin\left(\frac{2\pi}{a}x\right)\sin\left(\frac{\pi}{a}x\right)\left(x-\frac{a}{2}\right)\int_{0}^{t}\sin(\omega t)e^{i\omega t'}dt'\\
&= -\frac{i}{\hbar}\frac{2\lambda}{a}\sin\left(\frac{2\pi}{a}x\right)\sin\left(\frac{\pi}{a}x\right)\left(x-\frac{a}{2}\right)\left(\frac{2i\omega t - e^{2i\omega t}+1}{4\omega}\right)
\end{align*}
So now we can calculate the $P_{fi}$ by
\begin{align*}
P_{fi} = |c_f|^2 &= c_f^*c_f\\
&= \frac{4\lambda^2}{\hbar^2a^2}\sin^2\left(\frac{2\pi}{a}x\right)\sin^2\left(\frac{\pi}{a}x\right)\left(x-\frac{a}{2}\right)^2\left(\frac{2i\omega t - e^{2i\omega t}+1}{4\omega}\right)^*\left(\frac{2i\omega t - e^{2i\omega t}+1}{4\omega}\right)\\
&= \frac{4\lambda^2}{16\hbar^2a^2\omega^2}\sin^2\left(\frac{2\pi}{a}x\right)\sin^2\left(\frac{\pi}{a}x\right)\left(x-\frac{a}{2}\right)^2(-2i\omega t - e^{-2i\omega t}+1)(2i\omega t - e^{2i\omega t}+1)\\
&= \frac{\lambda^2}{4\hbar^2a^2\omega^2}\sin^2\left(\frac{2\pi}{a}x\right)\sin^2\left(\frac{\pi}{a}x\right)\left(x-\frac{a}{2}\right)^2(-2i\omega t - e^{-2i\omega t}+1)(2i\omega t - e^{2i\omega t}+1)
\end{align*}
Where
\begin{align*}
(-2i\omega t - e^{-2i\omega t}+1)(2i\omega t - e^{2i\omega t}+1) &= 4\omega^2t^2+ 2i\omega te^{2i\omega t} - \cancel{2i\omega t} - 2i\omega te^{-2i\omega t} + \cancelto{1}{e^{2i\omega t}e^{-2i\omega t}}\\
&\ \ \ \ \ \ -e^{-2i\omega t} + \cancel{2i\omega t} - e^{2i\omega t} + 1\\
&= 4\omega^2t^2+ 2i\omega t\left(e^{2i\omega t} - e^{-2i\omega t}\right) - 2\left(\frac{e^{2i\omega t} + e^{-2i\omega t}}{2} + 1\right)\\
&= 4\omega^2t^2+ 2i\omega t(2i\sin(2\omega t)) - 2\left(\cos(2\omega t) - 1\right)\\
&= 4\omega^2t^2 - 4\omega t\sin(2\omega t) - 2\left(\cos(2\omega t) - 1\right)
\end{align*}
Which yields the result
$$P_{21} = \frac{\lambda^2}{2\hbar^2a^2\omega^2}\sin^2\left(\frac{2\pi}{a}x\right)\sin^2\left(\frac{\pi}{a}x\right)\left(x-\frac{a}{2}\right)^2 (2\omega^2t^2 - 2\omega t\sin(2\omega t) - \cos(2\omega t) + 1)$$
note that this is a probability distribution function so we integrate over $x$ from $0$ to $a$ to get
\begin{align*}
P(t) &= \frac{\lambda^2}{2\hbar^2a^2\omega^2}(2\omega^2t^2 - 2\omega t\sin(2\omega t) - \cos(2\omega t) + 1)\int_0^a\sin^2\left(\frac{2\pi}{a}x\right)\sin^2\left(\frac{\pi}{a}x\right)\left(x-\frac{a}{2}\right)^2dx \\
&= \frac{\lambda^2}{2\hbar^2a^2\omega^2}(2\omega^2t^2 - 2\omega t\sin(2\omega t) - \cos(2\omega t) + 1)\left(\frac{a^3(6\pi^2-25)}{288\pi^2}\right)\\
&= \frac{\lambda^2a}{2\hbar^2\omega^2}\left(\frac{6\pi^2-25}{288\pi^2}\right)(2\omega^2t^2 - 2\omega t\sin(2\omega t) - \cos(2\omega t) + 1)
\end{align*}

\item
For the transition where we go from $\ket{i} = \ket{1}$ to $\ket{f} = \ket{3}$ we have the same result from part (a) except for a factor of $3$ in the sine function
$$P_{31} = \frac{\lambda^2}{2\hbar^2a^2\omega^2}\sin^2\left(\frac{3\pi}{a}x\right)\sin^2\left(\frac{\pi}{a}x\right)\left(x-\frac{a}{2}\right)^2 (2\omega^2t^2 - 2\omega t\sin(2\omega t) - \cos(2\omega t) + 1)$$
where we calculate
\begin{align*}
P(t) &= \frac{\lambda^2}{2\hbar^2a^2\omega^2}(2\omega^2t^2 - 2\omega t\sin(2\omega t) - \cos(2\omega t) + 1)\int_0^a\sin^2\left(\frac{3\pi}{a}x\right)\sin^2\left(\frac{\pi}{a}x\right)\left(x-\frac{a}{2}\right)^2dx \\
&= \frac{\lambda^2}{2\hbar^2a^2\omega^2}(2\omega^2t^2 - 2\omega t\sin(2\omega t) - \cos(2\omega t) + 1)\left(\frac{a^3(48\pi^2-275)}{2304\pi^2}\right)\\
&= \frac{\lambda^2a}{2\hbar^2\omega^2}\left(\frac{48\pi^2-275}{2304\pi^2}\right)(2\omega^2t^2 - 2\omega t\sin(2\omega t) - \cos(2\omega t) + 1)
\end{align*}
\end{enumerate}

\section{Problem \#2}
\begin{enumerate}[(a)]
\item
Given a spin-1/2 particle with gyromagnetic ration $\gamma$, at rest in a static magnetic field $B_0\hat{z}$ we know it precesses at the \emph{Larmor frequency} $\omega_0 = \gamma B_0$. Now if we add a transverse radio frequency field given by
$$\mathbf{B}_{\textnormal{rf}} = B_{\textnormal{rf}}(\cos(\omega t)\hat{x} - \sin(\omega t)\hat{y})$$
which results in a total magnetic field
$$\mathbf{B} = B_{\textnormal{rf}}\cos(\omega t)\hat{x} - B_{\textnormal{rf}}\sin(\omega t)\hat{y} +B_0\hat{z}$$
From this can construct the $2\times 2$ Hamiltonian matrix by
$$H = -\gamma\mathbf{B}\cdot\mathbf{S}$$
which results in
$$H = -\gamma(B_xS_x + B_yS_y + B_zS_z)$$
where we use the \emph{Pauli Spin Matrices}, $\sigma_i$, such that
$$S_i = \frac{\hbar}{2}\sigma_i$$
So we have 
\begin{align*}
H &= -\gamma\frac{\hbar}{2}(B_x\sigma_x + B_y\sigma_y + B_z\sigma_z)\\
&= -\gamma\frac{\hbar}{2}\left[
B_x\left(\begin{array}{cc}
0  &1\\
1  &0  \end{array}\right)
+ B_y\left(\begin{array}{cc}
0  &-i\\
i  &0  \end{array}\right)
+ B_z\left(\begin{array}{cc}
1  &0\\
0  &-1  \end{array}\right)
\right]\\
&= -\gamma\frac{\hbar}{2}\left(\begin{array}{cc}
                         B_z           &B_x - iB_y\\
                         B_x + iB_y    &-B_z      \\
                         \end{array}\right)\\
&= -\gamma\frac{\hbar}{2}\left(\begin{array}{cc}
                         B_0      &B_{\textnormal{rf}}[\cos(\omega t) - i\sin(\omega t)]\\
                         B_{\textnormal{rf}}[\cos(\omega t) + i\sin(\omega t)]    &-B_0 \\
                         \end{array}\right)\\
&= -\gamma\frac{\hbar}{2}\left(\begin{array}{cc}
                         B_0      &B_{\textnormal{rf}}e^{-i\omega t}\\
                         B_{\textnormal{rf}}e^{i\omega t}    &-B_0 \\
                         \end{array}\right)
\end{align*}

\item
So for a state
$$\chi(t) = \left(\begin{array}{c}a(t)\\b(t)\end{array}\right)$$
we can find the first time derivative by applying \emph{Schr\"{o}dinger's Equation}
$$i\hbar\partiald{}{t}\chi(t) = H\chi(t)$$
which yields
\begin{align*}
i\hbar\left(\begin{array}{c}\dot{a}(t)\\\dot{b}(t)\end{array}\right)  &= -\gamma\frac{\hbar}{2}\left(\begin{array}{cc}
               B_z      &B_{\textnormal{rf}}e^{-i\omega t}\\
               B_{\textnormal{rf}}e^{i\omega t}    &-B_z \\
               \end{array}\right)\left(\begin{array}{c}a(t)\\b(t)\end{array}\right)\\
&\Downarrow\\
\left(\begin{array}{c}\dot{a}(t)\\\dot{b}(t)\end{array}\right)  &= \gamma\frac{i}{2}\left(\begin{array}{c}
               B_0a(t) + \Brf e^{i\omega t}b(t)\\
               \Brf e^{-i\omega t}b(t) - B_0b(t)\\
               \end{array}\right)\\
&= \frac{i}{2}\left(\begin{array}{c}
               \omega_0a(t) + \gamma\Brf e^{i\omega t}b(t)\\
               \gamma\Brf e^{-i\omega t}a(t) - \omega_0b(t)\\
               \end{array}\right)
\end{align*}
Which yields the system of equations
\begin{align*}
\dot{a} &= \frac{i}{2}\left(\Omega e^{i\omega t}b + \omega_0a\right)\\
\dot{b} &= \frac{i}{2}\left(\Omega e^{-i\omega t}a - \omega_0b\right)
\end{align*}
where we define
$$\Omega\equiv\gamma\Brf$$

\item (Skipped)

\item 
Given the general form for $a(t)$ and $b(t)$
\begin{align*}
a(t) &= \left[a_0\cos(\omega' t/2)+\frac{i}{\omega '}(a_0(\omega_0-\omega)+b_0\Omega)\sin(\omega 't/2)\right]e^{i\omega t/2}\\
b(t) &= \left[b_0\cos(\omega' t/2)+\frac{i}{\omega '}(b_0(\omega-\omega_0)+a_0\Omega)\sin(\omega 't/2)\right]e^{-i\omega t/2}
\end{align*}
where 
$$\omega'\equiv\sqrt{(\omega-\omega_0)^2+\Omega^2}$$
we can calculate the probability of a transition from spin up to spin down as a function of time using the fact that
$$P(t) = |b(t)|^2$$
which states that the magnitude of $b(t)$ squared is the probability of finding the particle in that state. Where we apply the initial conditions that $a_0 = 1$ and $b_0 = 0$ to get
\begin{align*}
b(t) &= \left[\cancelto{0}{b_0\cos(\omega' t/2)}+\frac{i}{\omega '}(\cancelto{0}{b_0(\omega-\omega_0)}+a_0\Omega)\sin(\omega 't/2)\right]e^{-i\omega t/2}\\
&= \frac{i}{\omega '}\Omega\sin(\omega 't/2)e^{-i\omega t/2}
\end{align*}
So $P(t)$ is
\begin{align*}
P(t) = |b(t)|^2 &= b^*(t)b(t)\\
&= \left(\frac{i}{\omega '}\Omega\sin(\omega 't/2)e^{-i\omega t/2}\right)^*\frac{i}{\omega '}\Omega\sin(\omega 't/2)e^{-i\omega t/2}\\
&= -\frac{i}{\omega '}\Omega\sin(\omega 't/2)\cancel{e^{i\omega t/2}}\frac{i}{\omega '}\Omega\sin(\omega 't/2)\cancel{e^{-i\omega t/2}}\\
&= \frac{\Omega^2}{(\omega ')^2}\sin^2(\omega 't/2)
\end{align*}
Which if we write out $\omega'$ we get
$$P(t) = \frac{\Omega^2}{(\omega-\omega_0)^2+\Omega^2}\sin^2(\omega 't/2)$$



\item (Skipped)

\item
In a \emph{nuclear magnetic resonance} experiment the we can measure the $g$-factor of a proton by applying a static field $B_0=1\unit{T}$ with an rf field of $\Brf=1\times10^{-6}\unit{T}$. Using these values we can calculate the resonant frequency $\omega_0$ where the gyromagnetic ratio is given by
$$\gamma = \frac{g_pe}{2m_p}$$
where $g_p$ is given as $g_p = 5.59$ so we can calculate $\omega_0$ by
\begin{align*}
\nu_0 = \frac{\omega_0}{2\pi} = \frac{1}{2\pi}\gamma B_0 = \frac{g_pe}{4\pi m_p}B_0 = \frac{(5.59)(1.60\times10^{-19}\unit{C})}{4\pi(1.67\times10^{-27}\unit{kg})}1\unit{T} = 4.26\times10^{7}\unit{Hz}
\end{align*}
Note we can check the units
$$\left\langle\unit{C\ kg^{-1}\ T}\right\rangle = \left\langle\unit{C\ kg^{-1}\ kg\ C^{-1}\ s^{-1}}\right\rangle = \left\langle\unit{s^{-1}}\right\rangle = \left\langle\unit{Hz}\right\rangle$$
Now from part (d) we see the width of the Lorentzian, $\Delta\omega$ is given by $2\Omega$ which we can calculate as
$$\Delta\nu = \frac{2\Omega}{2\pi} = \frac{\gamma\Brf}{\pi} = \frac{g_pe}{2\pi m_p}\Brf = \frac{(5.59)(1.60\times10^{-19}\unit{C})}{2\pi(1.67\times10^{-27}\unit{kg})}1\times10^{-6}\unit{T} = 85.2\unit{Hz}$$ 
\end{enumerate}
\end{document}

