\documentclass[11pt]{article}
\usepackage{latexsym}
\usepackage{amssymb}
\usepackage{amsthm}
\usepackage{graphicx}
\usepackage{enumerate}
\usepackage{amsmath}
\usepackage{cancel}
\numberwithin{equation}{section}

\setlength{\evensidemargin}{.25in}
\setlength{\oddsidemargin}{-.25in}
\setlength{\topmargin}{-.75in}
\setlength{\textwidth}{6.5in}
\setlength{\textheight}{9.5in}
\newcommand{\due}{September 19th, 2012}
\newcommand{\HWnum}{2}
\newcommand{\grad}{\bold\nabla}
\newcommand{\vecE}{\vec{E}}
\newcommand{\scrptR}{\vec{\mathfrak{R}}}
\newcommand{\kapa}{\frac{1}{4\pi\epsilon_0}}
\newcommand{\emf}{\mathcal{E}}
\newcommand{\unit}[1]{\ensuremath{\, \mathrm{#1}}}
\newcommand{\real}{\textnormal{Re}}
\newcommand{\Erf}{\textnormal{Erf}}
\newcommand{\sech}{\textnormal{sech}}
\newcommand{\scrO}{\mathcal{O}}
\newcommand{\levi}{\widetilde{\epsilon}}
\newcommand{\partiald}[2]{\ensuremath{\frac{\partial{#1}}{\partial{#2}}}}
\newcommand{\norm}[2]{\langle{#1}|{#2}\rangle}
\newcommand{\inprod}[2]{\langle{#1}|{#2}\rangle}
\newcommand{\ket}[1]{|{#1}\rangle}
\newcommand{\bra}[1]{\langle{#1}|}





\begin{document}
\begin{titlepage}
\setlength{\topmargin}{1.5in}
\begin{center}
\Huge{Physics 3320} \\
\LARGE{Principles of Electricity and Magnetism II} \\
\Large{Professor Ana Maria Rey} \\[1cm]

\huge{Homework \#\HWnum}\\[0.5cm]

\large{Joe Becker} \\
\large{SID: 810-07-1484} \\
\large{\due} 

\end{center}

\end{titlepage}



\section{Problem \#1}
\begin{enumerate}[(a)]
\item
A system of two electrons in a spin singlet is represented by
$$\psi_{00} = \frac{1}{\sqrt{2}}\left(\chi_{+}^{1}\chi_{-}^{2} - \chi_{-}^{1}\chi_{+}^{2}\right)$$
Note that in general to find the probability of finding a particle in state $a$ we have the operator that acts as the projection
$$\hat{P}_a = \ket{a}\bra{a}$$
for measuring the particle in a spin up state the projection operator becomes
$$\hat{P}_{+}^1 = \chi_+^1(\chi_+^1)^{\dagger}$$
So we act $\hat{P}_+$ on $\psi_{00}$ to yield
\begin{align*}
\hat{P}_+^2\psi_{00} &= \chi_+^1(\chi_+^1)^{\dagger}\frac{1}{\sqrt{2}}\left(\chi_{+}^{1}\chi_{-}^{2} - \chi_{-}^{1}\chi_{+}^{2}\right)\\
&= \chi_+^1\frac{1}{\sqrt{2}}\left(\cancelto{1}{(\chi_+^1)^{\dagger}\chi_{+}^{1}}\chi_{-}^{2} - \cancelto{0}{(\chi_+^1)^{\dagger}\chi_{-}^{1}}\chi_{+}^{2}\right)\\
&= \frac{1}{\sqrt{2}}\chi_+^1\chi_{-}^{2}
\end{align*}
Now that we have the projected state we can find the probability that the second electron is spin up by finding
\begin{align*}
P_+^2 &= |(\chi_{+}^2)^{\dagger}\hat{P}_+^2\psi_{00}|^2\\
&= |(\chi_{+}^2)^{\dagger}\frac{1}{\sqrt{2}}\chi_+^1\chi_{-}^{2}|^2\\
&= |\frac{1}{\sqrt{2}}\chi_+^1(\chi_{+}^2)^{\dagger}\chi_{-}^{2}|^2\\
&= 0
\end{align*}

\item
We know that the eigenfunction of $\chi_y^{+}$ is given by
$$\chi_{y+} = \frac{1}{\sqrt{2}}\left(\chi_{+}+i\chi_{-}\right)$$
so our projection operator becomes
\begin{align*}
\hat{P}_{y+}^1 &= \chi_{y+}^1(\chi_{y+}^1)^{\dagger}\\
&= \frac{1}{2}(\chi_{+}^1+i\chi_{-}^1)((\chi_{+}^1)^{\dagger}-i(\chi_{-}^1)^{\dagger})
\end{align*}
Now we project $\psi_{00}$ onto this state by
\begin{align*}
\hat{P}_{y+}^{1}\psi_{00} &= \frac{1}{2}(\chi_{+}^1+i\chi_{-}^1)((\chi_{+}^1)^{\dagger}-i(\chi_{-}^1)^{\dagger})\frac{1}{\sqrt{2}}\left(\chi_{+}^{1}\chi_{-}^{2} - \chi_{-}^{1}\chi_{+}^{2}\right)\\
&= \frac{1}{2\sqrt{2}}(\chi_{+}^1+i\chi_{-}^1)\left(\chi_{-}^{2} + i\chi_{+}^{2}\right)\\
&= \frac{1}{2\sqrt{2}}(\chi_{+}^1+i\chi_{-}^1)\left(\chi_{-}^{2} + i\chi_{+}^{2}\right)\\
\end{align*}

\item
\end{enumerate}

\section{Problem \#2}
\begin{enumerate}[(a)]
\item
Given that $\mathbf{J}$ is the total angular momentum given by $\mathbf{J} = \mathbf{S}+\mathbf{L}$ we can say that
\begin{align*}
J^2 &= S^2 + L^2 + 2\mathbf{S}\cdot\mathbf{L}\\
&\Downarrow\\
\mathbf{S}\cdot\mathbf{L} &= \frac{1}{2}(J^2 - S^2 - L^2)
\end{align*}
And now we can act on a state $\ket{lsjm}$ to find
\begin{align*}
\mathbf{S}\cdot\mathbf{L}\ket{lsjm} &=\frac{1}{2}(J^2 - S^2 - L^2)\ket{lsjm}\\ 
&=\frac{1}{2}(J^2\ket{lsjm} - S^2\ket{lsjm} - L^2\ket{lsjm})\\ 
&=\frac{\hbar^2}{2}(j(j+1)-s(s+1)-l(l+1))\ket{lsjm}  
\end{align*}
And for $s = 1$ we have
$$\mathbf{S}\cdot\mathbf{L}\ket{lsjm} =\frac{\hbar^2}{2}(j(j+1)-l(l+1)-2)\ket{lsjm}$$
so for $J = L - 1$ we have

\item
\item
\end{enumerate}

\section{Problem \#3}

\section{Problem \#4}

\end{document}
