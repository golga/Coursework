\documentclass[11pt]{article}

\usepackage{latexsym}
\usepackage{amssymb}
\usepackage{amsthm}
\usepackage{enumerate}
\usepackage{amsmath}
\usepackage{cancel}
\numberwithin{equation}{section}

\setlength{\evensidemargin}{.25in}
\setlength{\oddsidemargin}{-.25in}
\setlength{\topmargin}{-.75in}
\setlength{\textwidth}{6.5in}
\setlength{\textheight}{9.5in}
\newcommand{\due}{September 30th, 2010}
\newcommand{\HWnum}{5}
\newcommand{\grad}{\bold\nabla}
\newcommand{\vecE}{\vec{E}}
\newcommand{\scrptR}{\vec{\mathfrak{R}}}
\newcommand{\kapa}{\frac{1}{4\pi\epsilon_0}}
\newcommand{\emf}{\mathcal{E}}
\newcommand{\unit}[1]{\ensuremath{\, \mathrm{#1}}}


\begin{document}
\begin{titlepage}
\setlength{\topmargin}{1.5in}
\begin{center}
\Huge{Physics 3320} \\
\LARGE{Principles of Electricity and Magnetism II} \\
\Large{Professor Ana Maria Rey} \\[1cm]

\huge{Homework \#\HWnum}\\[0.5cm]

\large{Joe Becker} \\
\large{SID: 810-07-1484} \\
\large{\due} 

\end{center}

\end{titlepage}



\section{Problem \#1}
We assume that we have a plasma with an isotropic velocity distribution is placed in a magnetic mirror with a \emph{Mirror Ratio} of $R_m = 4$. So using the relation 
\begin{equation}
\sin^2(\theta_m) = \frac{1}{R_m}
\label{MagMir}
\end{equation}
where $\theta_m$ is the angle of the \emph{loss cone}. We see that equation \ref{MagMir} gives us
\begin{align*}
\sin^2(\theta_m) &= \frac{1}{R_m}\\
&\Downarrow\\
\theta_m &= \arcsin\left(\frac{1}{\sqrt{R_m}}\right)
\end{align*}
So for $R_m=4$ we find that $\theta_m = \frac{\pi}{6}$. Now we know that the plasma is isotropic so we assume a particle can equally be moving in any direction, but the particles with velocities inside the \emph{loss cone} for $\theta_m=\frac{\pi}{6}$. Now if we only consider the velocity that is parallel to the magnetic field we see that the velocities that are trapped are in the angle $\frac{2\pi}{3}$ so the fraction of trapped particles is given by
$$\frac{4\theta_m}{2\pi} = \frac{2\pi}{3}\frac{1}{2\pi} = \frac{1}{3}$$

\section{Problem \#2}
\begin{enumerate}[(a)]
\item
Given a cosmic ray proton that is trapped between two moving magnetic mirrors with $R_m=5$ with initial energy $W=1\unit{keV}$ and $v_{\perp}=v_{\parallel}$. We know that by the invariance of $\mu$ that $v^i_{\perp}=v^f_{\perp}$ where the superscripts $i$ and $f$ represent the initial and final velocities respectively. This implies that the parallel component of the velocity will change as the proton is accelerated prior to it escaping. So the loss cone formula states that
\begin{equation}
\frac{v^2_{\perp0}}{v^2_0} = \frac{1}{R_m}
\label{ConeLoss}
\end{equation}
where $v_0$ is the total final velocity and $v_{\perp0}$ is the perpendicular velocity. Now we see that the final velocity split into components is given by
$$v_0 = \sqrt{(v^f_{\perp})^2+(v^f_{\parallel})^2}$$
So we see that equation \ref{ConeLoss} yields
\begin{align*}
\frac{v^2_{\perp0}}{v^2_0} &= \frac{1}{R_m}\\
\frac{(v^f_{\perp})^2}{(v^f_{\perp})^2+(v^f_{\parallel})^2} &= \frac{1}{5}\\
&\Downarrow\\
5(v^f_{\perp})^2 &= (v^f_{\perp})^2+(v^f_{\parallel})^2\\
4(v^f_{\perp})^2 &= (v^f_{\parallel})^2\\
2v^f_{\perp} &= v^f_{\parallel}
\end{align*}
Now we can find the kinetic energy of the particle after it has been accelerated to escaping is 
\begin{align*}
W_f &= \frac{1}{2}m\left[(v^f_{\perp})^2 + (v^f_{\parallel})^2\right]\\
&= \frac{1}{2}m\left[(v^f_{\perp})^2 + (2v^f_{\perp})^2\right]\\
&= \frac{1}{2}m\left[5(v^f_{\perp})^2\right]\\
&= \frac{5}{2}m(v^f_{\perp})^2
\end{align*}
Now we need to compare this to the initial velocity by saying that
\begin{align*}
W_i &= \frac{1}{2}m\left[(v^i_{\perp})^2 + (v^i_{\parallel})^2\right]\\
&= \frac{1}{2}m\left[(v^f_{\perp})^2 + (v^f_{\perp})^2\right]\\
&= \frac{1}{2}m\left[2(v^f_{\perp})^2\right]\\
&= m(v^f_{\perp})^2
\end{align*}
Now we can see the ratio between the finial and initial energy is given by
\begin{align*}
\frac{W_f}{W_i} &= \frac{5/2m(v^f_{\perp})^2}{m(v^f_{\perp})^2}\\
\frac{W_f}{W_i} &= \frac{5}{2}\\
&\Downarrow\\
W_f &= \frac{5}{2}W_i
\end{align*}
Now we are given the fact that $W_i=1\unit{keV}$ so we see that
$$W_f = \frac{5}{2}(1\unit{keV}) = \frac{5}{2}\unit{keV}$$

\item
\begin{enumerate}[1)]
\item
If we assume that the mirrors are flat pistons traveling toward the midplane at a rate of $v_m = 10\unit{km/s}$ and the proton is traveling toward the piston at a velocity of $v_0$ we see that in the frame of the piston before the proton collides the velocity of the proton is 
$$v^i = v_0 - v_m$$
note that $v_m$ is in the opposite direction of $v_0$. So we let $v_m$ be a negative quantity. Now after the proton collides with the piston the velocity of the particle in the frame of the piston becomes
$$v^f = -v_0 + v_m$$
Note that the signs on both velocities flipped because now the particle is traveling in the opposite direction. Now if we convert back to the lab frame we see that the final velocity of the particle is given by
\begin{align*}
v'^f &= v^f +v_m\\
&= (-v_0+v_m) +v_m\\
&= -v_0+2v_m
\end{align*}
So the change in velocity is by $2v_m$

\item
Now we are given $v_m=10\unit{km/s}$ so for each "bounce" the particle gains $20\unit{km/s}$ in velocity. If we first need to find the initial velocity from the initial energy $W_i = 1\unit{keV}$
\begin{align*}
W_i &= m(v^i_{\perp})^2
&\Downarrow\\
v^i_{\parallel} &= v_0 = \sqrt{\frac{W_i}{m_p}}\\
&= \sqrt{\frac{1.6\times10^{-16}\unit{J}}{1.7\times10^{-27}\unit{kg}}}\\
&= 3.1\times10^{5}\unit{m\ s^{-1}}
\end{align*}
Now to get to the final velocity which we know to be $v_f = 2v_0$ from part (a) we see that the change in velocity is given by 
$$\Delta v = v_f - v_0 = 2v_0 - v_0 = v_0$$
And if we assume that the change is from a bounce we see that 
$$\Delta v = 2nv_m$$
where $n$ is the number of bounces. We can find $n$ by
\begin{align*}
n &= \frac{\Delta v}{2v_m}\\
&= \frac{v_0}{2v_m}\\
&= \frac{3.1\times10^{5}\unit{m/s}}{2(1\times10^{4}\unit{m/s})}\\
&= 15.5 \approx 16 \unit{bounces}
\end{align*}

\item
Now to find the total time it takes for the proton to get to the finial energy we first need to find how long the particle travels between each bounce. To find this we first need to find the average velocity of the proton which is given by
\begin{align*}
\bar{v} &= \frac{v_f + v_0}{2}\\
&= \frac{2v_0 +v_0}{2}\\
&= \frac{3v_0}{2}\\
&= \frac{3}{2}(3.1\times10^{5}\unit{m/s}) = 4.7\times10^{5}\unit{m/s}
\end{align*}
Now we can just use the given fact that the length between the mirrors is $L=10^10\unit{km}$ and the fact that the proton will travel this distance a total of 16 times. We see that the total time is 
\begin{align*}
T &= \frac{nL}{\bar{v}} \\
&= \frac{16(1\times10^{13})\unit{m}}{4.7\times10^{5}}\\
&= 3.4\times10^8\unit{s}
\end{align*}
\end{enumerate}
\end{enumerate}

\section{Problem \#3}
\begin{enumerate}[(a)]
\item
We can use the invariance in $J$ to find the same result from the previous problem where
$$J = \int_{0}^{L}v_{\parallel}ds \approx v_{\parallel}L$$
Since we are invariant in $J$ we see that this term is constant so if we differentiate $v_{\parallel}L$ with respect to time we get
$$0 = \frac{dv_{\parallel}}{dt}L+v_{\parallel}\frac{dL}{dt}$$

\item
Now we can see that the change in velocity over the total time $T$ is the same as the change in the parallel velocity or
$$\frac{dv_{\parallel}}{dt} = \frac{\Delta v_{\parallel}}{T}$$
Now we can use the result from part (a) to get $T$ in terms of $L$ 
\begin{align*}
0 &= \frac{dv_{\parallel}}{dt}L+v_{\parallel}\frac{dL}{dt}\\
&\Downarrow\\
\frac{dv_{\parallel}}{dt} &= -\frac{v_{\parallel}}{L}\frac{dL}{dt}\\
\frac{\Delta v_{\parallel}}{T} &= -\frac{v_{\parallel}}{L}\frac{dL}{dt}\\
\end{align*}
Now if solve for $T$ we get
$$T = -\frac{\Delta v_{\parallel}}{\bar{v}_{\parallel}}\frac{L}{\dot{L}}$$
Note that we take the average parallel velocity since the proton is accelerating. Now we know from problem 2 that $\Delta v_{\parallel} = v_{\perp}^i$ and the average velocity is $\bar{v}_{\parallel} = \frac{3}{2}v_{\perp}^i$ and we assume that $\dot{L} = -2v_m$ so we can find $T$ by saying
\begin{align*}
T &= -\frac{\Delta v_{\parallel}}{\bar{v}_{\parallel}}\frac{L}{\dot{L}}\\
&= -\frac{v^i_{\perp}}{\frac{3}{2}v^i_{\perp}}\frac{L}{-2v_m}\\
&= -\frac{2}{3}\frac{L}{-2v_m}\\
&= \frac{1}{3}\frac{10^{10}\unit{km}}{10\unit{km/s}}\\
&= 3.3\times10^{8}\unit{s}
\end{align*}
\end{enumerate}

\section{Problem \#4}
The particle sees a changing electric field perpendicular to its velocity. This is what accounts for the changing perpendicular velocity.
\end{document}

