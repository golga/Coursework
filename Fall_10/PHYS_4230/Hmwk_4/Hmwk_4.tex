\documentclass[11pt]{article}

\usepackage{latexsym}
\usepackage{amssymb}
\usepackage{amsthm}
\usepackage{enumerate}
\usepackage{amsmath}
\usepackage{cancel}
\numberwithin{equation}{section}

\setlength{\evensidemargin}{.25in}
\setlength{\oddsidemargin}{-.25in}
\setlength{\topmargin}{-.75in}
\setlength{\textwidth}{6.5in}
\setlength{\textheight}{9.5in}
\newcommand{\due}{September 17th, 2010}
\newcommand{\HWnum}{4}
\newcommand{\grad}{\bold\nabla}
\newcommand{\vecE}{\vec{E}}
\newcommand{\scrptR}{\vec{\mathfrak{R}}}
\newcommand{\kapa}{\frac{1}{4\pi\epsilon_0}}
\newcommand{\emf}{\mathcal{E}}

\begin{document}
\begin{titlepage}
\setlength{\topmargin}{1.5in}
\begin{center}
\Huge{Physics 3310} \\
\LARGE{Principles of Electricity and Magnetism 1} \\
\Large{Professor Thomas R. Schibli} \\[1cm]

\huge{Homework \#\HWnum}\\[0.5cm]

\large{Joe Becker} \\
\large{SID: 810-07-1484} \\
\large{\due} 

\end{center}

\end{titlepage}



\section{Problem \#1}
\begin{enumerate}[(a)]
\item
If we flip 50 fair coins we see that the number of microstates is the number of possible outcomes. Since each coin can have 2 different outcomes (heads or tails) there is a total of $2^{50}$ microstates.

\item
The total number of ways to get 25 heads and 25 tails or the multiplicity $\Omega$ of getting 25 heads is given by
$$\Omega(25)={50\choose25} = \frac{50!}{(25!)(25!)}\approx 1.26\times10^{14}$$

\item
The probability of getting 25 heads is $\Omega(25)$ over the total number of outcomes 
$$P(H=25) = \frac{\Omega(25)}{2^{50}} \approx 0.11$$

\item
To find the probability of getting 30 heads we first need to find $\Omega(30)$ by
$$\Omega(30) = {50\choose30} =\frac{50!}{(30!)(20!)}\approx 4.71\times10^{13}$$
now we just divide out by the total number of outcomes or $2^{50}$ to get
$$P(H=30) = \frac{\Omega(30)}{2^{50}} \approx 0.042$$ 

\item
The multiplicity of getting 40 heads is given by
$$\Omega(40) = {50\choose40} =\frac{50!}{(40!)(10!)}\approx 1.03\times10^{10}$$
so the probability is 
$$P(H=40) = \frac{\Omega(40)}{2^{50}} \approx 9.1\times10^{-6}$$

\item
To get 50 heads out of 50 coins we see that
$$\Omega(50) = {50\choose50} = \frac{50!}{(50!)(0!)} = 1$$
So the probability of getting 50 heads is
$$P(H=50) = \frac{\Omega}{2^{50}} = \frac{1}{2^{50}} \approx 8.9\times10^{-16}$$

\item
See attached for the plot
\end{enumerate}

\section{Problem \#2}
The total number of 5 card poker hands that you can be dealt from a 52 card deck is given by
$${52\choose5} = \frac{52!}{(5!)(47!)} = 2598960$$
Now there is only 4 possible ways to get a royal flush (one for each suit) so the probability of getting a royal flush on the first deal is
$$P(\textnormal{Royal Flush}) = \frac{4}{2598960} \approx 1.54\times10^{-6}$$

\section{Problem \#3}
\begin{enumerate}[(a)]
\item
Each dipole $N$ can be either $+1$, $-1$, or $0$ so for each dipole there are 3 possible states. So for a system with $N$ dipoles there is a total of $3^N$ microstates.

\item
If we take the general form for finding the multiplicity for a 2 state system given by
$$\Omega(N,M) = {N\choose\frac{M+N}{2}}$$
Note that we can reduce the 3 body state to a 2 body state because we assume that $N_0=0$

\item
Given that we have a total of $N$ dipoles then we know that we have $N!$ different ways to order the dipoles. This does not take into account dipoles that are the same for example we have over count by $N_0!$ for all the $N_0$ dipoles so we can just divide these out to get the total number of microstates as
$$\frac{N!}{(N_0!)(N_{-1}!)(N_1!)}$$


\item
Using the assumption that $M=0$ and that there is an even number of dipoles we can infer that the total number of up dipoles ($N_1$) is equal to the total number of down dipoles $N_{-1}$ or
$$N_1 = N_{-1}$$
So the total number of microstates for $M=0$ is 
$$\frac{N!}{(N_0!)(N_{-1}!)^2}$$


\end{enumerate}
\end{document}

