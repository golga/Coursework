\documentclass[11pt]{article}

\usepackage{latexsym}
\usepackage{amssymb}
\usepackage{amsthm}
\usepackage{enumerate}
\usepackage{amsmath}
\usepackage{cancel}

\setlength{\evensidemargin}{.25in}
\setlength{\oddsidemargin}{-.25in}
\setlength{\topmargin}{-.75in}
\setlength{\textwidth}{6.5in}
\setlength{\textheight}{9.5in}
\newcommand{\due}{September 18th, 2009}
\newcommand{\HWnum}{3}

\begin{document}
\begin{titlepage}
\setlength{\topmargin}{1.5in}
\begin{center}
\Huge{Physics 3310} \\
\LARGE{Principles of Electricity and Magnetism 1} \\
\Large{Professor Thomas R. Schibli} \\[1cm]

\huge{Homework \#\HWnum}\\[0.5cm]

\large{Joe Becker} \\
\large{SID: 810-07-1484} \\
\large{\due} 

\end{center}

\end{titlepage}


\section{Chapter 3}
\begin{enumerate}
\item Problem 1\\
We are looking for
$$P(\textnormal{at least one}\ 6\ |\textnormal{the dice land on different numbers})$$
So out of the 36 possible out comes for 2 dice being rolled we can eliminate the 6 cases where the dice land on the same number. So our subspace is now the 30 possible out comes. And if we think about the case where the first die is 6 we know that the other die can be anything from 1 through 5. This same hold true for the case where the secound die is 6. So there is a total of 10 possible "good" outcomes so the the probiblity is 
$$P(\textnormal{at least one}\ 6\ |\textnormal{the dice land on different numbers}) =\frac{10}{30}$$
\begin{center}
\fbox{$=\dfrac{1}{3}$}
\end{center}

\item Problem 4\\
So we are looking for
$$P(6|\textnormal{the sum is}\ i)$$
for a pair of dice. Where $i = 2,3,...12$. So we can quickly see that any sum that is less than or equal to 6 cannot have a 6. So we can say
$$P(6|\textnormal{the sum is}\ j) =0;\ j=2,3...6$$
The next obvious probability is for the sum is equal to 11 or 12. For the only way to get this sum is to have a 6. So we can say
$$P(6|\textnormal{the sum is}\ k) =1;\ j=11,12$$
Now for the case where the sum is 7. We see that the reduced space only contains 6 events and with 2 that have a 6 rolled ($\{6,1\},\{1,6\}$). So we can say
$$P(6|\textnormal{the sum is}\ 7) = \frac{2}{6} = \frac{1}{3}$$
For the sum is 8 there is only 5 events of which 2 contain a 6 ($\{6,2\},\{2,6\}$). So
$$P(6|\textnormal{the sum is}\ 8) = \frac{2}{5}$$
For the sum is 9 the reduced space have 4 events of which 2 contain 6 ($\{6,3\},\{3,6\}$). So
$$P(6|\textnormal{the sum is}\ 9) = \frac{2}{4}=\frac{1}{2}$$
For the sum is 10 there is only 3 events of which 2 contain a 6 ($\{6,4\},\{4,6\}$). So
$$P(6|\textnormal{the sum is}\ 8) = \frac{2}{3}$$
\item Problem 6\\
So the condition reduces the sample space down to the situation where we have 3 white balls out of 4 total balls. So the probability of the first and the third being white no longer depends on how the balls are drawn (with or without replacement). So to find the probability that both the first and the third ball are white we look at the case where either the first or the third is not white. Since it is a $\frac{1}{4}$ chance for either of them a happening and it each event is mutually exclusive the probability is given by
$$P(\textnormal{1st not white}\cup\textnormal{3rd not white}) = P(\textnormal{1st not white})+P(\textnormal{3rd not white}) = \frac{1}{4}+\frac{1}{4}$$
So we get 50 percent chance of what we want not happening so the oppisite is
$$1 -\frac{1}{2}$$
\begin{center}
\fbox{$=\dfrac{1}{2}$}
\end{center}
\item Problem 7\\
$$\textnormal{Let} G_1 = \textnormal{Child 1 is a girl}$$
$$\textnormal{Let} G_2 = \textnormal{Child 2 is a girl}$$
$$\textnormal{Let} B_1 = \textnormal{Child 1 is a boy}$$
$$\textnormal{Let} B_2 = \textnormal{Child 2 is a boy}$$
And we know that $B_n = G_n^c$ and that 
$$P(G_1)=P(G_2)=P(B_1)=P(B_2)=\frac{1}{2}$$
So we want to say what the probability that one or the other child is a girl given that one or the other is a boy. This is written as
$$P(G_1\cup G_2|B_1\cup B_2)$$
And we know through the multiplication rule that
$$P(G_1\cup G_2|B_1\cup B_2) = \frac{P((G_1\cup G_2)(B_1\cup B_2))}{P(B_1\cup B_2)}$$
So we can calculate the probability that one of the two is a boy.
$$P(B_1\cup B_2) = P(B_1) + P(B_2) - P(B_1B_2)$$
Because they are independent events (the fact that one or the other is a boy does not change the probability of the other), give us
$$P(B_1\cup B_2) = P(B_1) + P(B_2) - P(B_1)P(B_2)$$
$$P(B_1\cup B_2) = \frac{1}{2}+\frac{1}{2}-\frac{1}{2}\frac{1}{2}$$
$$P(B_1\cup B_2) = \frac{3}{4}$$

Now we can find the probability that one is a girl and the other is a boy and vise versa
$$P((G_1\cup G_2)(B_1\cup B_2))=P(G_1B_2)+P(G_2B_1)$$ 
Since these events are independent we can say
$$P((G_1\cup G_2)(B_1\cup B_2))=P(G_1)P(B_2)+P(G_2)P(B_1)$$ 
$$P((G_1\cup G_2)(B_1\cup B_2))=\frac{1}{2}\frac{1}{2}+\frac{1}{2}\frac{1}{2}$$
$$P((G_1\cup G_2)(B_1\cup B_2))=\frac{1}{2}$$
So we can find 
$$P(G_1\cup G_2|B_1\cup B_2) = \frac{1/2}{3/4}$$
\begin{center}
\fbox{$=\dfrac{2}{3}$}
\end{center}
Note we can also look at it as the whole sample space as
$$S = \{(B,G),(G,B),(G,G),(B,B)\}$$
And we take the condition the one of the children is a boy we see the reduced sample space as
$$S = \{(B,G),(G,B),(B,B)\}$$
And we can quickly see that 2 out of the 3 cases have the other child as a girl.


\item Problem 8\\
We are looking for
$$P(\textnormal{both are girls}|\textnormal{older child is a girl})$$
so we know that the if the younger child is a girl then both are girls. So this probability is the probability of the younger child being a girl which is 
\begin{center}
\fbox{$=\dfrac{1}{2}$}
\end{center}

\item Problem 9\\
$$\textnormal{Let}\ A = \textnormal{drawing a white from urn A}$$
$$\textnormal{Let}\ B = \textnormal{drawing a white from urn B}$$
$$\textnormal{Let}\ C = \textnormal{drawing a white from urn C}$$
$$\textnormal{Let}\ E = \textnormal{drawing two white balls}$$
So we are trying to find $$P(A|E)$$
Thinking about the problem you have three ways to draw two white balls (the red ball can either come from urn A, B, or C). So we can express
$$E=A^cBC\cup AB^cC\cup ABC^c$$
And we now due to the multiplication rule that 
$$P(A|E) = \frac{P(AE)}{P(E)}$$
So we get 
$$P(A|E) = \frac{AB^cC\cup ABC^c}{P(A^cBC\cup AB^cC\cup ABC^c)}$$
And we know that each event in $E$ is mutually exclusive (there is no way that if you draw the red from urn A you will draw it from any other), this fact gives us 
$$P(A|E) = \frac{P(AB^cC)+ P(ABC^c)}{P(A^cBC)+P(AB^cC)+P(ABC^c)}$$
The last thing we need to realize is the fact that because they are separate urns events $A$, $B$, and $C$ are independent so their individual probabilities can just be multiplied together. This gives us
$$P(A|E) = \frac{\frac{2}{6}\frac{4}{12}\frac{1}{4}+\frac{2}{6}\frac{8}{12}\frac{3}{4}}{\frac{4}{6}\frac{8}{12}\frac{1}{4}+\frac{2}{6}\frac{4}{12}\frac{1}{4}+\frac{2}{6}\frac{8}{12}\frac{3}{4}}$$
$$P(A|E) = \frac{\dfrac{8+48}{\cancel{288}}}{\dfrac{32+8+48}{\cancel{288}}}$$
$$P(A|E) = \frac{56}{88}$$
\begin{center}
\fbox{$=\dfrac{7}{11}$}
\end{center}

\item Problem 10\\
So with the condition that the second and third cards are spades we reduce our sample space to 50, and we now only have 11 spades since 2 were taken from the 13 so our probability is
\begin{center}
\fbox{$=\dfrac{11}{50}$}
\end{center}

\item Problem 11
$$\textnormal{Let} B = \textnormal{both are aces}$$
$$\textnormal{Let} A_s = \textnormal{ace of spades is chosen}$$
$$\textnormal{Let} A = \textnormal{at least one ace is chosen}$$
\begin{enumerate}[(a)]
\item We are trying to find $P(B|A_s)$. Here we can see that the reduced sample space is reduced my one card and one ace. So out the remaining 51 cards theres 3 aces to pick from so
$$P(B|A_s) = \frac{3}{51}$$
\begin{center}
\fbox{$=\dfrac{1}{17}$}
\end{center}

\item We are trying to find $P(B|A)$. For this problem we can use the multiplication property 
$$P(B|A)=\frac{P(AB)}{P(B)}$$
We see that if you are going to draw 2 aces you have to choose at least one too so we see that 
$$B\subset A$$
This makes 
$$P(B|A)=\frac{P(A)}{P(B)}$$
So we can say that
$$P(A) = \frac{{4\choose2}}{{52\choose2}}$$
$$P(A) = \frac{6}{1326}$$
And that
$$P(B) = \frac{1326-1128}{1326}$$
$$P(B) = \frac{198}{1326}$$
So we can see the 1326 cancel and we get
$$P(B|A)=\frac{6}{198}$$
\begin{center}
\fbox{$=\dfrac{1}{33}$}
\end{center}
\end{enumerate}
\end{enumerate}
\end{document}

