\documentclass[11pt]{article}

\usepackage{latexsym}
\usepackage{amssymb}
\usepackage{amsthm}
\usepackage{enumerate}
\usepackage{amsmath}
\usepackage{cancel}

\setlength{\evensidemargin}{.25in}
\setlength{\oddsidemargin}{-.25in}
\setlength{\topmargin}{-.75in}
\setlength{\textwidth}{6.5in}
\setlength{\textheight}{9.5in}
\newcommand{\due}{October 2nd, 2009}
\newcommand{\HWnum}{5}

\begin{document}
\begin{titlepage}
\setlength{\topmargin}{1.5in}
\begin{center}
\Huge{Physics 3320} \\
\LARGE{Principles of Electricity and Magnetism II} \\
\Large{Professor Ana Maria Rey} \\[1cm]

\huge{Homework \#\HWnum}\\[0.5cm]

\large{Joe Becker} \\
\large{SID: 810-07-1484} \\
\large{\due} 

\end{center}

\end{titlepage}


\section{Chapter 4}
\begin{enumerate}
\item Problem 1\\
Let $X$ denote our winnings
Note all calculations are done with the assumption that the balls are not being replaced.

For the case where $X=4$, we see that for drawing a black ball first we have 4 black balls out of 14 total balls to choose from. Once we draw a black ball there are only 3 black remaining out of now 13 total.
$$P(X=4) = \frac{4}{14}\frac{3}{13}$$ 
$$P(X=4) = \frac{12}{182} = \frac{6}{91}$$
For the case where $X=-2$, we pick 8 possible white balls from the 14 total balls, and then after one white is drawn we pick a possible 7 white from 13 total.
$$P(X=-2) = \frac{8}{14}\frac{7}{13}$$ 
$$P(X=-2) = \frac{56}{182} = \frac{28}{91}$$
For the case where $X=0$, we pick 2 orange balls from the 14 total, and once one orange ball we have 1 orange from the 13 total.
$$P(X=0) = \frac{2}{14}\frac{1}{13}$$ 
$$P(X=0) =\frac{2}{182} =  \frac{1}{91}$$
For the case $X=1$, we draw one black ball and one white ball. So we pick from 8 possible white balls from a total of 14 balls then we pick from 4 possible black balls from a total of 13 balls. We can do this 2 choose 1 ways. 
$$P(X=1) = {2\choose1}\frac{8}{14}\frac{4}{13}$$ 
$$P(X=1) =\frac{64}{182} =  \frac{32}{91}$$
For the case where $X=-1$, we draw an orange ball and a white ball. So we pick from 8 possible white balls from a total of 14 balls then we pick from 2 possible orange balls from a total of 13 balls. We can do this 2 choose 1 ways. 
$$P(X=-1) = {2\choose1}2\frac{8}{14}\frac{2}{13}$$ 
$$P(X=-1) = \frac{32}{182} = \frac{16}{91}$$
For $X=2$ we have the case where we draw a black and a orange or
For the case where $X=2$, we draw an orange ball and a black ball. So we pick from 4 possible black balls from a total of 14 balls then we pick from 2 possible orange balls from a total of 13 balls. We can do this 2 choose 1 ways. 
$$P(X=2) = {2\choose2}1\frac{4}{14}\frac{2}{13}$$ 
$$P(X=2) = \frac{16}{182} = \frac{8}{91}$$
Note that for all other $i$ 
$$P(X=i)=0$$.

Here is the full list of outcomes
\begin{align*}
&P(X=4) = \frac{6}{91}\\
&P(X=2) = \frac{8}{91}\\
&P(X=1) = \frac{32}{91}\\
&P(X=0) = \frac{1}{91}\\
&P(X=-1) = \frac{16}{91}\\
&P(X=-2) = \frac{28}{91}\\
\end{align*}
Note that the sum of the probabilities is 1.

\item Problem 2\\
Let $X$ equal the product of 2 fair dice.
$$\left(\begin{array}{c|cccccc}
		&1	&2	&3	&4	&5	&6	\\
\hline
	1	&1	&2	&3	&4	&5	&6	\\	
	2	&2	&4	&6	&8	&10	&12	\\	
	3	&3	&6	&9	&12	&15	&18	\\	
	4	&4	&8	&12	&16	&20	&24	\\	
	5	&5	&10	&15	&20	&25	&30	\\	
	6	&6	&12	&18	&24	&30	&36	\\	
							\end{array}\right)$$

Counting from the array above we get	
\begin{align*}
&P\{X=i\} = \frac{1}{36} \textnormal{ for } i=1,9,16,25,36\\
&P\{X=i\} = \frac{2}{36} = \frac{1}{18} \textnormal{ for } i=2,3,4,8,10,15,18,20,24,30\\
&P\{X=i\} = \frac{3}{36}=\frac{1}{12} \textnormal{ for } i=4\\
&P\{X=i\} = \frac{4}{36}=\frac{1}{12} \textnormal{ for } i=6,12\\
&P\{X=i\}=0 \textnormal{ for } i=7,11,13,14,17,19,21,22,23,26,27,28,29,31,32,33,34,35\\ 
\end{align*}
%$$P(X=1)=\frac{1}{36}$$
%$$P(X=9)=\frac{1}{36}$$
%$$P(X=16)=\frac{1}{36}$$
%$$P(X=25)=\frac{1}{36}$$
%$$P(X=36)=\frac{1}{36}$$
%$$P(X=2)=\frac{2}{36}$$
%$$P(X=3)=\frac{2}{36}$$
%$$P(X=5)=\frac{2}{36}$$
%$$P(X=8)=\frac{2}{36}$$
%$$P(X=10)=\frac{2}{36}$$
%$$P(X=15)=\frac{2}{36}$$
%$$P(X=18)=\frac{2}{36}$$
%$$P(X=20)=\frac{2}{36}$$
%$$P(X=24)=\frac{2}{36}$$
%$$P(X=30)=\frac{2}{36}$$
%$$P(X=4)=\frac{3}{36}$$
%$$P(X=6)=\frac{4}{36}$$
%$$P(X=12)=\frac{4}{36}$$
Note that the sum of all the probabilities 
$$\sum_{i=1}^{36} P\{X=i\} = 1$$

\item Problem 3\\
We know that $X=i$ where $i=3,4,5...18$. So we can say that
\begin{align*}
&\{X=i\} = \frac{1}{216} \textnormal{ for } i=3,18\\
&\{X=i\} = \frac{3}{216} \textnormal{ for } i=4,17\\
&\{X=i\} = \frac{6}{216} \textnormal{ for } i=5,16\\
&\{X=i\} = \frac{10}{216} \textnormal{ for } i=6,15\\
&\{X=i\} = \frac{15}{216} \textnormal{ for } i=7,14\\
&\{X=i\} = \frac{21}{216} \textnormal{ for } i=8,13\\
&\{X=i\} = \frac{25}{216} \textnormal{ for } i=9,12\\
&\{X=i\} = \frac{27}{216} \textnormal{ for } i=10,11\\
\end{align*}
\item Problem 5\\
We know that the highest possible outcome is $n$ that is all heads, on the other side we see that all tails is $-n$ and for everyone that changes we get a change in 2 so
$$X=i$$
where
$$i = -n, -n+2, -n+4,...,-n+n,n+2,n+4,...,n$$
\item Problem 25\\
\begin{enumerate}
\item
We can say that the $P(X=1)$ is the probability we get heads on the first coin($H_1$) and tails on the 2nd coin ($T_2$) or tails on the first and heads on the first so
$$P(X=1) = P(H_1T_2\cup T_1H_2)$$
$H_1T_2$ is mutually exclusive from $T_1H_2$ so
$$P(X=1) = P(H_1T_2)+ P(T_1H_2)$$
Because $H_1$ and $T_2$ are independent as well as $H_2$ and $T_1$ we can say
$$P(X=1) = P(H_1)P(T_2)+ P(T_1)P(H_2)$$
$$P(X=1) = (0.6)(0.3)+(0.7)(0.4)$$
$$P(X=1) = 0.46$$
\item
The expectation is given by 
$$E(X) = \sum_{i}^{2}X_iP(X_i)$$
$$E(X) = 0(0.12)+1(0.46)+2(0.42)$$
$$E(X) = 1.3$$
\end{enumerate}
\item Problem 28\\
Let $X$ be the number of defective items when we draw 3. Where $X=i$ and $i=0,1,2,3$. So we can calculate the probabilities of each $X$. For $X=0$ we know that we have a 16 out of 20 chance on the first draw then 15 out of 19 and 14 out of 18 so
$$P\{X=0\} = \frac{16}{20}\frac{15}{19}\frac{14}{18}$$
$$P\{X=0\} = \frac{3360}{6840}$$
For $X=1$ we pick one defective out of the 4 possible which has a chance of 4 out of 20 and then 16 out of 19 then 15 out of 18 for the non defective items. With 3 choose 1 different ways 1 defective item can be picked so
$$P\{X=1\} = {3\choose1}\frac{4}{20}\frac{16}{19}\frac{15}{18}$$
$$P\{X=1\} = 3\frac{960}{6840}$$
For $X=2$ we pick 2 defective items. So 4 out of 20 then 3 out of 19 then 16 out of 18 for the non defective we do this 3 choose 2 different ways
$$P\{X=2\} = {3\choose2}\frac{4}{20}\frac{3}{19}\frac{16}{18}$$
$$P\{X=1\} = 3\frac{192}{6840}$$
For $X=3$ 
$$P\{X=2\} = {3\choose3}\frac{4}{20}\frac{3}{19}\frac{2}{18}$$
$$P\{X=1\} = \frac{24}{6840}$$
To calculate the expected value we use
$$E(X) = \sum_{i}^{2}X_iP(X_i)$$
$$E(X) = 0\left(\frac{3360}{6840}\right)+1\left(3\frac{960}{6840}\right)+2\left(3\frac{192}{6840}\right)+3\left(\frac{24}{6840}\right)$$
$$E(X) = \frac{4104}{6840}$$
$$E(X) = \frac{3}{5}$$
\end{enumerate}
\end{document}

