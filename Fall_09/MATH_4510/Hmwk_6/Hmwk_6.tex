\documentclass[11pt]{article}

\usepackage{latexsym}
\usepackage{amssymb}
\usepackage{amsthm}
\usepackage{enumerate}
\usepackage{amsmath}
\usepackage{cancel}

\setlength{\evensidemargin}{.25in}
\setlength{\oddsidemargin}{-.25in}
\setlength{\topmargin}{-.75in}
\setlength{\textwidth}{6.5in}
\setlength{\textheight}{9.5in}
\newcommand{\due}{October 9th, 2009}
\newcommand{\HWnum}{6}
\newcommand{\Var}{\textnormal{Var}}
\newcommand{\dsum} {\displaystyle\sum}

\begin{document}
\begin{titlepage}
\setlength{\topmargin}{1.5in}
\begin{center}
\Huge{Physics 3310} \\
\LARGE{Principles of Electricity and Magnetism 1} \\
\Large{Professor Thomas R. Schibli} \\[1cm]

\huge{Homework \#\HWnum}\\[0.5cm]

\large{Joe Becker} \\
\large{SID: 810-07-1484} \\
\large{\due} 

\end{center}

\end{titlepage}


\section{Chapter 4}
\begin{enumerate}
\item Problem 30
\begin{enumerate}[(a)]
\item
We can calculate the expectation using
$$E[X] = \sum_{n=1}^{\infty} x_nP(x_i)$$
where $x_n$ is the amount of money that you win. You win $2^n$ dollars, where $n$ is the number of flips it took to get your first tails. And we know that the probability of having a tail land on the $n$th flip is given by
$$P(x_n)=\frac{1}{2^n}$$
Because it would require $n-1$ heads in a row. So we can say the expectation is
$$E[X] = \sum_{n=1}^{\infty} 2^n\frac{1}{2^n}$$
$$E[X] = \sum_{n=1}^{\infty}1$$
$$E[X] = \infty$$
So if the expectation is $\infty$, but that is a misleading number because if you were to only play once for a million dollars it would be unlikely that you would get heads enough times in a row on you first try to break even.
\item If you got to play until you got lucky and made enough to be positive then it would be worth it.

\end{enumerate}
\item Problem 33\\
So if we let $n$ be the number of papers that the paperboy buys. And let $X$ be the number of papers he sells. And let $Y$ be the demand for papers. So we can say that if the demand is greater than the number of papers the paperboy has ($Y\ge n$) we can say that $$X = n$$ 
For the case where the demand does not exceed the number of papers the paperboy has we have
$$X=Y$$
Since we know that $Y$ is given by a binomial random variable we can say that
$$P(X=i) = \left\{\begin{array}{lc}
		&P(Y=i) = \dbinom{10}{i}\left(\dfrac{1}{3}\right)^i\left(\dfrac{2}{3}\right)^{10-i}\ \mbox{for } i<n\\
		&P(Y\ge n)= \dsum_{j=n}^{10}\dbinom{10}{j}\left(\dfrac{1}{3}\right)^j\left(\dfrac{2}{3}\right)^{10-j}\ \mbox{for } i=n\\
		\end{array}\right.$$
And we know that his profit is given by $0.15X-0.10n$ So if we calculate the expectation of his profit we get
$$E[0.15X-0.10n]=0.15E[X]-0.10n$$
Where $E[X]$ can be calculated as
$$E[X] = \sum_{i=0}^{n-1} \dbinom{10}{i}\left(\dfrac{1}{3}\right)^i\left(\dfrac{2}{3}\right)^{10-i} + \sum_{j=n}^{10} \dbinom{10}{j}\left(\dfrac{1}{3}\right)^j\left(\dfrac{2}{3}\right)^{10-j}$$
We can see that 3 yields the greatest profit.

\item Problem 38
\begin{enumerate}[(a)]
\item
Assuming that 
$E[X] = 1$
We can calculate 
$$E[(2+X)^2]$$
First we distribute the square
$$E[(2+X)^2]=E[X^2+4X+4]$$
Now we can use the property of expectation values
$$E[aX+b] = aE[X]+b$$
to distribute the function
$$E[(2+X)^2]=E[X^2]+4E[X]+4$$
So we know that $E[X]=1$ so
$$E[(2+X)^2]=E[X^2]+4+4$$
$$E[(2+X)^2]=E[X^2]+8$$
Now we can add and subtract a $(E[X])^2$
$$E[(2+X)^2]=(E[X^2]-(E[X])^2)+E[X]^2+8$$
We know from the definition of variance 
$$\Var(X) = E[X^2]-(E[X])^2$$
and we know from the problem that 
$$\Var(X) = 5$$
so we get
$$E[(2+X)^2]=\Var(X)+E[X]^2+8$$
$$E[(2+X)^2]=5+1^2+8$$
$$E[(2+X)^2]=14$$
\item
Assuming that
$$E[X] = 1$$
$$\Var(X) = 5$$
We can calculate 
$$\Var(4+3X)$$
We can use the property of $\Var$
$$\Var(aX+b) = a^2\Var(X)$$
So this identity yields
$$\Var(4+3X)=3^2\Var(X)$$
$$\Var(4+3X)=3^2(5)$$
$$\Var(4+3X)=45$$

\end{enumerate}
\item Problem 40\\
On a multiple-choice exam with 3 possible answers for each of the 5 questions. Let $X$ denote the number of correct answers a student got just by guessing. We wee that $X$ is represented by a binomial random variable where $n=5$ and $p=\frac{1}{3}$. 
So the probability is given by
$$P(X=i) = {n\choose i}p^i(1-p)^{n-i}$$
Where $i$ is the number of successful guesses. So to find the probability that $X\ge4$ we set $i=4$ and $i=5$
$$P(X\ge4) = {5\choose 4}\left(\frac{1}{3}\right)^4\left(1-\frac{1}{3}\right)^{5-4}+{5\choose 5}\left(\frac{1}{3}\right)^5\left(1-\frac{1}{3}\right)^{5-5}$$
$$P(X\ge4) = 5\left(\frac{1}{3}\right)^4\left(\frac{2}{3}\right)+1\left(\frac{1}{3}\right)^5\left(\frac{2}{3}\right)^{0}$$
$$P(X\ge4) = 5\frac{2}{3^5}+\frac{1}{3^5}$$
$$P(X\ge4) = \frac{10}{243}+\frac{1}{243}$$
$$P(X\ge4) = \frac{11}{243}$$

\item Problem 51
\begin{enumerate}[(a)]
\item
If we let $X$ be the number of typographical errors on a page of a certain magazine. We see that we have a \emph{Poisson} distribution which is given by
$$P(X=i) = e^{-\lambda}\frac{\lambda^i}{i!}$$
Where $\lambda$ is expected number of typographical errors given as $\lambda =0.2$. We can calculate the probability that the next page has zero errors or $i=0$
$$P(X=0) = e^{-\lambda}\frac{\lambda^0}{0!}$$
$$P(X=0) = e^{-\lambda}1$$
$$P(X=0) = e^{-0.2}$$
\item
Now for when $X\ge2$ we know that having more than 2 errors is the same as not having less that 1 error or
$$P(X\ge2) = 1 - P(X\le1)$$
Where $P(X\le1)=P(X=0)+P(X=1)$
So we can calculate
$$P(X\ge2) = 1 - (P(X=0)+P(X=1))$$
$$P(X=0)+P(X=1) = e^{-\lambda}\frac{\lambda^0}{0!} + e^{-\lambda}\frac{\lambda^1}{1!}$$
$$P(X=0)+P(X=1) = e^{-\lambda}+ e^{-\lambda}\lambda$$
$$P(X=0)+P(X=1) = (1+\lambda)e^{-\lambda}$$
Where $\lambda = 0.2$ so 
$$P(X\ge2) = 1 - (1+\lambda)e^{-\lambda}$$
$$P(X\ge2) = 1 - (1+0.2)e^{-0.2}$$
$$P(X\ge2) = 1 - 1.2e^{-0.2}$$
\end{enumerate} 

\item Problem 52
\begin{enumerate}[(a)]
\item
If we let $X$ be the number of plane crashes in a month we see that we have a \emph{Poisson} distribution. With $\lambda$ being the expected number of crashes in a month or $\lambda = 3.5$. So
$$P(X=i) = e^{-\lambda}\frac{\lambda^i}{i!}$$
Now for 2 crashes next month we have $i=2$ or
$$P(X=i) = e^{-\lambda}\frac{\lambda^2}{2!}$$
$$P(X=i) = e^{-3.5}\frac{(3.5)^2}{2}$$
$$P(X=i) = 0.0049$$
\item
Now for the case where we have at most 1 crash we have to sum the possibility that we have no crashes $i=0$ and we have one crash $i=1$
$$P(X\le1)=P(X=0)+P(X=1)= e^{-\lambda}\frac{\lambda^0}{0!}+e^{-\lambda}\frac{\lambda^1}{1!}$$
$$P(X=0)+P(X=1)= e^{-3.5}(1)+e^{-3.5}\frac{3.5}{1}$$
$$P(X=0)+P(X=1)= 0.136$$
\end{enumerate}

\item Problem 57
\begin{enumerate}[(a)]
\item
Let $X$ be the number of accidents occurring on a highway each day. Where $X$'s distribution is given by
$$P(X=i) = e^{-\lambda}\frac{\lambda^i}{i!}$$
Where $\lambda$ is given to equal $3$. So to find the probability that 3 or more accidents occur in a day we calculate $X\ge3$. We know this is the probability that we don't have less than 2 accidents or
$$P(X\ge3) = 1- P(X\le2)$$
We can calculate $P(X\le2)$ by saying
$$P(X\le2) = P(X=0)+P(X=1)+P(X=2)$$
$$P(X\le2) = e^{-\lambda}\frac{\lambda^0}{0!} + e^{-\lambda}\frac{\lambda^1}{1!} + e^{-\lambda}\frac{\lambda^2}{2!}$$
$$P(X\le2) = e^{-\lambda}\left(\frac{\lambda^0}{0!} + \frac{\lambda^1}{1!} + \frac{\lambda^2}{2!}\right)$$
$$P(X\le2) = e^{-\lambda}\left(1 + \lambda + \frac{\lambda^2}{2}\right)$$
$$P(X\le2) = e^{-3}\left(1 + 3 + \frac{3^2}{2}\right)$$
$$P(X\le2) = e^{-3}\left(\frac{17}{2}\right)$$
Now we calculate
$$P(X\ge3) = 1 - e^{-3}\left(\frac{17}{2}\right)$$
$$P(X\ge3) \approx 0.5768$$
\item
If we assume that we already had 1 crash today we can say that having more than three crashes can be represented by
$$P(X\ge3) = 1- (P(X=1)+P(X=2))$$
So 
$$P(X=1)+P(X=2) = e^{-\lambda}\frac{\lambda^1}{1!} + e^{-\lambda}\frac{\lambda^2}{2!}$$
$$P(X=1)+P(X=2) = e^{-\lambda}\left(\lambda^1+\frac{\lambda^2}{2}\right)$$
$$P(X=1)+P(X=2) = e^{-3}\left(3+\frac{3^2}{2}\right)$$
$$P(X=1)+P(X=2) = e^{-3}\left(\frac{15}{2}\right)$$
Now we calculate 
$$P(X\ge3) = 1- e^{-3}\left(\frac{15}{2}\right)$$
$$P(X\ge3) \approx 0.6266$$
\end{enumerate}

\end{enumerate}
\end{document}

