\documentclass[11pt]{article}

\usepackage{latexsym}
\usepackage{amssymb}
\usepackage{amsthm}
\usepackage{enumerate}
\usepackage{amsmath}
\usepackage{cancel}

\setlength{\evensidemargin}{.25in}
\setlength{\oddsidemargin}{-.25in}
\setlength{\topmargin}{-.75in}
\setlength{\textwidth}{6.5in}
\setlength{\textheight}{9.5in}
\newcommand{\due}{November 20th, 2009}
\newcommand{\HWnum}{11}

\begin{document}
\begin{titlepage}
\setlength{\topmargin}{1.5in}
\begin{center}
\Huge{Physics 3310} \\
\LARGE{Principles of Electricity and Magnetism 1} \\
\Large{Professor Thomas R. Schibli} \\[1cm]

\huge{Homework \#\HWnum}\\[0.5cm]

\large{Joe Becker} \\
\large{SID: 810-07-1484} \\
\large{\due} 

\end{center}

\end{titlepage}


\section{Chapter 8}
\begin{enumerate}
\item Problem 1\\
Suppose $X$ is a random variable where $E[X] = \mu = 20$ and $Var(X) = 20$. We can find the limit of $P\{0<X<40\}$. First we subtract the mean to yield
\begin{align*}
P\{0<X<40\} &= P\{0-\mu<X-\mu<40-\mu\}\\
&= P\{-20<X-\mu<20\}\\
&= P\{|X-\mu|<20\}
\end{align*}
Now we can apply the \emph{Chebyshev Inequality} which states
\begin{equation} 
P\{|X-\mu|\ge k\} \le \frac{\sigma_x^2}{k^2}
\label{chebineq} 
\end{equation} 
where $\sigma_x^2$ is the variance of $X$. To apply equation \ref{chebineq} we first have to say that
$$P\{|X-\mu|<20\} = 1 - P\{|X-\mu|>20\}$$
Note because we subtracted $1$ from both sides to get into the form of equation \ref{chebineq} the inequality flips to a greater or equal to so
\begin{align*}
P\{|X-\mu|<20\} &= 1 - P\{|X-\mu|>20\}\\
&\ge 1 - \frac{\sigma_x^2}{20^2}\\
&\ge 1 - \frac{20}{20^2}\\
&\ge 1 - \frac{1}{20}\\
&\ge \frac{19}{20}\\
\end{align*}
So we can say that
$$P\{0<X<20\} \ge \frac{19}{20}$$

\item Problem 2
\begin{enumerate}
\item
Let $X$ be the test score of a student where $E[X] = 75$ and $Var(X) = 25$ to calculate $P\{X>85\}$ we can use the \emph{Markov inequality} given by
\begin{equation}
P\{X\ge a\} \le \frac{E[X]}{a}
\label{markov}
\end{equation}
So equation \ref{markov} yields
\begin{align*}
P\{X>85\} &\le \frac{E[X]}{85}\\
&\le \frac{75}{85}\\
&\le \frac{15}{17}
\end{align*}

\item
To find $P\{65<X<85\}$ we subtract the mean
\begin{align*}
P\{65<X<85\} &= P\{65-\mu<X-\mu<85-\mu\} \\ 
&= P\{65-75<X-75<85-75\} \\ 
&= P\{-10<X-75<10\} \\ 
&= P\{|X-75|<10\} 
\end{align*}
Now we can apply equation \ref{chebineq} to get
\begin{align*}
P\{|X-75|<10\} &= 1 - P\{|X-75|>10\}\\
& \ge1-\frac{Var(X)}{10^2}\\
& \ge1-\frac{25}{10^2}\\
& \ge1-\frac{25}{100}\\
& \ge\frac{75}{100}\\
& \ge\frac{3}{4}
\end{align*}

\item
Let 
$$\frac{\sum_{i=1}^nX_i}{n}$$
be the average score for $n$ students where 
$$P\left\{\left|\frac{\sum_{i=1}^nX_i}{n}-\mu\right|<5\right\} = 0.9$$
we can apply equation \ref{chebineq} but first we need to say that
$$P\left\{\left|\frac{\sum_{i=1}^nX_i}{n}-\mu\right|<5\right\} = P\left\{\left|\sum_{i=1}^nX_i-\mu\right|<n5\right\}$$
Now we can say that
\begin{align*}
P\left\{\left|\sum_{i=1}^nX_i-\mu\right|<n5\right\} &= 1- P\left\{\left|\sum_{i=1}^nX_i-\mu\right|>n5\right\}\\
&\ge 1- \frac{Var(X)}{n5^2}\\
&\ge 1- \frac{25}{n25}\\
0.9 &\ge 1- \frac{1}{n}\\
0.1 &\le \frac{1}{n}\\
10 &\le {n}\\
\end{align*}
\end{enumerate}

\item Problem 3\\
If we use the \emph{Central Limit Theorem} for part (c) of problem 2 first we say 
\begin{align*}
P\left\{\left|\frac{\sum_{i=1}^nX_i}{n}-\mu\right|<5\right\} &= P\left\{-n5<\sum_{i=1}^nX_i-n\mu<n5\right\}\\
&= P\left\{-\frac{n5}{\sigma\sqrt{n}}<Z_n<\frac{n5}{\sigma\sqrt{n}}\right\}\\
&= P\left\{-\frac{\sqrt{n}5}{\sigma}<Z_n<\frac{\sqrt{n}5}{\sigma}\right\}
\end{align*}
Where $Z_n$ is normally distributed. So we can say 
\begin{align*}
P\left\{-\frac{\sqrt{n}5}{\sigma}<Z_n<\frac{\sqrt{n}5}{\sigma}\right\} &= \Phi\left(\frac{\sqrt{n}5}{\sigma}\right)-\Phi\left(-\frac{\sqrt{n}5}{\sigma}\right)\\
&\approx \Phi\left(\frac{\sqrt{n}5}{\sigma}\right)-\left(1 - \Phi\left(\frac{\sqrt{n}5}{\sigma}\right)\right)\\
0.9 &\approx 2\Phi\left(\frac{\sqrt{n}5}{\sigma}\right)-1\\
.95 &\approx \Phi\left(\frac{\sqrt{n}5}{5}\right)\\
.95 &\approx \Phi\left({\sqrt{n}}\right)
\end{align*}
Now by using the table we get that
\begin{align*}
\sqrt{n} &= 1.65\\
n &= 2.72
\end{align*}
So we would need $n\ge2.72$

\item Problem 4
\begin{enumerate}
\item
Using equation \ref{markov} and assuming $E[X]=1$ we see
\begin{align*}
P\left\{\sum_1^{20}X_i>15\right\} &\le \frac{nE[X]}{15}\\
&\le \frac{20(1)}{15}\\
&\le \frac{4}{3}
\end{align*}
Note that this result is trivial as $P\{X\}\le1$ is true for all probabilities.

\item
Using the \emph{Central Limit Theorem} we can say
\begin{align*}
P\left\{\sum_1^{20}X_i>15\right\} &= P\left\{\frac{\sum_1^{20}X_i-n\mu}{\sqrt{n}\sigma}>\frac{15-n\mu}{\sqrt{n}\sigma}\right\}\\
&= P\left\{Z_n>\frac{15-20(1)}{\sqrt{20}(1)}\right\}\\
&= P\left\{Z_n>\frac{-5}{\sqrt{20}}\right\}\\
&\approx 1-\Phi\left(-\frac{5}{\sqrt{20}}\right)\\
&\approx 1-\left(1-\Phi\left(\frac{5}{\sqrt{20}}\right)\right)\\
&\approx \Phi\left(\frac{5}{\sqrt{20}}\right)\\
&\approx 0.8686
\end{align*}
\end{enumerate}

\item Problem 5\\
Let $X_i$ be an individual rounding error that is uniformily distributed over $(-0.5,0.5)$. To calculate the probability that the resultant sum differs from the exact sum by more than 3 we can calculate
$$P\left\{-3>\sum_{i=1}^{50}X_i>3\right\}$$
We know that 
$$E[X_i] = \frac{-0.5+0.5}{2} = 0$$
and
\begin{align*}
Var(X_i) &= \frac{(0.5--0.5)^2}{12}\\
Var(X_i) &= \frac{1}{12}
\end{align*}
So to apply the \emph{Central Limit Theorem} we say
\begin{align*}
P\left\{-3>\sum_{i=1}^{50}X_i>3\right\} &= P\left\{-\frac{3-n\mu}{\sqrt{n}\sigma}>\frac{\sum_{i=1}^{50}X_i-n\mu}{\sqrt{n}\sigma}>\frac{3-n\mu}{\sqrt{n}\sigma}\right\}\\
&= P\left\{-\frac{3-n\mu}{\sqrt{n}\sigma} > Z_n > \frac{3\sqrt{12}}{\sqrt{50}}\right\}\\
&\approx \Phi\left(-\frac{3\sqrt{12}}{\sqrt{50}}\right) + 1 - \Phi\left(\frac{3\sqrt{12}}{\sqrt{50}}\right)\\
&\approx 1 - \Phi\left(\frac{3\sqrt{12}}{\sqrt{50}}\right) + 1 - \Phi\left(\frac{3\sqrt{12}}{\sqrt{50}}\right)\\
&\approx 2\left(1 - \Phi\left(\frac{3\sqrt{12}}{\sqrt{50}}\right)\right)\\
&\approx 2\left(1 - 0.9292\right)\\
&\approx 0.142
\end{align*}

\item Problem 6\\
Let $X_i$ be the value of one roll of the die. We know that $E[X_i] = 3.5$ and
\begin{align*}
Var(X) &= E[X_i^2]-E[X_i]^2\\
&= \left(\sum_{n=1}^6\frac{1}{6}n^2\right)-3.5^2\\
&= \frac{91}{6}-3.5^2\\
&= 2.92
\end{align*}
So we are try to find the probability that the sum of 80 dice is enough to sum to 300 or
$$P\left\{\sum_{i=1}^{79}X_i\le300\right\}$$
So we apply the \emph{Central Limit Theorem} note we add $0.5$ to account for the descrete probablities.
\begin{align*}
P\left\{\sum_{i=1}^{79}X_i\le300\right\} &= P\left\{\frac{\sum_{i=1}^{79}X_i - n\mu}{\sigma\sqrt{n}}\le\frac{300-n\mu+0.5}{\sigma\sqrt{n}}\right\}\\
&= P\left\{Z_n\le\frac{300-79(3.5)+0.5}{\sqrt{2.92}\sqrt{79}}\right\}\\
&= P\left\{Z_n\le1.58\right\}\\
&\approx \Phi(1.58)\\
&\approx 0.9429
\end{align*}

\end{enumerate}
\end{document}
