\documentclass[11pt]{article}

\usepackage{latexsym}
\usepackage{amssymb}
\usepackage{enumerate}
\usepackage{amsthm}
\usepackage{amsmath}
\usepackage{cancel}
\numberwithin{equation}{section}

\setlength{\evensidemargin}{.25in}
\setlength{\oddsidemargin}{-.25in}
\setlength{\topmargin}{-.75in}
\setlength{\textwidth}{6.5in}
\setlength{\textheight}{9.5in}
\newcommand{\due}{October 28th, 2009}
\newcommand{\HWnum}{8}

\begin{document}
\begin{titlepage}
\setlength{\topmargin}{1.5in}
\begin{center}
\Huge{Physics 3310} \\
\LARGE{Principles of Electricity and Magnetism 1} \\
\Large{Professor Thomas R. Schibli} \\[1cm]

\huge{Homework \#\HWnum}\\[0.5cm]

\large{Joe Becker} \\
\large{SID: 810-07-1484} \\
\large{\due} 

\end{center}

\end{titlepage}



\section{Problem \#1}
\begin{enumerate}[(a)]
\item
So we want to minimize $r$ in the equation
$$\frac{\alpha}{r} = 1+\epsilon\cos\phi$$
we see we need to maximize $\cos\phi$. This is when $\cos\phi=1$ or $\phi=0$. This is the case when the particle is aligned with the center of the planet. Now if we solve for when $r$ is a minimum we see
$$r=\frac{\alpha}{1+\epsilon}$$
where
$$\alpha = \frac{2b^2 T_0'}{k}$$
$$\epsilon = \sqrt{1+\frac{4T_0'^2b^2}{k^2}}$$
This follows from the fact that $l=b\sqrt{2\mu T_0'}$ so we can say that $r$ is
$$r= \frac{2b^2 T_0'}{k}\frac{1}{1+\sqrt{1+\dfrac{4T_0'^2b^2}{k^2}}}$$
now we know that when $r$ is at a minimum it is on the surface of the planet or $r\le R$ so it follows that
$$R \le \frac{2b^2 T_0'}{k}\frac{1}{1+\sqrt{1+\dfrac{4T_0'^2b^2}{k^2}}}$$
Now if we solve for $b$ we find the maximum impact parameter that will have the particles fall into the planet. With a little rearranging we can get
\begin{align*}
\frac{2b^2 T_0'}{kR} &\le {1+\sqrt{1+\dfrac{4T_0'^2b^2}{k^2}}}\\
\frac{2b^2 T_0'}{kR} -1 &\le \sqrt{1+\dfrac{4T_0'^2b^2}{k^2}}\\
\left(\frac{2b^2 T_0'}{kR} -1\right)^2 &\le 1+\dfrac{4T_0'^2b^2}{k^2}\\
\left(\frac{2b^2 T_0'}{kR}\right)^2-2\frac{2b^2 T_0'}{kR}+\cancel{1} &\le \cancel{1}+\dfrac{4T_0'^2b^2}{k^2}\\
\frac{2b^2 T_0'}{kR}\left(\frac{2b^2 T_0'}{kR}-2\right) &\le \dfrac{4T_0'^2b^2}{k^2}\\
\frac{\cancel{2b^2 T_0'}}{\cancel{k}R}\left(\frac{2b^2 T_0'}{kR}-2\right) &\le \dfrac{\cancelto{2T_0'}{4T_0'^2b^2}}{\cancelto{k}{k^2}}\\
\frac{1}{R}\left(\frac{2b^2 T_0'}{kR}-1\right) &\le \dfrac{2T_0'}{k}\\
\frac{2b^2 T_0'}{kR}-2 &\le \dfrac{2RT_0'}{k}\\
\frac{2b^2 T_0'}{kR} &\le \dfrac{2RT_0'}{k}+2
\end{align*}
\begin{align*}
b^2 &\le \frac{kR}{2 T_0'}\left(\dfrac{2RT_0'}{k}+2\right)\\
b^2 &\le \frac{kR}{T_0'}\left(\dfrac{RT_0'}{k}+1\right)\\
b &\le \sqrt{\frac{kR}{T_0'}\left(\dfrac{RT_0'}{k}+1\right)}\\
b &\le \sqrt{R^2+\frac{kR}{T_0'}}
\end{align*}
So we now know the upper maximum distance $b$ from the center of the planet that the particle will still fall to the planet. So  the total cross section is the circle that $b$ makes or
$$\sigma = \pi b^2$$
$$\sigma = \pi \left(R^2+\frac{kR}{T_0'}\right)$$

\item
To begin this problem we will first state that instead of the planet moving through a stationary dust environment. We are rather looking at the dust moving at a constant velocity $v$ that is crashing into a stationary planet. So we know that for a given instant we have the area $\sigma$ or particles crashing into the so over a given time the total volume particles is given by
$$V=\sigma vdt$$
Now we find the total number of particles in this volume by multiplying by our particle density $\rho$ so we have
$$N = \rho\sigma vdt$$
So for a given amount of time we have
$$\frac{N}{dt}=\rho\sigma v$$
$$\frac{N}{dt}=\rho v\left(R^2+\frac{kR}{T_0'}\right)$$
\end{enumerate}

\section{Problem \#2}
\begin{enumerate}[(a)]
\item
To numerically solve 
$$\vec{a}_{eff} = -\vec{\omega}\times(\vec{\omega}\times\vec{r})-2\vec{\omega}\times\vec{v}_r$$
We split everything into the components of $\vec{r}$ where
$$\vec{r}=x\hat{x}+y\hat{y}$$
$$\vec{v}_r=\dot{x}\hat{x}+\dot{y}\hat{y}$$
and 
$$\vec{\omega} = \omega\hat{z}$$
So we can calculate the cross products as
$$\vec{\omega}\times\vec{r} = \det\left(\begin{array}{ccc}
			\hat{x}		&\hat{y}	&\hat{z}\\		
			0		&0		&\omega\\
			x		&y		&0\\
				\end{array}\right)$$
$$\vec{\omega}\times\vec{r} = \hat{x}(-\omega y)+\hat{y}(\omega x)$$
$$\vec{\omega}\times(\vec{\omega}\times\vec{r})= \det\left(\begin{array}{ccc}
			\hat{x}		&\hat{y}	&\hat{z}\\		
			0		&0		&\omega\\
			-\omega y	&\omega x 	0\\
				\end{array}\right)$$
$$\vec{\omega}\times(\vec{\omega}\times\vec{r})= \hat{x}(\omega^2 x)+\hat{y}(\omega^2y)$$
And 
$$\vec{\omega}\times\vec{v}_r = \det\left(\begin{array}{ccc}
			\hat{x}		&\hat{y}	&\hat{z}\\		
			0		&0		&\omega\\
			\dot{x}		&\dot{y}	&0\\
					\end{array}\right)$$
$$\vec{\omega}\times\vec{r} = \hat{x}(-\omega \dot{y})+\hat{y}(\omega \dot{x})$$
Now if we split everything into their individual components we get
$$\ddot{\vec{x}} = \omega^2x+2\omega\dot{y}\hat{x}$$
$$\ddot{\vec{y}} = \omega^2y-2\omega\dot{x}\hat{y}$$
See Mathematica sheet attached for the numerical solution and plots of this differential equations.
\item
To make the puck appear to not be moving in the fixed frame we want
$$\left.\frac{ds}{dt}\right|_f = 0$$
So we know that the relationship between the fixed and rotated frames is given by
$$\left.\frac{ds}{dt}\right|_f = \left.\frac{ds}{dt}\right|_r \vec{\omega}\times\vec{s}$$
Where $\vec{\omega}$ is the vector of the rotation given by $\vec{\omega} = \omega\hat{z}$ and $\vec{s}$ is the position vector of the puck given by $\vec{s}=0.5R\hat{s}$ due to initial conditions. So if we can solve
\begin{align*}
\left.\frac{ds}{dt}\right|_f = 0 &= \left.\frac{ds}{dt}\right|_r \vec{\omega}\times\vec{s}\\
0 &= \left.\frac{ds}{dt}\right|_r \omega\hat{z}\times 0.5R\hat{s}\\
0 &= \left.\frac{ds}{dt}\right|_r \omega0.5R\hat{z}\times\hat{s}\\
0 &= \left.\frac{ds}{dt}\right|_r \omega0.5R\\
\left.\frac{ds}{dt}\right|_r &= -\omega0.5R\\
\end{align*}
So the puck appears to move in a clockwise circle of radius $0.5R$ to an observer in the rotated frame.
\end{enumerate}

\section{Problem \#3}
\begin{enumerate}
\item Thornton and Marion 10.12 \\
First we can say that change in the center of mass denoted by $R$ in the rotated frame is given by
$$\left.\frac{dR}{dt}\right|_f = \left.\frac{dR}{dt}\right|_r \vec{\omega}\times\vec{R}$$
We know that the center of mass does not move outside of the rotation so we see that $\dot{\vec{R}}_r = 0$ this implies
$$\dot{R}_f = \vec{\omega}\times\vec{R}$$
So if we wish to find $$\ddot{R}_f$$ we can say
$$\ddot{R}_f = \vec{\omega}\times\dot{\vec{R}}_f+\dot{\vec{\omega}}\times\vec{R}$$
But $\omega$ is constant in time so that term becomes zero, and we already found $\dot{\vec{R}}_f =  \vec{\omega}\times\vec{R}$. So we can say
$$\ddot{R}_f = \vec{\omega}\times(\vec{\omega}\times\vec{R})$$
Now if we say that the effective force $F_{eff}$ is given by
$$F_{eff} = F - m\ddot{R}_f - m\dot{\vec{\omega}}\times\vec{r} - m\vec{\omega}\times(\vec{\omega}\times\vec{r}) - 2m\vec{\omega}\times v_r$$
First we can see that the $\dot{\vec{\omega}}$ term goes to zero. As does the term with the relative velocity as the plumb bob is stationary so we get
\begin{align*}
F_{eff} &= F - m\ddot{R}_f - m\vec{\omega}\times(\vec{\omega}\times\vec{r})\\
&= F - m \vec{\omega}\times(\vec{\omega}\times\vec{R})- m\vec{\omega}\times(\vec{\omega}\times\vec{r})\\
&= mg_0 - m\vec{\omega}\times(\vec{\omega}\times\vec{R+r})
\end{align*}
Because the only force that the bob feels is the force of gravity we write $F=mg_0$ where $g_0$ is the actual force of gravity to the center of the earth. So if we say the effective force that the plumb bob feels is $F_{eff} = mg$ where $g$ is the felt gravitation so we have
$$ g = g_0 - \vec{\omega}\times(\vec{\omega}\times\vec{R+r})$$
where we can use the definition of a cross product to say
$$\vec{\omega}\times(\vec{\omega}\times\vec{R+r})=\omega^2R\sin(\pi/2-\lambda)\hat{s}$$
where $\lambda$ is the angle of rise from the center of the earth. Using a trig identity we can say that the centrifugal acceleration is
$$a_{cf}=-\omega^2R\cos(\lambda)\hat{s}$$
So we get $g$ as
$$g=\vec{g_0}+\omega^2R\cos(\lambda)\hat{s}$$
So by looking at the drawing attached we can see the deflection angle $\epsilon$ is given by the law of sines as
\begin{align*}
\frac{a_{cf}}{\sin(\epsilon)} &= \frac{g_0}{\sin(\pi-\lambda-\epsilon)}\\
 {\sin(\pi-\lambda-\epsilon)}&= \frac{g_0}{a_{cf}}{\sin(\epsilon)}\\
\end{align*}
Using the identity
$$\sin(u-v) = \sin(u)\cos(v)-\cos(u)\sin(v)$$
this gives us
\begin{align*}
 {\sin(\pi-\lambda)\cos(\epsilon)}- {\cos(\pi-\lambda)\sin(\epsilon)}&= \frac{g_0}{a_{cf}}{\sin(\epsilon)}\\
 {\sin(\lambda)\cos(\epsilon)}- -\cos(\lambda)\sin(\epsilon)&= \frac{g_0}{a_{cf}}{\sin(\epsilon)}\\
 {\sin(\lambda)\cos(\epsilon)}+\cos(\lambda)\sin(\epsilon)&= \frac{g_0}{a_{cf}}{\sin(\epsilon)}\\
 \sin(\lambda)\cot(\epsilon)+\cos(\lambda)&= \frac{g_0}{a_{cf}}\\
 \sin(\lambda)\cot(\epsilon)&= \frac{g_0}{a_{cf}}-\cos(\lambda)\\
 \sin(\lambda)\cot(\epsilon)&= \frac{g_0-a_{cf}\cos(\lambda)}{a_{cf}}\\
 \cot(\epsilon)&= \frac{g_0-a_{cf}\cos(\lambda)}{a_{cf}\sin(\lambda)}\\
 \tan(\epsilon)&= \frac{a_{cf}\sin(\lambda)}{g_0-a_{cf}\cos(\lambda)}\\
 \tan(\epsilon)&= \frac{\omega^2R\cos(\lambda)\sin(\lambda)}{g_0-\omega^2R\cos(\lambda)\cos(\lambda)}\\
 \tan(\epsilon)&= \frac{\omega^2R\cos(\lambda)\sin(\lambda)}{g_0-\omega^2R\cos^2(\lambda)}
\end{align*}
Now because $\epsilon$ is a small angle we can say that $\tan(\epsilon) \approx \epsilon$ so
$$\epsilon = \frac{\omega^2R\cos(\lambda)\sin(\lambda)}{g_0-\omega^2R\cos^2(\lambda)}$$
We see that due to the $\cos(\lambda)\sin(\lambda)$ term we will have $\epsilon$ at a max when $\lambda = \pi/2$.

\end{enumerate}
\end{document}

