\documentclass[11pt]{article}

\usepackage{latexsym}
\usepackage{amssymb}
\usepackage{enumerate}
\usepackage{amsthm}
\usepackage{amsmath}
\usepackage{cancel}
\numberwithin{equation}{section}

\setlength{\evensidemargin}{.25in}
\setlength{\oddsidemargin}{-.25in}
\setlength{\topmargin}{-.75in}
\setlength{\textwidth}{6.5in}
\setlength{\textheight}{9.5in}
\newcommand{\due}{September 30th, 2009}
\newcommand{\HWnum}{5}

\begin{document}
\begin{titlepage}
\setlength{\topmargin}{1.5in}
\begin{center}
\Huge{Physics 3320} \\
\LARGE{Principles of Electricity and Magnetism II} \\
\Large{Professor Ana Maria Rey} \\[1cm]

\huge{Homework \#\HWnum}\\[0.5cm]

\large{Joe Becker} \\
\large{SID: 810-07-1484} \\
\large{\due} 

\end{center}

\end{titlepage}



\section{Problem \#1}
So we know that our total energy is 
$$E = T+U$$
where $$T = \frac{1}{2}m_e(\dot{r}^2 +r^2\dot{\theta}^2)$$ we can see that the kinetic energy is not effected by the mass of the sun. And we know that
$$U = \frac{Gm_em_s}{r}$$
We see that if the sun's mass drops by one half the new potential energy becomes
$$U'= \frac{1}{2}U$$
So our energy becomes 
\begin{equation}
E = T+\frac{1}{2}U
\label{energy}
\end{equation}
once the sun's mass drops by one half.
Now we need to find a comparison between the kinetic energy and the potential energy. We know that because the motion is orbital we can describe the motion as
$$\frac{\alpha}{r} = 1 +\epsilon\cos\theta$$
Where
$$\epsilon = \sqrt{1+\dfrac{2El^2}{\mu k^2}}$$
and
$$\alpha = \frac{l^2}{\mu k}$$
Where $k=Gm_em_s$.
We know that because we are in a circular orbit that $\epsilon = 0$, and we can see that $\epsilon$ is dependent on $E$ so we can solve for $E$ to find the total energy of the orbit
$$\epsilon = 0 = \sqrt{1+\dfrac{2El^2}{\mu k^2}}$$
$$ 0 = 1+\dfrac{2El^2}{\mu k^2}$$
$$ -\dfrac{2El^2}{\mu k^2} = 1$$
$$ E =-\dfrac{\mu k^2}{2l^2}$$
So we now know that the total energy. We can express the total energy as
\begin{equation}
E = T + U
\label{totEn}
\end{equation}
where $U=-\dfrac{k}{r}$ and we can find $r$ from the fact that the orbit is circular and $\epsilon$ is zero. Our orbital motion is
$$\frac{\alpha}{r} = 1$$
So $r=\alpha$ or
$$r= \frac{l^2}{\mu k}$$
So now we can find the $U$ as
$$U = -k\frac{\mu k}{l^2}$$
$$U = -\frac{\mu k^2}{l^2}$$
Now if we look at $E$ again we see that
$$ E =-\dfrac{\mu k^2}{2l^2}$$
$$ E =\frac{1}{2}\left(-\dfrac{\mu k^2}{l^2}\right)$$
$$ E =\frac{1}{2}U$$
So if we replace $E$ in equation \ref{totEn} we get
$$\frac{1}{2}U = T + U$$
Solving for $T$ gives us
$$T = -\frac{1}{2}U$$
So if use this in equation \ref{energy} we get 
$$E = -\frac{1}{2}U + \frac{1}{2}U$$
$$E=0$$
Therefore the orbit becomes parabolic once the mass of the sun drops by one half. This also means that earth will escape the solar system.


\section{Problem \#2}
So initially we can say that the energy is given by
$$E_1 = \frac{-k}{2R}$$ 
where $R$ is the radius that the satellite is at initially given by the problem as a distance equal to the radius of the earth so $R=2R_e$. So we can find its initial velocity from the energy or
$$E_1 = \frac{1}{2}mv_1^2-\frac{k}{R}$$
Set the two energies equal and solve for $v_1$
$$\frac{-k}{2R} = \frac{1}{2}mv_1^2-\frac{k}{R}$$
$$\frac{1}{2}mv_1^2 = \frac{-k}{2R}+\frac{k}{R}$$
$$\frac{1}{2}mv_1^2 = \frac{-k+2k}{2R}$$
$$ v_1^2 = \frac{k}{mR}$$
$$v_1 = \sqrt{\frac{k}{mR}}$$
Now we are going to move to a radius that is twice the height or $R'=3R_e$ so our energy is now
$$E_t = \frac{-k}{R+R'}$$
Again we need to find the velocity of this orbit by using energies
$$E_t = \frac{1}{2}mv_t^2 -\frac{k}{R}$$
Setting them equal we get
$$\frac{-k}{2R_e+3R_e} = \frac{1}{2}mv_t^2 -\frac{k}{R}$$
Now solving for $v_t$
$$ \frac{1}{2}mv_t^2 = \frac{-k}{R+R'} +\frac{k}{R}$$
$$ \frac{1}{2}mv_t^2 = \frac{-kR}{(R+R')R} +\frac{k(R+R')}{R(R+R')}$$
$$ \frac{1}{2}mv_t^2 = \frac{-kR + k(R+R')}{R(R+R')}$$
$$ \frac{1}{2}mv_t^2 = \frac{-kR + kR + kR')}{R(R+R')}$$
$$ \frac{1}{2}mv_t^2 = \frac{kR'}{R(R+R')}$$
$$ \frac{1}{2}mv_t^2 = \frac{k}{R}\frac{R'}{R+R'}$$
$$ v_t^2 = \frac{k}{mR}\frac{2R'}{R+R'}$$
$$ v_t = \sqrt{\frac{k}{mR}\frac{2R'}{R+R'}}$$
So now we can find $\Delta v$ as
$$\Delta v = v_1 - v_t$$
$$\Delta v = \sqrt{\frac{k}{mR}} - \sqrt{\frac{k}{mR}\frac{2R'}{R+R'}}$$
Factoring out a $\sqrt{\dfrac{k}{mR}}$
$$\Delta v = \sqrt{\frac{k}{mR}} \left(1 - \sqrt{\frac{2R'}{R+R'}}\right)$$

\section{Problem \#3}
\begin{enumerate}[(a)]
\item
We know that we have a potential 
$$U(r) = \alpha r^n$$
So we can calculate the Lagrangian 
$$L=T-U$$
Where we know $U$ and we know that the kinetic energy is
$$T = \frac{1}{2}\mu(\dot{r}^2 +r^2\dot{\theta}^2) $$
So the Lagrangian is
$$L = \frac{1}{2}\mu(\dot{r}^2 +r^2\dot{\theta}^2) - \alpha r^n$$
Now we can find the Hamiltonian by 
$$p_r = \frac{\partial L}{\partial \dot{r}}$$
$$\frac{\partial L}{\partial \dot{r}} = \mu\dot{r}$$
$$p_r = \mu\dot{r}$$
$$\dot{r}= \frac{p_r}{\mu}$$
$$p_{\theta} = \frac{\partial L}{\partial \dot{\theta}}$$
$$\frac{\partial L}{\partial \dot{\theta}} = \mu r^2\dot{\theta}$$
We know that this is constant because the angular momentum is conserved so 
$$l=\mu r^2 \dot{\theta}$$
$$\dot{\theta} = \frac{l}{\mu r^2}$$
So now we can calculated the Hamiltonian using 
$$H = p_r\dot{r} + p_{\theta}\dot{\theta} - L$$
$$H = \frac{p_r^2}{\mu} + \frac{l^2}{\mu r^2} -\frac{1}{2}\mu\left(\frac{p_r^2}{\mu^2} +r^2\frac{l^2}{\mu^2r^4}\right) + \alpha r^n$$
$$H = \frac{p_r^2}{\mu} + \frac{l^2}{\mu r^2} - \frac{p_r^2}{2\mu} +\frac{l^2}{\mu r^2} + \alpha r^n$$
$$H = \frac{p_r^2}{2\mu} + \frac{l^2}{2\mu r^2} + \alpha r^n$$
So we can see that our effective potential is 
$$U_{eff} = \frac{l^2}{2\mu r^2} + \alpha r^n$$
So if we are looking for a stable circular orbit we are looking for the minimum of the effective potential. We can calculate that by finding where the first derivative is zero. 
$$\frac{d U_{eff}}{dr} = \frac{l^2}{2\mu}(-2r^{-3}) + \alpha nr^{n-1}$$
$$\frac{d U_{eff}}{dr} = \frac{-l^2}{\mu r^3} + \alpha nr^{n-1}$$
$$0 = \frac{-l^2}{\mu r^3} + \alpha nr^{n-1}$$
$$\frac{l^2}{\mu r^3} = \alpha nr^{n-1}$$
$$\frac{l^2}{\mu\alpha n} = r^3r^{n-1}$$
$$\frac{l^2}{\mu\alpha n} = r^{n-1+3}$$
$$\frac{l^2}{\mu\alpha n} = r^{n+2}$$
$$r_0=\left(\frac{l^2}{\alpha\mu n}\right)^{1/n+2} $$
Where $r_0$ is the radius of a circular orbit.
Now lets check the dimensions, we assume $$<r_0> = m$$
So we know that 
$$<l> = kg\ m^2\ s^{-1};\ <\mu>=kg;\ <\alpha> = kg\ m^{2-n}\ s^{-2}$$
Note that $n$ has no units and $<\alpha>$ is calculated from 
$$<U_{eff}> = <\alpha><r^n>$$
$$kg\ m^2\ s^{-2}= <\alpha>m^n$$
$$<\alpha>=m^{-n}\ kg\ m^2\ s^{-2}$$
$$<\alpha>= kg\ m^{2-n}\ s^{-2}$$
So we can calculate the units
$$\left<\left(\frac{l^2}{\alpha\mu n}\right)^{1/n+2}\right> = \left(\frac{kg^2\ m^4\ s^{-2}}{kg\ kg\ m^{2-n}\ s^{-2}}\right)^{1/n+2}$$
$$\left<\left(\frac{l^2}{\alpha\mu n}\right)^{1/n+2}\right> = \left(\frac{m^4}{m^{2-n}}\right)^{1/n+2}$$
$$\left<\left(\frac{l^2}{\alpha\mu n}\right)^{1/n+2}\right> = \left(m^{4-2+n}\right)^{1/n+2}$$
$$\left<\left(\frac{l^2}{\alpha\mu n}\right)^{1/n+2}\right> = \left(m^{2+n}\right)^{1/n+2}$$
$$\left<\left(\frac{l^2}{\alpha\mu n}\right)^{1/n+2}\right> = m$$
Good our units are in agreement.

\item
To find the ratio between the angular frequency and the frequency of small oscillations we need to calculate them. First the angular frequency, assuming that we are in a circular orbit we can say
$$a = \frac{r_0^2\Omega^2}{r_0}$$
$$a = {r_0\Omega^2}$$
And from Newton we get
$$F= \mu a$$
so now we need to calculate $F$ from $U$ using
$$F = -\frac{dU}{dr}$$
$$F = -\alpha n r_0^{n-1}$$
So we can say that 
$$-\alpha n r_0^{n-1}= \mu{r_0\Omega^2}$$
Solving for $\Omega^2$ we get
$$-\alpha n r_0^{n-1}= \mu{r_0\Omega^2}$$
$$\Omega^2 = \frac{-\alpha n r_0^{n-1}}{\mu{r_0}}$$
$$\Omega^2 = \frac{-\alpha n r_0^{n-2}}{\mu}$$
Now to calculate the frequency of small oscillations we use
$$\omega^2 = \frac{1}{\mu}\left. \frac{d^2 U_{eff}}{dr^2}\right|_{r_0}$$
Finding 
$$\frac{d^2 U_{eff}}{dr^2} = \frac{d}{dr}\frac{-l^2}{\mu r^3} + \alpha nr^{n-1}$$
$$\frac{d^2 U_{eff}}{dr^2} = \frac{-l^2}{\mu r^4}(-3) + \alpha n(n-1)r^{n-2}$$
$$\frac{d^2 U_{eff}}{dr^2} = \frac{3l^2}{\mu r^4} + \alpha n(n-1)r^{n-2}$$
Evaluated at $r_0$ 
$$\frac{d^2 U_{eff}}{dr^2} = \frac{3l^2}{\mu r_0^4} + \alpha n(n-1)r_0^{n-2}$$
So we can say $\omega^2$ is equal to 
$$\omega^2 = \frac{1}{\mu}\left(\frac{3l^2}{\mu r_0^4} + \alpha n(n-1)r_0^{n-2}\right)$$
Now to find the ratio
$$\frac{\omega^2}{\Omega^2} = \frac{\mu}{-\alpha n r_0^{n-2}}\frac{1}{\mu}\left(\frac{3l^2}{\mu r_0^4} + \alpha n(n-1)r_0^{n-2}\right)$$
$$\frac{\omega^2}{\Omega^2} = \frac{1}{-\alpha n r_0^{n-2}}\left(\frac{3l^2}{\mu r_0^4} + \alpha n(n-1)r_0^{n-2}\right)$$
$$\frac{\omega^2}{\Omega^2} = \frac{1}{-\alpha n r_0^{n-2}}\frac{3l^2}{\mu r_0^4} + \frac{1}{-\alpha n r_0^{n-2}}\alpha n(n-1)r_0^{n-2}$$
$$\frac{\omega^2}{\Omega^2} = \frac{1}{-\alpha n r_0^{n-2}}\frac{3l^2}{\mu r_0^4} -(n-1)$$
$$\frac{\omega^2}{\Omega^2} = \frac{3l^2}{-\alpha\mu n}\frac{1}{r_0^4}\frac{1}{r_0^{n-2}} -(n-1)$$
$$\frac{\omega^2}{\Omega^2} = \frac{3l^2}{-\alpha\mu n}r_0^{-4}r_0^{-(n-2)} -(n-1)$$
$$\frac{\omega^2}{\Omega^2} = \frac{3l^2}{-\alpha\mu n}r_0^{-4}r_0^{-n+2} -(n-1)$$
$$\frac{\omega^2}{\Omega^2} = \frac{3l^2}{-\alpha\mu n}r_0^{-4-n+2} -(n-1)$$
$$\frac{\omega^2}{\Omega^2} = \frac{3l^2}{-\alpha\mu n}r_0^{-2-n} -(n-1)$$
$$\frac{\omega^2}{\Omega^2} = \frac{3l^2}{-\alpha\mu n}r_0^{-(n+2)} -(n-1)$$
Now we can use what we found $r_0$ to be in part (a) which is 
$$r_0=\left(\frac{l^2}{\alpha\mu n}\right)^{1/n+2}$$
So we get 
$$\frac{\omega^2}{\Omega^2} = \frac{3l^2}{-\alpha\mu n}\left(\left(\frac{l^2}{\alpha\mu n}\right)^{1/n+2}\right)^{-(n+2)} -(n-1)$$
$$\frac{\omega^2}{\Omega^2} = \frac{3l^2}{-\alpha\mu n}\left(\frac{l^2}{\alpha\mu n}\right)^{-(n+2)/n+2} -(n-1)$$
$$\frac{\omega^2}{\Omega^2} = \frac{3l^2}{-\alpha\mu n}\left(\frac{l^2}{\alpha\mu n}\right)^{-1} -(n-1)$$
$$\frac{\omega^2}{\Omega^2} = \frac{3l^2}{-\alpha\mu n}\frac{\alpha\mu n}{l^2} -(n-1)$$
$$\frac{\omega^2}{\Omega^2} = \frac{3}{-1} -(n-1)$$
$$\frac{\omega^2}{\Omega^2} = -3 -(n-1)$$
$$\frac{\omega^2}{\Omega^2} = -3 - n + 1$$
$$\frac{\omega^2}{\Omega^2} = -(2 + n)$$
Checking the units is pretty easy. We know the ratio of two frequencies is unitless and that $n$ is unitless so our units agree.


\end{enumerate}
\end{document}

