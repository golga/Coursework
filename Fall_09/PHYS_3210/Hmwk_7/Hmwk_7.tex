\documentclass[11pt]{article}

\usepackage{latexsym}
\usepackage{amssymb}
\usepackage{enumerate}
\usepackage{amsthm}
\usepackage{amsmath}
\usepackage{cancel}
\numberwithin{equation}{section}

\setlength{\evensidemargin}{.25in}
\setlength{\oddsidemargin}{-.25in}
\setlength{\topmargin}{-.75in}
\setlength{\textwidth}{6.5in}
\setlength{\textheight}{9.5in}
\newcommand{\due}{October 21st, 2009}
\newcommand{\HWnum}{7}

\begin{document}
\begin{titlepage}
\setlength{\topmargin}{1.5in}
\begin{center}
\Huge{Physics 3310} \\
\LARGE{Principles of Electricity and Magnetism 1} \\
\Large{Professor Thomas R. Schibli} \\[1cm]

\huge{Homework \#\HWnum}\\[0.5cm]

\large{Joe Becker} \\
\large{SID: 810-07-1484} \\
\large{\due} 

\end{center}

\end{titlepage}



\section{Problem \#1}
We know that 
$$T_0 = T_1+T_2$$
and the relation
$$\frac{T_1}{T_0} = 1 - \frac{2m_1m_2}{(m_1+m_2)^2}(1-\cos\theta)$$
So if we replace $T_1$ with $T_2$ using the equality
$$T_1 = T_0- T_2$$
we get
\begin{align*}
\frac{T_0-T_2}{T_0} &= 1 - \frac{2m_1m_2}{(m_1+m_2)^2}(1-\cos\theta)\\
\frac{T_0}{T_0}-\frac{T_2}{T_0} &= 1 - \frac{2m_1m_2}{(m_1+m_2)^2}(1-\cos\theta)\\
1-\frac{T_2}{T_0} &= 1 - \frac{2m_1m_2}{(m_1+m_2)^2}(1-\cos\theta)\\
\frac{T_2}{T_0} &= \frac{2m_1m_2}{(m_1+m_2)^2}(1-\cos\theta)\\
\end{align*}
Now if we solve for the $\cos$ we get
\begin{align*}
\frac{T_2}{T_0} &= \frac{2m_1m_2}{(m_1+m_2)^2}(1-\cos\theta)\\
\frac{T_2}{T_0}\frac{(m_1+m_2)^2}{2m_1m_2} &= 1-\cos\theta\\
\cos\theta  &= 1-\frac{T_2}{T_0}\frac{(m_1+m_2)^2}{2m_1m_2} \\
\cos\theta  &= \frac{T_0}{T_0}-\frac{T_2}{T_0}\frac{(m_1+m_2)^2}{2m_1m_2} \\
\cos\theta  &= \frac{T_0-T_2\dfrac{(m_1+m_2)^2}{2m_1m_2}}{T_0} \\
\end{align*}
If we take the derivative of this
\begin{align*}
d(\cos\theta) &= d\left(\frac{T_0-T_2\dfrac{(m_1+m_2)^2}{2m_1m_2}}{T_0}\right) \\
\sin\theta d\theta &= d\left(\frac{T_0-T_2\dfrac{(m_1+m_2)^2}{2m_1m_2}}{T_0}\right) \\
\sin\theta d\theta &= \frac{dT_2}{\dfrac{2m_1m_2}{(m_1+m_2)^2}T_0} \\
\end{align*}
Now we can see we have the equation
$$\sigma(\theta) = \frac{1}{I}\frac{dN}{d\Omega'}$$
where
$$d\Omega' = 2\pi\sin\theta d\theta$$
So we see that we have a relation therefore we can say that
$$\sigma(\theta) \varpropto \frac{dN}{dT_2}$$

\section{Problem \#2}
First we can say that we know 
$$\frac{\alpha}{r} = 1+\epsilon\cos(\phi)$$
we need to look at this at the extremes. We know when $r$ goes to infinity we see
$$0= 1+\epsilon\cos(\phi_{max})$$
solving for $\phi_{max}$ we get
$$\phi_{max} = \arccos\left(-\frac{1}{\epsilon}\right)$$
We also know that at $r_{min}$ we have $\phi_{min} =0$
So we can say
\begin{align*}
\Theta &= \phi_{max} - \phi_{min}\\
\Theta &= \arccos\left(-\frac{1}{\epsilon}\right)-0\\
\cos(\Theta) &= -\frac{1}{\epsilon}
\end{align*}
where $\epsilon$ is defined as
$$\epsilon = \sqrt{1+\frac{2El^2}{\mu k^2}}$$
Using the fact that when we are at $r_{max}$ we have no potential energy or are total energy is just kinetic given by $T_0'$. This and the identity
$$l= b\sqrt{2\mu T_0'}$$
We can rewrite $\epsilon$ as
\begin{align*}
\epsilon &= \sqrt{1+\frac{2T_0'(b\sqrt{2\mu T_0'})^2}{\mu k^2}}\\
\epsilon &= \sqrt{1+\frac{4T_0'^2b^2\mu}{\mu k^2}}\\
\epsilon &= \sqrt{1+\frac{4T_0'^2b^2}{k^2}}
\end{align*}
So now we can say that
\begin{align*}
\cos(\Theta) &= -\frac{1}{\epsilon}\\
\cos(\Theta) &= -\frac{1}{\sqrt{1+\dfrac{4T_0'^2b^2}{k^2}}}
\end{align*}
Now if we define 
$$\kappa\equiv \frac{k}{2T_0'}$$
we can see
\begin{align*}
\cos(\Theta) &= -\frac{1}{\sqrt{1+\dfrac{4T_0'^2b^2}{k^2}}}\\
&= -\frac{1}{\sqrt{1+b^2\dfrac{4T_0'^2}{k^2}}}\\
&= -\frac{1}{\sqrt{1+b^2\dfrac{1}{\kappa^2}}}\\
&= -\frac{1}{\sqrt{1+\dfrac{b^2}{\kappa^2}}}\\
&= -\frac{\kappa/b}{\dfrac{\kappa}{b}\sqrt{1+\dfrac{b^2}{\kappa^2}}}\\
&= -\frac{\kappa/b}{\sqrt{\left(\dfrac{\kappa}{b}\right)^2+\left(\dfrac{\kappa}{b}\right)^2\dfrac{b^2}{\kappa^2}}}\\
\cos(\Theta) &= -\frac{\kappa/b}{\sqrt{\left(\dfrac{\kappa}{b}\right)^2+1}}
\end{align*}
Now if we that the angle $\Theta$ describes a triangle with the one leg as $\kappa/b$ and the opposite leg as 1. This means we can say
\begin{align*}
\tan(\Theta) &= \frac{b}{\kappa}\\
\tan^2(\Theta) &= \frac{b^2}{\kappa^2}\\
{b^2} &= \tan^2(\Theta){\kappa^2}
\end{align*}
Now we know that 
$$\Theta = \frac{\pi}{2}+\frac{\theta}{2}$$
so we can say that
\begin{align*}
{b^2} &= \tan^2\left(\frac{\pi}{2}+\frac{\theta}{2}\right){\kappa^2}\\
{b^2} &= \cot^2\left(\frac{\theta}{2}\right){\kappa^2}\\
{b} &= {\kappa}\cot\left(\frac{\theta}{2}\right)
\end{align*}
Using the trig identities
$$\cos(\theta+\pi/2) = \sin(\theta)$$
$$\sin(\theta+\pi/2) = \cos(\theta)$$
We know that
$$\tan(\theta) = \frac{\sin(\theta)}{\cos(\theta)}$$
so we can see that
\begin{align*}
\tan(\theta+\pi/2) &= \frac{\sin(\theta+\pi/2)}{\cos(\theta+\pi/2)}\\
&= \frac{\cos(\theta)}{\sin(\theta)}\\
&= \cot(\theta)
\end{align*}
So now we need to calculate
\begin{align*}
\frac{db}{d\theta} &= \frac{d}{d\theta}{\kappa}\cot\left(\frac{\theta}{2}\right)\\
\frac{db}{d\theta} &= {\kappa}\frac{d}{d\theta}\cot(\theta/2)\\
\frac{db}{d\theta} &= -\frac{\kappa}{2}\frac{1}{\sin^2(\theta/2)}
\end{align*}
So we know that
$$\sigma(\theta) = \frac{b}{\sin(\theta)}\left|\frac{db}{d\theta}\right|$$
so we can say 
\begin{align*}
\sigma(\theta) &= \frac{b}{\sin(\theta)}\frac{\kappa}{2}\frac{1}{\sin^2(\theta/2)}\\
&= \frac{{\kappa}\cot(\theta/2)}{\sin(\theta)}\frac{\kappa}{2}\frac{1}{\sin^2(\theta/2)}\\
&= \frac{{\kappa^2}}{2}\frac{\cot(\theta/2)}{\sin(\theta)\sin^2(\theta/2)}
\end{align*}
Where
$$\sin(\theta) = 2\sin(\theta/2)\cos(\theta/2)$$
so $\sigma(\theta)$ becomes
\begin{align*}
\sigma(\theta) &= \frac{{\kappa^2}}{2}\frac{\cot(\theta/2)}{\sin(\theta)\sin^2(\theta/2)}\\
\sigma(\theta) &= \frac{{\kappa^2}}{2}\frac{\cot(\theta/2)}{2\sin(\theta/2)\cos(\theta/2)\sin^2(\theta/2)}\\
\sigma(\theta) &= \frac{{\kappa^2}}{4}\frac{\cot(\theta/2)}{\cos(\theta/2)\sin^3(\theta/2)}\\
\sigma(\theta) &= \frac{{\kappa^2}}{4}\frac{\cos(\theta/2)}{\cos(\theta/2)\sin^4(\theta/2)}\\
\sigma(\theta) &= \frac{{\kappa^2}}{4}\frac{1}{\sin^4(\theta/2)}\\
\end{align*}
Writing $\kappa$ out we get
$$\sigma(\theta) = \frac{{k^2}}{(4T_0')^2}\frac{1}{\sin^4(\theta/2)}$$

\section{Problem \#3}
So we know that the effective potential due to
$$U(r) = -\frac{\alpha}{r^2}$$
is given by
$$U_{eff} = \frac{l^2}{2\mu r^2} + \frac{\alpha}{r^2}$$
So we can find the equilibrium point as where the first derivative is zero
\begin{align*}
\frac{d U_{eff}}{dr} = 0 &= \frac{d}{dr}\frac{l^2}{2\mu r^2} + \frac{\alpha}{r^2}\\
0 &= \frac{l^2}{2\mu r^3}(-2) + \frac{\alpha}{r^3}(-2)\\
0 &= -\frac{l^2}{\mu r^3}  -\frac{2\alpha}{r^3}\\
0 &= -\frac{l^2}{\mu}\frac{1}{r^3}  -{2\alpha}\frac{1}{r^3}\\
0 &= -\frac{1}{r^3}\left(\frac{l^2}{\mu}  + {2\alpha}\right)\\
0 &= \frac{1}{r^3}\\
0 &= {r^3}\\
r_0 &=0
\end{align*}
Now for the potential
$$U(r) = -\frac{\alpha}{r^n}$$
so the effective potential is
$$U_{eff} = \frac{l^2}{2\mu r^2} - \frac{\alpha}{r^n}$$
So to find the equilibrium point we have
\begin{align*}
\frac{d U_{eff}}{dr} = 0 &= \frac{d}{dr}\frac{l^2}{2\mu r^2} - \frac{\alpha}{r^n}\\
0 &= \frac{l^2}{2\mu r^3}(-2) - \frac{\alpha}{r^{n+1}}(-n)\\
0 &= -\frac{l^2}{\mu r^3} + \frac{n\alpha}{r^{n+1}}\\
\frac{l^2}{\mu r^3} &= \frac{n\alpha}{r^{n+1}}\\
\frac{l^2}{\mu n\alpha} &= \frac{r^3}{r^{n+1}}\\
\frac{l^2}{\mu n\alpha} &= \frac{1}{r^{n-2}}\\
\frac{\mu n\alpha}{l^2} &= {r^{n-2}}\\
r_0&= \left(\frac{\mu n\alpha}{l^2}\right)^{1/(n-2)}
\end{align*}
So we know that the equilibrium point is the minimum required energy make the particle fall into the center so we can say
\begin{align*}
T_0' &\ge U_{eff}(r_0)\\
&\ge \frac{l^2}{2\mu r_0^2} - \frac{\alpha}{r_0^n}\\
&\ge \frac{1}{r_0^2}\left(\frac{l^2}{2\mu} - \frac{\alpha}{r_0^{n-2}}\right)\\
&\ge \frac{1}{r_0^2}\left(\frac{l^2}{2\mu} - \alpha\frac{1}{\dfrac{\mu n\alpha}{l^2}^{n-2/n-2}}\right)\\
&\ge \frac{1}{r_0^2}\left(\frac{l^2}{2\mu} - \alpha\frac{1}{\dfrac{\mu n\alpha}{l^2}}\right)\\
&\ge \frac{1}{r_0^2}\left(\frac{l^2}{2\mu} - \alpha \dfrac{l^2}{\mu n\alpha}\right)\\
&\ge \frac{1}{r_0^2}\left(\frac{l^2}{2\mu} - \dfrac{l^2}{\mu n}\right)\\
&\ge \frac{1}{r_0^2}\left(\frac{2b^2 T_0'\mu}{2\mu} - \dfrac{2b^2T_0'\mu}{\mu n}\right)\\
&\ge \frac{1}{r_0^2}\left(b^2 T_0' - \dfrac{2b^2T_0'}{n}\right)\\
&\ge \frac{T_0'b^2}{r_0^2}\left(1 - \dfrac{2}{n}\right)\\
\frac{T_0'}{T_0'} &\le \frac{b^2}{r_0^2}\left(1 - \dfrac{2}{n}\right)\\
1 &\le \frac{b^2}{r_0^2}\left(1 - \dfrac{2}{n}\right)
\end{align*}
\begin{align*}
1 &\le \frac{b^2}{r_0^2}\left(\frac{n}{n} -\dfrac{2}{n}\right)\\
1 &\le \frac{b^2}{r_0^2}\left(\frac{n-2}{n}\right)\\
\left(\frac{n}{n-2}\right) &\le {b^2}r_0^{-2}\\
\left(\frac{n}{n-2}\right) &\le {b^2}\left(\left(\frac{\mu n\alpha}{l^2}\right)^{1/(n-2)}\right)^{-2}\\
\left(\frac{n}{n-2}\right) &\le {b^2}\left(\frac{\mu n\alpha}{l^2}\right)^{-2/(n-2)}\\
\left(\frac{n}{n-2}\right) &\le {b^2}\left(\frac{\mu n\alpha}{2b^2T_0'\mu}\right)^{-2/(n-2)}\\
\left(\frac{n}{n-2}\right) &\le {b^2}\left(\frac{n\alpha}{2b^2T_0'}\right)^{-2/(n-2)}\\
\left(\frac{n}{n-2}\right) &\le {b^2}\left(b^{-2}\right)^{-2/(n-2)}\left(\frac{n\alpha}{2T_0'}\right)^{-2/(n-2)}\\
\left(\frac{n}{n-2}\right)\left(\frac{n\alpha}{2T_0'}\right)^{(n-2)/-2}&\le {b^2}\left(b^{-2}\right)^{-2/(n-2)}\\
\left(\frac{n}{n-2}\right)\left(\frac{n\alpha}{2T_0'}\right)^{(n-2)/-2}&\le {b^2}\left(b\right)^{(-2)-2/(n-2)}\\
\left(\frac{n}{n-2}\right)\left(\frac{n\alpha}{2T_0'}\right)^{(n-2)/-2}&\le b^{2+4/(n-2)}\\
\left(\frac{n}{n-2}\right)\left(\frac{n\alpha}{2T_0'}\right)^{(n-2)/-2}&\le b^{2(n-2)/(n-2)+4/(n-2)}\\
\left(\frac{n}{n-2}\right)\left(\frac{n\alpha}{2T_0'}\right)^{(n-2)/-2}&\le b^{(2(n-2)+4)/(n-2)}\\
\left(\frac{n}{n-2}\right)\left(\frac{n\alpha}{2T_0'}\right)^{(n-2)/-2}&\le b^{(2n-4+4)/(n-2)}\\
\left(\frac{n}{n-2}\right)\left(\frac{n\alpha}{2T_0'}\right)^{(n-2)/-2}&\le b^{2n/(n-2)}\\
\left(\left(\frac{n}{n-2}\right)\left(\frac{n\alpha}{2T_0'}\right)^{(n-2)/-2}\right)^{(n-2)/2n} &\le b\\
\end{align*}


\end{document}

