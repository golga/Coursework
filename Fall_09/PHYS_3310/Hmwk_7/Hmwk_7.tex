\documentclass[11pt]{article}

\usepackage{latexsym}
\usepackage{amssymb}
\usepackage{amsthm}
\usepackage{enumerate}
\usepackage{amsmath}
\usepackage{cancel}
\numberwithin{equation}{section}

\setlength{\evensidemargin}{.25in}
\setlength{\oddsidemargin}{-.25in}
\setlength{\topmargin}{-.75in}
\setlength{\textwidth}{6.5in}
\setlength{\textheight}{9.5in}
\newcommand{\due}{October 14th, 2009}
\newcommand{\HWnum}{7}
\newcommand{\grad}{\bold\nabla}
\newcommand{\vecE}{\vec{E}}
\newcommand{\scrptR}{\vec{\mathfrak{R}}}

\begin{document}
\begin{titlepage}
\setlength{\topmargin}{1.5in}
\begin{center}
\Huge{Physics 3320} \\
\LARGE{Principles of Electricity and Magnetism II} \\
\Large{Professor Ana Maria Rey} \\[1cm]

\huge{Homework \#\HWnum}\\[0.5cm]

\large{Joe Becker} \\
\large{SID: 810-07-1484} \\
\large{\due} 

\end{center}

\end{titlepage}



\section{Problem \#1}
\begin{enumerate}[(i)]
\item
To calculate the potential due to a sphere with constant surface charge density $\sigma_0$. We use separation of variables in spherical coordinates. We see that the sphere has azimuthal symmetry so there is no dependence on $\phi$ and we can guess that the solution looks like
$$V(r,\theta) = R(r)\Theta(\theta)$$
When we apply \emph{Laplace's Equation}
$$\grad^2V(r,\theta)=0$$
we get
\begin{align*}
\grad^2V(r,\theta) &= \frac{1}{r^2}\frac{\partial}{\partial r}\left(r^2\frac{\partial V(r,\theta)}{\partial r}\right) + \frac{1}{r^2\sin(\theta)}\frac{\partial}{\partial \theta}\left(\sin(\theta)\frac{\partial V(r,\theta)}{\partial \theta}\right) = 0\\ 
&=\frac{1}{r^2}\frac{\partial}{\partial r}\left(r^2\frac{\partial }{\partial r}R(r)\Theta(\theta)\right) + \frac{1}{r^2\sin(\theta)}\frac{\partial}{\partial \theta}\left(\sin(\theta)\frac{\partial}{\partial \theta}R(r)\Theta(\theta)\right)\\
&=\Theta(\theta)\frac{1}{r^2}\frac{\partial}{\partial r}\left(r^2\frac{\partial }{\partial r}R(r)\right) + R(r)\frac{1}{r^2\sin(\theta)}\frac{\partial}{\partial \theta}\left(\sin(\theta)\frac{\partial}{\partial \theta}\Theta(\theta)\right)
\end{align*}
Now if we divide by $V(r,\theta)$ we get
\begin{align*}
0&=\frac{\Theta(\theta)}{R(r)\Theta(\theta)}\frac{1}{r^2}\frac{\partial}{\partial r}\left(r^2\frac{\partial }{\partial r}R(r)\right) + \frac{R(r)}{R(r)\Theta(\theta)}\frac{1}{r^2\sin(\theta)}\frac{\partial}{\partial \theta}\left(\sin(\theta)\frac{\partial}{\partial \theta}\Theta(\theta)\right)\\
&=\frac{1}{R(r)}\frac{1}{r^2}\frac{\partial}{\partial r}\left(r^2\frac{\partial }{\partial r}R(r)\right) + \frac{1}{\Theta(\theta)}\frac{1}{r^2\sin(\theta)}\frac{\partial}{\partial \theta}\left(\sin(\theta)\frac{\partial}{\partial \theta}\Theta(\theta)\right)
\end{align*}
We can cancel the $\frac{1}{r^2}$ from both terms to yield
$$0=\frac{1}{R(r)}\frac{\partial}{\partial r}\left(r^2\frac{\partial }{\partial r}R(r)\right) + \frac{1}{\Theta(\theta)}\frac{1}{\sin(\theta)}\frac{\partial}{\partial \theta}\left(\sin(\theta)\frac{\partial}{\partial \theta}\Theta(\theta)\right)$$ 
We can see that we have 2 terms each is only dependent on $r$ or $\theta$ that sum to 0. So we can say that they have to be constant this gives us 
$$\frac{\partial}{\partial r}\left(r^2\frac{\partial }{\partial r}R(r)\right) = C_1 R(r)$$
$$\frac{\partial}{\partial \theta}\left(\sin(\theta)\frac{\partial}{\partial \theta}\Theta(\theta)\right) = C_2\sin(\theta)\Theta(\theta)$$ 
We assume that $C_1 = l(l+1)$ where $l\in\mathbb{Z}$. This also implies that $C_2 = -l(l+1)$ due to the fact that $C_1+C_2=0$ from the solution of \emph{Laplace's Equation}. So
$$\frac{\partial}{\partial r}\left(r^2\frac{\partial }{\partial r}R(r)\right) = l(l+1)R(r)$$
$$\frac{\partial}{\partial \theta}\left(\sin(\theta)\frac{\partial}{\partial \theta}\Theta(\theta)\right) =-l(l+1)\sin(\theta)\Theta(\theta)$$ 
Where the general solutions to these differential equation are
$$R(r) = Ar^l+\frac{B}{r^{l+1}}$$
$$\Theta(\theta) = P_l(\cos\theta)$$
Where $P_l$ are the \emph{Legendre polynomials} which are given by
$$P_l(x) = \frac{1}{2^ll!}\left(\frac{d}{dx}\right)^l(x^2-1)^l$$
So we can say that
\begin{align*}
&P_0(\cos\theta) = 1\\
&P_1(\cos\theta) = \cos\theta \\
&P_2(\cos\theta) =  \frac{3}{2}\cos^2\theta-\frac{1}{2}\\ 
\end{align*}
So we know that $V$ is given by
$$V_l(r,\theta) = \left(Ar^l+\frac{B}{r^{l+1}}\right)P_l(\cos\theta)$$
$$V(r,\theta) = \sum_{l=0}^{\infty}\left(A_lr^l+\frac{B_l}{r^{l+1}}\right)P_l(\cos\theta)$$
Now before we apply the boundary conditions we need to find what the potential on the surface is. We can do this using the identity 
$$\sigma = -\epsilon_0 \frac{\partial V}{\partial n}$$
In this case the normal direction is $r$ and $\sigma=\sigma_0$ so we get
$$\sigma_0 = -\epsilon_0 \frac{\partial V}{\partial r}$$
We can use separation of variables to get
$$\sigma_0 \partial r = -\epsilon_0\partial V$$
$$\int\sigma_0 \partial r = -\epsilon_0\int\partial V$$
$$\sigma_0 r = -\epsilon_0 V$$
$$V = -\frac{\sigma_0 r}{\epsilon_0}$$
This is only when we are on the surface of the sphere or $r=R$ so 
$$V(R,\theta) = -\frac{\sigma_0 R}{\epsilon_0}$$
So for outside the sphere we can say that
$$V(r\rightarrow\infty,\theta) = 0$$
This implies that all $A_l$ terms are zero so we have
$$V(r,\theta) = \sum_{l=0}^{\infty}\left(\frac{B_l}{r^{l+1}}\right)P_l(\cos\theta)$$
Now if we apply the condition on the surface we have
$$V(R,\theta) = -\frac{\sigma_0 R}{\epsilon_0} = \sum_{l=0}^{\infty}\left(\frac{B_l}{R^{l+1}}\right)P_l(\cos\theta)$$
We can see that we have no $\cos$ terms to match so we can assume that $l=0$ this yields
\begin{align*}
-\frac{\sigma_0 R}{\epsilon_0} &= \left(\frac{B_0}{R^{0+1}}\right)P_0(\cos\theta)\\
 &= \left(\frac{B_0}{R}\right)\cancelto{1}{P_0(\cos\theta)}\\
-\frac{\sigma_0 R}{\epsilon_0} &= \left(\frac{B_0}{R}\right)
\end{align*}
If we solve for $B_0$ we get
$$B_0 = -\frac{\sigma_0 R^2}{\epsilon_0}$$
So we can say the potential outside the sphere is
$$V(r\ge R,\theta) = -\frac{\sigma_0 R^2}{\epsilon_0 r}$$

For inside the sphere we can see that for
$$V(r,\theta) = \sum_{l=0}^{\infty}\left(A_lr^l+\frac{B_l}{r^{l+1}}\right)P_l(\cos\theta)$$
all the $B_l$ terms must be zero or else at $r=0$ the potential is undefined. So we have
$$V(r,\theta) = \sum_{l=0}^{\infty}\left(A_lr^l\right)P_l(\cos\theta)$$
Now we can see that at the surface of the sphere we have
$$V(R,\theta) = -\frac{\sigma_0 R}{\epsilon_0} = \sum_{l=0}^{\infty}\left(A_lR^l\right)P_l(\cos\theta)$$
Again we can see that there are no $\cos$ terms so we can say that $l=0$ this yields
\begin{align*}
-\frac{\sigma_0 R}{\epsilon_0} &= \left(A_0R^0\right)P_0(\cos\theta)\\
&= \left(A_0R^0\right)\cancelto{1}{P_0(\cos\theta)}\\
&= A_0\cancelto{1}{R^0}\\
-\frac{\sigma_0 R}{\epsilon_0} &= A_0\\
\end{align*}
So we found that the potential is constant inside of the sphere and is given by
$$V(r\le R,\theta) = -\frac{\sigma_0 R}{\epsilon_0}$$

So the potential over all space is
$$V(r,\theta) = \left\{\begin{array}{cc}
	-\dfrac{\sigma_0 R}{\epsilon_0} &\mbox{for } r\le R\\
	-\dfrac{\sigma_0 R^2}{\epsilon_0 r} &\mbox{for } r\ge R
	\end{array}\right.$$

\item
This looks like a charged conductor and it is what we would expect from a sphere of constant charge distribution in principle we did not need to solve \emph{Laplace's Equation} we could have just used \emph{Gauss' Law} to find the electric field then the potential from there because the charge distribution is constant.
\item
First we need to find the potential on the surface. We know that the potential has to be continuous so we can say
$$\sigma_0 = -\epsilon_0 \left(\left.\frac{\partial V_{in}}{\partial r} - \frac{\partial B_{out}}{\partial r}\right|_R\right)$$
Where
$$V_{in} = A_lr^lP_l(\cos\theta)$$
$$V_{out} = \frac{B_l}{r^{l+1}}P_l(\cos\theta)$$
So we can calculate 
\begin{align*}
\sigma_0 &= -\epsilon_0 \left(\left.\frac{\partial V_{in}}{\partial r} - \frac{\partial B_{out}}{\partial r}\right|_R\right)\\
&= \left(\left.\frac{\partial V_{in}}{\partial r} - \frac{\partial B_{out}}{\partial r}\right|_R\right)\\
&= \frac{\partial}{\partial r}A_lr^lP_l(\cos\theta) - \frac{\partial}{\partial r}\frac{B_l}{r^{l+1}}P_l(\cos\theta)\\
&= A_lr^{l-1}lP_l(\cos\theta) - -(l+1)\frac{B_l}{r^{l+1+1}}P_l(\cos\theta)\\
&= A_lR^{l-1}lP_l(\cos\theta) +(l+1)\frac{B_l}{R^{l+2}}P_l(\cos\theta)\\
\end{align*}
Now we need to use the fact that on the surface $V_{in} = V_{out}$ this give us
\begin{align*}
A_lR^lP_l(\cos\theta) &= \frac{B_l}{R^{l+1}}P_l(\cos\theta)\\
A_lR^l&= \frac{B_l}{R^{l+1}}\\
A_l&= \frac{B_l}{R^{l+1}R^l}\\
&= \frac{B_l}{R^{l+1+l}}\\
&= \frac{B_l}{R^{2l+1}}\\
\end{align*}
So now we can replace our $A_l$ in our term to get
\begin{align*}
A_lR^{l-1}lP_l(\cos\theta) +(l+1)\frac{B_l}{R^{l+1+1}}P_l(\cos\theta) &= \frac{B_l}{R^{2l+1}}R^{l-1}lP_l(\cos\theta) +(l+1)\frac{B_l}{R^{l+1+1}}P_l(\cos\theta) \\
&= \frac{B_l}{R^{2l+1}R^{-(l-1)}}lP_l(\cos\theta) +(l+1)\frac{B_l}{R^{l+2}}P_l(\cos\theta) \\
&= \frac{B_l}{R^{2l+1}R^{-(l-1)}}lP_l(\cos\theta) +(l+1)\frac{B_l}{R^{l+2}}P_l(\cos\theta) \\
&= \frac{B_l}{R^{2l+1-l+1}}lP_l(\cos\theta) +(l+1)\frac{B_l}{R^{l+2}}P_l(\cos\theta) \\
&= \frac{B_l}{R^{l+2}}lP_l(\cos\theta) +(l+1)\frac{B_l}{R^{l+2}}P_l(\cos\theta) \\
&= P_l(\cos\theta)\left(\frac{B_l}{R^{l+2}}l +(l+1)\frac{B_l}{R^{l+2}}\right) \\
&= P_l(\cos\theta)\left(\frac{B_l}{R^{l+2}}(l +(l+1))\right) \\
&= P_l(\cos\theta)\left(\frac{B_l}{R^{l+2}}(2l+1)\right) \\
&= \frac{B_l(2l+1)}{R^{l+2}}P_l(\cos\theta)\\
\end{align*}
Note that this is only for the top hemisphere where $\sigma_0$ is posive for the bottom hemisphere its the same but oppisite. Now we can find our $B_l$ from 
$$\sum_{l=0}^{\infty}\frac{B_l(2l+1)}{R^{l+2}}P_l(\cos\theta) = \left\{\begin{array}{cc}
	\dfrac{\sigma_0}{\epsilon_0} &\mbox{for } 0\le\theta<\pi/2\\
\\
	-\dfrac{\sigma_0}{\epsilon_0} &\mbox{for } \pi/2\le\theta<\pi\\
	\end{array}\right.$$
So now we can use Fourier's trick with Lagandre polynomials which is given by
$$\int_0^{\pi} P_l(\cos\theta)P_{l'}(\cos\theta)\sin\theta d\theta = \left\{\begin{array}{cc}
		0 &\mbox{for } l\ne l'\\
		\dfrac{2}{2l+1} &\mbox{for } l\ne l'\\
		\end{array}\right.$$
Now we can say
$$\sum_{l=0}^{\infty}\frac{B_l(2l+1)}{R^{l+2}}\int_0^{\pi} P_l(\cos\theta)P_{l'}(\cos\theta)\sin\theta d\theta = \int_0^{\pi} V_R P_{l'}(\cos\theta)\sin\theta d\theta$$
Where $V_R$ is definded as
$$V_R = \left\{\begin{array}{cc}
	\dfrac{\sigma_0}{\epsilon_0} &\mbox{for } 0\le\theta<\pi/2\\
\\
	-\dfrac{\sigma_0}{\epsilon_0} &\mbox{for } \pi/2\le\theta<\pi\\
	\end{array}\right.$$
So we see that using Fourier's trick we get
$$\frac{B_l(2l+1)}{R^{l+2}}\frac{2}{2l+1} = \int V_R P_{l}(\cos\theta)\sin\theta d\theta$$
$$\frac{2B_l}{R^{l+2}}= \int V_R P_{l}(\cos\theta)\sin\theta d\theta$$
Notice how we isolated the a sigle $l$ from the sum. (It is actually $l'$ but the index is arbitrary so we just change it back to $l$). We can now see that the integral on the left can be split due to the piecewise function $V_R$ so 
$$\frac{2B_l}{R^{l+2}}= \frac{\sigma_0}{\epsilon_0}\int_0^{\pi/2} P_{l}(\cos\theta)\sin\theta d\theta - \frac{\sigma_0}{\epsilon_0}\int_{\pi/2}^{\pi} P_{l}(\cos\theta)\sin\theta d\theta$$
$$B_l= \frac{R^{l+2}}{2}\frac{\sigma_0}{\epsilon_0}\int_0^{\pi/2} P_{l}(\cos\theta)\sin\theta d\theta - \frac{\sigma_0}{\epsilon_0}\int_{\pi/2}^{\pi} P_{l}(\cos\theta)\sin\theta d\theta$$
$$B_l= \frac{R^{l+2}}{2}\frac{\sigma_0}{\epsilon_0}\left(\int_0^{\pi/2} P_{l}(\cos\theta)\sin\theta d\theta - \int_{\pi/2}^{\pi} P_{l}(\cos\theta)\sin\theta d\theta\right)$$
So find the first two non-zero terms we start with $l=0$
\begin{align*}
B_0 &= \frac{R^{0+2}}{2}\frac{\sigma_0}{\epsilon_0}\int_0^{\pi/2} P_{0}(\cos\theta)\sin\theta d\theta - \frac{\sigma_0}{\epsilon_0}\int_{\pi/2}^{\pi} P_{0}(\cos\theta)\sin\theta d\theta\\
&= \frac{R^{2}}{2}\frac{\sigma_0}{\epsilon_0}\int_0^{\pi/2} (1)\sin\theta d\theta - \frac{\sigma_0}{\epsilon_0}\int_{\pi/2}^{\pi} (1)\sin\theta d\theta\\
&= \frac{R^{2}}{2}\frac{\sigma_0}{\epsilon_0}\int_0^{\pi/2} \sin\theta d\theta - \frac{\sigma_0}{\epsilon_0}\int_{\pi/2}^{\pi} \sin\theta d\theta\\
&= \frac{R^{2}}{2}\frac{\sigma_0}{\epsilon_0}\left(\cos\theta\right|_0^{\pi/2} - \frac{\sigma_0}{\epsilon_0}\left(\cos\theta \right|_{\pi/2}^{\pi}\\
&= \frac{R^{2}}{2}\frac{\sigma_0}{\epsilon_0}\left(\cos(\pi/2)-\cos(0)\right) - \frac{\sigma_0}{\epsilon_0}\left(\cos(\pi)-\cos(\pi/2)\right)\\
&= \frac{R^{2}}{2}\frac{\sigma_0}{\epsilon_0}\left(0-1\right) - \frac{\sigma_0}{\epsilon_0}\left(1-0\right)\\
&= \frac{R^{2}}{2}\frac{-\sigma_0}{\epsilon_0}- \frac{\sigma_0}{\epsilon_0}\\
&=0
\end{align*}
Now for $l=1$
\begin{align*}
B_1 &= \frac{R^{1+2}}{2}\frac{\sigma_0}{\epsilon_0}\left(\int_0^{\pi/2} P_{1}(\cos\theta)\sin\theta d\theta - \int_{\pi/2}^{\pi} P_{1}(\cos\theta)\sin\theta d\theta\right)\\
&= \frac{R^{3}}{2}\frac{\sigma_0}{\epsilon_0}\left(\int_0^{\pi/2}\cos\theta\sin\theta d\theta - \int_{\pi/2}^{\pi} \cos\theta\sin\theta d\theta\right)
\end{align*}
Using a $u$ subsitiution where
$$u=\cos\theta$$
$$du=-\sin\theta d\theta$$
\end{enumerate}
\begin{align*}
\frac{R^{3}}{2}\frac{\sigma_0}{\epsilon_0}\left(\int_0^{\pi/2}\cos\theta\sin\theta d\theta - \int_{\pi/2}^{\pi} \cos\theta\sin\theta d\theta\right) &= \frac{R^3}{2}\frac{\sigma_0}{\epsilon_0}\left(-\int_{u(0)}^{u(\pi/2)}udu +\int_{u(\pi/2)}^{u(\pi)} udu\right)\\
&= \frac{R^{3}}{2}\frac{\sigma_0}{\epsilon_0}\left(-\left(\frac{1}{2}u^2\right|_{u(0)}^{u(\pi/2)} +\left(\frac{1}{2}u^2\right|_{u(\pi/2)}^{u(\pi)}\right)\\
&= \frac{R^{3}}{2}\frac{\sigma_0}{\epsilon_0}\left(-\left(\frac{1}{2}\cos^2(\theta)\right|_{0}^{\pi/2} +\left(\frac{1}{2}\cos^2(\theta)\right|_{\pi/2}^{\pi}\right)\\
&= \frac{R^{3}}{2}\frac{\sigma_0}{\epsilon_0}\left(-\frac{1}{2}\left(\cos^2(\pi/2)- \cos^2(0)\right) +\frac{1}{2}\left(\cos^2(\pi)- \cos^2(\pi/2)\right)\right)\\
&= \frac{R^{3}}{2}\frac{\sigma_0}{\epsilon_0}\left(-\frac{1}{2}\left(\cos^2(\pi/2)- \cos^2(0)\right) +\frac{1}{2}\left(\cos^2(\pi)- \cos^2(\pi/2)\right)\right)\\
&= \frac{R^{3}}{2}\frac{\sigma_0}{\epsilon_0}\left(-\frac{1}{2}\left(0- 1^2\right) +\frac{1}{2}\left((-1)^2- 0\right)\right)\\
&= \frac{R^{3}}{2}\frac{\sigma_0}{\epsilon_0}\left(-\frac{1}{2}\left(-1\right) +\frac{1}{2}\left(1\right)\right)\\
&= \frac{R^{3}}{2}\frac{\sigma_0}{\epsilon_0}\left(\frac{1}{2} +\frac{1}{2}\right)\\
&= \frac{R^{3}}{2}\frac{\sigma_0}{\epsilon_0}(1)\\
B_1&= \frac{R^{3}}{2}\frac{\sigma_0}{\epsilon_0}\\
\end{align*}
Now for $l=2$ we have
\begin{align*}
B_2 &= \frac{R^{2+2}}{2}\frac{\sigma_0}{\epsilon_0}\left(\int_0^{\pi/2} P_{2}(\cos\theta)\sin\theta d\theta - \int_{\pi/2}^{\pi} P_{2}(\cos\theta)\sin\theta d\theta\right)\\
&= \frac{R^{4}}{2}\frac{\sigma_0}{\epsilon_0}\left(\int_0^{\pi/2}\left( \frac{3}{2}\cos^2(\theta)-\frac{1}{2}\right)\sin\theta d\theta - \int_{\pi/2}^{\pi}\left(\frac{3}{2}\cos^2(\theta)-\frac{1}{2}\right)\sin\theta d\theta\right)\\
\end{align*}
Again we do a $u$ substitution where
$$u = \cos(\theta)$$
$$du = -\sin(\theta)d\theta$$
\begin{align*}
&\frac{R^{4}}{2}\frac{\sigma_0}{\epsilon_0}\left(\int_0^{\pi/2}\left( \frac{3}{2}\cos^2(\theta)-\frac{1}{2}\right)\sin\theta d\theta - \int_{\pi/2}^{\pi}\left(\frac{3}{2}\cos^2(\theta)-\frac{1}{2}\right)\sin\theta d\theta\right)\\
&=\frac{R^{4}}{2}\frac{\sigma_0}{\epsilon_0}\left(\int_{u(0)}^{u(\pi/2)}\left(-\frac{3}{2}u^2-\frac{1}{2}\right)du + \int_{u(\pi/2)}^{u(\pi)}\left(\frac{3}{2}u^2-\frac{1}{2}\right)du\right)\\
&=\frac{R^{4}}{2}\frac{\sigma_0}{\epsilon_0}\left(-\frac{3}{6}u^3-\frac{1}{2}u\right|_{u(0)}^{u(\pi/2)} + \left(\frac{3}{6}u^3-\frac{1}{2}u\right|_{u(\pi/2)}^{u(\pi)} \\
\end{align*}
\begin{align*}
&=\frac{R^{4}}{2}\frac{\sigma_0}{\epsilon_0}\left(-\frac{1}{2}\cos^3(\theta)-\frac{1}{2}\cos(\theta)\right|_{0}^{\pi/2} + \left(\frac{3}{2}\cos^3(\theta)-\frac{1}{2}\cos(\theta)\right|_{\pi/2}^{\pi} \\
&=\frac{R^{4}}{2}\frac{\sigma_0}{\epsilon_0}\left(-(\frac{1}{2}\cos^3(0)-\frac{1}{2}\cos(0)-\frac{1}{2}\cos^3(\pi/2)-\frac{1}{2}\cos(\pi/2)\right) + \left(\frac{1}{2}\cos^3(\theta)-\frac{1}{2}\cos(\theta)\right|_{\pi/2}^{\pi} \\
&=\frac{R^{4}}{2}\frac{\sigma_0}{\epsilon_0}\left(-(\frac{1}{2}1^3-\frac{1}{2}1-\frac{1}{2}(0)-\frac{1}{2}(0)\right) + \left(\frac{1}{2}\cos^3(\pi)-\frac{1}{2}\cos(\pi)- \frac{1}{2}\cos^3(\pi/2)-\frac{1}{2}\cos(\pi/2)\right) \\
&=\frac{R^{4}}{2}\frac{\sigma_0}{\epsilon_0}\left(-(\frac{1}{2}-\frac{1}{2}\right) + \left(\frac{1}{2}(-1)^3-\frac{1}{2}(-1)- \frac{1}{2}0 - \frac{1}{2}0\right) \\
&=\frac{R^{4}}{2}\frac{\sigma_0}{\epsilon_0}\left(-(\frac{1}{2}-\frac{1}{2}\right) + \left(-\frac{1}{2}+\frac{1}{2}\right) \\
&=\frac{R^{4}}{2}\frac{\sigma_0}{\epsilon_0}\left(0\right) + \left(0\right) \\
&=\frac{R^{4}}{2}\frac{\sigma_0}{\epsilon_0}\left(0\right) + \left(0\right) \\
&=0
\end{align*}
Ok now our now we can do $l=3$
\begin{align*}
B_3 &= \frac{R^{3+2}}{2}\frac{\sigma_0}{\epsilon_0}\left(\int_0^{\pi/2} P_{3}(\cos\theta)\sin\theta d\theta - \int_{\pi/2}^{\pi} P_{3}(\cos\theta)\sin\theta d\theta\right)\\
 &= \frac{R^{5}}{2}\frac{\sigma_0}{\epsilon_0}\left(\int_0^{\pi/2} \left(\frac{5}{2}\cos^3(\theta)-\frac{1}{2}\cos(\theta)\right)\sin\theta d\theta - \int_{\pi/2}^{\pi} \left(\frac{5}{2}\cos^3(\theta)-\frac{1}{2}\cos(\theta)\right)\sin\theta d\theta\right)\\
 &= \frac{R^{5}}{2}\frac{\sigma_0}{\epsilon_0}\left(-\int_{u(0)}^{u(\pi/2)} \left(\frac{5}{2}u^3-\frac{1}{2}u\right)du + \int_{u(\pi/2)}^{u(\pi)} \left(\frac{5}{2}u^3-\frac{1}{2}u\right)du\right)\\
 &= \frac{R^{5}}{2}\frac{\sigma_0}{\epsilon_0}\left(-\left(\frac{5}{6}u^4-\frac{1}{4}u^2\right|_{u(0)}^{u(\pi/2)} + \left(\frac{5}{6}u^4-\frac{1}{4}u^2\right|_{u(\pi/2)}^{u(\pi)} \right)\\
 &= \frac{R^{5}}{2}\frac{\sigma_0}{\epsilon_0}\left(-\left(\frac{5}{6}u^4-\frac{1}{4}u^2\right|_{1}^{0} + \left(\frac{5}{6}u^4-\frac{1}{4}u^2\right|_{0}^{-1} \right)\\
 &= \frac{R^{5}}{2}\frac{\sigma_0}{\epsilon_0}\left(-\left(\frac{5}{6}0^4-\frac{1}{4}0^2-\frac{5}{6}1^4-\frac{1}{4}1^2\right) + \left(\frac{5}{6}(-1)^4-\frac{1}{4}(-1)^2-0\right) \right)\\
 &= \frac{R^{5}}{2}\frac{\sigma_0}{\epsilon_0}\left(\left(\frac{5}{6}-\frac{1}{4}\right) + \left(\frac{5}{6}-\frac{1}{4}\right)\right)\\
 &= \frac{R^{5}}{2}\frac{\sigma_0}{\epsilon_0}\left(\left(\frac{10}{12}-\frac{3}{12}\right) + \left(\frac{10}{12}-\frac{3}{12}\right)\right)\\
 &= \frac{R^{5}}{2}\frac{\sigma_0}{\epsilon_0}\left(\left(\frac{7}{12}\right) + \left(\frac{7}{12}\right)\right)\\
 &= \frac{R^{5}}{2}\frac{\sigma_0}{\epsilon_0}\left(\frac{7}{6}\right)\\
B_3 &= \frac{7R^{5}}{12}\frac{\sigma_0}{\epsilon_0}
\end{align*}
Yay! Now we have our first two non-zero terms so we can say our potential is
$$V(r,\theta) = \sum_{l=0}^{\infty}\frac{B_l}{r^{l+1}}P_l(\cos\theta)$$
$$V(r,\theta) = \frac{R^{3}}{2}\frac{\sigma_0}{\epsilon_0}\frac{1}{r^{2}}P_1(\cos\theta) + \frac{7R^{5}}{12}\frac{\sigma_0}{\epsilon_0}\frac{1}{r^{l+1}}P_3(\cos\theta)+...$$

\section{Problem \#2}
We can assume the general solution for \emph{Laplace's Equation} applies here. So we assume that
$$V(r,\theta) = \sum_{l=0}^{\infty}\left(A_lr^l+\frac{B_l}{r^{l+1}}\right)P_l(\cos\theta)$$
is a solution. So we can apply the boundary conditions
$$V(a,\theta) = V_{in}$$
$$V(b,\theta) = V_{out} \cos(\theta)$$
to find the potential between the two concentric spheres. We see that we have a constant term and a term that has a first order $\cos$ so we are going to say that we need $l=0$ and $l=1$ this gives us the general solution 
\begin{align*}
V(r,\theta) &= \left(A_0r^0+\frac{B_0}{r^{0+1}}\right)P_0(\cos\theta) + \left(A_1r^1+\frac{B_1}{r^{1+1}}\right)P_1(\cos\theta)\\
&= \left(A_0+\frac{B_0}{r}\right) + \left(A_1r+\frac{B_1}{r^{2}}\right)\cos(\theta)\\
\end{align*}

So to apply the boundary condition we get
$$V(a,\theta) = V_{in} = \left(A_0+\frac{B_0}{a}\right) + \left(A_1a+\frac{B_1}{a^{2}}\right)\cos(\theta)$$
We see that we do not have any $\cos$ terms so we know that 
$$A_1a+\frac{B_1}{a^2} = 0$$
and
$$A_0+\frac{B_0}{a} = V_{in}$$

Now for the boundary on the outside sphere we get
$$V(b,\theta) = V_{out} \cos(\theta) = \left(A_0+\frac{B_0}{b}\right) + \left(A_1b+\frac{B_1}{b^{2}}\right)\cos(\theta)$$
We see that we have a $\cos$ term to the first order. This implies that
$$A_0+\frac{B_0}{b}=0 $$
$$A_1b+\frac{B_1}{b^{2}}=V_{out}$$
So we now have 2 system of equations to solve for our constants. First lets solve the ones for $l=1$.
$$A_1a+\frac{B_1}{a^2} = 0$$
$$-A_1a=\frac{B_1}{a^2}$$
$$B_1=-A_1 a^3$$
Now we can substitute $B_1$ into our next equation to get
\begin{align*}
V_{out} &= A_1b-\frac{1}{b^{2}}A_1 a^3\\
&=A_1\left(b-\frac{a^3}{b^{2}}\right)\\
&=A_1\left(\frac{b^3-a^3}{b^{2}}\right)\\
&=A_1\left(\frac{b^3-a^3}{b^{2}}\right)\\
A_1&=V_{out}\frac{b^{2}}{b^3-a^3}
\end{align*}
Now we can solve for $B_1$
\begin{align*}
B_1&=-A_1 a^3\\
&=-V_{out}\frac{b^{2}}{b^3-a^3}a^3\\
&=-V_{out}\frac{b^{2}a^3}{b^3-a^3}
\end{align*}
Now for the system of equations for $l=0$
$$A_0+\frac{B_0}{b}=0 $$
$$A_0=-\frac{B_0}{b}$$
Now if we substitute into the other equation
\begin{align*}
V_{in}&=A_0+\frac{B_0}{a} \\
&=-\frac{B_0}{b}+\frac{B_0}{a} \\
&=B_0\left(-\frac{1}{b}+\frac{1}{a}\right) \\
&=B_0\left(\frac{1}{a}-\frac{1}{b}\right) \\
B_0&=V_{in}\left(\frac{1}{a}-\frac{1}{b}\right)^{-1} \\
\end{align*}
Now to find $A_0$
\begin{align*}
A_0 &=-\frac{B_0}{b}\\
&=-\frac{1}{b}V_{in}\left(\frac{1}{a}-\frac{1}{b}\right)^{-1} \\
&=-V_{in}b^{-1}\left(\frac{1}{a}-\frac{1}{b}\right)^{-1} \\
&=-V_{in}\left(\frac{b}{a}-\frac{b}{b}\right)^{-1} \\
&=-V_{in}\left(\frac{b}{a}-1\right)^{-1} \\
\end{align*}
Now that we have all the boundary conditions met we can write the full function of the potential
$$V(r,\theta) = \left(-V_{in}\left(\frac{b}{a}-1\right)^{-1}+V_{in}\left(\frac{1}{a}-\frac{1}{b}\right)^{-1}\frac{1}{r}\right) + \left(V_{out}\frac{b^{2}}{b^3-a^3}r-V_{out}\frac{b^{2}a^3}{b^3-a^3}\frac{1}{r^{2}}\right)\cos(\theta)$$
$$V(r,\theta) = \frac{V_{in}}{\dfrac{1}{a}-\dfrac{1}{b}}\left(-\frac{1}{b}+\frac{1}{r}\right) + V_{out}\frac{b^{2}}{b^3-a^3}\left(r-\frac{a^3}{r^{2}}\right)\cos(\theta)$$
$$V(r,\theta) = \frac{V_{in}}{\dfrac{b-a}{ab}}\left(\frac{1}{r}-\frac{1}{b}\right) + V_{out}\frac{b^{2}}{b^3-a^3}\left(r-\frac{a^3}{r^{2}}\right)\cos(\theta)$$
$$V(r,\theta) = \frac{V_{in}ab}{b-a}\left(\frac{1}{r}-\frac{1}{b}\right) + V_{out}\frac{b^{2}}{b^3-a^3}\left(r-\frac{a^3}{r^{2}}\right)\cos(\theta)$$
$$V(r,\theta) = \frac{V_{in}}{b-a}\left(\frac{ab}{r}-\frac{ab}{b}\right) + V_{out}\frac{b^{2}}{b^3-a^3}\left(r-\frac{a^3}{r^{2}}\right)\cos(\theta)$$
$$V(r,\theta) = \frac{V_{in}}{b-a}\left(\frac{ab}{r}-a\right) + V_{out}\frac{b^{2}}{b^3-a^3}\left(r-\frac{a^3}{r^{2}}\right)\cos(\theta)$$

\section{Problem \#3}
\begin{enumerate}[(i)]
\item
Again we assume that the general solution
$$V(r,\theta) = \sum_{l=0}^{\infty}\left(A_lr^l+\frac{B_l}{r^{l+1}}\right)P_l(\cos\theta)$$
is valid for this problem. So we can apply the boundary condition that is given in the problem.
$$V(r,0) = \frac{\sigma_0}{2\epsilon_0}\left(\sqrt{r^2+R^2}-r\right)$$
We rewrite this function
\begin{align*}
V(r,0) &= \frac{\sigma_0}{2}\left(\sqrt{r^2+R^2}-r\right)\\
&= \frac{\sigma_0}{2\epsilon_0}\left(\frac{r}{r}\sqrt{r^2+R^2}-r\right)\\
&= \frac{\sigma_0}{2\epsilon_0}\left(r\sqrt{\frac{1}{r^2}(r^2+R^2)}-r\right)\\
&= \frac{\sigma_0}{2\epsilon_0}\left(r\sqrt{\frac{r^2}{r^2}+\frac{R^2}{r^2}}-r\right)\\
&= \frac{\sigma_0}{2\epsilon_0}\left(r\sqrt{1+\frac{R^2}{r^2}}-r\right)\\
&= \frac{\sigma_0r}{2\epsilon_0}\left(\sqrt{1+\frac{R^2}{r^2}}-1\right)\\
\end{align*}
We can see that if we are very far away ($r>>R$) $\dfrac{R^2}{r^2}$ is small. So we can say that we have 
$$\epsilon \equiv \frac{R^2}{r^2}$$
this gives us
$$f(\epsilon)=\sqrt{1+\epsilon}$$
We can expand this using a \emph{Taylor Series} around zero or
$$f(\epsilon) = f(0) + \frac{f'(0)}{1!}\epsilon + \frac{f''(0)}{2!}\epsilon^2 + ...+ \frac{f^{(n)}(0)}{n!}\epsilon^n$$
So we can calculate the derivatives of $f(\epsilon)$ to get
\begin{align*}
f(0) &= \sqrt{1+0} = 1\\
f'(0) &= \frac{1}{2}(1+0)^{-1/2} = \frac{1}{2}\\
f''(0) &= -\frac{1}{4}(1+0)^{-3/2} = -\frac{1}{4}\\
f^{(3)}(0) &= \frac{3}{8}(1+0)^{-5/2} = \frac{3}{8}
\end{align*}
So now we can write the expansion as
$$f(\epsilon) = 1+\frac{1}{2}\epsilon-\frac{1}{4}\frac{\epsilon^2}{2!}+\frac{3}{8}\frac{\epsilon^3}{3!}+...$$
Now if we replace our $\epsilon$ we get
$$f(r) = 1+\frac{1}{2}\frac{R^2}{r^2}-\frac{1}{8}\left(\frac{R^2}{r^2}\right)^2+\frac{1}{16}\left(\frac{R^2}{r^2}\right)^3+...$$
Now if we replace this expansion into our boundary condition we find
\begin{align*}
V(r,0) &= \frac{\sigma_0r}{2\epsilon_0}\left(1+\frac{1}{2}\frac{R^2}{r^2}-\frac{1}{8}\left(\frac{R^2}{r^2}\right)^2+\frac{1}{16}\left(\frac{R^2}{r^2}\right)^3+...-1\right)\\
&= \frac{\sigma_0r}{2\epsilon_0}\left(\frac{1}{2}\frac{R^2}{r^2}-\frac{1}{8}\left(\frac{R^2}{r^2}\right)^2+\frac{1}{16}\left(\frac{R^2}{r^2}\right)^3+...\right)\\
&= \frac{\sigma_0r}{2\epsilon_0}\left(\frac{1}{2}\frac{R^2}{r^2}-\frac{1}{8}\frac{R^4}{r^4}+\frac{1}{16}\frac{R^6}{r^6}+...\right)\\
&= \frac{\sigma_0}{2\epsilon_0}\left(\frac{1}{2}\frac{R^2r}{r^2}-\frac{1}{8}\frac{R^4r}{r^4}+\frac{1}{16}\frac{R^6r}{r^6}+...\right)\\
&= \frac{\sigma_0}{2\epsilon_0}\left(\frac{1}{2}\frac{R^2}{r^1}-\frac{1}{8}\frac{R^4}{r^3}+\frac{1}{16}\frac{R^6}{r^5}+...\right)
\end{align*}
So now we can apply the boundary condition to get
$$V(r,0) = \frac{\sigma_0}{2\epsilon_0}\left(\frac{1}{2}\frac{R^2}{r^1}-\frac{1}{8}\frac{R^4}{r^3}+\frac{1}{16}\frac{R^6}{r^5}+...\right) = \sum_{l=0}^{\infty}\left(A_lr^l+\frac{B_l}{r^{l+1}}\right)P_l(\cos0)$$
We can see right off the bat that there are no $r$ terms with positive exponents therefore we know that $A_l$ is zero for all $l$ so we have
$$V(r,0) = \frac{\sigma_0}{2\epsilon_0}\left(\frac{1}{2}\frac{R^2}{r^1}-\frac{1}{8}\frac{R^4}{r^3}+\frac{1}{16}\frac{R^6}{r^5}+...\right) = \sum_{l=0}^{\infty}\left(\frac{B_l}{r^{l+1}}\right)P_l(1)$$
We know that $P_l(1) =1$ for all $l$ so that term drops off.
$$V(r,0) = \frac{\sigma_0}{2\epsilon_0}\left(\frac{1}{2}\frac{R^2}{r^1}-\frac{1}{8}\frac{R^4}{r^3}+\frac{1}{16}\frac{R^6}{r^5}+...\right) = \sum_{l=0}^{\infty}\left(\frac{B_l}{r^{l+1}}\right)$$
Now we have 2 infinite sums so we can see that for any odd $l$ we do not have a term in the expansion so we know for all odd $l$ $B_l=0$. For the first three non-zero terms we see that 
$$V(r,0) = \frac{\sigma_0}{2\epsilon_0}\left(\frac{1}{2}\frac{R^2}{r^1}-\frac{1}{8}\frac{R^4}{r^3}+\frac{1}{16}\frac{R^6}{r^5}\right) = \frac{B_0}{r^1}+\frac{B_2}{r^3}+\frac{B_4}{r^5}$$
So we can see that 
\begin{align*}
B_0 &= \frac{\sigma_0R^2}{4\epsilon_0}\\
B_2 &= -\frac{\sigma_0R^4}{16\epsilon_0}\\
B_4 &= \frac{\sigma_0R^6}{32\epsilon_0}\\
\end{align*}
So we know that our potential takes the form of
$$V(r,\theta) = \frac{\sigma_0R^2}{4\epsilon_0}\frac{1}{r}P_0(\cos\theta) -\frac{\sigma_0R^4}{16\epsilon_0}\frac{1}{r^3}P_2(\cos\theta) + \frac{\sigma_0R^6}{32\epsilon_0}\frac{1}{r^5}P_4(\cos\theta)$$

\item
The potential calculated in part (i) does not have the dipole term in the expansion. This makes sense because we have a net charge due to the $\sigma$ on the disk.

\end{enumerate}

\section{Problem \#4}
\begin{enumerate}[(i)]
\item
The total charge of the four point charges is 
$$3q+(-q)+(-q)+(-q)=0$$
so we have no net charge. Now we can find the 2 dipole moments of this system of charges. In the $x$ direction we have
\begin{align*}
\vec{P_x} &= [a(-q)+(-a)(-q)]\hat{x}\\
&= [-aq+aq]\hat{x}\\
&= 0\hat{x}
\end{align*}
So we have no dipole moment in the $x$ direction. For the $z$ direction we have
\begin{align*}
\vec{P_z} &= [b(3q)+(-b)(-q)]\hat{z}\\
&= [3bq+bq]\hat{z}\\
&= 4bq\hat{z}
\end{align*}
Now that we see that this system has a dipole moment we know that the potential is going to drop off by $1/r^2$, and because we know that their is no net charge the monopole term
$$V_{mon}(\vec{r}) = \frac{1}{4\pi\epsilon_0}\frac{Q}{r}$$
becomes zero ($Q=0$). So the leading term for the multipole expansion is the dipole term. This is given by
$$V_{dip}(\vec{r}) = \frac{1}{r\pi\epsilon_0}\frac{\vec{P}\cdot\hat{r}}{r^2}$$
So we know that $\vec{P}$ only has a $z$ component and that
\begin{align*}
\vec{P}\cdot\hat{r} &= P\cos(\theta)\\
 &= 4bq\cos(\theta)
\end{align*}
So we can say that our potential is 
$$V(r,\theta) = \frac{1}{4\pi\epsilon_0}\frac{4bq\cos(\theta)}{r^2}$$
$$V(r,\theta) = \frac{bq}{\pi\epsilon_0}\frac{\cos(\theta)}{r^2}$$
\item
We can find the electric field using
$$\vecE = -\grad V(r,\theta)$$
Where $\grad$ in spherical coordinates is given by
$$\grad = \frac{\partial}{\partial r}\hat{r}+\frac{1}{r}\frac{\partial}{\partial \theta}\hat{\theta}$$
\begin{align*}
\vecE &= -\grad V(r,\theta)\\
&=  -\left(\frac{\partial}{\partial r}\hat{r}+\frac{1}{r}\frac{\partial}{\partial \theta}\hat{\theta}\right) V(r,\theta)\\
&=  -\left(\frac{\partial}{\partial r}\hat{r}+\frac{1}{r}\frac{\partial}{\partial \theta}\right)\frac{bq}{\pi\epsilon_0\hat{\theta}}\frac{\cos(\theta)}{r^2}\\
&=  -\frac{bq}{\pi\epsilon_0}\left(\frac{\partial}{\partial r}\frac{\cos(\theta)}{r^2}\hat{r}+\frac{1}{r}\frac{\partial}{\partial \theta}\frac{\cos(\theta)}{r^2}\hat{\theta}\right)\\
&=  -\frac{bq}{\pi\epsilon_0}\left(\frac{-2\cos(\theta)}{r^3}\hat{r}+\frac{1}{r^3}\frac{\partial}{\partial \theta}{\cos(\theta)}\hat{\theta}\right)\\
&=  -\frac{bq}{\pi\epsilon_0}\left(\frac{2\cos(\theta)}{r^3}\hat{r}-\frac{\sin(\theta)}{r^3}\hat{\theta}\right)\\
&=  \frac{bq}{\pi\epsilon_0}\left(\frac{2\cos(\theta)}{r^3}\hat{r}+\frac{\sin(\theta)}{r^3}\hat{\theta}\right)\\
\end{align*}
See attached for the sketch of this electric field.
\end{enumerate}

\section{Problem \#5}
\begin{enumerate}[(i)]
\item
To find the dipole moment we can use
$$\vec{P} = \int_V\vec{r}\rho(\vec{r})d\tau$$
Where $\rho$ is the volume charge distribution described by 
$$\rho(\vec{r}) = \sigma_0\delta(r-R)\cos(\theta)$$
Now we see that we have to have 3 components to $\vec{P}$ so we need to describe each component of $\vec{r}$ in spherical coordinates given by
\begin{align*}
r_x&=r\sin(\theta)\cos(\phi)\hat{x}\\
r_y&=r\sin(\theta)\sin(\phi)\hat{y}\\
r_z&=r\cos(\theta)\hat{z}
\end{align*}
Now we can take the integrals. Not that there are three different integrals, one for each component of $\vec{P}$. Also we use the limits of integration in spherical coordinates or
$$d\tau = r^2\sin(\theta)drd\theta d\phi$$
So for the $x$ direction we get
\begin{align*}
\vec{P_x} &= \int_V\vec{r_x}\rho(\vec{r})d\tau\\
 &= \int_0^R\int_0^{\pi}\int_0^{2\pi}(r\sin(\theta)\cos(\phi)\hat{x})(\sigma_0\delta(r-R)\cos(\theta))r^2\sin(\theta)drd\theta d\phi\\
 &= \sigma_0\int_0^Rr^3\delta(r-R)dr\int_0^{2\pi}\cos(\phi) d\phi\int_0^{\pi}\sin(\theta)\cos(\theta)\sin(\theta)d\theta\hat{x}\\
 &= \sigma_0\int_0^Rr^3\delta(r-R)dr\int_0^{2\pi}\cos(\phi) d\phi\int_0^{\pi}\sin^2(\theta)\cos(\theta)d\theta\hat{x}\\
 &= \sigma_0R^3\left(-\sin(\phi)\right|_0^{2\pi}\int_0^{\pi}\sin^2(\theta)\cos(\theta)d\theta\hat{x}\\
 &= \sigma_0R^3\left(-\sin(2\pi)-(-\sin(0))\right)\int_0^{\pi}\sin^2(\theta)\cos(\theta)d\theta\hat{x}\\
 &= \sigma_0R^3\left(0-0\right)\int_0^{\pi}\sin^2(\theta)\cos(\theta)d\theta\hat{x}\\
 &= \sigma_0R^3\left(0\right)\int_0^{\pi}\sin^2(\theta)\cos(\theta)d\theta\hat{x}\\
\vec{P_x}&=0
\end{align*}

So there is no dipole in the $x$ direction. And for the $y$ direction 
\begin{align*}
\vec{P_y} &= \int_V\vec{r_y}\rho(\vec{r})d\tau\\
 &= \int_0^R\int_0^{\pi}\int_0^{2\pi}(r\sin(\theta)\sin(\phi)\hat{y})(\sigma_0\delta(r-R)\cos(\theta))r^2\sin(\theta)drd\theta d\phi\\
 &= \sigma_0\int_0^Rr^3\delta(r-R)dr\int_0^{2\pi}\sin(\phi) d\phi\int_0^{\pi}\sin(\theta)\cos(\theta)\sin(\theta)d\theta\hat{y}\\
 &= \sigma_0\int_0^Rr^3\delta(r-R)dr\int_0^{2\pi}\sin(\phi) d\phi\int_0^{\pi}\sin^2(\theta)\cos(\theta)d\theta\hat{y}\\
 &= \sigma_0R^3\left(\cos(\phi)\right|_0^{2\pi}\int_0^{\pi}\sin^2(\theta)\cos(\theta)d\theta\hat{y}\\
 &= \sigma_0R^3\left(\cos(2\pi)-\cos(0)\right)\int_0^{\pi}\sin^2(\theta)\cos(\theta)d\theta\hat{y}\\
 &= \sigma_0R^3\left(1-1\right)\int_0^{\pi}\sin^2(\theta)\cos(\theta)d\theta\hat{y}\\
 &= \sigma_0R^3\left(0\right)\int_0^{\pi}\sin^2(\theta)\cos(\theta)d\theta\hat{y}\\
\vec{P_y}&=0
\end{align*}
So there is no dipole in the $y$ direction either. Now for the $z$ direction we have
\begin{align*}
\vec{P_z} &= \int_V\vec{r_z}\rho(\vec{r})d\tau\\
 &= \int_0^R\int_0^{\pi}\int_0^{2\pi}(r\cos(\theta)\hat{z})(\sigma_0\delta(r-R)\cos(\theta))r^2\sin(\theta)drd\theta d\phi\\
 &= \sigma_0\int_0^Rr^3\delta(r-R)dr\int_0^{2\pi} d\phi\int_0^{\pi}\sin(\theta)\cos(\theta)\cos(\theta)d\theta\hat{z}\\
 &= \sigma_0\int_0^Rr^3\delta(r-R)dr\int_0^{2\pi} d\phi\int_0^{\pi}\sin(\theta)\cos^2(\theta)d\theta\hat{z}\\
 &= \sigma_0R^3\int_0^{2\pi} d\phi\int_0^{\pi}\sin(\theta)\cos^2(\theta)d\theta\hat{z}\\
 &= \sigma_0R^32\pi \int_0^{\pi}\sin(\theta)\cos^2(\theta)d\theta\hat{z}\\
 &= 2\sigma_0R^3\pi \int_0^{\pi}\sin(\theta)\cos^2(\theta)d\theta\hat{z}\\
\end{align*}
We need to use a $u$ substitution to solve the last integral we can say that 
$$u=\cos(\theta)$$
$$du=-\sin(\theta)d\theta$$

\begin{align*}
\int_0^{\pi}\sin(\theta)\cos^2(\theta)d\theta &= -\int_{u(0)}^{u(\pi)}u^2du\\
&= -\left(\frac{1}{3}u^3\right|_{u(0)}^{u(\pi)}\\
&= -\left(\frac{1}{3}\cos^3(\theta)\right|_{0}^{\pi}\\
&= -\frac{1}{3}(\cos^3(\pi)-\cos^3(0))\\
&= -\frac{1}{3}((-1)^3-1^3)\\
&= -\frac{1}{3}((-1)-1)\\
&= -\frac{1}{3}((-2)\\
\int_0^{\pi}\sin(\theta)\cos^2(\theta)d\theta &= \frac{2}{3}
\end{align*}
Now we can find $\vec{P_z}$
\begin{align*}
\vec{P_z}  &= 2\sigma_0R^3\pi \int_0^{\pi}\sin(\theta)\cos^2(\theta)d\theta\hat{z}\\
&= 2\sigma_0R^3\pi \frac{2}{3}\hat{z}\\
\vec{P_z}&= \frac{4\pi}{3}\sigma_0R^3\hat{z}
\end{align*}
Now because $\vec{P_x}$ and $\vec{P_y}$ are zero we can say that
$$\vec{P} = \frac{4\pi}{3}\sigma_0R^3\hat{z}$$
and we see that the dipole moment only points in the $z$ direction. To find the total charge $Q$ we can use
$$Q = \int_V\rho(\vec{r})d\tau$$
Where we integrate over all space so we calculate
\begin{align*}
Q &= \int_V\rho(\vec{r})d\tau\\
 &= \int_V\sigma_0\delta(r-R)\cos(\theta)d\tau\\
 &= \int_0^{\infty}\int_0^{\pi}\int_0^{2\pi}\sigma_0\delta(r-R)\cos(\theta)r^2\sin(\theta)drd\theta d\phi\\
 &= \sigma_0\int_0^{\infty}\delta(r-R)r^2dr\int_0^{2\pi}d\phi\int_0^{\pi}\cos(\theta)\sin(\theta)d\theta\\
 &= \sigma_0R^2\int_0^{2\pi}d\phi\int_0^{\pi}\cos(\theta)\sin(\theta)d\theta\\
 &= \sigma_0R^22\pi\int_0^{\pi}\cos(\theta)\sin(\theta)d\theta\\
 &= 2\pi\sigma_0R^2\int_0^{\pi}\cos(\theta)\sin(\theta)d\theta\\
\end{align*}
again we need to use a $u$ sub with 
$$u=\cos(\theta)$$
$$du=-\sin(\theta)d\theta$$
\begin{align*}
\int_0^{\pi}\cos(\theta)\sin(\theta)d\theta &= -\int_{u(0)}^{u(\pi)}udu \\
&= -\left(\frac{1}{2}u^2\right|_{u(0)}^{u(\pi)}\\
&= -\left(\frac{1}{2}\cos^2(\theta)\right|_{0}^{\pi}\\
&= -\frac{1}{2}\left(\cos^2(\pi)-\cos^2(0)\right)\\
&= -\frac{1}{2}\left((-1)^2-1^2\right)\\
&= -\frac{1}{2}\left(1-1\right)\\
&= -\frac{1}{2}\left(0\right)\\
\int_0^{\pi}\cos(\theta)\sin(\theta)d\theta &= 0\\
\end{align*}
Therefore the total charge $Q$ is zero

\item
We can use the multipole expansion to estimate the potential due to this sphere.
$$V(r,\theta)= \frac{1}{4\pi\epsilon_0}\left(\frac{Q}{r}+\frac{\vec{P}\cdot\hat{r}}{r^2}+...\right)$$
Where $Q$ is the total charge which we found to be zero so the first term approximation of the potential is
\begin{align*}
V(r,\theta) &= \frac{1}{4\pi\epsilon_0}\frac{\vec{P}\cdot\hat{r}}{r^2}\\
&= \frac{1}{4\pi\epsilon_0}\frac{\left(\dfrac{4\pi}{3}\sigma_0R^3\hat{z}\right)\cdot\hat{r}}{r^2}\\
&= \frac{1}{4\pi\epsilon_0}\frac{\left(\dfrac{4\pi}{3}\sigma_0R^3\right)\cos(\theta)}{r^2}\\
&= \frac{1}{4\pi\epsilon_0}\left(\dfrac{4\pi}{3}\sigma_0R^3\right)\frac{\cos(\theta)}{r^2}\\
&= \frac{1}{\epsilon_0}\left(\dfrac{1}{3}\sigma_0R^3\right)\frac{\cos(\theta)}{r^2}\\
V(r,\theta) &= \frac{\sigma_0R^3}{3\epsilon_0}\frac{\cos(\theta)}{r^2}
\end{align*}
We see that using \emph{Laplace's Equation} to find the potential we get (Griffiths' equation (3.87))
$$V(r,\theta) = \frac{kR^3}{3\epsilon_0}\frac{1}{r^2}\cos(\theta)$$
this is the same result that we found with the multipole expansion! Note that $\sigma_0$ and $k$ are just constants represent the same value. This means that the higher order term of our expansion are zero.
\end{enumerate}

\section{Problem \#6}
So we can say that from Griffiths equation (4.2)
$$\alpha = 4\pi\epsilon_0 a^3$$
So we need to solve for $a$ or the radius of the atom
$$a=\sqrt[3]{\frac{\alpha}{4\pi\epsilon_0}}$$
where we are given $\dfrac{\alpha}{4\pi\epsilon} = 0.667\times10^{-30}$ so
$$a=\sqrt[3]{0.667\times10^{-30}}$$
$$a=8.74\times10^{-11}\ m$$
Now we can use the equation
$$E=\frac{1}{4\pi\epsilon_0}\frac{qd}{a^3}$$
to find $d$ the separation distance
$$d= \frac{E4\pi\epsilon_0a^3}{q}$$
where $q$ is the charge of an electron $e$ and $E$ is given by
$$E= \frac{V}{x}$$
$$E= \frac{500}{.001}$$
$$E= 500000$$
So we can calculate
$$d= \frac{(500000)4\pi\epsilon_0(8.74\times10^{-11})^3}{e}$$
$$d=2.31\times10^{-16}\ m$$
This is a very small distance it is less than one millionth of the radius of the atom. Now if we wanted to ionizes the atom we would want $d=a$ in the equation
$$E=\frac{1}{4\pi\epsilon_0}\frac{ed}{a^3}$$
$$E=\frac{1}{4\pi\epsilon_0}\frac{ea}{a^3}$$
$$E=\frac{1}{4\pi\epsilon_0}\frac{e}{a^2}$$
$$E=\frac{1}{4\pi\epsilon_0}\frac{e}{(8.74\times10^{-11})^2}$$
$$E=1.89\times10^{11}$$
So now we can solve for the voltage needed using 
$$E=\frac{V}{x}$$
$$V=E{x}$$
$$V=1.89\times10^{11}(0.001)$$
$$V=1.89\times10^{8}\ V$$


\end{document}

