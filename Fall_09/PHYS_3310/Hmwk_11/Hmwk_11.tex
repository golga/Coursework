\documentclass[11pt]{article}

\usepackage{latexsym}
\usepackage{amssymb}
\usepackage{amsthm}
\usepackage{enumerate}
\usepackage{amsmath}
\usepackage{cancel}
\numberwithin{equation}{section}

\setlength{\evensidemargin}{.25in}
\setlength{\oddsidemargin}{-.25in}
\setlength{\topmargin}{-.75in}
\setlength{\textwidth}{6.5in}
\setlength{\textheight}{9.5in}
\newcommand{\due}{November 17th, 2009}
\newcommand{\HWnum}{11}
\newcommand{\grad}{\bold\nabla}
\newcommand{\vecE}{\vec{E}}
\newcommand{\vecscrptR}{\vec{\mathfrak{R}}}
\newcommand{\scrptR}{\mathfrak{R}}
\newcommand{\kapa}{\frac{1}{4\pi\epsilon_0}}

\begin{document}
\begin{titlepage}
\setlength{\topmargin}{1.5in}
\begin{center}
\Huge{Physics 3320} \\
\LARGE{Principles of Electricity and Magnetism II} \\
\Large{Professor Ana Maria Rey} \\[1cm]

\huge{Homework \#\HWnum}\\[0.5cm]

\large{Joe Becker} \\
\large{SID: 810-07-1484} \\
\large{\due} 

\end{center}

\end{titlepage}



\section{Problem \#1}
\begin{enumerate}[(a)]
\item
To find the magnetic field due to an infinite sheet of uniform surface current density $K_0\hat{x}$ we use 
\begin{equation}
\vec{B}(\vec{r}) = \frac{\mu_0}{4\pi}\int\frac{\vec{K}\times\hat{\scrptR}}{\scrptR^2}da'
\label{Biosav}
\end{equation}
Where $\vecscrptR$ is given by $\vecscrptR = \vec{r}-\vec{r'}$. Where $\vec{r}$ is the vector that points to the point of interest, and $\vec{r'}$ points to the source. These are given by
$$\vec{r} = z\hat{z}$$
and
$$\vec{r'} = x'\hat{x} + y'\hat{y}$$
note that we put the origin under the point where we are trying to find the B-field to make things simpler. So if follows that
$$\vecscrptR = -x'\hat{x} - y'\hat{y} + z\hat{z}$$
where
$$|\vecscrptR| = \scrptR = \sqrt{x'^2+y'^2+z^2}$$
So we can now solve equation \ref{Biosav}
$$\vec{B}(\vec{r}) = \frac{\mu_0}{4\pi}\int\frac{K_0\hat{x}\times\vecscrptR}{\scrptR^3}da'$$
Note that we changed the unit vector $\hat{\scrptR}$ to $\vecscrptR/\scrptR$. Now we can calculate the vector product as
\begin{align*}
{K_0\hat{x}\times\vecscrptR} &= \det\left(\begin{array}{ccc}
			\hat{x}		&\hat{y}	&\hat{z}	\\
			K_0		&0		&0		\\
			-x'		&-y'		&z
			\end{array}\right)\\
&= -K_0z\hat{y}+K_0y'\hat{z}
\end{align*}
So now we can split equation \ref{Biosav} into the 2 vector components (note that we are actually doing all 3 but the $\hat{x}$ component is zero).
$$\vec{B}(\vec{r}) = \frac{\mu_0}{4\pi}\int\frac{-K_0z}{(x'^2+y'^2+z^2)^{3/2}}\hat{y}da' + \frac{\mu_0}{4\pi}\int\frac{K_0y'}{(x'^2+y'^2+z^2)^{3/2}}\hat{z}da'$$
Now we see that our point is in the center of the sheet so we can see that the symmetry of the sheet will make all the contributions from the $y'$ cancel out. Therefore we do not have a magnetic field component in the $\hat{z}$ direction so our magnetic field is given by
$$\vec{B}(\vec{r}) = \frac{\mu_0}{4\pi}\int\frac{-K_0z}{(x'^2+y'^2+z^2)^{3/2}}\hat{y}da'$$
where $da'=dx'dy'$ and we integrate over all space so we have
\begin{align*}
\vec{B}(\vec{r}) &= \frac{\mu_0}{4\pi}\int\frac{-K_0z}{(x'^2+y'^2+z^2)^{3/2}}\hat{y}da'\\
&= \frac{\mu_0}{4\pi}\int_{-\infty}^{\infty}\int_{-\infty}^{\infty}\frac{-K_0z}{(x'^2+y'^2+z^2)^{3/2}}\hat{y}dx'dy'\\
&= \frac{-\mu_0K_0z}{4\pi}\int_{-\infty}^{\infty}\frac{2dy'}{(y'^2+z^2)^{3/2}}\hat{y}\\
&= \frac{-\mu_0K_0z}{4\pi}\frac{2\pi}{z}\hat{y}\\
&= \frac{-\mu_0K_0}{2}\hat{y}
\end{align*}
This is the field for above the sheet. Now we see that the only change when we go below the sheet is $z\rightarrow-z$ so we just change the sign. So we can say the magnetic field is
$$\vec{B} = \left\{\begin{array}{lc}
	\dfrac{-\mu_0K_0}{2}\hat{y}	&z>0\\
\\
	\dfrac{\mu_0K_0}{2}\hat{y}	&z<0
	\end{array}\right.$$

\item
We can use \emph{Amp\`{e}re's Law} 
\begin{equation}
\oint\vec{B}\cdot d\vec{l} = \mu_0 I_{enc}
\label{amplaw}
\end{equation}
to find the magnetic field for the system in part (a). We know that the magnetic field points in the $y$ direction so if we make and Amp\`{e}rean loop that is parallel to the $yz$ plane we can see that the B-field points along the top and bottom of the square loop and does not contribute along the sides ($x$ direction). So we see that the B-field is constant and parallel to our loop we can say
\begin{align*}
\oint\vec{B}\cdot d\vec{l} &= \mu_0 I_{enc}\\
Bl+Bl &= \mu_0 I_{enc}\\
2Bl &= \mu_0 I_{enc}
\end{align*}
Where $l$ is the length of the top and bottom of the loop. To find the current enclosed we can say that $I_{enc} = K_0l$. This is from the definition of $\vec{K}$. So equation \ref{amplaw} becomes 
\begin{align*}
2Bl &= \mu_0K_0l\\\
2B &= \mu_0K_0\\
B &= \frac{\mu_0K_0}{2}
\end{align*}
Good this is in agreement with part (a). We know the direction of our B-field from our assumptions with its direction along $d\vec{l}$. So it follows that
$$\vec{B} = \left\{\begin{array}{lc}
	\dfrac{-\mu_0K_0}{2}\hat{y}	&z>0\\
\\
	\dfrac{\mu_0K_0}{2}\hat{y}	&z<0
	\end{array}\right.$$

\item
So if we add another sheet at $z=a$ we can quickly see that the field is reversed as we found in part (a) and (b) but the regions are slightly different.
$$\vec{B}_2 = \left\{\begin{array}{lc}
	\dfrac{\mu_0K_0}{2}\hat{y}	&z>a\\
\\
	\dfrac{-\mu_0K_0}{2}\hat{y}	&z<a
	\end{array}\right.$$
Where the original sheet is given by
$$\vec{B}_1 = \left\{\begin{array}{lc}
	\dfrac{-\mu_0K_0}{2}\hat{y}	&z>0\\
\\
	\dfrac{\mu_0K_0}{2}\hat{y}	&z<0
	\end{array}\right.$$
So by superposition we can see that if we just add the fields we get
$$\vec{B}_1 + \vec{B}_2 = \vec{B} = \left\{\begin{array}{lc}
	0			&z>a\\
	{\mu_0K_0}\hat{y}	&0<z<a\\
	0			&z<0
	\end{array}\right.$$
So we see that between the sheets the magnetic field cancles and outside the sheets they doubled. This is like a parallel plate capacitor in electrostatics.

\end{enumerate}

\section{Problem \#2}
For the regions outside of the slab we can treat the system just like the sheet of charge. With an Amp\`{e}rian loop that (again) is parallel to the $yz$ plane where the top and the bottoms are outside of the slab. Here we can say that current enclosed is given by 
\begin{align*}
I_{enc} &= \int_{-h}^h\int_0^{l}Jdydz\\
&= \int_{-h}^h\int_0^{l}J_0|z|dydz\\
&= \int_{-h}^hJ_0|z|ldz\\
&= 2\int_{0}^hJ_0zldz\\
&= 2J_0l\frac{1}{2}z^2|_0^h\\
&= J_0lh^2\\
\end{align*}
So now we can apply equation \ref{amplaw} to get
\begin{align*}
\oint\vec{B}\cdot d\vec{l} &= \mu_0 I_{enc}\\
\oint\vec{B}\cdot d\vec{l} &= \mu_0 J_0lh^2\\
2Bl &= \mu_0 J_0lh^2\\
B &= \frac{\mu_0 J_0h^2}{2}
\end{align*}
Now for inside of the slab we assume that the magnetic field points in the same way but now our current enclosed is 
\begin{align*}
I_{enc} &= \int_{-z}^z\int_0^{l}Jdydz'\\
&= \int_{-z}^zlJ_0|z'|dz'\\
&= 2\int_{0}^zlJ_0z'dz'\\
&= lJ_0\left.z'^2\right|_0^z\\
&= lJ_0z^2
\end{align*}
Where $z$ is the half the hight of the loop we are integrating over. Note that the Amp\`{e}rian loop is centered at $z=0$. So equation \ref{amplaw} is
\begin{align*}
\oint\vec{B}\cdot d\vec{l} &= \mu_0 I_{enc}\\
2Bl &= \mu_0 lJ_0z^2\\
B &= \frac{\mu_0 J_0z^2}{2}
\end{align*}
So we can say that the total magnetic field over all space is given by
$$\vec{B} = \left\{\begin{array}{lc}
	\dfrac{-\mu_0 J_0h^2}{2}\hat{y}	&z>h\\
\\
	\dfrac{-\mu_0 J_0z^2}{2}\hat{y}	&0<z<h\\
\\
	\dfrac{\mu_0 J_0z^2}{2}	\hat{y}	&-h<z<0\\
\\
	\dfrac{\mu_0 J_0h^2}{2}	\hat{y}	&z<-h
		\end{array}\right.$$	


\section{Problem \#3}
\begin{enumerate}[(a)]
\item
The electric field due to a infinitely long wire with a linear charge density $\lambda$ is given by 
$$\vec{E} = \kapa\frac{2\lambda}{s}$$
where $s$ is the radial distance from the wire. So we can say that the force due to the electric field is given by
\begin{align*}
\vec{F}_E &= \vec{E}Q\\
&= \frac{1}{2\pi\epsilon_0}\frac{\lambda}{d}\lambda\\
&= \frac{1}{2\pi\epsilon_0}\frac{\lambda^2}{d}
\end{align*}
Now we know that the force due to the magnetic field from a infinitely long wire is given by
$$\vec{F}_M = \frac{\mu_0}{2\pi}\frac{I_1I_2}{d}$$
note that both of these forces are per unit length and not the total force due to the infinite wire as that would be an infinite force. We know that $I_1=I_2=\lambda v$ where $v$ is the velocity of the charges. So the magnetic force is
$$\vec{F}_M = \frac{\mu_0}{2\pi}\frac{\lambda^2v^2}{d}$$
So if we want to find the velocity $v$ that will make the forces equal we solve
\begin{align*}
\vec{F}_E &= \vec{F}_M\\
\frac{1}{2\pi\epsilon_0}\frac{\lambda^2}{d} &= \frac{\mu_0}{2\pi}\frac{\lambda^2v^2}{d}\\
\frac{1}{\epsilon_0} &= {\mu_0}v^2\\
v^2 &= \frac{1}{\epsilon_0\mu_0}\\
v &= \frac{1}{\sqrt{\epsilon_0\mu_0}}\\
v &= c
\end{align*}
so the charge would have to go the speed of light to have the forces equal.
\item
If we take the full version of Amp\`{e}re's law which is 
$$\grad\times\vec{B} = \mu_0\vec{J}+\frac{1}{c^2}\frac{\partial\vec{E}}{\partial t}$$
and take the divergence of both sides we get
$$\grad\cdot(\grad\times\vec{B}) = \grad\cdot\mu_0\vec{J}+\grad\cdot\frac{1}{c^2}\frac{\partial\vec{E}}{\partial t}$$
we know that the divergence of a curl is zero so it follows that
$$\grad\cdot(\grad\times\vec{B})= 0 = \mu_0\grad\cdot\vec{J}+\frac{1}{c^2}\grad\cdot\frac{\partial\vec{E}}{\partial t}$$
Now we pull the divergence into the partial derivative because the gradient operator has no time dependence so we get
$$0 = \mu_0\grad\cdot\vec{J}+\frac{1}{c^2}\frac{\partial}{\partial t}\grad\cdot\vec{E}$$
and we know that the divergence of $\vec{E}$ is given by
$$\grad\cdot\vec{E} = \frac{\rho}{\epsilon_0}$$
so if follows that
\begin{align*}
0 &= \mu_0\grad\cdot\vec{J}+\frac{1}{\epsilon_0c^2}\frac{\partial\rho}{\partial t}\\
\mu_0\grad\cdot\vec{J} &= -\frac{1}{\epsilon_0}\frac{1}{c^2}\frac{\partial\rho}{\partial t}\\
\grad\cdot\vec{J} &= -\frac{1}{\epsilon_0\mu_0}\frac{1}{c^2}\frac{\partial\rho}{\partial t}\\
\grad\cdot\vec{J} &= -c^2\frac{1}{c^2}\frac{\partial\rho}{\partial t}\\
\grad\cdot\vec{J} &= -\frac{\partial\rho}{\partial t}\\
\end{align*}
This is the continuity equation and shows that electric charge is conserved globally.
\end{enumerate}

\section{Problem \#4}
\begin{enumerate}[(a)]
\item
For the infinitely long wire of radius $R$ we assume that the vector potential of the system is given by 
$$\vec{A} = cI_0\frac{1-s^2}{R^2}\hat{z}$$
Note that we assume that $\vec{A}$ points in the $\hat{z}$ direction because we assume that $\vec{A}$ points along the same direction as the current flow. So if we wish to find the magnetic field from this vector potential we take the curl of $\vec{A}$ in cylindrical coordinates which is as follows
$$\grad\times\vec{A} = \left[\frac{1}{s}\frac{\partial A_z}{\partial\phi}-\frac{\partial A_{\phi}}{\partial z}\right]\hat{s}+\left[\frac{\partial A_s}{\partial z}-\frac{\partial A_z}{\partial s}\right]\hat{\phi}+\frac{1}{s}\left[\frac{\partial}{\partial s}(sA_{\phi})-\frac{\partial A_s}{\partial\phi}\right]\hat{z}$$
We see that $\vec{A}$ only has a $\hat{z}$ component and that component is only dependent on $s$ so we see that all the terms in the cross product drop off. So we are left with
$$\grad\times\vec{A} = -\frac{\partial A_z}{\partial s}\hat{\phi}$$
We can calculate this partial derivative as
\begin{align*}
\frac{\partial A_z}{\partial s} &= \frac{\partial}{\partial s}cI_0\frac{1-s^2}{R^2}\\
&= \frac{cI_0}{R^2}\frac{\partial}{\partial s}1-s^2\\
&= \frac{-2cI_0s}{R^2}
\end{align*}
So we can see that the magnetic field is given by 
$$\vec{B} = \frac{2cI_0s}{R^2}\hat{\phi}$$
Now if we use equation \ref{amplaw} to find the magnetic field. First we know that the B-field points in the $\hat{\phi}$ direction so if we pick our Amp\`{e}rian loop to be a circle with radius $s\le R$ centered at the origin. Where the current enclosed is given by 
\begin{align*}
I_{enc} &= \frac{I_0}{\pi R^2}\pi s^2\\
I_{enc} &= \frac{I_0s^2}{R^2}
\end{align*}
So we can calculate the magnetic field as
\begin{align*}
\oint\vec{B}\cdot\vec{dl} &= \mu_0I_{enc}\\
\oint\vec{B}\cdot\vec{dl} &= \mu_0\frac{I_0s^2}{R^2}\\
B(2\pi s) &= \mu_0\frac{I_0s^2}{R^2}\\
B &= \mu_0\frac{I_0s^2}{2\pi sR^2}\\
\vec{B} &= \mu_0\frac{I_0s}{2\pi R^2}\hat{\phi}\\
\end{align*}
So now if we want to find $c$ we can say
\begin{align*}
\mu_0\frac{I_0s}{2\pi R^2} &= \frac{2cI_0s}{R^2}\\
\mu_0\frac{1}{2\pi} &= {2c}\\
c &= \frac{\mu_0}{4\pi}
\end{align*}
Where the units of $c$ are the units of $\mu_0$ or 
$$<c> = N\ A^{-2}$$
Now we should note that $\vec{A}$ has an ambiguity due to the fact that we could shift the vector with another vector that does not have any divergence.

\item
Finding the vector potential outside of the cylinder is a tad bit difficult because our current goes to infinity. So we do not have a localized current distribution we have to use the equation
$$\oint\vec{A}\cdot d\vec{l} = \int \vec{B}\cdot d\vec{a}$$
So before we can find the flux of the magnetic field we first need to find the magnetic field. We can do this by using equation \ref{amplaw}. If we pick an Amp\`{e}rian loop that has a radius $s>R$ we can see that $I_{enc} = I_0$ so if follows that
\begin{align*}
\oint\vec{B}\cdot d\vec{l} &= \mu_0I_0\\
B\oint dl &= \mu_0I_0\\
B(2\pi s) &= \mu_0I_0\\
B &= \frac{\mu_0I_0}{2\pi s}
\end{align*}
So now if we pick a loop (not the same loop that we just used) that is represented in the attached sketch. We see that the magnetic flux is given by
\begin{align*}
\int\vec{B}\cdot d\vec{a} &= \int_R^s\int_0^l\frac{\mu_0I_0}{2\pi s'}ds'dz\\
&= \int_R^s\int_0^l\frac{\mu_0I_0}{2\pi s'}ds'dz\\
&= \int_R^s\frac{\mu_0I_0l}{2\pi s'}ds'\\
&= \frac{\mu_0I_0l}{2\pi} \int_R^s\frac{1}{s'}ds'\\
&= \frac{\mu_0I_0l}{2\pi} \left.\ln(s')\right|_R^s\\
&= \frac{\mu_0I_0l}{2\pi} \left(\ln(s)-\ln(R)\right)\\
&= \frac{\mu_0I_0l}{2\pi} \ln\left(\frac{s}{R}\right)
\end{align*}
So now we know the flux of the B-field through the loop. Now we just need to find the line integral so we assume that $\vec{A}$ only points along the $z$ direction so only the length contributes to the integral so
$$\oint\vec{A}\dot d\vec{l} = A(2l)$$
So now we can find the vector potential as
\begin{align*}
\oint\vec{A}\cdot d\vec{l} = \int \vec{B}\cdot d\vec{a}\\
A(2l) &= \frac{\mu_0I_0l}{2\pi} \ln\left(\frac{s}{R}\right)\\
A &= \frac{\mu_0I_0l}{4l\pi} \ln\left(\frac{s}{R}\right)\\
&= \frac{\mu_0I_0}{4\pi} \ln\left(\frac{s}{R}\right)\\
\end{align*}
So we can say that
$$\vec{A} =  \frac{\mu_0I_0}{4\pi} \ln\left(\frac{s}{R}\right)\hat{z}$$
Now if we check to see that $\grad\cdot\vec{A} = 0$ we get
\begin{align*}
\grad\cdot\vec{A} &= \frac{\partial}{\partial z}\frac{\mu_0I_0}{4\pi} \ln\left(\frac{s}{R}\right)\\
&=0
\end{align*}
Note this is because $\vec{A}$ has no $z$ dependence and only has a $z$ component. We also know that this vector potential like in part (a) could be shifted by a vector with zero divergence.
\end{enumerate}

\section{Problem \#5}
\begin{enumerate}[(a)]
\item
To find a relation between $\vec{A}$ and $\vec{B}$ we look at the relations between these two values and $\vec{J}$ namely 
$$\vec{A} = \frac{\mu_0}{4\pi}\int\frac{\vec{J}}{\scrptR}d\tau$$
and 
$$\grad\times\vec{B} = \mu_0\vec{J}$$
we can see that there is a $\mu_0\vec{J}$ term in both equation so it is easy to relate them by saying
$$\vec{A} = \frac{1}{4\pi}\int\frac{\grad\times\vec{B}}{\scrptR}d\tau$$

\item
From part (a) we found that the relation between $\vec{A}$ and $\vec{B}$ is given by
$$\vec{A} = \frac{1}{4\pi}\int\frac{\grad\times\vec{B}}{\scrptR}d\tau$$
no for the situation described in the problem we can immediately see that this is very similar to a solenoid with B-field acting like the surface current. We found that the B-Field due to some $\vec{J}$ in this system is 
$$\vec{B} = \left\{\begin{array}{lc}
		\mu_0nI\hat{z}	&s<R\\
		0		&s>R
		\end{array}\right.$$
In this case $\vec{J} = nI\delta(s-R)\hat{\phi}$ this is analogous to the B-field we have in this problem which is $\vec{B} = C\delta(s-R)\hat{\phi}$ so we can quickly say that
$$\vec{A} = \left\{\begin{array}{lc}
		C\hat{z}	&s<R\\
		0		&s>R
		\end{array}\right.$$

\end{enumerate}

\end{document}

