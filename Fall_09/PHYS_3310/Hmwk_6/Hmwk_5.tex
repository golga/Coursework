\documentclass[11pt]{article}

\usepackage{latexsym}
\usepackage{amssymb}
\usepackage{amsthm}
\usepackage{enumerate}
\usepackage{amsmath}
\usepackage{cancel}
\numberwithin{equation}{section}

\setlength{\evensidemargin}{.25in}
\setlength{\oddsidemargin}{-.25in}
\setlength{\topmargin}{-.75in}
\setlength{\textwidth}{6.5in}
\setlength{\textheight}{9.5in}
\newcommand{\due}{October 7th, 2009}
\newcommand{\HWnum}{6}
\newcommand{\grad}{\bold\nabla}
\newcommand{\vecE}{\vec{E}}
\newcommand{\scrptR}{\vec{\mathfrak{R}}}

\begin{document}
\begin{titlepage}
\setlength{\topmargin}{1.5in}
\begin{center}
\Huge{Physics 3320} \\
\LARGE{Principles of Electricity and Magnetism II} \\
\Large{Professor Ana Maria Rey} \\[1cm]

\huge{Homework \#\HWnum}\\[0.5cm]

\large{Joe Becker} \\
\large{SID: 810-07-1484} \\
\large{\due} 

\end{center}

\end{titlepage}



\section{Problem \#1}
So if we start with \emph{Green's Identity}
\begin{equation}
\int_V[T\grad^2 U + (\grad T)\cdot(\grad U)]d\tau= \oint_S (T\grad U)\cdot d\vec{a}
\label{greens}
\end{equation}
And if we define a potential $V_3$ as
$$V_3 = V_1-V_2$$
Where $V_1$ and $V_2$ satisfy \emph{Laplace's Equation}
$$\grad^2 V_1 = 0$$
$$\grad^2 V_2 = 0$$
Now if use equation \ref{greens} with $T$ and $U$ replaced with $V_3$ we get
$$\int_V[V_3\grad^2 V_3 + (\grad V_3)\cdot(\grad V_3)]d\tau= \oint_S (V_3\grad V_3)\cdot d\vec{a}$$
Now we know that the electric field due to $V_3$ is given by $$\vecE_3 = -\grad V_3$$
so we now have
$$\int_V[V_3\grad^2 V_3 + (-\vecE_3)\cdot(-\vecE_3)]d\tau= \oint_S (V_3\grad V_3)\cdot d\vec{a}$$
$$\int_V[V_3\grad^2 V_3 + E_3^2]d\tau= \oint_S (V_3\grad V_3)\cdot d\vec{a}$$
If we split the integral up and rewrite the $\grad^2$ we get
$$\int_V \grad\cdot (V_3\grad V_3)d\tau + \int_VE_3^2d\tau= \oint_S (V_3\grad V_3)\cdot d\vec{a}$$
Now we can apply the \emph{Divergence Theorem}
\begin{equation}
\int_V (\grad\cdot\vec{v})d\tau = \oint_S \vec{v}\cdot d\vec{a}
\label{divtheo}
\end{equation}
So using equation \ref{divtheo} where $\vec{v}$ is $V_3\grad V_3$ we get 
$$\oint_S(V_3\grad V_3)\cdot d\vec{a} + \int_VE_3^2d\tau= \oint_S (V_3\grad V_3)\cdot d\vec{a}$$
Now we can see that the surface integrals cancel and we are left with 
$$\int_VE_3^2d\tau= 0$$
and since $E_3^2$ is always positive we know that 
$$E_3 = 0$$
everywhere inside of this volume. Now we know that 
$$\vecE_3 = \vecE_1-\vecE_2$$
from the fact that $V_3= V_1-V_2$ and if we take the negative gradient of $V_3$ we get
$$-\grad V_3 = -\grad(V_1-V_2)$$
$$ \vecE_3 = -\grad(V_1-V_2)$$
Because $\grad$ is a linear operator we can distribute
$$ \vecE_3 = (-\grad V_1) - (-\grad V_2)$$
$$ \vecE_3 = \vecE_1-\vecE_2$$
And we know that $\vecE=0$ so
$$0 = \vecE_1-\vecE_2$$
$$ \vecE_1=\vecE_2$$
Therefore $V_1$ and $V_2$ are the same, and there is only one unique solution to \emph{Laplace's Equation}.

\section{Problem \#2}
\begin{enumerate}[(i)]
\item
We need to use the \emph{Law of Cosines}
$$c^2 = a^2+b^2-2ab\cos(\theta)$$
to rewrite 
\begin{equation}
V(\vec{r}) = \frac{1}{4\pi\epsilon_0}\left(\frac{q}{|\scrptR|}+\frac{q'}{|\scrptR'|}\right)
\label{pot2}
\end{equation}
We can see from the attached drawing that
$$|\scrptR|=\sqrt{a^2+r^2-2ar\cos(\theta)}$$
$$|\scrptR'|=\sqrt{b^2+r^2-2br\cos(\theta)}$$
Where $b=\dfrac{R^2}{a}$ (note that $R$ is the radius of the sphere).
$$|\scrptR'|=\sqrt{\left(\frac{R^2}{a}\right)^2+r^2-2\frac{R^2}{a}r\cos{\theta}}$$
So we can see that equation \ref{pot2} becomes 
$$V(\vec{r}) = \frac{1}{4\pi\epsilon_0}\left(\frac{q}{\sqrt{a^2+r^2-2ar\cos(\theta)}}+\frac{q'}{\sqrt{\left(\dfrac{R^2}{a}\right)^2+r^2-2\dfrac{R^2}{a}r\cos{\theta}}}\right)$$
Where we know from Griffiths equation 3.15
$$q'=-\frac{R}{a}q$$
so
$$V(\vec{r}) = \frac{1}{4\pi\epsilon_0}\left(\frac{q}{\sqrt{a^2+r^2-2ar\cos(\theta)}}-\frac{\dfrac{R}{a}q}{\sqrt{\left(\dfrac{R^2}{a}\right)^2+r^2-2\dfrac{R^2}{a}r\cos{\theta}}}\right)$$
$$V(\vec{r}) = \frac{1}{4\pi\epsilon_0}\left(\frac{q}{\sqrt{a^2+r^2-2ar\cos(\theta)}}-\frac{q}{\dfrac{a}{R}\sqrt{\left(\dfrac{R^2}{a}\right)^2+r^2-2\dfrac{R^2}{a}r\cos(\theta)}}\right)$$
$$V(\vec{r}) = \frac{1}{4\pi\epsilon_0}\left(\frac{q}{\sqrt{a^2+r^2-2ar\cos(\theta)}}-\frac{q}{\sqrt{\dfrac{a^2}{R^2}\left(\dfrac{R^2}{a}\right)^2+\dfrac{a^2}{R^2}r^2-2\dfrac{a^2}{R^2}\dfrac{R^2}{a}r\cos{\theta}}}\right)$$
$$V(\vec{r}) = \frac{1}{4\pi\epsilon_0}\left(\frac{q}{\sqrt{a^2+r^2-2ar\cos(\theta)}}-\frac{q}{\sqrt{R^2+\left(\dfrac{ar}{R}\right)^2-2ar\cos{\theta}}}\right)$$

\item
We can use
$$\sigma= -\epsilon_0\frac{\partial V}{\partial n}$$ 
to find the surface charge distribution $\sigma$. Where $\partial n$ is the partial derivative in the normal direction. In this case the direction that is normal is $r$ so we get 
$$\sigma= -\epsilon_0\frac{\partial V}{\partial r}$$ 
$$\sigma= -\epsilon_0\frac{\partial}{\partial r}\frac{1}{4\pi\epsilon_0}\left(\frac{q}{\sqrt{a^2+r^2-2ar\cos(\theta)}}-\frac{q}{\sqrt{R^2+\left(\dfrac{ar}{R}\right)^2-2ar\cos{\theta}}}\right)$$
$$\sigma= -\frac{q}{4\pi}\frac{\partial}{\partial r}\left(\frac{1}{\sqrt{a^2+r^2-2ar\cos(\theta)}}-\frac{1}{\sqrt{R^2+\left(\dfrac{ar}{R}\right)^2-2ar\cos{\theta}}}\right)$$
$$\sigma= -\frac{q}{4\pi}\frac{\partial}{\partial r}\left(\left({a^2+r^2-2ar\cos(\theta)}\right)^{-1/2}-\left(R^2+\left(\dfrac{ar}{R}\right)^2-2ar\cos{\theta}\right)^{-1/2}\right)$$
$$\sigma= -\frac{q}{4\pi}\left(\frac{-1}{2}\dfrac{2r-2a\cos(\theta)}{\left(a^2+r^2-2ar\cos(\theta)\right)^{3/2}}-\frac{-1}{2}\dfrac{\dfrac{2a^2r}{R^2} - 2a\cos(\theta)}{\left(R^2+\left(\dfrac{ar}{R}\right)^2-2ar\cos{\theta}\right)^{3/2}}\right)$$
$$\sigma= \frac{q}{8\pi}\left(\dfrac{2r-2a\cos(\theta)}{\left(a^2+r^2-2ar\cos(\theta)\right)^{3/2}}-\dfrac{\dfrac{2a^2r}{R^2} - 2a\cos(\theta)}{\left(R^2+\left(\dfrac{ar}{R}\right)^2-2ar\cos{\theta}\right)^{3/2}}\right)$$
If we check the units we assume that $<\sigma> = C\ m^{-2}$ and we know that 
$$<r>=<a>=<R>=m;\ <q>=C$$
So we can say the units of the left hand side are
$$C\ \frac{m-m}{(m^2+m^2+m^2)^{3/2}}-\frac{m-m}{(m^2+m^2+m^2)^{3/2}}$$
$$C\ \frac{m}{(m^2)^{3/2}}-\frac{m}{(m^2)^{3/2}}$$
$$C\ \frac{m}{m^3}-\frac{m}{m^3}$$
$$C\ \frac{1}{m^2}-\frac{1}{m^2}$$
$$C\ m^{-2}$$
Good our units agree.
\end{enumerate}

\section{Problem \#3}
\begin{enumerate}[(i)]
\item
Lets assume that $V(x,y,z)$ can be written in the form
$$V(x,y,z) = X(x)Y(y)Z(z)$$
Where $X$ is a function of only $x$, $Y$ is only a function of $y$ and $Z$ is only a function of $z$. So if we apply \emph{Laplace's Equation} to the potential
\begin{equation}
\grad^2V(x,y,z)=0
\label{lapace}
\end{equation}
Where $\grad^2$ (in Cartesian coordinates) is given  by
$$\grad^2 = \frac{\partial^2}{\partial x^2}+\frac{\partial^2}{\partial y^2}+\frac{\partial^2}{\partial z^2}$$
So we can say
$$\grad^2V(x,y,z) =0= \left(\frac{\partial^2}{\partial x^2}+\frac{\partial^2}{\partial y^2}+\frac{\partial^2}{\partial z^2}\right)X(x)Y(y)Z(z)$$
$$\grad^2V(x,y,z) =0= \frac{\partial^2}{\partial x^2}X(x)Y(y)Z(z)+\frac{\partial^2}{\partial y^2}X(x)Y(y)Z(z)+\frac{\partial^2}{\partial z^2}X(x)Y(y)Z(z)$$
We can see that because $Y(y)$ does not have $x$ dependence it can be pulled out of the partial derivative. This holds yields
$$\grad^2V(x,y,z) =0= Y(y)Z(z)\frac{\partial^2X(x)}{\partial x^2}+X(x)Z(z)\frac{\partial^2Y(y)}{\partial y^2}+X(x)Y(y)\frac{\partial^2Z(z)}{\partial z^2}$$
Now if we divide by $V(x,y,z)$ note that the second order partial derivative has been written with primes as a short hand
$$\frac{0}{X(x)Y(y)Z(z)}= \frac{Y(y)Z(z)X''(x)}{X(x)Y(y)Z(z)}+\frac{X(x)Z(z)Y''(y)}{X(x)Y(y)Z(z)}+\frac{X(x)Y(y)Z''(z)}{X(x)Y(y)Z(z)}$$
We cancel the like terms
$$0= \frac{X''(x)}{X(x)}+\frac{Y''(y)}{Y(y)}+\frac{Z''(z)}{Z(z)}$$
We can see that because we are constant in $z$ that $Z(z)$ is constant and $Z''(z)=0$ so
$$0= \frac{X''(x)}{X(x)}+\frac{Y''(y)}{Y(y)}+\cancelto{0}{\frac{Z''(z)}{Z(z)}}$$
$$0= \frac{X''(x)}{X(x)}+\frac{Y''(y)}{Y(y)}$$
And we also can see that two functions that only depend on $x$ and $y$ must be constant if their sum is equal to zero. So we can say that
$$0= C_1+C_2$$
Where $C_1$ and $C_2$ are constants and
$$C_1 = \frac{X''(x)}{X(x)}$$
$$C_2 = \frac{Y''(y)}{Y(y)}$$
So now we have 2 ordinary differential equations. We also can see that $C_1=-C_2$ so if we define a new constant $k^2=C_1$ so this yields
$$X''(x)= k^2X(x)$$
$$Y''(y)= -k^2Y(y)$$
Note that we picked the $y$ component to have the negative constant because we assume that that will give us a sinusoidal function and we can see from the problem that $y$ varies from zero to zero. So to solve these differential equations we can guess that the solution is $X(x)=e^{rx}$ so $X''(x)=r^2e^{rx}$ so we get
$$r^2e^{rx}=k^2e^{rx}$$
$$r^2=k^2$$
$$r=k$$
So we can say the general solution for $X(x)$ is 
$$X(x) = Ae^{kx}+Be^{-kx}$$
Now if we pick the same guess for $Y(y)$ we get
$$r^2e^{ry}=-k^2e^{ry}$$
$$r^2=-k^2$$
$$r=ik$$
Where $i$ is the imaginary number $i=\sqrt{-1}$. So our general solution is 
$$Y(y) = Ce^{iky}+De^{-iky}$$
Using \emph{Euler's equation}
$$e^{i\theta} = \cos{\theta}+i\sin{\theta}$$ 
we get
$$Y(y) = C(\cos(ky)+i\sin(ky))+D(\cos(-ky)+i\sin(-ky))$$
Because cosine is an even function we know that
$$\cos(-ky)=\cos(ky)$$
and because sine is an odd function we know that
$$\sin(-ky)=-\sin(ky)$$
So this fact gives us
$$Y(y) = C(\cos(ky)+i\sin(ky))+D(\cos(ky)-i\sin(ky))$$
$$Y(y) = C\cos(ky)+Ci\sin(ky))+D\cos(ky)-Di\sin(ky)$$
$$Y(y) = C\cos(ky)+D\cos(ky)+Ci\sin(ky))-Di\sin(ky)$$
$$Y(y) = (C+D)\cos(ky)+i(C-D)(\sin(ky))$$
We can redefine our constants as $C=C+D$ and $D=i(C-D)$. Note that we can absorb the $i$ into the constant $D$. We could also have just picked a constant that would cancel the $i$ later when we set our boundary conditions we know that this is possible because we need to have real answers at the end. So
$$Y(y) = C\cos(ky)+D\sin(ky)$$
So we can write $V(x,y)$ as
$$V(x,y)=\left(Ae^{kx}+Be^{-kx}\right)\left(C\cos(ky)+D\sin(ky)\right)$$
Now we can apply the boundary conditions 
\begin{enumerate}
\item $V(x,0)=0$
\item $V(x,a)=0$
\item $\dfrac{\partial V(0,y)}{\partial x}=0$
\item $V(a,y)=V_0$
\end{enumerate}
So for boundary condition (a) we get
$$V(x,0)=0=\left(Ae^{kx}+Be^{-kx}\right)\left(C\cos(k0)+D\sin(k0)\right)$$
$$V(x,0)=0=\left(Ae^{kx}+Be^{-kx}\right)C$$
And we see that $C=0$ so
$$V(x,y)=\left(Ae^{kx}+Be^{-kx}\right)\sin(ky)$$
Note that we absorbed $D$ into both $A$ and $B$ for simplicity. For the boundary condition (b) we get
$$V(x,a)=0=\left(Ae^{kx}+Be^{-kx}\right)\sin(ka)$$
We see that
$$V(x,a)=0=\sin(ka)$$
$$\arcsin(0)=ka$$
$$n\pi=ka$$
$$k_n=\frac{n\pi}{a}$$
Where $n=1,2,3...$, so we know have 
$$V(x,y)=\left(Ae^{k_nx}+Be^{-k_nx}\right)\sin(k_ny)$$
Now for boundary condition (c) we need to find the first partial with respect to $x$
$$\frac{\partial}{\partial x}V(x,y) = \frac{\partial}{\partial x}\left(Ae^{k_nx}+Be^{-k_nx}\right)\sin(k_ny)$$
$$\frac{\partial}{\partial x}V(x,y) = k_n\left(Ae^{k_nx}-Be^{-k_nx}\right)\sin(k_ny)$$
If we evaluate the boundary condition
$$\frac{\partial}{\partial x}V(0,y) =0= k_n\left(Ae^{k_n0}-Be^{-k_n0}\right)\sin(k_ny)$$
$$\frac{\partial}{\partial x}V(0,y) =0= k_n\left(A-B\right)\sin(k_ny)$$
Because we know that $\sin(k_ny)$ and $k_n$ are not zero we get 
$$\frac{\partial}{\partial x}V(0,y) =0= A-B$$
So we can see that $A=B$ and we can rewrite $V(x,y)$ as
$$V(x,y)=A\left(e^{k_nx}+e^{-k_nx}\right)\sin(k_ny)$$
But this is only for a specific $n$ so we actually have`
$$V_n(x,y)=A_n\left(e^{k_nx}+e^{-k_nx}\right)\sin(k_ny)$$
Where 
$$V(x,y) = \sum_{n=1}^{\infty} V_n(x,y)$$
So 
$$V(x,y) = \sum_{n=1}^{\infty} A_n\left(e^{k_nx}+e^{-k_nx}\right)\sin(k_ny)$$
Now to apply the last boundary condition we need to use the fact that 
\begin{equation}
\int_0^a\sin(k_ny)\sin(k_{n'}y)dy = \left\{ \begin{array}{rcc}
					0 &\mbox{for} &n\neq n' \\
					\dfrac{a}{2} &\mbox{for} &n=n'
					\end{array}\right.
\label{Fouri}
\end{equation}
This is Fourier's trick that uses the orthogonality of $\sin$ so if we apply it to our last boundary condition
$$V(a,y)= V_0 = \sum_{n=1}^{\infty} A_n\left(e^{k_na}+e^{-k_na}\right)\sin(k_ny)$$
$$\int_0^aV_0\sin(k_{n'}y)dy = \sum_{n=1}^{\infty} \int_0^aA_n\left(e^{k_na}+e^{-k_na}\right)\sin(k_ny)\sin(k_{n'}y)dy$$
$$\int_0^aV_0\sin(k_{n'}y)dy = \sum_{n=1}^{\infty} A_n\left(e^{k_na}+e^{-k_na}\right)\int_0^a\sin(k_ny)\sin(k_{n'}y)dy$$
Note that the integral was pulled in the sum. This could go either way the sum of an integral is the same as the integral of a sum. We see that from equation \ref{Fouri} the sum is zero for all $n$ but $n'$ this effectively isolates $n'$ from the sum so 
$$\int_0^aV_0\sin(k_{n'}y)dy = A_{n'}\left(e^{k_{n'}a}+e^{-k_{n'}a}\right)\frac{a}{2}$$
Now we can integrate the right hand side to get
$$\left.-V_0\cos(k_{n}y)\frac{a}{n\pi}\right]_0^a = A_{n}\left(e^{k_{n}a}+e^{-k_{n}a}\right)\frac{a}{2}$$
$$-V_0\frac{a}{n\pi}(\cos(n\pi)-1) = A_{n}\left(e^{k_{n}a}+e^{-k_{n}a}\right)\frac{a}{2}$$
So if we solve for $A_n$ we get 
$$A_n= -V_0\frac{a}{n\pi}\frac{2}{a}\left(e^{k_{n}a}+e^{-k_{n}a}\right)^{-1}(\cos(n\pi)-1)$$
$$A_n= -V_0\frac{2}{n\pi}\left(e^{k_{n}a}+e^{-k_{n}a}\right)^{-1}(\cos(n\pi)-1)$$
Now if we use $k_n = \dfrac{n\pi}{a}$ we get
$$A_n= -V_0\frac{2}{n\pi}\left(e^{n\pi}+e^{-n\pi}\right)^{-1}(\cos(n\pi)-1)$$
$$A_n= -\frac{2V_0}{n\pi\left(e^{n\pi}+e^{-n\pi}\right)}(\cos(n\pi)-1)$$
So we can see for $n=2,4,6...$ we get
$$A_n= -\frac{2V_0}{n\pi\left(e^{n\pi}+e^{-n\pi}\right)}(1-1)$$
$$A_n= 0$$
And for $n=1,3,5...$ we get
$$A_n= -\frac{2V_0}{n\pi\left(e^{n\pi}+e^{-n\pi}\right)}(-1-1)$$
$$A_n= -\frac{2V_0}{n\pi\left(e^{n\pi}+e^{-n\pi}\right)}(-2)$$
$$A_n= \frac{4V_0}{n\pi\left(e^{n\pi}+e^{-n\pi}\right)}$$
So we know know out last constant so we can say that
$$V(x,y) = \sum_{n=1,3,5...}^{\infty}\frac{4V_0}{n\pi\left(e^{n\pi}+e^{-n\pi}\right)}\left(e^{k_nx}+e^{-k_nx}\right)\sin(k_ny)$$
Where
$$k_n = \frac{n\pi}{a}$$

\item
The left wall is constant with respect to $x$, but it could potentially be any function of $y$ and still satisfy $$\frac{\partial V}{\partial x} = 0$$ This surface is not a conductor because a conductor has a constant potential across it and this plate does necessarily fulfill that condition.

See attached for the sketch of the electric field and the equipotential contours inside the pipe.

\item
To find the charge density at $y=0$ or along the bottom plate of the pipe we need to use the identity
$$\sigma= -\epsilon_0\frac{\partial V}{\partial n}$$ 
where $n$ is the normal direction. In this case it the normal direction is $y$ so
$$\sigma= -\epsilon_0\frac{\partial V(x,0)}{\partial y}$$ 
$$\sigma= -\epsilon_0\frac{\partial }{\partial y}\sum_{n=1,3,5...}^{\infty}\frac{4V_0}{n\pi\left(e^{n\pi}+e^{-n\pi}\right)}\left(e^{k_nx}+e^{-k_nx}\right)\sin(k_ny)$$
$$\sigma= -\epsilon_0\sum_{n=1,3,5...}^{\infty}\frac{4V_0}{n\pi\left(e^{n\pi}+e^{-n\pi}\right)}\left(e^{k_nx}+e^{-k_nx}\right)\frac{\partial }{\partial y}\sin(k_ny)$$
$$\sigma= -\epsilon_0\sum_{n=1,3,5...}^{\infty}\frac{4V_0}{n\pi\left(e^{n\pi}+e^{-n\pi}\right)}\left(e^{k_nx}+e^{-k_nx}\right)\cos(k_ny)k_n$$
So we can evaluate at the point $V(x,0)$
$$\sigma= -\epsilon_0\sum_{n=1,3,5...}^{\infty}\frac{4V_0}{n\pi\left(e^{n\pi}+e^{-n\pi}\right)}\left(e^{k_nx}+e^{-k_nx}\right)\cos(k_n0)k_n$$
$$\sigma= -\epsilon_0\sum_{n=1,3,5...}^{\infty}\frac{4V_0}{n\pi\left(e^{n\pi}+e^{-n\pi}\right)}\left(e^{k_nx}+e^{-k_nx}\right)(1)k_n$$
$$\sigma= -\epsilon_0\sum_{n=1,3,5...}^{\infty}\frac{n\pi}{a}\frac{4V_0}{n\pi\left(e^{n\pi}+e^{-n\pi}\right)}\left(e^{k_nx}+e^{-k_nx}\right)$$
$$\sigma= \sum_{n=1,3,5...}^{\infty}\frac{-4V_0\epsilon_0}{a\left(e^{n\pi}+e^{-n\pi}\right)}\left(e^{k_nx}+e^{-k_nx}\right)$$
Now we can check our units we assume that $<\sigma> = C\ m^{-2}$ and we know that
$$<k_n> = m^{-1};\ <x>=<a>=m;\ <\epsilon_0>=C^2\ N^{-1}\ m^{-2};\ <V_0> = N\ m\ C^{-1}$$
As a quick check we see that
$$\left<e^{k_nx}+e^{-k_nx}\right> = e^{m^{-1}\ m} + e^{m^{-1}\ m}$$
therefore this term has no units
$$\left<\frac{-4V_0\epsilon_0}{a\left(e^{n\pi}+e^{-n\pi}\right)}\right>=\frac{N\ m\ C^{-1}\ C^2\ N^{-1}\ m^{-2}}{m}$$
$$\left<\frac{-4V_0\epsilon_0}{a\left(e^{n\pi}+e^{-n\pi}\right)}\right>=\frac{C\ m^{-1}}{m}$$
$$\left<\frac{-4V_0\epsilon_0}{a\left(e^{n\pi}+e^{-n\pi}\right)}\right>=C\ m^{-2}$$
Good our units agree.

\end{enumerate}

\section{Problem \#4}
To see the rigorous work through the separation of variables see problem 3 for this problem we will start with
$$\grad^2V(x,y,z) = 0 = \frac{X''(x)}{X(x)}+\frac{Y''(y)}{Y(y)}+\frac{Z''(z)}{Z(z)}$$
So we can say that
$${X''(x)}=C_1 {X(x)}$$
$${Y''(y)}=C_2{Y(y)}$$
$${Z''(z)}=C_3{Z(z)}$$
Where $C_1+C_2+C_3=0$. Again we see that we have 2 ($y$ and $z$) components that go from zero to zero, so we have to say that $C_2 = -k^2$ and $C_3 = -l^2$. So from the fact that $C_1 = -C_2-C_3$ we get $C_1= k^2+l^2$ so we can say that.
$$X(x) = Ae^{\sqrt{k^2+l^2}x}+Be^{-\sqrt{k^2+l^2}x}$$
$$Y(y) = C\cos(ky)+D\sin(ky)$$
$$Z(z) = E\cos(lz)+F\sin(lz)$$
So we can say that $V(x,y,z)$ is
$$V(x,y,z) = \left(Ae^{\sqrt{k^2+l^2}x}+Be^{-\sqrt{k^2+l^2}x}\right) \left(C\cos(ky)+D\sin(ky)\right) \left(E\cos(lz)+F\sin(lz)\right)$$
Again to see the rigorous solutions to these differential equation see problem 3. Now we can apply our boundary conditions
\begin{enumerate}[(a)]
\item $V(x,0,z) = 0$
\item $V(x,y,0) = 0$
\item $V(x,a,z) = 0$
\item $V(x,y,a) = 0$
\item $V(a/2,y,z) = V_0$
\item $V(-a/2,y,z) = V_0$
\end{enumerate}
So for (a) we get
$$V(x,0,z)=0= \left(Ae^{\sqrt{k^2+l^2}x}+Be^{-\sqrt{k^2+l^2}x}\right) \left(C\cos(k0)+D\sin(k0)\right) \left(E\cos(lz)+F\sin(lz)\right)$$
$$0= \left(Ae^{\sqrt{k^2+l^2}x}+Be^{-\sqrt{k^2+l^2}x}\right) \left(C\right) \left(E\cos(lz)+F\sin(lz)\right)$$
We see that for this to be true $C=0$ so $V$ becomes 
$$V(x,y,z) = \left(Ae^{\sqrt{k^2+l^2}x}+Be^{-\sqrt{k^2+l^2}x}\right) \left(D\sin(ky)\right) \left(E\cos(lz)+F\sin(lz)\right)$$
For (b) we get
$$V(x,y,0) = 0 = \left(Ae^{\sqrt{k^2+l^2}x}+Be^{-\sqrt{k^2+l^2}x}\right) \left(D\sin(ky)\right) \left(E\cos(l0)+F\sin(l0)\right)$$
$$V(x,y,0) = 0 = \left(Ae^{\sqrt{k^2+l^2}x}+Be^{-\sqrt{k^2+l^2}x}\right) \left(D\sin(ky)\right) \left(E\right)$$
So we see that $E=0$ so $V$ becomes
$$V(x,y,z) = \left(Ae^{\sqrt{k^2+l^2}x}+Be^{-\sqrt{k^2+l^2}x}\right)D\sin(ky)F\sin(lz)$$
We can combine $D$ and $F$ into $A$ and $B$ to yield
$$V(x,y,z) = \left(Ae^{\sqrt{k^2+l^2}x}+Be^{-\sqrt{k^2+l^2}x}\right)\sin(ky)\sin(lz)$$
Now if we use (c) we get 
$$V(x,a,z) = 0 = \left(Ae^{\sqrt{k^2+l^2}x}+Be^{-\sqrt{k^2+l^2}x}\right)\sin(ka)\sin(lz)$$
Now we see that the only term that can be zero is
$$0=\sin(ka)$$
That means 
$$k_n = \frac{n\pi}{a}$$
for the boundary condition (d) we see
$$V(x,y,a) = 0 = \left(Ae^{\sqrt{k^2+l^2}x}+Be^{-\sqrt{k^2+l^2}x}\right)\sin(k_ny)\sin(la)$$
Again the only term that can be zero is
$$0=\sin(la)$$
So 
$$l_m = \frac{m\pi}{a}$$
Now our $V$ is 
$$V_{n,m}(x,y,z) = \left(Ae^{\sqrt{k_n^2+l_m^2}x}+Be^{-\sqrt{k_n^2+l_m^2}x}\right)\sin(k_ny)\sin(l_mz)$$
And our full function of $V$ is given by a double sum
$$V(x,y,z) = \sum_{n=1}^{\infty}\sum_{m=1}^{\infty}\left(Ae^{\sqrt{k_n^2+l_m^2}x}+Be^{-\sqrt{k_n^2+l_m^2}x}\right)\sin(k_ny)\sin(l_mz)$$
Factoring our the common $\pi$ in the exponents 
$$V(x,y,z) = \sum_{n=1}^{\infty}\sum_{m=1}^{\infty}\left(Ae^{\sqrt{(n\pi/a)^2+(m\pi/a)^2}x}+Be^{-\sqrt{(n\pi/a)^2+(m\pi/a)^2}x}\right)\sin(k_ny)\sin(l_mz)$$
$$V(x,y,z) = \sum_{n=1}^{\infty}\sum_{m=1}^{\infty}\left(Ae^{\pi\sqrt{(n/a)^2+(m/a)^2}x}+Be^{-\pi\sqrt{(n/a)^2+(m/a)^2}x}\right)\sin(k_ny)\sin(l_mz)$$
Now to fit our last two boundary conditions, For (e) we get
$$V(a/2,y,z) = V_0 = \sum_{n=1}^{\infty}\sum_{m=1}^{\infty}\left(Ae^{\pi\sqrt{(n/a)^2+(m/a)^2}a/2}+Be^{-\pi\sqrt{(n/a)^2+(m/a)^2}a/2}\right)\sin(k_ny)\sin(l_mz)$$
$$V_0 = \sum_{n=1}^{\infty}\sum_{m=1}^{\infty}\left(Ae^{\pi\sqrt{(a/2)^2(n/a)^2+(a/2)^2(m/a)^2}}+Be^{-\pi\sqrt{(a/2)^2(n/a)^2+(a/2)^2(m/a)^2}}\right)\sin(k_ny)\sin(l_mz)$$
$$V_0 = \sum_{n=1}^{\infty}\sum_{m=1}^{\infty}\left(A_{n,m}e^{\pi\sqrt{(n/2)^2+(m/2)^2}}+B_{n.m}e^{-\pi\sqrt{(n/2)^2+(m/2)^2}}\right)\sin(k_ny)\sin(l_mz)$$
and for (f) we get
$$V(-a/2,y,z) = V_0 = \sum_{n=1}^{\infty}\sum_{m=1}^{\infty}\left(Ae^{\pi\sqrt{(n/a)^2+(m/a)^2}(-a/2)}+Be^{-\pi\sqrt{(n/a)^2+(m/a)^2}(-a/2)}\right)\sin(k_ny)\sin(l_mz)$$
$$V_0 = \sum_{n=1}^{\infty}\sum_{m=1}^{\infty}\left(Ae^{-\pi\sqrt{(a/2)^2(n/a)^2+(a/2)^2(m/a)^2}}+Be^{\pi\sqrt{(a/2)^2(n/a)^2+(a/2)^2(m/a)^2}}\right)\sin(k_ny)\sin(l_mz)$$
$$V_0 = \sum_{n=1}^{\infty}\sum_{m=1}^{\infty}\left(A_{n,m}e^{-\pi\sqrt{(n/2)^2+(m/2)^2}}+B_{n,m}e^{\pi\sqrt{(n/2)^2+(m/2)^2}}\right)\sin(k_ny)\sin(l_mz)$$
We see that these equal the same constant but only differ in a flipped exponent sign so we can infer that
$$A_{n,m}e^{-\pi\sqrt{(n/2)^2+(m/2)^2}}+B_{n,m}e^{\pi\sqrt{(n/2)^2+(m/2)^2}}=A_{n,m}e^{\pi\sqrt{(n/2)^2+(m/2)^2}}+B_{n,m}e^{-\pi\sqrt{(n/2)^2+(m/2)^2}}$$
$$A_{n,m}e^{-\pi\sqrt{(n/2)^2+(m/2)^2}}-A_{n,m}e^{\pi\sqrt{(n/2)^2+(m/2)^2}}=B_{n,m}e^{-\pi\sqrt{(n/2)^2+(m/2)^2}}-B_{n,m}e^{\pi\sqrt{(n/2)^2+(m/2)^2}}$$
$$A_{n,m}\left(e^{-\pi\sqrt{(n/2)^2+(m/2)^2}}-e^{\pi\sqrt{(n/2)^2+(m/2)^2}}\right)=B_{n,m}\left(e^{-\pi\sqrt{(n/2)^2+(m/2)^2}}-e^{\pi\sqrt{(n/2)^2+(m/2)^2}}\right)$$
$$A_{n,m}= B_{n,m}$$
This means we can rewrite our function with just one constant factored out to give us
$$V(x,y,z) = \sum_{n=1}^{\infty}\sum_{m=1}^{\infty}A_{n,m}\left(e^{\sqrt{k_n^2+l_m^2}x}+e^{-\sqrt{k_n^2+l_m^2}x}\right)\sin(k_ny)\sin(l_mz)$$

Now if we use equation \ref{Fouri} we can find the constant $A_{n,m}$. So
$$\sum_{n=1}^{\infty}\sum_{m=1}^{\infty}A_{m,n}\left(e^{\pi\sqrt{(n/2)^2+(m/2)^2}}+e^{-\pi\sqrt{(n/2)^2+(m/2)^2}}\right)\int_0^a\sin(k_ny)\sin(k_{n'}y)dy\int_0^a\sin(l_mz)\sin(l_{m'}z)$$
$$V_0 \int_0^a\sin(k_{n}y)dy\int_0^a\sin(l_{m'}y)dz= \sum_{m=1}^{\infty}A_{n,m}\left(e^{\pi\sqrt{(n/2)^2+(m/2)^2}}+e^{-\pi\sqrt{(n/2)^2+(m/2)^2}}\right)\frac{a}{2}\int_0^a\sin(l_mz)\sin(l_{m'}z)$$
$$V_0 \int_0^a\sin(k_{n}y)dy\int_0^a\sin(l_{m}y)dz= A_{n,m}\left(e^{\pi\sqrt{(n/2)^2+(m/2)^2}}+e^{-\pi\sqrt{(n/2)^2+(m/2)^2}}\right)\frac{a^2}{4}$$
Now if we solve the right hand side we get
$$V_0 \left.\frac{a}{n\pi}\cos(k_ny)\right]_0^a\left.\frac{a}{m\pi}\cos(l_my)\right]_0^a= A_{n,m}\left(e^{\pi\sqrt{(n/2)^2+(m/2)^2}}+e^{-\pi\sqrt{(n/2)^2+(m/2)^2}}\right)\frac{a^2}{4}$$
Note that the negatives from the cosines canceled with each other.
$$\frac{V_0a^2}{nm\pi^2}\left(\cos(n\pi)-1\right)\left(\cos(m\pi)-1\right)= A_{n,m}\left(e^{\pi\sqrt{(n/2)^2+(m/2)^2}}+e^{-\pi\sqrt{(n/2)^2+(m/2)^2}}\right)\frac{a^2}{4}$$
Now if we solve for $A_{n,m}$ we get
$$\frac{4V_0}{nm\pi^2}\left(\cos(n\pi)-1\right)\left(\cos(m\pi)-1\right)= A_{n,m}\left(e^{\pi\sqrt{(n/2)^2+(m/2)^2}}+e^{-\pi\sqrt{(n/2)^2+(m/2)^2}}\right)$$
$$A_{n,m}=\frac{4V_0}{nm\pi^2}\left(\cos(n\pi)-1\right)\left(\cos(m\pi)-1\right) \left(e^{\pi\sqrt{(n/2)^2+(m/2)^2}}+e^{-\pi\sqrt{(n/2)^2+(m/2)^2}}\right)^{-1}$$
We see that for $n=2,4,6...$ or $m=2,4,6...$ $\cos-1$ becomes zero so the whole term is zero. And for $n=1,3,5...$ or  $m=1,3,5$ the $\cos-1$ becomes $-2$ so
$$A_{n,m}=\frac{4V_0}{nm\pi^2}(-2)(-2)\left(e^{\pi\sqrt{(n/2)^2+(m/2)^2}}+e^{-\pi\sqrt{(n/2)^2+(m/2)^2}}\right)^{-1}$$
$$A_{n,m}=\frac{16V_0}{nm\pi^2\left(e^{\pi\sqrt{(n/2)^2+(m/2)^2}}+e^{-\pi\sqrt{(n/2)^2+(m/2)^2}}\right)}$$
So our full solution for $V(x,y,z)$ is 
$$\sum_{n=1,3,5...}^{\infty}\sum_{m=1,3,5...}^{\infty}\frac{16V_0}{nm\pi^2\left(e^{\pi\sqrt{(n/2)^2+(m/2)^2}}+e^{-\pi\sqrt{(n/2)^2+(m/2)^2}}\right)}\left(e^{\sqrt{k_n^2+l_m^2}x}+e^{-\sqrt{k_n^2+l_m^2}x}\right)\sin(k_ny)\sin(l_mz)$$
where $k_n=\dfrac{n\pi}{a}$ and $l_m=\dfrac{m\pi}{a}$

We can see that at the center of the cube or $V(0,a/2,a/2)$ we get 
$$\sum_{n=1,3,5...}^{\infty}\sum_{m=1,3,5...}^{\infty}\frac{16V_0}{nm\pi^2\left(e^{\pi\sqrt{(n/2)^2+(m/2)^2}}+e^{-\pi\sqrt{(n/2)^2+(m/2)^2}}\right)}\left(e^{\sqrt{k_n^2+l_m^2}0}+e^{-\sqrt{k_n^2+l_m^2}0}\right)\sin(k_na/2)\sin(l_ma/2)$$
$$\sum_{n=1,3,5...}^{\infty}\sum_{m=1,3,5...}^{\infty}\frac{16V_0}{nm\pi^2\left(e^{\pi\sqrt{(n/2)^2+(m/2)^2}}+e^{-\pi\sqrt{(n/2)^2+(m/2)^2}}\right)}(2)\sin(\frac{n\pi}{a} a/2)\sin(\frac{m\pi}{a} a/2)$$
$$\sum_{n=1,3,5...}^{\infty}\sum_{m=1,3,5...}^{\infty}\frac{32V_0}{nm\pi^2\left(e^{\pi\sqrt{(n/2)^2+(m/2)^2}}+e^{-\pi\sqrt{(n/2)^2+(m/2)^2}}\right)}\sin(\frac{n\pi}{2})\sin(\frac{m\pi}{2})$$
We can see that the sines will never be zero, because the indexes are always odd. So $V$ is not zero at the center of the cube and neither is $E$.

\section{Problem \#5}
\begin{enumerate}[(i)]
\item
So if we start by saying that
$$V(r,\theta,\phi) = R(r)\Theta(\theta)\Phi(\phi)$$
We are going to solve \emph{Laplace's Equation} with spherical coordinates 
$$\grad^2V(r.\theta,\phi) = 0$$
where $\grad^2$ in spherical coordinates is given by
$$\grad^2 = \frac{1}{r^2}\frac{\partial}{\partial r}\left(r^2\frac{\partial}{\partial r}\right) + \frac{1}{r^2\sin(\theta)}\frac{\partial}{\partial \theta}\left(\sin(\theta)\frac{\partial}{\partial \theta}\right) + \frac{1}{r^2\sin^2(\theta)}\frac{\partial}{\partial r}\frac{\partial^2}{\partial \phi^2}$$
So to solve \emph{Laplace's Equation} we first can assume that $V$ has no dependence on $\phi$ so that the problem is reduced to 2 dimensions.
$$\grad^2V(r,\theta) = \frac{1}{r^2}\frac{\partial}{\partial r}\left(r^2\frac{\partial V(r,\theta)}{\partial r}\right) + \frac{1}{r^2\sin(\theta)}\frac{\partial}{\partial \theta}\left(\sin(\theta)\frac{\partial V(r,\theta)}{\partial \theta}\right) = 0$$ 
$$\frac{1}{r^2}\frac{\partial}{\partial r}\left(r^2\frac{\partial }{\partial r}R(r)\Theta(\theta)\right) + \frac{1}{r^2\sin(\theta)}\frac{\partial}{\partial \theta}\left(\sin(\theta)\frac{\partial}{\partial \theta}R(r)\Theta(\theta)\right) = 0$$ 
$$\Theta(\theta)\frac{1}{r^2}\frac{\partial}{\partial r}\left(r^2\frac{\partial }{\partial r}R(r)\right) + R(r)\frac{1}{r^2\sin(\theta)}\frac{\partial}{\partial \theta}\left(\sin(\theta)\frac{\partial}{\partial \theta}\Theta(\theta)\right) = 0$$ 
Now if we divide by $V(r,\theta)$ we get 
$$\frac{\Theta(\theta)}{R(r)\Theta(\theta)}\frac{1}{r^2}\frac{\partial}{\partial r}\left(r^2\frac{\partial }{\partial r}R(r)\right) + \frac{R(r)}{R(r)\Theta(\theta)}\frac{1}{r^2\sin(\theta)}\frac{\partial}{\partial \theta}\left(\sin(\theta)\frac{\partial}{\partial \theta}\Theta(\theta)\right) = 0$$ 
$$\frac{1}{R(r)}\frac{1}{r^2}\frac{\partial}{\partial r}\left(r^2\frac{\partial }{\partial r}R(r)\right) + \frac{1}{\Theta(\theta)}\frac{1}{r^2\sin(\theta)}\frac{\partial}{\partial \theta}\left(\sin(\theta)\frac{\partial}{\partial \theta}\Theta(\theta)\right) = 0$$ 
We can cancel the $\frac{1}{r^2}$ from both terms
$$\frac{1}{R(r)}\frac{\partial}{\partial r}\left(r^2\frac{\partial }{\partial r}R(r)\right) + \frac{1}{\Theta(\theta)}\frac{1}{\sin(\theta)}\frac{\partial}{\partial \theta}\left(\sin(\theta)\frac{\partial}{\partial \theta}\Theta(\theta)\right) = 0$$ 
Now we have 2 terms that only depend on one variable. These also sum to zero so we know that they are constant so we can say
$$\frac{\partial}{\partial r}\left(r^2\frac{\partial }{\partial r}R(r)\right) = C_1 R(r)$$
$$\frac{\partial}{\partial \theta}\left(\sin(\theta)\frac{\partial}{\partial \theta}\Theta(\theta)\right) = C_2\sin(\theta)\Theta(\theta)$$ 
We assume that $C_1 = l(l+1)$ where $l\in\mathbb{Z}$. This also implies that $C_2 = -l(l+1)$ due to the fact that $C_1+C_2=0$ from the solution of \emph{Laplace's Equation}. So
$$\frac{\partial}{\partial r}\left(r^2\frac{\partial }{\partial r}R(r)\right) = l(l+1)R(r)$$
$$\frac{\partial}{\partial \theta}\left(\sin(\theta)\frac{\partial}{\partial \theta}\Theta(\theta)\right) =-l(l+1)\sin(\theta)\Theta(\theta)$$ 
Where the general solutions to these differential equation are
$$R(r) = Ar^l+\frac{B}{r^{l+1}}$$
$$\Theta(\theta) = P_l(\cos\theta)$$
Where $P_l$ are the \emph{Legendre polynomials} which are given by
$$P_l(x) = \frac{1}{2^ll!}\left(\frac{d}{dx}\right)^l(x^2-1)^l$$
So we can say that
\begin{align*}
&P_0(\cos\theta) = 1\\
&P_1(\cos\theta) = \cos\theta \\
&P_2(\cos\theta) =  \frac{3}{2}\cos^2\theta-\frac{1}{2}\\ 
\end{align*}
So we know that $V$ is given by
$$V_l(r,\theta) = \left(Ar^l+\frac{B}{r^{l+1}}\right)P_l(\cos\theta)$$
$$V(r,\theta) = \sum_{l=0}^{\infty}\left(A_lr^l+\frac{B_l}{r^{l+1}}\right)P_l(\cos\theta)$$

So we can apply the boundary conditions
\begin{enumerate}
\item $V(r\rightarrow\infty, \theta)=0$
\item$V(R,\theta) = V_0\cos(2\theta)$
\end{enumerate}
Now for inside the sphere we can see that $B_l$ needs to be zero for all $l$ otherwise the potential at the center will by undefined so we see that
$$V(r,\theta) = \sum_{l=0}^{\infty}A_lr^lP_l(\cos\theta)$$
So if we apply boundary condition (b) we can say
$$V(R,\theta) =V_0\cos(2\theta) =  \sum_{l=0}^{\infty}A_lR^lP_l(\cos\theta)$$
We can use the identity that $$\cos(2\theta) = 2\cos^2(\theta)-1$$
So if we look at that identity we see that we can just use $l=0$ and $l=2$ to make that function so we do not need to use the infinite sum.
$$V(R,\theta) =V_0\cos(2\theta) =  A_0R^0P_0(\cos\theta)+A_2R^2P_2(\cos\theta)$$
$$V(R,\theta) =V_0\cos(2\theta) =  A_0+A_2R^2\left(\frac{3}{2}\cos^2\theta-\frac{1}{2}\right)$$
So if we apply the identity we get
$$V(R,\theta) =V_0(2\cos^2(\theta)-1)=  A_0+A_2R^2\left(\frac{3}{2}\cos^2\theta-\frac{1}{2}\right)$$
So we need to pick a $A_0$ and $A_2$ such that we get the left hand side. So lets get a factor of 2 in front of the $\cos^2(\theta)$ so lets pick 
$$A_2 = \frac{4V_0}{3R^2}$$
Note the $V_0$ is to account for the $V_0$ on the right hand side.
$$V(R,\theta) =V_0(2\cos^2(\theta)-1)=  A_0+\frac{4}{3R^2}R^2\left(\frac{3}{2}\cos^2\theta-\frac{1}{2}\right)$$
$$V(R,\theta) =V_0(2\cos^2(\theta)-1)=  A_0+\frac{4V_0}{3}\left(\frac{3}{2}\cos^2\theta-\frac{1}{2}\right)$$
$$V(R,\theta) =V_0(2\cos^2(\theta)-1)=  A_0+V_0\left(\frac{4}{2}\cos^2\theta-\frac{2}{3}\right)$$
$$V(R,\theta) =V_0(2\cos^2(\theta)-1)=  A_0+V_0\left(2\cos^2\theta-\frac{2}{3}\right)$$
Now we can pick $A_0$ as $\frac{-1}{3}V_0$ so we get
$$V(R,\theta) =V_0(2\cos^2(\theta)-1)=  -\frac{1}{3}V_0+V_0\left(2\cos^2\theta-\frac{2}{3}\right)$$
$$V(R,\theta) =V_0(2\cos^2(\theta)-1)=  V_0\left(-\frac{1}{3}+2\cos^2\theta-\frac{2}{3}\right)$$
$$V(R,\theta) =V_0(2\cos^2(\theta)-1)=  V_0\left(2\cos^2\theta-1\right)$$
Good we our final value of $V$ is 
$$V(r,\theta) = -\frac{V_0}{3} + \frac{4V_0}{3R^2}r^2P_2(\cos\theta)\ \mbox{for}\ r<R$$


Now for outside the sphere we have the boundary condition (a)
$$V(r\rightarrow\infty,\theta)=0 = \sum_{l=0}^{\infty}\left(A_lr^l+\frac{B_l}{r^{l+1}}\right)P_l(\cos\theta)$$
We see that $A_l$ has to equal zero or else our potential cannot go to zero so
$$V(r,\theta) = \sum_{l=0}^{\infty}\frac{B_l}{r^{l+1}}P_l(\cos\theta)$$
Now we can apply boundary condition (b) to yield
$$V(R,\theta) = V_0\cos(2\theta)= \sum_{l=0}^{\infty}\frac{B_l}{R^{l+1}}P_l(\cos\theta)$$
Like before we can use the trig identity to convert the left hand side
$$V(R,\theta) = V_0(2\cos^2(\theta)-1)= \sum_{l=0}^{\infty}\frac{B_l}{R^{l+1}}P_l(\cos\theta)$$
And again we will only need to use $l=0$ and $l=2$ and not the infinite sum.
$$V(R,\theta) = V_0(2\cos^2(\theta)-1)= \frac{B_0}{R^{0+1}}P_0(\cos\theta)+\frac{B_2}{R^{2+1}}P_2(\cos\theta)$$
$$V(R,\theta) = V_0(2\cos^2(\theta)-1)= \frac{B_0}{R}+\frac{B_2}{R^{3}}\left(\frac{3}{2}\cos^2(\theta) - \frac{1}{2}\right)$$
We can pick $B_2$ to be
$$B_2 = \frac{4V_0R^3}{3}$$
this gives us
$$V(R,\theta) = V_0(2\cos^2(\theta)-1)= \frac{B_0}{R}+\frac{4V_0R^3}{3}\frac{1}{R^{3}}\left(\frac{3}{2}\cos^2(\theta) - \frac{1}{2}\right)$$
$$V(R,\theta) = V_0(2\cos^2(\theta)-1)= \frac{B_0}{R}+\frac{4V_0}{3}\left(\frac{3}{2}\cos^2(\theta) - \frac{1}{2}\right)$$
$$V(R,\theta) = V_0(2\cos^2(\theta)-1)= \frac{B_0}{R}+V_0\left(\frac{4}{3}\frac{3}{2}\cos^2(\theta) - \frac{4}{3}\frac{1}{2}\right)$$
$$V(R,\theta) = V_0(2\cos^2(\theta)-1)= \frac{B_0}{R}+V_0\left(2\cos^2(\theta) - \frac{2}{3}\right)$$
Now we can pick $B_0$ to be
$$B_0 = \frac{-V_0R}{3}$$
This yields
$$V(R,\theta) = V_0(2\cos^2(\theta)-1)= \frac{-V_0R}{3}\frac{1}{R}+V_0\left(2\cos^2(\theta) - \frac{2}{3}\right)$$
$$V(R,\theta) = V_0(2\cos^2(\theta)-1)= V_0\frac{-1}{3}+V_0\left(2\cos^2(\theta) - \frac{2}{3}\right)$$
Factoring out the $V_0$
$$V(R,\theta) = V_0(2\cos^2(\theta)-1)= V_0\left(\frac{-1}{3}+2\cos^2(\theta) - \frac{2}{3}\right)$$
$$V(R,\theta) = V_0(2\cos^2(\theta)-1)= V_0\left(2\cos^2(\theta) - 1\right)$$
Good we picked a $B_0$ and $B_2$ so we can say that the potential outside of the sphere is 
$$V(r,\theta) = \frac{-V_0R}{3r}+\frac{4V_0R^3}{3r^3}P_2(\cos\theta)\ \mbox{for}\ R<r$$

\item
$$V(r,\theta) = -\frac{V_0}{3} + \frac{4V_0}{3R^2}r^2P_2(\cos\theta)\ \mbox{for}\ r<R$$
$$V(r,\theta) = \frac{-V_0R}{3r}+\frac{4V_0R^3}{3r^3}P_2(\cos\theta)\ \mbox{for}\ R<r$$
So to find the charge distribution we need to use
$$\sigma = -\epsilon_0\frac{\partial V(r,\theta)}{\partial n}$$
we evaluate this at $r=R$ and we take the derivative with respect to $r$ because that is the normal direction on the surface of the sphere so
$$\sigma = -\epsilon_0\frac{\partial V(R,\theta)}{\partial r}$$
$$\sigma = -\epsilon_0\frac{\partial}{\partial r}\left(-\frac{V_0}{3} + \frac{4V_0}{3R^2}r^2P_2(\cos\theta)\right)$$
$$\sigma = -\epsilon_0\left(\frac{4V_0}{3R^2}(2r)P_2(\cos\theta)\right)$$
Evaluate at $r=R$
$$\sigma = -\epsilon_0\left(\frac{8V_0}{3R^2}RP_2(\cos\theta)\right)$$
$$\sigma(\theta) = \frac{-8V_0\epsilon_0}{3R}P_2(\cos\theta)$$

\end{enumerate}


\end{document}

