\documentclass[11pt]{article}

\usepackage{latexsym}
\usepackage{amssymb}
\usepackage{enumerate}
\usepackage{amsthm}
\usepackage{amsmath}
\usepackage{ulem}
\usepackage{cancel}

\setlength{\evensidemargin}{.25in}
\setlength{\oddsidemargin}{-.25in}
\setlength{\topmargin}{-.75in}
\setlength{\textwidth}{6.5in}
\setlength{\textheight}{9.5in}
\newcommand{\due}{October 28th, 2009}
\newcommand{\HWnum}{8}
\newcommand{\CC}{\mathbb{C}}
\newcommand{\ZZ}{\mathbb{Z}}
\newcommand{\zbar}{\overline{z}}
\newcommand{\Let}{\textnormal{Let }}
\newcommand{\Arg}{\textnormal{arg}}
\newcommand{\Log}{\textnormal{Log}}
\newcommand{\dd}{\displaystyle}

\begin{document}
\begin{titlepage}
\setlength{\topmargin}{1.5in}
\begin{center}
\Huge{Physics 3310} \\
\LARGE{Principles of Electricity and Magnetism 1} \\
\Large{Professor Thomas R. Schibli} \\[1cm]

\huge{Homework \#\HWnum}\\[0.5cm]

\large{Joe Becker} \\
\large{SID: 810-07-1484} \\
\large{\due} 

\end{center}

\end{titlepage}


\subsection*{Written Problems}
\begin{enumerate}
\item 4.4: 18, 19  
\begin{enumerate}[(i)]
\item 18)
Let 
$$I:=\oint_{|z|=2} \frac{dz}{z^2(z-1)^3}$$
\begin{enumerate}[(a)]
\item For every $R>2$, $I=I(R)$, where
$$I(R) := \oint_{|z|=R}\frac{1}{z^2(z-1)^3}dz$$
This is true because the contour $\Gamma$ where $\Gamma$ is represented by $|z|=2$ can be continuously deformed to be any other contour, because the points of discontinuity are inside the circle already. So we continuously deform $\Gamma$ into the form $|z|=R$
\item
$$|I(R)| \le \frac{2\pi}{R(R-1)^3}$$ for $R>2$. 

First we see that
$$|I(R)|= \left|\oint_{|z|=R}\frac{1}{z^2(z-1)^3}dz\right|$$
Now we can say
$$\left|\oint_{|z|=R}\frac{1}{z^2(z-1)^3}dz\right|\le \max_{|z|=R}\left\{\left|\frac{1}{z^2(z-1)^3}\right|\right\}l(\Gamma)$$
Where $l(\Gamma)$ is the length of the circle $|z|=R$ or $2\pi R$ so now we need to calculate the maximum of the magnitude of the integrand. We can use the triangle inequality to say that
$$\left|\frac{1}{z^2(z-1)^3}dz\right| \le \frac{1}{|z|^2(|z|-1)^3}dz$$
So we know that we are bound by the contour $|z|=R$ so we see that
$$\left|\frac{1}{z^2(z-1)^3}dz\right| \le \frac{1}{R^2(R-1)^3}dz$$
So we see that the bound of the integral is 
\begin{align*}
\left|\oint_{|z|=R}\frac{1}{z^2(z-1)^3}dz\right| &\le \max_{|z|=R}\left\{\left|\frac{1}{z^2(z-1)^3}\right|\right\}l(\Gamma)\\
&\le \frac{1}{R^2(R-1)^3}2\pi R\\
&\le \frac{2\pi}{R(R-1)^3}\\
|I(R)| &\le \frac{2\pi}{R(R-1)^3}
\end{align*}

\item
$$\lim_{R\rightarrow+\infty} I(R)=0$$
So if we apply the limit to the bound we found in part (b) we see
$$\lim_{R\rightarrow+\infty} |I(R)|\le \lim_{R\rightarrow+\infty}\frac{2\pi}{R(R-1)^3}$$
We see that the limit makes the bound go to zero so we see that
$$\lim_{R\rightarrow+\infty} |I(R)|\le 0$$
This implies that
$$\lim_{R\rightarrow+\infty} I(R) = 0$$

\item
$I=0$

So if we see that if for any contour where $|z|>2$ the integral around that contour is zero. Therefore we can deform said contour back to the original $\Gamma$ and say that 
$$I=0$$
\end{enumerate}
\item 19) \textit{Using the method of proof in Prob. 18, establish the following theorem. If $P$ is a polynomial of degree at least 2 and $P$ has all its zeros inside the circle $|z| = r$, then}
$$\oint_{|z|=r}\frac{1}{P(z)}dz=0$$
First we can see that assuming all the zeros of $P$ are contained in $|z|=r$ we can continuously deform the contour such that $|z|=R$ where $R>r$. Now if we find the bound of the integral we can see that
$$\left|\oint_{|z|=R}\frac{1}{P(z)}dz\right|\le \max\left\{\left|\frac{1}{P(z)}\right|\right\}l(|z|=R)$$
We know that the length of the circle is given by $2\pi R$ and the max of $P$ goes by $\frac{1}{R^n}$ where $n$ is the degree of the polynomial we assume that $n\ge2$ So we can see that
$$\left|\oint_{|z|=R}\frac{1}{P(z)}dz\right|\le \frac{1}{f(R^n)}2\pi R$$
So given our assumption that $P$ is of degree 2 or higher we can cancel the $R$ from the length giving us
$$\left|\oint_{|z|=R}\frac{1}{P(z)}dz\right|\le \frac{2\pi}{f(R^{n-1})}$$
now if we allow the radius of our circle to go to infinity we get
$$\lim_{R\rightarrow+\infty}\left|\oint_{|z|=R}\frac{1}{P(z)}dz\right|\le \lim_{R\rightarrow+\infty}\frac{2\pi}{f(R^{n-1})}$$
And because we only have $R$ terms in the denominator the upper bound goes to zero or
$$\lim_{R\rightarrow+\infty}\left|\oint_{|z|=R}\frac{1}{P(z)}dz\right|\le 0$$
Because the upper bound is zero the magnitude of the integral must be zero
$$\lim_{R\rightarrow+\infty}\left|\oint_{|z|=R}\frac{1}{P(z)}dz\right| = 0$$
$$\oint_{|z|=R}\frac{1}{P(z)}dz= 0$$
Now because we can continuously deform our contour to this point we can reverse the process back to where $|z|=r$ but this does not change the integral due to the \emph{Deformation Invariance Theorem} so we can say
$$\oint_{|z|=r}\frac{1}{P(z)}dz=0$$

\end{enumerate}
\end{enumerate}

\subsection*{Problems}
\begin{enumerate}

\item 4.4: 10, 20
\begin{enumerate}[(i)]
\item 10)\textit{Determine the domain of analyticity for each of the given functions $f$ and explain why}
$$\oint_{|z|=2}f(z)dz=0$$
\begin{enumerate}[(a)]
\item 
$$f(z) = \frac{z}{z^2+25}$$
the only points where $f(z)$ is not analytic is at $z=5$ and $z=-5$ or the zeros of the polynomial in the denominator. These points are not on the contour $|z|=2$ or inside it so \emph{Theorem 9} states that: \textit{If $f$ is analytic in a simply connected domain $D$ and $\Gamma$ is a any loop in $D$ then}
$$\int_{\Gamma}f(z)dz = 0$$

\item 
For
$$f(z) = \frac{\cos z}{z^2-6z+10}$$
we see that the points where $f(z)$ is not analytic are again the zeros of the polynomial $z^2-6z+10$. We can factor this polynomial as $(z-3+i)(z-3-i)$ in this form we see that there are zeros at $z=3+i$ and $z=3-i$. Again these points are not in or on the circle made by $|z|=2$. Therefore we can say that $f(z)$ is analytic in and on $\Gamma$ and we can apply \emph{Theorem 9} and say
$$\oint_{|z|=2}\frac{\cos z}{z^2-6z+10}dz=0$$
\item 
For $$f(z) = \sec\left(\frac{z}{2}\right)$$
we see that the $f(z)$ is not analytic when $\cos(z/2)=0$ this is when $z=\pi\pm2n\pi$ where $n=1,2,3...$. We see that this set of points is outside of the circle $|z|=2$ so again by \emph{Theorem 9} 
$$\oint_{|z|=2}\sec\left(\frac{z}{2}\right) dz=0$$

\item  The function $$f(z) = e^{-z}(2z+1)$$
is analytic for all $z$. So this implies that \emph{Theorem 9} is applicable. It follows from \emph{Theorem 9} that
$$\oint_{|z|=2}e^{-z}(2z+1)dz=0$$

\item The function $$f(z) = \Log(z+3)$$ is not analytic at $z=-3$ and if we assume that we are using the principle branch of $\log$ then there is an interval $z\in(-\infty,-3]$ where $f(z)$ is not analytic. But this branch of $\log$ is still analytic in and on the circle $|z|=2$. Therefore we know that \emph{Theorem 9} holds true and 
$$\oint_{|z|=2}\Log(z+3)dz=0$$
\end{enumerate}

\item 20)\\
First we can say that
$$\int_{\Gamma} \frac{1}{z^4-1}dz=0$$
because we see (by looking at figure 4.50) that $\Gamma$ includes all the zeros of the polynomial $P=z^4-1$. So by \emph{Deformation Invariance Theorem} we can deform $\Gamma$ into a circle that has all the zeros of $P$. And if we expand the radius of the circle to infinity we see that the upper bound of the integral becomes zero.

Now if we use partial fractions we can say
\begin{align*}
\int_{\Gamma} \frac{1}{z^4-1}dz &= \int_{\Gamma} \frac{1}{(z-1)(z+1)(z-i)(z+i)}dz\\
&= \int_{\Gamma} \frac{A}{(z-1)}+\frac{B}{(z+1)}+\frac{C}{(z-i)}+\frac{D}{(z+i)}dz
\end{align*}
We can see that $A=C=1$ and $B=D=-1$ so the integral becomes
$$\int_{\Gamma} \frac{1}{(z-1)}-\frac{1}{(z+1)}+\frac{1}{(z-i)}-\frac{1}{(z+i)}dz$$
Now each of for each fraction we can deform the contour so we integrate over a circle around each point to give us $2\pi i$ so we have
\begin{align*}
\int_{\Gamma} \frac{1}{(z-1)}-\frac{1}{(z+1)}+\frac{1}{(z-i)}-\frac{1}{(z+i)}dz &= 2\pi i - 2\pi i+2\pi i -2\pi i\\
&= 0
\end{align*}


\end{enumerate}

\item 4.5:  2, 4 
\begin{enumerate}[(i)]
\item 2)\\
\newtheorem{Theo2}{Theorem}
\begin{Theo2}
Let $f$ and $g$ be analytic inside and on the simple loop $\Gamma$. If $f(z) = g(z)$ for all $z$ on $\Gamma$, the $f(z) = g(z)$ for all $z$ inside $\Gamma$.
\end{Theo2}
\begin{proof}
So if we assume that $f(z)$ and $g(z)$ are analytic on and in the simple loop $\Gamma$ then it follows from \emph{Theorem 9} that
$$f^{(n)}(z) = \frac{n!}{2\pi i}\int_{\Gamma}\frac{f(w)}{(w-z)^{n+1}}dw$$
and
$$g^{(n)}(z) = \frac{n!}{2\pi i}\int_{\Gamma}\frac{g(w)}{(w-z)^{n+1}}dw$$
And because we assume that $f(z) = g(z)$ for all $z$ on $\Gamma$ then if follows that
$$f^{(n)}(z) = \frac{n!}{2\pi i}\int_{\Gamma}\frac{g(w)}{(w-z)^{n+1}}dw$$
and
$$g^{(n)}(z) = \frac{n!}{2\pi i}\int_{\Gamma}\frac{f(w)}{(w-z)^{n+1}}dw$$
or
$$g^{(n)}(z) = f^{(n)}(z)$$
for $n=1,2,3...$ so every derivative of $f(z)$ equals every derivative of $g(z)$ in and on $\Gamma$. There is only one unique function that this holds true, therefore $f(z)=g(z)$ for all $z$ in $\Gamma$
\end{proof}

\item 4)
\textit{Compute}
$$\int_C \frac{z+i}{z^3+2z^2}dz$$
where $C$ is
\begin{enumerate}[(a)]
\item\textit{the circle $|z|=1$ traversed once anticlockwise.}
First if we factor the denominator we get
$$\int_C \frac{z+i}{z^2(z+2)}dz$$
Now we say
$$f(z) = \frac{z+i}{z+2}$$
Note that $f(z)$ is analytic on and in the circle $|z|=1$ so we apply \emph{Theorem 9}
$$f^{(n)}(z)=\frac{n!}{2\pi i}\int_{\Gamma}\frac{f(w)}{(w-z)^{n+1}}dw$$
where $n=1$ and $z=0$ so
$$f^{'}(2)=\frac{1!}{2\pi i}\int_{\Gamma}\frac{f(w)}{z^{2}}dw$$
Where
\begin{align*}
f'(z) &= \frac{d}{dx}\frac{z+i}{z+2}\\
&= \frac{d}{dx}(z+i)(z+2)^{-1}\\
&= (1)(z+2)^{-1}+(z+i)(-1)(z+2)^{-2}\\
&= \frac{1}{z+2}+\frac{-z+i}{(z+2)^{2}}
\end{align*}
We evaluate at $z=0$
\begin{align*}
f'(0) &= \frac{1}{0+2}+\frac{0+i}{(0+2)^{2}}\\
f'(0) &= \frac{1}{2}+\frac{i}{4}\\
f'(0) &= \frac{2}{4}+\frac{i}{4}\\
f'(0) &= \frac{2+i}{4}
\end{align*}
So
$$\frac{2+i}{4}=\frac{1}{2\pi i}\int_{\Gamma}\frac{w+i}{(w+2)z^2}dw$$
Solving for the integral yields
\begin{align*}
\int_{\Gamma}\frac{w+i}{(w+2)z^2}dw &= \frac{2+i}{4}(2\pi i)\\
 &= \frac{2+i}{2}(\pi i)\\
 &= \pi\frac{2i+i^2}{2}\\
 &= \pi\frac{2i-1}{2}
\end{align*}

\item\textit{the circle $|z+2-i|=2$ traversed once anticlockwise.}
Again we have
$$\int_C \frac{z+i}{z^2(z+2)}dz$$
now we pick $f(z)$ as
$$f(z) = \frac{z+i}{z^2}$$
again $f(z)$ is analytic in and on the contour $\Gamma$ so we apply the \emph{Cauchy's Integral Formula} 
$$f(z_0) = \frac{1}{2\pi i}\int_{\Gamma}\frac{f(z)}{z-z_0}$$
where $z_0=-2$ so we calculate
\begin{align*}
f(-2) &= \frac{-2+i}{(-2)^2}\\
&= \frac{-2+i}{4}
\end{align*}
Solving for the integral 
\begin{align*}
\int_{\Gamma}\frac{f(z)}{z-z_0} &= \frac{-2+i}{4}{2\pi i}\\
&= \frac{-2+i}{2}{\pi i}\\
&= \pi\frac{-2i+i^2}{2}\\
&= \pi\frac{-2i-1}{2}\\
&= -\pi\frac{1+2i}{2}
\end{align*}

\item\textit{the circle $|z-2i|=1$ traversed once anticlockwise.}
For
$$\int_C \frac{z+i}{z^2(z+2)}dz$$
where $C$ is the circle $|z-2i|=i$ we pick $f(z)$ as
$$f(z) = \frac{z+i}{z+2}$$
Again \emph{Theorem 9} applys where $n=1$ and $z=0$ so we calculate
\begin{align*}
f'(z) &= \frac{d}{dz}\frac{z+i}{z+2}\\
&= \frac{1}{z+2}+\frac{-z+i}{(z+2)^{2}}\\
f'(0) &= \frac{2+i}{4}
\end{align*}
(See part (a) for rigorous calculation of the derivative). So if we solve for the integral we get
\begin{align*}
\int_C \frac{z+i}{z^2(z+2)}dz &= \frac{2+i}{4}2\pi i\\
&= \frac{2i-1}{2}\pi
\end{align*}
\end{enumerate}

\end{enumerate}
\item Let 
\begin{align*}
\Delta_1 &= \{z\in \CC\ \mid\ |z|< 2\}\\
\Delta_2 &= \left\{z\in \CC\ \mid\ |z-i|<\frac{1}{2}\right\},
\end{align*}
and let $\partial\Delta_j$ be the boundary of $\Delta_j$.  Evaluate the following integrals.
\begin{enumerate}
\item $\dd\int_{\partial\Delta_1}\frac{dz}{z^4-1}$,
We see that the contour $\partial\Delta_1$ contains all the discontinuities of 
$$\frac{1}{z^4-1}$$
Note that the discontinuities are $z=1,-1,i,-i$. We apply the theorem from problem 19 and get that
$$\int_{\partial\Delta_1}\frac{dz}{z^4-1}=0$$

\item $\dd\int_{\partial\Delta_2}\frac{dz}{z^4-1}$,
We see that the function 
$$\frac{1}{z^4-1}$$
has a discontinuity at $z=i$ which is the center of the circle $|z-1|=\frac{1}{2}$ or the contour $\partial\Delta_2$. This implies that that
$$\int_{\partial\Delta_2}\frac{dz}{z^4-1} = 2\pi i$$

\item $\dd\int_{\partial\Delta_1}z^m(1-z)^n dz$, for $m,n\in\ZZ$,
We can write $z^m(1-z)^n$ as
$$z^m(1-z)^n = e^{m\Log(z)}e^{n\Log(1-z)}$$
or
$$z^m(1-z)^n = e^{mn\Log (z(1-z))}$$
Note that we are taking the principle branch of $\log$ and that $\Log(z(1-z))$ is not analytic on the real axis, except for the interval $(0,1)$

\item $\dd\int_{\partial\Delta_2}|z|^2dz$
If we write $z$ in the form $z=x+iy$ we see that 
$$|z|^2 = x^2+y^2$$
Note that this is a real valued function and is analytic everywhere in and on the closed contour $\partial\Delta_2$ so we integrate over a closed loop to yield
$$\int_{\partial\Delta_2}|z|^2dz=0$$

\end{enumerate}
\end{enumerate}

\end{document}

