\documentclass[11pt]{article}

\usepackage{latexsym}
\usepackage{amssymb}
\usepackage{enumerate}
\usepackage{amsthm}
\usepackage{amsmath}
\usepackage{ulem}
\usepackage{cancel}

\setlength{\evensidemargin}{.25in}
\setlength{\oddsidemargin}{-.25in}
\setlength{\topmargin}{-.75in}
\setlength{\textwidth}{6.5in}
\setlength{\textheight}{9.5in}
\newcommand{\due}{November 4th, 2009}
\newcommand{\HWnum}{9}
\newcommand{\CC}{\mathbb{C}}
\newcommand{\ZZ}{\mathbb{Z}}
\newcommand{\zbar}{\overline{z}}
\newcommand{\Let}{\textnormal{Let }}
\newcommand{\Arg}{\textnormal{arg}}

\begin{document}
\begin{titlepage}
\setlength{\topmargin}{1.5in}
\begin{center}
\Huge{Physics 3320} \\
\LARGE{Principles of Electricity and Magnetism II} \\
\Large{Professor Ana Maria Rey} \\[1cm]

\huge{Homework \#\HWnum}\\[0.5cm]

\large{Joe Becker} \\
\large{SID: 810-07-1484} \\
\large{\due} 

\end{center}

\end{titlepage}


\subsection*{Written Problems}
\begin{enumerate}
\item Suppose that $f:\CC\rightarrow \CC$ is continuous on a contour $\Gamma\subseteq \CC$.  Show that if 
$$H(z)=\int_\Gamma \frac{f(w)dw}{(w-z)^2},$$
then 
$$H'(z)=2\int_\Gamma \frac{f(w)dw}{(w-z)^3}.$$
\item Prove the generalized Cauchy integral formula by induction on $n$.  In particular, show that if we assume that 
$$f^{(n-1)}(z)=\frac{(n-1)!}{2\pi i}\int_\Gamma \frac{f(w)dw}{(w-z)^{n}},$$
then 
$$f^{(n)}(z)=\frac{n!}{2\pi i}\int_\Gamma \frac{f(w)dw}{(w-z)^{n+1}}.$$
Hint: Write
$$\frac{\int_\Gamma \frac{f(w)dw}{(w-z)^{n}}-\int_\Gamma \frac{f(w)dw}{(w-a)^{n}}}{z-a}=\frac{\int_\Gamma \frac{f(w)dw}{(w-z)^{n-1}(w-a)}-\int_\Gamma \frac{f(w)dw}{(w-a)^{n}}}{z-a}+\int_\Gamma \frac{f(w)dw}{(w-z)^{n}(w-a)}.$$
For the first term, define a function $h(w)=f(w)/(w-a)$, and use induction.  For the last term prove that 
$$\lim_{z\rightarrow a} \int_\Gamma \frac{f(w)dw}{(w-z)^{n}(w-a)}=\int_\Gamma \frac{f(w)dw}{(w-a)^{n+1}}.$$
\end{enumerate}

\subsection*{Problems}


\begin{enumerate}
\item 4.5: 6, 8, 13, 14
\begin{enumerate}[(i)]
\item 5)\\
\item 8)\\
\item 13)\\
\item 14)\\
\end{enumerate}

\item 4.6:  5, 7
\begin{enumerate}[(i)]
\item 5)\\
\newtheorem{theo4.65}{Theorem}
\begin{theo4.65}
Let $f$ be entire and suppose that Re$f(z)\le M$ for all $z$. Then $f$ must be a constant function
\end{theo4.65}
\begin{proof}
Assume that $f$ takes the form
$$f(z(x,y)) = u(x,y)+iv(x,y)$$
and we raise $f$ to an exponent or $e^f$ we see that
$$e^f = e^ue^{iv}.$$
We can see that the magnitude of this function is
$$|e^f| = |e^ue^{iv}| = |e^u|.$$
We assumed that the real part of $f$ is bounded by $M$. So it follows that $|e^u|\le M$. Therefore
$$|e^f|\le M$$
now we see that $e^f$ is bounded so we know from \emph{Liouville's Theorem} that
$e^f$ is constant, because $e$ is constant we can infer that $f$ is a constant function.
\end{proof}

\item 7)\\
\newtheorem{theo4.67}{Theorem}
\begin{theo4.67}
Suppose that $f$ is entire and that $|f(z)|\le|z|^2$ for all sufficiently large values of $|z|$, say $|z|>r_0$. Then $f$ must be a polynomial of degree at most 2.
\end{theo4.67}
\begin{proof}

\end{proof}
\end{enumerate}
\end{enumerate}

\end{document}

