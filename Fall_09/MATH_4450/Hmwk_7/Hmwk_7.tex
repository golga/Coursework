\documentclass[11pt]{article}

\usepackage{latexsym}
\usepackage{amssymb}
\usepackage{enumerate}
\usepackage{amsthm}
\usepackage{amsmath}
\usepackage{ulem}
\usepackage{cancel}

\setlength{\evensidemargin}{.25in}
\setlength{\oddsidemargin}{-.25in}
\setlength{\topmargin}{-.75in}
\setlength{\textwidth}{6.5in}
\setlength{\textheight}{9.5in}
\newcommand{\due}{October 21st, 2009}
\newcommand{\HWnum}{7}
\newcommand{\CC}{\mathbb{C}}
\newcommand{\ZZ}{\mathbb{Z}}
\newcommand{\zbar}{\overline{z}}
\newcommand{\Let}{\textnormal{Let }}
\newcommand{\Arg}{\textnormal{arg}}
\newcommand{\Log}{\textnormal{Log}}

\begin{document}
\begin{titlepage}
\setlength{\topmargin}{1.5in}
\begin{center}
\Huge{Physics 3310} \\
\LARGE{Principles of Electricity and Magnetism 1} \\
\Large{Professor Thomas R. Schibli} \\[1cm]

\huge{Homework \#\HWnum}\\[0.5cm]

\large{Joe Becker} \\
\large{SID: 810-07-1484} \\
\large{\due} 

\end{center}

\end{titlepage}


\subsection*{Written Problems}
\begin{enumerate}
\item 4.2: 14\\
\textit{For each of the following, use Theorem 5 to establish the indicated estimate.}
\begin{enumerate}[(a)]
\item \textit{If $C$ is the circle $|z|=3$ traversed once, then}
$$\left|\int_C\frac{dz}{z^2-i}\right|\le \frac{3\pi}{4}$$
So by \emph{Theorem 5} we can say that
$$\left|\int_C\frac{dz}{z^2-i}\right|\le \max_{z on C}|f(z)|l(C)$$
We know the length of the circle is given by $2\pi r$ so for the circle C we get
$$l(C) = 6\pi$$
Now if we take the max of $f(z)$, by the triangle inequality we get
$$\left|\frac{1}{z^2-i}\right|\le \frac{1}{|z|^2-|i|}$$
$$\left|\frac{1}{z^2-i}\right|\le \frac{1}{3^2-1}$$
$$\left|\frac{1}{z^2-i}\right|\le \frac{1}{8}$$
So \emph{Theorem 5} yields
$$\left|\int_C\frac{dz}{z^2-i}\right|\le \frac{1}{8}6\pi$$
$$\left|\int_C\frac{dz}{z^2-i}\right|\le \frac{3\pi}{4}$$

\item
\textit{If $\gamma$ is the vertical line segment from $z=R (>0)$ to $z=R+2\pi i$, then}
$$\left|\int_{\gamma} \frac{e^{3z}}{1+e^z}dz\right| \le \frac{2\pi e^{3R}}{e^R-1}$$
So we can first see that the length of the line is given by
$$l(\gamma) = 2\pi$$
So now we need to find the max of $|f(z)|$ by the triangle inequality
\begin{align*}
\left|\frac{e^{3z}}{1+e^z}\right| &\le \frac{|e^{3z}|}{|1|+|e^z|}\\
&\le \frac{|e^{3(x+iy)}|}{1+|e^{x+iy}|}\\
&\le \frac{e^{3R}}{1+e^{R}}
\end{align*}
So by \emph{Theorem 5} we can say
$$\left|\int_{\gamma} \frac{e^{3z}}{1+e^z}dz\right| \le \frac{e^{3R}}{1+e^R}2\pi$$
$$\left|\int_{\gamma} \frac{e^{3z}}{1+e^z}dz\right| \le \frac{2\pi e^{3R}}{1+e^R}$$

\item
\textit{If $\Gamma$ is the arc of the circle $|z| =1$ that lies in the first quadrant, then}
$$\left|\int_{\Gamma} \Log(z)dz\right| \le \frac{\pi^{2}}{4}$$
First we can say the length of a quarter of a circle of radius 1 is
\begin{align*}
l(\Gamma) &= \frac{1}{4}2\pi(1)\\
&= \frac{1}{2}\pi\\
\end{align*}
Now if we find the max of $|f(z)|$ using the triangle inequality we get
\begin{align*}
|\Log(z)| &\le \Log|z|\\
&\le \Log(1)\\
&\le \frac{\pi}{2}
\end{align*}
Note that the maximum of the $\Log(1)$ is $\frac{\pi}{2}$ on $\Gamma$. So \emph{Theorem 5} yields
\begin{align*}
\left|\int_{\Gamma}\Log z\right| &\le \max_{z on C}|f(z)|l(C)\\
 &\le \frac{\pi}{2}\frac{1}{2}\pi\\
 &\le \frac{\pi^2}{4}
\end{align*}

\item
\textit{If $\gamma$ is the line segment from $z=0$ to $z=i$, then}
$$\left|\int_{\gamma} e^{\sin z}dz\right| \le 1$$
Again we first find the length of $\gamma$ which is
$$l(\gamma) = 1$$
Now to find the max of $f(z) = e^{\sin z}$ we say that
$$|e^{\sin z}| \le e^{|\sin z|}$$ 
$$|e^{\sin z}| \le e^{\sin |z|}$$ 
We see that the magnitude of $\sin z$ over the line from $z=0$ to $z=i$ is when $z=0$ or $\sin z = 0$ so we get
\begin{align*}
|e^{\sin z}| &\le e^{\sin |0|}\\ 
&\le e^{0}\\ 
&\le 1 
\end{align*}
So by \emph{Theorem 5} we can say 
$$\left|\int_{\gamma}e^{\sin z}\right|\le \max_{z on C}|f(z)|l(C)$$
$$\left|\int_{\gamma}e^{\sin z}\right|\le 1$$
\end{enumerate}

\item 4.3: 6\\
For the function $f(z)$
$$f(z) = \frac{1}{z-z_0}$$
we see that if we have a circle $C$ with a branch cut from $z_0$ to $\infty$ we can see that $f(z)$ has an antiderivative given by
$$F(z) = \Log(z-z_0)$$
therefore we can apply \emph{Theorem 6} or
$$\int_C f(z)dz = F(z_2)-F(z_1)$$
For the our particular $C$ we have endpoints of $\alpha$ and $\beta$ so we see that \emph{Theorem 6} yields
$$\int_C \frac{1}{z-z_0}dz =\Log(\beta) - \Log(\alpha)$$
Now if we take the limit as $\alpha$ and $\beta$ go to $\tau$, where $\tau$ is the point at which our branch cut meets the circle $C$ we get
$$\int_C \frac{1}{z-z_0}dz =\lim_{\alpha,\beta\rightarrow\tau}\Log(\beta) - \Log(\alpha)$$
We see that if $\alpha$ and $\beta$ approach from the same direction the integral becomes zero. If they approach from different directions the integral yields $2\pi i$ which is the result when we apply \emph{Theorem 6} to a closed circle around $z_0$.
\end{enumerate}

\subsection*{Problems}

\begin{enumerate}
\item 4.1:  2, 8, 10
\begin{enumerate}[(i)]
\item 2)\\
\textit{Show why the condition that $z'(t)$ never vanishes is necessary to ensure that smooth curves have no cusps.}
If we take a parametrization that has a cusp we can see that $z'(t)=0$ at some point in the interval. Say we have the curve
$$z(t) = t^2 +it^3,\ (-1\le t\le 1)$$
We can find that 
$$z'(t) = 2t+3it^2$$
we see that when $t=0$ (which is in the interval $-1\le t\le 1$) $z'(t) = 0$. And we know that $z(t)$ has a cusp at $t=0$ this shows that when $z'(t)=0$ we have a cusp. Therefore we know that $z'(t)$ cannot vanish if we wish to ensure that our curve has no cusps.


\item 8)\\
\textit{Parametrize the contour $\Gamma$ indicated in Fig. 4.14. Also give the parametrization for the opposite contour $-\Gamma$}

For the positive $\Gamma$ we first go through the line from $-2+2i$ to $-1$. This is parametrized as
$$z_1(t) = (-2+2i)(-t)-(1+t),\ (-1\le t\le 0)$$
Now for the circle with radius of $1$ that is centered on the origin. This is parametrized as
$$z_2(t) = e^{-\pi i(1-t)},\ (0\le t\le 1)$$
So we can see these two smooth curves are components of $\Gamma$. Therefore we can represent $\Gamma$ as
$$z(t)=\left\{\begin{array}{ll}
	(-2+2i)(-t)-(1+t), &(-1\le t\le 0)\\
	e^{-\pi i(1-t)}, &(0\le t\le 1)
	\end{array}\right.$$
Now for negative $\Gamma$ we start with the circle, which we can parametrize as
$$z_1(t) = e^{-\pi i(1+t)},\ (-1\le t\le 0)$$
and for the line we can say
$$z_2(t) = (-2+2i)(t)-(1-t),\ (0\le t\le 1)$$
So for the contour $-\Gamma$ we have the parametrization
$$z(t)=\left\{\begin{array}{ll}
	e^{-\pi i(1+t)}, &(-1\le t\le 0)\\
 	(-2+2i)(t)-(1-t), &(0\le t\le 1)
	\end{array}\right.$$
	

\item 10)\\
\textit{Using an admissible parametrization verify from formula (1) that}
\begin{enumerate}[(a)]
\item
\textit{the length of the line segment from $z_1$ to $z_2$ is $|z_2-z_1|$}

So if we can parametrize a line that goes from $z_1$ to $z_2$ as
$$z(t) = z_1(1-t) + z_2t,\ (0\le t\le 1)$$
We can calculate the length using
$$l(\gamma) = \int_a^b \left|\frac{dz}{dt}\right| dt$$
where
$$\frac{dz}{dt} = z_2-z_1$$
Solving the integral yields
\begin{align*}
l(\gamma) &= \int_a^b \left|\frac{dz}{dt}\right| dt\\
&= \int_0^1 \left|z_2-z_1\right| dt\\
&= \left.\left|z_2-z_1\right|t\right|_0^1 \\
&= \left|z_2-z_1\right|1 - \left|z_2-z_1\right|0\\
l(\gamma) &= \left|z_2-z_1\right|
\end{align*}
\item
\textit{the length of a circle $|z-z_0|=r$ is $2\pi r$}

We can parametrize the circle $|z-z_0|=r$ as 
$$z(t) = z_0 + re^{it},\ (0\le t\le 2\pi)$$
Again we calculate the length by
$$l(\gamma) = \int_a^b \left|\frac{dz}{dt}\right| dt$$
where
$$\frac{dz}{dt} = ire^{it}$$
Note the magnitude of $z'(t)$ is $r$. Now solving the integral yields
\begin{align*}
l(\gamma) &= \int_a^b \left|\frac{dz}{dt}\right| dt\\
&= \int_0^{2\pi} \left|ire^{it}\right| dt\\
&= \int_0^{2\pi} r dt\\
&= \left.rx\right|_0^{2\pi}\\
&= r(2\pi) - r(0)\\
l(\gamma) &= 2\pi r
\end{align*}
\end{enumerate}
\end{enumerate}

\item 4.2:  6, 8, 10 
\begin{enumerate}[(i)]
\item 6)
\textit{Compute
$$\int_{\Gamma} \zbar dz$$
where}
\begin{enumerate}[(a)]
\item \textit{$\Gamma$ is the circle $|z|=2$ traversed once anti-clockwise}

We can represent $\Gamma$ as a single parametrization of the curve
$$z(t) = 2e^{it},\ (0\le t\le 2\pi)$$
where
$$z'(t) = 2ie^{it},\ (0\le t\le 2\pi)$$
With this parametrization we can calculate
$$\int_{\Gamma} f(z) dz = \int_a^b f(z(t))z'(t)dt$$
where $$f(z) = \zbar$$
So we calculate
\begin{align*}
\int_{\Gamma} \zbar dz &= \int_0^{2\pi} \overline{z(t)}z'(t)dt\\
&= \int_0^{2\pi} \overline{2e^{it}}\left(2ie^{it}\right)dt\\
&= (2)(2)(i)\int_0^{2\pi} e^{-it}\left(e^{it}\right)dt\\
&= 4i\int_0^{2\pi} e^{-it+it}dt\\
&= 4i\int_0^{2\pi} e^0dt\\
&= 4i\int_0^{2\pi} dt\\
&= 4i(t|_0^{2\pi}\\
&= 4i(2\pi-0)\\
&= 4i(2\pi)\\
&= 8i\pi
\end{align*}
\item \textit{$\Gamma$ is the circle $|z|=2$ traversed once clockwise}

For this $\Gamma$ we have a single curve parametrized as
$$z(t) = 2e^{-it},\ (-\pi\le t\le \pi)$$
where
$$z'(t) = -2ie^{-it},\ (-\pi\le t\le \pi)$$
We can calculate
\begin{align*}
\int_{\Gamma} \zbar dz &= \int_{-\pi}^{\pi} \overline{z(t)}z'(t)dt\\
&= \int_{-\pi}^{\pi} \overline{2e^{-it}}\left(-2ie^{it}\right)dt\\
&= (2)(-2i)\int_0^{-\pi} e^{it}\left(e^{-it}\right)dt\\
&= -4i\int_{-\pi}^{\pi} e^{it-it}dt\\
&= -4i\int_{-\pi}^{\pi} e^0dt\\
&= -4i\int_{-\pi}^{\pi} dt\\
&= -4i(t|_{-\pi}^{\pi}\\
&= -4i(\pi-{-\pi})\\
&= -4i(2\pi)\\
&= -8i\pi
\end{align*}

\item \textit{$\Gamma$ is the circle $|z|=2$ traversed three times clockwise}
We see that $\Gamma$ is 3 smooth curves described in part (b)
$$z(t) = 2e^{-it},\ (-\pi\le t\le \pi)$$
where
$$z'(t) = -2ie^{-it},\ (-\pi\le t\le \pi)$$
Notice that we can say
\begin{align*}
\int_{\Gamma}f(z)dz &= \int_{\gamma} f(z)dz + \int_{\gamma} f(z)dz + \int_{\gamma} f(z)dz\\
&= 3\int_{\gamma} f(z)dz\\
&= 3\int_{\gamma} f(z(t))z'(t)dt
\end{align*}
We can calculate
\begin{align*}
\int_{\Gamma} \zbar dz &= 3\int_{-\pi}^{\pi} \overline{z(t)}z'(t)dt\\
&= 3\int_{-\pi}^{\pi} \overline{2e^{-it}}\left(-2ie^{it}\right)dt\\
&= (3)(2)(-2i)\int_0^{-\pi} e^{it}\left(e^{-it}\right)dt\\
&= -12i\int_{-\pi}^{\pi} e^{it-it}dt\\
&= -12i\int_{-\pi}^{\pi} e^0dt\\
&= -12i\int_{-\pi}^{\pi} dt\\
&= -12i(t|_{-\pi}^{\pi}\\
&= -12i(\pi-{-\pi})\\
&= -12i(2\pi)\\
&= -24i\pi
\end{align*}
\end{enumerate}

\item 8)\\
\textit{Let $C$ be the perimeter of the square with vertices at the points $z=0$, $z=1$, $z=1+i$, $z=i$ traversed once in that order. Show that
$$\int_C d^z dz =0$$}

We can say that $C$ is sum of the smooth curves 
$$C = \gamma_1+\gamma_2+\gamma_3+\gamma_4$$
where 
\begin{align*}
z_1(t) &= t,\ (0\le t\le 1)\\
z_2(t) &= 1+it,\ (0\le t\le 1)\\
z_3(t) &= (i+1)(1-t)+it,\ (0\le t\le 1)\\
z_4(t) &= i(1-t),\ (0\le t\le 1)\\
\end{align*}
and
\begin{align*}
z_1'(t) &= 1,\ (0\le t\le 1)\\
z_2'(t) &= i,\ (0\le t\le 1)\\
z_3'(t) &= -1,\ (0\le t\le 1)\\
z_4'(t) &= -i,\ (0\le t\le 1)\\
\end{align*}
are the parametrizations for each of the respective smooth curves. So we can say that
\begin{align*}
\int_Cf(z)dz &=\int_{\gamma_1}f(z_1(t))z_1'(t)dz+\int_{\gamma_2}f(z_2(t))z_2'(t)dz+\int_{\gamma_3}f(z_3(t))z_3'(t)dz+\int_{\gamma_4}f(z_4(t))z_4'(t)dz
\end{align*}
Where $f(z) = e^z$. So we can calculate each integral
\begin{align*}
\int_{\gamma_1}f(z_1(t))z_1'(t)dz &= \int_0^1 e^{t}1dt\\
&= e^{t}|_0^1 \\
&= e^{1}-e^0 \\
\int_{\gamma_1}f(z_1(t))z_1'(t)dz &= e-1 
\end{align*}
\begin{align*}
\int_{\gamma_2}f(z_2(t))z_2'(t)dz &= \int_0^1 e^{1+it}idt\\
 &= i\int_0^1 e^{1+it}dt\\
 &= i\left(\frac{1}{i}e^{1+it}\right|_0^1 \\
 &= \frac{i}{i}\left(e^{1+it}\right|_0^1 \\
 &= e^{1+i(1)}-e^{1+i(0)} \\
 &= e^{1+i}-e^{1} \\
\int_{\gamma_2}f(z_2(t))z_2'(t)dz &= e^{1+i}-e 
\end{align*}
\begin{align*}
\int_{\gamma_3}f(z_3(t))z_3'(t)dz &= \int_0^1 e^{(i+1)(1-t)+it}(-1)dt\\
&= -\int_0^1 e^{(i+1)(1-t)+it}dt\\
&= -\left(-e^{(i+1)(1-t)+it}\right|_0^1 \\
&= e^{(i+1)(1-1)+i1}- e^{(i+1)(1-0)+i0}\\
&= e^{(i+1)(0)+i}- e^{i+1}\\
\int_{\gamma_3}f(z_3(t))z_3'(t)dz &= e^{i}- e^{i+1}
\end{align*}
\begin{align*}
\int_{\gamma_4}f(z_4(t))z_4'(t)dz &= \int_0^1 e^{i(1-t)}(-i)dt\\
&= -i\int_0^1 e^{i(1-t)}dt\\
&= -i\left(\frac{1}{-i}e^{i(1-t)}\right|_0^1 \\
&= -i\frac{1}{-i}\left(e^{i(1-1)}-e^{i(1-0)}\right) \\
&= \frac{-i}{-i}\left(e^{(i+1)(0)}-e^{i(1)}\right) \\
&= e^{0}-e^{i}\\
\int_{\gamma_4}f(z_4(t))z_4'(t)dz &= 1-e^{i}
\end{align*}
So we can calculate
\begin{align*}
\int_Cf(z)dz &=\int_{\gamma_1}f(z_1(t))z_1'(t)dz+\int_{\gamma_2}f(z_2(t))z_2'(t)dz+\int_{\gamma_3}f(z_3(t))z_3'(t)dz+\int_{\gamma_4}f(z_4(t))z_4'(t)dz\\
&= e-1 + e^{1+i}-e + e^{i}- e^{i+1} + 1-e^{i}\\
&= 1 - 1 + e-e + e^{1+i}- e^{i+1} + e^{i}-e^{i}\\
\int_Cf(z)dz &=0
\end{align*}
 
\item 10)\\
\textit{Compute $$\int_C \zbar^2dz$$ along the perimeter of the square in Problem 8}

Recall that we parametrized the square contour $C$ as
\begin{align*}
z_1(t) &= t,\ (0\le t\le 1)\\
z_2(t) &= 1+it,\ (0\le t\le 1)\\
z_3(t) &= (i+1)(1-t)+it,\ (0\le t\le 1)\\
z_4(t) &= i(1-t),\ (0\le t\le 1)\\
\end{align*}
and
\begin{align*}
z_1'(t) &= 1,\ (0\le t\le 1)\\
z_2'(t) &= i,\ (0\le t\le 1)\\
z_3'(t) &= -1,\ (0\le t\le 1)\\
z_4'(t) &= -i,\ (0\le t\le 1)\\
\end{align*}
\end{enumerate}
So we know that
$$\int_Cf(z)dz =\int_{\gamma_1}f(z_1(t))z_1'(t)dz+\int_{\gamma_2}f(z_2(t))z_2'(t)dz+\int_{\gamma_3}f(z_3(t))z_3'(t)dz+\int_{\gamma_4}f(z_4(t))z_4'(t)dz$$
So we see that we need to calculate each individual integral as
\begin{align*}
\int_{\gamma_1}\overline{z_1(t)}^2z_1'(t)dz &=  \int_0^1\overline{t}^2(1)dt\\
&= \int_0^1t^2dt\\
&= \left.\frac{1}{3}t^3\right|_0^1\\
&= \frac{1}{3}(1^3-0^3)\\
&= \frac{1}{3}(1-0)\\
\int_{\gamma_1}\overline{z_1(t)}^2z_1'(t)dz &= \frac{1}{3}
\end{align*}

\begin{align*}
\int_{\gamma_2}\overline{z_2(t)}^2z_2'(t)dz &=  \int_0^1\overline{1+it}^2(i)dt\\
&= i\int_0^1(1-it)^2dt\\
&= i\int_0^1(1^2+i^2t^2-2it)dt\\
&= i\int_0^1(1+(-1)2t^2-2it)dt\\
&= i\int_0^1(1-2t^2-2it)dt\\
&= i\left(t-\frac{2}{3}t^3-it^2\right|_0^1\\
&= i\left(1-\frac{2}{3}1^3-i1^2 - 0 + \frac{2}{3}0^3 + i0^2\right)\\
&= i\left(1-\frac{2}{3}-i - 0 \right)\\
&= i\left(\frac{1}{3}-i\right)\\
&= \frac{1}{3}i-i^2\\
\int_{\gamma_2}\overline{z_2(t)}^2z_2'(t)dz &= 1+ \frac{1}{3}i
\end{align*}
\begin{align*}
\int_{\gamma_3}\overline{z_3(t)}^2z_3'(t)dz &=  \int_0^1\overline{(i+1)(1-t)+it}^2(-1)dt\\
&= -\int_0^1((-i+1)(1-t)-it)^2dt\\
&= -\int_0^1(-i+1+it-t-it)^2dt\\
&= -\int_0^1(-i+1-t)^2dt\\
&= -\frac{-1}{3}\left((-i+1-t)^3\right|_0^1\\
&= \frac{1}{3}\left((-i+1-1)^3-(-i+1-0)^3\right)\\
&= \frac{1}{3}\left((-i)^3-(-i+1)^3\right)\\
&= \frac{1}{3}\left(-i-(-2-2i)\right)\\
&= \frac{1}{3}\left(-i+2+2i)\right)\\
&= \frac{1}{3}\left(2+i)\right)\\
\int_{\gamma_3}\overline{z_3(t)}^2z_3'(t)dz &= \frac{2}{3}+\frac{1}{3}i
\end{align*}
\begin{align*}
\int_{\gamma_4}\overline{z_4(t)}^2z_4'(t)dz &=  \int_0^1\overline{i(1-t)}^2(-i)dt\\
&= -i\int_0^1(-i(1-t))^2dt\\
&= -i\int_0^1-i^2(1-t)^2dt\\
&= -i\int_0^1(1)(1-t)^2dt\\
&= -i\int_0^1(1-t)^2dt\\
&= -i\left(\frac{-1}{3}(1-t)^3\right|_0^1\\
&= \frac{i}{3}((1-1)^3-(1-0)^3)\\
&= \frac{i}{3}((0)^3-(1)^3)\\
&= \frac{i}{3}(-1)\\
\int_{\gamma_4}\overline{z_4(t)}^2z_4'(t)dz &= \frac{-i}{3}
\end{align*}
So the integral over the whole contour $C$ is given by
\begin{align*}
\int_C\zbar^2dz &= \int_{\gamma_1}\overline{z_1(t)}^2z_1'(t)dz+\int_{\gamma_2}\overline{z_2(t)}^2z_2'(t)dz+\int_{\gamma_3}\overline{z_3(t)}^2z_3'(t)dz+\int_{\gamma_4}\overline{z_4(t)}^2z_4'(t)dz\\
&= \frac{1}{3} + 1 + \frac{1}{3}i+ \frac{2}{3}+ \frac{i}{3}i - \frac{i}{3}\\
&= \frac{1}{3} +\frac{2}{3}+ 1 + \frac{1}{3}i\\
&= 1+ 1 + \frac{1}{3}i\\
\int_C\zbar^2dz &= 2 + \frac{1}{3}i
\end{align*}

\item 4.3:  4, 11, 12
\begin{enumerate}[(i)]
\item 4)\\
\textit{True or false: If $f$ is analytic at each point of a closed contour $\Gamma$, then $$\int_{\Gamma}f(z)dz=0$$}
If $f$ is analytic at each point in $\Gamma$ then that implies that $f$ is continuous and has an anti-derivative for each point of a closed contour $\Gamma$. Therefore one can apply 
$$\int_{\Gamma}f(z)dz = F(z_T) - F(z_I)$$
for a closed loop we see that $z_T = z_I$ therefore we get
\begin{align*}
\int_{\Gamma}f(z)dz &= F(z_T) - F(z_I)\\
&= F(z_T) - F(z_T)\\
&= 0
\end{align*}

\item 11)\\
\newtheorem{Theo11}{Theorem}
\begin{Theo11}
If $f$ and $g$ have continuous first derivatives in a domain containing the contour $\Gamma$, then
$$\int_{\Gamma} f'(z)g(z) = f(z)g(z)\vert-\int_{\Gamma} f(z)g'(z)dz$$
\end{Theo11}
\begin{proof}
Let us define a function $h(z)$ as
$$h(z) = f(z)g(z)$$
If we take the derivative of $h(z)$ we have to apply the product rule which yields
$$\frac{dh(z)}{dz} = f'(z)g(z)+f(z)g'(z)$$
So we can say that
$$\int_{\Gamma}dh(z) =\int_{\Gamma} f'(z)g(z)dz+\int_{\Gamma}f(z)g'(z)dz$$
$$f(z)g(z)|_{zI}^{zT} =\int_{\Gamma} f'(z)g(z)dz+\int_{\Gamma}f(z)g'(z)dz$$
Now solving for the integral we get
$$\left.\int_{\Gamma} f'(z)g(z)dz = f(z)g(z)\right|_{zI}^{zT} -\int_{\Gamma}f(z)g'(z)dz$$
\end{proof}

\item 12)\\
\textit{Let $f$ be an analytic function with a continuous derivative satisfying $|f'(z)|\le M$ for all $z$ in the disk $D:|z|<1$. Show that}
$$|f(z_2) - f(z_1)|\le M|z_2-z_1|\ (z_1,z_2 \mbox{ in } D)$$
We see that 
$$f(z_2)-f(z_1) = \int_{z_1}^{z_2} f'(z)dz$$
so by the definition of an integral we can say that
$$\int_{z_1}^{z_2} f'(z)dz = \lim_{n\rightarrow\infty} \sum_{j=1}^{n+1}f'(z_j)(z_j-z_{j-1})$$
So if we take the magnitude of the sume we see that by the triangle formula we get
$$\left|\lim_{n\rightarrow\infty} \sum_{j=1}^{n+1}f'(z_j)(z_j-z_{j-1})\right|\le\lim_{n\rightarrow\infty} \sum_{j=1}^{n+1}\left|f'(z_j)\right|\left|z_j-z_{j-1}\right|$$
we know that the $|f'(z)|\le M$ so we can see that we have
$$\left|\lim_{n\rightarrow\infty} \sum_{j=1}^{n+1}f'(z_j)(z_j-z_{j-1})\right|\le\lim_{n\rightarrow\infty} \sum_{j=1}^{n+1}M\left|z_j-z_{j-1}\right|$$
Now if we take the limit we see that 
$$|z_j-z_{j-1}| \rightarrow l(\gamma)$$
in this case the length of $\gamma$ is the magnitude of the difference of the endpoints or $|z_2-z_1|$. So we see that
$$\left|\int_{z_1}^{z_2} f'(z)dz\right|\le M\left|z_2-z_{1}\right|$$
or 
$$|f(z_2) - f(z_1)|\le M|z_2-z_1|$$
\end{enumerate}
\end{enumerate}

\end{document}

