\documentclass[11pt]{article}

\usepackage{latexsym}
\usepackage{amssymb}
\usepackage{enumerate}
\usepackage{amsthm}
\usepackage{amsmath}
\usepackage{ulem}
\usepackage{cancel}

\setlength{\evensidemargin}{.25in}
\setlength{\oddsidemargin}{-.25in}
\setlength{\topmargin}{-.75in}
\setlength{\textwidth}{6.5in}
\setlength{\textheight}{9.5in}
\newcommand{\due}{November 11th, 2009}
\newcommand{\HWnum}{10}
\newcommand{\CC}{\mathbb{C}}
\newcommand{\ZZ}{\mathbb{Z}}
\newcommand{\zbar}{\overline{z}}
\newcommand{\Let}{\textnormal{Let }}
\newcommand{\Arg}{\textnormal{arg}}
\newcommand{\Log}{\textnormal{Log}}

\begin{document}
\begin{titlepage}
\setlength{\topmargin}{1.5in}
\begin{center}
\Huge{Physics 3320} \\
\LARGE{Principles of Electricity and Magnetism II} \\
\Large{Professor Ana Maria Rey} \\[1cm]

\huge{Homework \#\HWnum}\\[0.5cm]

\large{Joe Becker} \\
\large{SID: 810-07-1484} \\
\large{\due} 

\end{center}

\end{titlepage}


\subsection*{Written Problems}
\begin{enumerate}
\item 4.6: 14\\
\newtheorem{Theo14}{Theorem}
\begin{Theo14}
Let $f$ be analytic in a bounded domain $D$ and continuous up to and including its boundary. Then if $f$ is nonzero in $D$, the modulus $|f(z)|$ attains its minimum balue on the boundary of $D$.
\end{Theo14}
\begin{proof}
If we consider the function $1/f(z)$ in $D$, while assuming that $f$ is nonzero. We can infer that $1/f(z)$ is analytic in $D$ as well. Due to the fact that $f$ is nonzero we can see that where ever $|1/f(z)|$ is at a maximum, we know that $|f|$ is at a minimum. So we see that $|1/f(z)|$ attains its maximum on the boundary of $D$ by definition. Therefore we see that $|f(z)|$ is at a minimum on $D$. Consider the function $f(z) = z$ on a disk centered at the origin. We see that if we allow $f(z)$ to be zero then the minimum of $f$ is at zero, and not the boundary of $D$.
\end{proof}

\item 5.1: 12\\
\textit{Let 
$$F_n(z) = \frac{nz}{n+1}+\frac{3}{n},$$
$n=1,2,3...$. Prove that the sequence $\{F_n(z)\}_1^{\infty}$ converges uniformly to $F(z) = z$ on every closed disk $|z|\le R$.}

So we take the magnitude of the difference $|F_n(z) - F(z)|$ or
\begin{align*}
\left|\frac{nz}{n+1}+\frac{3}{n} - z\right| &= \left|\frac{nz}{n+1}+\frac{3}{n} - \frac{z(n+1)}{n+1}\right|\\
&= \left|\frac{nz}{n+1}+\frac{3}{n} - \frac{nz+z)}{n+1}\right|\\
&= \left|\frac{nz-nz-z}{n+1}+\frac{3}{n}\right|\\
&= \left|\frac{-z}{n+1}+\frac{3}{n}\right|
\end{align*}
Using the \emph{Triangle Inequality} we see that
\begin{align*}
\left|\frac{-z}{n+1}+\frac{3}{n}\right| \le \frac{|-z|}{n+1}+\frac{3}{n} 
\end{align*}
Now we assume we are on a closed disk given by $|z|\le R$ so we see that
$$\left|\frac{-z}{n+1}+\frac{3}{n}\right| \le \frac{R}{n+1}+\frac{3}{n}$$
Now if we take the limit as $n$ goes to infinity we see
\begin{align*}
\lim_{n\rightarrow\infty} \left|F_n(z) - F(z)\right| &\le \lim_{n\rightarrow\infty} \frac{R}{n+1}+\frac{3}{n}\\
\lim_{n\rightarrow\infty} \left|F_n(z) - F(z)\right| &\le 0
\end{align*}
Therefore the sequence $\{F_n(z)\}_1^{\infty}$ converges to $F(z) = z$ uniformly.

\end{enumerate}

\subsection*{Problems}
\begin{enumerate}
\item 4.6: 10 \\
We see that $f(z)$ reaches a maximum in $D: |z|<R$ because of the assumption that $|f(z)|\le 1$. So we see that at $z=0$ $f(z=0) = i$. We see that the magnitude of $f$ at this point is
$$|f(0)| = |i| = 1.$$
So we see that $f$ is at a maximum in $D$ so by 
\newtheorem{Theo23}{Theorem 23}
\begin{Theo23}
If $f$ is analytic in a domain $D$ and $|f(z)|$ achieves its maximum value at a point $z_0$ in $D$, then $f$ is constant in $D$.
\end{Theo23}
We can say the $f$ is constant in $D$ and we know that $f(0)=i$ so $$f(z) = i$$ for all $z\in D$.

\item 5.1: 7 (be sure to supply justification), 8, 10, 16  
\begin{enumerate}[(i)]
\item 7)
\begin{enumerate}[(a)]
\item 
The series 
$$\sum_{k=0}^{\infty}\left(\frac{1+2i}{1-i}\right)^k$$
diverges. The is due to the fact that if we take the limit of the sequence 
$$a_k = \left(\frac{1+2i}{1-i}\right)^k$$
We see that 
$$\lim{k\rightarrow\infty}\left(\frac{1+2i}{1-i}\right)^k= \infty$$
because 
$$\left|\frac{1+2i}{1-i}\right| > 1$$

\item
The series 
$$\sum_{j=1}^{\infty}\frac{1}{j^23^j}$$
converges, because 
$$\lim_{j\rightarrow\infty}\frac{1}{j^23^j}=0$$
So we know that the sequence converges and by definition a series converges if the sequence has a limit.

\item
The series 
$$\sum_{n=1}^{\infty}\frac{ni^n}{2n+1}$$
diverges. This is due to the fact that the limit as $n$ goes to infinity of the sequence 
$$\lim_{n\rightarrow\infty}\frac{ni^n}{2n+1}$$
goes to infinity. Therefore the series diverges as well.

\item To see if the series
$$\sum_{j=1}^{\infty}\frac{j!}{5^j}$$
diverges or converges we apply the \emph{ratio test}. So we take the limit as $j\rightarrow\infty$ of the ratio of $c_{j+1}/c_j$ or
\begin{align*}
\lim_{j\rightarrow\infty} \frac{c_{j+1}}{c_j} &=  \lim_{j\rightarrow\infty} \frac{(j+1)!}{5^{j+1}}\frac{5^j}{j!}\\
&= \lim_{j\rightarrow\infty} \frac{j+1}{5}
\end{align*}
We see that the ratio goes to infinity, so we can say the series diverges.

\item
To see if series
$$\sum_{k=1}^{\infty}\frac{(-1)^kk^3}{(1+i)^k}$$
converges or diverges we can say that 
$$\sum_{k=1}^{\infty}\left(\frac{-1}{1+i}\right)^kk^3$$
and we see that $|-1/(1+i)|<1$ so we can say that the series converges.

\item
The series
$$\sum_{k=1}^{\infty}\left(i^k-\frac{1}{k^2}\right)$$
diverges because the limit 
$$\lim_{k\rightarrow\infty}\left(i^k-\frac{1}{k^2}\right)$$
does not converge to a value. Therefore the series diverges.

\end{enumerate}

\item 8)
\begin{enumerate}[(a)]
\item
If we assume that
$$\sum_{j=0}^{\infty}c_j = S$$
and we assume that $c_J$ and $S$ can be separated into their real and imaginary parts so they are in the form $c_j = x_j+iy_j$ and $S = a+ib$ we see that
$$\sum_{j=0}^{\infty}x_j + i\sum_{j=0}^{\infty}y_j = a+ib$$
so if we group the components together we get the two sums
$$\sum_{j=0}^{\infty}x_j = a$$
and
$$\sum_{j=0}^{\infty}y_j = b$$
So if we take $\overline{c_j}$ we see that we have
\begin{align*}
\sum_{j=0}^{\infty}c_j &= \sum_{j=0}^{\infty}x_j - i\sum_{j=0}^{\infty}y_j\\
&= a - ib\\
&= \overline{S}
\end{align*}

\item
If the sum
$$\sum_{j=0}^{\infty}c_j$$
sums to $S$ we see that for any complex number $\lambda$ the sum
$$\sum_{j=0}^{\infty}\lambda c_j$$
becomes
$$\lambda\sum_{j=0}^{\infty} c_j$$
because we can factor the constant out. Now we just have the sum which we assumed to be $S$ so we can say that
$$\sum_{j=0}^{\infty}\lambda c_j=\lambda S$$
\item
If we assume that
$$\sum_{j=0}^{\infty}c_j=S$$
and that
$$\sum_{j=0}^{\infty}d_j=T.$$
Then the sum
$$\sum_{j=0}^{\infty}\left(c_j+d_j\right)$$
can be factored such that 
$$\sum_{j=0}^{\infty}\left(c_j+d_j\right) = \sum_{j=0}^{\infty}c_j + \sum_{j=0}^{\infty}d_j$$
because the sum is a linear operator . It follows from our initial assumptions that this equals $S+T$, so we see that
$$\sum_{j=0}^{\infty}\left(c_j+d_j\right)=S+T$$
\end{enumerate}

\item 10)\\
The sequence of functions
$$F_n(z) = \frac{z^n}{z^n-3^n}$$
can be rewritten as
\begin{align*}
F_n(z) &= \frac{z^n}{z^n-3^n}\\
&= \frac{z^n}{z^n/z^n(z^n-3^n)}\\
&= \frac{z^n}{z^n(z^n/z^n-3^n/z^n)}\\
&= \frac{z^n}{z^n(1-{3}/{z}^n)}\\
&= \frac{1}{1-({3}/{z})^n}\\
\end{align*}
Now for the case when $|z|<3$ we take the limit as $n$ goes the infinity and we see that
$$\lim_{n\rightarrow\infty}\frac{1}{1-({3}/{z})^n} = 0$$
this is because the term $(3/z)^n$ goes to infinity as $n\rightarrow\infty$. Now for the case when $|z|>3$ the term $(3/z)^n$ goes to zero as $n$ goes to infinity. So it follows that
$$\lim_{n\rightarrow\infty}\frac{1}{1-({3}/{z})^n} = 1$$

\item 16)
\newtheorem{theo16}{Theorem}
\begin{theo16}
The sequence $\{z_n\}_1^{\infty}$ converges if and only if the series $\sum_{k=1}^{\infty}(z_{k+1}-z_k)$ converges.
\end{theo16}
\begin{proof}
We see that the series
$$\sum_{k=1}^{\infty}(z_{k+1}-z_k)$$
telescopes such that 
\begin{align*}
\sum_{k=1}^{\infty}(z_{k+1}-z_k) &= \cancel{z_2}-z_1+\cancel{z_3}-\cancel{z_2}+...+z_{n}-\cancel{z_{n-1}}\\
&= z_n - z_1
\end{align*}
So we see that the leading term remaining is $z_n$ so the series converges if and only if $z_n$ converges. Therefore $\{z_n\}_1^{\infty}$ converges if and only if the series $\sum_{k=1}^{\infty}(z_{k+1}-z_k)$ converges.
\end{proof}

\end{enumerate}

\item 5.2: 4, 13
\begin{enumerate}[(i)]
\item 4)\\
We see that if we write $(1+z)^{\alpha}$ as $e^{\alpha\Log(1+z)}$ we see that this function is analytic everywhere, but at $z=-1$ and the negative real axis below $-1$. This implies that $(1+z)^{\alpha}$ is analytic in $|z|<1$. Therefore we can expand $(1+z)^{\alpha}$ using a \emph{Taylor Series} or
$$f(z) = f(0) + f'(0)z + \frac{f''(0)}{2!}z^2 +...+ \frac{f^{(n)}(0)}{n!}z^n$$ 
So we can calculate the first 3 derivatives of $f(z) = (1+z)^{\alpha}$
\begin{align*}
f(0) &= (1+0)^{\alpha} = 1\\
f'(0) &= \alpha(1+0)^{\alpha-1} = \alpha\\
f''(0) &= \alpha(\alpha-1)(1+0)^{\alpha-2}= \alpha(\alpha-1)\\
f^{(3)}(0) &= \alpha(\alpha-1)(\alpha-2)(1+0)^{\alpha-3}= \alpha(\alpha-1)(\alpha-2)
\end{align*}
So we see that the \emph{Taylor Series} is
$$(1+z)^{\alpha} = 1 + {\alpha}z + \frac{\alpha(\alpha-1)}{2}z^2 + \frac{\alpha(\alpha-1)(\alpha-2)}{6}z^3 +...$$

\item 13)\\
We assume that 
$$(1-z)^{-1} = \sum_{k=0}^{\infty}z^k$$
now if we take the derivative of both sides get
\begin{align*}
(1-z)^{-1} &= \sum_{k=0}^{\infty}z^k\\
\frac{d}{dz}(1-z)^{-1} &=\frac{d}{dz}\sum_{k=0}^{\infty}z^k\\
(1-z)^{-2} &= \sum_{k=0}^{\infty}kz^{k-1}
\end{align*}
Now we multiply by $z$ to raise the power of $z$ from $k-1$ back to $k$
$$z(1-z)^{-2} = \sum_{k=0}^{\infty}kz^{k}$$
Now if we take the derivative again we get
\begin{align*}
\frac{d}{dz} z(1-z)^{-2} &=\frac{d}{dz} \sum_{k=0}^{\infty}kz^{k}\\
2z(1-z)^{-3}+(1-z)^{-2} &= \sum_{k=0}^{\infty}k^2z^{k-1}
\end{align*}
Now if we raise the power of $z$ again by multiplying by $z$ we get
\begin{align*}
\sum_{k=0}^{\infty}k^2z^{k} &= 2z^2(1-z)^{-3}+z(1-z)^{-2} \\
&= (2z(1-z)^{-1}+1)z(1-z)^{-2}\\
\end{align*}

\end{enumerate}
\end{enumerate}

\end{document}

